\biblebook{Liber Ecclesiastes}

\begin{biblechapter}   
\verse Verba Ecclesiastes filii David regis Ierusalem.” 
\verse “Vanitas vanitatum, dixit Ecclesiastes, vanitas vanitatum et omnia vanitas". 
\verse Quid lucri est homini de universo labore suo, quo laborat sub sole? 
\verse Generatio praeterit, et generatio advenit, terra autem in aeternum stat. 
\verse Oritur sol, et occidit sol et ad locum suum anhelat ibique renascitur. 
\verse Gyrat per meridiem et flectitur ad aquilonem, lustrans universa in circuitu pergit spiritus et in circulos suos revertitur. 
\verse Omnia flumina pergunt ad mare, et mare non redundat; ad locum, unde exeunt, flumina illuc revertuntur in cursu suo. 
\verse Cunctae res difficiles; non potest eas homo explicare sermone. Non saturatur oculus visu, nec auris auditu impletur. 
\verse Quod fuit, ipsum est, quod futurum est. Quod factum est, ipsum est, quod faciendum est: 
\verse nihil sub sole novum. Si de quadam re dicitur: “Ecce hoc novum est", iam enim praecessit in saeculis, quae fuerunt ante nos. 
\verse Non est priorum memoria, sed nec eorum quidem, qui postea futuri sunt, erit recordatio apud eos, qui futuri sunt in novissimo. 
\verse Ego Ecclesiastes fui rex Israel in Ierusalem 
\verse et proposui in animo meo quaerere et investigare sapienter de omnibus, quae fiunt sub sole. Hanc occupationem pessimam dedit Deus filiis hominum, ut occuparentur in ea. 
\verse Vidi cuncta, quae fiunt sub sole; et ecce universa vanitas et afflictio spiritus. 
\verse Quod est curvum, rectum fieri non potest; et, quod deficiens est, numerari non potest. 
\verse Locutus sum ego in corde meo dicens: “Ecce ego magnificavi et apposui sapientiam super omnes, qui fuerunt ante me in Ierusalem; et mens mea contemplata est multam sapientiam et scientiam". 
\verse Dedique cor meum, ut scirem sapientiam et scientiam, insipientiam et stultitiam. Et agnovi quod in his quoque esset afflictio spiritus, eo quod 
\verse in multa sapientia multus sit maeror; et, qui addit scientiam, addit et laborem. 
\end{biblechapter}

\begin{biblechapter}  
\verse Dixi ego in corde meo: “Veni, tentabo te gaudio: fruere bonis"; et ecce hoc quoque vanitas. 
\verse De risu dixi: “Insania" et de gaudio: “Quid prodest?". 
\verse Tractavi in corde meo detinere in vino carnem meam, cum cor meum duceretur in sapientia, et amplecti stultitiam, donec viderem quid esset utile filiis hominum, ut faciant sub sole paucis diebus vitae suae. 
\verse Magnificavi opera mea: aedificavi mihi domos et plantavi vineas, 
\verse feci hortos et pomaria et consevi ea arboribus cuncti generis fructuum 
\verse et exstruxi mihi piscinas aquarum, ut irrigarem silvam lignorum germinantium. 
\verse Possedi servos et ancillas et habui multam familiam, habui armenta quoque et magnos ovium greges ultra omnes, qui fuerunt ante me in Ierusalem. 
\verse Coacervavi mihi etiam argentum et aurum et substantias regum ac provinciarum, feci mihi cantores et cantatrices et delicias filiorum hominum, scyphos et urceos in ministerio ad vina fundenda 
\verse et crevi, supergressus sum omnes, qui ante me fuerunt in Ierusalem; sapientia quoque mea perseveravit mecum. 
\verse Et omnia, quae desideraverunt oculi mei, non negavi eis nec prohibui cor meum ab omni voluptate, et oblectatum est ex omnibus laboribus, et hanc ratus sum partem meam ab omnibus aerumnis meis. 
\verse Cumque me convertissem ad universa opera, quae fecerant manus meae, et ad labores, in quibus sudaveram, et ecce in omnibus vanitas et afflictio spiritus, et nihil lucri esse sub sole. 
\verse Verti me ad contemplandam sapientiam et insipientiam et stultitiam: “Quid faciet, inquam, homo, qui veniet post regem? Id quod antea fecerunt". 
\verse Et vidi quod tantum praecederet sapientia stultitiam, quantum lux praecedit tenebras. 
\verse “Sapientis oculi in capite eius, stultus in tenebris ambulat"; et didici quod unus utriusque esset interitus. 
\verse Et dixi in corde meo: “Si unus et stulti et meus occasus erit, quid mihi prodest quod maiorem sapientiae dedi operam?". Locutusque cum mente mea, animadverti quod hoc quoque esset vanitas. 
\verse Non enim erit memoria sapientis similiter ut stulti in perpetuum; siquidem futura tempora oblivione cuncta pariter operient: moritur doctus similiter ut indoctus. 
\verse Et idcirco taeduit me vitae meae, quia malum mihi est, quod sub sole fit; cuncta enim vanitas et afflictio spiritus. 
\verse Rursus detestatus sum omnem laborem meum, quo sub sole laboravi, quem relicturus sum homini, qui erit post me; 
\verse et quis scit utrum sapiens an stultus futurus sit? Et dominabitur in laboribus meis, quibus desudavi et sollicitus fui sub sole. Hoc quoque vanitas. 
\verse Verti me exasperans cor meum de omni labore, quo laboravi sub sole.  
\verse Nam est qui laborat in sapientia et doctrina et sollicitudine, et homini, qui non laboraverit, dabit portionem suam; et hoc ergo vanitas et magnum malum. 
\verse Quid enim proderit homini de universo labore suo et afflictione cordis, qua sub sole laboravit? 
\verse Cuncti dies eius dolores sunt, et aerumnae occupatio eius, nec per noctem cor eius requiescit; et hoc quoque vanitas est. 
\verse Nihil melius est homini quam comedere et bibere et ostendere animae suae bona de laboribus suis. Et hoc vidi de manu Dei esse. 
\verse Quis enim comedet et deliciis affluet sine eo? 
\verse Quia homini bono in conspectu suo dedit sapientiam et scientiam et laetitiam; peccatori autem dedit afflictionem colligendi et congregandi, ut tradat ei, qui placuit Deo; sed et hoc vanitas est et afflictio spiritus. 
\end{biblechapter}

\begin{biblechapter}  
\verse Omnia tempus habent, et momentum suum cuique negotio sub caelo: 
\verse tempus nascendi et tempus moriendi, tempus plantandi et tempus evellendi quod plantatum est, 
\verse tempus occidendi et tempus sanandi, tempus destruendi et tempus aedificandi, 
\verse tempus flendi et tempus ridendi, tempus plangendi et tempus saltandi, 
\verse tempus spargendi lapides et tempus eos colligendi, tempus amplexandi et tempus longe fieri ab amplexibus, 
\verse tempus quaerendi et tempus perdendi, tempus custodiendi et tempus abiciendi, 
\verse tempus scindendi et tempus consuendi, tempus tacendi et tempus loquendi, 
\verse tempus dilectionis et tempus odii, tempus belli et tempus pacis. 
\verse Quid lucri habet, qui operatur, de labore suo? 
\verse Vidi occupationem, quam dedit Deus filiis hominum, ut occuparentur in ea.  
\verse Cuncta fecit bona in tempore suo; et mundum tradidit cordi eorum, et non inveniet homo opus, quod operatus est Deus ab initio usque ad finem. 
\verse Cognovi quod nihil boni esset in eis nisi laetari et facere bene in vita sua.  
\verse Omnis enim homo, qui comedit et bibit et videt bonum de labore suo, hoc donum Dei est. 
\verse Didici quod omnia opera, quae fecit Deus, perseverent in perpetuum; non possumus eis quidquam addere nec auferre, quae fecit Deus, ut timeatur. 
\verse Quod iam fuit, ipsum est; et, quod futurum est, iam fuit; et Deus requirit, quod abiit. 
\verse Et adhuc vidi sub sole: in loco iudicii ibi impietas, et in loco iustitiae ibi iniquitas; 
\verse et dixi in corde meo: “Iustum et impium iudicabit Deus, quia tempus omni rei et omnibus occasio". 
\verse Dixi in corde meo de filiis hominum, ut probaret eos Deus et ostenderet eos in semetipsis similes esse bestiis. 
\verse Quoniam sors filiorum hominis et iumentorum una est atque eadem: sicut moritur homo, sic et illa moriuntur; et idem spiritus omnibus: nihil habet homo iumento amplius, quia omnia vanitas.  
\verse Et omnia pergunt ad unum locum: de terra facta sunt omnia, et in terram omnia pariter revertuntur. 
\verse Quis novit, si spiritus filiorum hominis ascendat sursum, et si spiritus iumentorum descendat deorsum in terram? 
\verse Et deprehendi nihil esse melius quam laetari hominem in opere suo; nam haec est pars illius. Quis enim eum adducet, ut post se futura cognoscat? 
\end{biblechapter}

\begin{biblechapter}  
\verse Verti me ad alia et vidi calumnias, quae sub sole geruntur, et ecce lacrimae oppressorum, et nemo consolator; et ex parte opprimentium violentia, et nemo consolator. 
\verse Et laudavi magis mortuos, qui iam defuncti sunt, quam viventes, qui adhuc vitam agunt, 
\verse et feliciorem utroque iudicavi, qui necdum natus est nec vidit opera mala, quae sub sole fiunt. 
\verse Rursum contemplatus sum omnes labores et omnem successum operis, et hoc esse zelum in proximum suum. Et in hoc ergo vanitas et afflictio spiritus. 
\verse Stultus complicat manus suas et comedit carnes suas. 
\verse Melior est pugillus cum requie quam plena utraque manus cum labore et afflictione spiritus. 
\verse Iterum repperi et aliam vanitatem sub sole: 
\verse unus est et secundum non habet, non filium, non fratrem, et tamen laborare non cessat, nec satiantur oculi eius divitiis, nec recogitat dicens: “Cui laboro et fraudo animam meam bonis?”. In hoc quoque vanitas est et occupatio pessima. 
\verse Melius est duos esse simul quam unum: habent enim emolumentum in labore suo, 
\verse quia si unus ceciderit, ab altero fulcietur. Vae soli! Cum ceciderit, non habet sublevantem se. 
\verse Insuper, si dormierint duo, fovebuntur mutuo; unus quomodo calefiet?  
\verse Et, si quispiam praevaluerit contra unum, duo resistent ei. Et fu niculus triplex non cito rumpitur. 
\verse Melior est puer pauper et sapiens rege sene et stulto, qui iam nescit erudiri. 
\verse Ille enim de domo carceris exivit, ut regnaret, etiamsi in regno istius natus sit pauper. 
\verse Vidi cunctos viventes, qui ambulant sub sole, cum adulescente illo secundo, qui consurgebat pro eo. 
\verse Infinitus numerus erat populi, omnium, quos ipse praecedebat; sed qui postea futuri sunt, non laetabuntur in eo. Et hoc vanitas et afflictio spiritus. 
\verse Custodi pedem tuum ingrediens domum Dei, nam accedere, ut audias, melius est quam cum stulti offerunt victimas: multo enim melior est oboedientia quam stultorum victimae, qui nesciunt se malum facere. 
\end{biblechapter}

\begin{biblechapter}  
\verse Ne temere quid loquaris, neque cor tuum sit velox ad proferen dum sermonem coram Deo; Deus enim in caelo, et tu super terram: idcirco sint pauci sermones tui. 
\verse Multas curas sequuntur somnia, et in multis sermonibus invenietur stultitia. 
\verse Si quid vovisti Deo, ne moreris reddere: displicet enim ei stulta promissio; sed, quodcumque voveris, redde. 
\verse Multoque melius est non vovere, quam post votum promissa non reddere. 
\verse Ne dederis os tuum, ut peccare faciat carnem tuam, neque dicas coram angelo: “Error fuit"; ne forte iratus Deus contra sermones tuos dissipet opera manuum tuarum. 
\verse Ubi multa sunt somnia, plurimae sunt vanitates et sermones innumeri; tu vero Deum time. 
\verse Si videris calumnias egenorum et subreptionem iudicii et iustitiae in provincia, non mireris super hoc negotio, quia excelso excelsior vigilat, et super hos quoque eminentiores sunt alii; 
\verse et terrae lucrum in omnibus est rex, cuius agri culti sunt. 
\verse Qui diligit pecuniam, pecunia non implebitur; et, qui amat divitias, fructum non capiet ex eis; et hoc ergo vanitas. 
\verse Ubi multae sunt opes, multi et qui comedunt eas; et quid prodest possessori, nisi quod cernit divitias oculis suis? 
\verse Dulcis est somnus operanti, sive parum sive multum comedat; saturitas autem divitis non sinit eum dormire. 
\verse Est et infirmitas pessima, quam vidi sub sole: divitiae conservatae in malum domini sui. 
\verse Perierunt enim in negotio pessimo; si generavit filium, in summa egestate erit. 
\verse Sicut egressus est de utero matris suae, nudus iterum abibit, sicut venit, et nihil auferet secum de labore suo, quod tollat in manu sua. 
\verse Miserabilis prorsus infirmitas: quomodo venit, sic revertetur. Quid ergo prodest ei quod laboravit in ventum? 
\verse Cunctis enim diebus vitae suae comedit in tenebris et in curis multis et in aerumna atque tristitia.  
\verse Ecce quod ego vidi bonum, quod pulchrum, ut comedat quis et bibat et fruatur laetitia ex labore suo, quo laboravit ipse sub sole, numero dierum vitae suae, quos dedit ei Deus; haec enim est pars illius. 
\verse Et quidem omni homini, cui dedit Deus divitias atque substantiam, potestatemque ei tribuit, ut comedat ex eis et tollat partem suam et laetetur de labore suo: hoc est donum Dei. 
\verse Non enim satis recordabitur dierum vitae suae, eo quod Deus occupet deliciis cor eius. 
\end{biblechapter}

\begin{biblechapter}  
\verse Est et aliud malum, quod vidi sub sole, et quidem grave apud homines:  
\verse vir, cui dedit Deus divitias et substantiam et honorem, et nihil deest animae suae ex omnibus, quae desiderat; nec tribuit ei potestatem Deus, ut comedat ex eo, sed homo extraneus vorabit illud: hoc vanitas et miseria mala est. 
\verse Si genuerit quispiam centum liberos et vixerit multos annos et plures dies aetatis habuerit, et anima illius non sit satiata bonis substantiae suae, immo et sepultura careat, de hoc ego pronuntio quod melior illo sit abortivus. 
\verse Frustra enim venit et pergit ad tenebras, et in tenebris abscondetur nomen eius.  
\verse Etsi non vidit solem neque cognovit, maior est requies isti quam illi. 
\verse Etiamsi duobus milibus annis vixerit et non fuerit perfruitus bonis, nonne ad unum locum properant omnes? 
\verse “Omnis labor hominis est ad os eius, sed anima eius non implebitur". 
\verse Quid habet amplius sapiens prae stulto? Et quid pauper, qui sciat ambulare coram vivis? 
\verse “Melior est oculorum visio quam vana persequi desideria"; sed et hoc vanitas est et afflictio spiritus. 
\verse Quidquid est, iam vocatum est nomen eius; et scitur quod homo sit et non possit contra fortiorem se in iudicio contendere. 
\verse Ubi verba sunt plurima, multiplicant vanitatem; quid lucri habet homo? 
\verse Quoniam quis scit quid homini bonum sit in vita, in paucis diebus vanitatis suae, quos peragit velut umbra? Aut quis ei poterit indicare quid post eum futurum sub sole sit? 
\end{biblechapter}

\begin{biblechapter}  
\verse Melius est nomen bonum quam unguenta pretiosa, et dies mortis die nativitatis. 
\verse Melius est ire ad domum luctus quam ad domum convivii; in illa enim finis cunctorum hominum, et vivens hoc conferet in corde. 
\verse Melior est tristitia risu, quia per tristitiam vultus corrigitur animus. 
\verse Cor sapientium in domo luctus, et cor stultorum in domo laetitiae. 
\verse Melius est a sapiente corripi quam laetari stultorum canticis, 
\verse quia sicut sonitus spinarum ardentium sub olla, sic risus stulti. Sed et hoc vanitas. 
\verse Quia calumnia stultum facit sapientem, et munus cor insanire facit. 
\verse “Melior est finis negotii quam principium, melior est patiens arrogante". 
\verse Ne sis velox in animo ad irascendum, quia ira in sinu stulti requiescit.  
\verse Ne dicas: “Quid, putas, causae est quod priora tempora meliora fuere quam nunc sunt?". Non enim ex sapientia interrogas de hoc. 
\verse Bona est sapientia cum divitiis et prodest videntibus solem. 
\verse Sicut enim protegit sapientia, sic protegit pecunia; hoc autem plus habet eruditio, quod sapientia vitam tribuit possessori suo. 
\verse Considera opera Dei: quod nemo possit corrigere, quod ille curvum fecerit. 
\verse In die bona fruere bonis et in die mala considera: sicut hanc, sic et illam fecit Deus, ita ut non inveniat homo quidquam de futuro. 
\verse Cuncta vidi in diebus vanitatis meae: est iustus, qui perit in iustitia sua, et impius, qui multo vivit tempore in malitia sua. 
\verse Noli esse nimis iustus neque sapiens supra modum! Cur te perdere vis? 
\verse Ne agas nimis impie et noli esse stultus! Cur mori debeas in tempore non tuo? 
\verse Bonum est ut, quod habes, teneas, sed et ab illo ne subtrahas manum tuam, quia qui timet Deum, utrumque devitat. 
\verse Sapientia confortabit sapientem super decem principes civitatis. 
\verse Nullus enim homo iustus in terra, qui faciat bonum et non peccet. 
\verse Sed et cunctis sermonibus, qui dicuntur, ne accommodes cor tuum, ne forte audias servum tuum maledicentem tibi; 
\verse scit enim conscientia tua, quia et tu crebro maledixisti aliis. 
\verse Cuncta tentavi in sapientia, dixi: “Sapiens efficiar". 
\verse Et ipsa longius recessit a me. Longe est, quod fuit; et alta est profunditas. Quis inveniet eam? 
\verse Lustravi universa animo meo, ut scirem et considerarem et quaererem sapientiam et rationem et ut cognoscerem impietatem esse stultitiam et errorem imprudentiam. 
\verse Et invenio amariorem morte mulierem, quae laqueus venatorum est, et sagena cor eius, vincula sunt manus illius. Qui placet Deo, effugiet eam; qui autem peccator est, capietur ab illa. 
\verse Ecce hoc inveni, dixit Ecclesiastes, unum et alterum, ut invenirem rationem, 
\verse quam adhuc quaerit anima mea, et non inveni: Hominem de mille unum repperi, mulierem ex omnibus non inveni. 
\verse Ecce solummodo hoc inveni: Quod fecerit Deus hominem rectum, et ipsi quaesierint infinitas quaestiones. 
\end{biblechapter}

\begin{biblechapter}  
\verse Quis talis, ut sapiens est? Et quis cognovit solutionem rerum? Sapientia hominis illuminat vultum eius, et durities faciei illius commutatur. 
\verse Os regis observa et propter iuramenta Dei 
\verse ne festines recedere a facie eius neque permaneas in re mala, quia omne, quod voluerit, faciet. 
\verse Quia sermo illius potestate plenus est, nec dicere ei quisquam potest: “Quare ita facis?". 
\verse Qui custodit praeceptum, non experietur quidquam mali; tempus et iudicium cor sapientis intellegit. 
\verse Omni enim negotio tempus est et iudicium, et multa hominis afflictio; 
\verse ignorat enim quid futurum sit, nam quomodo sit futurum, quis nuntiabit ei? 
\verse Non est in hominis potestate dominari super spiritum nec cohibere spiritum, nec habet potestatem supra diem mortis, nec ulla remissio est ingruente bello, neque salvabit impietas impium. 
\verse Omnia haec consideravi et dedi cor meum cunctis operibus, quae fiunt sub sole, quo tempore dominatur homo homini in malum suum. 
\verse Et ita vidi impios sepultos, discedentes de loco sancto; in oblivionem cadere in civitate, quod ita egerunt: sed et hoc vanitas est. 
\verse Etenim, quia non profertur cito sententia contra opera mala, ideo cor filiorum hominum repletur, ut perpetrent mala. 
\verse Nam peccator centies facit malum et prolongat sibi dies; verumtamen novi quod erit bonum timentibus Deum, qui verentur faciem eius. 
\verse Non sit bonum impio, nec prolongabit dies suos quasi umbram, qui non timet faciem Domini. 
\verse Est vanitas, quae fit super terram: sunt iusti, quibus mala proveniunt, quasi opera egerint impiorum, et sunt impii, quibus bona proveniunt, quasi iustorum facta habeant; sed et hoc vanissimum iudico. 
\verse Laudavi igitur laetitiam quod non esset homini bonum sub sole, nisi quod comederet et biberet atque gauderet et hoc solum secum auferret de labore suo in diebus vitae suae, quos dedit ei Deus sub sole. 
\verse Cum apposui cor meum, ut scirem sapientiam et intellegerem occupationem, quae versatur in terra, quod diebus et noctibus somnum non capit oculis, 
\verse ecce intellexi quod omnium operum Dei nullam possit homo invenire rationem eorum, quae fiunt sub sole; et quanto plus laboraverit homo ad quaerendum, tanto minus inveniet; etiamsi dixerit sapiens se nosse, non poterit reperire. 
\end{biblechapter}

\begin{biblechapter}  
\verse Omnia haec contuli in corde meo, ut curiose intellegerem quod iusti atque sapientes et opera eorum sunt in manu Dei. Utrum amor sit an odium, omnino nescit homo: coram illis omnia. 
\verse Sicut omnibus sors una: iusto et impio, bono et malo, mundo et immundo, immolanti victimas et non immolanti. Sicut bonus sic et peccator; ut qui iurat, ita et ille qui iuramentum timet. 
\verse Hoc est pessimum inter omnia, quae sub sole fiunt, quia sors eadem cunctis; unde et corda filiorum hominum implentur malitia et stultitia in vita sua, et novissima eorum apud mortuos. 
\verse Qui enim sociatur omnibus viventibus, habet fiduciam: melior est canis vivus leone mortuo. 
\verse Viventes enim sciunt se esse morituros; mortui vero nihil noverunt amplius nec habent ultra mercedem, quia oblivioni tradita est memoria eorum. 
\verse Amor quoque eorum et odium et invidiae simul perierunt, nec iam habent partem in hoc saeculo et in opere, quod sub sole geritur. 
\verse Vade ergo et comede in laetitia panem tuum et bibe cum gaudio vinum tuum, etenim iam diu placuerunt Deo opera tua. 
\verse Omni tempore sint vestimenta tua candida, et oleum de capite tuo non deficiat. 
\verse Perfruere vita cum uxore, quam diligis, cunctis diebus vitae instabilitatis tuae, qui dati sunt tibi sub sole omni tempore vanitatis tuae: haec est enim pars in vita et in labore tuo, quo laboras sub sole. 
\verse Quodcumque facere potest manus tua, instanter operare, quia nec opus nec ratio nec sapientia nec scientia erunt apud inferos, quo tu properas. 
\verse Verti me ad aliud et vidi sub sole nec velocium esse cursum nec fortium bellum nec sapientium panem nec doctorum divitias nec prudentium gratiam, sed tempus casumque in omnibus. 
\verse Insuper nescit homo finem suum, sed sicut pisces capiuntur sagena mala, et sicut aves laqueo comprehenduntur, sic capiuntur homines in tempore malo, cum eis extemplo supervenerit. 
\verse Hanc quoque sub sole vidi sapientiam et probavi maximam: 
\verse civitas parva, et pauci in ea viri; venit contra eam rex magnus et vallavit eam exstruxitque munitiones magnas per gyrum. 
\verse Inventusque est in ea vir pauper et sapiens et liberavit urbem per sapientiam suam; et nullus deinceps recordatus est hominis illius pauperis. 
\verse Et dicebam ego meliorem esse sapientiam fortitudine, sed sapientia pauperis contemnitur, et verba eius non sunt audita. 
\verse Verba sapientium cum lenitate audiuntur plus quam clamor principis inter stultos. 
\verse Melior est sapientia quam arma bellica; sed unus, qui peccaverit, multa bona perdet. 
\end{biblechapter}

\begin{biblechapter}  
\verse Muscae morientes perdunt et corrumpunt oleum unguentarii. Gravior quam sapientia et gloria est parva stultitia. 
\verse Cor sapientis in dextera eius, et cor stulti in sinistra illius. 
\verse Sed et in via stultus ambulans, cum ipse insipiens sit, omnes stultos aestimat. 
\verse Si spiritus potestatem habentis ascenderit contra te, locum tuum ne dimiseris, quia lenitas faciet cessare peccata maxima. 
\verse Est malum, quod vidi sub sole quasi errorem egredientem a facie principis:  
\verse positum stultum in dignitate sublimi et divites sedere deorsum. 
\verse Vidi servos in equis et principes ambulantes super terram quasi servos. 
\verse Qui fodit foveam, incidet in eam; et, qui dissipat murum, mordebit eum coluber. 
\verse Qui excidit lapides, affligetur in eis; et, qui scindit ligna, periclitabitur ex eis. 
\verse Si retusum fuerit ferrum, et aciem eius non exacueris, labor multiplicabitur, sed lucrum industriae erit sapientia. 
\verse Si mordeat serpens incantatione neglecta, nihil lucri habet incantator. 
\verse Verba oris sapientis gratia, et labia insipientis praecipitabunt eum. 
\verse Initium verborum eius stultitia, et novissimum oris illius insipientia mala. 
\verse Stultus verba multiplicat: “Ignorat homo quid futurum sit; et, quid post se futurum sit, quis ei poterit indicare?”. 
\verse Labor stultorum affliget eos, qui nesciunt in urbem pergere. 
\verse Vae tibi, terra, cuius rex puer est, et cuius principes mane comedunt. 
\verse Beata terra, cuius rex nobilis est, et cuius principes vescuntur in tempore suo ad reficiendum et non ad luxuriam. 
\verse In pigris manibus humiliabitur contignatio, et in remissis perstillabit domus. 
\verse In risum faciunt epulas; vinum laetificat vitam, et pecunia praestat omnia. 
\verse In cogitatione tua regi ne detrahas et in secreto cubiculi tui ne maledixeris diviti, quia et aves caeli portabunt vocem tuam, et, qui habet pennas, annuntiabit sententiam. 
\end{biblechapter}

\begin{biblechapter}  
\verse Mitte panem tuum super transeuntes aquas, quia post tempora multa invenies illum. 
\verse Da partem septem necnon et octo, quia ignoras, quid futurum sit mali super terram. 
\verse Si repletae fuerint nubes, imbrem super terram effundent; si ceciderit lignum ad austrum aut ad aquilonem, in quocumque loco ceciderit, ibi erit. 
\verse Qui observat ventum, non seminat, et, qui considerat nubes, numquam metet. 
\verse Quomodo ignoras, quae sit via spiritus, et qua ratione compingantur ossa in ventre praegnantis, sic nescis opera Dei, qui fabricator est omnium. 
\verse Mane semina semen tuum, et vespere ne cesset manus tua, quia nescis quid magis prosit, hoc aut illud, et si utrumque simul melius erit. 
\verse Dulce lumen, et delectabile est oculis videre solem. 
\verse Si annis multis vixerit homo et in his omnibus laetatus fuerit, meminisse debet tenebrosi temporis, quod multum erit: omne, quod venerit, vanitas. 
\verse Laetare ergo, iuvenis, in adulescentia tua, et in bono sit cor tuum in diebus iuventutis tuae, et ambula in viis cordis tui et in intuitu oculorum tuorum et scito quod pro omnibus his adducet te Deus in iudicium. 
\verse Aufer curam a corde tuo et amove malum a carne tua; adulescentia enim et iuventus vanae sunt. 
\end{biblechapter}

\begin{biblechapter}  
\verse Memento Creatoris tui in diebus iuventutis tuae, antequam veniat tempus afflictionis, et appropinquent anni, de quibus dicas: “Non mihi placent"; 
\verse antequam tenebrescat sol et lumen et luna et stellae, et revertantur nubes post pluviam; 
\verse quando commovebuntur custodes domus, et nutabunt viri fortissimi, et otiosae erunt molentes imminuto numero, et tenebrescent videntes per foramina, 
\verse et claudentur ostia in platea submissa voce molentis, et consurgent ad vocem volucris, et subsident omnes filiae carminis; 
\verse excelsa quoque timebunt et formidabunt in via. Florebit amygdalus, reptabit locusta, et dissipabitur capparis, quoniam ibit homo in domum aeternitatis suae, et circuibunt in platea plangentes, 
\verse antequam rumpatur funiculus argenteus, et frangatur lecythus aureus, et conteratur hydria super fontem, et confringatur rota super cisternam, 
\verse et revertatur pulvis in terram suam, unde erat, et spiritus redeat ad Deum, qui dedit illum. 
\verse Vanitas vanitatum, dixit Ecclesiastes, et omnia vanitas. 
\verse Cumque esset sapientissimus, Ecclesiastes docuit insuper populum scientiam; ponderavit et investigans composuit parabolas multas. 
\verse Quaesivit Ecclesiastes verba delectabilia et conscripsit sermones rectissimos ac veritate plenos. 
\verse Verba sapientium sicut stimuli, et quasi clavi defixi sunt magistri collationum; data sunt a pastore uno. 
\verse His amplius, fili mi, ne requiras: faciendi plures libros nullus est finis, frequensque meditatio carnis afflictio est. 
\verse Finis loquendi, omnibus auditis: Deum time et mandata eius observa; hoc est enim omnis homo. 
\verse Et cuncta, quae fiunt, adducet Deus in iudicium circa omne occultum, sive bonum sive malum.
\end{biblechapter}
