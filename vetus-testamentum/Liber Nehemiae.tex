\biblebook{Liber Nehemiae}

\begin{biblechapter}   
\verse Verba Nehemiae filii Hachaliae. Et factum est in mense Casleu, anno vicesimo, et ego eram in castro Susan. 
\verse Et venit Hanani unus de fratribus meis, ipse et viri ex Iuda; et interrogavi eos de Iudaeis, qui salvati erant et supererant de captivitate, et de Ierusalem. 
\verse Et dixerunt mihi: “Superstites, qui supererant de captivitate ibi in provincia, in afflictione magna sunt et in opprobrio; et murus Ierusalem dissipatus est, et portae eius combustae sunt igne". 
\verse Cumque audissem verba huiuscemodi, sedi et flevi et luxi diebus multis; ieiunabam et orabam ante faciem Dei caeli. 
\verse Et dixi: “Quaeso, Domine, Deus caeli, Deus fortis, magne atque terribilis, qui custodis pactum et misericordiam cum his, qui te diligunt et custodiunt mandata tua; 
\verse fiat auris tua auscultans, et oculi tui aperti, ut audias orationem servi tui, quam ego oro coram te hodie, die et nocte pro filiis Israel servis tuis, et confiteor pro peccatis filiorum Israel, quibus peccaverunt tibi. Ego quoque et domus patris mei peccavimus, 
\verse delinquentes deliquimus contra te et non custodivimus praecepta et mandata et iudicia, quae praecepisti Moysi famulo tuo. 
\verse Memento verbi, quod mandasti Moysi servo tuo dicens: “Cum transgressi fueritis, ego dispergam vos in populos; 
\verse si autem revertamini ad me et custodiatis praecepta mea et faciatis ea, etiamsi abducti fueritis in extrema caeli, inde congregabo vos et reducam in locum quem elegi, ut habitaret nomen meum ibi”. 
\verse Ipsi enim sunt servi tui et populus tuus, quos redemisti in fortitudine tua magna et in manu tua valida. 
\verse Obsecro, Domine, sit auris tua attendens ad orationem servi tui et ad orationem servorum tuorum, qui volunt timere nomen tuum; et fac servum tuum prosperari hodie et da ei gratiam ante virum hunc". Ego enim eram pincerna regis. 
\end{biblechapter}

\begin{biblechapter}  
\verse Factum est autem in mense Nisan, anno vicesimo Artaxerxis regis, dum biberet, levavi vinum et dedi regi; non enim eram ingratus coram eo. 
\verse Dixitque mihi rex: “Quare vultus tuus tristis est, cum te aegrotum non videam? Nihil est aliud nisi tristitia cordis". Et timui valde 
\verse et dixi regi: “Rex, in aeternum vive! Quare non maereat vultus meus, quia civitas sepulcrorum patrum meorum deserta est, et portae eius combustae sunt igne?". 
\verse Et ait mihi rex: “Pro qua re postulas?". Et oravi Deum caeli 
\verse et dixi ad regem: “Si videtur regi bonum, et si placet servus tuus ante faciem tuam, ut mittas me in Iudaeam ad civitatem sepulcrorum patrum meorum, et aedificabo eam". 
\verse Dixitque mihi rex, et regina sedebat iuxta eum: “Usque ad quod tempus erit iter tuum, et quando reverteris?". Et placuit regi mittere me; et constitui ei tempus. 
\verse Et dixi regi: “Si regi videtur bonum, epistulae dentur mihi ad duces regionis trans flumen, ut me transire permittant, donec veniam in Iudaeam; 
\verse et epistulam ad Asaph custodem saltus regis, ut det mihi ligna, ut contignare possim portas turris domus et muri civitatis et domus, in qua habitabo". Et dedit mihi rex, quia manus Dei mei bona super me. 
\verse Et veni ad duces regionis trans flumen dedique eis epistulas regis. Miserat autem rex mecum principes militum et equites. 
\verse Et audierunt Sanaballat Horonites et Thobias servus Ammanites et contristati sunt afflictione magna, quod venisset homo, qui quaereret prosperitatem filiorum Israel. 
\verse Et veni Ierusalem et eram ibi tribus diebus. 
\verse Et surrexi nocte ego, et viri pauci mecum, et non indicavi cuiquam quid Deus meus dedisset in corde meo, ut facerem in Ierusalem; et iumentum non erat mecum, nisi animal cui sedebam.  
\verse Et egressus sum per portam Vallis nocte et ad fontem Draconis et portam Sterquilinii et considerabam murum Ierusalem dissipatum et portas eius consumptas igne. 
\verse Et transivi ad portam Fontis et ad piscinam Regis, et non erat locus iumento cui sedebam, ut transiret. 
\verse Et ascendi per torrentem nocte et considerabam murum; et iterum veni ad portam Vallis et reversus sum. 
\verse Magistratus autem nesciebant quo abissem aut quid ego facerem, sed et Iudaeis et sacerdotibus et optimatibus et magistratibus et reliquis, qui faciebant opus, usque ad id loci nihil indicaveram. 
\verse Et dixi eis: “Vos nostis afflictionem, in qua sumus, quia Ierusalem deserta est, et portae eius consumptae sunt igne; venite et aedificemus murum Ierusalem et non simus ultra opprobrium". 
\verse Et indicavi eis quod manus Dei mei bona esset super me et verba regis, quae locutus esset mihi, et dixerunt: “Surgamus et aedificemus!". Et confortatae sunt manus eorum in bonum. 
\verse Audierunt autem Sanaballat Horonites et Thobias servus Ammanites et Gosem Arabs et subsannaverunt nos et despexerunt dixeruntque: “Quae est haec res, quam facitis? Numquid contra regem vos rebellatis?". 
\verse Et dedi eis responsum dicens: “Deus caeli ipse nos facit prosperari, et nos servi eius sumus; surgamus et aedificemus. Vobis autem non est pars et ius et memoria in Ierusalem". 
\end{biblechapter}

\begin{biblechapter}  
\verse Et surrexit Eliasib sacerdos magnus et fratres eius sacerdotes et aedificaverunt portam Gregis; contignaverunt eam et statuerunt valvas eius et usque ad turrim Meah et turrim Hananeel. 
\verse Et iuxta eos aedificaverunt viri Iericho, et iuxta eos aedificavit Zacchur filius Imri. 
\verse Portam autem Piscium aedificaverunt filii Asnaa; ipsi contignaverunt eam et statuerunt valvas eius et seras et vectes. 
\verse Et iuxta eos restauravit Meremoth filius Uriae filii Accos, et iuxta eum restauravit Mosollam filius Barachiae filii Mesezabel, et iuxta eum restauravit Sadoc filius Baana, 
\verse et iuxta eum restauraverunt Thecueni; optimates autem eorum non supposuerunt colla sua in opere Domini sui. 
\verse Et portam Veterem restauraverunt Ioiada filius Phasea et Mosollam filius Besodia; ipsi contignaverunt eam et statuerunt valvas eius et seras et vectes.  
\verse Et iuxta eos restauraverunt Meltias Gabaonites et Iadon Meronathites, viri de Gabaon et Maspha, qui erant ad solium ducis, qui erat in regione trans flumen;  
\verse et iuxta eos restauravit Oziel filius Araia de aurificibus, et iuxta eum restauravit Hananias de pigmentariis et firmaverunt Ierusalem usque ad murum latiorem. 
\verse Et iuxta eum restauravit Raphaia filius Hur, princeps dimidiae partis vici Ierusalem; 
\verse et iuxta eum restauravit Iedaia filius Haromaph contra domum suam, et iuxta eum restauravit Hattus filius Hasabneia. 
\verse Alteram partem restauravit Melchias filius Harim et Hassub filius Phahathmoab usque ad turrim Furnorum. 
\verse Et iuxta eos restauravit Sellum filius Alohes, princeps mediae partis vici Ierusalem, ipse et filiae eius. 
\verse Portam Vallis restauravit Hanun et habitatores Zanoa; ipsi aedificaverunt eam et statuerunt valvas eius et seras et vectes et mille cubitos in muro usque ad portam Sterquilinii. 
\verse Et portam Sterquilinii restauravit Melchias filius Rechab, princeps vici Bethcharem; ipse aedificavit eam et statuit valvas eius et seras et vectes. 
\verse Et portam Fontis restauravit Sellum filius Cholhoza princeps pagi Maspha; ipse aedificavit eam et texit et statuit valvas eius et seras et vectes et murum piscinae Siloae iuxta hortum regis et usque ad gradus, qui descendunt de civitate David. 
\verse Post eum restauravit Nehemias filius Azboc princeps dimidiae partis vici Bethsur usque contra sepulcra David et usque ad piscinam, quae repleta est, et usque ad domum Fortium. 
\verse Post eum restauraverunt Levitae, Rehum filius Bani; iuxta eum restauravit Hasabias princeps dimidiae partis vici Ceilae pro vico suo; 
\verse post eum aedificaverunt fratres eorum Bavai filius Henadad princeps dimidiae partis vici Ceilae. 
\verse Et restauravit iuxta eum Ezer filius Iesua princeps Maspha mensuram alteram contra ascensum armentarii in angulo. 
\verse Post eum restauravit Baruch filius Zachai mensuram alteram ab angulo usque ad portam domus Eliasib sacerdotis magni. 
\verse Post eum restauravit Meremoth filius Uriae filii Aecos mensuram secundam a porta domus Eliasib usque ad extremitatem domus Eliasib. 
\verse Et post eum restauraverunt sacerdotes viri de campestribus. 
\verse Post eos restauravit Beniamin et Hassub contra domum suam; post eos restauravit Azarias filius Maasiae filii Ananiae iuxta domum suam.  
\verse Post eum restauravit Bennui filius Henadad mensuram alteram a domo Azariae usque ad angulum et flexuram. 
\verse Phalel filius Ozi contra angulum turris, quae eminet de domo regis excelsa in atrio carceris; post eum Phadaia filius Pharos restauravit 
\verse usque contra portam Aquarum ad orientem et turrim, quae prominebat. 
\verse Post eum restauraverunt Thecueni mensuram alteram a regione contra magnam turrim eminentem usque ad murum templi. 
\verse Sursum autem a porta Equorum restauraverunt sacerdotes, unusquisque contra domum suam. 
\verse Post eos restauravit Sadoc filius Emmer contra domum suam; et post eum restauravit Semeia filius Secheniae custos portae orientalis. 
\verse Post eum restauravit Hanania filius Selemiae et Hanun filius Seleph sextus mensuram alteram. Post eum restauravit Mosollam filius Barachiae contra cellam suam. 
\verse Post eum restauravit Melchias de aurificibus usque ad domum oblatorum et mercatorum, contra portam Iudicialem, et usque ad cenaculum anguli; 
\verse et inter cenaculum anguli et portam Gregis restauraverunt aurifices et negotiatores. 
\verse Factum est autem, cum audisset Sanaballat quod aedificaremus murum, iratus est et indignatus est nimis et subsannavit Iudaeos 
\verse et dixit coram fratribus suis et optimatibus Samariae: “Quid Iudaei faciunt imbecilles? Num hoc conceditur eis? Num, quia sacrificant, complebunt in una die? Numquid vivificare poterunt lapides de acervis pulveris, qui combusti sunt?". 
\verse Sed et Thobias Ammanites, qui erat ad latus eius, ait: “Sine aedificare; si ascenderit vulpes, diruet murum eorum lapideum". 
\verse Audi, Deus noster, quia facti sumus irrisio! Converte contumeliam eorum super caput eorum et da eos in irrisionem in terra captivitatis! 
\verse Ne operias iniquitatem eorum, et peccatum eorum coram facie tua non deleatur, quia offenderunt te coram aedificantibus. 
\verse Itaque aedificavimus murum, et compositus est totus murus usque ad partem dimidiam, et populus dabat cor suum, ut operaretur. 
\end{biblechapter}

\begin{biblechapter}  
\verse Factum est autem cum audisset Sanaballat et Thobias et Arabes et Ammanitae et Azotii quod prosperaretur restauratio muri Ierusalem et quod coepissent interrupta concludi, irati sunt nimis; 
\verse et conspiraverunt omnes pariter, ut venirent et pugnarent contra Ierusalem et facerent confusionem.  
\verse Et oravimus Deum nostrum et posuimus custodiam die ac nocte contra eos. 
\verse Dixit autem Iudas: “Debilitata est fortitudo portantis, et humus nimia est; et nos non poterimus aedificare murum". 
\verse Et dixerunt hostes nostri: “Nesciant et ignorent, donec veniamus in medium eorum et interficiamus eos et cessare faciamus opus". 
\verse Factum est autem venientibus Iudaeis, qui habitabant iuxta eos, et dicentibus nobis per decem vices ex omnibus locis, quibus venerant ad nos, 
\verse statuimus nos in inferioribus post murum in locis apertis, et ordinavi populum secundum familias cum gladiis suis et lanceis suis et arcubus suis. 
\verse Et perspexi atque surrexi, et aio ad optimates et magistratus et ad reliquam partem vulgi: “Nolite timere a facie eorum; Domini magni et terribilis mementote et pugnate pro fratribus vestris, filiis vestris et filiabus vestris et uxoribus vestris et domibus vestris". 
\verse Factum est autem cum audissent inimici nostri nuntiatum esse nobis, dissipavit Deus consilium eorum, et reversi sumus omnes ad murum, unusquisque ad opus suum. 
\verse Et factum est a die illa, media pars iuvenum meorum faciebat opus, et media tenebat lanceas et scuta et arcus et loricas, et principes post omnem domum Iudae. 
\verse Aedificantium in muro et portantium onera et imponentium, una manu sua faciebat opus et altera tenebat gladium;  
\verse aedificantium enim unusquisque gladio erat accinctus renes, et sic aedificabant; et, qui clangebat bucina, iuxta me. 
\verse Et dixi ad optimates et ad magistratus et ad reliquam partem vulgi: “Opus grande est et latum, et nos separati sumus in muro procul alter ab altero; 
\verse in loco quocumque audieritis clangorem tubae, illuc concurrite ad nos. Deus noster pugnabit pro nobis". 
\verse Et sic nos fecimus opus, et media pars nostrum tenebat lanceas ab ascensu aurorae, donec egrediantur astra. 
\verse In tempore quoque illo dixi populo: “Unusquisque cum puero suo pernoctet in medio Ierusalem; et erit nobis custodia per noctem, et opus per diem". 
\verse Ego autem et fratres mei et pueri mei et custodes, qui erant post me, non deponebamus vestimenta nostra; unusquisque tenebat gladium in dextera sua. 
\end{biblechapter}

\begin{biblechapter}  
\verse Et factus est clamor populi et uxorum eius magnus adversus fratres suos Iudaeos. 
\verse Et erant qui dicerent: “Filios nostros et filias nostras pignoravimus, ut acciperemus frumentum et comederemus et viveremus!". 
\verse Et erant qui dicerent: “Agros nostros et vineas et domos nostras opposuimus, ut acciperemus frumentum in fame!". 
\verse Et alii dicebant: “Mutuo sumpsimus pecunias in tributa regis pro agris nostris et vineis nostris. 
\verse Et nunc sicut caro fratrum nostrorum sic caro nostra est, et sicut filii eorum ita et filii nostri; ecce nos subiugamus filios nostros et filias nostras in servitutem, et de filiabus nostris quaedam iam in servitute subiugatae sunt, nec habemus unde possint redimi, quia agros nostros et vineas nostras alii possident". 
\verse Et iratus sum nimis, cum audissem clamorem eorum secundum verba haec. 
\verse Cogitavique in corde meo et increpavi optimates et magistratus et dixi eis: “Usuras singuli a fratribus vestris exigitis!". Et congregavi adversum eos contionem magnam 
\verse et dixi eis: “Nos, ut scitis, redemimus fratres nostros Iudaeos, qui venditi fuerant gentibus, secundum possibilitatem nostram; quin potius et vos vendetis fratres vestros, ut vendentur nobis?". Et siluerunt nec invenerunt quid responderent. 
\verse Dixique ad eos: “Non est bona res, quam facitis. Quare non in timore Dei nostri ambulatis, ne exprobretur nobis a gentibus inimicis nostris? 
\verse Et ego et fratres mei et pueri mei commodavimus plurimis pecuniam et frumentum; non repetamus usuras istas. 
\verse Reddite eis hodie agros suos et vineas suas et oliveta sua et domos suas et centesimam pecuniae frumenti vini et olei, quam exigere soletis ab eis". 
\verse Et dixerunt: “Reddemus et ab eis nihil quaeremus; sicque faciemus, ut loqueris". Et vocavi sacerdotes et feci eos iurare, ut facerent, sicut dictum erat. 
\verse Insuper excussi sinum meum et dixi: “Sic excutiat Deus omnem virum, qui non compleverit verbum istud, de domo sua et de laboribus suis; sic excutiatur et vacuus fiat!". Et dixit universa multitudo: “Amen!". Et laudaverunt Deum. Fecit ergo populus, sicut erat dictum. 
\verse A die autem illa, qua praeceperat rex mihi, ut essem dux in terra Iudae, ab anno vicesimo usque ad annum tricesimum secundum Artaxerxis regis, per annos duodecim ego et fratres mei annonas, quae ducibus debebantur, non comedimus.  
\verse Duces autem priores, qui fuerant ante me, gravaverunt populum et acceperunt ab eis cotidie pro pane siclos argenti quadraginta; sed et ministri eorum depresserunt populum. Ego autem non feci ita propter timorem Dei, 
\verse quin potius in opere muri restauravi et agrum non emi; et omnes pueri mei congregati ad opus erant. 
\verse Iudaei quoque et magistratus, centum quinquaginta viri, et qui veniebant ad nos de gentibus, quae in circuitu nostro sunt, in mensa mea erant. 
\verse Parabatur autem mihi per dies singulos bos unus, arietes sex electi, exceptis volatilibus; et inter dies decem vina diversa multa. Insuper et annonas ducatus mei non quaesivi; gravis enim erat servitus populi huius. 
\verse Memento mei, Deus meus, in bonum, secundum omnia, quae feci populo huic. 
\end{biblechapter}

\begin{biblechapter}  
\verse Factum est autem cum audisset Sanaballat et Thobias et Gosem Arabs et ceteri inimici nostri, quod aedificassem ego murum, et non esset in ipso residua interruptio — usque ad tempus autem illud valvas non posueram in portis — 
\verse miserunt Sanaballat et Gosem ad me dicentes: “Veni, et conveniamus in Cephirim in campo Ono". Ipsi autem cogitabant, ut facerent mihi malum. 
\verse Misi ergo ad eos nuntios dicens: “Opus grande ego facio et non possum descendere; cur cessare oportet opus, si desistero et descendero ad vos?". 
\verse Miserunt autem ad me secundum verbum hoc per quattuor vices, et respondi eis iuxta sermonem priorem. 
\verse Et misit ad me Sanaballat iuxta verbum prius quinta vice puerum suum, et epistulam non obsignatam habebat in manu sua, in qua erat scriptum: 
\verse “In gentibus auditum est, et Gosem dixit quod tu et Iudaei cogitetis rebellare, et propterea aedifices murum et levare te velis super eos regem; iuxta hanc vocem  
\verse et prophetas posueris, qui praedicent de te in Ierusalem dicentes: “Rex in Iudaea est!”. Nunc autem auditurus est rex verba haec; idcirco nunc veni, ut ineamus consilium pariter". 
\verse Et misi ad eum dicens: “Non est factum secundum verba haec, quae tu loqueris; de corde enim tuo tu componis haec".  
\verse Omnes enim hi terrebant nos cogitantes: “Fatigabuntur manus eorum ab opere, et non complebitur". Quam ob causam magis confortavi manus meas. 
\verse Et ingressus sum domum Semeiae filii Dalaiae filii Meetabel, ubi erat detentus. Qui ait: “Tractemus nobiscum in domo Dei, in medio templi, et claudamus portas aedis, quia venturi sunt, ut interficiant te; utique nocte venturi sunt ad occidendum te”. 
\verse Et dixi: “Num quisquam similis mei fugit? Et quis ut ego ingredietur templum et vivet? Non ingrediar". 
\verse Et intellexi quod Deus non misisset eum, sed quasi vaticinans locutus esset ad me, quia Thobias et Sanaballat conduxerant eum. 
\verse Acceperat enim pretium, ut territus sic agerem et peccarem, et haberent malum, quod exprobrarent mihi. 
\verse Memento, Deus meus, Thobiae et Sanaballat iuxta opera eorum talia, sed et Noadiae prophetae et ceterorum prophetarum, qui terrebant me! 
\verse Completus est autem murus vicesimo quinto die mensis Elul, quinquaginta duobus diebus. 
\verse Factum est ergo, cum audissent omnes inimici nostri, et vidissent universae gentes, quae erant in circuitu nostro, ut conciderent intra semetipsos et scirent quod a Deo factum esset opus hoc. 
\verse Sed et in diebus illis, multae optimatum Iudaeorum epistulae mittebantur ad Thobiam, et a Thobia veniebant ad eos. 
\verse Multi enim in Iudaea coniurationem fecerunt cum eo, quia gener erat Secheniae filii Area, et Iohanan filius eius acceperat filiam Mosollam filii Barachiae. 
\verse Sed et laudabant eum coram me et verba mea nuntiabant ei; et Thobias mittebat epistulas, ut terreret me. 
\end{biblechapter}

\begin{biblechapter}  
\verse Postquam autem aedificatus est murus, et posui valvas et recensui ianitores et cantores et Levitas, 
\verse praeposui Hanani fratrem meum et Hananiam principem arcis supra Ierusalem — ipse enim quasi vir verax et timens Deum plus ceteris videbatur — 
\verse et dixi eis: “Non aperiantur portae Ierusalem usque ad calorem solis. Dum adhuc calor permanet, claudantur portae et oppilentur; et ponant custodes de habitatoribus Ierusalem, singulos per vices suas et unumquemque contra domum suam". 
\verse Civitas autem erat lata nimis et grandis, et populus parvus in medio eius, et non erant domus aedificatae. 
\verse Deus autem meus dedit in corde meo, et congregavi optimates et magistratus et vulgus, ut recenserem eos; et inveni librum census eorum, qui ascenderant primum, et inventum est scriptum in eo: 
\verse Isti filii provinciae, qui ascenderunt de captivitate migrantium, quos transtulerat Nabuchodonosor rex Babylonis, et reversi sunt in Ierusalem et in Iudaeam unusquisque in civitatem suam. 
\verse Qui venerunt cum Zorobabel, Iesua, Nehemias, Azarias, Raamias, Nahamani, Mardochaeus, Belsan, Mespharath, Beguai, Nahum, Baana. Numerus virorum populi Israel: 
\verse filii Pharos duo milia centum septuaginta duo; 
\verse filii Saphatia trecenti septuaginta duo; 
\verse filii Area sescenti quinquaginta duo; 
\verse filii Phahathmoab, hi sunt filii Iesua et Ioab, duo milia octingenti decem et octo; 
\verse filii Elam mille ducenti quinquaginta quattuor; 
\verse filii Zethua octingenti quadraginta quinque; 
\verse filii Zachai septingenti sexaginta; 
\verse filii Bennui sescenti quadraginta octo;  
\verse filii Bebai sescenti viginti octo; 
\verse filii Azgad duo milia trecenti viginti duo; 
\verse filii Adonicam sescenti sexaginta septem; 
\verse filii Beguai duo milia sexaginta septem; 
\verse filii Adin sescenti quinquaginta quinque; 
\verse filii Ater, qui erant ex Ezechia, nonaginta octo; 
\verse filii Hasum trecenti viginti octo; 
\verse filii Besai trecenti viginti quattuor;  
\verse filii Hareph centum duodecim; 
\verse filii Gabaon nonaginta quinque; 
\verse filii Bethlehem et Netopha centum octoginta octo; 
\verse viri Anathoth centum viginti octo; 
\verse viri Bethazmaveth quadraginta duo; 
\verse viri Cariathiarim, Cephira et Beroth septingenti quadraginta tres; 
\verse viri Rama et Gabaa sescenti viginti unus; 
\verse viri Machmas centum viginti duo; 
\verse viri Bethel et Hai centum viginti tres; 
\verse viri Nabo alterius quinquaginta duo; 
\verse viri Elam alterius mille ducenti quinquaginta quattuor; 
\verse filii Harim trecenti viginti; 
\verse filii Iericho trecenti quadraginta quinque;  
\verse filii Lod, Hadid et Ono septingenti viginti unus; 
\verse filii Senaa tria milia nongenti triginta. 
\verse Sacerdotes: filii Iedaia de domo Iesua nongenti septuaginta tres; 
\verse filii Emmer mille quinquaginta duo; 
\verse filii Phassur mille ducenti quadraginta septem; 
\verse filii Harim mille decem et septem. 
\verse Levitae: filii Iesua, hi sunt filii Cadmihel, Bennui et Odoviae, septuaginta quattuor. 
\verse Cantores: filii Asaph centum quadraginta octo. 
\verse Ianitores: filii Sellum, filii Ater, filii Telmon, filii Accub, filii Hatita, filii Sobai, centum triginta octo. 
\verse Oblati: filii Siha, filii Hasupha, filii Tabbaoth, 
\verse filii Ceros, filii Siaa, filii Phadon, 
\verse filii Lebana, filii Hagaba, filii Selmai, 
\verse filii Hanan, filii Giddel, filii Gaher, 
\verse filii Raaia, filii Rasin, filii Necoda, 
\verse filii Gazam, filii Oza, filii Phasea, 
\verse filii Besai, filii Meunitarum, filii Nephusorum, 
\verse filii Bacbuc, filii Hacupha, filii Harhur,  
\verse filii Basluth, filii Mahida, filii Harsa, 
\verse filii Bercos, filii Sisara, filii Thema, 
\verse filii Nasia, filii Hatipha. 
\verse Filii servorum Salomonis: filii Sotai, filii Sophereth, filii Pheruda, 
\verse filii Iaala, filii Darcon, filii Giddel, 
\verse filii Saphatia, filii Hatil, filii Phochereth Hassebaim, filii Amon. 
\verse Omnes oblati et filii servorum Salomonis trecenti nonaginta duo. 
\verse Hi sunt autem, qui ascenderunt de Thelmela, Thelharsa, Cherub, Addon et Emmer et non potuerunt indicare domum patrum suorum et semen suum, utrum ex Israel essent: 
\verse filii Dalaia, filii Thobia, filii Necoda sescenti quadraginta duo. 
\verse Et de sacerdotibus: filii Hobia, filii Accos, filii Berzellai, qui accepit de filiabus Berzellai Galaaditis uxorem et vocatus est nomine eorum.  
\verse Hi quaesierunt tabulas genealogiae suae et non invenerunt; et eiecti sunt de sacerdotio; 
\verse dixitque praepositus eis, ut non manducarent de sanctificatis sanctuarii, donec staret sacerdos pro Urim et Tummim. 
\verse Omnis multitudo simul quadraginta duo milia trecenti sexaginta, 
\verse absque servis et ancillis eorum, qui erant septem milia trecenti triginta septem; insuper et cantores et cantatrices ducenti quadraginta quinque. 
\verse Equi eorum septingenti triginta sex, muli eorum ducenti quadraginta quinque, 
\verse cameli eorum quadringenti triginta quinque, asini sex milia septingenti viginti. 
\verse Nonnulli autem de principibus familiarum dederunt in opus: praepositus dedit in thesaurum auri drachmas mille, phialas quinquaginta, tunicas sacerdotales quingentas triginta; 
\verse et de principibus familiarum dederunt in thesaurum operis auri drachmas viginti milia et argenti minas duo milia ducentas. 
\verse Et quod dedit reliquus populus, auri drachmas viginti milia et argenti minas duo milia et tunicas sacerdotales sexaginta septem. Habitaverunt autem ibi sacerdotes et Levitae; ianitores autem et cantores et quidam de populo et oblati et omnis Israel habitaverunt in civitatibus suis. Et venerat mensis septimus; filii autem Israel erant in civitatibus suis. 
\end{biblechapter}

\begin{biblechapter}  
\verse Congregatusque est omnis populus quasi vir unus ad plateam, quae est ante portam Aquarum, et dixerunt Esdrae scribae, ut afferret librum legis Moysi, quam praece perat Dominus Israeli. 
\verse Attulit ergo Esdras sacerdos legem coram multitudine virorum et mulierum cunctisque, qui poterant intellegere, in die prima mensis septimi. 
\verse Et legit in eo in platea, quae erat ante portam Aquarum, de mane usque ad mediam diem in conspectu virorum et mulierum et eorum, qui intellegere poterant; et aures omnis populi erant erectae ad librum legis. 
\verse Stetit autem Esdras scriba super gradum ligneum, quem ad hoc fecerant; et steterunt iuxta eum Matthathias et Sema et Anaia et Uria et Helcia et Maasia ad dexteram eius, et ad sinistram Phadaia, Misael et Melchia et Hasum et Hasbadana, Zacharia et Mosollam. 
\verse Et aperuit Esdras librum coram omni populo — super universum quippe populum eminebat — et, cum aperuisset eum, stetit omnis populus. 
\verse Et benedixit Esdras Domino, Deo magno; et respondit omnis populus: “Amen, amen", elevans manus suas. Et incurvati sunt et adoraverunt Deum proni in terram. 
\verse Porro Iesua et Bani et Serebia, Iamin, Accub, Sabethai, Hodia, Maasia, Celita, Azarias, Iozabad, Hanan, Phalaia et Levitae erudiebant populum in lege; populus autem stabat in gradu suo. 
\verse Et legerunt in libro legis Dei distincte et aperierunt sensum et explicaverunt lectionem. 
\verse Dixit autem Nehemias, ipse est praepositus, et Esdras sacerdos et scriba et Levitae instruentes populum universo populo: “Dies iste sanctificatus est Domino Deo nostro! Nolite lugere et nolite flere". Flebat enim omnis populus, cum audiret verba legis. 
\verse Et dixit eis: “Ite, comedite pinguia et bibite mulsum et mittite partes his, qui non praeparaverunt sibi, quia sanctus dies Domini nostri est; et nolite contristari, gaudium etenim Domini est fortitudo vestra". 
\verse Levitae autem silentium faciebant in omni populo dicentes: “Tacete, quia dies sanctus est, et nolite dolere". 
\verse Abiit itaque omnis populus, ut comederet et biberet et mitteret partes et faceret laetitiam magnam, quia intellexerant verba, quae docuerat eos. 
\verse Et in die secundo congregati sunt principes familiarum universi populi, sacerdotes et Levitae ad Esdram scribam, ut intellegerent verba legis. 
\verse Et invenerunt scriptum in lege, quam praecepit Dominus per Moysen, ut habitent filii Israel in tabernaculis in die sollemni mense septimo 
\verse et ut praedicent et divulgent vocem in universis urbibus suis et in Ierusalem dicentes: “Egredimini in montem et afferte frondes olivae et frondes oleastri, frondes myrti et ramos palmarum et frondes ligni nemorosi, ut fiant tabernacula, sicut scriptum est". 
\verse Et egressus est populus, et attulerunt feceruntque sibi tabernacula, unusquisque in domate suo et in atriis suis et in atriis domus Dei et in platea portae Aquarum et in platea portae Ephraim. 
\verse Fecit ergo universa ecclesia eorum, qui redierant de captivitate, tabernacula et habitaverunt in tabernaculis. Non enim fecerant a diebus Iosue filii Nun taliter filii Israel usque ad diem illum; et fuit laetitia magna nimis. 
\verse Legit autem in libro legis Dei per dies singulos, a die primo usque ad diem novissimum; et fecerunt sollemnitatem septem diebus et in die octavo conventum iuxta ordinationem. 
\end{biblechapter}

\begin{biblechapter}  
\verse In die autem vicesimo quarto mensis huius convenerunt filii Israel in ieiunio et in saccis, et humus super eos. 
\verse Et separatum est semen filiorum Israel ab omni alienigena; et steterunt et confitebantur peccata sua et iniquitates patrum suorum. 
\verse Et consurrexerunt ad standum et legerunt in volumine legis Domini Dei sui per quartam partem diei; et per quartam partem confitebantur et adorabant Dominum Deum suum. 
\verse Surrexerunt autem super gradum Levitarum Iesua et Bani et Cadmihel, Sebania, Bunni, Serebia, Bani et Chanani et clamaverunt voce magna ad Dominum Deum suum. 
\verse Et dixerunt Levitae Iesua et Cadmihel, Bani, Hasabneia, Serebia, Hodia, Sebania, Phethahia: “Surgite, benedicite Domino Deo vestro ab aetemo usque in aeternum, et benedicant nomini gloriae tuae excelso super omnem benedictionem et laudem. 
\verse Tu ipse, Domine, solus; tu fecisti caelum et caelum caelorum et omnem exercitum eorum, terram et universa, quae in ea sunt, maria et omnia, quae in eis sunt; et tu vivificas omnia haec, et exercitus caeli te adorat. 
\verse Tu ipse, Domine Deus, qui elegisti Abram et eduxisti eum de Ur Chaldaeorum et posuisti nomen eius Abraham. 
\verse Et invenisti cor eius fidele coram te et percussisti cum eo foedus, ut dares terram Chananaei, Hetthaei et Amorraei et Pherezaei et Iebusaei et Gergesaei, nempe ut dares semini eius; et implesti verba tua, quoniam iustus es. 
\verse Et vidisti afflictionem patrum nostrorum in Aegypto clamoremque eorum audisti iuxta mare Rubrum. 
\verse Et dedisti signa atque portenta in pharaone et in universis servis eius et in omni populo terrae illius; cognovisti enim quia superbe egerant contra eos, et fecisti tibi nomen, sicut et in hac die. 
\verse Et mare divisisti ante eos, et transierunt per medium maris in sicco; persecutores autem eorum proiecisti in profundum, quasi lapidem in aquas validas. 
\verse Et in columna nubis ductor eorum fuisti per diem et in columna ignis per noctem, ut illuminaret eis viam, per quam ingrediebantur. 
\verse Ad montem quoque Sinai descendisti et locutus es cum eis de caelo; et dedisti eis iudicia recta et legem rectam, mandata et praecepta bona. 
\verse Et sabbatum sanctificatum tuum ostendisti eis et praecepta et mandata et legem praecepisti eis in manu Moysi servi tui. 
\verse Panem quoque de caelo dedisti eis in fame eorum et aquam de petra eduxisti eis in siti eorum; et dixisti eis, ut ingrederentur et possiderent terram, super quam levasti manum tuam, ut traderes eis. 
\verse Ipsi vero patres nostri superbe egerunt et induraverunt cervices suas et non audierunt mandata tua. 
\verse Et noluerunt audire et non sunt recordati mirabilium tuorum, quae feceras eis, et induraverunt cervices suas et posuerunt caput suum, ut reverterentur ad servitutem suam in Aegyptum. Tu autem Deus propitius, clemens et misericors, longanimis et multae miserationis, non dereliquisti eos. 
\verse Et quidem, cum fecissent sibi vitulum conflatilem et dixissent: “Iste est Deus tuus, qui eduxit te de Aegypto” feceruntque blasphemias magnas; 
\verse tu autem in misericordiis tuis multis non dimisisti eos in deserto: columna nubis non recessit ab eis per diem, ut duceret eos in viam; et columna ignis per noctem, ut illuminaret eis iter, per quod ingrederentur. 
\verse Et spiritum tuum bonum dedisti, qui doceret eos, et manna tuum non prohibuisti ab ore eorum et aquam dedisti eis in siti eorum. 
\verse Quadraginta annis pavisti eos in deserto, nihilque eis defuit; vestimenta eorum non inveteraverunt, et pedes eorum non intumuerunt. 
\verse Et dedisti eis regna et populos et partitus es eis sortes; et possederunt terram Sehon et terram regis Hesebon et terram Og regis Basan. 
\verse Et multiplicasti filios eorum sicut stellas caeli; et adduxisti eos ad terram, de qua dixeras patribus eorum, ut ingrederentur et possiderent. 
\verse Et venerunt filii et possederunt terram, et humiliasti coram eis habitatores terrae Chananaeos; et dedisti eos in manu eorum et reges eorum et populos terrae, ut facerent eis, sicut placebat illis. 
\verse Ceperunt itaque urbes munitas et humum pinguem; et possederunt domos plenas cunctis bonis, cisternas ab aliis fabricatas, vineas et oliveta et ligna pomifera multa. Et comederunt et saturati sunt et impinguati sunt et delectati sunt in bonitate tua magna. 
\verse Vexaverunt autem te et rebellaverunt contra te et proiecerunt legem tuam post terga sua; et prophetas tuos occiderunt, qui contestabantur eos, ut reverterentur ad te; feceruntque blasphemias grandes. 
\verse Et dedisti eos in manu hostium suorum, et afflixerunt eos; et in tempore tribulationis suae clamaverunt ad te, et tu de caelo audisti et secundum miserationes tuas multas dedisti eis salvatores, qui salvarent eos da manu hostium suorum. 
\verse Cumque requievissent, reversi sunt, ut facerent malum in conspectu tuo; et dereliquisti eos in manu inimicorum suorum, et dominati sunt eis. Conversique sunt et clamaverunt ad te; tu autem de caelo exaudisti et liberasti eos in misericordiis tuis multis vicibus. 
\verse Et contestatus es eos, ut reduceres eos ad legem tuam; ipsi vero superbe egerunt et non audierunt mandata tua et in iudicia tua peccaverunt, quae si fecerit homo, vivet in eis, et dederunt umerum rebellem et cervicem suam induraverunt nec audierunt. 
\verse Et pepercisti eis annos multos et contestatus es eos in spiritu tuo per manum prophetarum tuorum, et non audierunt; et tradidisti eos in manu populorum terrarum. 
\verse In misericordiis autem tuis plurimis non fecisti eos in consumptionem nec dereliquisti eos; quoniam Deus misericors et clemens es tu. 
\verse Nunc itaque, Deus noster magne, fortis et terribilis, custodiens pactum et misericordiam, ne parvipendas omnem laborem, qui invenit nos, reges nostros et principes nostros et sacerdotes nostros et prophetas nostros et patres nostros et omnem populum tuum a diebus regum Assyriae usque in diem hanc. 
\verse Et tu iustus es in omnibus, quae venerunt super nos, quia recte fecisti, nos autem impie egimus. 
\verse Reges nostri, principes nostri, sacerdotes nostri et patres nostri non fecerunt legem tuam et non attenderunt mandata tua et testimonia tua, quae testificatus es in eis. 
\verse Et ipsi in regnis suis et in bonitate tua multa, quam dederas eis, et in terra latissima et pingui, quam tradideras in conspectu eorum, non servierunt tibi nec reversi sunt a studiis suis pessimis. 
\verse Ecce nos ipsi hodie servi sumus; et in terra, quam dedisti patribus nostris, ut comederent fructum eius et bona eius, nos ipsi servi sumus. 
\verse Et fruges eius multiplicantur regibus, quos posuisti super nos propter peccata nostra, et corporibus nostris dominantur et iumentis nostris secundum voluntatem suam, et in tribulatione magna sumus". 
\end{biblechapter}

\begin{biblechapter}  
\verse “Super omnibus ergo his nos ipsi percutimus foedus et scribimus, et signant principes nostri, Levitae nostri et sacerdotes nostri". 
\verse Signatores autem fuerunt: Nehemias praepositus, filius Hachaliae, et Sedecias, 
\verse Saraias, Azarias, Ieremias, 
\verse Phassur, Amarias, Melchias, 
\verse Hattus, Sebania, Melluch, 
\verse Harim, Meremoth, Abdias, 
\verse Daniel, Genthon, Baruch, 
\verse Mosollam, Abia, Miamin, 
\verse Maazia, Belgai, Semeia; hi sacerdotes. 
\verse Porro Levitae: Iesua filius Azaniae, Bennui de filiis Henadad, Cadmihel 
\verse et fratres eorum Sebania, Hodia, Celita, Phalaia, Hanan, 
\verse Micha, Rohob, Hasabia, 
\verse Zacchur, Serebia, Sebania, 
\verse Hodia, Bani, Baninu. 
\verse Capita populi: Pharos, Phahathmoab, Elam, Zethua, Bani, 
\verse Bunni, Azgad, Bebai, 
\verse Adonia, Beguai, Adin, 
\verse Ater, Ezechia, Azur, 
\verse Hodia, Hasum, Besai, 
\verse Hareph, Anathoth, Nebai, 
\verse Megphias, Mosollam, Hezir, 
\verse Mesezabel, Sadoc, Ieddua, 
\verse Pheltia, Hanan, Anaia, 
\verse Osee, Hanania, Hassub, 
\verse Alohes, Phalea, Sobec, 
\verse Rehum, Hasabna, Maasia, 
\verse Ahia, Hanan, Anan, 
\verse Melluch, Harim, Baana. 
\verse Et reliqui de populo, sacerdotes, Levitae, ianitores et cantores, oblati et omnes, qui se separaverunt de populis terrarum ad legem Dei, uxores eorum, filii eorum et filiae eorum, omnes, qui poterant sapere, 
\verse adhaeserunt fratribus suis optimatibus pollicentes et iurantes, ut ambularent in lege Dei, quam dederat in manu Moysi servi Dei, et ut facerent et custodirent universa mandata Domini Dei nostri et iudicia eius et praecepta eius, 
\verse et ut non daremus filias nostras populo terrae et filias eorum non acciperemus filiis nostris. 
\verse Et si populi terrae importaverint venalia et omnia cibaria per diem sabbati, ut vendant, non accipiemus ab eis in sabbato et in die sanctificato; et dimittemus annum septimum et omnem exactionem. 
\verse Et statuimus super nos praecepta, ut demus tertiam partem sicli per annum ad opus domus Dei nostri,  
\verse ad panes propositionis et ad oblationem sempiternam et in holocaustum sempiternum in sabbatis, in calendis, in sollemnitatibus et in sanctificata et in sacrificium pro peccato, ut expietur pro Israel, et in omnem usum domus Dei nostri. 
\verse Sortes ergo misimus super oblationem lignorum inter sacerdotes et Levitas et populum, ut inferrentur in domum Dei nostri per domos patrum nostrorum, in temporibus constitutis ab anno in annum, ut arderent super altare domini Dei nostri, sicut scriptum est in lege; 
\verse et ut afferremus primogenita terrae nostrae et primitiva universi fructus omnis ligni ab anno in annum in domo Domini, 
\verse et primitiva filiorum nostrorum et pecorum nostrorum, sicut scriptum est in lege, et primitiva boum nostrorum et ovium nostrarum, ut afferrentur in domum Dei nostri sacerdotibus, qui ministrant in domo Dei nostri;  
\verse et primitias ciborum nostrorum et libaminum nostrorum et poma omnis ligni, vindemiae quoque et olei, afferemus sacerdotibus ad gazophylacium Dei nostri, et decimam partem terrae nostrae Levitis. Ipsi Levitae decimas accipient ex omnibus civitatibus agriculturae nostrae. 
\verse Erit autem sacerdos filius Aaron cum Levitis in decimis Levitarum colligendis, et Levitae offerent decimam partem decimae in domo Dei nostri ad gazophylacium thesauri. 
\verse Ad gazophylacium enim deportabunt filii Israel et filii Levi primitias frumenti, vini et olei; et ibi erunt vasa sanctificata et sacerdotes, qui ministrabant, et ianitores et cantores. Et non dimittemus domum Dei nostri. 
\end{biblechapter}

\begin{biblechapter}  
\verse Habitaverunt autem principes populi in Ierusalem; reliqua vero plebs misit sortem, ut adducerent unum virum de decem ad habitandum in Ierusalem civitate sancta, novem vero partes in civitatibus. 
\verse Benedixit autem populus omnibus viris, qui se sponte obtulerant, ut habitarent in Ierusalem. 
\verse Hi sunt itaque principes provinciae, qui habitaverunt in Ierusalem et in civitatibus Iudae. Habitavit autem unusquisque in possessione sua, in urbibus suis, Israel, sacerdotes, Levitae, oblati et filii servorum Salomonis. 
\verse Et in Ierusalem habitaverunt de filiis Iudae et de filiis Beniamin. De filiis Iudae: Athaias filius Oziam filii Zachariae filii Amariae filii Saphatiae filii Malaleel, de filiis Phares; 
\verse et Maasia filius Baruch filius Cholhoza filius Hazia filius Adaia filius Ioiarib filius Zachariae filius Silonitis. 
\verse Omnes filii Phares, qui habitaverunt in Ierusalem, quadringenti sexaginta octo viri fortes. 
\verse Hi sunt autem filii Beniamin: Sallu filius Mosollam filius Ioed filius Phadaia filius Colaia filius Maasia filius Etheel filius Iesaia; 
\verse et fratres eius viri fortes, nongenti viginti octo. 
\verse Et Ioel filius Zechri praepositus eorum, et Iudas filius Asana super civitatem secundus. 
\verse Et de sacerdotibus: Iedaia filius Ioiarib filius 
\verse Saraia filius Helciae filius Mosollam filius Sadoc filius Meraioth filius Achitob princeps domus Dei; 
\verse et fratres eorum facientes opera templi, octingenti viginti duo. Et Adaia filius Ieroham filius Phelelia filius Amsi filius Zachariae filius Phassur filius Melchiae; 
\verse et fratres eius principes familiarum ducenti quadraginta duo. Et Amassai filius Azareel filius Ahazi filius Mosollamoth filius Emmer;  
\verse et fratres eorum potentes nimis, centum viginti octo; et praepositus eorum Zabdiel vir nobilis. 
\verse Et de Levitis: Semeia filius Hassub filius Ezricam filius Hasabia filius Bunni; 
\verse et Sabethai et Iozabad super omnia opera, quae erant forinsecus in domo Dei, de principibus Levitarum; 
\verse et Matthania filius Micha filius Zebedaei filius Asaph magister chori incohabat orationem; et Becbecia secundus de fratribus eius, et Abda filius Sammua filius Galal filius Idithun. 
\verse Omnes Levitae in civitate sancta ducenti octoginta quattuor. 
\verse Et ianitores: Accub, Telmon et fratres eorum, qui custodiebant ostia, centum septuaginta duo. 
\verse Et reliqui ex Israel sacerdotes et Levitae in universis civitatibus Iudae, unusquisque in possessione sua. 
\verse Et oblati habitabant in Ophel; et Siha et Gaspha super oblatos. 
\verse Et praefectus Levitarum in Ierusalem Ozi filius Bani filius Hasabiae filius Matthaniae filius Michae de filiis Asaph, cantores in ministerio domus Dei.  
\verse Praeceptum quippe regis super eos erat, et ordo in cantoribus per dies singulos. 
\verse Et Phethahia filius Mesezabel de filiis Zara filii Iudae, legatus regis in omni negotio populi. 
\verse Et in viculis per omnes regiones eorum, de filiis Iudae habitaverunt in Cariatharbe et in pagis eius et in Dibon et in pagis eius et in Cabseel et in viculis eius 
\verse et in Iesua et in Molada et in Bethpheleth 
\verse et in Asarsual et in Bersabee et in pagis eius 
\verse et in Siceleg et in Mochona et in pagis eius 
\verse et in Remmon et in Saraa et in Ierimoth, 
\verse Zanoa, Odollam et in villis earum, Lachis et regionibus eius et Azeca et pagis eius. Et habitaverunt a Bersabee usque ad vallem Ennom. 
\verse Filii autem Beniamin in Gabaa, Machmas et Hai et Bethel et pagis eius, 
\verse Anathoth, Nob, Anania, 
\verse Asor, Rama, Getthaim, 
\verse Hadid, Seboim et Neballat, 
\verse Lod et Ono et valle Artificum. 
\verse Et de Levitis portiones in Iuda et Beniamin. 
\end{biblechapter}

\begin{biblechapter}  
\verse Hi sunt autem sacerdotes et Levitae, qui ascenderunt cum Zorobabel filio Salathiel et Iesua: Saraia, Ieremias, Esdras, 
\verse Amaria, Melluch, Hattus,  
\verse Sechenias, Rehum, Meremoth, 
\verse Addo, Genthon, Abia, 
\verse Miamin, Maadia, Belga, 
\verse Semeia et Ioiarib, Iedaia, 
\verse Sallu, Amoc, Helcias, Iedaia. Isti principes sacerdotum et fratrum eorum in diebus Iesua. 
\verse Porro Levitae: Iesua, Bennui, Cadmihel, Serebia, Iuda, Matthanias, super hymnos ipse et fratres eius; 
\verse et Becbecia atque Hanni fratres eorum coram eis per vices suas. 
\verse Iesua autem genuit Ioachim, et Ioachim genuit Eliasib, et Eliasib genuit Ioiada, 
\verse et Ioiada genuit Ionathan, et Ionathan genuit Ieddua. 
\verse In diebus autem Ioachim erant sacerdotes principes familiarum: Saraiae Maraia, Ieremiae Hanania, 
\verse Esdrae Mosollam, Amariae Iohanan, 
\verse Milicho Ionathan, Sebaniae Ioseph, 
\verse Harim Edna, Meraioth Helci, 
\verse Adaiae Zacharia, Genthon Mosollam, 
\verse Abiae Zechri, Miamin Maadiae Phelti,  
\verse Belgae Sammua, Semeiae Ionathan, 
\verse Ioiarib Matthanai, Iedaiae Ozi,  
\verse Sellai Celai, Amoc Heber, 
\verse Helciae Hasabia, Iedaiae Nathanael. 
\verse Levitae in diebus Eliasib et Ioiada et Iohanan et Ieddua scripti principes familiarum et sacerdotes usque ad regnurn Darii Persae. 
\verse Filii Levi principes familiarum scripti in libro Chronicorum usque ad dies Ionathan filii Eliasib. 
\verse Et principes Levitarum Hasabia, Serebia, Iesua, Bennui et Cadmihel et fratres eorum coram eis, ut laudarent et confiterentur iuxta praeceptum David viri Dei per vices suas; 
\verse Matthania et Becbecia, Abdia, Mosollam, Telmon, Accub ianitores ad custodiam horreorum iuxta portas. 
\verse Hi in diebus Ioachim filii Iesua filii Iosedec et in diebus Nehemiae ducis et Esdrae sacerdotis scribaeque. 
\verse In dedicatione autem muri Ierusalem requisierunt Levitas de omnibus locis suis, ut adducerent eos in Ierusalem et facerent dedicationem in laetitia, in actione gratiarum et cantico et cymbalis, psalteriis et citharis. 
\verse Congregati sunt autem cantores de campestribus circa Ierusalem et de villis Netophathitarum 
\verse et de Bethgalgala et de regionibus Gabaa et Azmaveth, quoniam villas aedificaverunt sibi cantores in circuitu Ierusalem. 
\verse Et mundati sunt sacerdotes et Levitae et mundaverunt populum et portas et murum. 
\verse Ascendere autem feci principes Iudae super murum et statui duos magnos choros laudantium, quorum unus ivit ad dexteram super murum ad portam Sterquilinii. 
\verse Et ivit post eos Osaias et media pars principum Iudae  
\verse et Azarias, Esdras et Mosollam, 
\verse Iudas et Beniamin et Semeia et Ieremias. 
\verse Et de sacerdotibus cum tubis et Zacharias filius Ionathan filius Semeiae filius Matthaniae filius Michaiae filius Zacchur filius Asaph;  
\verse et fratres eius Semeia et Azareel, Malalai, Galalai, Maai, Nathanael et Iudas et Hanani cum musicis David viri Dei; et Esdras scriba ante eos et in porta Fontis. 
\verse Processerunt per gradus civitatis David in ascensu muri super domum David et usque ad portam Aquarum ad orientem. 
\verse Et chorus secundus gratias referentium ibat ex adverso, et ego post eum, et media pars populi super murum et super turrim Furnorum et usque ad murum latissimum 
\verse et super portam Ephraim et super portam Antiquam et super portam Piscium et turrim Hananeel et turrim Meah et usque ad portam Gregis; et steterunt in porta Custodiae. 
\verse Steteruntque duo chori laudantium in domo Dei, et ego et dimidia pars magistratuum mecum. 
\verse Et sacerdotes Eliachim, Maasia, Miamin, Michaia, Elioenai, Zacharia, Hanania in tubis; 
\verse et Maasia et Semeia et Eleazar et Ozi et Iohanan et Melchia et Elam et Ezer. Et clare cecinerunt cantores et Izrahia praepositus. 
\verse Et obtulerunt in die illa sacrificia magna et laetati sunt; Deus enim laetificaverat eos laetitia magna; sed et uxores eorum et liberi gavisi sunt, et audita est laetitia Ierusalem procul. 
\verse Praeposuerunt quoque in die illa viros super gazophylacia ad thesaurum, ad libamina et ad primitias et ad decimas, ut colligerent in ea de agris civitatum partes legitimas pro sacerdotibus et Levitis; quia laetificatus est Iuda in sacerdotibus et Levitis, qui adstiterunt 
\verse et servierunt in ministerio Dei sui et in ministerio purificationis simul cum cantoribus et ianitoribus iuxta praeceptum David et Salomonis filii eius; 
\verse quia in diebus David et Asaph ab exordio erant catervae cantorum et carmina laudis et actionis gratiarum Deo.  
\verse Et omnis Israel in diebus Zorobabel et in diebus Nehemiae dabant partes cantoribus et ianitoribus per dies singulos partem suam et partes consecrabant Levitis, et Levitae consecrabant filiis Aaron. 
\end{biblechapter}

\begin{biblechapter}  
\verse In die autem illo lectum est in volumine Moysi, audiente populo, et inventum est scriptum in eo quod non debeant introire Ammonites et Moabites in ecclesiam Dei usque in aeternum, 
\verse eo quod non occurrerint filiis Israel cum pane et aqua et conduxerint adversum eos Balaam ad maledicendum eis, et convertit Deus noster maledictionem in benedictionem. 
\verse Factum est autem, cum audissent legem, separaverunt omnem promiscuum ab Israel. 
\verse Ante hoc autem erat Eliasib sacerdos, qui fuerat praepositus in gazophylacio domus Dei nostri et proximus Thobiae; 
\verse fecerat ei gazophylacium grande, ubi antea reponebant munera et tus et vasa et decimam frumenti, vini et olei, partes Levitarum et cantorum et ianitorum et tributa sacerdotum. 
\verse In omnibus autem his non fui in Ierusalem, quia anno tricesimo secundo Artaxerxis regis Babylonis veni ad regem et in fine dierum rogavi, ut abirem a rege, 
\verse et veni in Ierusalem. Et intellexi malum, quod fecerat Eliasib Thobiae: fecerat enim ei thesaurum in vestibulis domus Dei. 
\verse Et malum mihi visum est valde, et proieci vasa domus Thobiae foras de gazophylacio; 
\verse praecepique, et emundaverunt gazophylacia, et rettuli ibi vasa domus Dei, oblationem et tus. 
\verse Et cognovi quod partes Levitarum non fuissent datae, et fugisset unusquisque in campum suum de Levitis et cantoribus, qui ministrabant. 
\verse Et egi causam adversus magistratus et dixi: “Quare dereliquimus domum Dei?". Et congregavi eos et feci stare in stationibus suis. 
\verse Et omnis Iuda apportabat decimam frumenti, vini et olei in horrea. 
\verse Et constitui super horrea Selemiam sacerdotem et Sadoc scribam et Phadaiam de Levitis et iuxta eos Hanan filium Zacchur, filium Matthaniae, quoniam fideles comprobati sunt; et ipsi curam habebant distribuendi partes fratribus suis. 
\verse Memento mei, Deus meus, pro hoc; et ne deleas opera mea bona, quae feci in domo Dei mei et in ministeriis eius! 
\verse In diebus illis vidi in Iuda calcantes torcularia in sabbato, portantes acervos et onerantes super asinos vinum et uvas et ficus et omne onus et inferentes in Ierusalem die sabbati; et contestatus sum, quando vendebant cibaria. 
\verse Et ibi Tyrii habitaverunt in ea inferentes pisces et omnia venalia et vendebant in sabbatis filiis Iudae in Ierusalem. 
\verse Et obiurgavi optimates Iudae et dixi eis: “Quae est haec res mala, quam vos facitis, et profanatis diem sabbati? 
\verse Numquid non haec fecerunt patres nostri, et adduxit Deus noster super nos omne malum hoc et super civitatem hanc? Et vos additis iracundiam super Israel profanando sabbatum!". 
\verse Factum est autem, cum obscuratae essent portae Ierusalem ante diem sabbati, dixi, et clauserunt ianuas; et praecepi, ut non aperirent eas usque post sabbatum. Et de pueris meis constitui super portas, ut nullus inferret onus in die sabbati. 
\verse Et manserunt negotiatores et vendentes universa venalia foris Ierusalem semel et bis. 
\verse Et contestatus sum eos et dixi eis: “Quare manetis ex adverso muri? Si iterum hoc feceritis, manum mittam in vos". Itaque ex tempore illo non venerunt in sabbato. 
\verse Dixi quoque Levitis, ut mundarentur et venirent ad custodiendas portas et sanctificandam diem sabbati. Et pro hoc ergo memento mei, Deus meus, et parce mihi secundum multitudinem miserationum tuarum! 
\verse Sed et in diebus illis vidi Iudaeos, qui duxerant uxores Azotidas, Ammonitidas et Moabitidas. 
\verse Et filii eorum ex media parte loquebantur Azotice et nesciebant loqui Iudaice vel loquebantur iuxta linguam unius vel alterius populi. 
\verse Et obiurgavi eos et maledixi et cecidi quosdam ex eis et decalvavi eos; et adiuravi in Deo, ut non darent filias suas filiis eorum et non acciperent de filiabus eorum filiis suis et sibimetipsis dicens: 
\verse “Numquid non in huiuscemodi re peccavit Salomon rex Israel? Et certe in gentibus multis non erat rex similis ei, et dilectus Deo suo erat, et posuit eum Deus regem super omnem Israel; et ipsum ergo duxerunt ad peccatum mulieres alienigenae. 
\verse Numquid et vobis obsequentes faciemus omne malum grande hoc, ut praevaricemur in Deo nostro et ducamus uxores peregrinas?". 
\verse Unus autem de filiis Ioiada filii Eliasib sacerdotis magni gener erat Sanaballat Horonites, quem fugavi a me. 
\verse Recordare, Domine Deus meus, adversum eos, qui polluunt sacerdotium et pactum sacerdotale et leviticum!  
\verse Igitur mundavi eos ab omnibus alienigenis et constitui ordines pro sacerdotibus et Levitis, unumquemque in ministerio suo, 
\verse et pro oblatione lignorum in temporibus constitutis et pro primitiis. Memento mei, Deus meus, in bonum.
\end{biblechapter}
