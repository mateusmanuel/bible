\biblebook{ Prophetia Ionae}
\begin{biblechapter}
 \verse Et factum est verbum Domini ad Ionam filium Amathi dicens: 
\verse “ Surge et vade in Nineven civitatem grandem et praedica in ea, quia ascendit malitia eius coram me ”. 
\verse Et surrexit Ionas, ut fugeret in Tharsis a facie Domini; et descendit Ioppen et invenit navem euntem in Tharsis et dedit naulum eius et descendit in eam, ut iret cum eis in Tharsis a facie Domini.
 \verse Dominus autem misit ventum magnum in mare, et facta est tempestas magna in mari, et navis periclitabatur conteri. 
\verse Et timuerunt nautae et clamaverunt unusquisque ad deum suum et miserunt vasa, quae erant in navi, in mare, ut alleviaretur ab eis. Ionas autem descenderat ad interiora navis et, cum recubuisset, dormiebat sopore gravi. 
\verse Et accessit ad eum gubernator et dixit ei: “ Quid? Tu sopore deprimeris? Surge, invoca Deum tuum, si forte recogitet Deus de nobis, et non pereamus ”.
 \verse Et dixit unusquisque ad collegam suum: “ Venite, et mittamus sortes, ut sciamus quare hoc malum sit nobis ”. Et miserunt sortes, et cecidit sors super Ionam. 
\verse Et dixerunt ad eum: “ Indica nobis cuius causa malum istud sit nobis. Quod est opus tuum, et unde venis? Quae terra tua, et ex quo populo es tu? ”. 
\verse Et dixit ad eos: “ Hebraeus ego sum et Dominum, Deum caeli, ego timeo, qui fecit mare et aridam ”. 
\verse Et timuerunt viri timore magno et dixerunt ad eum: “ Quid hoc fecisti? ”. Cognoverant enim viri quod a facie Domini fugeret, quia indicaverat eis.
 \verse Et dixerunt ad eum: “ Quid faciemus tibi, ut conticescat mare a nobis? ”. Mare enim magis ac magis intumescebat. 
\verse Et dixit ad eos: “ Tollite me et mittite in mare, et cessabit mare a vobis; scio enim ego quoniam propter me tempestas haec grandis super vos ”.
 \verse Et remigabant viri, ut reverterentur ad aridam; et non valebant, quia mare magis intumescebat super eos. 
\verse Et clamaverunt ad Dominum et dixerunt: “ Quaesumus, Domine, ne pereamus in anima viri istius, et ne des super nos sanguinem innocentem; quia tu, Domine, sicut voluisti, fecisti ”. 
\verse Et tulerunt Ionam et miserunt in mare; et stetit mare a fervore suo. 
\verse Et timuerunt viri timore magno Dominum et immolaverunt hostias Domino et voverunt vota.
 
\begin{biblechapter}
\verse Et praeparavit Dominus piscem grandem, ut deglutiret Ionam; et erat Ionas in ventre piscis tribus diebus et tribus noctibus.
 \verse Et oravit Ionas ad Dominum Deum suum de ventre piscis 
\verse et dixit: “ Clamavi de tribulatione mea ad Dominum,
 et respondit mihi;
 de ventre inferi clamavi,
 et exaudisti vocem meam.
 \verse Et proiecisti me in profundum in corde maris,
 et flumen circumdedit me;
 omnes gurgites tui et fluctus tui
 super me transierunt.
 \verse Et ego dixi: “Abiectus sum
 a conspectu oculorum tuorum;
 verumtamen rursus videbo
 templum sanctum tuum”.
 \verse Circumdederunt me aquae usque ad guttur,
 abyssus vallavit me,
 iuncus alligatus est capiti meo.
 \verse Ad extrema montium descendi,
 terrae vectes concluserunt me in aeternum,
 sed eduxisti de fovea vitam meam,
 Domine Deus meus.
 \verse Cum angustiaretur in me anima mea,
 Domini recordatus sum,
 et venit ad te oratio mea,
 ad templum sanctum tuum.
 \verse Qui colunt idola vana,
 pietatem suam derelinquunt;
 \verse ego autem in voce laudis
 immolabo tibi,
 quaecumque vovi, reddam;
 salus Domini est ”.
 \verse Et dixit Dominus pisci, et evomuit Ionam in aridam.
 
\begin{biblechapter}
\verse Et factum est verbum Domini ad Ionam secundo dicens:
 \verse “ Surge, vade in Nineven civitatem magnam et praedica in ea praedicationem, quam ego loquor ad te ”. 
\verse Et surrexit Ionas et abiit in Nineven iuxta verbum Domini.
 Et Nineve erat civitas magna coram Deo, itinere trium dierum. 
\verse Et coepit Ionas introire in civitatem itinere diei unius; et clamavit et dixit: “ Adhuc quadraginta dies, et Nineve subvertetur ”.
 \verse Et crediderunt viri Ninevitae in Deo; et praedicaverunt ieiunium et vestiti sunt saccis a maiore usque ad minorem. 
\verse Et pervenit verbum ad regem Nineve; et surrexit de solio suo et abiecit pallium suum a se et indutus est sacco et sedit in cinere. 
\verse Et clamavit et dixit in Nineve decreto regis et principum eius dicens: “ Homines et iumenta et boves et pecora non gustent quidquam nec pascantur et aquam non bibant; 
\verse et operiantur saccis homines et iumenta et clament ad Deum in fortitudine, et convertatur vir a via sua mala et a violentia, quae est in manibus eorum. 
\verse Quis scit si convertatur et ignoscat Deus et revertatur a furore irae suae, et non peribimus? ”.
 \verse Et vidit Deus opera eorum, quia conversi sunt de via sua mala; et misertus est Deus super malum, quod lo cutus fuerat ut faceret eis, et
 non fecit. 
\begin{biblechapter}
\verse Et afflictus est Ionas afflictione magna et iratus est; 
\verse et oravit ad Dominum et dixit: “ Obsecro, Domine, numquid non hoc est verbum meum, cum adhuc essem in terra mea? Propter hoc praeoccupavi ut fugerem in Tharsis. Sciebam enim quia tu Deus clemens et misericors es, longanimis et multae miserationis et ignoscens super malitia. 
\verse Et nunc, Domine, tolle, quaeso, animam meam a me, quia melior est mihi mors quam vita”. 
\verse Et dixit Dominus: “ Putasne bene irasceris tu? ”.
 \verse Et egressus est Ionas de civitate et sedit contra orientem civitatis et fecit sibimet umbraculum ibi et sedebat subter illud in umbra, donec videret quid accideret in civitate. 
\verse Et praeparavit Dominus Deus hederam, et ascendit super Ionam, ut esset umbra super caput eius et protegeret eum ab afflictione sua. Et laetatus est Ionas super hedera laetitia magna.
 \verse Et paravit Deus vermem, cum surgeret aurora in crastinum, et percussit hederam, quae exaruit. 
\verse Et, cum ortus fuisset sol, praecepit Deus vento orientali calido; et percussit sol super caput Ionae, et elanguit; et petivit animae suae, ut moreretur, et dixit: “ Melius est mihi mori quam vivere ”
 \verse Et dixit Deus ad Ionam: “ Putasne bene irasceris tu super hedera? ”. Et dixit: “ Bene irascor ego usque ad mortem ”. 
\verse Et dixit Dominus: “ Tu doles super hederam, in qua non laborasti neque fecisti, ut cresceret, quae sub una nocte nata est et sub una nocte periit. 
\verse Et ego non parcam Nineve civitati magnae, in qua sunt plus quam centum viginti milia hominum, qui nesciunt quid sit inter dexteram et sinistram suam, et iumenta multa? ”.
    
\end{biblechapter}
