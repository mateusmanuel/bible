\biblebook{ Liber I Samuelis}
\begin{biblechapter}
 \verse Fuit vir unus de Ramathaim Suphita de monte Ephraim, et nomen eius Elcana filius Ieroham filii Eliu filii Thohu filii Suph, Ephrathaeus. 
\verse Et habuit duas uxores: nomen uni Anna et nomen secundae Phenenna. Fueruntque Phenennae filii, Annae autem non erant liberi.
 \verse Et ascendebat vir ille de civitate sua singulis annis, ut adoraret et sacrificaret Domino exercituum in Silo. Erant autem ibi duo filii Heli, Ophni et Phinees, sacerdotes Domini.
 \verse Venit ergo dies, et immolavit Elcana dabatque Phenennae uxori suae et cunctis filiis eius et filiabus partes; 
\verse Annae autem dabat unam partem electam, quia Annam diligebat; Dominus autem concluserat vulvam eius. 
\verse Affligebat quoque eam aemula eius et vehementer angebat, ut conturbaret eam, quod conclusisset Dominus vulvam eius. 
\verse Sicque faciebat per singulos annos, cum, redeunte tempore, ascenderent templum Domini, et sic provocabat eam. Porro illa flebat et non capiebat cibum. 
\verse Dixit ergo ei Elcana vir suus: “ Anna, cur fles et quare non comedis? Et quam ob rem affligitur cor tuum? Numquid non ego melior sum tibi quam decem filii? ”.
 \verse Surrexit autem Anna, postquam comederant et biberant in Silo, et Heli sacerdote sedente super sellam ante postes templi Domini. 
\verse Cum esset Anna amaro animo, oravit Dominum flens largiter 
\verse et votum vovit dicens: “ Domine exercituum, si respiciens videris afflictionem famulae tuae et recordatus mei fueris nec oblitus ancillae tuae dederisque servae tuae sexum virilem, dabo eum Domino omnes dies vitae eius, et novacula non ascendet super caput eius ”.
 \verse Factum est ergo, cum illa multiplicaret preces coram Domino, ut Heli observaret os eius. 
\verse Porro Anna loquebatur in corde suo; tantumque labia illius movebantur, et vox penitus non audiebatur. Aestimavit igitur eam Heli temulentam 
\verse dixitque ei: “ Usquequo ebria eris? Digere paulisper vinum, quo mades! ”. 
\verse Respondens Anna: “ Nequaquam, inquit, domine mi; nam mulier infelix nimis ego sum: vinumque et omne, quod inebriare potest, non bibi, sed effudi animam meam in conspectu Domini. 
\verse Ne reputes ancillam tuam quasi unam de filiabus Belial, quia ex multitudine doloris et maeroris mei locuta sum usque in praesens ”. 
\verse Tunc Heli ait ei: “ Vade in pace, et Deus Israel det tibi petitionem, quam rogasti eum ”. 
\verse Et illa dixit: “ Utinam inveniat ancilla tua gratiam in oculis tuis ”. Et abiit mulier in viam suam et comedit; vultusque illius non fuerunt amplius sicut prius. 
\verse Et surrexerunt mane et adoraverunt coram Domino.
 Reversique sunt et venerunt in domum suam in Rama. Cognovit autem Elcana Annam uxorem suam, et recordatus est eius Dominus. 
\verse Et factum est post circulum dierum concepit Anna et peperit filium vocavitque nomen eius Samuel, eo quod a Domino postulasset eum.
 \verse Ascendit autem vir Elcana et omnis domus eius, ut immolaret Domino hostiam annuam et votum suum. 
\verse Et Anna non ascendit; dixit enim viro suo: “ Non vadam, donec ablactetur infans, et ducam eum, et appareat ante conspectum Domini et maneat ibi iugiter ”. 
\verse Et ait ei Elcana vir suus: “ Fac, quod bonum tibi videtur, et mane, donec ablactes eum; precorque, ut impleat Dominus verbum suum ”. Mansit ergo mulier et lactavit filium suum, donec amoveret eum a lacte.
 \verse Et adduxit eum secum, postquam ablactaverat, cum vitulo trium annorum et tribus modiis farinae et utre vini; et adduxit eum ad domum Domini in Silo. Puer autem erat adhuc infantulus. 
\verse Et immolaverunt vitulum et obtulerunt puerum Heli, 
\verse et ait Anna: “ Obsecro, mi domine; vivit anima tua, domine, ego sum illa mulier, quae steti coram te hic orans Dominum. 
\verse Pro puero isto oravi, et dedit mihi Dominus petitionem meam, quam postulavi eum. 
\verse Idcirco et ego commodavi eum Domino; cunctis diebus, quibus vivet, postulatus erit pro Domino ”.
 Et adoraverunt ibi Dominum.
 
\begin{biblechapter}
\verse Et oravit Anna et ait:
 “ Exsultavit cor meum in Domino,
 exaltatum est cornu meum in Deo meo;
 dilatatum est os meum super inimicos meos,
 quoniam laetata sum in salutari tuo.
 \verse Non est sanctus ut est Dominus;
 neque enim est alius extra te,
 et non est fortis sicut Deus noster.
 \verse Nolite multiplicare loqui sublimia gloriantes.
 Recedant superba de ore vestro,
 quia Deus scientiarum Dominus est, et ab eo ponderantur actiones.
 \verse Arcus fortium confractus est,
 et infirmi accincti sunt robore.
 \verse Saturati prius pro pane se locaverunt,
 et famelici non eguerunt amplius.
 Sterilis peperit plurimos,
 et, quae multos habebat filios, emarcuit.
 \verse Dominus mortificat et vivificat,
 deducit ad infernum et reducit.
 \verse Dominus pauperem facit et ditat,
 humiliat et sublevat;
 \verse suscitat de pulvere egenum
 et de stercore elevat pauperem,
 ut sedeat cum principibus
 et solium gloriae teneat.
 Domini enim sunt cardines terrae, et posuit super eos orbem.
 \verse Pedes sanctorum suorum servabit,
 et impii in tenebris conticescent,
 quia non in fortitudine sua roborabitur vir.
 \verse Dominus conteret adversarios suos;
 super ipsos in caelis tonabit.
 Dominus iudicabit fines terrae
 et dabit imperium regi suo
 et sublimabit cornu christi sui ”.
 \verse Et abiit Elcana in Rama in domum suam. Puer autem erat minister in conspectu Domini ante faciem Heli sacerdotis.
 \verse Porro filii Heli filii Belial nescientes Dominum 
\verse neque officium sacerdotum ad populum, sed, quicumque immolasset victimam, veniebat puer sacerdotis, dum coquerentur carnes, et habebat fuscinulam tridentem in manu sua 
 \verse et mittebat eam in lebetem vel in caldariam aut in ollam sive in cacabum et omne, quod levabat fuscinula, tollebat sacerdos sibi. Sic faciebant universo Israeli venienti in Silo. 
\verse Etiam, antequam adolerent adipem, veniebat puer sacerdotis et dicebat immolanti: “ Da mihi carnem, ut coquam sacerdoti; non enim accipiet a te carnem coctam sed crudam ”. 
\verse Dicebatque illi immolans: “ Incendatur primum iuxta morem hodie adeps, et tolle tibi, quantumcumque desiderat anima tua ”. Qui respondens aiebat ei: “ Nequaquam; nunc enim dabis, alioquin tollam vi ”. 
\verse Erat ergo peccatum puerorum grande nimis coram Domino, quia detrahebant sacrificio Domini.
 \verse Samuel autem ministrabat ante faciem Domini, puer accinctus ephod lineo. 
 \verse Et tunicam parvam faciebat ei mater sua, quam afferebat ei singulis annis ascendens cum viro suo, ut immolaret hostiam annuam. 
\verse Et benedicebat Heli Elcanae et uxori eius dicebatque: “ Reddat tibi Dominus semen de muliere hac pro petitione, quae postulata est pro Domino ”. Et abierunt in locum suum. 
\verse Visitavit ergo Dominus Annam, et concepit et peperit tres filios et duas filias. Et crevit puer Samuel apud Dominum.
 \verse Heli autem erat senex valde et audivit omnia, quae faciebant filii sui universo Israeli et quomodo dormiebant cum mulieribus, quae ministrabant ad ostium tabernaculi, 
\verse et dixit eis: “ Quare facitis res huiuscemodi, quas ego audio, res pessimas, ab omni populo? 
\verse Nolite, filii mei; non enim est bona fama, quam ego audio, ut transgredi faciatis populum Domini.
 \verse Si peccaverit vir in virum,
 arbiter ei potest esse Deus;
 si autem in Dominum peccaverit vir,
 quis intercedet pro eo? ”.
 Et non audierunt vocem patris sui, quia voluit Dominus occidere eos.
 \verse Puer autem Samuel proficiebat atque crescebat et placebat tam Domino quam hominibus.
 \verse Venit autem vir Dei ad Heli et ait ad eum: “ Haec dicit Dominus: Numquid non aperte revelatus sum domui patris tui, cum esset in Aegypto in domo pharaonis? 
 \verse Et elegi eum ex omnibus tribubus Israel mihi in sacerdotem, ut ascenderet ad altare meum et adoleret mihi incensum et portaret ephod coram me; et dedi domui patris tui omnia de sacrificiis filiorum Israel. 
\verse Quare calce abicitis victimam meam et munera mea, quae praecepi, ut offerrentur in templo, et magis honorasti filios tuos quam me, ut impinguaremini primitiis omnis sacrificii Israel populi mei?
 \verse Propterea ait Dominus, Deus Israel: Loquens locutus sum, ut domus tua et domus patris tui ministraret in conspectu meo usque in sempiternum. Nunc autem, dicit Dominus, absit hoc a me. Sed quicumque glorificaverit me, glorificabo eum; qui autem contemnunt me, erunt ignobiles. 
\verse Ecce dies veniunt, et praecidam brachium tuum et brachium domus patris tui, ut non sit senex in domo tua. 
\verse Et videbis aemulum tuum in templo in universis prosperis Israel; et non erit senex in domo tua omnibus diebus. 
\verse Verumtamen non auferam penitus virum ex te ab altari meo; sed ut deficiant oculi tui, et tabescat anima tua, et pars magna domus tuae morietur, cum ad virilem aetatem venerit. 
\verse Hoc autem erit tibi signum, quod venturum est duobus filiis tuis Ophni et Phinees: in die uno morientur ambo.
 \verse Et suscitabo mihi sacerdotem fidelem, qui iuxta cor meum et animam meam faciat; et aedificabo ei domum fidelem, et ambulabit coram christo meo cunctis diebus. 
\verse Futurum est autem ut quicumque remanserit in domo tua, veniat, ut procidat ante illum pro nummo argenteo et torta panis dicatque: “Dimitte me, obsecro, ad unam partem sacerdotalem, ut comedam buccellam panis” ”.
 
\begin{biblechapter}
\verse Puer autem Samuel ministrabat Domino coram Heli. Et sermo Domini erat pretiosus in diebus illis: non erat visio frequens. 
\verse Factum est ergo in die quadam, Heli iacebat in loco suo, et oculi eius caligaverant, nec poterat videre. 
\verse Lucerna Dei nondum exstincta erat, et Samuel dormiebat in templo Domini, ubi erat arca Dei. 
\verse Et vocavit Dominus Samuel, qui respondens ait: “ Ecce ego ”. 
\verse Et cucurrit ad Heli et dixit: “ Ecce ego; vocasti enim me ”. Qui dixit: “ Non vocavi. Revertere; dormi! ”. Et abiit et dormivit.
 \verse Et Dominus rursum vocavit Samuel. Consurgensque Samuel abiit ad Heli et dixit: “ Ecce ego, quia vocasti me ”. Qui respondit: “ Non vocavi te, fili mi. Revertere et dormi! ”. 
\verse Porro Samuel necdum sciebat Dominum, neque revelatus fuerat ei sermo Domini.
 \verse Et Dominus rursum vocavit Samuel tertio, qui consurgens abiit ad Heli 
\verse et ait: “ Ecce ego, quia vocasti me ”. Intellexit igitur Heli quia Dominus vocaret puerum, et ait ad Samuel: “ Vade et dormi; et, si deinceps vocaverit te, dices: “ Loquere, Domine, quia audit servus tuus” ”. Abiit ergo Samuel et dormivit in loco suo.
 \verse Et venit Dominus et stetit et vocavit, sicut vocaverat prius: “ Samuel, Samuel ”. Et ait Samuel: “ Loquere, quia audit servus tuus ”. 
\verse Et dixit Dominus ad Samuel: “ Ecce ego facio verbum in Israel, quod quicumque audierit, tinnient ambae aures eius. 
\verse In die illo suscitabo adversum Heli omnia, quae locutus sum super domum eius: incipiam et complebo. 
\verse Praedixi enim ei quod iudicaturus essem domum eius in aeternum propter iniquitatem, eo quod noverat filios suos contemnere Deum et non corripuit eos. 
\verse Idcirco iuravi domui Heli quod non expietur iniquitas domus eius victimis et muneribus usque in aeternum ”.
 \verse Dormivit autem Samuel usque mane aperuitque ostia domus Domini. Et Samuel timebat indicare visionem Heli. 
\verse Vocavit ergo Heli Samuelem et dixit: “ Samuel, fili mi ”. Qui respondens ait: “ Praesto sum ”. 
\verse Et interrogavit eum: “ Quis est sermo, quem locutus est ad te? Oro te, ne celaveris me. Haec faciat tibi Deus et haec addat, si absconderis a me sermonem ex omnibus verbis, quae dicta sunt tibi ”. 
\verse Indicavit itaque ei Samuel universos sermones et non abscondit ab eo. Et ille respondit: “ Dominus est! Quod bonum est in oculis suis, faciat ”.
 \verse Crevit autem Samuel, et Dominus erat cum eo, et non cecidit ex omnibus verbis eius in terram. 
\verse Et cognovit universus Israel a Dan usque Bersabee quod constitutus esset Samuel propheta Domini. 
\verse Et addidit Dominus ut appareret in Silo, quoniam revelatus fuerat Dominus Samueli in Silo iuxta verbum Domini. Et evenit sermo Samuelis universo Israeli.
 
\begin{biblechapter}
\verse Et factum est in diebus illis, convenerunt Philisthim in pugnam; et egressus est Israel obviam Philisthim in proelium et castrametatus est iuxta Abenezer. Porro Philisthim venerunt in Aphec 
\verse et instruxerunt aciem contra Israel. Crescente autem certamine, terga vertit Israel Philisthaeis; et caesi sunt in illo certamine passim per agros quasi quattuor milia virorum.
 \verse Et reversus est populus ad castra, dixeruntque maiores natu de Israel: “ Quare percussit nos Dominus hodie coram Philisthim? Afferamus ad nos de Silo arcam foederis Domini, et veniat in medium nostri, ut salvet nos de manu inimicorum nostrorum ”. 
\verse Misit ergo populus in Silo, et tulerunt inde arcam foederis Domini exercituum sedentis super cherubim; erantque duo filii Heli cum arca foederis Dei, Ophni et Phinees.
 \verse Cumque venisset arca foederis Domini in castra, vociferatus est omnis Israel clamore grandi, et personuit terra. 
\verse Et audierunt Philisthim vocem clamoris dixeruntque: “ Quaenam est haec vox clamoris magni in castris Hebraeorum? ”. Et cognoverunt quod arca Domini venisset in castra. 
\verse Timueruntque Philisthim dicentes: “ Venit Deus in castra! ”. Et ingemuerunt dicentes: 
\verse “ Vae nobis! Non enim fuit tanta exsultatio heri et nudiustertius. Vae nobis! Quis nos servabit de manu deorum sublimium istorum? Hi sunt dii, qui percusserunt Aegyptum omni plaga in deserto. 
\verse Confortamini et estote viri, Philisthim, ne serviatis Hebraeis, sicut illi servierunt vobis. Estote viri et bellate! ”.
 \verse Pugnaverunt ergo Philisthim, et caesus est Israel, et fugit unusquisque in tabernaculum suum; et facta est plaga magna nimis, et ceciderunt de Israel triginta milia peditum. 
\verse Et arca Dei capta est; duoque filii Heli mortui sunt, Ophni et Phinees.
 \verse Currens autem vir de Beniamin ex acie venit in Silo in die illo scissa veste et conspersus pulvere caput. 
\verse Cumque ille venisset, Heli sedebat super sellam iuxta portam aspectans viam; erat enim cor eius pavens pro arca Dei. Vir autem ille, postquam ingressus est, nuntiavit urbi; et ululavit omnis civitas. 
 \verse Et audivit Heli sonitum clamoris dixitque: “ Quis est hic sonitus tumultus huius? ”. At ille festinavit et venit et nuntiavit Heli. 
\verse Heli autem erat nonaginta et octo annorum, et oculi eius caligaverant, et videre non poterat. 
 \verse Et dixit ad Heli: “ Ego sum qui veni de proelio et ego qui de acie fugi hodie ”. Cui ille ait: “ Quid actum est, fili mi? ”. 
\verse Respondens autem, qui nuntiabat: “ Fugit, inquit, Israel coram Philisthim, et ruina magna facta est in populo; insuper et duo filii tui mortui sunt, Ophni et Phinees, et arca Dei capta est ”.
 \verse Cumque ille nominasset arcam Dei, cecidit de sella retrorsum iuxta ostium et, fractis cervicibus, mortuus est; senex enim erat vir et gravis. Et ipse iudicavit Israel quadraginta annis.
 \verse Nurus autem eius, uxor Phinees, praegnans erat vicinaque partui. Et, audito nuntio quod capta esset arca Dei et mortuus socer suus et vir suus, incurvavit se et peperit; irruerant enim in eam dolores subiti. 
\verse In ipso autem momento mortis eius dixerunt ei, quae stabant circa eam: “ Ne timeas, quia filium peperisti ”. Quae non respondit eis neque animadvertit. 
\verse Et vocavit puerum Ichabod dicens: “ Translata est gloria de Israel! ”, quia capta est arca Dei et pro socero suo et pro viro suo. 
\verse Et ait: “ Translata est gloria ab Israel, eo quod capta est arca Dei! ”.
 
\begin{biblechapter}
\verse Philisthim autem tulerunt arcam Dei et asportaverunt eam a Abenezer in Azotum. 
\verse Tulerunt Philisthim arcam Dei et intulerunt eam in templum Dagon et statuerunt eam iuxta Dagon. 
\verse Cumque surrexissent Azotii altera die, ecce Dagon iacebat pronus in terram ante arcam Domini; et tulerunt Dagon et restituerunt eum in loco suo. 
\verse Rursumque mane die altera consurgentes invenerunt Dagon iacentem super faciem suam in terram coram arca Domini; caput autem Dagon et duae palmae manuum eius abscisae erant super limen: 
\verse porro Dagon truncus solus remanserat in loco suo. Propter hanc causam non calcant sacerdotes Dagon et omnes, qui ingrediuntur templum eius, super limen Dagon in Azoto usque in hodiernum diem.
 \verse Aggravata est autem manus Domini super Azotios, et demolitus est eos et percussit eos tumoribus, Azotum et fines eius. 
\verse Videntes autem viri Azotii huiuscemodi plagam dixerunt: “ Non maneat arca Dei Israel apud nos, quoniam dura est manus eius super nos et super Dagon deum nostrum ”. 
\verse Et mittentes congregaverunt omnes principes Philisthinorum ad se et dixerunt: “ Quid faciemus de arca Dei Israel? ”. Responderuntque: “ In Geth circumducatur arca Dei Israel ”. Et circumduxerunt arcam Dei Israel. 
\verse Postquam autem circumduxerunt eam, facta est manus Domini super civitatem, pavor magnus nimis; et percussit viros urbis a parvo usque ad maiorem, et eruperunt eis tumores. 
\verse Miserunt ergo arcam Dei in Accaron.
 Cumque venisset arca Dei in Accaron, exclamaverunt Accaronitae dicentes: “ Adduxerunt ad nos arcam Dei Israel, ut interficiat nos et populum nostrum! ”. 
 \verse Miserunt itaque et congregaverunt omnes principes Philisthinorum et dixerunt: “ Dimittite arcam Dei Israel, et revertatur in locum suum et non interficiat nos cum populo nostro ”. 
\verse Fiebat enim pavor mortis in tota civitate, et gravissima valde manus Dei. Viri quoque, qui mortui non fuerant, percutiebantur tumoribus, et ascendebat ululatus civitatis in caelum.
 
\begin{biblechapter}
\verse Fuit ergo arca Domini in regione Philisthinorum septem mensibus; 
\verse et vocaverunt Philisthim sacerdotes et divinos dicentes: “ Quid faciemus de arca Domini? Indicate nobis quomodo remittemus eam in locum suum ”.
 Qui dixerunt: 
\verse “ Si remittitis arcam Dei Israel, nolite dimittere eam vacuam, sed, quod debetis, reddite ei pro peccato, et tunc curabimini; scietis quare non recedat manus eius a vobis ”. 
\verse Qui dixerunt: “ Quid est quod pro delicto reddere debeamus ei? ”. Responderuntque illi: 
\verse “ Iuxta numerum principum Philisthinorum quinque tumores aureos facietis et quinque mures aureos, quia plaga una fuit omnibus vobis et principibus vestris. Facietisque similitudines tumorum vestrorum et similitudines murium, qui demoliti sunt terram, et dabitis Deo Israel gloriam, si forte relevet manum suam a vobis et a diis vestris et a terra vestra. 
\verse Quare gravatis corda vestra, sicut aggravavit Aegyptus et pharao cor suum? Nonne, postquam percussit eos, tunc dimiserunt eos, et abierunt? 
\verse Nunc ergo arripite et facite plaustrum novum unum et duas vaccas fetas, quibus non est impositum iugum, iungite in plaustro; et recludite vitulos earum domi. 
\verse Tolletisque arcam Domini et ponetis in plaustro; et similitudines aureas, quas exsolvistis ei pro delicto, ponetis in capsella ad latus eius et dimittite eam, ut vadat, 
\verse et aspicietis. Et siquidem per viam finium suorum ascenderit contra Bethsames, ipse fecit nobis hoc malum grande; sin autem minime, sciemus quia nequaquam manus eius tetigit nos, sed casu accidit ”.
 \verse Fecerunt ergo illi hoc modo et tollentes duas vaccas, quae lactabant vitulos, iunxerunt ad plaustrum vitulosque earum concluserunt domi; 
\verse et posuerunt arcam Dei super plaustrum et capsellam, quae habebat mures aureos et similitudines tumorum.
 \verse Ibant autem in directum vaccae per viam, quae ducit Bethsames, et itinere uno gradiebantur pergentes et mugientes et non declinabant neque ad dextram neque ad sinistram. Sed et principes Philisthim sequebantur usque ad terminos Bethsames. 
\verse Porro Bethsamitae metebant triticum in valle; et elevantes oculos viderunt arcam et gavisi sunt, cum vidissent.
 \verse Et plaustrum venit in agrum Iosue Bethsamitae et stetit ibi. Erat autem ibi lapis magnus; et conciderunt ligna plaustri vaccasque imposuerunt super ea holocaustum Domino. 
\verse Levitae autem deposuerunt arcam Dei et capsellam, quae erat iuxta eam, in qua erant similitudines aureae; et posuerunt super lapidem grandem. Viri autem Bethsamitae obtulerunt holocausta et immolaverunt victimas in die illa Domino. 
\verse Et quinque principes Philisthinorum viderunt et reversi sunt in Accaron in die illa.
 \verse Hi sunt autem tumores aurei, quos reddiderunt Philisthim pro delicto Domino: Azotus unum, Gaza unum, Ascalon unum, Geth unum, Accaron unum; 
\verse et mures aureos secundum numerum urbium Philisthim quinque principum, ab urbe murata usque ad villam, quae erat absque muro; et lapis ille magnus, super quem posuerunt arcam Domini, testis est usque in hunc diem in agro Iosue Bethsamitis.
 \verse Filii autem Iechoniae non sunt gavisi super viros Bethsamites quia viderant arcam Domini; et percussit Dominus de populo septuaginta viros. Luxitque populus eo quod Dominus percussisset plebem plaga magna; 
\verse et dixerunt viri Bethsamitae: “ Quis poterit stare in conspectu Domini, Dei sancti huius? Et ad quem ascendet a nobis? ”. 
\verse Miseruntque nuntios ad habitatores Cariathiarim dicentes: “ Reduxerunt Philisthim arcam Domini. Descendite et ducite eam sursum ad vos ”.
 
\begin{biblechapter}
\verse Venerunt ergo viri Cariathiarim et duxerunt arcam Domini sursum et intulerunt eam in domum Abinadab in colle; Eleazarum autem filium eius sanctificaverunt, ut custodiret arcam Domini.
 \verse Et factum est, ex qua die mansit arca Domini in Cariathiarim, multiplicati sunt dies; erat quippe iam annus vicesimus, et ingemuit omnis domus Israel post Dominum. 
\verse Ait autem Samuel ad universam domum Israel dicens: “ Si in toto corde vestro revertimini ad Dominum, auferte deos alienos de medio vestri et Astharoth et praeparate corda vestra Domino et servite ei soli, et eruet vos de manu Philisthim ”. 
\verse Abstulerunt ergo filii Israel Baalim et Astharoth et servierunt Domino soli.
 \verse Dixit autem Samuel: “ Congregate universum Israel in Maspha, ut orem pro vobis Dominum ”. 
\verse Et convenerunt in Maspha hauseruntque aquam et effuderunt in conspectu Domini et ieiunaverunt in die illa et dixerunt ibi: “ Peccavimus Domino ”. Iudicavitque Samuel filios Israel in Maspha.
 \verse Et audierunt Philisthim quod congregati essent filii Israel in Maspha, et ascenderunt principes Philisthinorum ad Israel. Quod cum audissent filii Israel, timuerunt a facie Philisthinorum 
\verse dixeruntque ad Samuel: “ Ne cesses pro nobis clamare ad Dominum Deum nostrum, ut salvet nos de manu Philisthinorum ”. 
\verse Tulit ergo Samuel agnum lactantem unum et obtulit illum holocaustum integrum Domino; et clamavit Samuel ad Dominum pro Israel, et exaudivit eum Dominus. 
\verse Factum est autem cum Samuel offerret holocaustum, Philisthim iniere proelium contra Israel. Intonuit autem Dominus fragore magno in die illa super Philisthim et exterruit eos, et caesi sunt a facie Israel. 
\verse Egressique viri Israel de Maspha persecuti sunt Philisthaeos et percusserunt eos usque ad locum, qui erat subter Bethchar. 
\verse Tulit autem Samuel lapidem unum et posuit eum inter Maspha et inter Sen et vocavit nomen loci illius Abenezer (id est Lapis adiutorii) dixitque: “ Hucusque auxiliatus est nobis Dominus ”.
 \verse Et humiliati sunt Philisthim nec apposuerunt ultra ut venirent in terminos Israel. Facta est itaque manus Domini super Philisthaeos cunctis diebus Samuel. 
\verse Et redditae sunt urbes, quas tulerant Philisthim ab Israel, Israeli ab Accaron usque Geth; et terminos earum liberavit Israel de manu Philisthinorum. Eratque pax inter Israel et Amorraeum.
 \verse Iudicabat quoque Samuel Israel cunctis diebus vitae suae 
\verse et ibat per singulos annos circumiens Bethel et Galgala et Maspha et iudicabat Israelem in supradictis locis. Revertebaturque in Rama; ibi enim erat domus eius, et ibi iudicabat Israelem. Aedificavit etiam ibi altare Domino.
 
\begin{biblechapter}
\verse Factum est autem cum senuis set, Samuel posuit filios suos iudices Israel. 
\verse Fuitque nomen filii eius primogeniti Ioel et nomen secundi Abia; iudicabant in Bersabee. 
\verse Et non ambulaverunt filii illius in viis eius, sed declinaverunt post avaritiam acceperuntque munera et perverterunt iudicium.
 \verse Congregati ergo universi maiores natu Israel venerunt ad Samuel in Rama 
\verse dixeruntque ei: “ Ecce tu senuisti, et filii tui non ambulant in viis tuis; nunc ergo constitue nobis regem, ut iudicet nos, sicut universae habent nationes ”. 
 \verse Displicuitque sermo in oculis Samuelis, eo quod dixissent: “ Da nobis regem, ut iudicet nos ”. Et oravit Samuel ad Dominum. 
\verse Dixit autem Dominus ad Samuel: “ Audi vocem populi in omnibus, quae loquuntur tibi; non enim te abiecerunt, sed me abiecerunt, ne regnem super eos. 
\verse Iuxta omnia opera sua, quae fecerunt a die, qua eduxi eos de Aegypto, usque ad diem hanc, sicut dereliquerunt me et servierunt diis alienis, sic faciunt etiam tibi. 
\verse Nunc ergo audi vocem eorum; verumtamen contestare eos et praedic eis ius regis, qui regnaturus est super eos ”.
 \verse Dixit itaque Samuel omnia verba Domini ad populum, qui petierat a se regem, 
\verse et ait: “ Hoc erit ius regis, qui imperaturus est vobis: Filios vestros tollet et ponet in curribus suis facietque sibi equites, et current ante quadrigas eius; 
\verse et constituet sibi tribunos et centuriones et aratores agrorum suorum et messores segetum et fabros armorum et curruum suorum. 
\verse Filias quoque vestras faciet sibi unguentarias et focarias et panificas. 
\verse Agros quoque vestros et vineas et oliveta optima tollet et dabit servis suis. 
 \verse Sed et segetes vestras et vinearum reditus addecimabit, ut det eunuchis et famulis suis. 
\verse Servos etiam vestros et ancillas et boves vestros optimos et asinos auferet et ponet in opere suo. 
\verse Greges vestros addecimabit, vosque eritis ei servi. 
\verse Et clamabitis in die illa a facie regis vestri, quem elegistis vobis, et non exaudiet vos Dominus in die illa ”.
 \verse Noluit autem populus audire vocem Samuel, sed dixerunt: “ Nequaquam: rex enim erit super nos, 
\verse et erimus nos quoque sicut omnes gentes; et iudicabit nos rex noster et egredietur ante nos et pugnabit bella nostra pro nobis ”. 
\verse Et audivit Samuel omnia verba populi et locutus est ea in auribus Domini. 
\verse Dixit autem Dominus ad Samuel: “ Audi vocem eorum et constitue super eos regem ”. Et ait Samuel ad viros Israel: “ Vadat unusquisque in civitatem suam ”.
 
\begin{biblechapter}
\verse Et erat vir de Beniamin nomine Cis filius Abiel filii Seror filii Bechorath filii Aphia, Beniaminita vir potens. 
\verse Et erat ei filius vocabulo Saul electus et bonus, et non erat vir de filiis Israel melior illo; ab umero et sursum eminebat super omnem populum.
 \verse Perierant autem asinae Cis patris Saul, et dixit Cis ad Saul filium suum: “ Tolle tecum unum de pueris et consurgens vade et quaere asinas ”. Qui cum transissent per montem Ephraim 
\verse et per terram Salisa et non invenissent, transierunt etiam per terram Salim, et non erant, sed et per terram Iemini et minime reppererunt. 
\verse Cum autem venissent in terram Suph, dixit Saul ad puerum suum, qui erat cum eo: “ Veni, et revertamur, ne forte dimiserit pater meus asinas et sollicitus sit pro nobis ”. 
\verse Qui ait ei: “ Ecce est vir Dei in civitate hac, vir nobilis. Omne quod loquitur, absque ambiguitate venit. Nunc ergo eamus illuc, si forte indicet nobis de via nostra, propter quam venimus ”. 
\verse Dixitque Saul ad puerum suum: “ Ecce ibimus; quid feremus ad virum? Panis defecit in sitarciis nostris, et sportulam non habemus, ut demus homini Dei. Quid habemus? ”. 
\verse Rursum puer respondit Sauli et ait: “ Ecce inventa est in manu mea quarta pars sicli argenti; demus homini Dei, ut indicet nobis viam nostram ”. — 
\verse Olim in Israel sic loquebatur unusquisque vadens consulere Deum: “ Venite, et eamus ad videntem ”; qui enim propheta dicitur hodie, vocabatur olim videns. — 
\verse Et dixit Saul ad puerum suum: “ Optimus sermo tuus; veni, eamus ”. Et ierunt in civitatem, in qua erat vir Dei.
 \verse Cumque ascenderent clivum civitatis, invenerunt puellas egredientes ad hauriendam aquam et dixerunt eis: “ Num hic est videns? ”. 
\verse Quae respondentes dixerunt illis: “ Hic est: ecce ante te, festina nunc; hodie enim venit in civitatem, quia sacrificium est hodie populo in excelso. 
\verse Ingredientes urbem statim invenietis eum, antequam ascendat excelsum ad vescendum; neque enim comesurus est populus, donec ille veniat, quia ipse benedicit hostiae, et deinceps comedunt, qui vocati sunt. Nunc ergo conscendite, quia statim reperietis eum ”.
 \verse Et ascenderunt in civitatem. Cumque illi intrarent in urbem, apparuit Samuel egrediens obviam eis, ut ascenderet in excelsum.
 \verse Dominus autem revelaverat Samuel, ante unam diem quam veniret Saul, dicens: 
\verse “ Hac ipsa, quae nunc est hora, cras mittam ad te virum de terra Beniamin, et unges eum ducem super populum meum Israel, et salvabit populum meum de manu Philisthinorum, quia respexi populum meum; venit enim clamor eorum ad me ”. 
\verse Cumque aspexisset Samuel Saulem, Dominus ait ei: “ Ecce vir, quem dixeram tibi; iste dominabitur populo meo ”.
 \verse Accessit autem Saul ad Samuelem in medio portae et ait: “ Indica, oro, mihi: Ubi est domus videntis? ”. 
\verse Et respondit Samuel Sauli dicens: “ Ego sum videns. Ascende ante me in excelsum, ut comedatis mecum hodie. Et dimittam te mane et omnia, quae sunt in corde tuo, indicabo tibi; 
\verse et de asinis, quas perdidisti nudiustertius, ne sollicitus sis, quia inventae sunt. Et cuius erunt optima quaeque Israel? Nonne tibi et omni domui patris tui? ”. 
\verse Respondens autem Saul ait: “Numquid non Beniaminita ego sum de minima tribu Israel, et cognatio mea novissima inter omnes familias de tribu Beniamin? Quare ergo locutus es mihi sermonem istum? ”.
 \verse Assumens itaque Samuel Saulem et puerum eius introduxit eos in triclinium et dedit eis locum in capite eorum, qui fuerant invitati: erant enim quasi triginta viri. 
\verse Dixitque Samuel coco: “ Da partem, quam dedi tibi et praecepi, ut reponeres seorsum apud te ”. 
\verse Levavit autem cocus armum et caudam et posuit ante Saul. Dixitque Samuel: “ Ecce quod remansit; pone ante te et comede, quia de industria servatum est tibi, quando populum vocavi ”. Et comedit Saul cum Samuel in die illa.
 \verse Et descenderunt de excelso in oppidum. Et straverunt pro Saul in solario, et dormivit.
 \verse Cumque mane surrexissent, et iam elucesceret, vocavit Samuel Saul in solario dicens: “ Surge, ut dimittam te ”. Et surrexit Saul. Egressique sunt ambo, ipse videlicet et Samuel. 
\verse Cumque descenderent in extrema parte civitatis, Samuel dixit ad Saul: “ Dic puero, ut antecedat nos — et ille antecessit C; tu autem subsiste paulisper, ut indicem tibi verbum Domini ”.
 
\begin{biblechapter}
\verse Tulit autem Samuel lenticulam olei et effudit super caput eius et deosculatus eum ait: “ Ecce unxit te Dominus in principem super populum suum, super Israel. Et tu dominaberis populo Domini et tu liberabis eum de manu inimicorum eius, qui in circuitu eius sunt. Et hoc tibi signum quia unxit te Deus in principem super hereditatem suam: 
\verse cum abieris hodie a me, invenies duos viros iuxta sepulcrum Rachel in finibus Beniamin, dicentque tibi: “Inventae sunt asinae, ad quas ieras perquirendas; et intermissis pater tuus asinis sollicitus est pro vobis et dicit: Quid faciam de filio meo?”. 
\verse Cumque abieris inde et ultra transieris et veneris ad quercum Thabor, invenient te ibi tres viri ascendentes ad Deum in Bethel: unus portans tres haedos et alius tres tortas panis et alius portans utrem vini. 
\verse Cumque te salutaverint, dabunt tibi duos panes, et accipies de manu eorum. 
\verse Post haec venies in Gabaa Dei, ubi est statio Philisthinorum; et, cum ingressus fueris ibi urbem, obviam habebis gregem prophetarum descendentium de excelso et ante eos psalterium et tympanum et tibiam et citharam ipsosque prophetantes. 
\verse Et insiliet in te spiritus Domini, et prophetabis cum eis et mutaberis in virum alium. 
\verse Quando ergo evenerint signa haec omnia tibi, fac, quaecumque invenerit manus tua, quia Dominus tecum est. 
\verse Et descendes ante me in Galgala. Ego quippe descendam ad te, ut offeram oblationem et immolem victimas pacificas. Septem diebus exspectabis, donec veniam ad te et ostendam tibi, quae facias ”.
 \verse Itaque, cum avertisset umerum suum, ut abiret a Samuele, immutavit ei Deus cor aliud, et venerunt omnia signa haec in die illa. 
\verse Veneruntque inde in Gabaa, et ecce grex prophetarum obvius ei; et insiluit super eum spiritus Dei, et prophetavit in medio eorum. 
\verse Videntes autem omnes, qui noverant eum heri et nudiustertius, quod esset cum prophetis et prophetaret, dixerunt ad invicem: “ Quaenam res accidit filio Cis? Num et Saul inter prophetas? ”. 
\verse Responditque vir loci illius dicens: “ Et quis pater eorum? ”. Propterea versum est in proverbium: “ Num et Saul inter prophetas? ”.
 \verse Cessavit autem prophetare et venit in Gabaa; 
\verse dixitque patruus Saul ad eum et ad puerum eius: “ Quo abistis? ”. Qui respondit: “ Quaerere asinas; quas cum non repperissemus, venimus ad Samuelem”. 
\verse Et dixit ei patruus suus: “Indica mihi quid dixerit tibi Samuel”. 
\verse Et ait Saul ad patruum suum: “ Indicavit nobis quia inventae essent asinae ”. De sermone autem regni non indicavit ei, quem locutus illi fuerat Samuel.
 \verse Et convocavit Samuel populum ad Dominum in Maspha 
\verse et ait ad filios Israel: “ Haec dicit Dominus, Deus Israel: Ego eduxi Israel de Aegypto et erui vos de manu Aegyptiorum et de manu omnium regnorum, quae affligebant vos. 
\verse Vos autem hodie proiecistis Deum vestrum, qui solus salvavit vos de universis malis et tribulationibus vestris, et dixistis: “Nequaquam, sed regem constitue super nos!”. Nunc ergo state coram Domino per tribus vestras et per familias ”.
 \verse Et applicuit Samuel omnes tribus Israel; et cecidit sors in tribum Beniamin. 
\verse Et applicuit tribum Beniamin et cognationes eius; et cecidit in cognationem Metri et pervenit usque ad Saul filium Cis. Quaesierunt ergo eum, et non est inventus. 
\verse Et consuluerunt post haec Dominum, utrumnam venisset illuc vir. Responditque Dominus: “ Ecce absconditus est inter sarcinas ”. 
\verse Cucurrerunt itaque et tulerunt eum inde; stetitque in medio populi et altior fuit universo populo ab umero et sursum. 
\verse Et ait Samuel ad omnem populum:
 “ Certe videtis, quem elegit Dominus, quoniam non sit similis ei in omni populo ”. Et clamavit cunctus populus et ait: “ Vivat rex! ”.
 \verse Locutus est autem Samuel ad populum legem regni et scripsit in libro et reposuit coram Domino; et dimisit Samuel omnem populum, singulos in domum suam.
 \verse Sed et Saul abiit in domum suam in Gabaa; et abierunt cum eo viri fortes, quorum tetigerat Deus corda. 
\verse Filii vero Belial dixerunt: “ Num salvare nos poterit iste? ”. Et despexerunt eum et non attulerunt ei munera; ille vero dissimulabat se audire.
 
\begin{biblechapter}
\verse Ascendit autem Naas Ammonites et pugnare coepit ad versum Iabes Galaad. Dixeruntque omnes viri Iabes ad Naas: “ Habeto nos foederatos, et serviemus tibi ”. 
\verse Et respondit ad eos Naas Ammonites: “ In hoc feriam vobiscum foedus, ut eruam omnium vestrum oculos dextros ponamque vos opprobrium in universo Israel ”. 
\verse Et dixerunt ad eum seniores Iabes: “ Concede nobis septem dies, ut mittamus nuntios in universos terminos Israel; et, si non fuerit qui defendat nos, egrediemur ad te ”. 
\verse Venerunt ergo nuntii in Gabaa Saulis et locuti sunt verba audiente populo; et levavit omnis populus vocem suam et flevit.
 \verse Et ecce Saul veniebat sequens boves de agro et ait: “ Quid habet populus quod plorat? ”. Et narraverunt ei verba virorum Iabes. 
\verse Et insilivit spiritus Domini in Saul, cum audisset verba haec; et iratus est furor eius nimis. 
\verse Et assumens par boum concidit in frusta misitque in omnes terminos Israel per manum nuntiorum dicens: “ Quicumque non exierit secutusque fuerit Saul et Samuel, sic fiet bobus eius ”. Invasit ergo timor Domini populum, et egressi sunt quasi vir unus. 
\verse Et recensuit eos in Bezec: fueruntque filiorum Israel trecenta milia; virorum autem Iudae triginta milia. 
\verse Et dixit nuntiis, qui venerant: “ Sic dicetis viris, qui sunt in Iabes Galaad: Cras erit vobis salus, cum incaluerit sol ”. Venerunt ergo nuntii et annuntiaverunt viris Iabes, qui laetati sunt 
\verse et dixerunt: “Mane exibimus ad vos, et facietis nobis omne, quod placuerit vobis ”.
 \verse Et factum est, cum venisset dies crastinus, constituit Saul populum in tres partes; et ingressi sunt media castra in vigilia matutina et percusserunt Ammon, usque dum incalesceret dies. Reliqui autem dispersi sunt, ita ut non relinquerentur in eis duo pariter.
 \verse Et ait populus ad Samuel: “ Quis est iste qui dixit: “Saul num regnabit super nos?”. Date viros, et interficiemus eos ”. 
\verse Et ait Saul: “ Non occidetur quisquam in die hac, quia hodie fecit Dominus salutem in Israel”. 
\verse Dixit autem Samuel ad populum: “ Venite, et eamus in Galgala et innovemus ibi regnum ”.
 \verse Et perrexit omnis populus in Galgala, et fecerunt ibi regem Saul coram Domino in Galgala; et immolaverunt ibi victimas pacificas coram Domino. Et laetatus est ibi Saul et cuncti viri Israel nimis.
 
\begin{biblechapter}
\verse Dixit autem Samuel ad universum Israel: “ Ecce audivi vocem vestram iuxta omnia, quae locuti estis ad me, et constitui super vos regem; 
\verse et nunc rex graditur ante vos. Ego autem senui et incanui; porro filii mei vobiscum sunt. Itaque conversatus coram vobis ab adulescentia mea usque ad hanc diem; 
\verse ecce praesto sum. Loquimini contra me coram Domino et coram christo eius, utrum bovem cuiusquam tulerim an asinum, si quempiam calumniatus sum, si oppressi aliquem, si de manu cuiusquam munus accepi, ut oculos meos clauderem in eius causa. Restituam vobis ”. 
\verse Et dixerunt: “ Non es calumniatus nos neque oppressisti neque tulisti de manu alicuius quippiam ”. 
\verse Dixitque ad eos: “ Testis Dominus adversum vos, et testis christus eius in die hac, quia non inveneritis in manu mea quippiam ”. Et dixerunt: “ Testis ”.
 \verse Et ait Samuel ad populum: “ Testis est Dominus, qui fecit Moysen et Aaron et eduxit patres nostros de terra Aegypti. 
\verse Nunc ergo state, ut iudicio contendam adversum vos coram Domino de omnibus misericordiis Domini, quas fecit vobiscum et cum patribus vestris: 
\verse quomodo ingressus est Iacob in Aegyptum, et oppresserunt eos Aegyptii; et clamaverunt patres vestri ad Dominum, et misit Dominus Moysen et Aaron et eduxit patres vestros ex Aegypto et collocavit eos in loco hoc; 
\verse qui obliti sunt Domini Dei sui, et tradidit eos in manu Sisarae magistri militiae Asor et in manu Philisthinorum et in manu regis Moab, et pugnaverunt adversum eos.
 \verse Postea autem clamaverunt ad Dominum et dixerunt: “Peccavimus, quia dereliquimus Dominum et servivimus Baalim et Astharoth; nunc ergo erue nos de manu inimicorum nostrorum, et serviemus tibi”. 
\verse Et misit Dominus Ierobbaal et Barac et Iephte et Samuel et eruit vos de manu inimicorum vestrorum per circuitum; et habitastis confidenter.
 \verse Videntes autem quod Naas rex filiorum Ammon venisset adversum vos, dixistis mihi: “Nequaquam, sed rex imperabit nobis!”, cum Dominus Deus vester regnaret in vobis. 
\verse Nunc ergo praesto est rex vester, quem elegistis et petistis; ecce dedit vobis Dominus regem. 
\verse Si timueritis Dominum et servieritis ei et audieritis vocem eius et non contempseritis sermonem Domini, eritis et vos et rex, qui imperat vobis, sequentes Dominum Deum vestrum. 
\verse Si autem non audieritis vocem Domini, sed contempseritis sermonem Domini, erit manus Domini super vos et super regem vestrum, ut disperdat vos.
 \verse Sed et nunc state et videte rem istam grandem, quam facturus est Dominus in conspectu vestro. 
\verse Numquid non messis tritici est hodie? Invocabo Dominum, et dabit tonitrua et pluvias; et scietis et videbitis quia grande malum feceritis vobis in conspectu Domini petentes super vos regem ”.
 \verse Et clamavit Samuel ad Dominum, et dedit Dominus tonitrua et pluviam in die illa. 
\verse Et timuit omnis populus nimis Dominum et Samuel; dixitque universus populus ad Samuel: “ Ora pro servis tuis ad Dominum Deum tuum, ut non moriamur: addidimus enim universis peccatis nostris malum, ut peteremus nobis regem ”.
 \verse Dixit autem Samuel ad populum: “ Nolite timere. Vos fecistis universum malum hoc; verumtamen nolite recedere a tergo Domini et servite Domino in omni corde vestro; 
\verse et nolite declinare post vana, quae non proderunt vobis neque eruent vos, quia vana sunt; 
\verse profecto non derelinquet Dominus populum suum propter nomen suum magnum, quia dignatus est Dominus facere vos sibi populum. 
 \verse Absit autem a me hoc peccatum in Dominum, ut cessem orare pro vobis et docere vos viam bonam et rectam. 
\verse Igitur timete Dominum et servite ei in veritate et ex toto corde vestro; vidistis enim magnifica, quae in vobis gesserit. 
\verse Quod si perseveraveritis in malitia, et vos et rex vester pariter peribitis ”.
 
\begin{biblechapter}
\verse Filius annorum Saul, cum regnare coepisset; duobus autem annis regnavit super Israel.
 \verse Et elegit sibi Saul tria milia de Israel: et erant cum Saul duo milia in Machmas et in monte Bethel, mille autem cum Ionathan in Gabaa Beniamin. Porro ceterum populum remisit unumquemque in tabernacula sua. 
\verse Et percussit Ionathan stationem Philisthinorum, quae erat in Gabaa. Quod audierunt Philisthim; Saul autem cecinit bucina in omni terra dicens: “ Audiant Hebraei! ”. 
\verse Et universus Israel audivit huiuscemodi famam: “ Percussit Saul stationem Philisthinorum; et factus est Israel odiosus Philisthim ”. Ergo populus congregatus est post Saul in Galgala.
 \verse Et Philisthim congregati sunt ad proeliandum contra Israel: tria milia curruum et sex milia equitum et reliquum vulgus plurimum sicut arena, quae est in litore maris. Et ascendentes castrametati sunt in Machmas ad orientem Bethaven. 
\verse Quod cum vidissent viri Israel se in arto sitos — afflictus est enim populus — absconderunt se in speluncis et in abditis, in petris quoque et in antris et in cisternis. 
\verse Hebraei autem transierunt Iordanem in terram Gad et Galaad.
 Cumque adhuc esset Saul in Galgalis, universus populus perterritus est, qui sequebatur eum. 
\verse Et exspectavit septem diebus iuxta placitum Samuel, et non venit Samuel in Galgala; dilapsusque est populus ab eo. 
\verse Ait ergo Saul: “ Afferte mihi holocaustum et pacifica ”. Et obtulit holocaustum.
 \verse Cumque complesset offerens holocaustum, ecce Samuel veniebat; et egressus est Saul obviam ei, ut salutaret eum. 
\verse Locutusque est ad eum Samuel: “ Quid fecisti? ”. Respondit Saul: “ Quia vidi quod dilaberetur populus a me, et tu non veneras iuxta placitos dies, porro Philisthim congregati fuerant in Machmas, 
\verse dixi: Nunc descendent Philisthim ad me in Galgala, et faciem Domini non placavi. Necessitate compulsus obtuli holocaustum ”.
 \verse Dixitque Samuel ad Saul: “ Stulte egisti. Utinam custodisses mandata Domini Dei tui, quae praecepit tibi! Profecto nunc confirmasset Dominus regnum tuum super Israel in sempiternum; 
\verse sed nequaquam regnum tuum ultra consurget. Quaesivit sibi Dominus virum iuxta cor suum; et constituit eum Dominus ducem super populum suum, eo quod non servaveris, quae praecepit Dominus ”.
 \verse Surrexit autem Samuel et ascendit de Galgalis et abiit per viam suam. Et reliquus populus ascendit post Saul obviam exercitui bellatorum. Et venerunt de Galgalis in Gabaa Beniamin. Et recensuit Saul populum, qui inventi fuerant cum eo, quasi sescentos viros.
 \verse Et Saul et Ionathan filius eius populusque, qui erat cum eis, erat in Gabaa Beniamin; porro Philisthim consederant in Machmas. 
\verse Et egressi sunt ad praedandum de castris Philisthinorum tres cunei: unus cuneus pergebat contra viam Ophra ad terram Sual, 
\verse porro alius ingrediebatur per viam Bethoron, tertius autem verterat se ad iter termini imminentis valli Seboim contra desertum.
 \verse Porro faber ferrarius non inveniebatur in omni terra Israel; caverant enim Philisthim, ne forte facerent Hebraei gladium aut lanceam. 
\verse Descendebat ergo omnis Israel ad Philisthim, ut exacueret unusquisque vomerem suum et ligonem et securim et falcem. 
\verse Pretium autem exacutionis erat: pro vomeribus et ligonibus duae partes sicli, et tertia pars sicli ad acuendas secures et ad stimulum corrigendum. 
\verse Cumque venisset dies proelii Machmas, non est inventus ensis et lancea in manu totius populi, qui erat cum Saul et cum Ionathan, excepto Saul et Ionathan filio eius.
 \verse Egressa est autem statio Philisthim ad fauces Machmas.
 
\begin{biblechapter}
\verse Et accidit quadam die, ut diceret Ionathan filius Saul ad adulescentem armigerum suum: “ Veni, et transeamus ad stationem Philisthim, quae est ibi ex adverso ”. Patri autem suo hoc ipsum non indicavit. 
\verse Porro Saul morabatur in extrema parte Gabaa sub malogranato, quae erat in Magron; et erat populus cum eo quasi sescentorum virorum. 
\verse Et Ahias filius Achitob fratris Ichabod filii Phinees, qui ortus fuerat ex Heli sacerdote Domini in Silo, portabat ephod. Sed et populus ignorabat quod isset Ionathan.
 \verse Erant autem inter ascensus, per quos nitebatur Ionathan transire ad stationem Philisthinorum, dens rupis hinc ex una parte et dens rupis illinc ex altera parte: nomen uni Boses et nomen alteri Sene; 
\verse unus scopulus prominens ad aquilonem ex adverso Machmas et alter a meridie contra Gabaa. 
\verse Dixit autem Ionathan ad adulescentem armigerum suum: “ Veni, transeamus ad stationem incircumcisorum horum, si forte faciat Dominus pro nobis; quia non est Domino difficile salvare vel in multitudine vel in paucis ”. 
\verse Dixitque ei armiger suus: “ Fac omnia, quae placent animo tuo. Perge quo cupis; ego ero tecum ubicumque volueris ”. 
\verse Et ait Ionathan: “ Ecce nos transimus ad viros istos. Cumque apparuerimus eis, 
\verse si taliter locuti fuerint ad nos: “Manete, donec veniamus ad vos”, stemus in loco nostro nec ascendamus ad eos. 
\verse Si autem dixerint: “Ascendite ad nos”, ascendamus, quia tradidit eos Dominus in manibus nostris; hoc erit nobis signum ”.
 \verse Apparuit igitur uterque stationi Philisthinorum. Dixeruntque Philisthim: “ En Hebraei egrediuntur de cavernis, in quibus absconditi fuerant ”. 
\verse Et locuti sunt viri de statione ad Ionathan et ad armigerum eius dixeruntque: “ Ascendite ad nos, et ostendimus vobis rem ”. Et ait Ionathan ad armigerum suum: “ Ascendamus; sequere me, tradidit enim eos Dominus in manu Israel ”. 
\verse Ascendit autem Ionathan reptans manibus et pedibus et armiger eius post eum; Philisthim cadebant ante Ionathan, et eos armiger eius interficiebat sequens eum. 
\verse Et facta est plaga prima, qua percussit Ionathan et armiger eius quasi viginti viros in media fere parte iugeri. 
\verse Et factus est terror in castris per agros; sed et omnis populus stationis eorum et, qui ierant ad praedandum, obstupuerunt; et conturbata est terra, et factus est terror a Deo.
 \verse Et respexerunt speculatores Saul, qui erant in Gabaa Beniamin; et ecce multitudo fluctuabat huc illucque diffugiens. 
\verse Et ait Saul populo, qui erat cum eo: “ Requirite et videte quis abierit ex nobis ”. Cumque requisissent, repertum est non adesse Ionathan et armigerum eius. 
\verse Et ait Saul ad Ahiam: “ Applica ephod ”. Ipse enim portabat ephod in die illa in conspectu filiorum Israel. 
\verse Cumque loqueretur Saul ad sacerdotem, tumultus maior fiebat in castris Philisthinorum, crescebatque paulatim et clarius reboabat. Et ait Saul ad sacerdotem: “ Contrahe manum tuam ”.
 \verse Congregati ergo sunt Saul et omnis populus, qui erat cum eo, et venerunt usque ad locum certaminis. Et ecce versus fuerat gladius uniuscuiusque ad proximum suum: perturbatio magna nimis. 
\verse Sed et Hebraei, qui fuerant cum Philisthim heri et nudiustertius ascenderantque cum eis in castris, reversi sunt et ipsi, ut essent cum Israel, qui erant cum Saul et Ionathan. 
\verse Omnes quoque Israelitae, qui se absconderant in monte Ephraim, audientes quod fugissent Philisthim, sociaverunt se et ipsi cum suis in proelio. 
\verse Et salvavit Dominus in die illa Israel; pugna autem pervenit ultra Bethaven.
 \verse Et viri Israel comprimebant se in die illa. Adiuravit autem Saul populum dicens: “ Maledictus vir, qui comederit panem usque ad vesperam, donec ulciscar de inimicis meis! ”. Et non manducavit universus populus panem. 
\verse Omneque terrae vulgus venit in saltum, in quo erat mel super faciem agri. 
\verse Ingressus est itaque populus saltum, et apparuit fluens mel. Nullusque applicuit manum ad os suum; timebat enim populus iuramentum.
 \verse Porro Ionathan non audierat, cum adiuraret pater eius populum; extenditque summitatem virgae, quam habebat in manu, et intinxit in favo mellis et convertit manum suam ad os suum, et illuminati sunt oculi eius. 
\verse Respondensque unus de populo ait: “ Iureiurando constrinxit pater tuus populum dicens: “Maledictus, qui comederit panem hodie!”. Defecit autem populus ”. 
\verse Dixitque Ionathan: “ Turbavit pater meus terram! Videte quia illuminati sunt oculi mei, eo quod gustaverim paululum de melle isto; 
\verse quanto magis si comedisset hodie populus de praeda inimicorum suorum, quam repperit? Nonne nunc maior facta fuisset plaga in Philisthim? ”.
 \verse Percusserunt ergo in die illa Philisthaeos a Machmis usque in Aialon; defatigatus est autem populus nimis. 
\verse Et versus ad praedam tulit oves et boves et vitulos; et mactaverunt in terra, comeditque populus cum sanguine. 
 \verse Nuntiaverunt autem Saul dicentes: “ Ecce populus peccat Domino comedens cum sanguine ”. Qui ait: “ Praevaricati estis! Volvite ad me huc saxum grande ”. 
 \verse Et dixit Saul: “ Dispergimini in vulgus et dicite eis, ut adducat ad me unusquisque bovem suum et arietem, et occidite super istud et vescimini; et non peccabitis Domino comedentes cum sanguine ”. Adduxit itaque omnis populus, unusquisque quod erat in manu sua illa nocte, et occiderunt ibi. 
\verse Aedificavit autem Saul altare Domino. Tuncque primum coepit aedificare altare Domino.
 \verse Et dixit Saul: “ Irruamus super Philisthim nocte et vastemus eos, usquedum illucescat mane; nec relinquamus de eis virum ”. Dixitque populus: “ Omne, quod bonum videtur in oculis tuis, fac ”. Et ait sacerdos: “ Accedamus huc ad Deum ”. 
 \verse Et consuluit Saul Deum: “ Num persequar Philisthim? Numquid trades eos in manu Israel? ”. Et non respondit ei in die illa. 
\verse Dixitque Saul: “ Accedite huc, universi duces populi, et scitote et videte per quem acciderit peccatum hoc hodie. 
\verse Vivit Dominus, salvator Israel, quia si per Ionathan filium meum factum est, absque retractatione morietur ”. Ad quod nullus contradixit ei de omni populo.
 \verse Et ait ad universum Israel: “ Separamini vos in partem unam, et ego cum Ionathan filio meo ero in parte altera ”. Respondit populus ad Saul: “ Quod bonum videtur in oculis tuis, fac ”. 
\verse Et dixit Saul ad Dominum, Deum Israel: “ Quid est quod non responderis servo tuo hodie? Si est in me aut in Ionathan filio meo iniquitas ista, Domine, Deus Israel, da Urim; sed, si est haec iniquitas in populo tuo Israel, da Tummim ”. Et deprehensus est Ionathan et Saul; populus autem salvus evasit. 
\verse Et ait Saul: “ Mittite sortem inter me et inter Ionathan filium meum ”. Et captus est Ionathan.
 \verse Dixit autem Saul ad Ionathan: “ Indica mihi quid feceris ”. Et indicavit ei Ionathan et ait: “ Gustans gustavi in summitate virgae, quae erat in manu mea, paululum mellis et ecce ego morior ”. 
\verse Et ait Saul: “ Haec faciat mihi Deus et haec addat, nisi morte morieris, Ionathan ”. 
\verse Dixitque populus ad Saul: “ Ergone Ionathan morietur, qui fecit salutem hanc magnam in Israel? Hoc nefas est; vivit Dominus, quia non cadet capillus de capite eius in terram, quia cum Deo operatus est hodie ”. Liberavit ergo populus Ionathan, ut non moreretur.
 \verse Recessitque Saul nec persecutus est Philisthim; porro Philisthim abierunt in loca sua.
 \verse At Saul, confirmato regno super Israel, pugnabat per circuitum adversum omnes inimicos eius: contra Moab et filios Ammon et Edom et reges Soba et Philisthaeos; et, quocumque se verterat, superabat. 
\verse Fortiter egit et percussit Amalec et eruit Israel de manu vastatorum eius.
 \verse Fuerunt autem filii Saul Ionathan et Isui et Melchisua. Nomina duarum filiarum eius: nomen primogenitae Merob et nomen minoris Michol. 
\verse Et nomen uxoris Saul Achinoam filia Achimaas, et nomen principis militiae eius Abner filius Ner patrui Saul. 
\verse Porro Cis pater Saul et Ner pater Abner fuerunt filii Abiel.
 \verse Erat autem bellum potens adversum Philisthaeos omnibus diebus Saul; nam, quemcumque viderat Saul virum fortem et aptum ad proelium, sociabat eum sibi.
 
\begin{biblechapter}
\verse Et dixit Samuel ad Saul: “ Me misit Dominus, ut ungerem te in regem super populum eius Israel. Nunc ergo audi vocem Domini. 
\verse Haec dicit Dominus exercituum: “Recensui, quaecumque fecit Amalec Israeli, quomodo restitit ei in via, cum ascenderet de Aegypto. 
\verse Nunc igitur vade et demolire Amalec et percute anathemate universa eius; non parcas ei, sed interfice a viro usque ad mulierem et parvulum atque lactantem, bovem et ovem, camelum et asinum” ”.
 \verse Convocavit itaque Saul populum et recensuit eos in Telem: ducenta milia peditum et decem milia virorum Iudae. 
\verse Cumque venisset Saul usque ad civitatem Amalec, tetendit insidias in torrente 
\verse dixitque Saul Cinaeo: “ Abite, recedite atque descendite ab Amalec, ne forte perdam te cum eo; tu enim fecisti misericordiam cum omnibus filiis Israel, cum ascenderent de Aegypto ”. Et recessit Cinaeus de medio Amalec.
 \verse Percussitque Saul Amalec ab Hevila usque ad Sur, quae est e regione Aegypti. 
 \verse Et apprehendit Agag regem Amalec vivum; omne autem vulgus interfecit in ore gladii. 
\verse Et pepercit Saul et populus Agag et optimis gregibus ovium et armentorum, pinguibus scilicet pecoribus et agnis et universis, quae pulchra erant, nec voluerunt disperdere ea; quidquid vero vile fuit et reprobum, hoc demoliti sunt.
 \verse Factum est autem verbum Domini ad Samuel dicens: 
\verse “ Paenitet me quod constituerim Saul regem, quia dereliquit me et verba mea opere non implevit ”. Contristatusque est Samuel et clamavit ad Dominum tota nocte.
 \verse Cumque de nocte surrexisset Samuel, ut iret ad Saul mane, nuntiatum est Samueli quod venisset Saul in Carmel et erexisset sibi trophaeum et reversus transisset descendissetque in Galgala. 
\verse Et cum venisset Samuel ad Saul, dixit ei Saul: “ Benedictus tu Domino; implevi verbum Domini ”. 
\verse Dixitque Samuel: “ Et quae est haec vox gregum, quae resonat in auribus meis, et armentorum, quam ego audio? ”. 
\verse Et ait Saul: “ De Amalec adduxerunt ea; pepercit enim populus melioribus ovibus et armentis, ut immolarentur Domino Deo tuo; reliqua vero occidimus ”.
 \verse Dixit autem Samuel ad Saul: “ Sine me, et indicabo tibi, quae locutus sit Dominus ad me nocte ”. Dixitque ei: “ Loquere ”. 
\verse Et ait Samuel: “ Nonne, cum parvulus esses in oculis tuis, caput in tribubus Israel factus es? Unxitque te Dominus regem super Israel 
\verse et misit te Dominus in viam et ait: “ Vade et interfice peccatores Amalec et pugnabis contra eos usque ad internecionem eorum ”. 
\verse Quare ergo non audisti vocem Domini, sed versus ad praedam es et fecisti malum in oculis Domini? ”. 
\verse Et ait Saul ad Samuelem: “ Immo audivi vocem Domini et ambulavi in via, per quam misit me Dominus; et adduxi Agag regem Amalec et Amalec interfeci. 
\verse Tulit autem populus de praeda oves et boves, primitias eorum, quae caesa sunt, ut immolet Domino Deo tuo in Galgalis ”.
 \verse Et ait Samuel: “ Numquid vult Dominus holocausta aut victimas et non potius ut oboediatur voci Domini? Melior est enim oboedientia quam victimae, et auscultare magis quam offerre adipem arietum. 
\verse Vere peccatum hariolandi est repugnare, et scelus idololatriae nolle acquiescere: pro eo ergo quod abiecisti sermonem Domini, abiecit te, ne sis rex ”.
 \verse Dixitque Saul ad Samuelem: “ Peccavi, quia praevaricatus sum sermonem Domini et verba tua timens populum et oboediens voci eorum; 
\verse sed nunc tolle, quaeso, peccatum meum et revertere mecum, ut adorem Dominum ”. 
\verse Et ait Samuel ad Saul: “ Non revertar tecum, quia proiecisti sermonem Domini; et proiecit te Dominus, ne sis rex super Israel ”. 
\verse Et conversus est Samuel, ut abiret; ille autem apprehendit summitatem pallii eius, quae et scissa est.
 \verse Et ait ad eum Samuel: “ Scidit Dominus regnum Israel a te hodie et tradidit illud proximo tuo meliori te. 
\verse Porro Gloria Israel non mentitur et paenitudine non flectitur; neque enim homo est, ut agat paenitentiam ”.
 \verse At ille ait: “ Peccavi, sed nunc honora me coram senibus populi mei et coram Israel; et revertere mecum, ut adorem Dominum Deum tuum ”. 
\verse Reversus ergo Samuel secutus est Saulem et adoravit Saul Dominum.
 \verse Dixitque Samuel: “ Adducite ad me Agag regem Amalec ”. Et oblatus est ei Agag tremens. Et dixit Agag: “ Certe secessit amaritudo mortis! ”. 
\verse Et ait Samuel: “ Sicut fecit absque liberis mulieres gladius tuus, sic absque liberis erit inter mulieres mater tua ”. Et in frusta concidit Samuel Agag coram Domino in Galgalis.
 \verse Abiit autem Samuel in Rama; Saul vero ascendit in domum suam in Gabaa Saulis. 
\verse Et non vidit Samuel ultra Saul usque ad diem mortis suae; verumtamen lugebat Samuel Saul, quoniam Dominum paenitebat quod constituisset Saul regem super Israel.
 
\begin{biblechapter}
\verse Dixitque Dominus ad Samuelem: “ Usquequo tu luges Saul, cum ego proiecerim eum, ne regnet super Israel? Imple cornu tuum oleo et veni, ut mittam te ad Isai Bethlehemitem; providi enim in filiis eius mihi regem ”. 
\verse Et ait Samuel: “ Quomodo vadam? Audiet enim Saul et interficiet me ”. Et ait Dominus: “ Vitulam de armento tolles in manu tua et dices: “Ad immolandum Domino veni”. 
 \verse Et vocabis Isai ad victimam; et ego ostendam tibi quid facias, et unges quemcumque monstravero tibi ”.
 \verse Fecit ergo Samuel, sicut locutus est ei Dominus, venitque in Bethlehem. Et expaverunt seniores civitatis occurrentes ei dixeruntque: “ Pacificusne ingressus tuus? ”. 
\verse Et ait: “ Pacificus; ad immolandum Domino veni. Sanctificamini et venite mecum, ut immolem ”. Sanctificavit ergo Isai et filios eius et vocavit eos ad sacrificium.
 \verse Cumque ingressi essent, vidit Eliab et ait: “ Absque dubio coram Domino est christus eius! ”. 
\verse Et dixit Dominus ad Samuelem: “ Ne respicias vultum eius neque altitudinem staturae eius, quoniam abieci eum; nec iuxta intuitum hominis iudico: homo enim videt ea, quae parent, Dominus autem intuetur cor ”. 
\verse Et vocavit Isai Abinadab et adduxit eum coram Samuele, qui dixit: “ Nec hunc elegit Dominus ”. 
\verse Adduxit autem Isai Samma, de quo ait: “ Etiam hunc non elegit Dominus ”. 
\verse Adduxit itaque Isai septem filios suos coram Samuele, et ait Samuel ad Isai: “ Non elegit Dominus ex istis ”.
 \verse Dixitque Samuel ad Isai: “ Numquid iam completi sunt filii? ”. Qui respondit: “ Adhuc reliquus est minimus et pascit oves ”. Et ait Samuel ad Isai: “ Mitte et adduc eum; nec enim discumbemus prius quam huc ille venerit ”. 
\verse Misit ergo et adduxit eum; erat autem rufus et pulcher aspectu decoraque facie. Et ait Dominus: “ Surge, unge eum; ipse est enim ”. 
\verse Tulit igitur Samuel cornu olei et unxit eum in medio fratrum eius; et directus est spiritus Domini in David a die illa et in reliquum. Surgensque Samuel abiit in Rama.
 \verse Spiritus autem Domini recessit a Saul, et exagitabat eum spiritus nequam a Domino. 
\verse Dixeruntque servi Saul ad eum: “ Ecce spiritus Dei malus exagitat te. 
\verse Iubeat dominus noster, et servi tui, qui coram te sunt, quaerant hominem scientem psallere cithara, ut, quando arripuerit te spiritus Dei malus, psallat manu sua, et levius feras ”. 
\verse Et ait Saul ad servos suos: “ Providete mihi aliquem bene psallentem et adducite eum ad me ”. 
\verse Et respondens unus de pueris ait: “ Ecce vidi filium Isai Bethlehemitae scientem psallere et fortissimum robore et virum bellicosum et prudentem in verbis et virum pulchrum; et Dominus est cum eo ”. 
\verse Misit ergo Saul nuntios ad Isai dicens: “ Mitte ad me David filium tuum, qui est in pascuis ”. 
\verse Tulitque Isai asinum cum pane et utre vini et haedo de capris uno et misit per manum David filii sui Sauli. 
\verse Et venit David ad Saul et stetit coram eo; at ille dilexit eum nimis, et factus est eius armiger. 
\verse Misitque Saul ad Isai dicens: “ Stet David in conspectu meo; invenit enim gratiam in oculis meis ”. 
 \verse Igitur, quandocumque spiritus Dei arripiebat Saul, David tollebat citharam et percutiebat manu sua; et refocillabatur Saul et levius habebat: recedebat enim ab eo spiritus malus.
 
\begin{biblechapter}
\verse Congregantes vero Philisthim agmina sua in proelium, convenerunt in Socho Iudae et castrametati sunt inter Socho et Azeca in Aphesdommim. 
\verse Porro Saul et viri Israel congregati venerunt in vallem Terebinthi et instruxerunt aciem ad pugnandum contra Philisthim. 
\verse Et Philisthim stabant super montem ex hac parte, et Israel stabat super montem ex altera parte; vallisque erat inter eos.
 \verse Et egressus est vir propugnator de castris Philisthinorum nomine Goliath de Geth altitudinis sex cubitorum et palmi. 
\verse Et cassis aerea super caput eius, et lorica squamata induebatur; porro pondus loricae eius quinque milia siclorum aeris. 
\verse Et ocreas aereas habebat in cruribus, et acinaces aereus erat inter umeros eius. 
\verse Hastile autem hastae eius erat quasi liciatorium texentium, ipsum autem ferrum hastae eius sescentos siclos habebat ferri; et armiger eius antecedebat eum. 
\verse Stansque clamabat adversum agmina Israel et dicebat eis: “ Quare venitis parati ad proelium? Numquid ego non sum Philisthaeus, et vos servi Saul? Eligite ex vobis virum, et descendat ad singulare certamen! 
\verse Si quiverit pugnare mecum et percusserit me, erimus vobis servi; si autem ego praevaluero et percussero eum, vos servi eritis et servietis nobis ”. 
\verse Et aiebat Philisthaeus: “ Ego exprobravi agminibus Israel hodie: Date mihi virum, et ineat mecum singulare certamen! ”.
 \verse Audiens autem Saul et omnes Israelitae sermones Philisthaei huiuscemodi stupebant et metuebant nimis.
 \verse David autem erat filius viri Ephrathaei, de quo supra dictum est, de Bethlehem Iudae, cui erat nomen Isai; qui habebat octo filios et erat vir in diebus Saul senex et grandaevus inter viros. 
\verse Abierunt autem tres filii eius maiores post Saul in proelium; et nomina trium filiorum eius, qui perrexerant ad bellum: Eliab primogenitus et secundus Abinadab tertiusque Samma. 
 \verse David autem erat minimus; tribus ergo maioribus secutis Saulem, 
\verse ibat David et revertebatur a Saul, ut pasceret gregem patris sui in Bethlehem.
 \verse Procedebat vero Philisthaeus mane et vespere et stabat quadraginta diebus.
 \verse Dixit autem Isai ad David filium suum: “ Accipe fratribus tuis ephi frumenti tosti et decem panes istos et curre in castra ad fratres tuos. 
\verse Et decem formellas casei has deferes ad tribunum, et fratres tuos visitabis, si recte agant; et pignus ab eis referes ”. 
\verse Saul autem et illi et omnes filii Israel in valle Terebinthi pugnabant adversum Philisthim.
 \verse Surrexit itaque David mane et commendavit gregem custodi et onustus abiit, sicut praeceperat ei Isai. Et venit ad carraginem, dum exercitus egrediebatur ad pugnam et vociferabatur in certamine. 
\verse Direxerunt ergo Israel et Philisthim aciem adversus aciem. 
\verse Derelinquens autem David vasa, quae attulerat, sub manu custodis ad sarcinas, cucurrit ad locum certaminis et interrogabat, si omnia recte agerentur erga fratres suos.
 \verse Cumque adhuc ille loqueretur eis, apparuit vir ille propugnator ascendens, Goliath nomine, Philisthaeus de Geth, ex castris Philisthinorum; et loquente eo haec eadem verba, audivit David. 
\verse Omnes autem Israelitae, cum vidissent virum, fugerunt a facie eius timentes eum valde. 
\verse Et dixit unus quispiam de Israel: “ Num vidistis virum hunc, qui ascendit? Ad exprobrandum enim Israeli ascendit. Virum ergo, qui percusserit eum, ditabit rex divitiis magnis et filiam suam dabit ei; et domum patris eius faciet absque tributo in Israel ”.
 \verse Et ait David ad viros, qui stabant secum, dicens: “ Quid dabitur viro, qui percusserit Philisthaeum hunc et tulerit opprobrium de Israel? Quis est enim hic Philisthaeus incircumcisus, qui exprobravit acies Dei viventis? ”. 
\verse Referebat autem ei populus eundem sermonem dicens: “ Haec dabuntur viro, qui percusserit eum ”. 
\verse Quod cum audisset Eliab frater eius maior, loquente eo cum aliis, iratus est contra David et ait: “ Quare venisti et cui dereliquisti pauculas oves illas in deserto? Ego novi superbiam tuam et nequitiam cordis tui, quia ut videres proelium descendisti ”. 
\verse Et dixit David: “ Quid feci? Numquid non verbum est? ”. 
\verse Et declinavit paululum ab eo ad alium dixitque eundem sermonem; et respondit ei populus verbum sicut prius.
 \verse Audita sunt autem verba, quae locutus est David, et annuntiata in conspectu Saul. 
\verse Ad quem cum fuisset adductus, locutus est ei: “ Non concidat cor cuiusquam in eo; ego servus tuus vadam et pugnabo adversus Philisthaeum istum ”. 
 \verse Et ait Saul ad David: “ Non vales resistere Philisthaeo isti nec pugnare adversus eum, quia puer es; hic autem vir bellator ab adulescentia sua ”. 
\verse Dixitque David ad Saul: “ Pascebat servus tuus patris sui gregem, et veniebat leo vel ursus tollebatque arietem de medio gregis. 
\verse Et sequebar eos et percutiebam eruebamque de ore eorum; et illi consurgebant adversum me, et apprehendebam mentum eorum et percutiebam interficiebamque eos. 
\verse Nam et leonem et ursum interfecit servus tuus; erit igitur et Philisthaeus hic incircumcisus quasi unus ex eis, quia ausus est maledicere exercitum Dei viventis ”. 
\verse Et ait David: “ Dominus, qui eruit me de manu leonis et de manu ursi, ipse liberabit me de manu Philisthaei huius ”. Dixit autem Saul ad David: “ Vade, et Dominus tecum sit ”.
 \verse Et induit Saul David vestimentis suis et imposuit galeam aeream super caput eius et vestivit eum lorica. 
\verse Accinctus ergo David gladio eius super vestem suam coepit tentare, si armatus posset incedere; non enim habebat consuetudinem. Dixitque David ad Saul: “ Non possum sic incedere, quia nec usum habeo ”. Et deposuit ea 
\verse et tulit baculum suum in manu sua; et elegit sibi quinque levissimos lapides de torrente et misit eos in peram pastoralem, qua ut sacculo lapidum utebatur, et fundam manu tulit et processit adversum Philisthaeum.
 \verse Ibat autem Philisthaeus incedens et appropinquans adversum David, et armiger eius ante eum. 
\verse Cumque inspexisset Philisthaeus et vidisset David, despexit eum; erat enim adulescens rufus et pulcher aspectu. 
\verse Et dixit Philisthaeus ad David: “ Numquid ego canis sum, quod tu venis ad me cum baculo? ”. Et maledixit Philisthaeus David in diis suis; 
\verse dixitque ad David: “ Veni ad me, et dabo carnes tuas volatilibus caeli et bestiis terrae ”. 
\verse Dixit autem David ad Philisthaeum: “ Tu venis ad me cum gladio et hasta et acinace; ego autem venio ad te in nomine Domini exercituum, Dei agminum Israel, quibus exprobrasti. 
\verse Hodie dabit te Dominus in manu mea, et percutiam te et auferam caput tuum a te; et dabo cadaver tuum et cadavera castrorum Philisthim hodie volatilibus caeli et bestiis terrae, ut sciat omnis terra quia est Deus in Israel, 
\verse et noverit universa ecclesia haec quia non in gladio nec in hasta salvat Dominus: ipsius enim est bellum, et tradet vos in manus nostras ”.
 \verse Cum ergo surrexisset Philisthaeus et veniret et appropinquaret contra David, festinavit David et cucurrit ad pugnam adversum Philisthaeum. 
\verse Et misit manum suam in peram tulitque unum lapidem et funda iecit; et percussit Philisthaeum in fronte, et infixus est lapis in fronte eius, et cecidit in faciem suam super terram. 
\verse Praevaluitque David adversum Philisthaeum in funda et in lapide; percussumque Philisthaeum interfecit. Cumque gladium non haberet in manu, David 
\verse cucurrit et stetit super Philisthaeum; et tulit gladium eius et eduxit eum de vagina sua et interfecit eum praeciditque caput eius.
 Videntes autem Philisthim quod mortuus esset fortissimus eorum fugerunt. 
\verse Et consurgentes viri Israel et Iudae vociferati sunt et persecuti Philisthaeos usque dum venirent ad Geth et usque ad portas Accaron. Cecideruntque vulnerati de Philisthim in via a Saarim usque ad Geth et usque ad Accaron. 
\verse Et revertentes filii Israel, postquam persecuti fuerant Philisthaeos, praedati sunt castra eorum. 
\verse Assumens autem David caput Philisthaei attulit illud in Ierusalem; arma vero eius posuit in tabernaculo.
 \verse Eo autem tempore, quo viderat Saul David egredientem contra Philisthaeum, ait ad Abner principem militiae: “ De qua stirpe descendit hic adulescens, Abner? ”. Dixitque Abner: “ Vivit anima tua, rex, quia non novi ”. 
\verse Et ait rex: “ Interroga tu, cuius filius sit iste puer ”. 
\verse Cumque regressus esset David, percusso Philisthaeo, tulit eum Abner et introduxit coram Saul caput Philisthaei habentem in manu. 
\verse Et ait ad eum Saul: “ De qua progenie es, o adulescens? ”. Dixitque David: “ Filius servi tui Isai Bethlehemitae ego sum ”.
 
\begin{biblechapter}
\verse Et factum est cum complesset loqui ad Saul, anima Ionathan colligata est animae David, et dilexit eum Ionathan quasi animam suam. 
\verse Tulitque eum Saul in die illa et non concessit ei, ut reverteretur in domum patris sui. 
\verse Inierunt autem Ionathan et David foedus; diligebat enim eum quasi animam suam. 
 \verse Et exspoliavit se Ionathan tunicam, qua erat vestitus, et dedit eam David et reliqua vestimenta sua usque ad gladium et arcum suum et usque ad balteum. 
\verse Egrediebatur quoque David ad omnia, quaecumque misisset eum Saul, et prospere agebat; posuitque eum Saul super viros belli, et acceptus erat in oculis universi populi, etiam in conspectu famulorum Saul.
 \verse Porro cum reverterentur, cum rediret David, percusso Philisthaeo, egressae sunt mulieres de universis urbibus Israel cantantes chorosque ducentes in occursum Saul regis in tympanis et in canticis laetitiae et in sistris. 
\verse Et praecinebant mulieres ludentes atque dicentes:
 “ Percussit Saul milia sua,
 et David decem milia sua ”.
 \verse Iratus est autem Saul nimis, et displicuit in oculis eius iste sermo, dixitque: “ Dederunt David decem milia et mihi dederunt milia; quid ei superest nisi solum regnum? ”. 
\verse Non rectis ergo oculis Saul aspiciebat David ex die illa et deinceps.
 \verse Post diem autem alteram invasit spiritus Dei malus Saul, et vaticinabatur in medio domus suae; David autem psallebat manu sua sicut per singulos dies, tenebatque Saul lanceam. 
\verse Et sustulit eam putans quod configere posset David cum pariete; et declinavit David a facie eius secundo.
 \verse Et timuit Saul David, eo quod esset Dominus cum eo et a se recessisset. 
 \verse Amovit ergo eum Saul a se et fecit eum tribunum super mille viros; et egrediebatur et intrabat in conspectu populi. 
\verse In omnibus quoque viis suis David prospere agebat, et Dominus erat cum eo. 
\verse Vidit itaque Saul quod prospere ageret nimis et coepit pavere eum; 
\verse omnis autem Israel et Iuda diligebat David; ipse enim egrediebatur et ingrediebatur ante eos.
 \verse Dixit autem Saul ad David: “ Ecce filia mea maior Merob, ipsam dabo tibi uxorem; tantummodo esto mihi vir fortis et proeliare bella Domini ”. Saul autem reputabat dicens: “ Non sit manus mea in eo, sed sit super illum manus Philisthinorum ”. 
\verse Ait autem David ad Saul: “ Quis ego sum, aut quae est vita mea aut cognatio patris mei in Israel, ut fiam gener regis? ”. 
\verse Factum est autem tempus, cum deberet dari Merob filia Saul David, data est Hadriel Molathitae uxor.
 \verse Dilexit autem Michol filia Saul altera David, et nuntiatum est Saul, et placuit ei; 
\verse dixitque Saul: “ Dabo eam illi, ut fiat ei in scandalum, et sit super eum manus Philisthinorum ”. Dixit ergo Saul ad David altera vice: “ Gener meus eris hodie ”. 
\verse Et mandavit Saul servis suis: “ Loquimini ad David secreto dicentes: “Ecce places regi, et omnes servi eius diligunt te; nunc ergo esto gener regis” ”. 
\verse Et locuti sunt servi Saul in auribus David omnia verba haec, et ait David: “ Num parum vobis videtur generum esse regis? Ego autem sum vir pauper et tenuis ”. 
\verse Et renuntiaverunt servi Saul dicentes: “ Huiuscemodi verba locutus est David ”. 
\verse Dixit autem Saul: “ Sic loquimini ad David: “Non habet necesse rex sponsalia, nisi tantum centum praeputia Philisthinorum, ut fiat ultio de inimicis regis” ”. Porro Saul cogitabat tradere David in manibus Philisthinorum.
 \verse Cumque renuntiassent servi eius David verba, quae dixerat Saul, placuit sermo in oculis David, ut fieret gener regis. 
\verse Et nondum erant dies impleti, cum David surgens abiit cum viris, qui sub eo erant, et percussit ex Philisthim ducentos viros; et attulit praeputia eorum, et annumeraverunt ea regi, ut esset gener eius.
 Dedit itaque ei Saul Michol filiam suam uxorem. 
\verse Et vidit Saul et intellexit quia Dominus esset cum David; Michol autem filia Saul diligebat eum. 
 \verse Et Saul magis coepit timere David; factusque est Saul inimicus David cunctis diebus. 
\verse Et egressi sunt principes Philisthinorum; et, quotiescumque egrediebantur, prospere agebat David magis quam omnes servi Saul, et celebre factum est nomen eius nimis.
 
\begin{biblechapter}
\verse Locutus est autem Saul ad Ionathan filium suum et ad omnes servos suos de occisione David; porro Ionathan filius Saul diligebat David valde. 
\verse Et indicavit Ionathan David dicens: “ Quaerit Saul pater meus occidere te; quapropter observa te, quaeso, mane; et manebis clam et absconderis. 
\verse Ego autem egrediens stabo iuxta patrem meum in agro, ubicumque fueris; et ego loquar de te ad patrem meum et, quodcumque videro, nuntiabo tibi ”. 
\verse Locutus est ergo Ionathan de David bona ad Saul patrem suum dixitque ad eum: “ Ne peccet rex in servum suum David, quia non peccavit tibi, et opera eius bona sunt tibi valde. 
\verse Et posuit animam suam in manu sua et percussit Philisthaeum, et fecit Dominus victoriam magnam universo Israeli; vidisti et laetatus es. Quare ergo peccas in sanguine innoxio interficiens David, qui est absque culpa? ”. 
 \verse Quod cum audisset Saul, placatus voce Ionathan iuravit: “ Vivit Dominus quia non occidetur ”. 
\verse Vocavit itaque Ionathan David et indicavit ei omnia verba haec; et introduxit lonathan David ad Saul, et fuit ante eum, sicut fuerat heri et nudiustertius.
 \verse Motum est autem rursum bellum, et egressus David pugnavit adversum Philisthim percussitque eos plaga magna; et fugerunt a facie eius. 
\verse Et factus est spiritus Domini malus in Saul; sedebat autem in domo sua et tenebat lanceam, porro David psallebat in manu sua. 
\verse Nisusque est Saul configere lancea David in pariete; et declinavit David a facie Saul, lancea autem, casso vulnere, perlata est in parietem. Et David fugit et salvatus est nocte illa. 
\verse Misit ergo Saul satellites suos in domum David, ut custodirent eum, et interficeretur mane.
 Quod cum annuntiasset David Michol uxor sua dicens: “ Nisi salvaveris te nocte hac, cras morieris ”, 
\verse deposuit eum per fenestram. Porro ille abiit et aufugit atque salvatus est.
 \verse Tulit autem Michol theraphim et posuit eum super lectum; et pellem pilosam caprarum posuit ad caput eius et operuit eum vestimentis. 
\verse Misit autem Saul nuntios, qui raperent David, et responsum est quod aegrotaret. 
\verse Rursumque misit Saul nuntios, ut viderent David, dicens: “ Afferte eum ad me in lecto, ut occidatur ”. 
\verse Cumque venissent nuntii, inventus est theraphim super lectum, et pellis caprarum ad caput eius. 
\verse Dixitque Saul ad Michol: “ Quare sic illusisti mihi et dimisisti inimicum meum, ut fugeret? ”. Et respondit Michol ad Saul: “ Quia ipse locutus est mihi: “Dimitte me, alioquin interficiam te” ”.
 \verse David autem fugiens salvatus est et venit ad Samuel in Rama et nuntiavit ei omnia, quae fecerat sibi Saul. Et abierunt ipse et Samuel et morati sunt in Naioth.
 \verse Nuntiatum est autem Sauli a dicentibus: “ Ecce David in Naioth in Rama ”. 
\verse Misit ergo Saul nuntios, ut raperent David. Qui cum vidissent cuneum prophetarum vaticinantium et Samuel stantem super eos, factus est in illis spiritus Dei, et vaticinari coeperunt etiam ipsi. 
\verse Quod cum nuntiatum esset Sauli, misit alios nuntios; vaticinati sunt autem et illi. Et rursum Saul misit tertios nuntios, qui et ipsi vaticinati sunt.
 \verse Abiit autem etiam ipse in Rama et venit usque ad cisternam magnam, quae est in Socho; et interrogavit et dixit: “ In quo loco sunt Samuel et David? ”. Dictumque est ei: “ Ecce in Naioth sunt in Rama ”. 
\verse Et abiit inde in Naioth in Rama; et factus est etiam super eum spiritus Dei, et ambulabat ingrediens et vaticinans, usquedum veniret in Naioth in Rama. 
\verse Et exspoliavit se etiam ipse vestimentis suis et vaticinatus est cum ceteris coram Samuel; et cecidit nudus tota die illa et nocte, unde et exivit proverbium: “ Num et Saul inter prophetas? ”.
 
\begin{biblechapter}
\verse Fugit autem David de Naioth, quae est in Rama, veniensque locutus est coram Ionathan: “ Quid feci? Quae est iniquitas mea et quod peccatum meum in patrem tuum, quia quaerit animam meam? ”. 
\verse Qui dixit ei: “ Absit, non morieris; neque enim faciet pater meus quidquam grande vel parvum, nisi prius indicaverit mihi; hoc ergo celavit me pater meus tantummodo? Nequaquam erit istud ”. 
\verse Et rursum respondit David et ait: “ Scit profecto pater tuus quia inveni gratiam in oculis tuis et dixit: “Nesciat hoc Ionathan, ne forte tristetur”. Quinimmo vivit Dominus, et vivit anima tua, quia uno tantum gradu ego morsque dividimur ”.
 \verse Et ait Ionathan ad David: “ Quid desiderat anima tua, ut faciam tibi? ”. 
\verse Dixit autem David ad Ionathan: “ Ecce neomenia est crastino, et ego ex more sedere soleo iuxta regem ad vescendum; dimitte ergo me, ut abscondar in agro usque ad vesperam diei tertiae. 
\verse Si requisierit me pater tuus, respondebis ei: “Rogavit me David, ut iret celeriter in Bethlehem civitatem suam, quia victimae annuae ibi sunt universis contribulibus eius”. 
\verse Si dixerit: “Bene”, pax erit servo tuo; si autem fuerit iratus, scito quia malum decretum est ab eo. 
\verse Fac ergo misericordiam in servum tuum, quia foedus Domini me famulum tuum tecum inire fecisti; si autem est in me aliqua iniquitas, tu me interfice et ad patrem tuum ne introducas me ”. 
\verse Et ait Ionathan: “ Absit hoc a te; neque enim fieri potest ut, si certo cognovero malum decretum esse a patre meo contra te, non annuntiem tibi ”. 
\verse Responditque David ad Ionathan: “ Quis nuntiabit mihi, si quid forte responderit tibi pater tuus dure? ”.
 \verse Et ait Ionathan ad David: “ Veni, egrediamur foras in agrum ”. Cumque exissent ambo in agrum, 
\verse ait Ionathan ad David: “ Vivit Dominus, Deus Israel, investigabo sententiam patris mei hoc fere tempore cras vel perendie; et si aliquid boni fuerit super David, et non statim miserim ad te et notum tibi fecerim, 
\verse haec faciat Dominus in Ionathan et haec augeat! Si autem perseveraverit patris mei malitia adversum te, hoc quoque notum faciam tibi et dimittam te, ut vadas in pace. Et sit Dominus tecum, sicut fuit cum patre meo. 
\verse Et, si vixero, facies mihi misericordiam Domini; si vero mortuus fuero, 
 \verse non auferas misericordiam tuam a domo mea usque in sempiternum, quando eradicaverit Dominus inimicos David unumquemque de terra ”. 
\verse Pepigit ergo foedus Ionathan cum domo David dicens: “ Requirat Dominus de manu inimicorum David! ”. 
\verse Et addidit Ionathan ut faceret David iurare per dilectionem suam erga illum; sicut animam enim suam, ita diligebat eum.
 \verse Dixitque ad eum Ionathan: “ Cras neomenia est, et requireris; 
\verse vacua erit enim sessio tua. Perendie descendes festinus et venies in locum, ubi abscondisti te in die facti illius; et sedebis iuxta acervum illum. 
\verse Et ego tres sagittas mittam iuxta eum et iaciam quasi exercens me ad signum. 
\verse Mittam quoque et puerum dicens ei: “Vade et affer mihi sagittas”. 
\verse Si dixero puero: “Ecce sagittae intra te sunt, tolle eas”, tu veni ad me, quia pax tibi est, et nihil est mali, vivit Dominus. Si autem sic locutus fuero puero: “Ecce sagittae ultra te sunt”, vade, quia dimisit te Dominus. 
\verse De verbo autem, quod locuti fuimus, ego et tu, sit Dominus inter me et te usque in sempiternum ”.
 \verse Absconditus est ergo David in agro; et venit neomenia, et sedit rex ad mensam ad comedendum. 
\verse Cumque sedisset rex super cathedram suam secundum consuetudinem, quae erat iuxta parietem, sedit Ionathan ex adverso, et sedit Abner ex latere Saul; vacuusque apparuit locus David. 
\verse Et non est locutus Saul quidquam in die illa; cogitabat enim quod forte evenisset ei, ut non esset mundus nec purificatus. 
\verse Cumque illuxisset dies secunda post neomeniam, rursum vacuus apparuit locus David; dixitque Saul ad Ionathan filium suum: “ Cur non venit filius Isai nec heri nec hodie ad vescendum? ”. 
\verse Et respondit Ionathan Sauli: “ Rogavit me obnixe, ut iret in Bethlehem, 
\verse et ait: “Dimitte me, quoniam sacrificium familiae est in civitate, et frater meus ipse accersivit me; nunc ergo, si inveni gratiam in oculis tuis, vadam cito et videbo fratres meos”. Ob hanc causam non venit ad mensam regis ”.
 \verse Iratus autem Saul adversum Ionathan dixit ei: “ Fili mulieris perversae, numquid ignoro quia diligis filium Isai in confusionem tuam et in confusionem nuditatis matris tuae? 
\verse Omnibus enim diebus, quibus filius Isai vixerit super terram, non stabilieris tu neque regnum tuum; itaque iam nunc mitte et adduc eum ad me, quia filius mortis est ”. 
\verse Respondens autem Ionathan Sauli patri suo ait: “ Quare morietur? Quid fecit? ”. 
\verse Et arripuit Saul lanceam, ut percuteret eum; et intellexit Ionathan quod definitum esset patri suo, ut interficeret David. 
\verse Surrexit ergo Ionathan a mensa in ira furoris et non comedit in die neomeniae secunda panem; contristatus est enim super David, eo quod confudisset eum pater suus.
 \verse Cumque illuxisset mane, venit Ionathan in agrum ad locum constitutum a David et puer parvulus cum eo; 
\verse et ait ad puerum suum: “ Vade et affer mihi sagittas, quas ego iacio ”. Cumque puer cucurrisset, iecit sagittam trans puerum. 
\verse Venit itaque puer ad locum sagittae, quam miserat Ionathan, et clamavit Ionathan post tergum pueri et ait: “ Ecce ibi est sagitta porro ultra te ”. 
\verse Clamavitque Ionathan post tergum pueri: “ Festina velociter, ne steteris ”. Sustulit autem puer Ionathae sagittam et attulit ad dominum suum 
\verse et quid ageretur penitus ignorabat, tantummodo enim Ionathan et David rem noverant.
 \verse Dedit igitur Ionathan arma sua puero et dixit ei: “ Vade, defer in civitatem ”. 
\verse Cumque abisset puer, surrexit David de latere acervi et cadens pronus in terram adoravit tertio; et osculantes alterutrum fleverunt pariter, David autem amplius. 
\verse Dixit ergo Ionathan ad David: “ Vade in pace; iuravimus enim ambo in nomine Domini dicentes: Dominus erit inter me et te et inter semen meum et semen tuum usque in sempiternum ”.
 
\begin{biblechapter}
\verse Et surrexit David et abiit; sed et Ionathan ingressus est civitatem. 
 \verse Venit autem David in Nob ad Achimelech sacerdotem, et obstupuit Achimelech eo quod venisset David, et dixit ei: “ Quare tu solus et nullus est tecum? ”. 
\verse Et ait David ad Achimelech sacerdotem: “ Rex praecepit mihi negotium et dixit: “Nemo sciat rem, propter quam a me missus es, et cuiusmodi tibi praecepta dederim”; pueris vero condixi in illum et illum locum. 
\verse Nunc igitur, si habes ad manum quinque panes, da mihi, aut quidquid inveneris ”.
 \verse Et respondens sacerdos David ait ei: “ Non habeo panes laicos ad manum, sed tantum panem sanctum; si mundi sunt pueri maxime a mulieribus? ”. 
\verse Et respondit David sacerdoti et dixit ei: “ Equidem, si de mulieribus agitur, continuimus nos ab heri et nudiustertius. Quando egrediebar, fuerunt corpora puerorum sancta, quamvis iter esset profanum. Quanto magis hodie sunt sancti quoad corpora ”. 
\verse Dedit ergo ei sacerdos sanctificatum panem; neque enim erat ibi panis, nisi tantum panes propositionis, qui sublati fuerant a facie Domini, ut ponerentur panes calidi.
 \verse Erat autem ibi vir de servis Saul in die illa retentus ante Dominum; et nomen eius Doeg Idumaeus, potentissimus pastorum Saul.
 \verse Dixit autem David ad Achimelech: “ Si habes hic ad manum hastam aut gladium? Quia gladium meum et arma mea non tuli mecum; negotium enim regis urgebat ”. 
\verse Et dixit sacerdos: “ Ecce hic gladius Goliath Philisthaei, quem percussisti in valle Terebinthi; est involutus pallio post ephod. Si istum vis tollere, tolle, neque enim est alius hic absque eo ”. Et ait David: “ Non est huic alter similis; da mihi eum ”.
 \verse Surrexit itaque David et fugit in die illa a facie Saul et venit ad Achis regem Geth. 
\verse Dixeruntque ei servi Achis: “ Numquid non iste est David rex terrae? Nonne huic cantabant per choros dicentes: “Percussit Saul milia sua, et David decem milia sua”? ”.
 \verse Posuit autem David sermones istos in corde suo et extimuit valde a facie Achis regis Geth. 
\verse Et immutavit os suum coram eis; et insaniebat inter manus eorum et impingebat in ostia portae, defluebantque salivae in barbam. 
 \verse Et ait Achis ad servos suos: “ Vidistis hominem insanum. Quare adduxistis eum ad me? 
\verse An desunt nobis furiosi, quod introduxistis istum, ut fureret, me praesente? Hicine ingredietur domum meam? ”.
 
\begin{biblechapter}
\verse Abiit ergo inde David et fugit in speluncam Odollam; quod cum audissent fratres eius et omnis domus patris eius, descenderunt ad eum illuc. 
\verse Et convenerunt ad eum omnes, qui erant in angustia constituti et oppressi aere alieno et amaro animo; et factus est eorum princeps, fueruntque cum eo quasi quadringenti viri.
 \verse Et profectus est David inde in Maspha, quae est Moab, et dixit ad regem Moab: “ Maneat, oro, pater meus et mater mea vobiscum, donec sciam quid faciat mihi Deus ”. 
\verse Et reliquit eos ante faciem regis Moab; manseruntque apud eum cunctis diebus, quibus David fuit in praesidio.
 \verse Dixitquc Gad propheta ad David: “ Noli manere in praesidio. Proficiscere et vade in terram Iudae ”. Et profectus David venit in saltum Haret.
 \verse Et audivit Saul quod detectus fuisset David et viri, qui erant cum eo. Saul autem, cum maneret in Gabaa et esset sub myrice, quae est in excelso, hastam manu tenens, cunctique servi eius circumstarent eum, 
\verse ait ad servos suos, qui assistebant ei: “ Audite, Beniaminitae. Etiam omnibus vobis dabit filius Isai agros et vineas et universos vos faciet tribunos et centuriones, 
\verse quoniam coniurastis omnes adversum me. Et non est qui mihi renuntiet quod filius meus foedus iunxerit cum filio Isai; non est qui vicem meam doleat ex vobis, nec qui annuntiet mihi quod suscitaverit filius meus servum meum adversum me insidiantem mihi sicut hodie ”.
 \verse Respondens autem Doeg Idumaeus, qui assistebat cum servis Saul: “ Vidi, inquit, filium Isai in Nob apud Achimelech filium Achitob; 
\verse qui consuluit pro eo Dominum et cibaria dedit ei, sed et gladium Goliath Philisthaei dedit illi ”.
 \verse Misit ergo rex ad accersendum Achimelech sacerdotem filium Achitob et omnem domum patris eius, sacerdotum, qui erant in Nob; qui venerunt universi ad regem. 
\verse Et ait Saul: “ Audi, fili Achitob ”. Qui respondit: “ Praesto sum, domine ”. 
\verse Dixitque ad eum Saul: “ Quare coniurastis adversum me, tu et filius Isai, et dedisti ei panes et gladium et consuluisti pro eo Deum, ut consurgeret adversum me insidiator, sicut est hodie? ”. 
\verse Respondensque Achimelech regi ait: “ Et quis in omnibus servis tuis sicut David fidelis et gener regis et dux satellitum tuorum et gloriosus in domo tua? 
\verse Num hodie coepi consulere pro eo Deum? Absit hoc a me, ne suspicetur rex adversus servum suum rem huiuscemodi, adversus universam domum patris mei; non enim scivit servus tuus quidquam super hoc negotio, vel modicum vel grande ”. 
\verse Dixitque rex: “ Morte morieris, Achimelech, tu et omnis domus patris tui ”.
 \verse Et ait rex emissariis, qui circumstabant eum: “ Convertimini et interficite sacerdotes Domini, nam manus eorum cum David est; scientes quod fugisset, non indicaverunt mihi ”. Noluerunt autem servi regis extendere manum suam in sacerdotes Domini. 
\verse Et ait rex ad Doeg: “ Convertere tu et irrue in sacerdotes ”. Conversusque Doeg Idumaeus irruit in sacerdotes; et trucidavit in die illa octoginta quinque viros vestitos ephod lineo. 
\verse Nob autem civitatem sacerdotum percussit in ore gladii, viros et mulieres, parvulos et lactantes, bovem et asinum et ovem in ore gladii.
 \verse Evadens autem unus filius Achimelech filii Achitob, cuius nomen erat Abiathar, fugit ad David 
\verse et annuntiavit ei quod occidisset Saul sacerdotes Domini. 
\verse Et ait David ad Abiathar: “ Sciebam in die illa quod, cum ibi esset Doeg Idumaeus, procul dubio annuntiaret Saul; ego sum reus omnium animarum domus patris tui. 
\verse Mane mecum, ne timeas; qui enim quaerit animam meam, quaerit et animam tuam, mecumque servaberis ”.
 
\begin{biblechapter}
\verse Et nuntiaverunt David dicentes: “ Ecce Philisthim oppugnant Ceila et diripiunt areas ”. 
\verse Consuluit igitur David Dominum dicens: “ Num vadam et percutiam Philisthaeos istos? ”. Et ait Dominus ad David: “ Vade et percuties Philisthaeos et salvabis Ceila ”. 
\verse Et dixerunt viri, qui erant cum David, ad eum: “ Ecce nos hic in Iuda consistentes timemus; quanto magis si ierimus in Ceila adversum agmina Philisthinorum? ”. 
\verse Rursum ergo David consuluit Dominum, qui respondens ei ait: “ Surge et vade in Ceila; ego enim tradam Philisthaeos in manu tua ”. 
\verse Abiit ergo David et viri eius in Ceila et pugnavit adversum Philisthaeos et abegit iumenta eorum et percussit eos plaga magna: et salvavit David habitatores Ceilae. 
\verse Porro cum fugisset Abiathar filius Achimelcch ad David, et ipse cum David in Ceila ephod secum habens descenderat.
 \verse Nuntiatum est autem Saul quod venisset David in Ceila, et ait Saul: “ Tradidit eum Deus in manus meas; conclususque est introgressus urbem, in qua portae et serae sunt ”. 
\verse Et convocavit Saul omnem populum, ut ad pugnam descenderet in Ceila et obsideret David et viros eius. 
\verse Quod cum rescisset David quia praepararet ei Saul clam malum, dixit ad Abiathar sacerdotem: “ Applica ephod ”. 
 \verse Et ait David: “ Domine, Deus Israel, audivit famam servus tuus quod disponat Saul venire ad Ceila, ut evertat urbem propter me. 
\verse Si tradent me viri Ceilae in manus eius? Et si descendet Saul, sicut audivit servus tuus? Domine, Deus Israel, indica servo tuo ”. Et ait Dominus: “ Descendet ”. 
\verse Dixitque David: “ Si tradent viri Ceilae me et viros, qui sunt mecum, in manu Saul? ”. Et dixit Dominus: “ Tradent ”.
 \verse Surrexit ergo David et viri eius quasi sescenti et egressi de Ceila huc atque illuc vagabantur incerti. Nuntiatumque est Saul quod fugisset David de Ceila, quam ob rem destitit exire.
 \verse Morabatur autem David in deserto in locis firmissimis mansitque in monte, in deserto Ziph; et quaerebat eum Saul cunctis diebus, sed non tradidit eum Deus in manus eius. 
\verse Et cognovit David quod egressus esset Saul, ut quaereret animam eius; porro David erat in deserto Ziph in Horesa. 
\verse Et surrexit Ionathan filius Saul et abiit ad David in Horesa; et confortavit manus eius in Deo dixitque ei: 
\verse “ Ne timeas, neque enim inveniet te manus Saul patris mei; et tu regnabis super Israel, et ego ero tibi secundus; sed et Saul pater meus scit hoc ”. 
 \verse Percussit igitur uterque foedus coram Domino; mansitque David in Horesa, Ionathan autem reversus est in domum suam.
 \verse Ascenderunt autem Ziphaei ad Saul in Gabaa dicentes: “ Nonne David latitat apud nos in locis tutissimis in Horesa, in colle Hachila, quae est ad meridiem deserti? 
\verse Nunc ergo, si desideravit anima tua, rex, ut descenderes, descende; nostrum autem erit ut tradamus eum in manus regis ”. 
\verse Dixitque Saul: “ Benedicti vos a Domino, quia doluistis vicem meam. 
\verse Abite, oro, et diligentius praeparate et curiosius agite; et considerate locum, ubi sit pes eius, vel quis viderit eum ibi; dictum est enim ad me quod callidus sit valde. 
 \verse Considerate et videte omnia latibula eius, in quibus absconditur, et revertimini ad me ad certum locum, ut vadam vobiscum; quodsi fuerit in regione, perscrutabor eum in cunctis regionibus Iudae ”. 
\verse At illi surgentes abierunt in Ziph ante Saul.
 David autem et viri eius erant in deserto Maon, in Araba ad meridiem deserti. 
 \verse Ivit ergo Saul et socii eius ad quaerendum eum, et nuntiatum est David; descenditque ad petram et versabatur in deserto Maon. Quod cum audisset Saul, persecutus est David in deserto Maon. 
\verse Et ibat Saul ad latus montis ex parte una, David autem et viri eius erant in latere montis ex parte altera; porro David praeceps fugiebat a facie Saul. Itaque Saul et viri eius in modum coronae cingebant David et viros eius, ut caperent eos. 
\verse Et nuntius venit ad Saul dicens: “ Festina et veni, quoniam infuderunt se Philisthim super terram ”. 
\verse Reversus est ergo Saul desistens persequi David; et perrexit in occursum Philisthinorum. Propter hoc vocaverunt locum illum: “ Petram dividentem ”.
 
\begin{biblechapter}
\verse Ascendit ergo David inde et habitavit in locis tutissimis Engaddi. 
\verse Cumque reversus esset Saul, postquam persecutus est Philisthaeos, nuntiaverunt ei dicentes: “ Ecce David in deserto est Engaddi ”. 
\verse Assumens ergo Saul tria milia electorum virorum ex omni Israel perrexit ad investigandum David et viros eius ad rupes ibicum. 
\verse Et venit ad caulas ovium, quae se offerebant vianti.
 Eratque ibi spelunca, quam ingressus est Saul, ut purgaret ventrem; porro David et viri eius in interiore parte speluncae latebant. 
\verse Et dixerunt viri David ad eum: “ Ecce dies, de qua locutus est Dominus ad te: “Ego trado tibi inimicum tuum, ut facias ei sicut placuerit in oculis tuis” ”. Surrexit ergo David et praecidit oram chlamydis Saul silenter. 
\verse Post haec cor David percussit eum, eo quod abscidisset oram chlamydis Saul, 
\verse dixitque ad viros suos: “ Propitius mihi sit Dominus, ne faciam hanc rem domino meo, christo Domini, ut mittam manum meam in eum, quoniam christus Domini est ”. 
\verse Et cohibuit David viros suos sermonibus et non permisit eos, ut consurgerent in Saul.
 Porro Saul exsurgens de spelunca pergebat coepto itinere. 
\verse Surrexit autem et David post eum et egressus de spelunca clamavit post tergum Saul dicens: “ Domine mi rex! ”. Et respexit Saul post se, et inclinans se David pronus in terram adoravit 
\verse dixitque ad Saul: “ Quare audis verba hominum loquentium: “David quaerit malum adversum te?”. 
\verse Ecce hodie viderunt oculi tui quod tradiderit te Dominus hodie in manu mea in spelunca; et dictum est mihi, ut occiderem te, sed pepercit tibi oculus meus. Dixi enim: Non extendam manum meam in dominum meum, quia christus Domini est 
\verse et pater meus. Quin potius vide et cognosce oram chlamydis tuae in manu mea, quoniam, cum praeciderem summitatem chlamydis tuae, nolui occidere te. Animadverte et vide quoniam non est in manu mea malum neque iniquitas, neque peccavi in te; tu autem insidiaris animae meae, ut auferas eam. 
\verse Iudicet Dominus inter me et te et ulciscatur me Dominus ex te; manus autem mea non sit in te. 
\verse Sicut et in proverbio antiquo dicitur: “Ab impiis egredietur impietas”, manus ergo mea non sit in te. 
\verse Quem sequitur rex Israel? Quem persequeris? Canem mortuum et pulicem unum. 
 \verse Sit Dominus iudex et iudicet inter me et te et videat et diiudicet causam meam et eruat me de manu tua ”. 
\verse Cum autem complesset David loquens sermones huiuscemodi ad Saul, dixit Saul: “ Numquid vox haec tua est, fili mi David? ”. Et levavit Saul vocem suam et flevit. 
\verse Dixitque ad David: “ Iustior tu es quam ego; tu enim tribuisti mihi bona, ego autem reddidi tibi mala. 
\verse Et tu indicasti hodie, quae feceris mihi bona, quomodo tradiderit me Dominus in manu tua, et non occideris me. 
\verse Quis enim, cum invenerit inimicum suum, dimittet eum in via bona? Sed Dominus reddat tibi vicissitudinem hanc, pro eo quod hodie operatus es in me. 
\verse Et nunc, quia scio quod certissime regnaturus sis et habiturus in manu tua regnum Israel, 
\verse iura mihi in Domino, ne deleas semen meum post me neque auferas nomen meum de domo patris mei ”. 
\verse Et iuravit David Sauli. Abiit ergo Saul in domum suam, et David et viri eius ascenderunt ad praesidium.
 
\begin{biblechapter}
\verse Mortuus est autem Samuel; et congregatus est universus Israel, et planxerunt eum et sepelierunt eum in domo sua in Rama.
 Consurgensque David descendit in desertum Maon. 
\verse Erat autem vir quispiam in solitudine Maon, et possessio eius in Carmel; et homo ille magnus nimis; erantque ei oves tria milia et mille caprae. Et accidit ut tonderet gregem suum in Carmel. 
\verse Nomen autem viri illius erat Nabal et nomen uxoris eius Abigail. Eratque mulier illa prudentissima et speciosa; porro vir eius durus et moribus malis; erat autem de genere Chaleb.
 \verse Cum ergo audisset David in deserto quod tonderet Nabal gregem suum, 
\verse misit decem iuvenes et dixit eis: “ Ascendite in Carmel et venietis ad Nabal et salutabitis eum ex nomine meo pacifice 
\verse et dicetis fratri meo: “Et tibi pax et domui tuae pax et omnibus, quaecumque habes, sit pax! 
\verse Et nunc audivi quod tonsores essent apud te. Pastores autem tui erant nobiscum in deserto; numquam eis molesti fuimus, nec aliquando defuit eis quidquam de grege omni tempore, quo fuerunt nobiscum in Carmel. 
\verse Interroga pueros tuos, et indicabunt tibi. Nunc ergo inveniant pueri isti gratiam in oculis tuis, in die enim bona venimus; quodcumque invenerit manus tua, da servis tuis et filio tuo David” ”.
 \verse Cumque venissent pueri David, locuti sunt ad Nabal omnia verba haec ex nomine David et siluerunt. 
\verse Respondens autem Nabal pueris David ait: “ Quis est David, et quis est filius Isai? Hodie increverunt servi, qui fugiunt dominos suos. 
\verse Tollam ergo panes meos et aquas meas et carnes pecorum, quae occidi, tonsoribus meis et dabo viris, quos nescio unde sint? ”. 
\verse Regressi sunt itaque pueri David per viam suam et reversi venerunt et nuntiaverunt ei omnia verba haec. 
\verse Tunc David ait viris suis: “ Accingatur unusquisque gladio suo! ”. Et accincti sunt singuli gladio suo, accinctusque est et David ense suo, et secuti sunt David quasi quadringenti viri; porro ducenti remanserunt ad sarcinas.
 \verse Abigail autem uxori Nabal nuntiavit unus de pueris suis dicens: “ Ecce misit David nuntios de deserto, ut benedicerent domino nostro, sed aversatus est eos. 
 \verse Homines isti boni satis fuerunt nobis et non molesti; nec quidquam aliquando periit omni tempore, quo sumus conversati cum eis in deserto. 
\verse Pro muro erant nobis tam in nocte quam in die omnibus diebus, quibus pavimus apud eos greges. 
\verse Quam ob rem considera et recogita quid facias, quoniam malum decretum est adversus dominum nostrum et adversus domum eius universam. Et ipse filius Belial est, ita ut nemo ei possit loqui ”.
 \verse Festinavit igitur Abigail et tulit ducentos panes et duos utres vini et quinque arietes coctos et quinque sata frumenti tosti et centum ligaturas uvae passae et ducentas massas caricarum et imposuit super asinos. 
\verse Dixitque pueris suis: “ Praecedite me, ecce ego post tergum sequar vos ”. Viro autem suo Nabal non indicavit. 
\verse Cum ergo ascendisset asinum et descenderet in tegmine montis, David et viri eius descendebant in occursum eius; quibus et illa occurrit. 
\verse Et aiebat David: “ Vere frustra servavi omnia, quae huius erant in deserto, et non periit quidquam de cunctis, quae ad eum pertinebant; et reddidit mihi malum pro bono. 
\verse Haec faciat Deus inimicis David et haec addat, si reliquero de omnibus, quae ad eum pertinent, usque mane quidquid masculini sexus ”.
 \verse Cum autem vidisset Abigail David, festinavit et descendit de asino et procidit coram David super faciem suam et adoravit super terram. 
\verse Et cecidit ad pedes eius et dixit: “ In me sit, domine mi, haec iniquitas; loquatur, obsecro, ancilla tua in auribus tuis, et audi verba famulae tuae. 
 \verse Ne ponat, oro, dominus meus cor suum super virum istum iniquum Nabal, quia secundum nomen suum stultus est, et est stultitia cum eo; ego autem ancilla tua non vidi pueros domini mei, quos misisti. 
\verse Nunc ergo, domine mi, vivit Dominus, et vivit anima tua, quia Dominus prohibuit te, ne venires in sanguine et salvares te manu tua; et nunc fiant sicut Nabal inimici tui et qui quaerunt domino meo malum. 
\verse Quapropter suscipe benedictionem hanc, quam attulit ancilla tua domino meo, et da pueris, qui sequuntur dominum meum. 
\verse Aufer iniquitatem famulae tuae. Faciens enim faciet Dominus domino meo domum fidelem, quia proelia Domini dominus meus proeliatur; malitia ergo non inveniatur in te omnibus diebus vitae tuae. 
\verse Si enim surrexerit aliquando homo persequens te et quaerens animam tuam, erit anima domini mei custodita in fasciculo vitae apud Dominum Deum tuum; sed inimicorum tuorum animam ipse iaciat in impetu et circulo fundae. 
\verse Cum ergo fecerit Dominus domino meo omnia, quae locutus est, bona de te et constituerit te ducem super Israel, 
\verse non erit tibi hoc in singultum et in scrupulum cordis domino meo, quod effuderis sanguinem innoxium et ipse te ultus fueris; et cum benefecerit Dominus domino meo, recordaberis ancillae tuae ”.
 \verse Et ait David ad Abigail: “ Benedictus Dominus, Deus Israel, qui misit te hodie in occursum meum. Et benedicta prudentia tua, 
\verse et benedicta tu, quae prohibuisti me hodie, ne irem ad sanguinem et ulciscerer me manu mea. 
\verse Alioquin, vivit Dominus, Deus Israel, qui prohibuit me malum facere tibi, nisi cito venisses in occursum mihi, non remansisset Nabal usque ad lucem matutinam quidquid masculini sexus ”. 
\verse Suscepit ergo David de manu eius omnia, quae attulerat ei, dixitque ei: “ Vade pacifice in domum tuam. Ecce audivi vocem tuam et honoravi faciem tuam ”.
 \verse Venit autem Abigail ad Nabal; et ecce erat ei convivium in domo eius quasi convivium regis, et cor Nabal iucundum; erat enim ebrius nimis. Et non indicavit ei verbum pusillum aut grande usque in mane. 
\verse Diluculo autem, cum digessisset vinum Nabal, haec indicavit ei uxor sua; et emortuum est cor eius intrinsecus, et factus est quasi lapis. 
\verse Cumque pertransissent decem dies, percussit Dominus Nabal, et mortuus est.
 \verse Quod cum audisset David mortuum Nabal, ait: “ Benedictus Dominus, qui iudicavit causam opprobrii mei de manu Nabal et servum suum custodivit a malo et malitiam Nabal reddidit Dominus in caput eius ”.
 Misit ergo David et locutus est ad Abigail, ut sumeret eam sibi in uxorem. 
 \verse Et venerunt pueri David ad Abigail in Carmel et locuti sunt ad eam dicentes: “ David misit nos ad te, ut accipiat te sibi in uxorem ”. 
\verse Quae consurgens adoravit prona in terram et ait: “ Ecce famula tua sit in ancillam, ut lavet pedes servorum domini mei ”. 
\verse Et festinavit et surrexit Abigail et ascendit super asinum, et quinque puellae ierunt cum ea pedisequae eius; et secuta est nuntios David et facta est illi uxor.
 \verse Sed et Achinoam accepit David de Iezrahel, et fuit utraque uxor eius. 
\verse Saul autem dedit Michol filiam suam uxorem David Phalti filio Lais, qui erat de Gallim.
 
\begin{biblechapter}
\verse Et venerunt Ziphaei ad Saul in Gabaa dicentes: “ Ecce David absconditus est in colle Hachila, quae est ex adverso solitudinis ”. 
\verse Et surrexit Saul et descendit in desertum Ziph, et cum eo tria milia virorum de electis Israel, ut quaereret David in deserto Ziph. 
\verse Et castrametatus est Saul in colle Hachila, quae erat ex adverso solitudinis in via. David autem habitabat in deserto; videns autem quod venisset Saul post se in desertum, 
\verse misit exploratores et didicit quod illuc venisset certissime. 
\verse Et surrexit David et venit ad locum, ubi erat Saul. Cumque vidisset locum, in quo dormiebat Saul et Abner filius Ner princeps militiae eius, Saulem dormientem in carragine et reliquum vulgus per circuitum eius, 
\verse ait David ad Achimelech Hetthaeum et Abisai filium Sarviae fratrem Ioab dicens: “ Quis descendet mecum ad Saul in castra? ”. Dixitque Abisai: “ Ego descendam tecum ”.
 \verse Venerunt ergo David et Abisai ad populum nocte et invenerunt Saul iacentem et dormientem in carragine et hastam fixam in terra ad caput eius, Abner autem et populum dormientes in circuitu eius. 
\verse Dixitque Abisai ad David: “ Conclusit Deus hodie inimicum tuum in manus tuas; nunc ergo perfodiam eum lancea in terra semel, et secundo opus non erit ”. 
\verse Et dixit David ad Abisai: “ Ne interficias eum; quis enim extendit manum suam in christum Domini et innocens erit? ”. 
\verse Et dixit David: “ Vivit Dominus quia Dominus percutiet eum, aut dies eius veniet, ut moriatur, aut in proelium descendens peribit. 
\verse Propitius mihi sit Dominus, ne extendam manum meam in christum Domini. Nunc igitur tolle hastam, quae est ad caput eius, et scyphum aquae, et abeamus ”. 
 \verse Tulit ergo David hastam et scyphum aquae, qui erat ad caput Saul, et abierunt; et non erat quisquam, qui videret et intellegeret et vigilaret, sed omnes dormiebant, quia sopor Domini irruerat super eos.
 \verse Cumque transisset David ex adverso et stetisset in vertice montis de longe, et esset grande intervallum inter eos, 
\verse clamavit David ad populum et ad Abner filium Ner dicens: “ Nonne respondebis, Abner? ”. Et respondens Abner ait: “ Quis es tu? Clamasti ad regem! ”. 
\verse Et ait David ad Abner: “ Numquid non vir tu es? Et quis alius similis tui in Israel? Quare ergo non custodisti dominum tuum regem? Ingressus est enim unus de turba, ut interficeret regem dominum tuum. 
\verse Non est bonum hoc, quod fecisti. Vivit Dominus quoniam filii mortis estis vos, qui non custodistis dominum vestrum, christum Domini. Nunc ergo vide, ubi sit hasta regis et ubi scyphus aquae, qui erat ad caput eius ”.
 \verse Cognovit autem Saul vocem David et dixit: “ Num vox tua haec est, fili mi David? ”. Et ait David: “ Vox mea, domine mi rex ”. 
\verse Et ait: “ Quam ob causam dominus meus persequitur servum suum? Quid feci? Aut quod est in manu mea malum? 
\verse Nunc ergo audiat, oro, dominus meus rex verba servi sui: Si Dominus incitat te adversum me, odoretur sacrificium; si autem filii hominum, maledicti sint in conspectu Domini, quia eiecerunt me hodie, ut non habitem in hereditate Domini dicentes: “Vade, servi diis alienis”. 
\verse Et nunc non effundatur sanguis meus in terra longe a facie Domini; quia egressus est rex Israel, ut quaerat pulicem unum, sicut persequitur quis perdicem in montibus ”.
 \verse Et ait Saul: “ Peccavi. Revertere, fili mi David; nequaquam enim ultra malefaciam tibi, eo quod pretiosa fuerit anima mea in oculis tuis hodie; apparet quod stulte egerim et erraverim multum nimis ”. 
\verse Et respondens David ait: “ Ecce hasta regis; transeat unus de pueris et tollat eam. 
\verse Dominus autem retribuet unicuique secundum iustitiam suam et fidem; tradidit enim te Dominus hodie in manu mea, et nolui extendere manum meam in christum Domini. 
\verse Et sicut magnificata est anima tua hodie in oculis meis, sic magnificetur anima mea in oculis Domini, et liberet me de omni angustia ”.
 \verse Ait ergo Saul ad David: “ Benedictus tu, fili mi David; et quidem faciens facies et potens poteris ”. Abiit autem David in viam suam, et Saul reversus est in locum suum.
 
\begin{biblechapter}
\verse Et ait David in corde suo: “ Aliquando incidam in uno die in manu Saul; nonne melius est ut fugiam et salver in terra Philisthinorum, ut desperet Saul cessetque me quaerere in cunctis finibus Israel? Fugiam ergo manus eius ”. 
\verse Et surrexit David et abiit ipse et sescenti viri cum eo ad Achis filium Maoch regem Geth. 
\verse Et habitavit David cum Achis in Geth ipse et viri eius unusquisque cum domo sua; David et duae uxores eius, Achinoam Iezrahelites et Abigail uxor Nabal de Carmel. 
\verse Et nuntiatum est Saul quod fugisset David in Geth, et non addidit ultra ut quaereret eum.
 \verse Dixit autem David ad Achis: “ Si inveni gratiam in oculis tuis, detur mihi locus in una urbium regionis huius, ut habitem ibi. Cur enim manet servus tuus in civitate regis tecum? ”. 
\verse Dedit itaque ei Achis in die illa Siceleg; propter quam causam facta est Siceleg regum Iudae usque in diem hanc. 
\verse Fuit autem numerus dierum, quibus habitavit David in regione Philisthinorum, annus et quattuor menses.
 \verse Et ascendit David et viri eius et agebant praedas de Gesuri et de Gerzi et de Amalecitis; hae enim gentes habitabant terram, quae est a Telem in via Sur et usque ad terram Aegypti. 
\verse Et percutiebat David omnem terram nec relinquebat viventem virum et mulierem; tollensque oves et boves et asinos et camelos et vestes revertebatur et veniebat ad Achis. 
\verse Dicebat autem ei Achis: “ In quem irruistis hodie? ”. Respondebatque David: “ Contra Nageb Iudae vel contra Nageb Ierameel vel contra Nageb Ceni ”. 
\verse Viro et mulieri non parcebat David nec adducebat in Geth dicens: “ Ne forte loquantur adversum nos: “Haec fecit David” ”. Et hoc erat decretum illi omnibus diebus, quibus habitavit in regione Philisthinorum. 
\verse Credidit ergo Achis David dicens: “ Valde odiosum se fecit populo suo Israel; erit
 igitur mihi servus sempiternus ”. 
\begin{biblechapter}
\verse Factum est autem in diebus illis, congregaverunt Philisthim agmina sua, ut praepararentur ad bellum contra Israel. Dixitque Achis ad David: “ Sciens nunc scito quoniam mecum egredieris in castris tu et viri tui ”. 
\verse Dixitque David ad Achis: “ Ideo tu quoque scies, quae facturus est servus tuus ”. Et ait Achis ad David: “ Ideo custodem capitis mei ponam te cunctis diebus ”.
 \verse Samuel autem mortuus erat; planxeratque eum omnis Israel, et sepelierant eum in Rama urbe sua. Et Saul abstulerat magos et hariolos de terra.
 \verse Congregatique sunt Philisthim et venerunt et castrametati sunt in Sunam. Congregavit autem et Saul universum Israel, et castrametati sunt in Gelboe. 
 \verse Et vidit Saul castra Philisthim et timuit, et expavit cor eius nimis. 
\verse Consuluitque Dominum, et non respondit ei neque per somnia neque per Urim neque per prophetas.
 \verse Dixitque Saul servis suis: “ Quaerite mihi mulierem habentem pythonem, et vadam ad eam et sciscitabor per illam ”. Et dixerunt servi eius ad eum: “ Est mulier habens pythonem in Endor ”. 
\verse Mutavit ergo habitum suum vestitusque est aliis vestimentis et abiit ipse et duo viri cum eo; veneruntque ad mulierem nocte, et ait: “ Divina mihi in pythone et suscita mihi, quem dixero tibi ”. 
\verse Et ait mulier ad eum: “ Ecce tu nosti, quanta fecerit Saul et quomodo eraserit magos et hariolos de terra; quare ergo insidiaris animae meae, ut occidar? ”. 
\verse Et iuravit ei Saul in Domino dicens: “ Vivit Dominus quia non veniet tibi quidquam mali propter hanc rem ”. 
\verse Dixitque ei mulier: “ Quem suscitabo tibi? ”. Qui ait: “ Samuelem suscita mihi ”.
 \verse Cum autem vidisset mulier Samuelem, exclamavit voce magna et dixit ad Saul: “ Quare imposuisti mihi? Tu es enim Saul! ”. 
\verse Dixitque ei rex: “ Noli timere. Quid vidisti? ”. Et ait mulier ad Saul: “ Hominem divinum vidi ascendentem de terra ”. 
\verse Dixitque ei: “ Qualis est forma eius? ”. Quae ait: “ Vir senex ascendit et ipse amictus est pallio ”. Intellexit Saul quod Samuel esset et inclinavit se super faciem suam in terra et adoravit.
 \verse Dixit autem Samuel ad Saul: “ Quare inquietasti me, ut suscitarer? ”. Et ait Saul: “ Coartor nimis. Siquidem Philisthim pugnant adversum me, et Deus recessit a me et exaudire me noluit neque in manu prophetarum neque per somnia; vocavi ergo te, ut ostenderes mihi quid faciam ”. 
\verse Et ait Samuel: “ Quid interrogas me, cum Dominus recesserit a te et factus est adversarius tuus? 
 \verse Fecit enim Dominus, sicut locutus est in manu mea, et scidit regnum de manu tua et dedit illud proximo tuo David, 
\verse quia non oboedisti voci Domini neque fecisti iram furoris eius in Amalec. Idcirco quod pateris, fecit tibi Dominus hodie. 
\verse Et dabit Dominus etiam Israel tecum in manu Philisthim; cras autem tu et filii tui mecum eritis, sed et castra Israel tradet Dominus in manu Philisthim ”.
 \verse Statimque Saul cecidit porrectus in terram; extimuerat enim verba Samuel, et robur non erat in eo, quia non comederat panem tota die illa et tota nocte illa. 
\verse Accessit itaque mulier ad Saul et vidit quod conturbatus esset valde; dixitque ad eum: “ Ecce audivit ancilla tua vocem tuam, et posui animam meam in manu mea et oboedivi sermonibus tuis, quos locutus es ad me. 
\verse Nunc igitur audi et tu vocem ancillae tuae, ut ponam coram te buccellam panis, et comedens convalescas, ut possis iter facere ”. 
\verse Qui renuit et ait: “ Non comedam ”. Coegerunt autem eum servi sui et mulier; et tandem, audita voce eorum, surrexit de terra et sedit super lectum. 
\verse Mulier autem illa habebat vitulum pascualem in domo; et festinavit et occidit eum, tollensque farinam miscuit eam et coxit azyma. 
\verse Et posuit ante Saul et ante servos eius. Qui cum comedissent, surrexerunt et abierunt hac eadem nocte.
 
\begin{biblechapter}
\verse Congregata sunt ergo Philisthim universa agmina in Aphec; sed et Israel castrametatus est super fontem, qui erat in Iezrahel. 
\verse Et principes quidem Philisthim incedebant in centuriis et milibus; David autem et viri eius incedebant in novissimo agmine cum Achis. 
\verse Dixeruntque principes Philisthim: “ Quid sibi volunt Hebraei isti? ”. Et ait Achis ad principes Philisthim: “ Nonne iste est David, qui fuit servus Saul regis Israel et est apud me multis diebus vel annis, et non inveni in eo quidquam ex die, qua transfugit ad me, usque ad diem hanc? ”. 
\verse Irati sunt autem adversus eum principes Philisthim et dixerunt ei: “ Revertatur vir iste et sedeat in loco suo, in quo constituisti eum, et non descendat nobiscum in proelium, ne fiat nobis adversarius, cum proeliari coeperimus. Quomodo enim aliter placare poterit dominum suum nisi in capitibus horum virorum? 
\verse Nonne iste est David, cui cantabant in choris dicentes: “Percussit Saul milia sua, et David decem milia sua”? ”.
 \verse Vocavit ergo Achis David et ait ei: “ Vivit Dominus quia rectus es tu, et bonus est in conspectu meo exitus tuus et introitus tuus mecum in castris, et non inveni in te quidquam mali ex die, qua venisti ad me, usque ad diem hanc. Sed principibus non places. 
\verse Revertere ergo et vade in pace et non offendes oculos principum Philisthim ”. 
\verse Dixitque David ad Achis: “ Quid enim feci, et quid invenisti in me servo tuo a die, qua fui in conspectu tuo, usque in diem hanc, ut non veniam et pugnem contra inimicos domini mei regis? ”. 
\verse Respondens autem Achis locutus est ad David: “ Scio quia bonus es tu in oculis meis sicut angelus Dei; sed principes Philisthim dixerunt: “Non ascendet nobiscum in proelium”. 
\verse Igitur consurge mane, tu et servi domini tui, qui venerunt tecum, et, cum de nocte surrexeritis et coeperit dilucescere, pergite ”. 
\verse Surrexit itaque de nocte David ipse et viri eius, ut proficiscerentur mane et reverterentur ad terram Philisthim. Philisthim autem ascenderunt in Iezrahel.
 
\begin{biblechapter}
\verse Cumque venissent David et viri eius in Siceleg die tertia, Amalecitae impetum fecerant contra Nageb et contra Siceleg et percusserant Siceleg et succenderant eam igni; 
\verse et captivas duxerant mulieres et omnes in ea a minimo usque ad magnum et non interfecerant quemquam, sed secum duxerant et pergebant in itinere suo. 
\verse Cum ergo venisset David et viri eius ad civitatem et invenissent eam succensam igni et uxores suas et filios suos et filias ductas esse captivas, 
\verse levaverunt David et populus, qui erat cum eo, voces suas et planxerunt, donec deficerent in eis lacrimae. 
\verse Siquidem et duae uxores David captivae ductae fuerant, Achinoam Iezrahelites et Abigail uxor Nabal de Carmel.
 \verse Et angustiatus est David valde; volebat enim eum populus lapidare, quia amara erat anima uniuscuiusque viri super filiis suis et filiabus. Confortatus est autem David in Domino Deo suo 
\verse et ait ad Abiathar sacerdotem filium Achimelech: “ Applica ad me ephod ”. Et applicuit Abiathar ephod ad David. 
\verse Et consuluit David Dominum dicens: “ Persequar latrunculos hos et comprehendam eos an non? ”. Dixitque ei: “ Persequere; absque dubio enim comprehendes eos et excuties praedam ”. 
\verse Abiit ergo David ipse et sescenti viri, qui erant cum eo, et venerunt usque ad torrentem Besor, et lassi quidam substiterunt. 
\verse Persecutus est autem David ipse et quadringenti viri; et reliqui substiterunt: ducenti, qui lassi transire non poterant torrentem Besor.
 \verse Et invenerunt virum Aegyptium in agro et adduxerunt eum ad David; dederuntque ei panem, et comedit, et dederunt ei aquam bibere, 
\verse sed et dederunt ei fragmen massae caricarum et duas ligaturas uvae passae. Quae cum comedisset, reversus est spiritus eius; non enim comederat panem neque biberat aquam tribus diebus et tribus noctibus. 
\verse Dixit itaque ei David: “ Cuius es tu vel unde? ”. Qui ait ei: “ Puer Aegyptius ego sum servus viri Amalecitae; dereliquit autem me dominus meus, quia aegrotare coepi nudiustertius. 
\verse Siquidem nos erupimus contra Nageb Cherethi et contra Nageb Iudae et Nageb Chaleb et Siceleg succendimus igni ”. 
\verse Dixitque ei David: “ Potes me ducere ad istum cuneum? ”. Qui ait: “ Iura mihi per Deum quod non occidas me et non tradas me in manu domini mei, et ducam te ad cuneum istum ”. Et iuravit ei David.
 \verse Qui cum duxisset eum, ecce illi discumbebant super faciem universae terrae comedentes et bibentes et festum celebrantes pro cuncta praeda et spoliis, quae ceperant de terra Philisthim et de terra Iudae. 
\verse Et percussit eos David die altera a diluculo usque ad vesperam, et non evasit ex eis quisquam, nisi quadringenti viri adulescentes, qui ascenderant camelos et fugerant.
 \verse Eruit ergo David omnia, quae ceperant Amalecitae, et duas uxores suas eruit. 
\verse Nec defuit quidquam a parvo usque ad magnum tam de filiis quam de filiabus et de spoliis, et, quaecumque rapuerant, omnia reduxit David. 
\verse Cepit ergo David universos greges et armenta, et minaverunt ante faciem eius possessionem hanc dixeruntque: “ Haec est praeda David ”.
 \verse Venit autem David ad ducentos viros, qui lassi substiterant nec sequi potuerant David, et residere eos iusserat in torrente Besor. Qui egressi sunt obviam David et populo, qui erat cum eo. Accedens autem David ad populum salutavit eos pacifice. 
\verse Respondensque omnis vir pessimus et iniquus de viris, qui ierant cum David, dixit: “ Quia non venerunt nobiscum, non dabimus eis quidquam de praeda, quam eruimus; sed sufficiat unicuique uxor sua et filii; quos cum acceperint, recedant ”. 
\verse Dixit autem David: “ Non sic facietis, fratres mei, de his, quae tradidit Dominus nobis, et custodivit nos et dedit latrunculos, qui eruperant adversum nos, in manu nostra; 
\verse nec audiet vos quisquam super sermone hoc; aequa enim pars erit descendentis ad proelium et remanentis ad sarcinas, et similiter divident ”. 
\verse Et factum est hoc ex die illa et deinceps constitutum ut praeceptum et quasi lex in Israel usque ad diem hanc.
 \verse Venit ergo David in Siceleg et misit dona de praeda senioribus Iudae proximis suis dicens: “ Accipite benedictionem de praeda hostium Domini ”; 
\verse his, qui erant in Bethul et qui in Ramathnageb et qui in Iether 
\verse et qui in Aroer et qui in Sephamoth et qui in Esthemo 
\verse et qui in Carmel et qui in urbibus Ierameeli et qui in urbibus Ceni 
\verse et qui in Horma et qui in Borasan et qui in Athach 
\verse et qui in Hebron et reliquis locis, in quibus commoratus fuerat David ipse et viri eius.
 
\begin{biblechapter}
\verse Philisthim autem pugnabant adversum Israel; et fugerunt viri Israel ante faciem Philisthim et ceciderunt interfecti in monte Gelboe. 
\verse Irrueruntque Philisthim in Saul et filios eius et percusserunt Ionathan et Abinadab et Melchisua filios Saul.
 \verse Totumque pondus proelii versum est in Saul; et consecuti sunt eum viri arcu, et vulneratus est vehementer a sagittariis. 
\verse Dixitque Saul ad armigerum suum: “ Evagina gladium tuum et percute me, ne forte veniant incircumcisi isti et confodiant me et illudant mihi ”. Et noluit armiger eius; erat enim nimio timore perterritus. Arripuit itaque Saul gladium et irruit super eum. 
\verse Quod cum vidisset armiger eius, videlicet quod mortuus esset Saul, irruit etiam ipse super gladium suum et mortuus est cum eo. 
\verse Mortuus est ergo Saul et tres filii eius et armiger illius et universi viri eius in die illa pariter. 
\verse Videntes autem viri Israel, qui erant trans vallem et trans Iordanem, quod fugissent viri Israelitae et quod mortuus esset Saul et filii eius, reliquerunt civitates suas et fugerunt. Veneruntque Philisthim et habitaverunt ibi.
 \verse Facta autem die altera, venerunt Philisthim, ut spoliarent interfectos, et invenerunt Saul et tres filios eius iacentes in monte Gelboe. 
\verse Et praeciderunt caput Saul et exspoliaverunt eum armis, quae miserunt in terram Philisthinorum per circuitum, ut annuntiaretur in templis idolorum suorum et populo. 
\verse Et posuerunt arma eius in templo Astharoth, corpus vero eius suspenderunt in muro Bethsan.
 \verse Quod cum audissent habitatores Iabes Galaad, quaecumque fecerant Philisthim Saul, 
\verse surrexerunt omnes viri fortissimi et ambulaverunt tota nocte et tulerunt cadaver Saul et cadavera filiorum eius de muro Bethsan; veneruntque Iabes et combusserunt ea ibi. 
\verse Et tulerunt ossa eorum et sepelierunt sub myrice in Iabes et ieiunaverunt septem diebus.
\end{biblechapter}
