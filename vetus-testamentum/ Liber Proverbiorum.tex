\biblebook{ Liber Proverbiorum}
\begin{biblechapter}
 \verse Parabolae Salomonis filii David regis Israel
 \verse ad sciendam sapientiam et disciplinam,
 ad intellegenda verba prudentiae;
 \verse ad suscipiendam eruditionem doctrinae,
 iustitiam et iudicium et aequitatem,
 \verse ut detur parvulis astutia,
 adulescenti scientia et recogitatio.
 \verse Audiat sapiens et addet doctrinam,
 et intellegens dispositiones possidebit:
 \verse animadvertet parabolam et allegoriam,
 verba sapientium et aenigmata eorum.
 \verse Timor Domini principium scientiae.
 Sapientiam atque doctrinam stulti despiciunt.
 \verse Audi, fili mi, disciplinam patris tui
 et ne reicias legem matris tuae,
 \verse quia diadema gratiae sunt capiti tuo,
 et torques collo tuo.
 \verse Fili mi, si te lactaverint peccatores,
 ne acquiescas eis.
 \verse Si dixerint: “ Veni nobiscum, insidiemur sanguini,
 abscondamus tendiculas contra insontem frustra;
 \verse deglutiamus eos sicut infernus viventes
 et integros quasi descendentes in lacum:
 \verse omnem pretiosam substantiam reperiemus,
 implebimus domos nostras spoliis;
 \verse sortem mitte nobiscum,
 marsupium unum sit omnium nostrum ”;
 \verse fili mi, ne ambules cum eis,
 prohibe pedem tuum a semitis eorum.
 \verse Pedes enim illorum ad malum currunt
 et festinant, ut effundant sanguinem.
 \verse Frustra autem iacitur rete ante oculos pinnatorum.
 \verse Ipsique contra sanguinem suum insidiantur
 et moliuntur fraudes contra animas suas.
 \verse Sic semitae omnis ad rapinam intenti:
 animam ipsius possidentis rapiunt.
 \verse Sapientia foris praedicat,
 in plateis dat vocem suam,
 \verse in capite viarum frequentium clamitat,
 in foribus portarum urbis profert verba sua:
 \verse “ Usquequo, parvuli, diligitis infantiam,
 et derisores sibi derisionem cupient,
 et imprudentes odibunt scientiam?
 \verse Convertimini ad correptionem meam;
 en proferam vobis spiritum meum
 et ostendam vobis verba mea.
 \verse Quia vocavi, et renuistis,
 extendi manum meam, et non fuit qui aspiceret;
 \verse despexistis omne consilium meum
 et increpationes meas neglexistis.
 \verse Ego quoque in interitu vestro ridebo
 et subsannabo, cum terror vobis advenerit,
 \verse cum irruerit ut procella terror,
 et interitus quasi tempestas ingruerit,
 quando venerit super vos tribulatio et angustia ”.
 \verse Tunc invocabunt me, et non exaudiam,
 instanter quaerent me et non invenient me,
 \verse eo quod exosam habuerint disciplinam
 et timorem Domini non elegerint
 \verse nec acquieverint consilio meo
 et despexerint universam correptionem meam.
 \verse Comedent igitur fructus viae suae
 suisque consiliis saturabuntur.
 \verse Aversio parvulorum interficiet eos,
 et securitas stultorum perdet illos.
 \verse Qui autem me audierit, absque terrore requiescet
 et tranquillus erit timore malorum sublato.
 
\begin{biblechapter}
 \verse Fili mi, si susceperis sermones meos
 et mandata mea absconderis penes te,
 \verse intendens ad sapientiam aurem tuam,
 inclinans cor tuum ad cognoscendam prudentiam;
 \verse si enim sapientiam invocaveris
 et dederis vocem tuam prudentiae, 
 \verse si quaesieris eam quasi pecuniam
 et sicut thesauros conquisieris illam,
 \verse tunc intelleges timorem Domini
 et scientiam Dei invenies.
 \verse Quia Dominus dat sapientiam,
 et ex ore eius scientia et prudentia. 
 \verse Thesaurizabit rectis sollertiam
 et clipeus erit gradientibus simpliciter
 \verse servans semitas iustitiae
 et vias sanctorum custodiens.
 \verse Tunc intelleges iustitiam et iudicium
 et aequitatem et omnem semitam bonam,
 \verse quia intrabit sapientia cor tuum,
 et scientia animae tuae placebit.
 \verse Consilium custodiet te,
 et prudentia servabit te,
 \verse ut eruaris a via mala
 et ab homine, qui perversa loquitur;
 \verse qui relinquunt iter rectum,
 ut ambulent per vias tenebrosas;
 \verse qui laetantur, cum malefecerint,
 et exsultant in rebus pessimis:
 \verse quorum viae perversae sunt,
 et pravi gressus eorum.
 \verse Ut eruaris a muliere aliena
 et ab extranea, quae mollit sermones suos
 \verse et relinquit ducem pubertatis suae
 et pacti Dei sui oblita est.
 \verse Inclinata est enim ad mortem domus eius,
 et ad inferos semitae ipsius;
 \verse omnes, qui ingrediuntur ad eam, non revertentur
 nec apprehendent semitas vitae.
 \verse Ut ambules in via bonorum
 et calles iustorum custodias:
 \verse qui enim recti sunt, habitabunt in terra,
 et simplices permanebunt in ea;
 \verse impii vero de terra perdentur,
 et, qui inique agunt, auferentur ex ea.
 
\begin{biblechapter}
 \verse Fili mi, ne obliviscaris legis meae,
 et praecepta mea cor tuum custodiat;
 \verse longitudinem enim dierum et annos vitae
 et pacem apponent tibi.
 \verse Misericordia et veritas te non deserant;
 circumda eas gutturi tuo
 et describe in tabulis cordis tui,
 \verse et invenies gratiam et successum bonum
 coram Deo et hominibus.
 \verse Habe fiduciam in Domino ex toto corde tuo
 et ne innitaris prudentiae tuae.
 \verse In omnibus viis tuis cogita illum,
 et ipse diriget gressus tuos.
 \verse Ne sis sapiens apud temetipsum;
 time Dominum et recede a malo.
 \verse Sanitas quippe erit umbilico tuo,
 et irrigatio ossibus tuis.
 \verse Honora Dominum de tua substantia
 et de primitiis omnium frugum tuarum,
 \verse et implebuntur horrea tua frumento,
 et vino torcularia tua redundabunt.
 \verse Disciplinam Domini, fili mi, ne abicias
 nec asperneris, cum ab eo corriperis:
 \verse quem enim diligit, Dominus corripit
 et quasi pater in filio complacet sibi.
 \verse Beatus homo, qui invenit sapientiam
 et qui affluit prudentia:
 \verse melior est acquisitio eius negotiatione argenti,
 et auro primo fructus eius.
 \verse Pretiosior est cunctis gemmis,
 et omnia pretiosa tua huic non valent comparari;
 \verse longitudo dierum in dextera eius,
 et in sinistra illius divitiae et gloria.
 \verse Viae eius viae pulchrae,
 et omnes semitae illius pacificae.
 \verse Lignum vitae est his, qui apprehenderint eam;
 et, qui tenuerit eam, beatus.
 \verse Dominus sapientia fundavit terram,
 stabilivit caelos prudentia;
 \verse sapientia illius eruperunt abyssi,
 et nubes rorem stillant.
 \verse Fili mi, ne effluant haec ab oculis tuis;
 custodi prudentiam atque consilium,
 \verse et erit vita animae tuae,
 et gratia collo tuo;
 \verse tunc ambulabis fiducialiter in via tua,
 et pes tuus non impinget.
 \verse Si dormieris, non timebis;
 quiesces, et suavis erit somnus tuus.
 \verse Ne paveas repentino terrore
 et irruentem tibi turbinem impiorum, cum venerit.
 \verse Dominus enim erit in latere tuo
 et custodiet pedem tuum, ne capiaris.
 \verse Noli prohibere beneficium ab eo, cui debetur,
 si in potestate manus tuae est, ut facias.
 \verse Ne dicas amico tuo: “ Vade et revertere,
 cras dabo tibi ”, cum statim possis dare.
 \verse Ne moliaris amico tuo malum,
 cum ille apud te sedeat cum fiducia.
 \verse Ne contendas adversus hominem frustra,
 cum ipse tibi nihil mali fecerit.
 \verse Ne aemuleris hominem iniustum
 nec imiteris omnes vias eius,
 \verse quia abominatio Domini est omnis pravus,
 et cum simplicibus societas eius.
 \verse Maledictio a Domino in domo impii,
 habitacula autem iustorum benedicentur.
 \verse Ipse deludet illusores
 et mansuetis dabit gratiam;
 \verse gloriam sapientes possidebunt,
 stultorum exaltatio ignominia.
 
\begin{biblechapter}
 \verse Audite, filii, disciplinam patris
 et attendite, ut sciatis prudentiam;
 \verse quoniam doctrinam bonam tribuam vobis,
 legem meam ne derelinquatis.
 \verse Nam et ego filius fui patris mei,
 tenellus et unigenitus coram matre mea;
 \verse et docebat me atque dicebat:
 “ Suscipiat verba mea cor tuum,
 custodi praecepta mea et vives.
 \verse Posside sapientiam, posside prudentiam,
 ne obliviscaris neque declines a verbis oris mei.
 \verse Ne dimittas eam, et custodiet te,
 dilige eam, et servabit te.
 \verse Principium sapientiae: posside sapientiam
 et in omni possessione tua acquire prudentiam.
 \verse Arripe illam, et exaltabit te,
 glorificaberis ab ea, cum eam fueris amplexatus.
 \verse Dabit capiti tuo diadema gratiae,
 et corona inclita proteget te ”.
 \verse Audi, fili mi, et suscipe verba mea,
 ut multiplicentur tibi anni vitae.
 \verse Viam sapientiae monstravi tibi;
 duxi te per semitas aequitatis,
 \verse quas cum ingressus fueris, non arctabuntur gressus tui,
 et currens non habebis offendiculum.
 \verse Tene disciplinam nec laxes;
 custodi illam, quia ipsa est vita tua.
 \verse Ne ingrediaris in semitas impiorum
 nec procedas in malorum via.
 \verse Fuge ab ea nec transeas per illam;
 declina et desere eam.
 \verse Non enim dormiunt, nisi malefecerint,
 et rapitur somnus ab eis, nisi supplantaverint.
 \verse Comedunt enim panem impietatis
 et vinum iniquitatis bibunt.
 \verse Iustorum autem semita quasi lux splendens
 procedit et crescit usque ad perfectam diem.
 \verse Via impiorum tenebrosa;
 nesciunt, ubi corruant.
 \verse Fili mi, ausculta sermones meos
 et ad eloquia mea inclina aurem tuam;
 \verse ne recedant ab oculis tuis,
 custodi ea in medio cordis tui:
 \verse vita enim sunt invenientibus ea,
 et universae carni sanitas.
 \verse Omni custodia serva cor tuum,
 quia ex ipso vita procedit.
 \verse Remove a te os pravum,
 et detrahentia labia sint procul a te.
 \verse Oculi tui recta videant,
 et palpebrae tuae dirigantur coram te.
 \verse Observa semitam pedum tuorum,
 et omnes viae tuae stabilientur.
 \verse Ne declines ad dexteram neque ad sinistram,
 averte pedem tuum a malo.
 
\begin{biblechapter}
 \verse Fili mi, attende ad sapientiam meam,
 et prudentiae meae inclina aurem tuam,
 \verse ut custodias cogitationes,
 et disciplinam labia tua conservent.
 \verse Favum enim stillant labia meretricis,
 et nitidius oleo guttur eius;
 \verse novissima autem illius amara quasi absinthium
 et acuta quasi gladius biceps.
 \verse Pedes eius descendunt in mortem,
 et ad inferos gressus illius tendunt;
 \verse cum non observet semitam vitae,
 vagi sunt gressus eius, et ipsa nescit.
 \verse Nunc ergo, fili mi, audi me
 et ne recedas a verbis oris mei.
 \verse Longe fac ab ea viam tuam
 et ne appropinques foribus domus eius.
 \verse Ne des alienis honorem tuum
 et annos tuos crudeli,
 \verse ne forte impleantur extranei viribus tuis,
 et labores tui sint in domo aliena,
 \verse et gemas in novissimis,
 quando consumpseris carnes tuas et corpus tuum
 \verse et dicas: “ Cur detestatus sum disciplinam,
 et increpationes renuit cor meum,
 \verse nec audivi vocem docentium me
 et magistris non inclinavi aurem meam?
 \verse Paene fui in omni malo,
 in medio ecclesiae et synagogae ”.
 \verse Bibe aquam de cisterna tua
 et fluenta putei tui,
 \verse ne deriventur fontes tui foras,
 et in plateis rivi aquarum;
 \verse habeto eas solus,
 nec sint alieni participes tui.
 \verse Sit vena tua benedicta,
 et laetare cum muliere adulescentiae tuae;
 \verse cerva carissima et gratissimus hinnulus,
 blanditiae eius inebrient te in omni tempore,
 in amore eius delectare iugiter.
 \verse Quare seduceris, fili mi, ab aliena
 et foveris in sinu extraneae?
 \verse Quoniam ante Dominum viae hominis,
 et omnes gressus eius considerat.
 \verse Iniquitates suae capient impium,
 et funibus peccatorum suorum constringetur.
 \verse Ipse morietur, quia non habuit disciplinam,
 et in multitudine stultitiae suae decipietur.
 
\begin{biblechapter}
 \verse Fili mi, si spoponderis pro amico tuo,
 defixisti apud extraneum manum tuam;
 \verse illaqueatus es verbis oris tui
 et captus propriis sermonibus.
 \verse Fac ergo, quod dico, fili mi, et temetipsum libera,
 quia incidisti in manum proximi tui;
 discurre, prosternere, insta amico tuo.
 \verse Ne dederis somnum oculis tuis
 nec palpebris tuis dormitationem.
 \verse Eruere quasi dammula de rete,
 et quasi avis de manu aucupis.
 \verse Vade ad formicam, o piger,
 et considera vias eius et disce sapientiam.
 \verse Quae, cum non habeat ducem
 nec praeceptorem nec principem,
 \verse parat in aestate cibum sibi
 et congregat in messe, quod comedat.
 \verse Usquequo, piger, dormies?
 Quando consurges e somno tuo?
 \verse Paululum dormis, paululum dormitas,
 paululum conseres manus, ut dormias;
 \verse et veniet tibi quasi viator egestas,
 et pauperies quasi vir armatus.
 \verse Homo iniquus, vir inutilis,
 graditur ore perverso;
 \verse annuit oculis, terit pede,
 digito loquitur.
 \verse Prava in corde suo machinatur,
 malum in omni tempore, iurgia seminat.
 \verse Ideo extemplo veniet perditio sua,
 et subito conteretur nec habebit medicinam.
 \verse Sex sunt, quae odit Dominus,
 et septem detestatur anima eius:
 \verse oculos sublimes, linguam mendacem,
 manus effundentes innoxium sanguinem,
 \verse cor machinans cogitationes pravas,
 pedes veloces ad currendum in malum,
 \verse proferentem mendacia, testem fallacem
 et eum, qui seminat inter fratres discordias.
 \verse Conserva, fili mi, praecepta patris tui
 et ne reicias legem matris tuae;
 \verse liga ea in corde tuo iugiter
 et circumda gutturi tuo.
 \verse Cum ambulaveris, dirigent te,
 cum dormieris, custodient te
 et, cum evigilaveris, colloquentur tecum.
 \verse Quia mandatum lucerna est, et lex lux,
 et via vitae increpatio disciplinae,
 \verse ut custodiant te a muliere mala
 et a blanda lingua extraneae;
 \verse non concupiscat pulchritudinem eius cor tuum,
 nec capiaris nutibus illius:
 \verse pretium enim scorti vix est torta panis,
 mulier autem viri pretiosam animam capit.
 \verse Numquid potest homo abscondere ignem in sinu suo,
 et vestimenta illius non ardebunt?
 \verse Aut ambulare super prunas,
 et non comburentur plantae eius?
 \verse Sic qui ingreditur ad mulierem proximi sui;
 non erit mundus, quicumque tetigerit eam.
 \verse Non contemptui erit fur, cum furatus fuerit,
 ut esurientem impleat animam.
 \verse Deprehensus quoque reddet septuplum
 et omnem substantiam domus suae tradet.
 \verse Qui autem adulter est cum muliere, vecors est;
 perdet animam suam, qui hoc fecerit.
 \verse Plagam et ignominiam congregat sibi,
 et opprobrium illius non delebitur.
 \verse Quia zelus est furor viri,
 et non parcet in die vindictae
 \verse nec accipiet personam tuam in piaculum
 nec suscipiet dona plurima.
 
\begin{biblechapter}
 \verse Fili mi, custodi sermones meos
 et praecepta mea reconde tibi.
 \verse Serva mandata mea et vives,
 et legem meam quasi pupillam oculi tui.
 \verse Liga ea in digitis tuis,
 scribe illa in tabulis cordis tui.
 \verse Dic sapientiae: “ Soror mea es ”
 et prudentiam voca Amicam,
 \verse ut custodiat te a muliere extranea
 et ab aliena, quae verba sua dulcia facit.
 \verse De fenestra enim domus meae
 per cancellos prospexi
 \verse et video inter parvulos;
 considero inter filios vecordem iuvenem,
 \verse qui transit per plateam iuxta angulum
 et prope viam domus illius graditur
 \verse in obscuro advesperascente die,
 in mediis tenebris et caligine.
 \verse Et ecce, occurrit illi mulier ornatu meretricio,
 cauta corde, garrula et rebellans,
 \verse quietis impatiens
 nec valens in domo consistere pedibus suis:
 \verse nunc foris, nunc in plateis
 et iuxta angulos insidians.
 \verse Apprehensumque deosculatur iuvenem
 et procaci vultu blanditur dicens:
 \verse “ Victimas pro salute vovi,
 hodie reddidi vota mea;
 \verse idcirco egressa sum in occursum tuum
 desiderans te videre et repperi.
 \verse Stragulatis vestibus lectulum meum stravi,
 linteis pictis ex Aegypto;
 \verse aspersi cubile meum myrrha
 et aloe et cinnamomo.
 \verse Veni, inebriemur voluptatibus,
 usque mane fruamur amoribus.
 \verse Non est enim vir in domo sua;
 abiit via longissima,
 \verse sacculum pecuniae secum tulit,
 in die plenae lunae reversurus est in domum suam ”.
 \verse Irretivit eum multis sermonibus
 et blanditiis labiorum protraxit illum.
 \verse Stultus eam sequitur quasi bos ductus ad victimam,
 sicut irretitur vinculo cervus,
 \verse donec transfigat sagitta iecur eius;
 velut si avis festinet ad laqueum
 et nescit quod de periculo animae illius agitur.
 \verse Nunc ergo, fili mi, audi me
 et attende verbis oris mei.
 \verse Ne abstrahatur in viis illius mens tua,
 neque decipiaris semitis eius.
 \verse Multos enim vulneratos deiecit,
 et fortissimi quique interfecti sunt ab ea:
 \verse viae inferi domus eius
 penetrantes in interiora mortis.
 
\begin{biblechapter}
 \verse Numquid non sapientia clamitat,
 et prudentia dat vocem suam?
 \verse In summis verticibus
 supra viam in mediis semitis stans,
 \verse iuxta portas ad introitum civitatis,
 in ipsis foribus conclamat:
 \verse “ O viri, ad vos clamito,
 et vox mea ad filios hominum.
 \verse Intellegite, parvuli, astutiam;
 et insipientes, animadvertite.
 \verse Audite, quoniam de rebus magnis locutura sum,
 et aperientur labia mea, ut recta praedicent.
 \verse Veritatem meditabitur guttur meum,
 et labia mea detestabuntur impium.
 \verse Iusti sunt omnes sermones oris mei,
 non est in eis pravum quid neque perversum;
 \verse omnes recti sunt intellegentibus
 et aequi invenientibus scientiam.
 \verse Accipitc disciplinam meam et non pecuniam,
 doctrinam magis quam aurum electum.
 \verse Melior est enim sapientia gemmis,
 et omne desiderabile ei non potest comparari ”.
 \verse Ego sapientia habito cum prudentia
 et artem excogitandi invenio.
 \verse Timor Domini odisse malum;
 arrogantiam et superbiam et viam pravam
 et os bilingue detestor.
 \verse Meum est consilium et prudentia,
 mea est intellegentia, mea est fortitudo.
 \verse Per me reges regnant,
 et principes iusta decernunt;
 \verse per me duces imperant,
 et potentes decernunt iustitiam.
 \verse Ego diligentes me diligo;
 et, qui mane vigilant ad me, invenient me.
 \verse Mecum sunt divitiae et gloria,
 opes superbae et iustitia.
 \verse Melior est enim fructus meus auro et obryzo,
 et genimina mea argento electo.
 \verse In viis iustitiae ambulo,
 in medio semitarum iudicii,
 \verse ut ditem diligentes me
 et thesauros eorum repleam.
 \verse Dominus possedit me in initio viarum suarum,
 antequam quidquam faceret a principio;
 \verse ab aeterno ordinata sum
 et ex antiquis, antequam terra fieret.
 \verse Nondum erant abyssi, et ego iam concepta eram,
 necdum fontes graves aquis,
 \verse priusquam montes demergerentur,
 ante colles ego parturiebar.
 \verse Adhuc terram non fecerat et campos
 et initium glebae orbis terrae.
 \verse Quando praeparabat caelos, aderam,
 quando certa lege et gyro vallabat abyssos,
 \verse quando nubes firmabat sursum,
 et praevaluerunt fontes abyssi,
 \verse quando circumdabat mari terminum suum
 et aquis, ne transirent fines suos,
 quando iecit fundamenta terrae,
 \verse cum eo eram ut artifex:
 delectatio eius per singulos dies,
 ludens coram eo omni tempore,
 \verse ludens in orbe terrarum,
 et deliciae meae esse cum filiis hominum.
 \verse Nunc ergo, filii, audite me:
 beati, qui custodiunt vias meas;
 \verse audite disciplinam et estote sapientes
 et nolite abicere eam.
 \verse Beatus homo, qui audit me
 et qui vigilat ad fores meas cotidie
 et observat ad postes ostii mei.
 \verse Qui me invenerit, inveniet vitam
 et hauriet delicias a Domino.
 \verse Qui autem in me peccaverit, laedet animam suam:
 omnes, qui me oderunt, diligunt mortem.
 
\begin{biblechapter}
 \verse Sapientia aedificavit sibi domum,
 excidit columnas septem;
 \verse immolavit victimas suas, miscuit vinum
 et proposuit mensam suam.
 \verse Misit ancillas suas, ut vocarent
 ad arcem et ad excelsa civitatis:
 \verse “ Si quis est parvulus, veniat ad me ”.
 Et vecordi locuta est:
 \verse “ Venite, comedite panem meum
 et bibite vinum, quod miscui vobis; 
 \verse relinquite infantiam et vivite
 et ambulate per vias prudentiae ”.
 \verse Qui erudit derisorem, ipse iniuriam sibi facit;
 et, qui arguit impium, sibi maculam generat.
 \verse Noli arguere derisorem, ne oderit te;
 argue sapientem, et diliget te.
 \verse Da sapienti, et sapientior fiet;
 doce iustum, et addet doctrinam.
 \verse Principium sapientiae timor Domini,
 et scientia Sancti est prudentia.
 \verse Per me enim multiplicabuntur dies tui,
 et addentur tibi anni vitae.
 \verse Si sapiens fueris, tibimetipsi eris;
 si autem illusor, solus portabis malum.
 \verse Mulier stulta est clamosa,
 fatua et nihil sciens;
 \verse sedit in foribus domus suae
 super sellam in excelsis urbis,
 \verse ut vocaret transeuntes per viam
 et pergentes itinere suo:
 \verse “ Qui est parvulus, declinet ad me ”.
 Et vecordi locuta est:
 \verse “ Aquae furtivae dulciores sunt,
 et panis in abscondito suavior ”.
 \verse Et ignoravit quod ibi sint umbrae,
 et in profundis inferni convivae eius.
 
\begin{biblechapter}
 \verse Parabolae Salomonis.
 Filius sapiens laetificat patrem,
 filius vero stultus maestitia est matris suae.
 \verse Nil proderunt thesauri impietatis,
 iustitia vero liberabit a morte.
 \verse Non affliget Dominus fame animam iusti
 et cupiditatem impiorum subvertet.
 \verse Egestatem operata est manus remissa,
 manus autem fortium divitias parat.
 \verse Qui congregat in messe, filius sapiens est;
 qui autem stertit aestate, filius confusionis.
 \verse Benedictiones Domini super caput iusti,
 os autem impiorum operit violentiam.
 \verse Memoria iusti in benedictione erit,
 et nomen impiorum putrescet.
 \verse Sapiens corde praecepta suscipit,
 et stultus labiis corruet.
 \verse Qui ambulat simpliciter, ambulat confidenter;
 qui autem depravat vias suas, manifestus erit.
 \verse Qui annuit oculo, dabit dolorem,
 et stultus labiis corruet.
 \verse Vena vitae os iusti,
 et os impiorum operit violentiam.
 \verse Odium suscitat rixas,
 et universa delicta operit caritas.
 \verse In labiis sapientis invenitur sapientia,
 et virga in dorso eius, qui indiget corde.
 \verse Sapientes recondunt scientiam,
 os autem stulti ruinae proximum est.
 \verse Substantia divitis urbs fortitudinis eius,
 ruina pauperum egestas eorum.
 \verse Opus iusti ad vitam,
 fructus autem impii ad peccatum.
 \verse Graditur ad vitam, qui custodit disciplinam;
 qui autem increpationes relinquit, errat.
 \verse Abscondunt odium labia mendacia;
 qui profert contumeliam, insipiens est.
 \verse In multiloquio non deerit peccatum;
 qui autem moderatur labia sua, prudentissimus est.
 \verse Argentum electum lingua iusti,
 cor autem impiorum pro nihilo.
 \verse Labia iusti erudiunt plurimos;
 qui autem indocti sunt, in cordis egestate morientur.
 \verse Benedictio Domini divites facit,
 nec addet ei labor quidquam.
 \verse Quasi per risum stultus operatur scelus,
 sapientia autem est viro prudentiae.
 \verse Quod timet impius, veniet super eum;
 desiderium suum iustis dabitur.
 \verse Quasi tempestas transiens non erit impius,
 iustus autem quasi fundamentum sempiternum.
 \verse Sicut acetum dentibus et fumus oculis,
 sic piger his, qui miserunt eum.
 \verse Timor Domini apponet dies,
 et anni impiorum breviabuntur.
 \verse Exspectatio iustorum laetitia,
 spes autem impiorum peribit.
 \verse Fortitudo simplici via Domini
 et ruina his, qui operantur malum.
 \verse Iustus in aeternum non commovebitur,
 impii autem non habitabunt super terram.
 \verse Os iusti germinabit sapientiam,
 lingua prava abscindetur.
 \verse Labia iusti considerant placita,
 et os impiorum perversa.
 
\begin{biblechapter}
 \verse Statera dolosa abominatio est apud Dominum,
 et pondus aequum voluntas eius.
 \verse Venit superbia, veniet et contumelia;
 apud humiles autem sapientia.
 \verse Simplicitas iustorum diriget eos,
 et supplantatio perversorum vastabit illos.
 \verse Non proderunt divitiae in die ultionis,
 iustitia autem liberabit a morte.
 \verse Iustitia simplicis diriget viam eius,
 et in impietate sua corruet impius.
 \verse Iustitia rectorum liberabit eos,
 et in insidiis suis capientur iniqui.
 \verse Mortuo homine impio, nulla erit ultra spes;
 et exspectatio divitiarum peribit.
 \verse Iustus de angustia liberatus est,
 et tradetur impius pro eo.
 \verse Simulator ore decipit amicum suum,
 iusti autem liberabuntur scientia.
 \verse In bonis iustorum exsultabit civitas,
 et in perditione impiorum erit laudatio.
 \verse Benedictione iustorum exaltabitur civitas
 et ore impiorum subvertetur.
 \verse Qui despicit amicum suum, indigens corde est,
 vir autem prudens tacebit.
 \verse Qui ambulat susurrans, revelat arcana;
 qui autem fidelis est animi, celat commissum.
 \verse Ubi non adsunt dispositiones, populus corruet;
 salus autem, ubi multa consilia.
 \verse Affligetur malo, qui fidem facit pro extraneo;
 qui autem odit sponsores, securus erit.
 \verse Mulier gratiosa inveniet gloriam,
 et robusti habebunt divitias.
 \verse Benefacit animae suae vir misericors;
 qui autem crudelis est, carnem suam affligit.
 \verse Impius facit opus fallax,
 seminanti autem iustitiam merces fidelis.
 \verse Firmus in iustitia praeparat vitam,
 et sectator malorum mortem.
 \verse Abominabile Domino cor pravum,
 et voluntas eius in iis, qui simpliciter ambulant.
 \verse Manus in manu, non erit impunitus malus,
 semen autem iustorum salvabitur.
 \verse Circulus aureus in naribus suis
 mulier pulchra et fatua.
 \verse Desiderium iustorum omne bonum est,
 praestolatio impiorum furor.
 \verse Alii dividunt propria et ditiores fiunt,
 alii parciores iusto semper in egestate sunt.
 \verse Anima, quae benedicit, impinguabitur;
 et, qui inebriat, ipse quoque inebriatur.
 \verse Qui abscondit frumenta, maledicetur in populis,
 benedictio autem super caput vendentium.
 \verse Qui instanter quaerit bonum, quaerit beneplacitum;
 qui autem investigator malorum est, haec advenient ei.
 \verse Qui confidit in divitiis suis, corruet,
 iusti autem quasi virens folium germinabunt.
 \verse Qui conturbat domum suam, possidebit ventos;
 et, qui stultus est, serviet sapienti.
 \verse Fructus iusti lignum vitae;
 et suscipit animas, qui sapiens est.
 \verse Si iustus in terra rependitur,
 quanto magis impius et peccator.
 
\begin{biblechapter}
 \verse Qui diligit disciplinam, diligit scientiam;
 qui autem odit increpationes, insipiens est.
 \verse Qui bonus est, hauriet gratiam a Domino,
 virum autem versutum ipse condemnabit.
 \verse Non roborabitur homo ex impietate,
 et radix iustorum non commovebitur.
 \verse Mulier diligens corona est viro suo,
 et quasi putredo in ossibus eius, quae est inhonesta.
 \verse Cogitationes iustorum iudicia,
 et consilia impiorum fraudulentia.
 \verse Verba impiorum insidiantur sanguini,
 os iustorum liberabit eos.
 \verse Subvertuntur impii et iam non sunt,
 domus autem iustorum permanebit.
 \verse Ad doctrinam suam laudabitur vir;
 qui autem perversus corde est, patebit contemptui.
 \verse Melior est pauper, qui ministrat sibi,
 quam gloriosus et indigens pane.
 \verse Curat iustus iumentorum suorum animas,
 viscera autem impiorum crudelia.
 \verse Qui operatur terram suam, satiabitur panibus;
 qui autem sectatur vana, vecors est.
 \verse Desiderat impius laqueum pessimorum,
 radix autem iustorum proficiet.
 \verse Propter peccata labiorum irretitur malus,
 effugiet autem iustus de angustia.
 \verse De fructu oris sui unusquisque replebitur bonis,
 et iuxta opera manuum suarum retribuetur ei.
 \verse Via stulti recta in oculis eius;
 qui autem sapiens est, audit consilia.
 \verse Fatuus statim indicat iram suam,
 dissimulat autem iniuriam callidus.
 \verse Qui spirat veritatem, index iustitiae est,
 testis autem mendax, fraudulentiae.
 \verse Est qui temere loquitur et quasi gladio pungit,
 lingua autem sapientium sanitas est.
 \verse Labium veritatis firmum erit in perpetuum,
 ad momentum autem lingua mendacii.
 \verse Dolus in corde cogitantium mala;
 qui autem pacis ineunt consilia, sequitur eos gaudium.
 \verse Nulla calamitas obveniet iusto,
 impii autem replebuntur malo.
 \verse Abominatio est Domino labia mendacia,
 qui autem fideliter agunt, placent ei.
 \verse Homo versutus celat scientiam,
 et cor insipientium provocat stultitiam.
 \verse Manus fortium dominabitur,
 quae autem remissa est, tributis serviet.
 \verse Maeror in corde viri humiliabit illum,
 et sermo bonus laetificabit eum.
 \verse In rectum ducit amicum iustus,
 iter autem impiorum decipiet eos.
 \verse Non assabit ignavia praedam suam,
 sed substantia pretiosa erit viro industrio.
 \verse In semita iustitiae vita,
 est autem etiam iter apertum ad mortem.
 
\begin{biblechapter}
 \verse Filius sapiens disciplina patris;
 qui autem illusor est, non audit, cum arguitur.
 \verse De fructu oris sui homo satiabitur bonis,
 anima autem praevaricatorum violentia.
 \verse Qui custodit os suum, custodit animam suam;
 qui autem incautus est eloquio, ruina est ei.
 \verse Vult et non habet piger,
 anima autem operantium impinguabitur.
 \verse Verbum mendax iustus detestabitur,
 impius autem confundit et dehonestat.
 \verse Iustitia custodit innocentem in via,
 impietas autem peccatorem supplantat.
 \verse Est qui quasi dives habetur, cum nihil habeat;
 et est qui quasi pauper, cum in multis divitiis sit.
 \verse Redemptio animae viri divitiae suae;
 qui autem pauper est, increpationem non sustinet.
 \verse Lux iustorum laetificat,
 lucerna autem impiorum exstinguetur.
 \verse Inter superbos tantum iurgia sunt,
 et apud humiles sapientia.
 \verse Substantia festinata minuetur;
 qui autem colligit manu, multiplicat.
 \verse Spes, quae differtur, affligit animam,
 lignum vitae desiderium veniens.
 \verse Qui contemnit verbum, ipse se obligat;
 qui autem timet praeceptum, retribuetur ei.
 \verse Lex sapientis fons vitae,
 ut declinet a laqueis mortis.
 \verse Intellegentia bona dabit gratiam,
 in itinere infidelium vorago.
 \verse Omnis astutus agit cum consilio;
 qui autem fatuus est, aperit stultitiam.
 \verse Nuntius impius cadet in malum,
 legatus autem fidelis sanitas.
 \verse Egestas et ignominia ei, qui deserit disciplinam;
 qui autem acquiescit arguenti, glorificabitur.
 \verse Desiderium, si compleatur, delectat animam;
 detestantur stulti fugere mala.
 \verse Qui cum sapientibus graditur, sapiens erit;
 amicus stultorum malus efficietur.
 \verse Peccatores persequitur malum,
 et iustis retribuentur bona.
 \verse Bonus relinquit heredes filios et nepotes;
 et custoditur iusto substantia peccatoris.
 \verse Multi cibi in novalibus pauperum,
 et est qui perit, deficiente iudicio.
 \verse Qui parcit virgae, odit filium suum;
 qui autem diligit illum, instanter erudit.
 \verse Iustus comedit et replet animam suam,
 venter autem impiorum insaturabilis.
 
\begin{biblechapter}
 \verse Sapientia mulierum aedificat domum suam,
 insipientia eam manibus destruet.
 \verse Ambulans recto itinere timet Deum;
 despicit illum, qui infami graditur via.
 \verse In ore stulti virga superbiae,
 labia autem sapientium custodiunt eos.
 \verse Ubi non sunt boves, praesepe vacuum est;
 plurimae autem segetes in fortitudine bovis.
 \verse Testis fidelis non mentitur,
 profert autem mendacium dolosus testis.
 \verse Quaerit derisor sapientiam et non invenit;
 doctrina prudentibus facilis.
 \verse Cede coram viro stulto,
 quia nescies labia prudentiae.
 \verse Sapientia callidi est intellegere viam suam,
 et imprudentia stultorum errans.
 \verse Stulti parvipendent peccatum,
 et inter iustos morabitur gratia.
 \verse Cor novit amaritudinem animae suae,
 in gaudio eius non miscebitur extraneus.
 \verse Domus impiorum delebitur,
 tabernacula vero iustorum germinabunt.
 \verse Est via, quae videtur homini recta,
 novissima autem eius deducunt ad mortem.
 \verse Etiam in risu cor dolore miscebitur,
 et extrema gaudii luctus occupat.
 \verse Viis suis replebitur stultus,
 et super eum erit vir bonus.
 \verse Simplex credit omni verbo,
 astutus considerat gressus suos.
 \verse Sapiens timet et declinat a malo,
 stultus transilit et confidit.
 \verse Impatiens operabitur stultitiam,
 et vir versutus odiosus est.
 \verse Possidebunt simplices stultitiam,
 et astuti coronabuntur scientia.
 \verse Procumbunt mali ante bonos,
 et impii ante portas iustorum.
 \verse Etiam proximo suo pauper odiosus erit,
 amici vero divitum multi.
 \verse Qui despicit proximum suum, peccat;
 qui autem miseretur pauperis, beatus erit.
 \verse Nonne errant, qui operantur malum?
 Misericordia et veritas iis, qui praeparant bona.
 \verse In omni labore erit abundantia;
 verbum autem labiorum tendit tantummodo ad egestatem.
 \verse Corona sapientium divitiae eorum,
 fatuitas stultorum fatuitas est.
 \verse Liberat animas testis fidelis,
 et profert mendacia versipellis.
 \verse In timore Domini fiducia fortis,
 et filiis eius erit spes.
 \verse Timor Domini fons vitae,
 declinans a laqueis mortis.
 \verse In multitudine populi dignitas regis,
 et in paucitate plebis ruina principis.
 \verse Qui patiens est, multa gubernatur prudentia;
 qui autem impatiens est, exaltat stultitiam.
 \verse Vita carnium sanitas cordis,
 putredo ossium invidia.
 \verse Qui calumniatur egentem, exprobrat Factori eius;
 honorat autem eum, qui miseretur pauperis.
 \verse In malitia sua impelletur impius,
 sperat autem iustus in integritate sua.
 \verse In corde prudentis requiescit sapientia,
 at in medio stultorum agnoscetur?
 \verse Iustitia elevat gentem,
 vituperium autem populorum est peccatum.
 \verse Acceptus est regi minister intellegens,
 et iracundia ei, qui turpiter agit.
 
\begin{biblechapter}
 \verse Responsio mollis frangit iram,
 sermo durus suscitat furorem.
 \verse Lingua sapientium stillat scientiam,
 os fatuorum ebullit stultitiam.
 \verse In omni loco oculi Domini
 contemplantur malos et bonos.
 \verse Lingua placabilis lignum vitae,
 sed obliquitas in ea conteret spiritum.
 \verse Stultus irridet disciplinam patris sui;
 qui autem custodit increpationes, astutior fiet.
 \verse In domo iusti divitiae plurimae,
 et in fructibus impii conturbatio.
 \verse Labia sapientium disseminabunt scientiam;
 cor stultorum non rectum erit.
 \verse Victimae impiorum abominabiles Domino;
 vota iustorum grata sunt ei.
 \verse Abominatio est Domino via impii;
 qui sequitur iustitiam, diligetur.
 \verse Admonitio mala deserenti viam;
 qui increpationes odit, morietur.
 \verse Infernus et Perditio coram Domino,
 quanto magis corda filiorum hominum!
 \verse Non amat derisor eum, qui se corripit,
 nec ad sapientes graditur.
 \verse Cor gaudens exhilarat faciem,
 in maerore animi deicitur spiritus.
 \verse Cor sapientis quaerit doctrinam,
 et os stultorum pascitur stultitia.
 \verse Omnes dies pauperis mali;
 hilaris autem corde quasi iuge convivium.
 \verse Melius est parum cum timore Domini
 quam thesauri magni cum sollicitudine.
 \verse Melius est demensum holerum cum caritate
 quam vitulus saginatus cum odio.
 \verse Vir iracundus provocat rixas;
 qui patiens est, mitigat lites.
 \verse Iter pigrorum quasi saepes spinarum,
 via sollertium complanata.
 \verse Filius sapiens laetificat patrem,
 et stultus homo despicit matrem suam.
 \verse Stultitia gaudium sensu carenti;
 et vir prudens dirigit gressus suos.
 \verse Dissipantur cogitationes, ubi non est consilium;
 ubi vero sunt plures consiliarii, confirmantur.
 \verse Laetatur homo in responsione oris sui,
 et sermo opportunus est optimus.
 \verse Semita vitae sursum est viro erudito,
 ut declinet de inferno deorsum.
 \verse Domum superborum demolietur Dominus
 et firmos faciet terminos viduae.
 \verse Abominatio Domini cogitationes malae,
 et purus sermo pulcherrimus.
 \verse Conturbat domum suam, qui sectatur avaritiam;
 qui autem odit munera, vivet.
 \verse Mens iusti meditatur, ut respondeat;
 os impiorum redundat malis.
 \verse Longe est Dominus ab impiis
 et orationes iustorum exaudiet.
 \verse Lux oculorum laetificat animam,
 fama bona impinguat ossa.
 \verse Auris, quae audit increpationes vitae,
 in medio sapientium commorabitur.
 \verse Qui abicit disciplinam, despicit animam suam;
 qui autem acquiescit increpationibus, possessor est cordis.
 \verse Timor Domini disciplina sapientiae,
 et gloriam praecedit humilitas.
 
\begin{biblechapter}
 \verse Hominis est animum praeparare,
 et Domini est responsio linguae.
 \verse Omnes viae hominis purae sunt oculis eius,
 spirituum ponderator est Dominus.
 \verse Revela Domino opera tua,
 et dirigentur cogitationes tuae.
 \verse Universa secundum proprium finem operatus est Dominus;
 impium quoque ad diem malum.
 \verse Abominatio Domini est omnis arrogans;
 manus in manu, non erit innocens.
 \verse Misericordia et veritate redimitur iniquitas,
 et in timore Domini declinatur a malo.
 \verse Cum placuerint Domino viae hominis,
 inimicos quoque eius convertet ad pacem.
 \verse Melius est parum cum iustitia
 quam multi fructus sine aequitate.
 \verse Cor hominis disponit viam suam,
 sed Domini est dirigere gressus eius.
 \verse Divinatio in labiis regis,
 in iudicio non errabit os eius.
 \verse Pondus et statera iusta Domini sunt,
 et opera eius omnes lapides sacculi.
 \verse Abominantur reges agere impie,
 quoniam iustitia firmatur solium.
 \verse Voluntas regum labia iusta;
 qui recta loquitur, diligetur.
 \verse Indignatio regis nuntii mortis,
 et vir sapiens placabit eam.
 \verse In lumine vultus regis vita,
 et voluntas eius quasi imber serotinus.
 \verse Possidere sapientiam quanto melius est auro;
 et acquirere prudentiam pretiosius est argento.
 \verse Semita iustorum declinare a malo;
 custos animae suae, qui servat viam suam.
 \verse Contritionem praecedit superbia,
 et ante ruinam exaltatio spiritus.
 \verse Melius est humiliari cum mitibus
 quam dividere spolia cum superbis.
 \verse Eruditus in verbo reperiet bona;
 et, qui sperat in Domino, beatus est.
 \verse Qui sapiens est corde, appellabitur prudens;
 et dulcedo labiorum addet doctrinam.
 \verse Fons vitae eruditio possidentis;
 poena stultorum stultitia.
 \verse Cor sapientis erudiet os eius
 et labiis eius addet doctrinam.
 \verse Favus mellis composita verba,
 dulcedo animae et sanitas ossium.
 \verse Est via, quae videtur homini recta,
 et novissima eius ducunt ad mortem.
 \verse Anima laborantis laborat sibi,
 quia compulit eum os suum.
 \verse Vir impius fodit malum,
 et in labiis eius quasi ignis ardens.
 \verse Homo perversus suscitat lites,
 et mussitator separat familiares.
 \verse Vir iniquus lactat amicum suum
 et ducit eum per viam non bonam.
 \verse Qui attonitis oculis cogitat prava,
 comprimens labia sua perficit malum.
 \verse Corona dignitatis canities,
 quae in viis iustitiae reperietur.
 \verse Melior est patiens viro forti,
 et, qui dominatur animo suo, expugnatore urbium.
 \verse Sortes mittuntur in sinum,
 sed a Domino temperantur.
 
\begin{biblechapter}
 \verse Melior est buccella sicca cum pace
 quam domus plena victimis cum iurgio.
 \verse Servus sapiens dominabitur filiis inhonestis
 et inter fratres hereditatem dividet.
 \verse Sicut igne probatur argentum et aurum camino,
 ita corda probat Dominus.
 \verse Malus oboedit labio iniquo,
 et fallax obtemperat linguae mendaci.
 \verse Qui despicit pauperem, exprobrat Factori eius;
 et, qui in ruina laetatur alterius, non erit impunitus.
 \verse Corona senum filii filiorum,
 et gloria filiorum patres eorum.
 \verse Non decent stultum verba composita,
 nec principem labium mentiens.
 \verse Gemma gratissima munus in oculis domini eius;
 quocumque se verterit, prospere aget.
 \verse Qui celat delictum, quaerit amicitias;
 qui sermone repetit, separat foederatos.
 \verse Plus proficit correptio apud prudentem
 quam centum plagae apud stultum.
 \verse Semper iurgia quaerit malus;
 angelus autem crudelis mittetur contra eum.
 \verse Expedit magis ursae occurrere, raptis fetibus,
 quam fatuo confidenti in stultitia sua.
 \verse Qui reddit mala pro bonis,
 non recedet malum de domo eius.
 \verse Aquarum proruptio initium est iurgiorum;
 et, antequam exacerbetur contentio, desere.
 \verse Qui iustificat impium et qui condemnat iustum,
 abominabilis est uterque apud Dominum.
 \verse Ad quid pretium in manu stulti?
 Ad emendam sapientiam, cum careat corde?
 \verse Omni tempore diligit, qui amicus est,
 et frater ad angustiam natus est.
 \verse Stultus homo iungit manus,
 cum spoponderit pro amico suo.
 \verse Qui diligit delictum, diligit rixas;
 et, qui exaltat ostium, quaerit effracturam.
 \verse Qui perversi cordis est, non inveniet bonum;
 et, qui vertit linguam, incidet in malum.
 \verse Qui generat stultum, maerorem generat sibi,
 sed nec pater in fatuo laetabitur.
 \verse Animus gaudens aetatem floridam facit,
 spiritus tristis exsiccat ossa.
 \verse Munera de sinu impius accipit,
 ut pervertat semitas iudicii.
 \verse In facie prudentis lucet sapientia,
 oculi stultorum in finibus terrae.
 \verse Ira patris filius stultus
 et dolor matris, quae genuit eum.
 \verse Non est bonum multam inferre iusto
 nec percutere principem contra rectitudinem.
 \verse Qui moderatur sermones suos, novit scientiam,
 et lenis spiritu est vir prudens.
 \verse Stultus quoque, si tacuerit, sapiens reputabitur
 et, si compresserit labia sua, intellegens.
 
\begin{biblechapter}
 \verse Occasiones quaerit, qui vult recedere ab amico;
 omni consilio exacerbatur.
 \verse Non delectatur stultus prudentia
 sed in revelatione cordis sui.
 \verse Cum venerit impius, veniet et contemptio,
 et cum ignominia opprobrium.
 \verse Aqua profunda verba ex ore viri,
 et torrens redundans fons sapientiae.
 \verse Accipere personam impii non est bonum,
 ut declines iustum in iudicio.
 \verse Labia stulti miscent se rixis,
 et os eius plagas provocat.
 \verse Os stulti ruina eius,
 et labia ipsius laqueus animae eius.
 \verse Verba susurronis quasi dulcia,
 et ipsa perveniunt usque ad interiora ventris.
 \verse Qui mollis et dissolutus est in opere suo,
 frater est viri dissipantis.
 \verse Turris fortissima nomen Domini;
 ad ipsum currit iustus et exaltabitur.
 \verse Substantia divitis urbs roboris eius
 et quasi murus excelsus in cogitatione eius.
 \verse Antequam conteratur, exaltatur cor hominis;
 et, antequam glorificetur, humiliatur.
 \verse Qui prius respondet quam audiat,
 stultitia est ei et contumelia.
 \verse Spiritus viri sustentat imbecillitatem suam;
 spiritum vero confractum, quis poterit sustinere?
 \verse Cor prudens possidebit scientiam,
 et auris sapientium quaerit doctrinam.
 \verse Donum hominis dilatat viam eius
 et ante principes deducit eum.
 \verse Qui prior in contentione loquitur, putatur iustus;
 venit amicus eius et arguet eum.
 \verse Lites comprimit sors
 et inter potentes quoque diiudicat.
 \verse Frater, qui offenditur, durior est civitate firma,
 et lites quasi vectes urbium.
 \verse De fructu oris viri replebitur venter eius,
 et genimina labiorum ipsius saturabunt eum.
 \verse Mors et vita in manu linguae;
 qui diligunt eam, comedent fructus eius.
 \verse Qui invenit mulierem bonam, invenit bonum
 et hausit gratiam a Domino.
 \verse Cum obsecrationibus loquetur pauper,
 et dives effabitur rigide.
 \verse Vir cum amicis concuti potest,
 sed est amicus, qui adhaereat magis quam frater.
 
\begin{biblechapter}
 \verse Melior est pauper, qui ambulat in simplicitate sua,
 quam qui torquet labia et est insipiens.
 \verse Ubi non est scientia animae, non est bonum;
 et, qui festinus est pedibus, offendit. 
 \verse Stultitia hominis supplantat gressus eius,
 et contra Deum fervet animo suo.
 \verse Divitiae addunt amicos plurimos;
 pauper autem ab amico suo separatur.
 \verse Testis falsus non erit impunitus;
 et, qui mendacia loquitur, non effugiet.
 \verse Multi blandiuntur faciei potentis,
 et omnes amici sunt dona tribuenti.
 \verse Omnes fratres hominis pauperis oderunt eum,
 insuverse per et amici procul recesserunt ab eo;
 qui tantum verba sectatur, nihil habebit.
 \verse Qui autem possessor est mentis, diligit animam suam,
 et custos prudentiae inveniet bona. 
 \verse Falsus testis non erit impunitus;
 et, qui loquitur mendacia, peribit.
 \verse Non decent stultum deliciae,
 nec servum dominari principibus.
 \verse Doctrina viri mitigat iram eius,
 et gloria eius est iniqua praetergredi.
 \verse Sicut fremitus leonis ita et regis ira,
 et sicut ros super herbam ita et gratia eius.
 \verse Calamitas patris filius stultus;
 et tecta iugiter perstillantia litigiosa mulier.
 \verse Domus et divitiae hereditas patrum,
 a Domino autem uxor prudens.
 \verse Pigredo immittit soporem,
 et anima dissoluta esuriet.
 \verse Qui custodit mandatum, custodit animam suam;
 qui autem neglegit viam suam, mortificabitur.
 \verse Feneratur Domino, qui miseretur pauperis,
 et vicissitudinem suam reddet ei.
 \verse Erudi filium tuum, dum spes est;
 ad interfectionem autem eius ne ponas animam tuam.
 \verse Qui impatiens est, sustinebit multam;
 et, si eum abripere vis, aliud appones.
 \verse Audi consilium et suscipe disciplinam,
 ut sis sapiens in novissimis tuis.
 \verse Multae cogitationes in corde viri,
 voluntas autem Domini permanebit.
 \verse Desiderabile in homine est misericordia eius;
 et melior est pauper quam vir mendax.
 \verse Timor Domini ad vitam,
 et in plenitudine commorabitur absque visitatione mali.
 \verse Abscondit piger manum suam in catino
 nec ad os suum applicat eam.
 \verse Derisore flagellato vel parvulus sapientior erit;
 si autem corripueris sapientem, intelleget disciplinam.
 \verse Qui affligit patrem et fugat matrem,
 filius inhonestus et ignominiosus.
 \verse Acquiesce, fili, ut audias doctrinam
 nec erres a sermonibus scientiae.
 \verse Testis iniquus deridet iudicium,
 et os impiorum devorat iniquitatem.
 \verse Paratae sunt derisoribus virgae,
 et plagae stultorum corporibus.
 
\begin{biblechapter}
 \verse Luxuriosa res vinum, et tumultuosa sicera;
 quicumque his delectatur, non erit sapiens.
 \verse Sicut rugitus leonis ita et terror regis:
 qui provocat eum, peccat in animam suam.
 \verse Honor est homini separari a contentionibus;
 omnes autem stulti miscentur contumeliis.
 \verse Propter frigus piger arare noluit;
 mendicabit ergo aestate, et non dabitur illi.
 \verse Sicut aqua profunda consilium in corde viri,
 sed homo sapiens exhauriet illud.
 \verse Multi homines misericordes vocantur;
 virum autem fidelem quis inveniet?
 \verse Iustus, qui ambulat in simplicitate sua,
 beatos post se filios derelinquet.
 \verse Rex, qui sedet in solio iudicii,
 dissipat omne malum intuitu suo.
 \verse Quis potest dicere: “ Mundavi cor meum,
 purus sum a peccato ”?
 \verse Pondus et pondus, mensura et mensura,
 utrumque abominabile est apud Dominum.
 \verse Ex studiis suis intellegitur puer,
 si munda et recta sint opera eius.
 \verse Aurem audientem et oculum videntem,
 Dominus fecit utrumque.
 \verse Noli diligere somnum, ne te egestas opprimat;
 aperi oculos tuos et saturare panibus.
 \verse “ Malum est, malum est! ” dicit omnis emptor
 et, cum recesserit, tunc gloriabitur.
 \verse Est aurum et multitudo gemmarum
 et vas pretiosum labia scientiae.
 \verse Tolle vestimentum eius, quia fideiussor exstitit alieni,
 et pro extraneis aufer pignus ab eo.
 \verse Suavis est homini panis mendacii,
 et postea implebitur os eius calculo. 
 \verse Cogitationes consiliis firmantur,
 et dispensationibus tractanda sunt bella.
 \verse Ei, qui revelat mysteria et calumniatur
 et dilatat labia sua, ne commiscearis.
 \verse Qui maledicit patri suo et matri,
 exstinguetur lucerna eius in mediis tenebris.
 \verse Hereditas, ad quam festinatur in principio,
 in novissimo benedictione carebit.
 \verse Ne dicas: “ Reddam malum ”;
 exspecta Dominum, et liberabit te. 
 \verse Abominatio est apud Dominum pondus et pondus;
 statera dolosa non est bona in oculis eius.
 \verse A Domino diriguntur gressus viri;
 quis autem hominum intellegere potest viam suam?
 \verse Laqueus est homini inconsulte dicere: “ Sanctum! ”
 et post vota retractare.
 \verse Ventilat impios rex sapiens
 et incurvat super eos rotam.
 \verse Lucerna Domini spiraculum hominis,
 quae investigat omnia secreta ventris.
 \verse Misericordia et veritas custodiunt regem,
 et roboratur clementia thronus eius.
 \verse Ornamentum iuvenum fortitudo eorum,
 et honor senum canities.
 \verse Livor vulneris absterget mala,
 et plagae in secretioribus ventris.
 
\begin{biblechapter}
 \verse Sicut rivi aquarum cor regis in manu Domini:
 quocumque voluerit, inclinabit illud.
 \verse Omnis via viri recta sibi videtur;
 appendit autem corda Dominus.
 \verse Facere misericordiam et iudicium
 magis placet Domino quam victimae.
 \verse Exaltatio oculorum et dilatatio cordis,
 lucerna impiorum: peccatum.
 \verse Cogitationes sollertis semper in abundantiam;
 omnis autem festinus semper in egestate est.
 \verse Qui congregat thesauros lingua mendacii,
 vento impingetur ad laqueos mortis.
 \verse Violentia impiorum detrahet eos,
 quia noluerunt facere iudicium.
 \verse Perversa via viri aliena est;
 qui autem mundus est, rectum opus eius.
 \verse Melius est sedere in angulo domatis
 quam cum muliere litigiosa et in domo communi.
 \verse Anima impii desiderat malum;
 non miserebitur proximo suo.
 \verse Multato derisore sapientior erit parvulus;
 et, si instruatur sapiens, sumet scientiam.
 \verse Excogitat Iustus de domo impii,
 ut praecipitet impios in malum.
 \verse Qui obturat aurem suam ad clamorem pauperis,
 et ipse clamabit, et non exaudietur.
 \verse Munus absconditum exstinguit iras,
 et donum in sinu indignationem maximam.
 \verse Gaudium iusto est facere iudicium,
 et ruina operantibus iniquitatem.
 \verse Vir, qui erraverit a via prudentiae,
 in coetu umbrarum commorabitur.
 \verse Qui diligit convivia, in egestate erit;
 qui amat vinum et pinguia, non ditabitur.
 \verse Redemptio pro iusto impius,
 et pro rectis iniquus.
 \verse Melius est habitare in terra deserta
 quam cum muliere rixosa et iracunda.
 \verse Thesaurus desiderabilis et pinguis in habitaculo sapientis,
 et imprudens homo dissipabit illum.
 \verse Qui sequitur iustitiam et misericordiam,
 inveniet vitam et iustitiam et gloriam.
 \verse Civitatem fortium ascendit sapiens
 et destruit robur fiduciae eius.
 \verse Qui custodit os suum et linguam suam,
 custodit ab angustiis animam suam.
 \verse Superbus et arrogans vocatur derisor,
 qui operatur in ira superbiae.
 \verse Desideria occidunt pigrum;
 noluerunt enim quidquam manus eius operari:
 \verse tota die concupiscit et desiderat;
 qui autem iustus est, tribuet et non parcit.
 \verse Hostiae impiorum abominabiles,
 eo magis quia offeruntur ex scelere.
 \verse Testis mendax peribit;
 vir oboediens loquetur in victoriam.
 \verse Vir impius obfirmat vultum suum;
 qui autem rectus est, corrigit viam suam.
 \verse Non est sapientia, non est prudentia,
 non est consilium contra Dominum.
 \verse Equus paratur ad diem belli,
 Dominus autem salutem tribuit.
 
\begin{biblechapter}
 \verse Melius est nomen bonum quam divitiae multae,
 super argentum et aurum gratia bona.
 \verse Dives et pauper obviaverunt sibi:
 utriusque operator est Dominus.
 \verse Callidus vidit malum et abscondit se;
 simplices pertransierunt et afflicti sunt damno.
 \verse Praemium modestiae timor Domini,
 divitiae et gloria et vita.
 \verse Spinae et laquei in via perversi,
 custos autem animae suae longe recedit ab eis.
 \verse Institue adulescentem iuxta viam suam;
 etiam cum senuerit, non recedet ab ea.
 \verse Dives pauperibus imperat;
 et, qui accipit mutuum, servus est fenerantis.
 \verse Qui seminat iniquitatem, metet mala
 et virga irae suae consummabitur.
 \verse Qui bono oculo est, benedicetur,
 de panibus enim suis dedit pauperi.
 \verse Eice derisorem, et exibit cum eo iurgium;
 cessabuntque causae et contumeliae.
 \verse Qui diligit cordis munditiam,
 propter gratiam labiorum suorum habebit amicum regem.
 \verse Oculi Domini custodiunt scientiam,
 et supplantantur verba iniqui.
 \verse Dicit piger: “ Leo est foris,
 in medio platearum occidendus sum ”.
 \verse Fovea profunda os alienae;
 cui iratus est Dominus, incidet in eam.
 \verse Stultitia colligata est in corde pueri,
 et virga disciplinae fugabit eam.
 \verse Opprimis pauperem? Ipse augebit divitias suas.
 Donas ditiori? Ipse egebis.
 \verse Inclina aurem tuam et audi verba sapientium,
 appone autem cor ad doctrinam meam,
 \verse quia pulchra erunt, cum servaveris ea in ventre tuo,
 et redundabunt in labiis tuis.
 \verse Ut sit in Domino fiducia tua,
 ostendi ea tibi hodie.
 \verse Nonne descripsi ea tibi nudiustertius
 in cogitationibus et scientia,
 \verse ut ostenderem tibi firmitatem verborum veritatis,
 ut respondeas illi, qui misit te?
 \verse Non facias violentiam pauperi, quia pauper est,
 neque conteras egenum in porta,
 \verse quia iudicabit Dominus causam eorum,
 et anima spoliabit spoliatores.
 \verse Noli esse amicus homini iracundo
 neque ambules cum viro furioso,
 \verse ne forte discas semitas eius
 et sumas scandalum animae tuae.
 \verse Noli esse cum his, qui iungunt manus suas
 et qui vades se offerunt pro debitis:
 \verse si enim non habes unde restituas,
 quid causae est ut tollat lectum tuum subter te?
 \verse Ne transferas terminos antiquos,
 quos posuerunt patres tui.
 \verse Vidisti virum velocem in opere suo:
 coram regibus stabit nec erit ante ignobiles.
 
\begin{biblechapter}
 \verse Quando sederis, ut comedas cum principe,
 diligenter attende, quae apposita sunt ante faciem tuam,
 \verse et statue cultrum in gutture tuo,
 si avidus es.
 \verse Ne desideres de cibis eius,
 quia est panis mendacii.
 \verse Noli laborare, ut diteris,
 sed in prudentia tua acquiesce.
 \verse Si erigas oculos tuos ad opes, iam non sunt;
 quia facient sibi pennas quasi aquilae et volabunt in caelum.
 \verse Ne comedas cum homine invido
 et ne desideres cibos eius;
 \verse quoniam sicut aestimavit in animo suo,
 ita ipse est.
 “ Comede et bibe ” dicet tibi,
 et mens eius non est tecum.
 \verse Buccellam, quam comederas, evomes
 et perdes pulchros sermones tuos.
 \verse In auribus insipientium ne loquaris,
 quia despicient doctrinam eloquii tui.
 \verse Ne attingas terminos viduae
 et agrum pupillorum ne introeas:
 \verse redemptor enim illorum fortis est,
 et ipse iudicabit contra te causam illorum.
 \verse Introduc ad doctrinam cor tuum
 et aures tuas ad verba scientiae.
 \verse Noli subtrahere a puero disciplinam;
 si enim percusseris eum virga, non morietur:
 \verse tu virga percuties eum
 et animam eius de inferno liberabis.
 \verse Fili mi, si sapiens fuerit cor tuum,
 gaudebit tecum et cor meum,
 \verse et exsultabunt renes mei,
 cum locuta fuerint rectum labia tua.
 \verse Non aemuletur cor tuum peccatores,
 sed in timore Domini esto tota die,
 \verse quia est tibi posteritas,
 et praestolatio tua non auferetur.
 \verse Audi, fili mi, et esto sapiens
 et dirige in via animum tuum.
 \verse Noli esse in conviviis potatorum
 nec in comissationibus carnis,
 \verse quia vacantes potibus et comissatores consumentur,
 et vestietur pannis dormitatio.
 \verse Audi patrem tuum, qui genuit te,
 et ne contemnas, cum senuerit mater tua.
 \verse Veritatem eme et noli vendere;
 sapientiam eme et doctrinam et intellegentiam.
 \verse Exsultat gaudio pater iusti;
 qui sapientem genuit, laetabitur in eo;
 \verse gaudeat pater tuus et mater tua,
 et exsultet, quae genuit te.
 \verse Praebe, fili mi, cor tuum mihi,
 et oculi tui vias meas custodiant.
 \verse Fovea enim profunda est meretrix,
 et puteus angustus aliena,
 \verse nam insidiatur ipsa in via quasi latro
 et iniquos in hominibus addet.
 \verse Cui “ Vae ”? Cui “ Eheu ”?
 Cui rixae? Cui querela?
 Cui sine causa vulnera? Cui suffusio oculorum?
 \verse His, qui commorantur in vino
 et eunt, ut scrutentur mixtum.
 \verse Ne intuearis vinum, quando flavescit,
 cum splenduerit in calice color eius:
 ingreditur blande,
 \verse sed in novissimo mordebit ut coluber
 et sicut regulus vulnerat.
 \verse Oculi tui videbunt extranea,
 et cor tuum loquetur perversa;
 \verse et eris sicut dormiens in medio mari
 et quasi sopitus ad malum navis:
 \verse “ Verberaverunt me, sed non dolui,
 percusserunt me, et ego non sensi;
 quando evigilabo et rursus illud requiram? ”.
 
\begin{biblechapter}
 \verse Ne aemuleris viros malos
 nec desideres esse cum eis,
 \verse quia rapinas meditatur mens eorum,
 et perniciem labia eorum loquuntur.
 \verse Sapientia aedificabitur domus,
 et prudentia roborabitur.
 \verse In doctrina replebuntur cellaria,
 universa substantia pretiosa et pulcherrima.
 \verse Vir sapiens fortis est,
 et vir doctus firmat robur.
 \verse Quia cum dispositione parabis tibi bellum,
 et erit salus, ubi multa consilia sunt.
 \verse Excelsa stulto sapientia,
 in porta non aperiet os suum.
 \verse Qui cogitat mala facere,
 vir perniciosus vocabitur.
 \verse Cogitatio stulti peccatum est,
 et abominatio hominum detractor. 
 \verse Si fueris lassus in die angustiae,
 coartabitur fortitudo tua.
 \verse Erue eos, qui ducuntur ad mortem;
 et, qui trahuntur ad interitum, retine.
 \verse Si dixeris: “ Nesciebamus hoc ”;
 nonne qui ponderator est cordis, ipse intellegit,
 et servatorem animae tuae nihil fallit
 reddetque homini iuxta opera sua?
 \verse Comede, fili mi, mel, quia bonum est
 et favum dulcissimum gutturi tuo.
 \verse Sic, scito, est sapientia animae tuae;
 quam cum inveneris, erit tibi posteritas,
 et spes tua non peribit.
 \verse Ne insidieris, o nequam, domui iusti
 neque vastes requiem eius.
 \verse Septies enim cadet iustus et resurget;
 impii autem corruent in malum.
 \verse Cum ceciderit inimicus tuus, ne gaudeas,
 et in ruina eius ne exsultet cor tuum,
 \verse ne forte videat Dominus, et displiceat ei
 et auferat ab eo iram suam.
 \verse Ne succendas ira in pessimos
 nec aemuleris impios,
 \verse quoniam non erit posteritas maligno,
 et lucerna impiorum exstinguetur.
 \verse Time Dominum, fili mi, et regem
 et cum nova sectantibus non commiscearis,
 \verse quoniam repente consurget perditio eorum,
 et ruinam utriusque quis novit?
 \verse Haec quoque sapientibus:
 Dignoscere personam in iudicio non est bonum.
 \verse Qui dicit impio: “ Iustus es ”,
 maledicent ei populi, et detestabuntur eum tribus.
 \verse Qui vero arguunt eum, laudabuntur,
 et super ipsos veniet benedictio boni.
 \verse Labia deosculatur,
 qui recta verba respondet.
 \verse Praepara foris opus tuum
 et diligenter exerce illud in agro tuo,
 ut postea aedifices domum tuam.
 \verse Ne sis testis frustra contra proximum tuum
 nec decipias quemquam labiis tuis.
 \verse Ne dicas: “ Quomodo fecit mihi, sic faciam ei,
 reddam viro secundum opus suum ”.
 \verse Per agrum hominis pigri transivi
 et per vineam viri sensu carentis:
 \verse et ecce totum repleverant urticae,
 et operuerant superficiem eius spinae,
 et maceria lapidum destructa erat; 
 \verse quod cum vidissem, posui in corde meo,
 vidi, didici disciplinam:
 \verse “ Parum dormies, modicum dormitabis,
 pauxillum manus conseres, ut quiescas,
 \verse et veniet tibi quasi cursor egestas,
 et mendicitas quasi vir armatus ”.
 
\begin{biblechapter}
 \verse Hae quoque parabolae Salomonis, quas transcripserunt viri Ezechiae regis Iudae.
 \verse Gloria Dei est celare verbum,
 et gloria regum investigare sermonem.
 \verse Caelum prae altitudine et terra prae profunditate,
 et cor regum inscrutabile.
 \verse Aufer scorias de argento,
 et egredietur vas pro argentario.
 \verse Aufer impium de conspectu regis,
 et firmabitur iustitia thronus eius.
 \verse Ne gloriosus appareas coram rege
 et in loco magnorum ne steteris.
 \verse Melius est enim ut dicatur tibi: “ Ascende huc ”,
 quam ut humilieris coram principe.
 \verse Quae viderunt oculi tui,
 ne proferas in iurgio cito,
 quoniam quid facies postea,
 cum dehonestaverit te amicus tuus?
 \verse Causam tuam tracta cum amico tuo
 et secretum extranei ne reveles,
 \verse ne forte insultet tibi, cum audierit,
 et contumelia tua revocari non poterit.
 \verse Mala aurea in ornatibus argenteis,
 verbum prolatum in tempore suo.
 \verse Inauris aurea et margaritum fulgens
 sapiens, qui arguit super aurem audientem.
 \verse Sicut frigus nivis in die messis,
 ita legatus fidelis ei, qui misit eum:
 animam ipsius recreat.
 \verse Nubes et ventus et pluviae non sequentes
 vir gloriosus et promissa non complens.
 \verse Patientia lenietur princeps,
 et lingua mollis confringet ossa.
 \verse Mel invenisti? Comede, quod sufficit tibi,
 ne forte satiatus evomas illud.
 \verse Subtrahe pedem tuum de domo proximi tui,
 ne quando satiatus oderit te.
 \verse Malleus et gladius et sagitta acuta
 homo, qui loquitur contra proximum suum falsum testimonium.
 \verse Dens putridus et pes vacillans,
 qui sperat super infideli in die angustiae.
 \verse Sicut exuens pallium in die frigoris,
 sicut acetum in nitro,
 qui cantat carmina cordi tristi.
 \verse Si esurierit inimicus tuus, ciba illum;
 si sitierit, pota illum:
 \verse prunas enim congregabis super caput eius,
 et Dominus reddet tibi.
 \verse Ventus aquilo parturit pluvias,
 et faciem tristem lingua detrahens. 
 \verse Melius est sedere in angulo domatis
 quam cum muliere litigiosa et in domo communi.
 \verse Aqua frigida animae sitienti
 et nuntius bonus de terra longinqua.
 \verse Fons turbatus pede et vena corrupta
 iustus cadens coram impio.
 \verse Mel nimium comedere non est bonum,
 nec quaestus gloriae est gloria.
 \verse Urbs diruta et absque muro
 vir, qui non potest cohibere spiritum suum.
 
\begin{biblechapter}
 \verse Quomodo nix in aestate et pluvia in messe,
 sic indecens est stulto gloria.
 \verse Sicut avis ad alia transvolans et hirundo volitans,
 sic maledictum frustra prolatum non superveniet.
 \verse Flagellum equo et camus asino
 et virga dorso stultorum.
 \verse Ne respondeas stulto iuxta stultitiam suam,
 ne tu quoque efficiaris ei similis;
 \verse responde stulto iuxta stultitiam suam,
 ne sibi sapiens esse videatur.
 \verse Amputat sibi pedes et iniuriam bibit,
 qui mittit verba per manum stulti.
 \verse Quomodo molles claudo tibiae,
 sic in ore stultorum parabola.
 \verse Sicut qui celat lapidem in acervo,
 ita qui tribuit insipienti honorem.
 \verse Spina crescens in manu temulenti,
 sic parabola in ore stultorum.
 \verse Sagittarius, qui conicit ad omnia,
 ita qui stultum conducit et qui vagos conducit.
 \verse Sicut canis, qui revertitur ad vomitum suum,
 sic stultus, qui iterat stultitiam suam.
 \verse Vidisti hominem sapientem sibi videri?
 Magis illo spem habebit stultus.
 \verse Dicit piger: “ Leaena est in via,
 et leo in plateis ”.
 \verse Ostium vertitur in cardine suo,
 et piger in lectulo suo.
 \verse Abscondit piger manum in catino
 et laborat, si ad os suum eam converterit.
 \verse Sapientior sibi piger videtur
 septem viris respondentibus sententias.
 \verse Apprehendit auribus canem,
 qui transiens commiscetur rixae alterius.
 \verse Sicut insanit, qui mittit sagittas
 et lanceas in mortem,
 \verse ita vir, qui decipit amicum suum
 et dicit: “ Nonne ludens feci? ”.
 \verse Cum defecerint ligna, exstinguetur ignis,
 et, susurrone subtracto, iurgia conquiescent.
 \verse Sicut carbones ad prunas et ligna ad ignem,
 sic homo litigiosus ad inflammandas rixas.
 \verse Verba susurronis quasi dulcia
 et ipsa perveniunt ad intima ventris.
 \verse Sicut argentum sordidum ornans vas fictile,
 sic labia levia et cor malum.
 \verse Labiis suis se dissimulabit inimicus,
 cum in corde tractaverit dolos:
 \verse quando mollierit vocem suam, ne credideris ei,
 quoniam septem abominationes sunt in corde illius;
 \verse operiet odium fraudulenter,
 revelabitur autem malitia eius in concilio.
 \verse Qui fodit foveam, incidet in eam;
 et, qui volvit lapidem, revertetur ad eum.
 \verse Lingua fallax non amat veritatem,
 et os lubricum operatur ruinas.
 
\begin{biblechapter}
 \verse Ne glorieris in crastinum
 ignorans, quid superventura pariat dies.
 \verse Laudet te alienus et non os tuum,
 extraneus et non labia tua.
 \verse Grave est saxum et onerosa arena,
 sed ira stulti utroque gravior.
 \verse Saevitas et erumpens furor,
 et coram zelo consistere quis poterit?
 \verse Melior est manifesta correptio
 quam amor absconditus.
 \verse Veriora sunt vulnera diligentis
 quam fraudulenta oscula odientis.
 \verse Anima saturata calcabit favum,
 et anima esuriens etiam amarum pro dulci sumet.
 \verse Sicut avis transmigrans de nido suo,
 sic vir errans longe a loco suo.
 \verse Unguento et ture delectatur cor
 et dulcedine amici in consilio ex animo.
 \verse Amicum tuum et amicum patris tui ne dimiseris
 et domum fratris tui ne ingrediaris in die afflictionis tuae.
 Melior est vicinus iuxta quam frater procul.
 \verse Stude sapientiae, fili mi, et laetifica cor meum,
 ut possim exprobranti mihi respondere sermonem.
 \verse Astutus videns malum absconditus est;
 simplices transeuntes multati sunt.
 \verse Tolle vestimentum eius, qui spopondit pro extraneo,
 et pro alienis aufer ei pignus.
 \verse Qui benedicit proximo suo voce grandi mane consurgens,
 maledictio reputabitur ei.
 \verse Tecta perstillantia in die frigoris
 et litigiosa mulier comparantur;
 \verse qui retinet eam, quasi qui ventum teneat,
 et oleum dextera sua tenere reperietur.
 \verse Ferrum ferro exacuitur,
 et homo exacuit faciem amici sui.
 \verse Qui servat ficum, comedet fructus eius;
 et, qui custos est domini sui, glorificabitur.
 \verse Quomodo in aqua facies prospicit ad faciem,
 sic cor hominis ad hominem.
 \verse Infernus et Perditio numquam implentur,
 similiter et oculi hominum insatiabiles.
 \verse Quomodo probatur in conflatorio argentum et in fornace aurum,
 sic probatur homo ore laudantis.
 \verse Si pilo contuderis stultum in pila quasi ptisanas,
 non auferetur ab eo stultitia eius.
 \verse Diligenter agnosce vultum pecoris tui;
 appone cor tuum ad greges,
 \verse non enim habebis iugiter divitias.
 Num corona tribuetur in generationem et generationem?
 \verse Nudata sunt prata, et apparuerunt herbae virentes,
 et collecta sunt fena de montibus;
 \verse agni ad vestimentum tuum,
 et haedi ad agri pretium;
 \verse sufficiat tibi lac caprarum in cibum tuum
 et in cibum domus tuae et ad victum ancillis tuis.
 
\begin{biblechapter}
 \verse Fugit impius, nemine persequente;
 iustus autem quasi leo confidens.
 \verse Propter peccata terrae multi principes eius;
 et propter hominem intellegentem et sapientem
 rectus ordo longior erit.
 \verse Vir pauper et calumnians pauperes
 similis est imbri vehementi, in quo paratur fames.
 \verse Qui derelinquunt legem, laudant impium;
 qui custodiunt, succenduntur contra eum.
 \verse Viri mali non intellegunt iudicium;
 qui autem requirunt Dominum, animadvertunt omnia.
 \verse Melior est pauper ambulans in simplicitate sua
 quam perversus in viis suis, quamquam dives.
 \verse Qui custodit legem, filius sapiens est;
 qui autem comissatores pascit, confundit patrem suum.
 \verse Qui coacervat divitias suas usuris et fenore,
 liberali in pauperes congregat eas.
 \verse Qui declinat aures suas, ne audiat legem,
 oratio quoque eius erit exsecrabilis.
 \verse Qui decipit iustos in via mala, in interitu suo corruet,
 et simplices possidebunt bona eius.
 \verse Sapiens sibi videtur vir dives,
 pauper autem prudens scrutabitur eum.
 \verse In exsultatione iustorum multa gloria est,
 et, cum exaltantur impii, abscondit se homo.
 \verse Qui abscondit scelera sua, non prosperabit;
 qui autem confessus fuerit et reliquerit ea,
 misericordiam consequetur.
 \verse Beatus homo, qui semper est pavidus;
 qui vero indurat cor suum, corruet in malum.
 \verse Leo rugiens et ursus esuriens
 princeps impius super populum pauperem.
 \verse Dux indigens prudentia multos opprimet;
 qui autem odit avaritiam, longi fient dies eius.
 \verse Hominem, animae cuiusdam sanguine gravatum,
 si usque ad lacum fugerit, nemo sustineat.
 \verse Qui ambulat simpliciter, salvus erit;
 qui perversis graditur viis, subito concidet.
 \verse Qui operatur terram suam, satiabitur panibus;
 qui autem sectatur otium, replebitur egestate.
 \verse Vir fidelis multum laudabitur;
 qui autem festinat ditari, non erit innocens.
 \verse Qui dignoscit in iudicio faciem, non benefacit;
 et pro buccella panis praevaricatur homo.
 \verse Festinat ditari vir invidus,
 ignorat quod egestas superveniet ei.
 \verse Qui corripit hominem, gratiam postea inveniet
 magis quam ille, qui lingua blanditur.
 \verse Qui abripit aliquid a patre suo et a matre
 et dicit: “ Hoc non est peccatum ”,
 particeps homicidae est.
 \verse Qui desiderium dilatat, iurgia concitat;
 qui vero sperat in Domino, impinguabitur.
 \verse Qui confidit in corde suo, stultus est;
 qui autem graditur sapienter, ipse salvabitur.
 \verse Qui dat pauperi, non indigebit;
 qui autem occultat oculos, abundabit maledictis.
 \verse Cum surrexerint impii, abscondentur homines;
 cum illi perierint, multiplicabuntur iusti.
 
\begin{biblechapter}
 \verse Vir, qui correptiones dura cervice contemnit,
 subito conteretur absque sanatione.
 \verse In multiplicatione iustorum laetabitur vulgus;
 et in dominatione impii gemet populus.
 \verse Vir, qui amat sapientiam, laetificat patrem suum;
 qui autem nutrit scorta, perdet substantiam.
 \verse Rex in iustitia erigit terram;
 vir acceptor donorum destruet eam.
 \verse Homo, qui blanditur amico suo,
 rete expandit gressibus eius.
 \verse In peccato vir iniquus irretitur laqueo,
 et iustus exsultabit atque gaudebit.
 \verse Novit iustus causam pauperum,
 impius ignorat scientiam.
 \verse Homines pestilentes dissipant civitatem;
 sapientes vero avertunt furorem.
 \verse Vir sapiens, si cum stulto iudicio contenderit,
 sive irascatur sive rideat, non inveniet requiem.
 \verse Viri sanguinum oderunt simplicem;
 iusti autem quaerunt animam eius.
 \verse Totum spiritum suum profert stultus;
 sapiens mitigat eum in posterum.
 \verse Princeps, qui libenter audit verba mendacii,
 omnes ministros habet impios.
 \verse Pauper et oppressor obviaverunt sibi,
 utriusque oculorum illuminator est Dominus.
 \verse Rex, qui iudicat in veritate pauperes,
 thronus eius in aeternum firmabitur.
 \verse Virga atque correptio tribuit sapientiam;
 puer autem, qui dimittitur voluntati suae, confundit matrem suam.
 \verse In multiplicatione impiorum multiplicabuntur scelera,
 et iusti ruinas eorum videbunt.
 \verse Erudi filium tuum, et refrigerabit te
 et dabit delicias animae tuae.
 \verse Cum visio defecerit, dissipabitur populus;
 qui vero custodit legem, beatus est.
 \verse Servus verbis non potest erudiri,
 quia intellegit et respondere contemnit.
 \verse Vidisti hominem velocem ad loquendum?
 Magis illo spem habebit insipiens.
 \verse Qui delicate a pueritia nutrit servum suum,
 postea sentiet eum contumacem.
 \verse Vir iracundus provocat rixas;
 et, qui ad indignandum facilis est, erit ad peccandum proclivior.
 \verse Superbia hominis humiliabit eum,
 et humilis spiritu suscipiet gloriam.
 \verse Qui cum fure participat, odit animam suam;
 adiuramentum audit et non indicat.
 \verse Timor hominis inducit laqueum;
 qui sperat in Domino, sublevabitur.
 \verse Multi requirunt faciem principis;
 et iudicium a Domino egreditur singulorum.
 \verse Abominantur iusti virum impium;
 et abominantur impii eos, qui recta sunt via.
 
\begin{biblechapter}
 \verse Verba Agur filii Iaces ex Massa.
 Oraculum hominis ad Itiel,
 ad Itiel et Ucal.
 \verse Quoniam stultissimus sum virorum,
 et sapientia hominum non est mecum;
 \verse et non didici sapientiam
 et scientiam sanctorum non novi.
 \verse Quis ascendit in caelum atque descendit?
 Quis continuit spiritum in manibus suis?
 Quis colligavit aquas quasi in vestimento?
 Quis statuit omnes terminos terrae?
 Quod nomen est eius, et quod nomen filii eius, si nosti?
 \verse Omnis sermo Dei probatus
 clipeus est sperantibus in eum.
 \verse Ne addas quidquam verbis illius:
 et arguaris inveniarisque mendax.
 \verse Duo rogavi te,
 ne deneges mihi, antequam moriar:
 \verse vanitatem et verba mendacia longe fac a me,
 mendicitatem et divitias ne dederis mihi,
 tribue tantum victum demensum mihi,
 \verse ne forte satiatus illiciar ad negandum
 et dicam: “ Quis est Dominus? ”
 aut egestate compulsus furer
 et periurem nomen Dei mei.
 \verse Ne calumnieris servum ad dominum suum,
 ne forte maledicat tibi, et puniaris.
 \verse Generatio, quae patri suo maledicit
 et quae matri suae non benedicit.
 \verse Generatio, quae sibi munda videtur
 et non est lota a sordibus suis.
 \verse Generatio, cuius oculi quam excelsi sunt,
 et palpebrae eius in alta surrectae!
 \verse Generatio, quae pro dentibus gladios habet,
 et cultri molares eius,
 ut comedat inopes de terra
 et pauperes ex hominibus.
 \verse Sanguisugae duae sunt filiae:
 “ Affer, affer! ”.
 Tria sunt insaturabilia,
 et quattuor, quae numquam dicunt: “ Sufficit! ”:
 \verse infernus et venter sterilis,
 terra, quae non satiatur aqua,
 ignis, qui numquam dicit: “ Sufficit! ”.
 \verse Oculum, qui subsannat patrem
 et qui despicit obsequium matris suae,
 effodiant eum corvi de torrente,
 et comedant eum filii aquilae.
 \verse Tria sunt nimis difficilia mihi,
 et quattuor penitus ignoro:
 \verse viam aquilae in caelo,
 viam colubri super petram,
 viam navis in medio mari
 et viam viri in adulescentula.
 \verse Talis est et via mulieris adulterae,
 quae comedit et tergens os suum dicit:
 “ Non sum operata malum ”.
 \verse Per tria movetur terra,
 et quattuor non potest sustinere:
 \verse per servum, cum regnaverit,
 per stultum, cum saturatus fuerit cibo,
 \verse per odiosam mulierem, cum in matrimonio fuerit assumpta,
 et per ancillam, cum fuerit heres dominae suae.
 \verse Quattuor sunt minima terrae,
 et ipsa sunt sapientiora sapientibus:
 \verse formicae populus infirmus,
 quae praeparant in messe cibum sibi;
 \verse hyraces plebs invalida,
 qui collocant in petra cubile suum;
 \verse regem locusta non habet
 et egreditur universa per turmas suas;
 \verse stellio manibus nititur
 et moratur in aedibus regis.
 \verse Tria sunt, quae bene gradiuntur,
 et quattuor, quae incedunt feliciter:
 \verse leo fortissimus bestiarum
 ad nullius pavebit occursum,
 \verse gallus succinctus lumbos et aries
 et rex, qui secum habet exercitum.
 \verse Si stultum te praebuisti, postquam elevatus es in sublime,
 et si considerasti, ori impone manum.
 \verse Qui enim fortiter premit lac, exprimit butyrum,
 et, qui vehementer emungit nares, elicit sanguinem,
 et, qui provocat iras, producit discordias.
 
\begin{biblechapter}
 \verse Verba Lamuelis regis Massa, quae erudivit eum mater eius.
 \verse Quid, fili mi? Quid, fili uteri mei?
 Quid, fili votorum meorum?
 \verse Ne dederis mulieribus substantiam tuam
 et vias tuas illis, quae delent reges.
 \verse Non decet reges, o Lamuel, non decet reges bibere vinum,
 nec magistratus desiderare siceram,
 \verse ne forte bibant et obliviscantur iudiciorum
 et mutent causam omnium filiorum pauperis.
 \verse Date siceram pereunti
 et vinum his, qui amaro sunt animo:
 \verse bibat et obliviscatur egestatis suae
 et doloris sui non recordetur amplius.
 \verse Aperi os tuum pro muto
 et causis omnium filiorum, qui pereunt;
 \verse aperi os tuum, decerne, quod iustum est,
 et iudica inopem et pauperem.
 \verse ALEPH. Mulierem fortem quis inveniet?
 Longe super gemmas pretium eius.
 \verse BETH. Confidit in ea cor viri sui et spoliis non indigebit.
 \verse GHIMEL. Reddet ei bonum et non malum omnibus diebus vitae suae.
 \verse DALETH. Quaesivit lanam et linum
 et operata est delectatione manuum suarum.
 \verse HE. Facta est quasi navis institoris
 de longe portans panem suum.
 \verse VAU. Et de nocte surrexit
 deditque praedam domesticis suis
 et cibaria ancillis suis.
 \verse ZAIN. Consideravit agrum et emit eum;
 de fructu manuum suarum plantavit vineam.
 \verse HETH. Accinxit fortitudine lumbos suos
 et roboravit brachium suum.
 \verse TETH. Gustavit et vidit quia bona est negotiatio eius;
 non exstinguetur in nocte lucerna eius.
 \verse IOD. Manum suam misit ad colos,
 et digiti eius apprehenderunt fusum.
 \verse CAPH. Palmas suas aperuit inopi
 et manum suam extendit ad pauperem.
 \verse LAMED. Non timebit domui suae a frigoribus nivis:
 omnes enim domestici eius vestiti sunt duplicibus.
 \verse MEM. Stragulatam vestem fecit sibi;
 byssus et purpura indumentum eius.
 \verse NUN. Nobilis in portis vir eius,
 quando sederit cum senatoribus terrae.
 \verse SAMECH. Sindonem fecit et vendidit
 et cingulum tradidit Chananaeo.
 \verse Ain. Fortitudo et decor indumentum eius,
 et ridebit in die novissimo.
 \verse PHE. Os suum aperuit sapientiae,
 et lex clementiae in lingua eius.
 \verse SADE. Consideravit semitas domus suae
 et panem otiosa non comedit.
 \verse COPH. Surrexerunt filii eius et beatissimam praedicaverunt,
 vir eius et laudavit eam:
 \verse RES. “ Multae filiae fortiter operatae sunt,
 tu supergressa es universas ”.
 \verse SIN. Fallax gratia et vana est pulchritudo;
 mulier timens Dominum ipsa laudabitur.
 \verse TAU. Date ei de fructu manuum suarum,
 et laudent eam in portis opera eius.
\end{biblechapter}
