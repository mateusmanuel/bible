\biblebook{ Liber Ecclesiasticus}
\begin{biblechapter}
 \verse Omnis sapientia a Domino Deo est
 et cum illo fuit semper et est ante aevum.
 \verse Arenam maris et pluviae guttas
 et dies saeculi quis dinumeravit?
 Altitudinem caeli et latitudinem terrae
 et profundum abyssi quis mensus est?
 \verse Sapientiam Dei praecedentem omnia quis investigavit?
 \verse Prior omnium creata est sapientia,
 et intellectus prudentiae ab aevo.
 \verse Fons sapientiae verbum Dei in excelsis,
 et ingressus illius mandata aeterna.
 \verse Radix sapientiae cui revelata est?
 Et astutias illius quis agnovit?
 \verse Disciplina sapientiae cui revelata est et manifestata?
 Et multiplicem peritiam illius quis intellexit?
 \verse Unus est Altissimus, creator omnipotens
 et rex potens et metuendus nimis,
 sedens super thronum suum et dominans, Deus.
 \verse Ipse creavit illam in spiritu sancto
 et vidit et dinumeravit et mensus est;
 \verse et effudit illam super omnia opera sua
 et super omnem carnem secundum largitatem suam
 et praebuit illam diligentibus se.
 \verse Timor Domini gloria et gloriatio
 et laetitia et corona exsultationis.
 \verse Timor Domini delectabit cor
 et dabit laetitiam et gaudium et longitudinem dierum.
 \verse Timenti Dominum bene erit in extremis,
 et in die defunctionis suae benedicetur.
 \verse Dilectio Dei honorabilis sapientia;
 \verse quibus autem apparuerit, dispertit eam in visionem sui ipsius
 et in agnitione magnalium suorum.
 \verse Initium sapientiae timor Domini,
 et cum fidelibus in vulva concreata est;
 cum hominibus veritatis ab aevo fundata est
 et semini eorum se credet.
 \verse Timor Domini scientiae religiositas;
 \verse religiositas custodiet et iustificabit cor,
 iucunditatem atque gaudium dabit.(\verse)
 (\verse Plenitudo sapientiae est timere Deum;
 et inebriat eos fructibus suis.
 \verse Omnem domum illius implebit rebus pretiosis
 et receptacula thesauris illius.
 \verse Corona sapientiae timor Domini,
 repollens pacem et salutis fructum:
 \verse utraque autem sunt dona Dei.
 \verse Scientiam et intellectum prudentiae sapientia effundit quasi pluviam;
 et gloriam tenentium se exaltat.
 \verse Radix sapientiae est timere Dominum,
 et rami illius longaevi.
 \verse In thesauris sapientiae intellectus et scientiae religiositas;
 exsecratio autem peccatoribus sapientia.
 \verse Timor Domini expellit peccatum;
 cum autem adsit, omnem avertit iram.
 \verse Nam, qui sine timore est, non poterit iustificari;
 iracundia enim animositatis illius subversio illi erit.
 \verse Usque in tempus sustinebit patiens,
 et postea erit redditio iucunditatis.
 \verse Bonus sensus usque in tempus abscondet verba illius,
 et labia multorum enarrabunt sensum illius.
 \verse In thesauris sapientiae parabola disciplinae;
 \verse exsecratio autem peccatori cultura Dei.
 \verse Fili, concupiscens sapientiam conserva iustitiam,
 et Deus praebebit illam tibi.
 \verse Sapientia enim et disciplina timor Domini,
 et quod beneplacitum est illi,
 \verse fides et mansuetudo.
 \verse Ne sis incredibilis timori Domini
 et ne accesseris ad illum duplici corde.
 \verse Ne fueris hypocrita in conspectu hominum
 et cave a labiis tuis.
 \verse Ne extollas teipsum, ne forte cadas
 et adducas animae tuae inhonorationem,
 \verse et revelet Deus absconsa tua
 et in medio synagogae elidat te;
 \verse quoniam accessisti maligne ad timorem Domini,
 et cor tuum plenum est dolo et fallacia.
 
\begin{biblechapter}
\verse Fili, accedens ad servitutem Dei
 sta in iustitia et timore
 et praepara animam tuam ad tentationem.
 \verse Dirige cor tuum et sustine,
 inclina aurem tuam et suscipe verba intellectus
 et ne sollicitus sis in tempore calamitatis.
 \verse Sustine sustentationes Dei, coniungere Deo et ne laxes,
 ut sapiens fias in viis tuis.
 \verse Omne, quod tibi applicitum fuerit, accipe et in dolore sustine
 et in humilitate tua patientiam habe,
 \verse quoniam in igne probatur aurum et argentum,
 homines vero receptibiles in camino humiliationis.
 \verse Crede Deo, et recuperabit te,
 et spera in illum, et diriget viam tuam;
 serva timorem illius et in illo veterasce.
 \verse Metuentes Dominum, sustinete misericordiam eius
 et non deflectatis ab illo, ne cadatis.
 \verse Qui timetis Dominum, credite illi,
 et non evacuabitur merces vestra.
 \verse Qui timetis Dominum, sperate in bona
 et in oblectationem aevi et in misericordiam.
 \verse Qui timetis Dominum, diligite illum, et illuminabuntur corda vestra.
 \verse Respicite, filii, generationes antiquas et videte:
 quis speravit in Domino et confusus est?
 \verse Aut quis permansit in mandatis eius et derelictus est?
 Aut quis invocavit eum, et despexit illum?
 \verse Quoniam pius et misericors est Dominus
 et remittet in die tribulationis peccata
 et protector est omnibus exquirentibus se in veritate.
 \verse Vae duplici corde et labiis scelestis et manibus dissolutis
 et peccatori terram ingredienti duabus viis!
 \verse Vae dissolutis corde, qui non credunt,
 et ideo non protegentur!
 \verse Vae vobis, qui perdidistis sustinentiam
 et qui dereliquistis vias rectas et divertistis in vias pravas!
 \verse Et quid facietis, cum inspicere coeperit Dominus?
 \verse Qui timent Dominum, non erunt incredibiles verbo illius;
 et, qui diligunt illum, conservabunt viam illius.
 \verse Qui timent Dominum, inquirent quae beneplacita sunt ei;
 et, qui diligunt eum, replebuntur lege ipsius.
 \verse Qui timent Dominum, praeparabunt corda sua
 et in conspectu illius sanctificabunt animas suas.
 \verse Qui timent Dominum, custodiunt mandata illius
 et patientiam habebunt usque ad inspectionem illius
 \verse dicentes: “ Si paenitentiam non egerimus,
 incidemus in manus Domini et non in manus hominum;
 \verse secundum enim magnitudinem ipsius,
 sic et misericordia illius ”.
 
\begin{biblechapter}
\verse Filii sapientiae ecclesia iustorum,
 et natio illorum oboedientia et dilectio.
 \verse Indicium patris audite, filii,
 et sic facite, ut salvi sitis.
 \verse Deus enim honoravit patrem in filiis
 et iudicium matris firmavit in filios.
 \verse Qui honorat patrem, exorabit pro peccatis
 et continebit se ab illis
 et in oratione dierum exaudietur.
 \verse Et, sicut qui thesaurizat,
 ita et qui honorificat matrem suam.
 \verse Qui honorat patrem suum, iucundabitur in filiis
 et in die orationis suae exaudietur;
 \verse qui honorat patrem suum, vita vivet longiore,
 et, qui oboedit patri, refrigerabit matrem.
 \verse Qui timet Dominum, honorat parentes
 et quasi dominis serviet his, qui se genuerunt.
 \verse In opere et sermone honora patrem tuum,
 \verse ut superveniat tibi benedictio ab eo.
 \verse Benedictio patris firmat domos filiorum;
 maledictio autem matris eradicat fundamenta.
 \verse Ne glorieris in contumelia patris tui,
 non est enim tibi gloria eius confusio;
 \verse gloria enim hominis ex honore patris sui,
 et dedecus filii mater sine honore.
 \verse Fili, suscipe senectam patris tui
 et non contristes eum in vita illius;
 \verse et, si defecerit sensu, veniam da
 et ne spernas eum omnibus diebus vitae eius.
 Eleemosyna enim patris non erit in oblivione,
 \verse nam pro peccatis ipsa plantabitur
 \verse et in iustitia aedificabitur tibi;
 et in die tribulationis commemorabitur tui,
 et sicut in sereno glacies solventur tua peccata.
 \verse Quam malae famae est, qui derelinquit patrem;
 et maledictus a Deo, qui exasperat matrem.
 \verse Fili, in mansuetudine opera tua perfice
 et super hominem datorem diligeris.
 \verse Quanto magnus es, humilia te in omnibus
 et coram Deo invenies gratiam.
 Multi sunt excelsi et gloriosi,
 sed mansuetis revelat mysteria sua.
 \verse Quoniam magna potentia Dei solius,
 et ab humilibus honoratur.
 \verse Altiora te ne quaesieris
 et fortiora te ne scrutatus fueris;
 sed, quae praecepit tibi Deus, illa cogita semper
 et in pluribus operibus eius ne fueris curiosus.
 \verse Non est enim tibi necessarium
 ea, quae abscondita sunt, videre oculis tuis.
 \verse In supervacuis rebus noli scrutari multipliciter;
 \verse plurima enim super sensum hominum ostensa sunt tibi.
 \verse Multos quoque supplantavit suspicio illorum,
 et species vana decepit sensus illorum.
 Sine pupilla deerit lux,
 sine scientia deerit sapientia.
 \verse Cor durum male habebit in novissimo;
 et, qui amat periculum, in illo peribit.
 \verse Cor ingrediens duas vias non habebit successus,
 et pravus corde in illis scandalizabitur.
 \verse Cor nequam gravabitur doloribus,
 et peccator adiciet peccatum ad peccatum.
 \verse Plagis superborum non erit sanitas,
 frutex enim peccati radicabitur in illis et non intellegetur.
 \verse Cor sapientis intelleget verba sapientium,
 et auris audiens concupiscet sapientiam.
 \verse Sapiens cor et intellegibile abstinebit se a peccatis
 et in operibus iustitiae successus habebit.
 \verse Ignem ardentem exstinguit aqua,
 sic eleemosyna expiat peccata.
 \verse Deus prospector est eius, qui reddit gratiam;
 meminit eius in posterum,
 et in tempore casus sui inveniet firmamentum.
 
\begin{biblechapter}
\verse Fili, eleemosynam pauperis ne defraudes
 et oculos tuos ne transvertas a paupere.
 \verse Animam esurientem ne contristaveris
 et non exasperes pauperem in inopia sua.
 \verse Cor inopis ne afflixeris
 et non protrahas datum angustianti.
 \verse Rogationem contribulati ne abicias
 et non avertas faciem tuam ab egeno.
 \verse Ab inope ne avertas oculos tuos propter iram
 et non des ei locum tibi retro maledicendi;
 \verse maledicentis enim tibi in amaritudine animae,
 exaudietur precatio illius:
 exaudiet autem eum, qui fecit illum.
 \verse Congregationi affabilem te facito
 et presbytero humilia animam tuam
 et magnato humilia caput tuum.
 \verse Declina pauperi sine tristitia aurem tuam
 et redde debitum tuum
 et responde illi pacifica in mansuetudine.
 \verse Libera eum, qui iniuriam patitur, de manu opprimentis eum
 et non acide feras in anima tua in iudicando.
 \verse Esto pupillis misericors ut pater
 et pro viro matri illorum;
 \verse et eris velut filius Altissimi oboediens,
 et miserebitur tui magis quam mater.
 \verse Sapientia filiis suis vitam inspiravit
 et suscipit inquirentes se.
 \verse Qui illam diligit, diligit vitam;
 et, qui vigilaverint ad illam, complectentur placorem a Domino.
 \verse Qui tenuerint illam, gloriam hereditabunt,
 et, quo introibit, benedicet Deus.
 \verse Qui serviunt ei, obsequentes erunt Sancto,
 et eos, qui diligunt illam, diligit Deus.
 \verse Qui audit illam, iudicabit gentes;
 et, qui intuetur illam, permanebit confidens.
 \verse Si crediderit ei, hereditabit illam,
 et erunt in possessione generationes illius.
 \verse Quoniam in simulatione ambulat cum eo
 et in primis explorat eum,
 \verse timorem et tremorem inducet super illum
 et probabit illum in tentatione doctrinae suae,
 donec ipse teneat eam in cogitationibus suis
 et credat animae illius.
 \verse Et redibit recta ad illum et firmabit illum
 et laetificabit illum 
\verse et denudabit illi absconsa sua
 et thesaurizabit super illum scientiam et intellectum iustitiae.
 \verse Si autem oberraverit, derelinquet eum
 et tradet eum in manus inimici sui.
 \verse Fili, observa tempus et devita a malo
 \verse et pro anima tua ne confundaris;
 \verse est enim confusio adducens peccatum,
 et est confusio adducens gloriam et gratiam.
 \verse Ne accipias faciem adversus animam tuam
 nec adversus animam tuam mendacium.
 \verse Ne reverearis proximum tuum in casu suo
 \verse nec retineas verbum in tempore suo;
 non abscondas sapientiam tuam in decorem.
 \verse In verbo enim agnoscitur sapientia,
 et sensus in responsione linguae.
 \verse Non contradicas verbo veritatis ullo modo
 et de ineruditione tua confundere.
 \verse Non confundaris confiteri peccata tua
 et ne subicias te omni homini pro peccato.
 \verse Noli resistere contra faciem potentis
 nec coneris contra ictum fluvii.
 \verse Usque ad mortem agonizare pro iustitia,
 et Deus expugnabit pro te inimicos tuos.
 \verse Noli citatus esse in lingua tua
 et pavidus et remissus in operibus tuis.
 \verse Noli esse sicut leo in domo tua
 evertens domesticos tuos et opprimens subiectos tibi.
 \verse Non sit porrecta manus tua ad accipiendum
 et ad dandum collecta.
 
\begin{biblechapter}
\verse Ne innitaris possessionibus tuis
 et ne dixeris: “ Est mihi sufficiens vita ”.
 \verse Ne sequaris fortitudinem tuam,
 ut ambules in concupiscentiis cordis tui,
 \verse et ne dixeris: “ Quis praevalebit in me? ”
 aut “ Quis me subiciet propter facta mea? ”.
 Deus enim vindicans vindicabit.
 \verse Ne dixeris: “ Peccavi, et quid mihi accidit triste? ”.
 Altissimus enim est patiens redditor.
 \verse De propitiato peccato noli esse sine metu
 neque adicias peccatum super peccatum
 \verse et ne dicas: “ Miseratio Domini magna est,
 multitudinis peccatorum meorum miserebitur ”;
 \verse misericordia enim et ira ab illo cito proximant,
 et super peccatores requiescit ira illius.
 \verse Non tardes converti ad Dominum
 et ne differas de die in diem.
 \verse Subito enim veniet ira illius,
 et in tempore vindictae abripieris.
 \verse Ne innitaris divitiis iniustis,
 nihil enim proderunt in die calamitatis.
 \verse Non ventiles in omnem ventum
 et non eas in omnem viam;
 sic enim omnis peccator probatur in duplici lingua.
 \verse Esto firmus in sensu tuo
 et in veritate sensus tui et scientia;
 et prosequatur te verbum pacis et iustitiae.
 \verse Esto velox ad audiendum verbum, ut intellegas
 et cum tarditate proferas responsum.
 \verse Si est tibi intellectus, responde proximo;
 sin autem, sit manus tua super os tuum,
 ne capiaris in verbo indisciplinato et confundaris.
 \verse Honor et ignominia in sermone;
 lingua vero homini subversio est ipsius.
 \verse Non appelleris susurro,
 et lingua tua ne calumnieris.
 \verse Super furem enim est confusio,
 et denotatio pessima super bilinguem,
 susurratori autem odium et inimicitia et contumelia.
 
\begin{biblechapter}
\verse Nec pusillum nec multum noceas
 et noli fieri pro amico inimicus.
 Nomen enim malum et improperium et contumeliam hereditabis;
 sic omnis peccator invidus et bilinguis.
 \verse Non te extollas in cogitatione animae tuae velut taurus,
 ne forte elidatur virtus tua per stultitiam,
 \verse et folia tua comedat et fructus tuos perdat,
 et relinquaris velut lignum aridum in eremo.
 \verse Anima enim nequam disperdet eum, qui se habet,
 et in gaudium inimicis dat illum
 et deducet in sortem impiorum.
 \verse Os dulce multiplicat amicos et mitigat inimicos;
 et lingua eucharis salutem dicit.
 \verse Multi pacifici sint tibi,
 et consiliarius sit tibi unus de mille.
 \verse Si possides amicum, in tentatione posside eum
 et ne facile credas ei.
 \verse Est enim amicus secundum opportunitatem suam
 et non permanebit in die tribulationis.
 \verse Et est amicus, qui convertitur ad inimicitiam
 et rixam convicii tui denudabit.
 \verse Est autem amicus socius mensae
 et non permanebit in die necessitatis;
 \verse in prosperis erit tibi quasi coaequalis
 et in domesticis tuis fiducialiter aget.
 \verse Si humiliatus fueris, convertetur contra te
 et a facie tua abscondet se.
 \verse Ab inimicis tuis separare
 et de amicis tuis attende.
 \verse Amicus fidelis protectio fortis;
 qui autem invenit illum, invenit thesaurum.
 \verse Amico fideli nulla est comparatio,
 et non est ponderatio contra bonitatem illius.
 \verse Amicus fidelis medicamentum vitae,
 et, qui metuunt Dominum, invenient illum.
 \verse Qui timet Deum, aeque habebit amicitiam eius,
 quoniam secundum illum erit amicus illius.
 \verse Fili, a iuventute tua excipe doctrinam
 et usque ad canos invenies sapientiam.
 \verse Quasi is qui arat et seminat, accede ad eam
 et sustine bonos fructus illius.
 \verse In opere enim ipsius exiguum laborabis
 et cito edes de generationibus illius.
 \verse Quam aspera est nimium sapientia indoctis hominibus,
 et non permanebit in illa excors.
 \verse Quasi lapis probationis gravis erit super illum,
 et non demorabitur proicere illam.
 \verse Doctrina est enim secundum nomen eius
 et non est multis manifesta;
 quibus autem cognita est, permanet usque ad conspectum Dei.
 \verse Audi, fili, et accipe sententiam meam
 et ne abicias consilium meum.
 \verse Inice pedem tuum in compedes illius
 et in torques illius collum tuum;
 \verse subice umerum tuum et porta illam
 et ne acedieris vinculis eius.
 \verse In omni animo tuo accede ad illam
 et in omni virtute tua conserva vias eius.
 \verse Investiga et scrutare, exquire et invenies,
 et continens factus ne derelinquas eam.
 \verse In novissimis enim invenies requiem in ea,
 et convertetur tibi in oblectationem.
 \verse Et erunt tibi compedes eius in protectionem fortitudinis,
 et torques illius in stolam gloriae;
 \verse decor enim aureus est in illa,
 et vincula illius alligatura hyacinthina.
 \verse Stolam gloriae indues eam
 et coronam gratulationis superpones tibi.
 \verse Fili, si attenderis, disces;
 et, si accommodaveris animum tuum, prudens eris.
 \verse Si dilexeris audire, excipies doctrinam;
 et, si inclinaveris aurem tuam, sapiens eris.
 \verse In multitudine presbyterorum sta
 et sapientiae illorum ex corde coniungere,
 ut omnem narrationem Dei velis audire,
 et proverbia intellectus non effugiant a te.
 \verse Et, si videris sensatum, evigila ad eum,
 et gradus ostiorum illius exterat pes tuus.
 \verse Cogitatum tuum habe in praeceptis Dei
 et in mandatis illius maxime assiduus esto;
 et ipse firmabit tibi cor,
 et concupiscentia sapientiae dabitur tibi.
 
\begin{biblechapter}
\verse Noli facere mala, et mala non te apprehendent;
 \verse discede ab iniquitate, et deficiet abs te.
 \verse Fili, non semines in sulcis iniustitiae
 et non metes ea in septuplum.
 \verse Noli quaerere a Domino ducatum
 neque a rege cathedram honoris.
 \verse Non te iustifices ante Deum,
 quoniam agnitor cordis ipse est;
 et penes regem noli velle videri sapiens.
 \verse Noli quaerere fieri iudex,
 nisi valeas virtute irrumpere iniquitates;
 ne forte extimescas faciem potentis
 et ponas scandalum in aequitate tua.
 \verse Non pecces in multitudinem civitatis
 nec te immittas in populum.
 \verse Neque alliges duplicia peccata;
 nec enim in uno eris immunis.
 \verse Noli esse pusillanimis in oratione tua;
 \verse exorare et facere eleemosynam ne despicias.
 \verse Ne dicas: “ In multitudine munerum meorum respiciet Deus
 et, offerente me Deo altissimo, munera mea suscipiet ”.
 \verse Non irrideas hominem in amaritudine animae;
 est enim, qui humiliat et exaltat, circumspector Deus.
 \verse Noli arare mendacium adversus fratrem tuum
 neque in amicum similiter facias.
 \verse Noli velle mentiri omne mendacium,
 assiduitas enim illius non est bona.
 \verse Noli verbosus esse in multitudine presbyterorum
 et non iteres verbum in oratione tua.
 \verse Non oderis laboriosa opera
 et rusticationem creatam ab Altissimo.
 \verse Non te reputes in multitudine indisciplinatorum;
 \verse memento irae, quoniam non tardabit.
 \verse Humilia valde spiritum tuum,
 quoniam vindicta carnis impii ignis et vermis.
 \verse Noli commutare amicum cum pecunia
 neque fratrem carissimum cum auro Ophir.
 \verse Noli discedere a muliere sensata et bona,
 quam sortitus es in timore Domini;
 gratia enim verecundiae illius super aurum.
 \verse Non laedas servum operantem in veritate
 neque mercennarium dantem animam suam.
 \verse Servus sensatus sit tibi dilectus quasi anima tua;
 non defraudes illum libertate
 neque inopem derelinquas illum.
 \verse Pecora tibi sunt? Attende illis,
 et, si sunt utilia, perseverent apud te.
 \verse Filii tibi sunt? Erudi illos
 et curva a pueritia cervicem illorum.
 \verse Filiae tibi sunt? Serva corpus illarum
 et non ostendas hilarem faciem tuam ad illas.
 \verse Trade filiam, et grande opus feceris,
 et homini sensato da illam.
 \verse Mulier si est tibi secundum animam tuam, non proicias illam,
 sed odibili non credas teipsum.
 In toto corde tuo 
\verse honora patrem tuum
 et gemitus matris tuae ne obliviscaris.
 \verse Memento quoniam, nisi per illos, natus non fuisses;
 et quid retribues illis, quomodo et illi tibi?
 \verse In tota anima tua time Dominum
 et sacerdotes illius sanctifica.
 \verse In omni virtute tua dilige eum, qui te fecit,
 et ministros eius ne derelinquas.
 \verse Honora Deum ex tota anima tua
 et honorifica sacerdotes.
 \verse Da illis partem, sicut mandatum est tibi,
 primitiarum et purgationis et de neglegentia
 \verse et armum in oblationem
 et sacrificium sanctificationis et primitias sanctorum.
 \verse Et pauperi porrige manum tuam,
 ut perficiatur propitiatio et benedictio tua.
 \verse Gratia dati in conspectu omnis viventis,
 sed et mortuo non prohibeas gratiam.
 \verse Non desis plorantibus in consolatione
 et cum lugentibus luge.
 \verse Non te pigeat visitare infirmum,
 ex his enim in dilectione firmaberis.
 \verse In omnibus operibus tuis memorare novissima tua
 et in aeternum non peccabis.
 
\begin{biblechapter}
\verse Non litiges cum homine potente,
 ne forte incidas in manus illius.
 \verse Non contendas cum viro locuplete,
 ne forte contra te constituat pondus tuum:
 \verse multos enim perdidit aurum,
 et argentum etiam cor regum subvertit.
 \verse Non litiges cum homine linguato
 et non struas in ignem illius ligna.
 \verse Non communices homini indocto,
 ne contemnaris a principibus.
 \verse Ne despicias hominem avertentem se a peccato
 neque improperes ei;
 memento quoniam omnes in correptione sumus.
 \verse Ne spernas hominem in sua senectute,
 etenim ex nobis senescunt.
 \verse Noli de mortuo inimico tuo gaudere;
 memento quoniam omnes morimur et in gaudium nolumus venire.
 \verse Ne despicias narrationem presbyterorum sapientium
 et in proverbiis eorum conversare;
 \verse ab ipsis enim disces sapientiam et doctrinam intellectus
 et servire magnatis sine querela.
 \verse Non te praetereat narratio seniorum:
 ipsi enim didicerunt a patribus suis;
 \verse quoniam ab ipsis disces intellectum
 et in tempore necessitatis dare responsum.
 \verse Non incendas carbones peccatorum arguens eos
 et ne incendaris flamma ignis peccatorum illorum.
 \verse Ne contra faciem stes contumeliosi,
 ne sedeat quasi insidiator ori tuo.
 \verse Noli fenerari homini fortiori te;
 quod si feneraveris, quasi perditum habe.
 \verse Non spondeas super virtutem tuam;
 quod si spoponderis, quasi restituens cogita.
 \verse Non litiges contra iudicem,
 quoniam secundum placitum suum iudicat.
 \verse Cum audace non eas in via,
 ne forte aggraves mala tua:
 ipse enim secundum voluntatem suam vadit,
 et simul cum stultitia illius peries.
 \verse Cum iracundo non facias rixam
 et cum ipso non eas in desertum,
 quoniam quasi nihil est ante illum sanguis,
 et, ubi non est adiutorium, elidet te.
 \verse Cum fatuis consilium non habeas;
 non enim poterunt occultare secretum tuum.
 \verse Coram extraneo nihil facias cautum;
 nescis enim quid pariet.
 \verse Non omni homini cor tuum manifestes,
 ne forte repellas a te bonum.
 
\begin{biblechapter}
\verse Non zeles mulierem sinus tui,
 ne doceas contra te notitiam ne quam.
 \verse Non des mulieri potestatem animae tuae,
 ne ingrediatur in virtutem tuam, et confundaris.
 \verse Ne adeas ad mulierem multivolam, ne forte incidas in laqueos illius.
 \verse Cum psaltria ne assiduus sis nec audias illam,
 ne forte pereas in efficacitate illius.
 \verse Virginem ne conspicias,
 ne forte scandalizeris in decore illius.
 \verse Ne des fornicariis animam tuam in ullo,
 ne perdas te et hereditatem tuam.
 \verse Noli circumspicere in vicis civitatis,
 nec oberraveris in plateis illius.
 \verse Averte faciem tuam a muliere compta
 et ne circumspicias speciem alienam.
 \verse Propter speciem mulieris multi perierunt,
 et ex hoc concupiscentia quasi ignis exardescit. (\verse. \verse)
 \verse Cum alterius muliere ne sedeas omnino
 nec accumbas cum ea super cubitum in vino,
 \verse ne forte declinet cor tuum in illam, et sanguine tuo labaris in perditionem.
 \verse Ne derelinquas amicum antiquum:
 novus enim non erit similis illi.
 \verse Vinum novum amicus novus:
 veterascet, et cum suavitate bibes illud.
 \verse Non zeles gloriam et opes peccatoris;
 non enim scis quae futura sit illius subversio.
 \verse Non placeat tibi prosperitas iniustorum
 sciens quoniam usque ad inferos non iustificabuntur.
 \verse Longe abesto ab homine potestatem habente occidendi
 et non suspicaberis timorem mortis;
 \verse et, si accesseris ad illum, noli aliquid committere,
 ne forte auferat vitam tuam.
 \verse Communionem mortis scito,
 quoniam in medio laqueorum ingredieris
 et super retia ambulabis.
 \verse Secundum virtutem tuam conversare cum proximo tuo
 et cum sapientibus et prudentibus tracta.
 \verse Et cum sensato sit cogitatus tuus,
 et omnis enarratio tua in praeceptis Altissimi.
 \verse Viri iusti sint tibi convivae,
 et in timore Dei sit tibi gloriatio.
 \verse In manu artificum opera laudabuntur,
 et princeps populi in sapientia sermonis sui,
 in sensu vero seniorum verbum.
 \verse Terribilis est in civitate sua homo linguosus,
 et temerarius in verbo suo odibilis erit.
 
\begin{biblechapter}
\verse Iudex sapiens instituet populum suum,
 et principatus sensati stabilis erit.
 \verse Secundum iudicem populi sic et ministri eius,
 et qualis rector est civitatis, tales et inhabitantes in ea.
 \verse Rex insipiens perdet populum suum,
 et civitates inhabitabuntur per sensum potentium.
 \verse In manu Dei potestas terrae,
 et utilem rectorem suscitabit in tempus super illam.
 \verse In manu Dei prosperitas hominis,
 et super faciem scribae imponet honorem suum.
 \verse Pro omni iniuria proximi ne rependas
 et nihil agas in operibus superbiae.
 \verse Odibilis coram Deo est et hominibus superbia,
 et utrisque execrabilis omnis vexatio.
 \verse Regnum a gente in gentem transfertur
 propter iniustitias et contumelias et divitias dolosas.
 \verse Avaro autem nihil est scelestius,
 hic enim et animam suam venalem habet.
 \verse Quid superbit terra et cinis?
 Quoniam in vita sua proiecit intima sua.
 \verse Languor prolixior gravat medicum,
 brevis languor serenat medicum.
 \verse Omnis potentatus brevis vita,
 sic et rex hodie est et cras morietur.
 \verse Cum enim morietur homo,
 hereditabit serpentes et bestias et vermes.
 \verse Initium superbiae hominis apostatare a Deo;
 \verse et ab eo, qui fecit illum, recessit cor eius.
 Quoniam initium omnis peccati est superbia,
 qui tenuerit illam, ebulliet maledictum,
 et subvertet eum in finem.
 \verse Propterea mirabiles fecit Dominus plagas malorum
 et destruxit eos usque in finem.
 \verse Sedes ducum superborum destruxit Deus
 et sedere fecit mites pro eis.
 \verse Radices gentium superbarum eradicavit Deus
 et plantavit humiles pro ipsis.
 \verse Terras gentium evertit Dominus
 et perdidit eas usque ad fundamentum.
 \verse Arefecit ex ipsis et disperdidit eos
 et cessare fecit memoriam eorum a terra.
 \verse Memoriam superborum perdidit Deus
 et reliquit memoriam humilium sensu.
 \verse Non est creata hominibus superbia,
 neque iracundia nato mulierum.
 \verse Semen hominum honoratum hoc,
 quod timet Deum;
 semen autem hoc exhonorabitur,
 quod praeterit mandata Domini.
 \verse In medio fratrum rector illorum in honore;
 et, qui timent Dominum, erunt in oculis illius.
 \verse Peregrinus, advena et pauper:
 timor Dei est gloria eorum.
 \verse Noli despicere hominem iustum pauperem
 et noli magnificare virum peccatorem divitem.
 \verse Magnus et iudex et potens est in honore,
 sed non est maior illo, qui timet Deum.
 \verse Servo sensato liberi servient;
 et vir prudens et disciplinatus non murmurabit correptus.
 \verse Noli extollere te in faciendo opere tuo
 et noli gloriari in tempore angustiae tuae.
 \verse Melior est, qui operatur et abundat in omnibus,
 quam qui gloriatur et eget pane.
 \verse Fili, in mansuetudine honora animam tuam
 et da illi victum cultumque secundum meritum suum.
 \verse Peccantem in animam suam quis iustificabit?
 Et quis honorificabit exhonorantem animam suam?
 \verse Est pauper, qui honoratur propter disciplinam et timorem suum,
 et est homo, qui honorificatur propter substantiam suam.
 \verse Qui autem honoratur in paupertate, quanto magis in substantia!
 Et, qui exhonoratur in substantia, quanto magis in paupertate!
 
\begin{biblechapter}
\verse Sapientia humiliati exaltabit caput illius
 et in medio magnatorum consedere illum faciet.
 \verse Non laudes virum in specie sua
 neque spernas hominem deformem in visu suo.
 \verse Brevis in volatilibus est apis,
 et initium dulcoris habet fructus illius.
 \verse In vestitu ne glorieris umquam
 nec in die honoris tui extollaris,
 quoniam mirabilia opera Altissimi solius,
 et absconsa et invisa hominibus opera illius.
 \verse Multi tyranni sederunt in terra,
 et insuspicabilis portavit diadema.
 \verse Multi potentes exhonorati sunt valide,
 et gloriosi traditi sunt in manus alterorum.
 \verse Priusquam interroges, ne vituperes quemquam,
 sed, postquam interrogaveris, corripe iuste.
 \verse Priusquam audias, ne respondeas verbum
 et in medio sermonum ne adicias loqui.
 \verse De ea re, quae te non molestat, ne certeris
 et in iudicio peccantium ne consistas.
 \verse Fili, ne in multis sint actus tui;
 etsi festinaveris, non eris immunis a delicto:
 si enim persecutus fueris, non apprehendes,
 et non effugies, si discurreris.
 \verse Est homo laborans et festinans et dolens
 et tanto magis non abundabit.
 \verse Est homo marcidus egens susceptione,
 plus deficiens virtute et abundans paupertate;
 \verse et oculus Dei respexit illum in bono
 et erexit eum ab humilitate ipsius
 et exaltavit caput eius:
 et mirati sunt in illo multi.
 \verse Bona et mala, vita et mors,
 paupertas et honestas a Deo sunt.
 \verse Sapientia et disciplina et scientia legis apud Dominum;
 dilectio et viae bonorum apud ipsum.
 \verse Error et tenebrae peccatoribus concreata sunt;
 qui autem exsultant in malis, consenescunt in malo.
 \verse Datio Dei permanet iustis,
 et beneplacitum illius successus habebit in aeternum.
 \verse Est qui locupletatur parce agendo,
 et haec est pars mercedis illius
 \verse in eo quod dicit: “ Inveni requiem mihi,
 et nunc manducabo de bonis meis solus”;
 \verse et nescit quod tempus praeteriet, et mors appropinquet,
 et relinquet omnia aliis et morietur.
 \verse Sta in mandato tuo et in illo conversare
 et in opere mandatorum tuorum veterasce.
 \verse Ne mireris in operibus peccatorum;
 confide autem in Deo et mane in labore tuo.
 \verse Facile est enim in oculis Dei subito honestare pauperem.
 \verse Benedictio Dei in mercede iusti continuo,
 et in hora veloci successus illius fructificat.
 \verse Ne dicas: “ Quid est mihi opus?
 Et, quae erunt mihi ex hoc bona? ”.
 \verse Ne dicas: “ Sufficiens mihi sum;
 et, quid ex hoc nunc pessimabor? ”
 \verse In die bonorum ne immemor sis malorum
 et in die malorum ne immemor sis bonorum,
 \verse quoniam facile est coram Deo in die obitus
 retribuere unicuique secundum vias suas.
 \verse Malitia horae oblivionem facit luxuriae magnae,
 et in fine hominis denudatio operum illius.
 \verse Ante mortem ne laudes hominem quemquam,
 quoniam in extremis suis agnoscitur vir.
 \verse Non omnem hominem inducas in domum tuam,
 multae enim sunt insidiae dolosi.
 \verse Sicut enim eructant praecordia foetentium,
 et sicut perdix inducitur in caveam,
 et ut caprea in laqueum, sic et cor superborum,
 et sicut prospector videns casum proximi sui.
 \verse Bona enim in mala convertens insidiator
 et in electis imponet maculam.
 \verse A scintilla una augentur carbones,
 et ab uno doloso augetur sanguis;
 homo vero peccator sanguini insidiatur.
 \verse Attende tibi a pestifero, fabricat enim mala;
 ne forte inducat super te maculam in perpetuum.
 \verse Admitte ad te alienigenam, et subvertet te in turbore
 et abalienabit te a tuis propriis.
 
\begin{biblechapter}
\verse Si benefeceris, scito cui feceris,
 et erit gratia in bonis tuis multa.
 \verse Benefac iusto et invenies retributionem magnam
 et, si non ab ipso, certe a Domino.
 \verse Non est enim ei bene, qui assiduus est in malis
 et eleemosynas non dat,
 quoniam et Altissimus odio habet peccatores
 et misertus est paenitentibus.
 \verse Da misericordi et ne suscipias peccatorem;
 et impiis et peccatoribus reddet vindictam
 custodiens eos in diem vindictae.
 \verse Da bono et non receperis peccatorem.
 \verse Benefac humili et non dederis impio;
 vasa belli ne dederis illi,
 ne in ipsis potentior te sit.
 \verse Nam duplicia mala invenies
 in omnibus bonis quaecumque feceris illi,
 quoniam et Altissimus odio habet peccatores
 et impiis reddet vindictam.
 \verse Non agnoscetur in bonis amicus,
 et non abscondetur in malis inimicus.
 \verse In bonis viri etiam inimici illius sunt amici,
 et in malis etiam amicus discedit.
 \verse Non credas inimico tuo in aeternum,
 sicut enim aeramentum aeruginat nequitia illius
 \verse et, si humiliatus vadat curvus,
 adice animum tuum et custodi te ab illo
 et fias ei sicut qui extergit speculum,
 et cognosces quoniam in finem aeruginavit.
 \verse Non statuas illum penes te,
 nec sedeat ad dexteram tuam,
 ne forte conversus in locum tuum
 inquirat cathedram tuam;
 et in novissimo agnoscas verba mea et in sermonibus meis stimuleris.
 \verse Quis miserebitur incantatori a serpente percusso
 et omnibus, qui appropiant bestiis?
 Et sic qui comitatur cum viro iniquo
 et obvolutus est in peccatis eius:
 non evadet, donec incendat eum ignis.
 \verse Una hora tecum permanebit;
 si autem declinaveris, non supportabit.
 \verse In labiis suis indulcat inimicus
 et in corde suo insidiatur, ut subvertat te in foveam.
 \verse In oculis suis lacrimatur inimicus
 et, si invenerit opportunitatem, non satiabitur sanguine.
 \verse Si incurrerint tibi mala,
 invenies eum illic priorem,
 \verse et quasi adiuvans suffodiet plantas tuas.
 \verse Caput suum movebit et plaudet manu
 et multa susurrans commutabit vultum suum.
 
\begin{biblechapter}
\verse Qui tetigerit picem, inquinabitur ab ea;
 et, qui communicaverit superbo, induet superbiam.
 \verse Pondus super te ne tollas
 et honestiori et ditiori te ne socius fueris.
 \verse Quid communicabit caccabus ad ollam?
 Quando enim se colliserint, confringetur.
 \verse Dives iniuste egit et fremet,
 pauper autem laesus, ipse supplicabit.
 \verse Si utilis fueris, assumet te
 et, si non habueris, derelinquet te.
 \verse Si habes, convivet tecum et evacuabit te
 et ipse non dolebit super te.
 \verse Si necessarius illi fueris, ludet te
 et subridens spem dabit narrans tibi bona
 et dicet: “ Quid opus est tibi? ”.
 \verse Et confundet te in cibis suis,
 donec te exinaniat bis et ter
 et in novissimo deridebit te;
 et postea videns derelinquet te
 et caput suum movebit ad te.
 \verse Humiliare Deo et exspecta manus eius.
 \verse Attende, ne seductus in stultitiam humilieris.
 \verse Noli esse humilis in sapientia tua,
 ne humiliatus in stultitiam seducaris.
 \verse Advocatus a potentiore discede,
 et eo magis te advocabit.
 \verse Ne accedas, ne impingaris;
 et ne longe sis ab eo, ne eas in oblivionem.
 \verse Ne retineas ex aequo loqui cum illo
 nec credas multis verbis illius;
 ex multa enim loquela tentabit te
 et subridens inquiret de absconditis tuis.
 \verse Immitis animus illius conservabit verba tua
 et non parcet de malitia et de vinculis.
 \verse Cave tibi et attende diligenter auditui tuo,
 quoniam cum subversione tua ambulas.
 \verse Audiens vero illa
 ex somno evigila.
 \verse Omni vita tua dilige Deum
 et invoca illum in salutem tuam.
 \verse Omne animal diligit simile sibi:
 sic et omnis homo proximum sibi.
 \verse Omnis caro ad similem sibi coniungetur,
 et omnis homo simili sui sociabitur.
 \verse Quid communicabit lupus agno?
 Sic peccator iusto.
 \verse Quae pax hyaenae ad canem?
 Aut quae pars diviti ad pauperem?
 \verse Venatio leonis onager in eremo,
 sic et pascua divitum sunt pauperes.
 \verse Et sicut abominatio est superbo humilitas,
 sic et exsecratio divitis pauper.
 \verse Dives commotus confirmatur ab amicis suis,
 humilis autem, cum ceciderit, expelletur et a notis.
 \verse Diviti decepto multi recuperatores:
 locutus est nefaria, et iustificaverunt illum;
 \verse humilis deceptus est, insuper et arguitur:
 locutus est sensate, et non est datus ei locus.
 \verse Dives locutus est, et omnes tacuerunt,
 et verbum illius usque ad nubes perducent;
 \verse pauper locutus est, et dicunt: “ Quis est hic? ”
 et, si offenderit, insuper subvertent illum.
 \verse Bona est substantia, cui non est peccatum in conscientia,
 et nequissima paupertas in ore impii.
 \verse Cor hominis immutat faciem illius
 sive in bona sive in mala.
 \verse Vestigium cordis boni facies hilaris:
 difficile invenies et cum labore.
 
\begin{biblechapter}
\verse Beatus vir, qui non est lapsus verbo ex ore suo
 et non est stimulatus in tristitia delicti.
 \verse Felix, quem non condemnat anima sua,
 et non excidit a spe sua.
 \verse Viro tenaci sine ratione est substantia;
 et homini livido ad quid aurum?
 \verse Qui denegat animo suo iniuste, aliis congregat,
 et in bonis illius alius luxuriabitur.
 \verse Qui sibi nequam est, cui alii bonus erit?
 Et non iucundabitur in bonis suis.
 \verse Qui sibi invidet, nihil est illo nequius;
 et haec redditio est malitiae illius.
 \verse Et, si bene fecerit, ignoranter et non volens facit
 et in novissimo manifestat malitiam suam.
 \verse Nequam est oculus lividi
 et avertens faciem suam et despiciens animas.
 \verse Insatiabilis oculus cupidi in parte non satiabitur,
 donec consumat arefaciens animam suam.
 \verse Oculus malus lividus irruit in panem
 et neglegens est mensae suae.
 \verse Fili, si habes, benefac tecum
 et Deo dignas oblationes offer.
 \verse Memor esto quoniam mors non tardat,
 et decretum inferorum quia non demonstratum est tibi;
 decretum enim huius mundi: morte morietur.
 \verse Ante mortem benefac amico tuo
 et secundum vires tuas exporrigens da ei.
 \verse Non defrauderis a bono diei,
 et particula desiderii boni non te praetereat.
 \verse Nonne aliis relinques res dolore partas
 et labores tuos in divisione sortis?
 \verse Da et accipe et oblecta animam tuam;
 \verse ante obitum tuum operare iustitiam,
 quoniam non est apud inferos quaerere voluptates.
 \verse Omnis caro sicut vestimentum veterascet
 et sicut folium fructificans in arbore viridi:
 alia generantur, et alia deiciuntur;
 \verse sic generatio carnis et sanguinis:
 alia finitur, et alia nascitur.
 \verse Omne opus corruptibile in fine deficiet,
 et, qui illud operatur, ibit cum illo;
 \verse et omne opus electum iustificabitur,
 et, qui operatur illud, honorabitur in illo.
 \verse Beatus vir, qui in sapientia morabitur
 et qui in iustitia sua meditabitur
 et in sensu cogitabit circumspectionem Dei;
 \verse qui excogitat vias illius in corde suo
 et in absconditis suis intellegens,
 vadens post illam quasi investigator
 et in viis illius consistens;
 \verse qui respicit per fenestras illius
 et in ianuis illius audiens;
 \verse qui requiescit iuxta domum illius
 et in parietibus illius figens palum,
 statuet casulam suam ad manus illius
 et requiescet in deversorio bonorum per aevum.
 \verse Statuet filios suos sub tegmine illius
 et sub ramis eius morabitur;
 \verse protegetur sub tegmine illius a fervore
 et in gloria eius requiescet.
 
\begin{biblechapter}
\verse Qui timet Deum, faciet haec,
 et, qui continens est legis, apprehendet illam;
 \verse et obviabit illi quasi mater honorificata
 et quasi mulier a virginitate suscipiet illum.
 \verse Cibabit illum pane vitae et intellectus
 et aqua sapientiae salutaris potabit illum,
 et firmabitur in illa et non flectetur
 \verse et confidet in illam et non confundetur;
 et exaltabit illum prae proximis suis
 \verse et in medio ecclesiae aperiet os eius
 et adimplebit illum spiritu sapientiae et intellectus
 et stola gloriae vestiet illum;
 \verse iucunditatem et exsultationem thesaurizabit super illum
 et nomine aeterno hereditabit illum.
 \verse Homines stulti non apprehendent illam,
 et homines sensati obviabunt illi;
 homines peccatores non videbunt eam,
 longe enim abest a superbia et dolo.
 \verse Viri mendaces non erunt illius memores;
 et viri veraces invenientur in illa
 et successum habebunt usque ad inspectionem Dei.
 \verse Non est speciosa laus in ore peccatoris,
 \verse quoniam non a Deo tributa est ei;
 sapientiae enim Dei astabit laus,
 et in ore sapientis dicetur laus,
 et dominator illius docebit eam.
 \verse Ne dixeris: “ A Deo peccatum meum ”
 quae enim odit, ipse non facit.
 \verse Non dicas: “ Ille in me impegit ”;
 non enim necessarii sunt ei homines impii.
 \verse Omne exsecramentum erroris odit Dominus,
 et non erit amabile timentibus eum.
 \verse Deus ab initio constituit hominem
 et reliquit illum in manu consilii sui
 et dedit eum in manum concupiscentiae suae.
 \verse Adiecit mandata et praecepta sua
 et intellegentiam ad faciendum placitum eius.
 \verse Si volueris mandata servare, conservabunt te;
 si confidis in illo, etiam tu vives.
 \verse Apposuit tibi aquam et ignem;
 ad quod volueris, porrige manum tuam.
 \verse Ante hominem vita et mors, bonum et malum:
 quod placuerit ei, dabitur illi.
 \verse Quoniam multa sapientia Dei,
 et fortis in potentia videns omnes sine intermissione.
 \verse Oculi Domini ad timentes eum,
 et ipse agnoscit omnem operam hominis.
 \verse Nemini mandavit impie agere
 et nemini dedit spatium peccandi.
 \verse Non concupiscas multitudinem filiorum infidelium et iniquorum.
 
\begin{biblechapter}
\verse Ne iucunderis in filiis impiis; si multiplicentur, non oblecteris super ipsos,
 si non est timor Dei cum illis.
 \verse Non credas vitae illorum
 et ne respexeris in labores eorum.
 \verse Melior est enim unus timens Deum
 quam mille filii impii;
 \verse et potius est mori sine filiis
 quam relinquere filios impios.
 \verse Ab uno sensato inhabitabitur patria,
 tribus autem impiorum deseretur.
 \verse Multa talia vidit oculus meus,
 et fortiora horum audivit auris mea.
 \verse In synagoga peccantium exardebit ignis,
 et in gente incredibili exardescet ira.
 \verse Non exoraverunt eum antiqui gigantes,
 qui rebelles fuerunt confidentes suae virtuti.
 \verse Et non pepercit accolis Lot
 et exsecratus est eos prae superbia verbi illorum;
 \verse non misertus est gentis anathematis,
 qui depulsi sunt in peccatis suis.
 \verse Et sicut sescenta milia peditum,
 qui congregati sunt in duritia cordis sui;
 et, si unus fuisset cervicatus,
 mirum si fuisset immunis:
 \verse misericordia enim et ira est cum illo,
 sustinens, exorabilis et effundens iram.
 \verse ecundum multam misericordiam suam, sic et correptio illius:
 hominem secundum opera sua iudicat.
 \verse Non effugiet in rapina peccator,
 et non irrita erit sustinentia iusti.
 \verse Omni misericordiae erit merces:
 unusquisque secundum meritum operum suorum inveniet coram se
 et secundum intellectum peregrinationis ipsius.
 Dominus induravit cor pharaonis, ne agnosceret illum,
 ut opera sua innotescerent sub caelo.
 Misericordia eius apparuit omnibus creaturis eius,
 lucem suam et tenebras dispertiit filiis hominum.
 \verse Non dicas: “ A Deo abscondar!
 Et, ex summo quis mei memorabitur?
 \verse In populo magno non agnoscar;
 quae est enim anima mea in tam immensa creatura? ”.
 \verse Ecce caelum et caeli caelorum,
 abyssus et universa terra et quae in eis sunt,
 in visitatione illius commovebuntur;
 \verse montes simul et colles et fundamenta terrae,
 cum conspexerit illa Deus, tremore concutientur.
 \verse Et in omnibus his non apponet cor,
 etenim omne cor intellegitur ab illo.
 \verse Et vias illius quis intellegit
 et procellam, quam nec oculus videbit hominis?
 \verse Nam plurima illius opera sunt in absconsis,
 et opera iustitiae eius quis enuntiabit, aut quis sustinebit?
 Longe enim est decretum,
 et interrogatio omnium in consummatione est.
 \verse Qui minoratur corde, cogitat ista,
 et vir imprudens et errans cogitat stulta.
 \verse Audi me, fili, et disce prudentiam sensus,
 \verse et dicam in aequitate disciplinam
 et scrutabor enarrare sapientiam;
 et in verbis meis attende in corde tuo.
 Edico in aequitate spiritus virtutes,
 quas posuit Deus in opera sua ab initio,
 et in veritate enuntio scientiam eius.
 \verse Quando creavit Deus opera sua ab initio
 et ab institutione ipsorum distinxit partes illorum,
 \verse ornavit in aeternum opera illorum
 et dominatum eorum in generationibus suis.
 Nec esurierunt nec laboraverunt
 et non destiterunt ab operibus suis.
 \verse Unusquisque proximum sibi non angustiavit,
 \verse et usque in aeternum non erunt incredibiles verbo illius.
 \verse Post haec Deus in terram respexit
 et complevit illam bonis suis;
 \verse anima omnis vitalis cooperuit faciem ipsius,
 et in ipsam iterum reversio illorum.
 
\begin{biblechapter}
\verse Deus creavit de terra hominem
 et secundum imaginem suam fecit illum;
 \verse et iterum convertit illum in ipsam
 et secundum se vestivit illum virtute.
 \verse Numerum dierum et tempus dedit illi
 et dedit illi potestatem eorum, quae sunt super terram.
 \verse Posuit timorem illius super omnem carnem,
 ut dominaretur bestiarum et volatilium.
 \verse Creavit illis consilium et linguam et oculos et aures
 et cor dedit illis excogitandi
 et disciplina intellectus replevit illos.
 \verse Creavit illis scientiam spiritus,
 sensu implevit cor illorum
 et mala et bona ostendit illis.
 \verse Posuit timorem suum super corda illorum
 ostendens illis magnalia operum suorum
 \verse et dedit illis gloriari in mirabilibus illius,
 ut nomen sanctificationis collaudent
 et magnalia enarrent operum eius.
 \verse Addidit illis disciplinam
 et legem vitae hereditavit illos.
 \verse Testamentum aeternum constituit cum illis
 et iustitiam et iudicia sua ostendit illis.
 \verse Et magnalia honoris eius vidit oculus illorum,
 et honorem vocis eius audierunt aures illorum,
 et dixit illis: “ Attendite ab omni iniquo ”.
 \verse Et mandavit illis unicuique de proximo suo.
 \verse Viae illorum coram ipso sunt semper:
 non sunt absconsae ab oculis ipsius.
 \verse In unamquamque gentem praeposuit rectorem,
 \verse et pars Dei Israel factus est.
 \verse Et omnia opera illorum velut sol in conspectu eius;
 et oculi eius sine intermissione inspicientes in viis eorum.
 \verse Non sunt absconsae iniquitates illorum
 et omnia peccata eorum in conspectu Dei.
 \verse Eleemosyna viri quasi signaculum cum ipso,
 et gratiam hominis quasi pupillam conservabit.
 \verse Et postea resurget et retribuet illis
 et retributionem unicuique in caput ipsorum convertet.
 \verse Paenitentibus autem dedit viam reditus
 et confirmavit deficientes sustinere
 et destinavit illis sortem veritatis.
 \verse Convertere ad Dominum et relinque peccata tua;
 \verse precare ante faciem Domini et minue offendicula.
 \verse Revertere ad Altissimum et avertere ab iniustitia tua
 et nimis odito exsecrationem.
 \verse Et cognosce iustitias et iudicia Dei
 et sta in sorte propositionis et orationis altissimi Dei.
 \verse Altissimum quis laudabit in inferis
 pro vivis et dantibus confessionem Deo?
 \verse Non demoreris in errore impiorum;
 ante mortem confitere:
 a mortuo, quasi nihil sit, perit confessio.
 \verse Confiteberis vivens, vivus et sanus confiteberis
 et laudabis Deum et gloriaberis in miserationibus illius.
 \verse Quam magna misericordia Domini,
 et propitiatio illius convertentibus ad se!
 \verse Nec enim omnia possunt esse in hominibus,
 quoniam non est immortalis filius hominis.
 \verse Quid lucidius sole? Et hic deficiet.
 Aut quid nequius quam quod excogitavit caro et sanguis?
 Et hoc arguetur.
 \verse Virtutem altitudinis caeli ipse conspicit,
 et omnes homines terra et cinis.
 
\begin{biblechapter}
\verse Qui vivit in aeternum, creavit omnia simul.
 Deus solus iustificabitur et manet invictus rex in aeternum.
 \verse Quis sufficit enarrare opera illius?
 \verse Et quis investigabit magnalia eius?
 \verse Virtutem autem magnitudinis eius quis enuntiabit?
 Aut quis adiciet enarrare misericordiam eius?
 \verse Non est minuere neque adicere
 nec est invenire magnalia Dei;
 \verse cum consummaverit homo, tunc incipiet
 et, cum quieverit, aporiabitur.
 \verse Quid est homo, quis defectus, et quae est utilitas illius?
 Et quid est bonum, aut quid nequam illius?
 \verse Numerus dierum hominum ut multum centum anni,
 quasi gutta aquae maris deputati sunt,
 et sicut calculus arenae, sic exigui anni in die aevi.
 \verse Propter hoc patiens est Deus in illis
 et effundit super eos misericordiam suam.
 \verse Vidit praesumptionem cordis eorum, quoniam mala est;
 et cognovit subversionem illorum, quoniam nequam est.
 \verse Ideo adimplevit propitiationem suam in illis
 et ostendit eis viam aequitatis.
 \verse Miseratio hominis circa proximum suum,
 misericordia autem Dei super omnem carnem.
 \verse Qui reprehendit, docet et erudit
 quasi pastor dirigens gregem suum.
 \verse Miseretur excipientibus doctrinam miserationis
 et festinantibus in iudiciis eius.
 \verse Fili, in bonis non des querelam
 et in omni dato non des tristitiam verbi mali.
 \verse Nonne ardorem refrigerabit ros?
 Sic et verbum melius quam datum.
 \verse Nonne ecce verbum super datum bonum?
 Sed utraque cum homine gratioso.
 \verse Stultus acriter improperabit,
 et datus indisciplinati tabescere facit oculos.
 \verse Ante iudicium para advocatum tibi
 et, antequam loquaris, disce.
 \verse Ante languorem adhibe medicinam
 et ante iudicium interroga teipsum
 et in hora visitationis invenies propitiationem.
 \verse Ante languorem humilia te
 et in tempore peccati ostende conversionem tuam.
 \verse Non sinas te impediri reddere votum tempore opportuno
 et ne tardes usque ad mortem iustificari,
 quoniam merces Dei manet in aeternum.
 \verse Ante votum praepara animam tuam
 et noli esse quasi homo, qui tentat Dominum.
 \verse Memento irae in die consummationis
 et, suo tempore, retributionis in conversione faciei.
 \verse Memento famis in tempore abundantiae
 et necessitatum paupertatis in die divitiarum.
 \verse A mane usque ad vesperam mutatur tempus,
 et haec omnia citata in oculis Dei.
 \verse Homo sapiens in omnibus metuet
 et in diebus delictorum cavebit a malitia.
 \verse Omnis astutus agnoscit sapientiam
 et invenienti eam dabit confessionem.
 \verse Sensati in verbis et ipsi sapienter egerunt
 et intellexerunt veritatem et iustitiam
 et effuderunt tamquam pluviam proverbia et iudicia.
 \verse De continentia animae.
 Post concupiscentias tuas non eas
 et a voluptatibus tuis te contine;
 \verse si praestes animae tuae beneplacitum concupiscentiae,
 faciet te in gaudium inimicis tuis.
 \verse Ne oblecteris in multa epulatione;
 duplex enim portio est paupertas illius.
 \verse Ne fueris ganeo et potator,
 cum nihil tibi est in sacculo:
 eris enim invidus vitae tuae.
 
\begin{biblechapter}
\verse Operarius ebriosus non locupletabitur;
 et, qui spernit modica, paulatim decidet.
 \verse Vinum et mulieres apostatare faciunt sensatos;
 et, qui se iungit fornicariis, peribit:
 putredo et vermes hereditabunt illum.
 \verse Anima audax perdet dominum suum;
 et tolletur de numero anima eius,
 et extolletur in exemplum maius.
 \verse Qui credit cito, levis corde est et minorabitur;
 et, qui delinquit in animam suam, quis innoxium faciet?
 \verse Qui gaudet iniquitate, denotabitur,
 et, qui odit correptionem, minuetur vita,
 et, qui odit loquacitatem, exstinguit malitiam. (\verse)
 \verse Ne umquam iteres verbum nequam et durum
 et prorsus non minoraberis.
 \verse De amico et inimico noli narrare
 et, si notum est tibi delictum, noli denudare:
 \verse audiet enim te et cavebit te
 et quasi defendens peccatum odiet te.
 \verse Audisti verbum adversus proximum tuum?
 Commoriatur in te fidens quoniam non te dirumpet.
 \verse A facie verbi parturiet fatuus
 tamquam parturiens a facie infantis;
 \verse sagitta infixa femori carnis,
 sic verbum in corde stulti.
 \verse Corripe amicum, ne forte fecerit malum et ipse dicat: “ Non feci ”;
 aut, si fecerit, ne iterum addat facere.
 \verse Corripe proximum, ne forte dixerit
 et, si dixerit, ne forte iteret.
 \verse Corripe amicum, saepe enim fit criminatio,
 \verse et non omni verbo credas.
 Est qui labitur lingua sed non ex animo;
 \verse quis est enim qui non deliquerit in lingua sua?
 Corripe proximum, antequam commineris,
 \verse et da locum legi Altissimi.
 Quia omnis sapientia timor Dei et in illa timere Deum,
 et in omni sapientia dispositio legis.
 \verse Et non est sapientia nequitiae scientia,
 et non est consilium peccatorum prudentia.
 \verse Est astutia et ipsa exsecratio,
 et est insipiens, qui minuitur sapientia.
 \verse Melior est homo, qui minuitur sapientia et deficiens sensu in timore,
 quam qui abundat sensu et transgreditur legem Altissimi.
 \verse Est solertia certa et ipsa iniqua.
 \verse Et est qui pervertit gratiam, ut proferat iudicium;
 est qui videtur oppressus et fractus animo,
 et interiora eius plena sunt dolo.
 \verse Et est qui se nimium submittit a multa humilitate;
 et est qui inclinat faciem suam
 et fingit se non audire:
 ubi ignoratus est, praeveniet te.
 \verse Et, si ab imbecillitate virium vetetur peccare,
 si invenerit tempus malefaciendi, malefaciet.
 \verse Ex visu cognoscitur vir,
 et ab occursu faciei cognoscitur sensatus:
 \verse amictus corporis et risus dentium
 et gressus hominis enuntiant de illo.
 \verse Est correptio inopportuna,
 et est indicium, quod non probatur esse bonum;
 et est tacens, et ipse est prudens.
 
\begin{biblechapter}
\verse Quam bonum est arguere quam irasci,
 et confitentem in oratione non prohibere!
 \verse Concupiscentia spadonis devirginans iuvenculam:
 \verse sic qui facit per vim iudicium iniquum.
 \verse Quam bonum est correptum manifestare paenitentiam!
 Sic enim effugies voluntarium peccatum.
 \verse Est tacens, qui invenitur sapiens,
 et est odibilis, quia procax est ad loquendum.
 \verse Est tacens non habens responsum,
 et est tacens sciens tempus aptum.
 \verse Homo sapiens tacebit usque ad tempus,
 lascivus autem et imprudens non servabunt tempus.
 \verse Qui multis utitur verbis, exsecrabitur;
 et, qui potestatem sibi assumit iniuste, odietur.
 \verse Est processus in malis viro indisciplinato,
 et est inventio in detrimentum.
 \verse Est datum, quod non est utile,
 et est datum, cuius retributio duplex.
 \verse Est propter gloriam minoratio,
 et est qui ab humilitate levat caput.
 \verse Est qui multa redimat modico pretio
 et restituens ea in septuplum.
 \verse Sapiens in verbis seipsum amabilem facit,
 gratiae autem fatuorum effundentur.
 \verse Datum insipientis non erit utile tibi,
 oculi enim illius septemplices sunt:
 \verse exigua dabit et multa improperabit,
 et apertio oris illius quasi clamantis.
 \verse Hodie feneratur quis et cras expetit:
 odibilis est homo huiusmodi.
 \verse Fatuus dicit: “ Non est mihi amicus,
 et non est gratia bonis meis ”.
 \verse Qui enim edunt panem illius, falsae linguae sunt.
 Quoties et quanti irridebunt eum!
 \verse Neque enim, quod habendum erat, directo sensu distribuit,
 similiter et, quod non erat habendum, est indifferens ei.
 \verse Melius lapsus in pavimento quam lapsus linguae:
 sic casus malorum festinanter veniet.
 \verse Homo acharis quasi fabula importuna;
 in ore indisciplinatorum assidua erit.
 \verse Ex ore fatui reprobabitur parabola,
 non enim dicit illam in tempore suo.
 \verse Est qui vetatur peccare prae inopia, et in requie sua non stimulabitur.
 \verse Est qui perdit animam suam prae confusione,
 et ab imprudenti persona perdet eam;
 personae autem acceptione perdet se.
 \verse Est qui prae confusione promittit amico,
 et lucratus est eum inimicum gratis.
 \verse Opprobrium nequam in homine mendacium,
 et in ore indisciplinatorum assidue erit.
 \verse Potior fur quam assiduitas viri mendacis;
 perditionem autem ambo hereditabunt.
 \verse Mos hominis mendacis est sine honore,
 et confusio illius cum ipso sine intermissione.
 \verse Verbum parabolarum.
 Sapiens in verbis producet seipsum,
 et homo prudens placebit magnatis.
 \verse Qui operatur terram suam, inaltabit acervum frugum,
 et, qui operatur iustitiam, ipse exaltabitur;
 qui vero placet magnatis, effugiet iniquitatem.
 \verse Xenia et dona excaecant oculos iudicum
 et quasi camus in ore avertunt correptiones eorum.
 \verse Sapientia absconsa et thesaurus invisus,
 quae utilitas in utrisque?
 \verse Melior est, qui celat insipientiam suam,
 quam homo, qui abscondit sapientiam suam.
 
\begin{biblechapter}
\verse Fili, peccasti? Non adicias iterum,
 sed et de pristinis deprecare, ut tibi dimittantur.
 \verse Quasi a facie colubri fuge peccata:
 et, si accesseris ad illa, mordebunt te.
 \verse Dentes leonis dentes eorum
 interficientes animas hominum.
 \verse Quasi romphaea bis acuta omnis iniquitas:
 plagae illius non est sanitas.
 \verse Terror et iniuriae annullabunt substantiam,
 et domus, quae nimis locuples est, annullabitur superbia;
 sic substantia superbi eradicabitur.
 \verse Deprecatio pauperis ex ore usque ad aures Dei perveniet,
 et iudicium festinato adveniet illi.
 \verse Qui odit correptionem, in vestigio est peccatoris;
 et, qui timet Deum, convertet illam ad cor suum.
 \verse Notus a longe potens lingua audaci,
 et sensatus novit illum labi.
 \verse Qui aedificat domum suam impendiis alienis,
 quasi qui colligit lapides suos in hiemem.
 \verse Stuppa collecta synagoga peccantium,
 et consummatio illorum flamma ignis.
 \verse Via peccantium complanata lapidibus,
 et in fine illius fovea inferi.
 \verse Qui custodit legem, continebit sensum suum;
 \verse consummatio timoris Dei sapientia et sensus.
 \verse Non erudietur, qui non est prudens;
 \verse est autem astutia, quae abundat in malo,
 et non est sensus, ubi est amaritudo.
 \verse Scientia sapientis tamquam inundatio abundabit,
 et consilium illius sicut fons vitae permanet.
 \verse Cor fatui quasi vas confractum
 et omnem sapientiam non tenebit.
 \verse Verbum sapiens, quodcumque audierit scius,
 laudabit et ad illud adiciet;
 audivit luxuriosus, et displicebit illi
 et proiciet illud post dorsum suum.
 \verse Narratio fatui quasi sarcina in via,
 sed in labiis sensati invenietur gratia.
 \verse Os prudentis quaeretur in ecclesia,
 et verba illius cogitabunt in cordibus suis.
 \verse Tamquam domus exterminata sic fatuo sapientia;
 et scientia insensati inenarrabilia verba.
 \verse Compedes in pedibus stulto doctrina
 et quasi vincula manuum super manum dexteram.
 \verse Fatuus in risu exaltat vocem suam;
 vir autem sapiens vix tacite ridebit.
 \verse Tamquam ornamentum aureum prudenti doctrina
 et quasi brachiale in brachio dextro.
 \verse Pes fatui facilis in domum proximi,
 sed homo peritus verebitur personam.
 \verse Stultus a fenestra respiciet in domum,
 vir autem eruditus foris stabit.
 \verse Ineruditio hominis auscultare per ostium,
 et prudenti gravis contumelia.
 \verse Labia imprudentium stulta narrabunt,
 verba autem prudentium statera ponderabuntur.
 \verse In ore fatuorum cor illorum,
 et in corde sapientium os illorum.
 \verse Dum maledicit impius adversarium,
 maledicit ipse animam suam.
 \verse Susurro coinquinabit animam suam et in omnibus odietur;
 et, qui cum eo manserit, odiosus erit:
 tacitus et sensatus honorabitur.
 
\begin{biblechapter}
\verse Lapidi luteo comparatus est piger,
 et omnes sibilabunt super aspernationem illius;
 \verse fimo boum comparatus est piger:
 et omnis, qui tetigerit eum, excutiet manus.
 \verse Confusio patris est de filio indisciplinato,
 filia autem in deminorationem generatur.
 \verse Filia prudens hereditas viro suo,
 nam, quae confundit, in contumeliam fit genitoris.
 \verse Patrem et virum confundit filia audax,
 ab utrisque autem inhonorabitur.
 \verse Musica in luctu importuna narratio;
 disciplina et doctrina in omni tempore sapientia.
 \verse Qui docet fatuum, quasi qui conglutinat testam;
 \verse qui narrat verbum non audienti,
 quasi qui excitat dormientem de gravi somno.
 \verse Cum dormiente loquitur, qui enarrat stulto sapientiam,
 et in fine narrationis dicit: “ Quis est hic? ”.
 \verse Supra mortuum plora, defecit enim lux,
 et supra fatuum plora, defecit enim sensus.
 \verse Modicum plora supra mortuum, quoniam requievit;
 \verse nequissima enim vita fatui super mortem.
 \verse Luctus mortui septem dies,
 fatui autem et impii omnes dies vitae illorum.
 \verse Cum stulto ne multum loquaris
 et cum insensato ne abieris.
 \verse Serva te ab illo, ut non molestiam habeas,
 et non coinquinaberis impactione illius.
 \verse Deflecte ab illo et invenies requiem
 et non acediaberis in stultitia illius.
 \verse Super plumbum quid gravius?
 Et quod illi aliud nomen quam fatuus?
 \verse Arenam et salem et massam ferri facilius est ferre
 quam hominem imprudentem et fatuum et impium.
 \verse Loramentum ligneum colligatum in fundamento aedificii
 non dissolvetur;
 sic et cor confirmatum in cogitatione consilii,
 nullus timor illud commovebit.
 \verse Cor firmatum in cogitatu intellegentiae
 sicut ornatus in pariete polito.
 \verse Sicut pali in excelsis et caementa sine impensa posita
 contra faciem venti non permanebunt,
 \verse sic et cor timidum in cogitatione stulti
 contra impetum timoris non resistet. (\verse)
 \verse Pungens oculum deducit lacrimas,
 et, qui pungit cor, pellit amicitiam.
 \verse Mittens lapidem in volatilia fugat illa;
 sic et qui conviciatur amico, dissolvit amicitiam.
 \verse Ad amicum etsi produxeris gladium,
 non desperes: est enim regressus;
 ad amicum 
 \verse si aperueris os triste,
 non timeas: est enim concordatio,
 excepto convicio et improperio et superbia
 et mysterii revelatione et plaga dolosa;
 in his omnis effugiet amicus.
 \verse Fidem posside cum amico in paupertate illius,
 ut et in bonis illius communices;
 \verse in tempore tribulationis illius permane illi fidelis,
 ut et in hereditate illius coheres sis.
 \verse Ante ignem camini vapor et fumus,
 sic et ante sanguinem maledicta et contumeliae et minae.
 \verse Amicum tegere non confundar,
 a facie illius non me abscondam;
 et, si mala mihi evenerint per illum, sustinebo:
 \verse omnis, qui audiet, cavebit se ab eo.
 \verse Quis dabit ori meo custodiam
 et super labia mea signaculum aptum,
 ut non cadam ab ipsis, et lingua mea perdat me?
 
\begin{biblechapter}
\verse Domine, pater et dominator vitae meae,
 ne derelinquas me in consilio eorum
 nec sinas me cadere in illis.
 \verse Quis superponet in cogitatu meo flagella
 et in corde meo doctrinam sapientiae,
 ut ignorationibus meis non parcant mihi,
 et non appareant delicta mea,
 \verse et ne adincrescant ignorantiae meae,
 et multiplicentur delicta mea, et peccata mea abundent,
 et incidam in conspectu adversariorum meorum,
 et gaudeat super me inimicus meus?
 \verse Domine, pater et Deus vitae meae,
 ne derelinquas me in cogitatu illorum.
 \verse Extollentiam oculorum meorum ne dederis mihi
 et omne desiderium averte a me.
 \verse Aufer a me ventris concupiscentias,
 et concubitus concupiscentiae ne apprehendant me,
 et animae irreverenti et infrunitae ne tradas me.
 \verse De doctrina oris.
 Doctrinam oris audite, filii;
 et, qui custodierit illam, non capietur labiis
 nec scandalizabitur in operibus nequissimis.
 \verse In labiis suis apprehendetur peccator,
 et maledicus et superbus scandalizabitur in illis.
 \verse Iurationi non assuescas os tuum:
 multi enim casus in illa.
 \verse Nominatio vero Dei non sit assidua in ore tuo,
 et nominibus sanctorum non admiscearis,
 quoniam non eris immunis ab eis.
 \verse Sicut enim servus exquisitus assidue
 livore carere non poterit,
 sic omnis iurans et nominans in toto
 a peccato non purgabitur.
 \verse Vir multum iurans implebitur iniquitate,
 et non discedet a domo illius plaga.
 \verse Et, si frustraverit, delictum illius super ipsum erit;
 et, si dissimulaverit, delinquet dupliciter.
 \verse Et, si in vacuum iuraverit, non iustificabitur:
 replebitur enim malis domus illius.
 \verse Est et alia loquela morti comparanda:
 non inveniatur in hereditate Iacob.
 \verse Etenim a timoratis omnia haec sunt remota,
 et in delictis non volutabuntur.
 \verse Indisciplinatae turpitudini non assuescat os tuum:
 est enim in illa verbum peccati.
 \verse Memento patris et matris tuae,
 in medio enim magnatorum consistis;
 \verse ne forte obliviscaris tui in conspectu illorum
 et assiduitate tua infatuatus improperium patiaris
 et maluisses non nasci et diem nativitatis tuae maledicas.
 \verse Homo assuetus in verbis improperii
 in omnibus diebus suis non erudietur.
 \verse Duo genera abundant in peccatis,
 et tertium adducit iram et perditionem:
 \verse anima calida quasi ignis ardens
 non exstinguetur, donec consumatur;
 \verse et homo fornicarius in corpore carnis suae
 non desinet, donec incendat ignem.
 \verse Homini fornicario omnis panis dulcis:
 non cessabit nisi in morte.
 \verse Omnis homo, qui transgreditur super lectum suum
 contemnens in anima sua et dicens: “ Quis me videt?
 \verse Tenebrae circumdant me, et parietes cooperiunt me,
 et nemo circumspicit me; quem vereor?
 Delictorum meorum non memorabitur Altissimus ”
 \verse et non intellegit quoniam omnia videt oculus illius,
 quoniam expellit a se timorem Dei huiusmodi hominis timor.
 Et oculi hominum sunt timor illius,
 \verse et non cognovit quoniam oculi Domini
 multo plus lucidiores sunt super solem
 circumspicientes omnes vias hominum et profundum abyssi
 et hominum corda intuentes in absconditas partes.
 \verse Domino enim Deo, antequam crearentur, omnia sunt agnita;
 sic et, postquam perfecta sunt, respicit omnia.
 \verse Hic in plateis civitatis vindicabitur
 et quasi pullus equinus fugabitur
 et, ubi non speravit, apprehendetur;
 \verse et erit dedecus omnibus,
 eo quod non intellexerit timorem Domini.
 \verse Sic et mulier omnis relinquens virum suum
 et statuens hereditatem ex alieno matrimonio.
 \verse Primo enim in lege Altissimi incredibilis fuit,
 secundo in virum suum deliquit,
 tertio in adulterio fornicata est
 et ex alio viro filios statuit sibi.
 \verse Haec in ecclesiam adducetur
 et in filios eius respicietur;
 \verse non tradent filii eius radices,
 et rami eius non dabunt fructum:
 \verse derelinquet in maledictum memoriam suam,
 et dedecus illius non delebitur.
 \verse Et agnoscent, qui derelicti sunt,
 quoniam nihil melius est quam timor Dei,
 et nihil dulcius quam attendere mandatis Domini.
 \verse Gloria magna est sequi Dominum;
 longitudo enim dierum assumetur ab eo.
 
\begin{biblechapter}
\verse Laus sapientiae.
 Sapientia laudabit animam suam et in Deo honorabitur
 et in medio populi sui gloriabitur
 \verse et in ecclesia Altissimi aperiet os suum
 et in conspectu virtutis illius gloriabitur
 \verse et in medio populi sui exaltabitur
 et in plenitudine sancta admirabitur
 \verse et in multitudine electorum habebit laudem
 et inter benedictos benedicetur dicens:
 \verse “ Ego ex ore Altissimi prodivi,
 primogenita ante omnem creaturam.
 \verse Ego feci in caelis, ut oriretur lumen indeficiens,
 et sicut nebula texi omnem terram.
 \verse Ego in altissimis habitavi,
 et thronus meus in columna nubis.
 \verse Gyrum caeli circuivi sola
 et in profundum abyssi ambulavi,
 \verse in fluctibus maris et in omni terra steti
 \verse et in omni populo et in omni gente primatum habui
 \verse et omnium excellentium et humilium corda virtute calcavi.
 In his omnibus requiem quaesivi:
 cuius in hereditate morabor?
 \verse Tunc praecepit et dixit mihi Creator omnium,
 et, qui creavit me, quietem dedit tabernaculo meo
 \verse et dixit mihi: “In Iacob inhabita et in Israel hereditare
 et in electis meis mitte radices”.
 \verse Ab initio ante saecula creata sum
 et usque ad futurum saeculum non desinam.
 \verse Et in tabernaculo sancto coram ipso ministravi,
 et sic in Sion firmata sum
 et in civitate similiter dilecta requievi,
 et in Ierusalem potestas mea.
 \verse Et radicavi in populo honorificato
 et in parte Domini, in hereditate illius,
 et in plenitudine sanctorum detentio mea.
 \verse Quasi cedrus exaltata sum in Libano,
 et quasi cupressus in montibus Hermon.
 \verse Quasi palma exaltata sum in Engaddi,
 et quasi plantatio rosae in Iericho.
 \verse Quasi oliva speciosa in campis,
 et quasi platanus exaltata sum iuxta aquam in plateis.
 \verse Sicut cinnamomum et balsamum aromatizans odorem dedi;
 quasi myrrha electa dedi suavitatem odoris.
 \verse Et quasi storax et galbanus et ungula et gutta,
 et quasi libani vapor in tabernaculo.
 \verse Ego quasi terebinthus extendi ramos meos,
 et rami mei rami honoris et gratiae.
 \verse Ego quasi vitis germinavi gratiam,
 et flores mei fructus honoris et honestatis.
 \verse Ego mater pulchrae dilectionis et timoris
 et agnitionis et sanctae spei.
 \verse In me gratia omnis viae et veritatis,
 in me omnis spes vitae et virtutis.
 \verse Transite ad me, omnes, qui concupiscitis me,
 et a generationibus meis implemini.
 \verse Doctrina enim mea super mel dulcis, et hereditas mea super mel et favum;
 \verse memoria mea in generationes saeculorum.
 \verse Qui edunt me, adhuc esurient;
 et, qui bibunt me, adhuc sitient.
 \verse Qui audit me, non confundetur;
 et, qui operantur in me, non peccabunt:
 \verse qui elucidant me, vitam aeternam habebunt ”.
 \verse Haec omnia liber testamenti Altissimi,
 \verse lex, quam mandavit nobis Moyses,
 hereditas domui Iacob.
 \verse Posuit David puero suo excitare regem ex ipso fortissimum,
 et in throno honoris sedentem in sempiternum.
 \verse Lex, quae implet quasi Phison sapientiam
 et sicut Tigris in diebus novorum,
 \verse quae adimplet quasi Euphrates sensum
 et quasi Iordanis in tempore messis,
 \verse quae redundavit disciplina sicut Nilus
 et assistens quasi Geon in die vindemiae.
 \verse Non perfecit primus scire ipsam,
 sic nec ultimus investigavit eam.
 \verse Super mare enim abundavit cogitatio eius,
 et consilium illius super abyssum magnam.
 \verse Ego sapientia effudi flumina,
 \verse ego quasi trames aquae immensae de fluvio
 et sicut aquaeductus exivi in paradisum.
 \verse Dixi: “ Rigabo hortum meum plantationum
 et inebriabo prati mei fructum ”.
 \verse Et ecce factus est mihi trames in fluvium,
 et fluvius meus appropinquavit ad mare.
 \verse Quoniam doctrinam quasi antelucanum illuminabo
 et enarrabo illam usque ad longinquum.
 \verse Penetrabo omnes inferiores partes terrae
 et inspiciam omnes dormientes
 et illuminabo omnes sperantes in Domino;
 \verse adhuc doctrinam quasi prophetiam effundam
 et relinquam illam in generationes saeculorum
 et non desinam in progenies illorum usque in aevum sanctum.
 \verse Videte quoniam non soli mihi laboravi,
 sed omnibus exquirentibus illam.
 
\begin{biblechapter}
\verse In tribus placitum est spiritui meo,
 quae sunt probata coram Deo et hominibus:
 \verse concordia fratrum et amor proximorum
 et vir et mulier bene sibi consentientes.
 \verse Tres species odivit anima mea,
 et aggravor valde animae illorum:
 \verse pauperem superbum, divitem mendacem,
 senem fatuum et insensatum.
 \verse In iuventute tua non congregasti;
 quomodo in senectute tua invenies?
 \verse Quam speciosum canitiei iudicium,
 et presbyteris cognoscere consilium!
 \verse Quam speciosa veteranis sapientia,
 et gloriosis intellectus et consilium!
 \verse Corona senum multa peritia,
 et gloria illorum timor Dei.
 \verse Novem insuspicabilia cordis magnificavi,
 et decimum dicam in lingua hominibus:
 \verse homo, qui iucundatur in filiis,
 vivens et videns subversionem inimicorum suorum;
 \verse beatus, qui habitat cum muliere sensata
 et non arat in bove et in asino simul;
 et qui lingua sua non est lapsus,
 et qui non servivit indigno se;
 \verse beatus, qui invenit amicum verum,
 et qui enarrat iustitiam auri audienti;
 \verse quam magnus, qui invenit sapientiam et scientiam,
 sed non est super timentem Dominum.
 \verse Timor Dei super omnia excedit;
 \verse qui tenet illum, cui assimilabitur?
 \verse Timor Dei initium dilectionis eius;
 fides autem initium adhaesionis ei.
 \verse Omnis plaga tristitia cordis est,
 et omnis malitia nequitia mulieris.
 \verse Et omnem plagam et non plagam cordis;
 \verse et omnem nequitiam et non nequitiam mulieris;
 \verse et omnem calamitatem et non calamitatem odientium;
 \verse et omnem vindictam et non vindictam inimicorum.
 \verse Non est venenum nequius super venenum colubri,
 \verse et non est ira super iram mulieris;
 commorari leoni et draconi placebit
 quam habitare cum muliere nequam.
 \verse Nequitia mulieris immutat faciem eius
 et obscurat vultum eius tamquam ursus.
 In medio proximorum eius accumbet vir eius
 \verse et invitus suspirabit amare.
 \verse Parva omnis malitia prae malitia mulieris:
 sors peccatorum cadat super illam.
 \verse Sicut ascensus arenosus in pedibus veterani,
 sic mulier linguata homini quieto.
 \verse Ne incidas in mulieris speciem
 et non concupiscas mulieris opes.
 \verse Ira et irreverentia et confusio magna
 \verse mulier, si primatum habeat super virum suum.
 \verse Cor humile et facies tristis
 et plaga cordis mulier nequam.
 \verse Manus debiles et genua dissoluta
 mulier, quae non beatificat virum suum.
 \verse A muliere initium factum est peccati,
 et per illam omnes morimur.
 \verse Non des aquae tuae exitum, nec modicum,
 nec mulieri nequam veniam prodeundi.
 \verse Si non ambulaverit ad manum tuam,
 confundet te in conspectu inimicorum;
 \verse a carnibus tuis abscinde illam
 et dimitte illam de domo tua.
 
\begin{biblechapter}
\verse Mulieris bonae beatus vir:
 numerus enim dierum illius duplex.
 \verse Mulier fortis oblectat virum suum
 et annos vitae illius in pace implebit.
 \verse Pars bona mulier bona;
 in parte timentium Deum dabitur viro pro factis bonis.
 \verse Divitis autem vel pauperis cor bonum;
 in omni tempore vultus illorum hilaris.
 \verse A tribus timuit cor meum,
 et de quarto facies mea metuit:
 \verse delaturam civitatis et collectionem populi,
 \verse calumniam mendacem, super mortem omnia gravia;
 \verse dolor cordis et luctus mulier zelotypa in mulierem;
 \verse et flagellum linguae omnibus communicans.
 \verse Sicut boum iugum, quod movetur, ita et mulier nequam;
 qui tenet illam, quasi qui apprehendat scorpionem.
 \verse Mulier ebriosa ira magna et contumelia,
 et turpitudo illius non tegetur.
 \verse Fornicatio mulieris in extollentia oculorum
 et in palpebris illius agnoscetur.
 \verse In filia pervicaci firma custodiam,
 ne, inventa occasione, utatur sibi.
 \verse Ab omni irreverentia oculorum eius cave
 et ne mireris, si te neglexerit.
 \verse Sicut viator sitiens ad fontem os aperiet
 et ab omni aqua proxima bibet
 et contra omnem palum sedebit
 et contra omnem sagittam aperiet pharetram, donec deficiat.
 \verse Gratia mulieris sedulae delectabit virum suum,
 et ossa illius impinguabit 
\verse disciplina illius.
 Datum Dei est 
\verse mulier sensata et tacita;
 non est commutatio eruditae animae.
 \verse Gratia super gratiam mulier sancta et pudorata;
 \verse omnis autem ponderatio non est digna continentis animae.
 \verse Sicut sol oriens mundo in altissimis Dei,
 sic mulieris bonae species in ornamentum domus eius.
 \verse Lucerna splendens super candelabrum sanctum,
 et species faciei super staturam stabilem;
 \verse columnae aureae super bases argenteas,
 et pedes speciosi super plantas stabiles mulieris.
 \verse Fundamenta aeterna supra petram solidam,
 et mandata Dei in corde mulieris sanctae.
 \verse In duobus contristatum est cor meum,
 et in tertio iracundia mihi advenit:
 \verse vir bellator deficiens per inopiam
 et vir sensatus contemptus
 \verse et qui transgreditur a iustitia ad peccatum;
 Deus parabit eum ad romphaeam.
 \verse Duae species difficiles et periculosae mihi apparuerunt:
 difficile eruitur negotians a neglegentia,
 et non iustificabitur caupo a peccatis.
 
\begin{biblechapter}
\verse Propter lucrum multi deliquerunt;
 et, qui quaerit locupletari, avertet oculum suum.
 \verse Sicut in medio compaginis lapidum palus figitur,
 sic et inter medium venditionis et emptionis constringitur peccatum.(\verse)
 \verse Si non in timore Domini tenueris te,
 instanter cito subvertetur domus tua.
 \verse Sicut in percussura cribri remanent quisquiliae,
 sic peripsemata hominis in cogitatu illius.
 \verse Vasa figuli probat fornax,
 et homines iustos tentatio tribulationis.
 \verse Sicut rusticationem ligni ostendit fructus illius,
 sic verbum ex cogitatu cordis hominis.
 \verse Ante sermonem non laudes virum:
 haec enim tentatio est hominum.
 \verse Si sequaris iustitiam, apprehendes illam
 et indues quasi poderem honoris
 et inhabitabis cum ea, et proteget te in sempiternum,
 et in die agnitionis invenies firmamentum.
 \verse Volatilia ad sibi similia conveniunt,
 et veritas ad eos, qui operantur illam, revertetur.
 \verse Leo venationi insidiatur semper,
 sic peccata operantibus iniquitates.
 \verse Loquela timorati semper in sapientia manet;
 stultus autem sicut luna mutatur.
 \verse In medio insensatorum serva tempus,
 in medio autem cogitantium assiduus esto.
 \verse Loquela stultorum odiosa,
 et risus illorum in deliciis peccati.
 \verse Loquacitas multum iurantis horripilationem capiti statuet,
 et rixa illorum obturatio aurium.
 \verse Effusio sanguinis rixa superborum,
 et maledictio illorum auditus gravis.
 \verse Qui denudat arcana, amici fidem perdit
 et non inveniet amicum ad animum suum:
 \verse dilige amicum et coniungere fide cum illo;
 \verse quod, si denudaveris absconsa illius,
 non persequeris post eum.
 \verse Sicut enim homo, qui extulit mortuum suum,
 sic et qui perdit amicitiam proximi sui;
 \verse et sicut qui dimittit avem de manu sua,
 sic dereliquisti proximum tuum et non eum capies.
 \verse Non illum sequaris, quoniam longe abest;
 effugit enim quasi caprea de laqueo,
 quoniam vulnerata est anima eius;
 \verse ultra eum non poteris colligare.
 Et maledicti est concordatio,
 \verse denudare autem amici mysteria
 amputatio spei est.
 \verse Annuens oculo fabricat iniqua;
 qui novit eum, recedet ab illo.
 \verse In conspectu oculorum tuorum condulcabit os suum
 et super sermones tuos admirabitur;
 novissime autem pervertet os suum
 et in verbis tuis dabit scandalum.
 \verse Multa odivi et non coaequavi ei,
 et Dominus odiet illum.
 \verse Qui in altum mittit lapidem, super caput eius cadet,
 et plaga dolosa dolosi dividet vulnera.
 \verse Qui foveam fodit, incidet in eam,
 et, qui statuit lapidem proximo, offendet in eo,
 et, qui laqueum alii ponit, capietur in illo.
 \verse Facienti nequissimum consilium, super ipsum devolvetur,
 et non agnoscet, unde adveniat illi.
 \verse Illusio et improperium superbo,
 et vindicta sicut leo insidiabitur illi.
 \verse Laqueo capientur, qui oblectantur casu iustorum,
 dolor autem consumet illos, antequam moriantur.
 \verse Ira et furor utraque exsecrabilia sunt,
 et vir peccator continens erit illorum.
 
\begin{biblechapter}
\verse Qui vindicari vult, a Domino inveniet vindictam,
 et peccata illius servans servabit.
 \verse Dimitte proximo tuo nocenti te,
 et tunc deprecanti tibi peccata tua solventur.
 \verse Homo homini reservat iram
 et a Deo quaerit medelam?
 \verse In hominem similem sibi non habet misericordiam
 et de peccatis suis deprecatur?
 \verse Ipse, cum caro sit, reservat iram
 et propitiationem petit a Deo?
 Quis exorabit pro delictis illius?
 \verse Memento novissimorum et desine inimicari,
 \verse tabitudinis et mortis et permane in mandatis eius.
 \verse Memorare mandatorum et non irascaris proximo.
 \verse Memorare testamentum Altissimi
 et ne respicias ignorantiam proximi.
 \verse Abstine te a lite et minues peccata:
 \verse homo enim iracundus incendit litem,
 et vir peccator turbabit amicos
 et in medio pacem habentium immittet inimicitiam.
 \verse Secundum enim ligna silvae sic ignis exardescit,
 et secundum virtutem hominis sic iracundia illius erit,
 et secundum substantiam suam exaltabit iram suam.
 \verse Certamen festinatum incendit ignem,
 et lis festinans effundit sanguinem,
 et lingua testificans adducit mortem.
 \verse Si sufflaveris in scintillam, quasi ignis exardebit;
 et, si exspueris super illam, exstinguetur:
 utraque ex ore tuo proficiscuntur.
 \verse Susurro et bilinguis maledictus:
 multos enim perdidit pacem habentes.
 \verse Lingua tertia multos commovit
 et dispersit illos de gente in gentem;
 \verse civitates muratas divitum destruxit
 et domos magnatorum evertit;
 \verse virtutes populorum concidit
 et gentes fortes dissolvit.
 \verse Lingua tertia mulieres viratas eiecit
 et privavit illas laboribus suis.
 \verse Qui respicit illam, non habebit requiem
 nec habebit amicum, in quo requiescat.
 \verse Flagelli plaga livorem facit,
 plaga autem linguae comminuet ossa;
 \verse multi ceciderunt in ore gladii,
 sed non sic quasi qui interierunt per linguam suam.
 \verse Beatus, qui tectus est ab ea,
 qui in iracundiam illius non transivit,
 et qui non attraxit iugum illius
 et in vinculis eius non est ligatus.
 \verse Iugum enim illius iugum ferreum est,
 et vinculum illius vinculum aereum est;
 \verse mors illius mors nequissima,
 et utilis potius infernus quam illa.
 \verse Non obtinebit imperium iustorum,
 et in flamma eius non comburentur;
 \verse qui relinquunt Deum, incident in illam,
 et exardebit in illis et non exstinguetur
 et immittetur in illos quasi leo
 et quasi pardus laedet illos.
 \verse Saepi aures tuas spinis, linguam nequam noli audire
 et ori tuo facito ostia et seras.
 \verse Aurum tuum et argentum tuum obsigna
 et verbis tuis facito stateram et frenos ori tuo rectos.
 \verse Et attende, ne forte labaris in lingua,
 ne cadas in conspectu inimicorum insidiantium tibi,
 et sit casus tuus insanabilis in mortem.
 
\begin{biblechapter}
 \verse Qui facit misericordiam, feneratur proximo suo;
 et, qui confortat manu, mandata servat.
 \verse Fenerare proximo tuo in tempore necessitatis illius
 et iterum redde proximo in tempore suo.
 \verse Confirma verbum et fideliter age cum illo,
 et omni tempore invenies, quod tibi necessarium est.
 \verse Multi quasi inventionem aestimaverunt fenus
 et praestiterunt molestiam his, qui se adiuverunt.
 \verse Donec accipiat, osculatur manus dantis
 et de possessionibus proximi humiliat vocem suam;
 \verse et in tempore redditionis postulabit tempus,
 et reddet verba taedii et murmurationum et tempus causabitur.
 \verse Si autem potuerit reddere, adversabitur;
 solidi vix reddet dimidium
 et computabit illud quasi inventionem.
 \verse Sin autem, fraudabit illum pecunia sua
 et possidebit illum inimicum gratis.
 \verse Et convicia et maledicta reddet illi
 et pro honore et beneficio reddet illi contumeliam.
 \verse Multi non causa nequitiae non fenerati sunt,
 sed fraudari gratis timuerunt.
 \verse Verumtamen super humilem longanimis esto
 et pro eleemosyna non trahas illum.
 \verse Propter mandatum assume pauperem
 et propter inopiam eius ne dimittas eum vacuum.
 \verse Perde pecuniam propter fratrem et amicum tuum
 et non abscondas illam sub lapide in perditionem.
 \verse Pone thesaurum tuum in praeceptis Altissimi,
 et proderit tibi magis quam aurum.
 \verse Conclude eleemosynam in corde pauperis,
 et haec pro te exorabit ab omni malo. (\verse \verse)
 \verse Super scutum roboris et super lanceam ponderis
 adversus inimicum tuum pugnabit pro te.
 \verse Vir bonus fidem facit pro proximo suo;
 et, qui perdiderit confusionem, fugiet repromissorem.
 \verse Gratiam fideiussoris ne obliviscaris:
 dedit enim pro te animam suam. (\verse)
 \verse Bona repromissoris dissipabit peccator,
 et ingratus sensu derelinquet liberantem se. (\verse)
 \verse Repromissio multos perdidit recte agentes
 et commovit illos quasi fluctus maris;
 \verse viros potentes transmigrare fecit,
 et vagati sunt in gentibus alienis.
 \verse Peccator transgrediens mandata Domini
 incidet in repromissionem,
 et, qui conatur lucrum sectari, incidet in iudicium.
 \verse Sponde pro proximo secundum virtutem tuam,
 sed attende tibi, ne incidas.
 \verse Initium vitae hominis aqua et panis et vestimentum
 et domus protegens turpitudinem.
 \verse Melior est victus pauperis sub tegmine asserum
 quam epulae splendidae in peregre sine domicilio.
 \verse Super parvo et magno placeat tibi,
 et improperium peregrinationis non audies.
 \verse Vita nequam hospitandi de domo in domum,
 et ubi hospitabitur, non fiducialiter aget, nec aperiet os.
 \verse Hospitaberis et pasceris et potaberis sine gratia,
 et ad haec amara audiet:
 \verse “ Transi, hospes, et orna mensam
 et, si quae in manu habes, ciba me! ”.
 \verse “ Exi a facie honoratioris!
 Necessitudine domus meae hospitio mihi factus est frater ”.
 \verse Gravia haec homini habenti sensum:
 obiurgatio peregrinationis et improperium feneratoris.
 
\begin{biblechapter}
\verse De filiis.
 Qui diligit filium suum, assi duat illi flagella,
 ut laetetur in novissimo suo.
 \verse Qui docet filium suum, fructum habebit in illo
 et in medio domesticorum in illo gloriabitur.
 \verse Qui docet filium suum, in zelum mittet inimicum
 et in medio amicorum gloriabitur in illo.
 \verse Mortuus est pater eius et quasi non est mortuus:
 similem enim reliquit sibi post se.
 \verse In vita sua vidit et laetatus est in illo,
 in obitu suo non est contristatus.
 Nec confusus est coram inimicis:
 \verse reliquit enim defensorem domus contra inimicos
 et amicis reddentem gratiam.
 \verse Qui blanditur filio, colligabit vulnera eius,
 et super omnem vocem turbabuntur viscera sua.
 \verse Equus indomitus evadit durus,
 et filius remissus evadet praeceps.
 \verse Lacta filium, et paventem te faciet;
 lude cum eo, et contristabit te.
 \verse Non corrideas illi, ne doleas,
 et in novissimo obstupescent dentes tui.
 \verse Non des illi potestatem in iuventute
 et ne despicias errata illius.
 \verse Curva cervicem eius in iuventute
 et tunde latera eius, dum infans est,
 ne forte induret et non credat tibi,
 et erit tibi ab illo dolor animae.
 \verse Doce filium tuum et operare in illo,
 ne in turpitudinem illius offendas.
 \verse Melior est pauper sanus et fortis viribus
 quam dives imbecillis et flagellatus in carne sua.
 \verse Salus carnis melior est omni auro et argento,
 et spiritus validus quam census immensus.
 \verse Non est census super censum salutis corporis,
 et non est oblectatio super cordis gaudium.
 \verse Melior est mors quam vita amara,
 et requies aeterna quam languor perseverans.
 \verse Bona effusa in ore clauso
 quasi appositiones epularum circumpositae sepulcro.
 \verse Quid proderit libatio idolo?
 Nec enim manducabit nec odorabitur:
 \verse sic qui effugatur a Domino
 portans mercedes iniquitatis,
 \verse videns oculis et ingemiscens
 sicut spado complectens virginem et suspirans.
 \verse Tristitiam non des animae tuae
 et non affligas temetipsum in consilio tuo.
 \verse Iucunditas cordis haec est vita hominis
 et thesaurus sine defectione sanctitatis,
 et exsultatio viri est longaevitas.
 \verse Indulge animae tuae et consolare cor tuum
 et tristitiam longe repelle a te.
 \verse Multos enim occidit tristitia,
 et non est utilitas in illa;
 \verse zelus et iracundia minuunt dies,
 et ante tempus senectam adducet cogitatus.
 \verse Splendidum cor et bonum in epulis est;
 epulae enim illius diligenter fiunt.
 
\begin{biblechapter}
\verse Vigilia divitis tabefacit carnes,
 et cogitatus illius aufert somnum;
 \verse cogitatus victus avertit somnum,
 et infirmitas gravis a somno excitat.
 \verse Laboravit dives in congregatione substantiae
 et, si requiescit, replebitur deliciis suis;
 \verse laboravit pauper in penuria victus
 et, si requiescit, inops fit.
 \verse Qui aurum diligit, non iustificabitur,
 et, qui insequitur lucrum, in eo oberrabit.
 \verse Multi dati sunt in ruinam auri gratia,
 et facta est in facie ipsorum perditio illorum.
 \verse Lignum offensionis est aurum sacrificantium;
 vae illis, qui sectantur illud:
 et omnis imprudens capietur in illo.
 \verse Beatus dives, qui inventus est sine macula
 et qui post aurum non abiit
 nec speravit in pecunia et thesauris.
 \verse Quis est hic, et laudabimus eum?
 Fecit enim mirabilia in populo suo.
 \verse Quis probatus est in illo et perfectus est?
 Erit illi gloria aeterna.
 Quis potuit transgredi et non est transgressus,
 facere mala et non fecit?
 \verse Ideo stabilita sunt bona illius in Domino,
 et eleemosynas illius enarrabit omnis ecclesia sanctorum.
 \verse De continentia.
 Supra mensam magnam sedisti?
 Non aperias super illam faucem tuam prior.
 \verse Non dicas: “Multa sunt, quae super illam sunt!”.
 \verse Memento quoniam malum est oculus nequam;
 oculum nequam odit Deus.
 \verse Nequius oculo quid creatum est?
 Ideo ab omni facie lacrimatur.
 \verse Quocumque aspexerit, ne extendas manum tuam prior
 et invidia contaminatus erubescas;
 \verse nec comprimaris cum eo in catino.
 \verse Intellege, quae sunt proximi tui, ex teipso
 et de omni verbo cogita;
 \verse utere quasi homo frugi his, quae tibi apponuntur,
 ne, cum manducas multum, odio habearis.
 \verse Cessa prior disciplinae causa
 et noli nimius esse, ne forte offendas.
 \verse Et, si in medio multorum sedisti,
 prior illis ne extendas manum tuam nec prior poscas bibere.
 \verse Quam sufficiens est homini erudito vinum exiguum!
 Et in dormiendo non laborabis ab illo et non senties dolorem.
 \verse Vigilia, cholera et tortura ventris viro infrunito;
 \verse somnus sanitatis in homine parco:
 dormiet usque mane, et anima illius cum ipso delectabitur.
 \verse Et, si coactus fueris in edendo multum,
 surge e medio, evome, et refrigerabit te,
 et non adduces corpori tuo infirmitatem.
 \verse Audi me, fili, et ne spernas me
 et in novissimo invenies verba mea.
 \verse In omnibus operibus tuis esto modestus,
 et omnis infirmitas non occurret tibi.
 \verse Splendidum in panibus benedicent labia multorum,
 et testimonium bonitatis illius fidele;
 \verse nequissimo in pane murmurabit civitas,
 et testimonium nequitiae illius verum est.
 \verse In vino noli provocare;
 multos enim exterminavit vinum.
 \verse Fornax probat aciem ferri in intinctione:
 sic vinum in lite corda superborum.
 \verse Quasi vita hominibus vinum,
 si bibas illud moderate.
 \verse Quae est vita ei, qui minuitur vino?
 \verse Quid defraudat vitam? Mors.
 \verse Vinum in iucunditatem creatum est
 et non in ebrietatem, ab initio.
 \verse Exsultatio animae et gaudium et voluptas cordis
 vinum moderate potatum in tempore;
 \verse sanitas est animae et corpori sobrius potus.
 \verse Vinum multum potatum irritationem
 et iram et ruinas multas facit.
 \verse Amaritudo animae vinum multum potatum
 in irritatione et ruina.
 \verse Ebrietas multiplicat animositatem imprudentis in offensionem,
 minorans virtutem et faciens vulnera.
 \verse In convivio vini non arguas proximum
 et non despicias eum in iucunditate illius;
 \verse verba improperii non dicas illi
 et non premas illum in repetendo.
 
\begin{biblechapter}
\verse Rectorem te posuerunt? Noli extolli:
 esto in illis quasi unus ex ipsis.
 \verse Curam illorum habe et sic conside
 et omni cura tua explicita recumbe,
 \verse ut laeteris propter illos
 et decentiae gratia accipias coronam
 et dignationem consequaris corrogationis.
 \verse Loquere, maior natu: decet enim te
 \verse primum verbum in diligenti scientia;
 et non impedias musicam.
 \verse Ubi convivium, non effundas sermonem
 et importune noli extolli in sapientia tua.
 \verse Gemmula carbunculi in ornamento auri,
 et concentus musicorum in convivio vini;
 \verse sicut in fabricatione aurea signum est smaragdi,
 sic numerus musicorum in iucundo et moderato vino.
 \verse Audi tacens, et pro reverentia accedet tibi bona gratia.
 \verse Adulescens, loquere in tua causa vix;
 \verse bis, si interrogatus fueris.
 \verse Recapitula sermonem, in paucis multa;
 esto quasi scius et simul tacens.
 \verse In medio magnatorum non praesumas
 et, ubi sunt senes, non multum loquaris.
 \verse Ante tonitruum praeibit coruscatio,
 et ante verecundum praeibit gratia.
 \verse Et hora surgendi non te trices;
 praecurre autem prior in domum tuam
 et illic avocare et illic lude
 \verse et age conceptiones tuas
 et noli peccare in verbo superbo.
 \verse Et super his omnibus benedicito Dominum, qui fecit te
 et inebriantem te ab omnibus bonis suis.
 \verse Qui timet Dominum, excipiet doctrinam;
 et, qui vigilaverint ad illum, invenient benedictionem.
 \verse Qui quaerit legem, replebitur ab ea; et, qui insidiose agit, scandalizabitur in ea.
 \verse Qui timent Dominum, invenient iudicium iustum
 et iustitias quasi lumen accendent.
 \verse Peccator homo vitabit correptionem
 et secundum voluntatem suam inveniet comparationem.
 \verse Vir consilii non despiciet intellegentiam;
 alienus et superbus non pertimescet timorem. (\verse)
 \verse Fili, sine consilio nihil facias
 et post factum non paenitebis.
 \verse In via ruinae non eas
 et non offendes bis in lapides;
 ne credas te viae inexploratae,
 ne ponas animae tuae scandalum.
 \verse Et a filiis tuis cave
 et a domesticis tuis attende.
 \verse In omni opere tuo confide animae tuae:
 haec est enim conservatio mandatorum.
 \verse Qui credit legi, attendit mandatis;
 et, qui confidit in Domino, non minorabitur.
 
\begin{biblechapter}
\verse Timenti Dominum non occurrent mala,
 sed in tentatione Deus illum iterum conservabit et liberabit a malis.
 \verse Sapiens non odit mandata et iustitias
 et non illidetur quasi in procella navis.
 \verse Homo sensatus credit verbo Dei,
 et lex illi fidelis sicut qui interrogationem manifestat.
 \verse Para verba et sic deprecatus exaudieris
 et conservabis disciplinam et tunc respondebis.
 \verse Praecordia fatui quasi rota carri,
 et quasi axis versatilis cogitatus illius.
 \verse Equus admissarius, sic et amicus subsannator:
 sub omni suprasedente hinnit.
 \verse Quare dies diem superat,
 si omnis lux anni a sole?
 \verse A Domini scientia separati sunt,
 \verse et immutavit tempora et dies festos ipsorum.
 \verse Ex ipsis exaltavit et magnificavit Deus
 et ex ipsis posuit in numerum dierum.
 Et omnes homines de solo,
 et ex terra creatus est Adam.
 \verse In multitudine disciplinae Dominus separavit eos
 et immutavit vias eorum:
 \verse ex ipsis benedixit et exaltavit
 et ex ipsis sanctificavit et ad se applicavit;
 ex ipsis maledixit et humiliavit
 et convertit illos a statione ipsorum.
 \verse Quasi lutum figuli in manu ipsius
 plasmare illud et disponere
 \verse secundum beneplacitum eius,
 sic homo in manu illius, qui se fecit
 et reddet illi secundum iudicium suum.
 \verse Contra malum bonum est,
 et contra mortem vita;
 sic et contra virum iustum peccator.
 Et sic intuere in omnia opera Altissimi,
 duo et duo, et unum contra unum.
 \verse Et ego novissimus evigilavi
 et, quasi qui colligit acinos post vindemiatores,
 \verse in benedictione Dei et ipse praecessi
 et, quasi qui vindemiat, replevi torcular.
 \verse Respicite quoniam non mihi soli laboravi
 sed omnibus exquirentibus disciplinam.
 \verse Audite me, magnates populi;
 et rectores ecclesiae, auribus percipite.
 \verse Filio et mulieri, fratri et amico
 non des potestatem super te in vita tua;
 et non dederis alii possessionem tuam,
 ne forte paeniteat te et depreceris pro illis.
 \verse Dum adhuc superes et aspiras,
 non commutes teipsum cum omni carne.
 \verse Melius est enim, ut filii tui te rogent,
 quam te respicere in manus filiorum tuorum.
 \verse In omnibus operibus tuis praecellens esto,
 \verse ne dederis maculam in gloria tua.
 In die consummationis dierum vitae tuae
 et in tempore exitus tui distribue hereditatem tuam.
 \verse Cibaria et virga et onus asino,
 panis et disciplina et opus servo.
 \verse Operare per servum et invenies requiem;
 laxa manus illi, et quaeret libertatem.
 \verse Iugum et lorum curvant collum,
 et servum inclinant operationes assiduae.
 \verse Servo malevolo tortura et compedes,
 mitte illum in operationem, ne vacet:
 \verse multam enim malitiam docuit otiositas.
 \verse In opera constitue eum, sic enim condecet illum;
 quod si non obaudierit, curva illum compedibus,
 sed non immoderate in omnem carnem;
 verum sine iudicio nihil facias grave.
 \verse Si est tibi servus unicus, sit tibi quasi anima tua,
 quoniam sicut te indigebis illo.
 Si est tibi servus unicus, quasi fratrem sic eum tracta,
 ne in sanguinem animae tuae irascaris.
 \verse Si laeseris eum iniuste, in fugam convertetur;
 \verse et, si surgens discesserit,
 in qua via quaeras illum, nescis.
 
\begin{biblechapter}
\verse Vana spes et mendax viro insensato,
 et somnia extollunt imprudentes.
 \verse Quasi qui apprehendit umbram et persequitur ventum,
 sic et qui attendit ad visa noctis.
 \verse Hoc secundum hoc visio somniorum,
 ante faciem hominis similitudo faciei.
 \verse Ab immundo quid mundabitur?
 Et a mendace quid verum dicetur?
 \verse Divinationes et auguria et somnia vanitas est,
 \verse et, sicut parturientis, cor phantasias patitur.
 Nisi ab Altissimo fuerit emissa visitatio,
 ne dederis in illis cor tuum.
 \verse Multos enim errare fecerunt somnia,
 et exciderunt sperantes in illis.
 \verse Sine mendacio consummabitur verbum legis,
 et sapientia in ore fideli consummatio.
 \verse Vir, qui peregrinatus est, multa didicit,
 et, qui multa expertus est, enarrabit scienter.
 \verse Qui non est expertus, pauca recognoscit,
 qui autem peregrinatus est, multiplicat astutiam. (\verse)
 \verse Multa vidi errando
 et plurima verba intellexi;
 \verse aliquoties usque ad mortem periclitatus sum
 et horum causa liberatus sum.
 \verse Spiritus timentium Dominum vivet,
 et in respectu illius benedicetur.
 \verse Spes enim illorum in salvantem illos,
 et oculi Dei in diligentes se.
 \verse Qui timet Dominum, nihil trepidabit
 et non pavebit, quoniam ipse est spes eius.
 \verse Timentis Dominum beata est anima eius.
 \verse Ad quem respicit? Et quis est fortitudo eius?
 \verse Oculi Domini super timentes eum:
 protector potentiae, firmamentum virtutis,
 tegimen ardoris et umbraculum meridiani,
 \verse custodia offensionis et adiutorium casus,
 exaltans animam et illuminans oculos,
 dans sanitatem vitae et benedictionem.
 \verse Dominus solus sustinentibus se
 in via veritatis et iustitiae.
 \verse Immolantis ex iniquo, oblatio maculata,
 et non sunt beneplacitae hostiae iniustorum.
 \verse Dona iniquorum non probat Altissimus
 nec respicit in oblationes iniquorum
 nec in multitudine sacrificiorum eorum propitiabitur peccatis.
 \verse Qui offert sacrificium ex substantia pauperum,
 quasi qui victimat filium in conspectu patris sui.
 \verse Panis egentium vita pauperum est;
 qui defraudat illum, homo sanguinis est.
 \verse Qui aufert in sudore panem,
 quasi qui occidit proximum suum;
 \verse et effundit sanguinem,
 qui fraudem facit mercennario.
 \verse Unus aedificans et unus destruens;
 quid prodest illis nisi labor?
 \verse Unus orans et unus maledicens;
 cuius vocem exaudiet Deus?
 \verse Qui baptizatur a mortuo et iterum tangit eum:
 quid proficit lavatio illius?
 \verse Sic homo, qui ieiunat pro peccatis suis,
 et iterum vadens et eadem faciens.
 Orationem illius quis exaudiet?
 Aut quid proficit humiliando se?
 
\begin{biblechapter}
\verse Qui conservat legem, multiplicat oblationes:
 \verse sacrificium salutare est attendere mandatis. (\verse)
 \verse Qui retribuit gratiam, offert similaginem,
 et, qui facit eleemosynam, offert sacrificium laudis.
 \verse Beneplacitum est Domino recedere ab iniquitate,
 et deprecatio pro peccatis recedere ab iniustitia.
 \verse Non apparebis ante conspectum Domini vacuus:
 \verse haec enim omnia propter mandatum Dei fiunt.
 \verse Oblatio iusti impinguat altare,
 et odor suavitatis illius est in conspectu Altissimi;
 \verse sacrificium iusti acceptum est,
 et memoriam eius non obliviscetur Dominus.
 \verse Bono oculo gloriam redde Deo
 et non minuas primitias manuum tuarum;
 \verse in omni dato hilarem fac vultum tuum
 et in exsultatione sanctifica decimas tuas;
 \verse da Altissimo secundum datum eius
 et in bono oculo ad inventionem facito manuum tuarum,
 \verse quoniam Dominus retribuens est
 et septies tantum reddet tibi.
 \verse Noli offerre munera prava,
 non enim suscipiet illa;
 \verse et noli confidere in sacrificio iniusto,
 quoniam Dominus iudex est,
 et non est apud illum gloria personae.
 \verse Non accipiet personam in pauperem
 et deprecationem laesi exaudiet.
 \verse Non despiciet preces pupilli
 nec viduam, si effundat loquelam gemitus.
 \verse Nonne lacrimae viduae ad maxillam descendunt,
 et exclamatio eius super deducentem eas?
 \verse A maxilla enim ascendunt usque ad caelum,
 et Dominus exauditor non delectabitur in illis.
 \verse Qui adorat Deum, in beneplacito suscipietur,
 et deprecatio illius usque ad nubes propinquabit.
 \verse Oratio humilis nubes penetrabit
 et, donec propinquet, non consolabitur;
 et non discedet, donec Altissimus aspiciat,
 et iudex iustus faciat iudicium.
 \verse Et Dominus non tardabit,
 et Fortissimus non habebit in illis patientiam,
 donec contribulet dorsum crudelium
 \verse et gentibus reddet vindictam,
 donec tollat multitudinem superborum
 et sceptra iniquorum contribulet,
 \verse donec reddat hominibus secundum actus suos
 et opera Adae secundum praesumptionem illius,
 \verse donec iudicet iudicium plebis suae
 et oblectabit istos misericordia sua.
 \verse Speciosa misericordia in tempore tribulationis,
 quasi nubes pluviae in tempore siccitatis.
 
\begin{biblechapter}
\verse Miserere nostri, Deus omnium, et respice nos
 et ostende nobis lucem miserationum tuarum;
 \verse et immitte timorem tuum super gentes,
 quae non exquisierunt te,
 ut cognoscant quia non est Deus nisi tu,
 et enarrent magnalia tua.
 \verse Alleva manum tuam super gentes alienas,
 ut videant potentiam tuam.
 \verse Sicut enim in conspectu eorum sanctificatus es in nobis,
 sic in conspectu nostro magnificaberis in eis,
 \verse ut cognoscant, sicut et nos cognovimus,
 quoniam non est Deus praeter te, Domine.
 \verse Innova signa et itera mirabilia,
 \verse glorifica manum et firma brachium dextrum,
 \verse excita furorem et effunde iram,
 \verse tolle adversarium et afflige inimicum.
 \verse Festina tempus et memento praefinitionis,
 et enarrentur mirabilia tua.
 \verse In ira flammae devoretur, qui salvatur;
 et, qui pessimant plebem tuam, inveniant perditionem.
 \verse Contere caput principum inimicorum
 dicentium: “ Non est alius praeter nos!”.
 \verse Congrega omnes tribus Iacob
 et hereditabis eos sicut ab initio.
 \verse Miserere plebi tuae, super quam invocatum est nomen tuum,
 et Israel, quem coaequasti primogenito tuo.
 \verse Miserere civitati sanctificationis tuae,
 Ierusalem, loco requiei tuae.
 \verse Reple Sion maiestate tua
 et gloria tua templum tuum.
 \verse Da testimonium his, qui ab initio creaturae tuae sunt,
 et suscita praedicationes, quas locuti sunt in nomine tuo.
 \verse Da mercedem sustinentibus te,
 ut prophetae tui fideles inveniantur.
 Et exaudi orationes servorum tuorum,
 \verse secundum beneplacitum super populo tuo,
 et dirige nos in viam iustitiae,
 et sciant omnes, qui habitant terram,
 quia tu es Deus saeculorum.
 \verse Omnem escam manducabit venter,
 et est cibus cibo melior;
 \verse fauces percipiunt cibum ferae,
 et cor sensatum verba mendacia.
 \verse Cor pravum dabit tristitiam,
 et homo peritus retribuet illi.
 \verse Omnem masculum excipiet mulier,
 est autem filia melior filia.
 \verse Species mulieris exhilarat faciem viri sui,
 et super omnem concupiscentiam hominis superducit desiderium.
 \verse Insuper, si est super lingua eius curatio
 et mitigatio et misericordia,
 non est vir illius secundum filios hominum.
 \verse Qui possidet mulierem bonam, inchoat possessionem,
 adiutorium secundum illum est et columna requiei.
 \verse Ubi non est saepes, diripietur vinea,
 et ubi non est mulier, ingemiscet errans.
 \verse Quis credit ei, qui non habet nidum
 et deflectens ubicumque obscuraverit,
 quasi succinctus latro exsiliens de civitate in civitatem?
 
\begin{biblechapter}
\verse Omnis amicus dicet: “ Et ego amicitiam copulavi! ”;
 sed est amicus solo nomine amicus.
 Nonne tristitia appropinquans usque ad mortem:
 \verse sodalis et amicus ad inimicitiam conversus?
 \verse O praesumptio nequissima, unde creata es
 cooperire aridam malitia et dolositate illius?
 \verse Sodalis amico coniucundatur in oblectationibus
 et in tempore tribulationis adversarius erit;
 \verse sodalis amico condolet causa ventris
 et contra hostem accipiet scutum.
 \verse Non obliviscaris amici tui in animo tuo
 et non immemor sis illius in opibus tuis.
 \verse Noli consiliari cum eo, qui tibi insidiatur,
 et a zelantibus te absconde consilium.
 \verse Omnis consiliarius prodit consilium,
 sed est consiliarius pro semetipso.
 \verse A consiliario serva animam tuam
 et prius scito quae sit illius necessitas
 ­ et ipse enim animo suo cogitabit ­
 \verse ne forte mittat super te sortem
 et dicat tibi: 
\verse “ Bona est via tua ”
 et stet e contrario videre quid tibi eveniat.
 \verse Noli consiliari cum invido
 et a zelante te consilium absconde;
 nec cum muliere de ea, quae ei aemulatur,
 cum timido de bello,
 cum negotiatore de traiecticio,
 cum emptore de venditione,
 cum viro livido de gratiis agendis,
 \verse cum impio de pietate,
 cum inhonesto de honestate,
 cum operario otioso de omni opere,
 \verse cum mercennario annuali de consummatione anni,
 cum servo pigro de multa operatione:
 non attendas his in omni consilio.
 \verse Sed cum viro timorato assiduus esto,
 quemcumque cognoveris observantem mandata,
 \verse cuius anima est secundum animam tuam,
 et qui, cum titubaveris in tenebris, condolebit tibi.
 \verse Et consilium cordis statue tecum;
 non est enim tibi aliud fidelius illo.
 \verse Anima viri enuntiat aliquando vera
 quam septem circumspectores sedentes in excelso ad speculandum.
 \verse Et in his omnibus deprecare Altissimum,
 ut dirigat in veritate viam tuam.
 \verse Ante omnia opera verbum verax praecedat te,
 et ante omnem actum consilium stabile.
 \verse Radix consiliorum cor,
 ex quo partes quattuor oriuntur:
 bonum et malum, vita et mors;
 et dominatrix illorum est assidua lingua.
 \verse Est vir peritus multorum eruditor
 et animae suae inutilis est.
 \verse Est qui sophistice loquitur et odibilis est;
 in omni cibo voluptatis defraudabitur.
 \verse Non est illi data a Domino gratia:
 omni enim sapientia defraudatus est.
 \verse Est sapiens animae suae sapiens,
 et fructus sensus illius super corpus suum.
 \verse Vir sapiens plebem suam erudit,
 et fructus sensus illius fideles sunt.
 \verse Vir sapiens implebitur benedictionibus,
 et omnes videntes illum beatum dicent.
 \verse Vita viri in numero dierum;
 dies autem Israel innumerabiles sunt.
 \verse Sapiens in populo hereditabit honorem,
 et nomen illius erit vivens in aeternum.
 \verse Fili, in vita tua tenta animam tuam
 et vide si quid obnoxium ei est: non des illi.
 \verse Non enim omnia omnibus expediunt,
 et non omni animae omne genus placet.
 \verse Noli avidus esse in omni epulatione
 et non te effundas super omnem escam.
 \verse In multis enim escis erit infirmitas,
 et aviditas appropinquabit usque ad choleram.
 \verse Propter crapulam multi obierunt;
 qui autem abstinens est, adiciet vitam.
 
\begin{biblechapter}
\verse Honora medicum propter necessitatem;
 etenim illum creavit Altissimus.
 \verse A Deo est enim illi sapientia,
 et a rege accipiet donationem.
 \verse Disciplina medici exaltabit caput illius,
 et in conspectu magnatorum collaudabitur.
 \verse Altissimus creavit de terra medicamenta,
 et vir prudens non abhorrebit illa.
 \verse Nonne a ligno indulcata est aqua amara,
 \verse ut agnoscerent homines virtutem illius?
 Et dedit hominibus scientiam Altissimus,
 ut honoraretur in mirabilibus suis.
 \verse In his curans mitigabit dolorem,
 et unguentarius faciet pigmenta suavitatis,
 ut non consumantur opera eius:
 \verse et salus super faciem terrae.
 \verse Fili, in tua infirmitate ne despicias teipsum,
 sed ora Dominum, et ipse curabit te.
 \verse Averte a delicto et dirige manus
 et ab omni peccato munda cor tuum;
 \verse da suavitatem et memoriam similaginis
 et impingua oblationem pro opibus tuis.
 Et da locum medico, 
\verse etenim illum Dominus creavit;
 et non discedat a te, quia opera eius sunt necessaria.
 \verse Est enim tempus, quando per manus illorum est solacium.
 \verse Ipsi vero Dominum deprecabuntur,
 ut dirigat ad rectam cognitionem
 et prosperet curationem.
 \verse Qui delinquit in conspectu eius, qui fecit eum,
 incidet in manus medici.
 \verse Fili, in mortuum produc lacrimas
 et, quasi dira passus, incipe lamentationem
 et secundum iudicium contege corpus illius
 et non despicias sepulturam illius.
 \verse Amare fer fletum et perfice lamentum
 \verse et fac luctum secundum meritum eius,
 uno die vel duobus propter detractionem,
 et consolare propter tristitiam.
 \verse A tristitia enim festinat mors,
 et tristitia cordis flectit virtutem.
 \verse In abductione permanet tristitia,
 et vita inopis maledictio cordis.
 \verse Ne dederis in tristitia cor tuum,
 sed repelle eam a te et memento novissimorum.
 \verse Ne ultra memineris: neque enim est conversio;
 et huic nihil proderis et teipsum pessimabis.
 \verse Memor esto iudicii eius, sic enim erit et tuum:
 mihi heri, et tibi hodie.
 \verse In requie mortui requiescere fac memoriam eius
 et consolare in illo in exitu spiritus sui.
 \verse Sapientia scribae in opportunitate vacationis;
 et, qui minoratur operatione, ipse sapientia replebitur.
 Qua sapientia replebitur, 
\verse qui tenet aratrum
 et qui gloriatur in iaculo stimuli?
 Boves agitat et conversatur in operibus eorum,
 et enarratio eius in filiis taurorum.
 \verse Cor suum dabit ad versandos sulcos,
 et vigilia eius in sagina vaccarum.
 \verse Sic omnis faber et architectus,
 qui noctem tamquam diem transigit,
 qui sculpit signacula sculptilia,
 et assiduitas eius variare picturam;
 cor suum dabit in similitudinem picturae,
 et vigilia sua perficere opus.
 \verse Sic faber ferrarius sedens iuxta incudem
 et considerans opus ferri;
 vapor ignis uret carnes eius,
 et in calore fornacis concertatur.
 \verse Vox mallei exsurdat aurem eius,
 et contra similitudinem vasis oculus eius.
 \verse Cor suum dabit in consummationem operum
 et vigilia sua ornare in perfectionem.
 \verse Sic figulus sedens ad opus suum,
 convertens pedibus suis rotam,
 qui in sollicitudine positus est semper propter opus suum,
 et in numero est omnis operatio eius;
 \verse in brachio suo formabit lutum
 et ante canos suos curvabit virtutem suam:
 \verse cor suum dabit, ut consummet linitionem,
 et vigilia sua mundare fornacem.
 \verse Omnes hi in manibus suis speraverunt,
 et unusquisque in arte sua sapiens est.
 \verse Sine his omnibus non aedificabitur civitas,
 \verse et non inhabitabunt nec inambulabunt.
 Verumtamen in consilium populi non requirentur
 et in ecclesiam non transilient;
 \verse super sellam iudicis non sedebunt
 et decretum iudicii non intellegent
 neque palam facient disciplinam et iudicium
 et in parabolis non invenientur;
 \verse sed creaturam laboris confirmabunt,
 et sollicitudo illorum in operatione artis.
 
\begin{biblechapter}
\verse Qui autem accommodat animam suam ad timorem Dei
 et in lege Altissimi meditatur,
 sapientiam omnium antiquorum exquiret
 et in prophetiis vacabit.
 \verse Narrationem virorum nominatorum conservabit
 et in versutias parabolarum simul introibit.
 \verse Occulta proverbiorum exquiret
 et in absconditis parabolarum conversabitur.
 \verse In medio magnatorum ministrabit
 et in conspectu principum apparebit.
 \verse In terram alienigenarum gentium pertransiet;
 bona enim et mala in hominibus tentabit.
 \verse Cor suum tradet ad vigilandum diluculo
 ad Dominum, qui fecit illum,
 et in conspectu Altissimi deprecabitur.
 \verse Aperiet os suum in oratione
 et pro delictis suis deprecabitur.
 \verse Si enim Dominus magnus voluerit,
 spiritu intellegentiae replebitur.
 \verse Ipse tamquam imbres mittet eloquia sapientiae suae
 et in oratione confitebitur Domino.
 \verse Et ipse diriget consilium et disciplinam
 et in absconditis eius considerabit.
 \verse Ipse palam faciet disciplinam doctrinae suae
 et in lege testamenti Domini gloriabitur.
 \verse Collaudabunt multi sapientiam eius,
 et usque in saeculum non delebitur.
 \verse Non recedet memoria eius,
 et nomen eius requiretur a generatione in generationem;
 \verse sapientiam eius enarrabunt gentes,
 et laudem eius enuntiabit ecclesia.
 \verse Si permanserit, nomen derelinquet plus quam mille,
 et, si requieverit, proderit sibi.
 \verse Adhuc meditabor et enarrabo;
 ut luna die duodecimo repletus sum.
 \verse Obaudite me, filii pii,
 et quasi rosa plantata super rivos aquarum florebit caro vestra;
 \verse quasi libanus odorem suavitatis habete,
 \verse florete flores quasi lilium.
 Date vocem et collaudate canticum
 et benedicite Dominum in omnibus operibus suis.
 \verse Date nomini eius magnificentiam
 et confitemini illi in laudatione eius
 et in canticis labiorum et citharis;
 et sic dicetis in confessione:
 \verse “ Opera Domini universa bona valde,
 et omne, quod praecepit, tempore suo erit! ”.
 Non est dicere: “ Quid est hoc? ” aut “ Ad quid istud? ”;
 omnia enim in tempore suo conquirentur.
 \verse In verbo eius stetit aqua sicut congeries,
 et in sermone oris illius exceptoria aquarum;
 \verse quoniam in praecepto ipsius placor fit,
 et non est minoratio in salutare ipsius.
 \verse Opera omnis carnis coram illo,
 et non est quidquam absconditum ab oculis eius.
 \verse A saeculo usque in saeculum respicit,
 et nihil est mirabile in conspectu eius.
 \verse Non est dicere: “ Quid est hoc! ” aut “ Ad quid istud? ”;
 omnia enim in usum suum creata sunt.
 \verse Benedictio illius quasi fluvius inundavit
 \verse et sicut cataclysmus aridam inebriavit.
 Sic ira ipsius gentes, quae non exquisierunt eum, disperdet,
 \verse quomodo convertit aquas in salsuginem.
 Viae illius sanctis directae sunt;
 sic peccatoribus offensiones in ira eius.
 \verse Bona bonis creata sunt ab initio,
 sic peccatoribus bona et mala.
 \verse Primum necessaria vitae hominum aqua, ignis et ferrum,
 sal, lac et panis similagineus et mel et sanguis uvae et oleum et vestimentum:
 \verse haec omnia sanctis in bona,
 sic et impiis et peccatoribus in mala convertentur.
 \verse Sunt spiritus, qui ad vindictam creati sunt
 et in furore suo confirmaverunt tormenta sua;
 \verse in tempore consummationis effundent virtutem
 et furorem eius, qui fecit illos, placabunt:
 \verse ignis, grando, fames et mors,
 omnia haec ad vindictam creata sunt;
 \verse bestiarum dentes et scorpii et serpentes
 et romphaea vindicans in exterminium impios:
 \verse in mandatis eius gaudebunt
 et super terram in necessitates praeparabuntur
 et in temporibus suis non praeterient verbum.
 \verse Propterea ab initio confirmatus
 et consiliatus sum et cogitavi et scriptis mandavi:
 \verse “ Opera Domini omnia bona,
 et omnem usum hora sua subministrabit ”.
 \verse Non est dicere: “ Hoc illo nequius est ”:
 omnia enim in tempore suo comprobabuntur.
 \verse Et nunc in omni corde et ore collaudate
 et benedicite nomen Domini.
 
\begin{biblechapter}
\verse Occupatio magna creata est omnibus hominibus,
 et iugum grave super filios Adam
 a die exitus de ventre matris eorum
 usque in diem reditus in matrem omnium:
 \verse cogitationes eorum et timores cordis,
 adinventio exspectationis, dies finitionis.
 \verse A residente super sedem gloriosam,
 usque ad humiliatum in terra et cinere;
 \verse ab eo, qui portat hyacinthum et coronam,
 usque ad eum, qui operitur lino crudo:
 furor, zelus, tumultus, fluctuatio
 et timor mortis et iracundia perseverans et contentio.
 \verse Et in tempore requiei in cubili
 somnus noctis immutat scientiam eius.
 \verse Modicum tamquam nihil in requie,
 et ab eo in somnis quasi in die laborat
 \verse conturbatus in visu cordis sui
 tamquam qui evaserit a facie belli;
 in tempore somni necessarii exsurrexit
 et admirans ad nullum timorem.
 \verse Cum omni carne ab homine usque ad pecus;
 et super peccatores septuplum amplius:
 \verse ad haec mors, sanguis, contentio et romphaea,
 oppressiones, fames et contritio et flagella.
 \verse Super iniquos creata sunt haec omnia,
 et propter illos factus est cataclysmus.
 \verse Omnia, quae de terra sunt, in terram convertentur,
 et omnia, quae de aquis sunt, in mare revertentur.
 \verse Omne munus corruptionis et iniquitas delebitur,
 et fides in saeculum stabit.
 \verse Substantiae iniustorum sicut fluvius siccabuntur
 et sicut tonitruum magnum in pluvia evanescent.
 \verse In aperiendo manus suas laetabitur,
 sic praevaricatores in consummationem deficient.
 \verse Nepotes impiorum non multiplicabunt ramos,
 et radices immundae super cacumen petrae.
 \verse Viriditas super omnem aquam et ad oram fluminis
 ante omne fenum evelletur.
 \verse Gratia sicut paradisus in benedictionibus,
 et eleemosyna in saeculum permanet.
 \verse Vita sibi sufficientis et operarii condulcabitur,
 et super utrumque, eius qui inveniet thesaurum.
 \verse Filii et aedificatio civitatis confirmant nomen,
 et super haec mulier immaculata computabitur.
 \verse Vinum et musica laetificant cor,
 et super utraque dilectio sapientiae.
 \verse Tibiae et psalterium suavem faciunt melodiam,
 et super utraque lingua suavis.
 \verse Gratiam et speciem desiderabit oculus,
 et super haec virides sationes.
 \verse Amicus et sodalis in tempore convenientes,
 et super utrosque mulier cum viro.
 \verse Fratres et adiutorium in tempore tribulationis,
 et super utraque eleemosyna liberabit.
 \verse Aurum et argentum firmant pedem,
 et super utrumque consilium acceptum habetur.
 \verse Facultates et virtutes exaltant cor,
 et super haec timor Domini.
 \verse Non est in timore Domini minoratio,
 et non est super eo inquirere adiutorium.
 \verse Timor Domini sicut paradisus benedictionis,
 et super omnem gloriam obumbratio eius.
 \verse Fili, in tempore vitae tuae ne indigeas;
 melius est enim mori quam indigere.
 \verse Vir respiciens in mensam alienam,
 non est vita eius in computatione vitae.
 Contaminat enim animam suam cibis alienis;
 \verse vir autem disciplinatus et eruditus custodiet se.
 \verse In ore impudentis condulcabitur mendicatio,
 et in ventre eius ignis ardebit.
 
\begin{biblechapter}
\verse O mors, quam amara est memoria tua
 homini pacem habenti in substantiis suis,
 \verse viro quieto et, cuius viae directae sunt in omnibus,
 et adhuc valenti accipere voluptatem!
 \verse O mors, bonum est iudicium tuum
 homini indigenti et, qui minoratur viribus,
 \verse defecto aetate et, cui de omnibus cura est,
 qui fiduciam amisit et perdidit patientiam!
 \verse Noli metuere iudicium mortis;
 memento eorum, qui ante te fuerunt
 et qui superventuri sunt tibi:
 hoc iudicium a Domino omni carni;
 \verse et quid resistis beneplacito Altissimi?
 Sive decem sive centum sive mille anni,
 \verse non est enim in inferno accusatio vitae.
 \verse Filii abominationum fiunt filii peccatorum,
 et qui conversantur in sedibus impiorum;
 \verse filiorum peccatorum periet hereditas,
 et cum semine illorum assiduitas opprobrii.
 \verse De patre impio queruntur filii,
 quoniam propter illum sunt in opprobrio.
 \verse Vae vobis, viri impii,
 qui dereliquistis legem Domini Altissimi!
 \verse Et, si nati fueritis, in maledictione nascemini;
 et, si mortui fueritis, in maledictione erit pars vestra.
 \verse Omnia, quae de terra sunt, in terram convertentur,
 sic impii a maledicto in perditionem.
 \verse Luctus hominum in corpore ipsorum;
 nomen autem impiorum non bonum delebitur.
 \verse Curam habe de bono nomine;
 hoc enim magis permanebit tibi
 quam mille thesauri pretiosi et magni:
 \verse bonae vitae numerus dierum,
 bonum autem nomen permanebit in aevum.
 \verse Melior est homo, qui abscondit stultitiam suam,
 quam homo, qui abscondit sapientiam suam.
 Sapientia enim abscondita et thesaurus invisibilis,
 quae utilitas in utrisque?
 \verse Disciplinam in pace conservate, filii;
 \verse verumtamen reveremini iudicium meum:
 \verse non est enim bonum omnem reverentiam observare,
 et non omnis pudor probatus.
 \verse Erubescite a patre et a matre de fornicatione
 et a praesidente et a potente de mendacio,
 \verse a principe et a iudice de delicto,
 a synagoga et plebe de iniquitate,
 \verse a socio et amico de iniustitia
 et de loco, in quo habitas, 
\verse de furto,
 de veritate Dei et testamento,
 de impositione cubiti super mensam
 et a despectione dati et accepti,
 \verse a salutantibus de silentio,
 a respectu mulieris fornicariae
 et ab aversione vultus cognati
 \verse et ab auferendo partem et non restituendo
 \verse et a respiciendo mulierem alieni viri
 et a curiositate in ancillam eius,
 neque steteris ad lectum eius;
 \verse ab amicis de sermonibus improperii,
 et, cum dederis, ne improperes;
 
\begin{biblechapter}
\verse et ab iteratione sermonis auditus
 et a revelatione sermonis absconditi.
 Et eris vere sine confusione
 et invenies gratiam in conspectu omnium hominum.
 Ne pro his omnibus confundaris,
 ne accipias personam, ut delinquas:
 \verse de lege Altissimi et testamento
 et de iudicio, iustificans impium,
 \verse de ratione sociorum et viatorum
 et de datione hereditatis alienorum,
 \verse de aequalitate staterae et ponderum,
 de acquisitione multorum et paucorum,
 \verse de pretio emptionis negotiatorum
 et de multa disciplina filiorum
 et servo pessimo latus sanguinare.
 \verse Super mulierem nequam bonum est signum;
 \verse ubi manus multae sunt, claude
 et, quodcumque trades, numera et appende:
 datum vero et acceptum omne describe.
 \verse De disciplina insensati et fatui
 et de seniore, qui iudicatur de fornicatione;
 et eris eruditus in veritate
 et probatus in conspectu omnium vivorum.
 \verse Filia patri est abscondita vigilia,
 et sollicitudo eius aufert somnum:
 in adulescentia sua, ne forte adulta efficiatur,
 viro nuptum locata, ne odibilis fiat;
 \verse ne quando polluatur in virginitate sua
 et in paternis suis gravida inveniatur;
 ne forte viro desponsata transgrediatur
 aut, cum eo commorata, ne sterilis inveniatur.
 \verse Super filiam immodestam confirma custodiam,
 ne quando faciat te in opprobrium venire inimicis,
 in detractionem in civitate et obiectionem plebis,
 et confundat te in multitudine populi.
 \verse Omni homini ne det speciem
 et in medio mulierum non commoretur;
 \verse de vestimentis enim procedit tinea,
 et a muliere iniquitas mulieris.
 \verse Melior est enim iniquitas viri quam mulier benefaciens,
 et mulier confundens in opprobrium.
 \verse Memor ero igitur operum Domini
 et, quae vidi, annuntiabo:
 in sermonibus Domini opera eius,
 et factum est in voluntate sua iudicium.
 \verse Sol illuminans per omnia respexit,
 et gloria Domini plenum est opus eius.
 \verse Non valent sancti Domini
 enarrare omnia mirabilia eius.
 Confirmavit Dominus exercitus suos
 stabiliri coram gloria sua.
 \verse Abyssum et cor hominum investigavit
 et in astutia eorum excogitavit.
 \verse Cognovit enim Dominus omnem scientiam
 et inspexit in signum aevi
 annuntians, quae praeterierunt et quae superventura sunt,
 et revelans vestigia occultorum.
 \verse Non praeterit illum omnis cogitatus,
 et non abscondit se ab eo ullus sermo.
 \verse Magnalia sapientiae suae ordinavit,
 unicus est ante saeculum et usque in saeculum;
 neque augetur 
\verse neque minuitur
 et non eget alicuius consilio.
 \verse Quam desiderabilia omnia opera eius,
 et tamquam scintilla spectatu!
 \verse Omnia haec vivunt et manent in saeculum,
 et in omni necessitate omnia obaudiunt ei;
 \verse omnia duplicia, unum contra unum,
 et non fecit quidquam deficiens.
 \verse Alterum alterius confirmat bonum;
 et quis satiabitur videns gloriam eius?
 
\begin{biblechapter}
\verse Gloria altitudinis firmamentum puritatis,
 species caeli in visione gloriae.
 \verse Sol in apparitione annuntians in processu:
 vas admirabile, opus Excelsi.
 \verse In meridiano suo exurit terram;
 et in conspectu ardoris eius quis poterit sustinere?
 Fornacem ventilans in operibus ardoris tripliciter,
 \verse sol exurens montes, vapores igneos exsufflans
 et refulgens radiis suis obcaecat oculos.
 \verse Magnus Dominus, qui fecit illum
 et sermonibus eius festinavit iter.
 \verse Et luna stat in tempus suum,
 in ostensionem temporis et signum aevi.
 \verse A luna signum diei festi;
 luminare, quod minuitur in consummatione.
 \verse Mensis secundum nomen eius est,
 crescens mirabiliter in consummatione.
 \verse Vas castrorum in excelsis,
 in firmamento caeli resplendens gloriose.
 \verse Species caeli gloria stellarum,
 mundum illuminans in excelsis Domini.
 \verse In verbis Sancti stabunt iuxta praeceptum
 et non deficient in vigiliis suis.
 \verse Vide arcum et benedic eum, qui fecit illum;
 valde speciosus est in splendore suo.
 \verse Gyravit caelum in circuitu gloriae suae,
 manus Excelsi tetenderunt illum.
 \verse Imperio suo acceleravit nivem
 et properat coruscationes iudicii sui.
 \verse Propterea aperti sunt thesauri,
 et evolaverunt nebulae sicut aves.
 \verse In magnitudine sua firmavit nubes,
 et confracti sunt lapides grandinis. Vox tonitrui eius tremefacit terram,
 \verse in conspectu eius commovebuntur montes.
 In voluntate eius aspirabit notus,
 \verse tempestas aquilonis et congregatio spiritus.
 \verse Et, sicut aves deponentes ad sedendum, aspergit nivem,
 et, sicut locusta demergens, descensus eius:
 \verse pulchritudinem candoris eius admirabitur oculus,
 et super imbrem eius expavescet cor.
 \verse Gelu sicut salem effundet super terram,
 et, dum gelaverit, fit tamquam cacumina tribuli.
 \verse Frigidus ventus aquilo flabit,
 et gelabit crystallus super aquam;
 super omnem congregationem aquarum requiescet,
 et sicut lorica induet se aqua.
 \verse Devorabit montes et exuret desertum
 et exstinguet viridem sicut ignis.
 \verse Medicina omnium in festinatione nebulae,
 et ros obvians ab ardore hilarescet.
 \verse Cogitatione sua placavit abyssum
 et plantavit in illa insulas.
 \verse Qui navigant mare, enarrant pericula eius,
 et audientes auribus nostris admiramur.
 \verse Illic praeclara opera et mirabilia,
 varia bestiarum genera et omnium pecorum et creatura belluarum.
 \verse Propter ipsum iter prosperat angelus eius,
 et in sermone eius composita sunt omnia.
 \verse Multa dicemus et deficiemus verbis; consummatio autem sermonum: “ Ipse est omnia!”.
 \verse Glorificantes ad quid valebimus?
 Ipse enim Magnus super omnia opera sua.
 \verse Terribilis Dominus et magnus vehementer,
 et mirabilis potentia ipsius.
 \verse Glorificantes Dominum exaltate, quantumcumque potueritis:
 supervalebit enim adhuc,
 et admirabilis magnificentia eius.(\verse)
 \verse Exaltantes eum replemini virtute;
 ne laboretis, non enim pervenietis usquam.
 \verse Quis vidit eum et enarrabit?
 Et quis magnificabit eum sicut est?
 \verse Multa abscondita sunt maiora his;
 pauca enim vidimus operum eius.
 \verse Omnia autem Dominus fecit
 et pie agentibus dedit sapientiam.
 
\begin{biblechapter}
 \verse Laus patrum.
 Laudemus viros gloriosos
 et parentes nostros in generatione sua.
 \verse Multam gloriam fecit Dominus,
 magnificentiam suam a saeculo.
 \verse Dominantes in potestatibus suis,
 homines magni virtute
 et prudentia sua praediti,
 nuntiantes in prophetiis,
 \verse regentes populum in consiliis
 et peritia scripturae populos;
 verba sapientiae in disciplina eorum,
 \verse requirentes modos musicos
 et narrantes carmina scripturarum;
 \verse homines divites innixi virtute,
 pulchritudinis studium habentes,
 pacificantes in domibus suis.
 \verse Omnes isti in generationibus gentis suae gloriam adepti sunt
 et a diebus suis habentur in laudibus.
 \verse De illis nati sunt, qui reliquerunt nomen
 narrandi laudes eorum.
 \verse Et sunt quorum non est memoria:
 perierunt quasi qui non fuerint;
 et nati sunt quasi non nati,
 et filii ipsorum post ipsos.
 \verse Sed hi viri misericordiae sunt,
 quorum pietates non fuerunt in oblivione.
 \verse Cum semine eorum permanent,
 bona hereditas, nepotes eorum,
 \verse et in testamentis stetit semen eorum;
 \verse et filii eorum propter illos.
 Usque in aeternum manet semen eorum,
 et gloria eorum non derelinquetur.
 \verse Corpora ipsorum in pace sepulta sunt,
 et nomen eorum vivit in generationem et generationem;
 \verse sapientiam ipsorum narrent populi,
 et laudem eorum nuntiet ecclesia.
 \verse Henoch placuit Deo et translatus est in paradisum,
 ut det gentibus paenitentiam.
 \verse Noe inventus est perfectus iustus
 et in tempore iracundiae factus est reconciliatio;
 \verse propter eum dimissum est reliquum terrae,
 cum factum est diluvium:
 \verse testamenta saeculi posita sunt apud illum,
 ne deleri posset diluvio omnis caro.
 \verse Abraham magnus pater multitudinis gentium,
 et non est inventa macula in gloria eius;
 qui conservavit legem Excelsi
 et fuit in testamento cum illo.
 \verse In carne eius stare fecit testamentum,
 et in tentatione inventus est fidelis.
 \verse Ideo iure iurando statuit illi
 benedici gentes in semine eius,
 crescere illum quasi terrae cumulum
 \verse et ut stellas exaltare semen eius
 et hereditare illos a mari usque ad mare
 et a Flumine usque ad terminos terrae.
 \verse Et in Isaac eodem modo statuit
 propter Abraham patrem eius.
 \verse Benedictionem omnium gentium dedit illi Dominus
 et testamentum confirmavit super caput Iacob.
 \verse Agnovit eum in benedictionibus suis
 et dedit illi hereditatem
 et divisit illi partem in tribubus duodecim.
 \verse Et eduxit ex illo hominem misericordiae
 invenientem gratiam in oculis omnis carnis,
 
\begin{biblechapter}
\verse dilectum a Deo et hominibus
 Moysen, cuius memoria in benedictione est.
 \verse Similem illum fecit in gloria sanctorum
 et magnificavit eum in timore inimicorum
 et in verbis suis signa acceleravit.
 \verse Glorificavit illum in conspectu regum
 et ius dedit illi ad populum suum
 et ostendit illi gloriam suam.
 \verse In fide et lenitate ipsius sanctum fecit illum
 et elegit eum ex omni carne.
 \verse Auditam fecit illi vocem suam
 et induxit illum in nubem;
 \verse et dedit illi coram praecepta
 et legem vitae et disciplinae,
 docere Iacob testamentum suum
 et iudicia sua Israel.
 \verse Excelsum fecit Aaron sanctum similem ei,
 fratrem eius de tribu Levi.
 \verse Statuit illum in testamentum aeternum
 et dedit illi sacerdotium gentis,
 et beatificavit illum in gloria
 \verse et circumcinxit eum zona gloriae.
 Et induit eum perfecto decore
 et coronavit eum in vasis virtutis:
 \verse femoralia et tunicam et umerale posuit ei
 et cinxit illum tintinnabulis aureis,
 malis granatis plurimis in gyro,
 \verse ut daret sonitum in incessu suo,
 auditum faceret sonitum in templo
 in memoriam filiis gentis suae.
 \verse Stola sancta auro et hyacintho et purpura,
 opus textile rationale iudicii et cingulum,
 \verse tortum coccum opere artificis,
 gemmae pretiosae super rationale
 in ligatura auri, opere lapidarii sculptis,
 in memoriam cum scriptura sculpta secundum numerum tribuum Israel.
 \verse Corona aurea super mitram eius,
 lamina cum signo sanctitatis, gloria honoris,
 opus virtutis et desideria oculorum, perfecta pulchritudo.
 \verse Sic pulchra ante ipsum non fuerunt talia in sempiternum;
 \verse non induet illa alienigena aliquis,
 sed tantum filii ipsius soli
 et nepotes eius per omne tempus.
 \verse Sacrificium ipsius consumitur
 igne cotidie iuge bis.
 \verse Complevit Moyses manus eius
 et unxit illum oleo sancto.
 \verse Factum est illi in testamentum aeternum
 et semini eius sicut dies caeli:
 ministrare illi et fungi sacerdotio
 et benedicere populum suum in nomine eius.
 \verse Ipsum elegit ab omni vivente
 offerre sacrificium Deo, incensum et adipes
 et incendere bonum odorem et memoriale
 et placare pro populo suo.
 \verse Et dedit illi in praeceptis suis potestatem,
 in testamentis iudiciorum:
 docere Iacob testimonia
 et in lege sua lucem dare Israel.
 \verse Quia contra illum steterunt alieni,
 et propter invidiam circumdederunt illum homines in deserto,
 qui erant cum Dathan et Abiram,
 et congregatio Core in iracundia furoris sui.
 \verse Vidit Dominus Deus, et non placuit illi,
 et consumpti sunt in impetu iracundiae eius.
 \verse Fecit illis monstra
 et consumpsit illos in flamma ignis.
 \verse Et addidit Aaron gloriam
 et dedit illi hereditatem
 et primitias frugum terrae divisit illi.
 \verse Panem ipsis in primis paravit in satietatem,
 nam et sacrificia Domini edent,
 quae dedit illi et semini eius.
 \verse Ceterum in terra gentis non hereditabit,
 et pars non est illi in gente:
 ipse est enim pars eius et hereditas.
 \verse Et Phinees filius Eleazari tertius in gloria est,
 zelando timorem Domini;
 \verse et stetit in ruptura pro gente,
 in bonitate et alacritate animae suae placuit Deo pro lsrael.
 \verse Ideo statuit illi testamentum pacis,
 principem sanctorum et gentis suae,
 ut sit illi et semini eius
 sacerdotii dignitas in aeternum.
 \verse Et testamentum David regi filio Iesse de tribu Iudae,
 hereditas viri coram gloria eius,
 hereditas Aaron et semini eius.
 Det vobis sapientiam in cor vestrum
 iudicare gentem suam in iustitia,
 ne abolerentur bona ipsorum,
 et gloria ipsorum in generationes aeternas.
 
\begin{biblechapter}
\verse Fortis in bello Iesus filius Nun,
 successor Moysi in prophetis,
 qui fuit secundum nomen suum
 \verse maximus in salutem electorum Dei,
 qui vindicet in insurgentes hostes,
 ut heredem faciat Israel.
 \verse Quam gloriam adeptus est in tollendo manus suas
 et iactando contra civitatem acinacem!
 \verse Quis ante illum restitit?
 Nam bella Domini ipse perduxit.
 \verse An non in manu eius impeditus est sol,
 et una dies facta est quasi duo?
 \verse Invocavit Altissimum potentem
 in oppugnando inimicos undique;
 et audivit illum magnus Dominus
 in saxis grandinis virtutis valde fortis.
 \verse Impetum fecit contra gentem hostilem
 et in descensu perdidit contrarios,
 \verse ut cognoscant gentes armaturam eius,
 quia contra Deum bellum eorum est:
 etenim secutus est a tergo Potentis.
 \verse Et in diebus Moysi misericordiam fecit,
 ipse et Chaleb filius Iephonne,
 stare contra congregationem,
 prohibere gentem a peccatis
 et perfringere murmur malitiae.
 \verse Ideo et ipsi duo liberati sunt
 a numero sescentorum milium peditum,
 ut inducerent illos in hereditatem,
 in terram, quae manat lac et mel.
 \verse Et dedit Dominus ipsi Chaleb fortitudinem,
 et usque in senectutem permansit illi,
 ut ascenderet in excelsum terrae locum;
 et semen ipsius obtinuit hereditatem,
 \verse ut viderent omnes filii Israel
 quia bonum est obsequi Domino.
 \verse Et iudices singuli suo nomine,
 quorum non est corruptum cor,
 qui non aversi sunt a Domino:
 \verse sit memoria illorum in benedictione, et ossa eorum pullulent de loco suo,
 \verse et nomen eorum renovet
 filiis illorum sanctorum virorum.
 \verse Dilectus a Domino suo Samuel
 propheta Domini instituit imperium
 et unxit principes in gente sua.
 \verse In lege Domini congregationem iudicavit,
 et visitavit Deus Iacob,
 et in fide sua probatus est propheta;
 \verse et cognitus est in verbis suis fidelis in visione.
 \verse Et invocavit Dominum omnipotentem,
 in oppugnando hostes circumstantes undique,
 in oblatione agni lactentis.
 \verse Et intonuit de caelo Dominus
 et in sonitu magno auditam fecit vocem suam
 \verse et contrivit principes Tyriorum
 et omnes duces Philisthim.
 \verse Et ante tempus dormitionis aeternae
 testimonium praebuit in conspectu Domini et christi eius:
 “ Pecunias et usque ad calceamenta
 ab omni carne non accepi ”;
 et non accusavit illum homo.
 \verse Et, postquam dormivit, prophetavit
 et notum fecit regi et ostendit illi finem vitae suae
 et exaltavit vocem suam de terra in prophetia
 ad delendam impietatem gentis.
 
\begin{biblechapter}
\verse Post hunc surrexit Nathan
 propheta in diebus David.
 \verse Et quasi adeps separatus a sacrificio salutari,
 sic David a filiis Israel.
 \verse Cum leonibus lusit quasi cum agnis
 et in ursis similiter fecit, sicut in agnis ovium.
 \verse In iuventute sua numquid non occidit gigantem
 et abstulit opprobrium de gente?
 \verse In agitando manu fundam
 deiecit exsultationem Goliath.
 \verse Nam invocavit Dominum altissimum,
 et dedit in dextera eius virtutem
 tollere hominem fortem in bello
 et exaltare cornu gentis suae.
 \verse Sic de decem milibus glorificaverunt eum
 et laudaverunt eum in benedictionibus Domini,
 in offerendo illi coronam gloriae.
 \verse Contrivit enim inimicos undique
 et humiliavit Philisthim contrarios,
 usque in hodiernum diem contrivit cornu ipsorum.
 \verse In omni opere suo dedit confessionem
 Sancto et Excelso in verbo gloriae;
 \verse de omni corde suo laudavit Dominum
 et dilexit Deum, qui fecit illum.
 \verse Et stare fecit cantores contra altare
 et in sono eorum dulces fecit modos.
 \verse Dedit in celebrationibus decus
 et ornavit tempora usque ad consummationem anni,
 dum laudarent nomen sanctum Domini,
 et ante mane resonaret sanctuarium.
 \verse Dominus purgavit peccata ipsius
 et exaltavit in aeternum cornu eius et dedit illi testamentum regni
 et sedem gloriae in Israel.
 \verse Post ipsum surrexit filius sensatus,
 et propter illum habitavit in securitate.
 \verse Salomon imperavit in diebus pacis;
 cui Deus requiem dedit in circuitu,
 ut conderet domum in nomine suo et pararet sanctuarium in sempiternum. Quemadmodum eruditus es in iuventute tua 
 \verse et impletus es quasi flumen sapientia!
 Terram retexit anima tua,
 \verse et replesti eam in comparationibus aenigmatum.
 Ad insulas longe divulgatum est nomen tuum,
 et dilectus es in pace tua.
 \verse In cantilenis et proverbiis
 et comparationibus et interpretationibus
 te miratae sunt terrae.
 \verse In nomine Domini Dei,
 cui est cognomen Deus Israel,
 \verse collegisti quasi orichalcum aurum
 et ut plumbum cumulasti argentum.
 \verse Sed reclinasti femora tua mulieribus,
 subiugatus es in corpore tuo.
 \verse Dedisti maculam in gloria tua
 et profanasti semen tuum,
 inducens iracundiam ad liberos tuos
 et ingemiscere faciens super stultitia tua,
 \verse ut fieret imperium bipartitum,
 et ex Ephraim inciperet imperium infidele.
 \verse Deus autem non derelinquet misericordiam suam
 et non corrumpet nec delebit verba sua
 neque perdet a stirpe nepotes electi sui
 et semen eius, qui diligit Dominum, non corrumpet.
 \verse Dedit autem reliquum Iacob
 et David de ipso stirpem.
 \verse Et finem habuit Salomon cum patribus suis
 \verse et dereliquit post se de semine suo
 gentis stultitiam 
\verse et imminutum prudentia,
 Roboam, qui avertit gentem consilio suo.
 \verse Et Ieroboam filius Nabat, qui peccare fecit Israel
 et dedit viam peccandi Ephraim;
 et plurima redundaverunt peccata ipsorum valde,
 \verse ita ut expelleret illos a terra sua.
 \verse Et exquisierunt omnes nequitias,
 usque dum perveniret super illos vindicta.
 
\begin{biblechapter}
\verse Et surrexit Elias propheta quasi ignis,
 et verbum ipsius quasi facula ardebat.
 \verse Qui induxit in illos famem
 et zelo suo paucos fecit eos.
 \verse Verbo Domini continuit caelum
 et deiecit de caelo ignem ter.
 \verse Quam amplificatus es, Elias, in mirabilibus tuis!
 Et quis potest similiter gloriari tibi?
 \verse Qui suscitasti mortuum de sorte mortis
 ab inferis in verbo Domini.
 \verse Qui deiecisti reges ad perniciem
 et gloriosos de lecto suo
 et confregisti facile potentiam ipsorum;
 \verse qui audis in Sinai indicium
 et in Horeb iudicia vindictae.
 \verse Qui ungis reges ad retributionem
 et prophetas facis successores post te;
 \verse qui receptus es in turbine ignis
 et in curru equorum igneorum;
 \verse qui scriptus es paratus in tempora
 lenire iracundiam Domini ante furorem,
 convertere cor patris ad filium
 et restituere tribus Iacob.
 \verse Beati sunt, qui te viderunt
 et in amicitia tua dormierunt!
 \verse Nam et nos vita quidem vivemus,
 post mortem autem non erit tale nomen nostrum.
 \verse Elias quidem in turbine tectus est,
 et in Eliseo completus est spiritus eius.
 In diebus suis non pertimuit principem,
 et potentia nemo vicit illum;
 \verse nec superavit illum verbum aliquod,
 et mortuum prophetavit corpus eius.
 \verse In vita sua fecit monstra
 et in morte mirabilia operatus est.
 \verse In omnibus istis non paenituit populum,
 et non recesserunt a peccatis suis,
 usque dum eiecti sunt de terra sua
 et dispersi sunt in omnem terram;
 \verse et relicta est gens perpauca,
 et princeps in domo David.
 \verse Quidam ipsorum fecerunt, quod placeret Deo;
 alii autem multiplicaverunt peccata.
 \verse Ezechias munivit civitatem suam
 et induxit in medium ipsius aquam et fodit ferro rupem
 et aedificavit ad aquam puteum.
 \verse In diebus ipsius ascendit Sennacherib
 et misit Rabsacen et discessit,
 et sustulit manum suam in Sion
 et superbus factus est in exaltatione sua.
 \verse Tunc mota sunt corda et manus ipsorum,
 et doluerunt quasi parturientes mulieres
 \verse et invocaverunt Dominum misericordem
 et expandentes manus suas extulerunt ad eum,
 et Sanctus audivit cito vocem ipsorum.
 \verse Non est commemoratus peccatorum illorum
 neque dedit illos inimicis suis,
 sed purgavit eos in manu Isaiae sancti prophetae;
 \verse percussit castra Assyriorum
 et contrivit illos angelus eius.
 \verse Nam fecit Ezechias quod placuit Deo
 et fortiter ivit in via David patris sui,
 quam mandavit illi Isaias propheta, magnus et fidelis in visione sua.
 \verse In diebus ipsius retro rediit sol,
 et addidit regi vitam.
 \verse Spiritu magno vidit ultima
 et consolatus est lugentes in Sion;
 usque in sempiternum ostendit futura
 \verse et abscondita, antequam evenirent.
 
\begin{biblechapter}
\verse Memoria Iosiae in compositionem odoris
 facta opere pigmentarii;
 \verse in omni ore quasi mel indulcabitur eius memoria
 et ut musica in convivio vini.
 \verse Ipse est directus divinitus in paenitentia gentis
 et tulit abominationes impietatis;
 \verse et gubernavit ad Dominum cor suum
 et in diebus peccatorum corroboravit pietatem.
 \verse Praeter David et Ezechiam et Iosiam,
 omnes peccatum commiserunt;
 \verse nam reliquerunt legem Altissimi reges Iudae
 et contempserunt timorem Dei;
 \verse dederunt enim regnum suum aliis
 et gloriam suam alienae genti;
 \verse incenderunt electam sanctuarii civitatem
 et desertas fecerunt vias ipsius in manu Ieremiae.
 \verse Nam male tractaverunt illum,
 ipse autem a ventre matris consecratus est propheta
 evertere et eruere et perdere
 et iterum aedificare et plantare et renovare.
 \verse Ezechiel, qui vidit visionem gloriae,
 quam ostendit illi in curru cherubim.
 \verse Nam et commemoratus est inimicorum in imbre
 et benefacere illis, qui ostenderunt rectas vias.
 \verse Et duodecim prophetarum ossa pullulent de loco suo;
 nam corroboraverunt Iacob
 et redemerunt eos in fide virtutis.
 \verse Quomodo amplificemus Zorobabel?
 Nam et ipse quasi signum in dextera manu;
 \verse sic et Iesua filius Iosedec.
 Qui in diebus suis aedificaverunt domum
 et exaltaverunt templum sanctum Domino
 paratum in gloriam sempiternam.
 \verse Et Nehemias: in memoria multi temporis,
 qui erexit nobis muros eversos
 et stare fecit portas et seras;
 qui erexit domos nostras.
 \verse Nemo creatus est in terra qualis Henoch;
 nam et ipse assumptus est a terra.
 \verse Neque ut Ioseph natus est homo,
 princeps fratrum, firmamentum gentis;
 \verse et ossa ipsius visitata sunt
 et post mortem prophetaverunt.
 \verse Seth et Sem apud homines gloriam adepti sunt
 et super omnem animam in origine Adam.
 
\begin{biblechapter}
\verse Simon, Oniae filius, sacerdos magnus,
 qui in vita sua suffulsit domum
 et in diebus suis corroboravit templum.
 \verse Templi etiam altitudo ab ipso fundata est,
 substructura elata parietis templi.
 \verse In diebus ipsius excisa est piscina aquarum,
 lacus, quasi maris superficies eius.
 \verse Qui curavit gentem suam a latrone
 et firmavit eam ab obsidione.
 \verse Quam gloriosus apparuit, cum prospiceret e tabernaculo
 in egressu domus velamenti!
 \verse Quasi stella matutina in medio nebulae
 et quasi luna plena in diebus festi
 \verse et quasi sol refulgens super templum Dei.
 \verse Quasi arcus refulgens inter nebulas gloriae
 et quasi flos rosarum in diebus vernis
 et quasi lilia, quae sunt in transitu aquae,
 et quasi flos Libani in diebus aestatis;
 \verse quasi ignis effulgens et tus ardens in igne,
 \verse quasi vas auri solidum
 ornatum omni lapide pretioso,
 \verse quasi oliva pullulans fructibus
 et cupressus in nubes se extollens,
 in accipiendo ipsum stolam gloriae et vestiri eum in consummationem magnificentiae.
 \verse In ascensu altaris sancti,
 cum gloriam daret peribolo sanctuarii
 \verse et acciperet partes de manu sacerdotum,
 et ipse stans iuxta aram,
 et circa illum corona fratrum,
 quasi plantatio cedri in monte Libano,
 \verse sic circa illum steterunt quasi rami palmae
 omnes filii Aaron in gloria sua.
 \verse Oblatio autem Domini in manibus ipsorum
 coram omni synagoga Israel,
 et consummatione fungens in ara,
 ordinans oblationem Omnipotentis.
 \verse Porrexit manum suam in libatione
 et libavit de sanguine uvae;
 \verse effudit in fundamento altaris
 odorem divinum excelso Principi.
 \verse Tunc exclamaverunt filii Aaron,
 in tubis productilibus sonuerunt
 et auditam fecerunt vocem magnam
 in memoriam coram Deo altissimo.
 \verse Tunc omnis populus simul properaverunt
 et ceciderunt in faciem super terram
 adorare Dominum Deum suum
 et dare preces omnipotenti Deo excelso.
 \verse Et laudaverunt psallentes in vocibus suis,
 et magnus resonabat cantus suavitatis plenus.
 \verse Et rogavit populus Dominum excelsum in prece coram Misericorde,
 usque dum perfectus est honor Domini;
 et munus suum perfecerunt.
 \verse Tunc descendens manus suas extulit
 in omnem congregationem filiorum Israel
 dare benedictionem Domini in labiis suis
 et in nomine ipsius gloriari;
 \verse et iteraverunt adorationem suam,
 ut acciperent benedictionem Altissimi.
 \verse Et nunc benedicite Deum omnium,
 qui magna facit in omni terra,
 exaltans dies nostros a ventre matris nostrae
 et faciens nobiscum secundum suam misericordiam.
 \verse Det nobis iucunditatem cordis
 et fieri pacem in diebus nostris in Israel per dies sempiternos;
 \verse credere Israel nobiscum esse Dei misericordiam,
 ut liberet nos in diebus nostris.
 \verse Duas gentes odit anima mea,
 tertia autem non est quidem gens:
 \verse qui sedent in monte Seir et Philisthim
 et stultus populus, qui habitat in Sichimis.
 \verse Doctrinam sapientiae et disciplinae
 scripsit in codice isto Iesus filius Sirach, Ierosolymita,
 qui effudit sapientiam de corde suo.
 \verse Beatus, qui in istis versatur sermonibus;
 qui ponit illa in corde suo, sapiens erit semper.
 \verse Si enim haec fecerit, ad omnia valebit,
 quia timor Domini vestigium eius est.
 
\begin{biblechapter}
\verse Oratio Iesu filii Sirach.
 “ Confitebor tibi, Domine rex; et collaudabo te Deum salvatorem meum.
 \verse Confitebor nomini tuo,
 quoniam adiutor et protector factus es mihi
 \verse et liberasti corpus meum a perditione,
 a laqueo linguae iniquae
 et a labiis operantium mendacium,
 et in conspectu insurgentium factus es mihi adiutor.
 \verse Et liberasti me,
 secundum magnitudinem misericordiae et nominis tui,
 a laqueis praeparatis ad escam,
 \verse de manibus quaerentium animam meam
 et de multis tribulationibus, quae circumdederunt me,
 \verse a pressura flammae, quae circumdedit me,
 et de medio ignis, ubi non sum aestuatus;
 \verse de altitudine ventris inferi
 et a lingua coinquinata et a verbo mendacii,
 a iaculo linguae iniustae.
 \verse Appropinquavit usque ad mortem anima mea,
 \verse et vita mea appropinquans erat in inferno deorsum.
 \verse Circumdederunt me undique, et non erat qui adiuvaret;
 respiciens eram ad adiutorium hominum, et non erat.
 \verse Memoratus autem sum misericordiae tuae, Domine,
 et operationis tuae, quae a saeculo est,
 \verse quoniam eruis sustinentes te, Domine,
 et liberas eos de manibus iniquorum.
 \verse Exaltavi de terra supplicationem meam
 et pro morte defluente deprecatus sum.
 \verse Invocavi Dominum: “Pater meus es tu,
 ne derelinquas me in die tribulationis meae
 et in tempore superborum sine adiutorio.
 \verse Laudabo nomen tuum assidue
 et collaudabo illud in confessione”.
 Et exaudita est oratio mea.
 \verse Et liberasti me de perditione
 et eripuisti me de tempore iniquo.
 \verse Propterea confitebor et laudem dicam tibi
 et benedicam nomini Domini.
 \verse Cum adhuc iunior essem, priusquam oberrarem,
 quaesivi sapientiam palam in oratione mea;
 \verse ante templum postulabam pro illa
 et usque in novissimis inquiram eam,
 et effloruit tamquam praecox uva.
 \verse Laetatum est cor meum in ea,
 ambulavit pes meus iter rectum;
 a iuventute mea investigabam eam.
 \verse Inclinavi modice aurem meam et excepi illam
 \verse et multam inveni mihimetipsi sapientiam
 et multum profeci in ea:
 \verse danti mihi sapientiam dabo gloriam.
 \verse Consiliatus sum enim, ut facerem illam;
 et quaesivi bonum et non confundar.
 \verse Colluctata est anima mea in illa,
 et in faciendo legem diligens fui.
 \verse Manus meas extendi in altum
 et incognoscibilia eius intellexi.
 \verse Animam meam direxi ad illam
 et in purificatione inveni eam.
 \verse Possedi cum ipsa cor ab initio;
 propter hoc non derelinquar.
 \verse Venter meus conturbatus est quaerendo illam;
 propterea bonam possedi possessionem.
 \verse Dedit mihi Dominus linguam mercedem meam,
 et in ipsa laudabo eum.
 \verse Appropiate ad me, indocti,
 et congregate vos in domo disciplinae.
 \verse Quid adhuc retardatis in his,
 dum animae vestrae sitiunt vehementer?
 \verse Aperui os meum et locutus sum:
 “Comparate vobis sine argento
 \verse et collum vestrum subicite iugo,
 et suscipiat anima vestra disciplinam:
 in proximo est enim invenire eam.
 \verse Videte oculis vestris quia modicum laboravi
 et inveni mihi multam requiem.
 \verse Assumite disciplinam in multo numero argenti
 et copiosum aurum possidete in ea.
 \verse Laetetur anima vestra in misericordia eius,
 et non confundemini in laude ipsius.
 \verse Operamini opus vestrum ante tempus,
 et dabit vobis mercedem vestram in tempore suo” ”.
\end{biblechapter}
