\biblebook{ Liber I Regum}
\begin{biblechapter}
 \verse Et rex David senuerat habebat que aetatis plurimos dies; cum que operiretur vestibus, non calefiebat. 
\verse Dixerunt ergo ei servi sui: “ Quaeratur domino nostro regi adulescentula virgo et stet coram rege et curam eius agat dormiatque in sinu tuo et calefaciat dominum nostrum regem ”. 
\verse Quaesierunt igitur adulescentulam speciosam in omnibus finibus Israel et invenerunt Abisag Sunamitin et adduxerunt eam ad regem. 
\verse Erat autem puella pulchra nimis et curam agebat regis et ministrabat ei; rex vero non cognovit eam.
 \verse Adonias autem filius Haggith elevabatur dicens: “ Ego regnabo! ”. Fecitque sibi currum et equites et quinquaginta viros, qui ante eum currerent. 
\verse Nec corripuit eum pater suus aliquando dicens: “ Quare hoc fecisti? ”. Erat autem et ipse pulcher valde, secundus natu post Absalom. 
\verse Et sermo ei cum Ioab filio Sarviae et cum Abiathar sacerdote, qui adiuvabant partes Adoniae. 
\verse Sadoc vero sacerdos et Banaias filius Ioiadae et Nathan propheta et Semei et Rei et robur exercitus David non erat cum Adonia. 
\verse Immolatis ergo Adonias ovibus et vitulis et pinguibus iuxta lapidem Zoheleth, qui erat vicinus fonti Rogel, vocavit universos fratres suos filios regis et omnes viros Iudae servos regis; 
\verse Nathan autem prophetam et Banaiam et robustos quosque et Salomonem fratrem suum non vocavit.
 \verse Dixit itaque Nathan ad Bethsabee matrem Salomonis: “ Num audisti quod regnaverit Adonias filius Haggith, et dominus noster David hoc ignorat? 
\verse Nunc ergo veni, accipe a me consilium et salva animam tuam filiique tui Salomonis. 
\verse Vade et ingredere ad regem David et dic ei: Nonne tu, domine mi rex, iurasti mihi ancillae tuae dicens: “Salomon filius tuus regnabit post me et ipse sedebit in solio meo”? Quare ergo regnat Adonias? 
\verse Et, adhuc ibi te loquente cum rege, ego veniam post te et complebo sermones tuos ”.
 \verse Ingressa est itaque Bethsabee ad regem in cubiculo; rex autem senuerat nimis, et Abisag Sunamitis ministrabat ei. 
\verse Inclinavit se Bethsabee et adoravit regem; ad quam rex: “ Quid tibi, inquit, vis? ”. 
\verse Quae respondens ait: “ Domine mi, tu iurasti per Dominum Deum tuum ancillae tuae: “Salomon filius tuus regnabit post me, et ipse sedebit in solio meo”; 
\verse et ecce nunc Adonias regnat, te, domine mi rex, ignorante. 
\verse Mactavit boves et pinguia quaeque et oves plurimas et vocavit omnes filios regis, Abiathar quoque sacerdotem et Ioab principem militiae; Salomonem autem servum tuum non vocavit. 
\verse Verumtamen, domine mi rex, in te oculi respiciunt totius Israel, ut indices eis quis sedere debeat in solio tuo, domine mi rex, post te. 
\verse Eritque, cum dormierit dominus meus rex cum patribus suis, erimus ego et filius meus Salomon peccatores ”.
 \verse Adhuc illa loquente cum rege, Nathan propheta venit; 
\verse et nuntiaverunt regi dicentes: “ Adest Nathan propheta ”. Cumque introisset ante conspectum regis et adorasset eum pronus in terram, 
\verse dixit Nathan: “ Domine mi rex, tu ergo dixisti: “Adonias regnet post me, et ipse sedeat super thronum meum”? 
 \verse Quia descendit hodie et immolavit boves et pinguia et arietes plurimos et vocavit universos filios regis et principes exercitus, Abiathar quoque sacerdotem; illique vescentes et bibentes coram eo dixerunt: “Vivat rex Adonias!”. 
\verse Me autem servum tuum et Sadoc sacerdotem et Banaiam filium Ioiadae et Salomonem famulum tuum non vocavit. 
\verse Numquid a domino meo rege exivit hoc verbum, et mihi non indicasti servo tuo quis sessurus esset super thronum domini mei regis post eum? ”.
 \verse Et respondit rex David dicens: “ Vocate ad me Bethsabee ”. Quae cum fuisset ingressa coram rege et stetisset ante eum, 
\verse iuravit rex et ait: “ Vivit Dominus, qui eruit animam meam de omni angustia, 
\verse quia, sicut iuravi tibi per Dominum, Deum Israel, dicens: Salomon filius tuus regnabit post me et ipse sedebit super solium meum pro me, sic faciam hodie ”. 
\verse Summissoque Bethsabee in terram vultu, adoravit regem dicens: “ Vivat dominus meus rex David in aeternum! ”.
 \verse Dixit quoque rex David: “ Vocate mihi Sadoc sacerdotem et Nathan prophetam et Banaiam filium Ioiadae ”. Qui cum ingressi fuissent coram rege, 
\verse dixit ad eos: “ Tollite vobiscum servos domini vestri et imponite Salomonem filium meum, super mulam meam et ducite eum in Gihon, 
\verse et ungat eum ibi Sadoc sacerdos et Nathan propheta in regem super Israel, et canetis bucina atque dicetis: “Vivat rex Salomon!”. 
\verse Et ascendetis post eum, et veniet et sedebit super solium meum, et ipse regnabit pro me; illique praecipiam, ut sit dux super Israel et super Iudam ”. 
\verse Et respondit Banaias filius Ioiadae regi dicens: “ Amen, sic loquatur Dominus Deus domini mei regis. 
\verse Quomodo fuit Dominus cum domino meo rege, sic sit cum Salomone et sublimius faciat solium eius a solio domini mei regis David ”.
 \verse Descendit ergo Sadoc sacerdos et Nathan propheta et Banaias filius Ioiadae et Cherethi et Phelethi, et imposuerunt Salomonem super mulam regis David et adduxerunt eum in Gihon. 
\verse Sumpsitque Sadoc sacerdos cornu olei de tabernaculo et unxit Salomonem; et cecinerunt bucina, et dixit omnis populus: “ Vivat rex Salomon! ”. 
\verse Et ascendit universa multitudo post eum, et populus canebat tibiis et laetabatur gaudio magno, et insonuit terra ad clamorem eorum.
 \verse Audivit autem Adonias et omnes, qui invitati fuerant ab eo; iamque convivium finitum erat. Sed et Ioab, audita voce tubae, ait: “ Quid sibi vult clamor civitatis tumultuantis? ”. 
\verse Adhuc illo loquente, Ionathan filius Abiathar sacerdotis venit; cui dixit Adonias: “ Ingredere, quia vir strenuus es et bona nuntians ”. 
\verse Responditque Ionathan Adoniae: “ Nequaquam! Dominus enim noster, rex David, regem constituit Salomonem 
\verse misitque cum eo Sadoc sacerdotem et Nathan prophetam et Banaiam filium Ioiadae et Cherethi et Phelethi, et imposuerunt eum super mulam regis; 
\verse unxeruntque eum Sadoc sacerdos et Nathan propheta regem in Gihon. Et ascenderunt inde laetantes, et insonuit civitas; haec est vox, quam audistis. 
\verse Sed et Salomon sedit super solio regni, 
\verse et ingressi servi regis benedixerunt domino nostro regi David dicentes: “Amplificet Deus nomen Salomonis super nomen tuum et magnificet thronum eius super thronum tuum”. Et adoravit rex in lectulo suo. 
\verse Insuper et haec locutus est: “Benedictus Dominus, Deus Israel, qui dedit hodie sedentem in solio meo, videntibus oculis meis” ”.
 \verse Territi sunt ergo et surrexerunt omnes, qui invitati fuerant ab Adonia, et ivit unusquisque in viam suam. 
\verse Adonias autem timens Salomonem surrexit et abiit tenuitque cornua altaris.
 \verse Et nuntiaverunt Salomoni dicentes: “ Ecce Adonias timens regem Salomonem tenuit cornua altaris dicens: “Iuret mihi hodie rex Salomon quod non interficiat servum suum gladio””. 
\verse Dixitque Salomon: “ Si fuerit vir bonus, non cadet ne unus quidem capillus eius in terram; sin autem malum inventum fuerit in eo, morietur ”. 
\verse Misit ergo rex Salomon et eduxit eum ab altari, et ingressus adoravit regem Salomonem; dixitque ei Salomon: “ Vade in domum tuam ”.
 
\begin{biblechapter}
\verse Appropinquaverant autem dies David ut moreretur, praecepit que Salomoni filio suo dicens: 
\verse “ Ego ingredior viam universae terrae; confortare et esto vir 
\verse et observa decreta Domini Dei tui, ut ambules in viis eius et custodias statuta eius et praecepta eius et iudicia et testimonia, sicut scriptum est in lege Moysi, ut prospere agas in universis, quae facis et quocumque te verteris; 
\verse ut confirmet Dominus sermonem suum, quem locutus est de me dicens: “Si custodierint filii tui viam suam et ambulaverint coram me in veritate, in omni corde suo et in omni anima sua, non auferetur tibi vir de solio Israel”.
 \verse Tu quoque nosti, quae fecerit mihi Ioab filius Sarviae, quae fecerit duobus principibus exercitus Israel, Abner filio Ner et Amasae filio Iether, quos occidit; et effudit sanguinem belli in pace et posuit cruorem proelii in balteo suo, qui erat circa lumbos eius, et in calceamento suo, quod erat in pedibus eius. 
\verse Facies ergo iuxta sapientiam tuam et non deduces canitiem eius pacifice ad inferos. 
\verse Sed filiis Berzellai Galaaditis reddes gratiam, eruntque comedentes in mensa tua; occurrerunt enim mihi, quando fugiebam a facie Absalom fratris tui. 
\verse Habes quoque apud te Semei filium Gera de Beniamin de Bahurim, qui maledixit mihi maledictione pessima, quando ibam ad Mahanaim; sed quia descendit mihi in occursum ad Iordanem, et iuravi ei per Dominum dicens: Non te interficiam gladio. 
\verse Tu noli pati esse eum innoxium; vir autem sapiens es et scies, quae facias ei deducesque canos eius cum sanguine ad infernum ”.
 \verse Dormivit igitur David cum patribus suis et sepultus est in civitate David. 
 \verse Dies autem, quibus regnavit David super Israel, quadraginta anni sunt: in Hebron regnavit septem annis, in Ierusalem triginta tribus. 
\verse Salomon autem sedit super thronum David patris sui, et firmatum est regnum eius nimis.
 \verse Et ingressus est Adonias filius Haggith ad Bethsabee matrem Salomonis, quae dixit ei: “ Pacificusne ingressus tuus? ”. Qui respondit: “ Pacificus ”. 
\verse Addiditque: “ Sermo mihi est ad te ”. Cui ait: “ Loquere ”. Et ille: 
\verse “ Tu, inquit, nosti quia meum erat regnum, et me proposuerat omnis Israel sibi in regem, sed translatum est regnum et factum est fratris mei; a Domino enim constitutum est ei. 
\verse Nunc ergo petitionem unam deprecor a te; ne confundas faciem meam ”. Quae dixit ad eum: “ Loquere ”. 
\verse Et ille ait: “ Precor, ut dicas Salomoni regi — neque enim negare tibi quidquam potest — ut det mihi Abisag Sunamitin uxorem ”. 
\verse Et ait Bethsabee: “ Bene, ego loquar pro te regi ”.
 \verse Venit ergo Bethsabee ad regem Salomonem, ut loqueretur ei pro Adonia. Et surrexit rex in occursum eius adoravitque eam et sedit super thronum suum; positus quoque est thronus matri regis, quae sedit ad dexteram eius. 
\verse Dixitque ei: “ Petitionem unam parvulam ego deprecor a te; ne confundas faciem meam”. Dixit ei rex: “ Pete, mater mi, neque enim fas est, ut avertam faciem tuam ”. 
\verse Quae ait: “ Detur Abisag Sunamitis Adoniae fratri tuo uxor ”. 
\verse Responditque rex Salomon et dixit matri suae: “ Quare postulas Abisag Sunamitin Adoniae? Postula ei et regnum! Ipse est enim frater meus maior me et habet Abiathar sacerdotem et Ioab filium Sarviae ”. 
\verse Iuravit itaque rex Salomon per Dominum dicens: “ Haec faciat mihi Deus et haec addat, certe contra animam suam locutus est Adonias verbum hoc. 
\verse Et nunc, vivit Dominus, qui firmavit me et collocavit me super solium David patris mei et qui fecit mihi domum, sicut locutus est, certe hodie occidetur Adonias ”. 
\verse Misitque rex Salomon per manum Banaiae filii Ioiadae, qui interfecit eum, et mortuus est.
 \verse Abiathar quoque sacerdoti dixit rex: “ Vade in Anathoth ad agrum tuum; es quidem vir mortis, sed hodie te non interficiam, quia portasti arcam Domini Dei coram David patre meo et sustinuisti laborem in omnibus, in quibus laboravit pater meus ”. 
\verse Eiecit ergo Salomon Abiathar, ut non esset sacerdos Domini, ut impleretur sermo Domini, quem locutus est super domum Heli in Silo.
 \verse Venit autem nuntius ad Ioab. Ioab autem declinaverat post Adoniam, cum post Absalom non declinasset; fugit ergo Ioab in tabernaculum Domini et apprehendit cornua altaris. 
\verse Nuntiatumque est regi Salomoni, quod fugisset Ioab in tabernaculum Domini et esset iuxta altare; misitque Salomon Banaiam filium Ioiadae dicens: “ Vade, interfice eum! ”. 
\verse Venit Banaias ad tabernaculum Domini et dixit ei: “ Haec dicit rex: Egredere!”. Qui ait: “ Non egrediar, sed hic moriar ”. Renuntiavit Banaias regi sermonem dicens: “ Haec locutus est Ioab et haec respondit mihi ”. 
\verse Dixitque ei rex: “ Fac, sicut locutus est, et interfice eum et sepeli; et amovebis sanguinem innocentem, qui effusus est a Ioab, a me et a domo patris mei. 
\verse Et reddet Dominus sanguinem eius super caput eius, quia interfecit duos viros iustos melioresque se et occidit eos gladio, patre meo David ignorante: Abner filium Ner principem militiae Israel et Amasam filium Iether principem exercitus Iudae. 
\verse Et revertetur sanguis illorum in caput Ioab et in caput seminis eius in sempiternum; David autem et semini eius et domui et throno illius sit pax usque in aeternum a Domino ”. 
 \verse Ascendit itaque Banaias filius Ioiadae et aggressus eum interfecit; sepultusque est in domo sua in deserto. 
\verse Et constituit rex Banaiam filium Ioiadae pro eo super exercitum et Sadoc sacerdotem posuit pro Abiathar.
 \verse Misit quoque rex et vocavit Semei dixitque ei: “ Aedifica tibi domum in Ierusalem et habita ibi et non egredieris inde huc atque illuc; 
\verse quacumque autem die egressus fueris et transieris torrentem Cedron, scito te interficiendum; sanguis tuus erit super caput tuum ”. 
\verse Dixitque Semei regi: “ Bonus sermo; sicut locutus est dominus meus rex, sic faciet servus tuus ”. Habitavit itaque Semei in Ierusalem diebus multis.
 \verse Factum est autem post annos tres, ut fugerent duo servi Semei ad Achis filium Maacha regem Geth; nuntiatumque est Semei quod servi eius essent in Geth. 
\verse Et surrexit Semei et stravit asinum suum ivitque in Geth ad Achis ad requirendos servos suos et adduxit eos de Geth.
 \verse Nuntiatum est autem Salomoni quod isset Semei in Geth de Ierusalem et redisset. 
\verse Et mittens vocavit eum dixitque illi: “ Nonne testificatus sum tibi per Dominum et praedixi tibi: Quacumque die egressus ieris huc et illuc, scito te esse moriturum? Et respondisti mihi “Bonus sermo; audivi”. 
\verse Quare ergo non custodisti iusiurandum Domini et praeceptum, quod praeceperam tibi? ”. 
\verse Dixitque rex ad Semei: “ Tu nosti omne malum, cuius tibi conscium est cor tuum, quod fecisti David patri meo; reddit Dominus malitiam tuam in caput tuum. 
\verse Et rex Salomon benedictus, et thronus David erit stabilis coram Domino usque in sempiternum ”. 
\verse Iussit itaque rex Banaiae filio Ioiadae, qui egressus percussit eum, et mortuus est. Confirmatum est igitur regnum in manu Salomonis.
 
\begin{biblechapter}
\verse Et affinitate coniunctus est pharaoni regi Aegypti. Accepit namque filiam eius et adduxit in civitatem David, donec compleret aedificans domum suam et domum Domini et murum Ierusalem per circuitum.
 \verse Attamen populus immolabat in excelsis; non enim aedificatum erat templum nomini Domini usque in diem illum. 
\verse Dilexit autem Salomon Dominum ambulans in praeceptis David patris sui, excepto quod in excelsis immolabat et accendebat thymiama. 
\verse Abiit itaque in Gabaon, ut immolaret ibi; illud quippe erat excelsum maximum. Mille hostias in holocaustum obtulit Salomon super altare illud.
 \verse In Gabaon apparuit Dominus Salomoni per somnium nocte dicens: “ Postula quod vis, ut dem tibi ”. 
\verse Et ait Salomon: “ Tu fecisti cum servo tuo David patre meo misericordiam magnam, sicut ambulavit in conspectu tuo in veritate et iustitia et recto corde tecum; custodisti ei misericordiam tuam grandem et dedisti ei filium sedentem super thronum eius, sicut est hodie. 
\verse Et nunc, Domine Deus meus, tu regnare fecisti servum tuum pro David patre meo. Ego autem sum puer parvus et ignorans egressum et introitum meum; 
\verse et servus tuus in medio est populi, quem elegisti, populi infiniti, qui numerari et supputari non potest prae multitudine. 
\verse Da ergo servo tuo cor docile, ut iudicare possit populum tuum et discernere inter bonum et malum. Quis enim potest iudicare populum tuum hunc multum? ”.
 \verse Placuit ergo sermo coram Domino quod Salomon rem huiuscemodi postulasset, 
 \verse et dixit Deus Salomoni: “ Quia postulasti verbum hoc et non petisti tibi dies multos nec divitias aut animam inimicorum tuorum, sed postulasti tibi sapientiam ad discernendum iudicium, 
\verse ecce feci tibi secundum sermones tuos et dedi tibi cor sapiens et intellegens, in tantum ut nullus ante te similis tui fuerit nec post te surrecturus sit; 
\verse sed et haec, quae non postulasti, dedi tibi, divitias scilicet et gloriam, ut nemo fuerit similis tui in regibus cunctis diebus tuis. 
\verse Si autem ambulaveris in viis meis et custodieris praecepta mea et mandata mea, sicut ambulavit David pater tuus, longos faciam dies tuos ”. 
\verse Igitur evigilavit Salomon et intellexit quod esset somnium. Cumque venisset Ierusalem, stetit coram arca foederis Domini et obtulit holocausta et fecit victimas pacificas et convivium universis famulis suis.
 \verse Tunc venerunt duae mulieres meretrices ad regem steteruntque coram eo. 
\verse Quarum una ait: “ Obsecro, mi domine; ego et mulier haec habitabamus in domo una, et peperi apud eam in domo; 
\verse tertia vero die, postquam ego peperi, peperit et haec; et eramus simul, nullusque alius nobiscum in domo, exceptis nobis duabus. 
\verse Mortuus est autem filius mulieris huius nocte; dormiens quippe oppressit eum. 
\verse Et consurgens intempesta nocte, silentio tulit filium meum de latere meo ancillae tuae dormientis et collocavit in sinu suo; suum autem filium, qui erat mortuus, posuit in sinu meo. 
\verse Cumque surrexissem mane, ut darem lac filio meo, apparuit mortuus; quem diligentius intuens clara luce, deprehendi non esse meum, quem genueram ”. 
\verse Responditque altera mulier: “ Non est ita, sed filius meus vivit, tuus autem mortuus est ”. E contrario illa dicebat: “ Mentiris. Filius quippe tuus mortuus est, meus autem vivit ”. Atque in hunc modum contendebant coram rege.
 \verse Tunc rex ait: “ Haec dicit: “Filius meus vivit, et filius tuus mortuus est”; et ista respondit: “Non, sed filius tuus mortuus est, et filius meus vivit” ”. 
 \verse Dixit ergo rex: “ Afferte mihi gladium! ”. Cumque attulissent gladium coram rege: 
\verse “ Dividite, inquit, infantem vivum in duas partes, et date dimidiam partem uni et dimidiam partem alteri ”. 
\verse Dixit autem mulier, cuius filius erat vivus, ad regem — commota sunt quippe viscera eius super filio suo —: “ Obsecro, domine, date illi infantem vivum et nolite interficere eum ”. E contrario illa dicebat: “ Nec mihi nec tibi sit; dividatur ”. 
\verse Respondens rex ait: “ Date huic infantem vivum, et non occidatur; haec est mater eius ”.
 \verse Audivit itaque omnis Israel iudicium, quod iudicasset rex; et timuerunt regem videntes sapientiam Dei esse in eo ad faciendum iudicium.
 
\begin{biblechapter}
\verse Erat autem rex Salomon regnans super omnem Israel. 
\verse Et hi principes quos habebat: Azarias filius Sadoc sacerdos; 
\verse Elihoreph et Ahia filii Sisa scribae; Iosaphat filius Ahilud cancellarius; 
\verse Banaias filius Ioiadae super exercitum; Sadoc autem et Abiathar sacerdotes; 
\verse Azarias filius Nathan super praefectos; Zabud filius Nathan sacerdos amicus regis; 
\verse et Ahisar praepositus domus et Adoniram filius Abda super tributa.
 \verse Habebat autem Salomon duodecim praefectos super omnem Israel, qui praebebant annonam regi et domui eius; per singulos enim menses in anno singuli necessaria ministrabant. 
\verse Et haec nomina eorum: Benhur in monte Ephraim; 
\verse Bendecar in Maces et in Salebim et in Bethsames et in Elon et in Bethanan; 
 \verse Benhesed in Aruboth, ipsius erat Socho et omnis terra Epher; 
\verse Benabinadab, cuius omnis regio Dor, Tapheth filiam Salomonis habebat uxorem; 
 \verse Baana filius Ahilud regebat Thanach et Mageddo et universam Bethsan, quae est iuxta Sarthan subter Iezrahel, a Bethsan usque Abelmehula et usque ultra Iecmaam; 
\verse Bengaber in Ramoth Galaad habebat villas Iair filii Manasse in Galaad: ipse praeerat in omni regione Argob, quae est in Basan, sexaginta civitatibus magnis atque muratis, quae habebant seras aereas; 
\verse Ahinadab filius Addo praeerat in Mahanaim; 
\verse Achimaas in Nephthali, sed et ipse habebat Basemath filiam Salomonis in coniugio; 
\verse Baana filius Chusai in Aser et in Baloth; 
\verse Iosaphat filius Pharue in Issachar; 
\verse Semei filius Ela in Beniamin; 
\verse Gaber filius Uri in terra Galaad, in terra Sehon regis Amorraei et Og regis Basan, ut praefectus unus, qui erat in terra.
 \verse Iuda et Israel innumerabiles, sicut arena maris in multitudine, comedentes et bibentes atque laetantes.
 
\begin{biblechapter}
\verse Salomon autem erat in dicione sua habens omnia regna a Flumine usque ad terram Philisthim et ad terminum Aegypti offerentium sibi munera et servientium ei cunctis diebus vitae eius. 
\verse Erat autem cibus Salomonis per dies singulos triginta chori similae et sexaginta chori farinae, 
\verse decem boves pingues et viginti boves pascuales et centum oves, excepta venatione cervorum, caprearum atque bubalorum et avium altilium. 
\verse Ipse enim obtinebat omnem regionem, quae erat trans Flumen, a Thaphsa usque Gazam, et cunctos reges illarum regionum; et habebat pacem ex omni parte in circuitu. 
\verse Habitabatque Iuda et Israel absque timore ullo, unusquisque sub vite sua et sub ficu sua a Dan usque Bersabee cunctis diebus Salomonis. 
\verse Et habebat Salomon quattuor milia praesepia equorum currulium et duodecim milia equestres. 
\verse Et praebebant supradicti praefecti necessaria mensae regis Salomonis et convivarum eius cum ingenti cura, unusquisque in suo mense. 
\verse Hordeum quoque et paleas equorum et iumentorum deferebant in locum, ubi erat unicuique constitutum.
 \verse Dedit quoque Deus sapientiam Salomoni et prudentiam multam nimis et latitudinem cordis quasi arenam, quae est in litore maris. 
\verse Et praecedebat sapientia Salomonis sapientiam omnium Orientalium et Aegyptiorum; 
\verse et erat sapientior cunctis hominibus, sapientior Ethan Ezrahita et Heman et Chalchol et Darda filiis Mahol et erat nominatus in universis gentibus per circuitum. 
\verse Locutus est quoque Salomon tria milia parabolas, et fuerunt carmina eius quinque et mille. 
\verse Et disputavit super lignis, a cedro, quae est in Libano, usque ad hyssopum, quae egreditur de pariete; et disseruit de iumentis et volucribus et reptilibus et piscibus. 
\verse Et veniebant de cunctis populis ad audiendam sapientiam Salomonis, ab universis regibus terrae, qui audiebant sapientiam eius.
 \verse Misit quoque Hiram rex Tyri servos suos ad Salomonem; audivit enim quod ipsum unxissent regem pro patre eius, quia amicus fuerat Hiram David omni tempore. 
 \verse Misit autem et Salomon ad Hiram dicens: 
\verse “ Tu scis voluntatem David patris mei et quia non potuerit aedificare domum nomini Domini Dei sui propter bella imminentia per circuitum, donec daret Dominus eos sub vestigio pedum eius. 
 \verse Nunc autem requiem dedit Dominus Deus meus mihi per circuitum; non est adversarius neque occursus malus. 
\verse Quam ob rem cogito aedificare templum nomini Domini Dei mei, sicut locutus est Dominus David patri meo dicens: “Filius tuus, quem dabo pro te super solium tuum, ipse aedificabit domum nomini meo”. 
\verse Praecipe igitur, ut praecidant mihi cedros de Libano, et servi mei sint cum servis tuis; mercedem autem servorum tuorum dabo tibi quamcumque praeceperis; scis enim quoniam non est in populo meo vir, qui noverit ligna caedere sicut Sidonii ”.
 \verse Cum ergo audisset Hiram verba Salomonis, laetatus est valde et ait: “ Benedictus Dominus hodie, qui dedit David filium sapientissimum super populum hunc plurimum ”. 
\verse Et misit Hiram ad Salomonem dicens: “ Audivi, quaecumque mandasti mihi; ego faciam omnem voluntatem tuam in lignis cedrinis et abiegnis. 
\verse Servi mei deponent ea de Libano ad mare, et ego componam ea in ratibus in mari usque ad locum, quem significaveris mihi, et applicabo ea ibi, et tu tolles ea; praebebisque necessaria mihi, ut detur cibus domui meae ”.
 \verse Itaque Hiram dabat Salomoni ligna cedrina et ligna abiegna iuxta omnem voluntatem eius. 
\verse Salomon autem praebebat Hiram viginti milia chororum tritici in cibum domui eius et viginti choros purissimi olei; haec tribuebat Salomon Hiram per annos singulos. 
\verse Dedit quoque Dominus sapientiam Salomoni, sicut locutus est ei; et erat pax inter Hiram et Salomonem, et percusserunt foedus ambo.
 \verse Elegitque rex Salomon operas de omni Israel, et erat indictio triginta milia virorum. 
\verse Mittebatque eos in Libanum decem milia per menses singulos vicissim, ita ut duobus mensibus essent in domibus suis; et Adoniram erat super huiuscemodi indictione. 
\verse Fueruntque Salomoni septuaginta milia eorum, qui onera portabant, et octoginta milia latomorum in monte, 
\verse absque praepositis, qui praeerant singulis operibus numero trium milium et trecentorum praecipientium populo, his, qui faciebant opus. 
\verse Praecepitque rex, ut tollerent lapides grandes, lapides pretiosos in fundamentum templi, lapides quadratos; 
\verse dolaverunt ergo caementarii Salomonis, caementarii Hiram et Giblii ligna et lapides et praeparaverunt ad aedificandam domum.
 
\begin{biblechapter}
\verse Factum est igitur quadringentesimo et octogesimo anno egressionis filiorum Israel de terra Aegypti, in anno quarto, mense Ziv — ipse est mensis secundus — regni Salomonis super Israel, aedificare coepit domum Domino. 
\verse Domus autem, quam aedificabat rex Salomon Domino, habebat sexaginta cubitos in longitudine et viginti cubitos in latitudine et triginta cubitos in altitudine. 
\verse Et porticus erat ante templum viginti cubitorum longitudinis iuxta mensuram latitudinis templi et habebat decem cubitos latitudinis ante faciem templi. 
\verse Fecitque in templo fenestras cum marginibus et cancellis. 
\verse Et aedificavit contra parietem templi tabulata per gyrum in parietibus domus per circuitum templi et Dabir et fecit latera in circuitu. 
\verse Tabulatum, quod subter erat, quinque cubitos habebat latitudinis et medium tabulatum sex cubitorum latitudinis et tertium tabulatum septem habens cubitos latitudinis; gradus enim posuit in domo per circuitum forinsecus, ut non ingrederentur trabes in muros templi.
 \verse Domus autem cum aedificaretur, lapidibus dedolatis atque perfectis aedificata est; et malleus et securis et omne ferramentum non sunt audita in domo, cum aedificaretur. 
\verse Ostium lateris inferioris in parte erat domus dextrae, et per cochleam ascendebant in medium latus et a medio in tertium. 
\verse Et aedificavit domum et consummavit eam; texit quoque domum laquearibus cedrinis. 
 \verse Aedificavit ergo stratum contra omnem domum quinque cubitis altitudinis et iunxit domui lignis cedrinis.
 \verse Et factus est sermo Domini ad Salomonem dicens: 
\verse “ Domus haec, quam aedificas, si ambulaveris in praeceptis meis et iudicia mea feceris et custodieris omnia mandata mea gradiens per ea, firmabo sermonem meum tibi, quem locutus sum ad David patrem tuum; 
\verse et habitabo in medio filiorum Israel et non derelinquam populum meum Israel ”.
 \verse Igitur aedificavit Salomon domum et consummavit eam. 
\verse Et aedificavit parietes domus intrinsecus tabulis cedrinis; a pavimento domus usque ad summitatem parietum et usque ad laquearia operuit lignis intrinsecus et texit pavimentum domus tabulis abiegnis. 
\verse Aedificavitque viginti cubitorum a posteriore parte templi tabulis cedrinis a pavimento usque ad superiora; et fecit ei intrinsecus Dabir, id est sancta sanctorum. 
\verse Porro quadraginta cubitorum erat ipsum templum ante illud. 
\verse Et cedrus in domo intrinsecus sculptas habebat colocynthidas et calices apertos florum. Omnia cedrinis tabulis vestiebantur, nec omnino lapis apparere poterat in pariete.
 \verse Dabir autem in medio domus in interiori parte fecerat, ut poneret ibi arcam foederis Domini. 
\verse Habebat viginti cubitos longitudinis et viginti cubitos latitudinis et viginti cubitos altitudinis; et vestivit illud auro purissimo et fecit altare cedrinum ante Dabir. 
\verse Domum quoque operuit Salomon intrinsecus auro purissimo et posuit catenas aureas ante Dabir. 
\verse Nihilque erat in templo, quod non auro tegeretur; sed et totum altare Dabir texit auro.
 \verse Et fecit in Dabir duos cherubim de lignis oleastri decem cubitorum altitudinis. 
\verse Quinque cubitorum ala cherub una et quinque cubitorum ala cherub altera, id est decem cubitos habentes a summitate alae unius usque ad alae alterius summitatem. 
\verse Decem quoque cubitorum erat cherub secundus, mensura par et effigies una erat duobus cherubim; 
\verse altitudinem habebat unus cherub decem cubitorum et similiter cherub secundus. 
\verse Posuitque cherubim in medio templi interioris; extendebant autem alas suas cherubim, et tangebat ala una parietem et ala cherub secundi tangebat parietem alterum; alae autem alterae in media parte templi se invicem contingebant. 
\verse Texit quoque cherubim auro. 
\verse Et omnes parietes templi per circuitum scalpsit variis caelaturis; et fecit in eis cherubim et palmas et calices apertos florum intrinsecus et foras. 
\verse Sed et pavimentum domus texit auro intrinsecus et extrinsecus.
 \verse Et pro ingressu Dabir fecit valvas de lignis oleastri postesque cum marginibus quinque. 
\verse Et in duabus valvis de lignis oleastri scalpsit cherubim et palmas et calices apertos florum et vestivit ea auro operiens tam cherubim quam palmas et cetera auro. 
\verse Fecitque eodem modo pro introitu templi postes cum quattuor marginibus de lignis oleastri 
\verse et duas valvas de lignis abiegnis; et utraque valva duplex erat et versatilis. 
\verse Et scalpsit cherubim et palmas et calices apertos florum operuitque omnia laminis aureis.
 \verse Et aedificavit atrium interius tribus ordinibus lapidum politorum et uno ordine lignorum cedri.
 \verse Anno quarto fundata est domus Domini in mense Ziv; 
\verse et in anno undecimo, mense Bul — ipse est mensis octavus — perfecta est domus in omni opere suo et in universis utensilibus; aedificavitque eam annis septem.
 
\begin{biblechapter}
\verse Domum autem suam aedificavit Salomon tredecim annis et ad perfectum usque perduxit. 
\verse Aedificavit quoque domum Saltus Libani centum cubitorum longitudinis et quinquaginta cubitorum latitudinis et triginta cubitorum altitudinis super quattuor ordines columnarum cedrinarum, et ligna cedrina super columnas. 
\verse Et erat tectum cedrinum in alto super tabulas quadraginta quinque, quae erant super columnas, quindecim in uno ordine, 
\verse et marginum tres ordines, fenestra iuxta fenestram tribus vicibus. 
\verse Ostia, id est postes, habebant quadruplicem marginem. 
\verse Et porticum columnarum fecit quinquaginta cubitorum longitudinis et triginta cubitorum latitudinis, et alteram porticum in facie maioris porticus et columnas et cancellos ante eas. 
 \verse Porticum quoque solii, in qua tribunal erat, fecit et texit lignis cedrinis a pavimento usque ad pavimentum. 
\verse Et domus, in qua habitabat, erat in altero atrio intro a porticu et simili opere. Domum quoque fecit filiae pharaonis, quam uxorem duxerat Salomon, tali opere quali et hanc porticum.
 \verse Omnia lapidibus pretiosis, qui ad normam quandam atque mensuram tam intrinsecus quam extrinsecus serrati erant, a fundamento usque ad summitatem parietum, et extrinsecus usque ad atrium maius. 
\verse Fundamenta autem de lapidibus pretiosis, lapidibus magnis decem sive octo cubitorum. 
\verse Et desuper lapides pretiosi secundum mensuram secti et ligna cedrina. 
\verse Et atrium maius in circuitu habebat tres ordines de lapidibus sectis et unum ordinem de dolata cedro; necnon et atrium domus Domini interius et porticus domus.
 \verse Misit quoque rex Salomon et tulit Hiram de Tyro, 
\verse filium mulieris viduae de tribu Nephthali, patre Tyrio, artificem aerarium et plenum sapientia et intellegentia et doctrina ad faciendum omne opus ex aere. Qui, cum venisset ad regem Salomonem, fecit omne opus eius.
 \verse Et finxit duas columnas aereas, decem et octo cubitorum altitudinis columnam unam, et linea duodecim cubitorum ambiebat columnam, et grossitudo eius quattuor digitorum, et intrinsecus cava erat; sic et columna altera. 
\verse Duo quoque capitella fecit, quae ponerentur super capita columnarum, fusili aere; quinque cubitorum altitudinis capitellum unum et quinque cubitorum altitudinis capitellum alterum, 
\verse et serta quasi in modum texturae, fimbriae in modum catenarum sibi invicem miro opere contextarum in capitellis, quae erant super caput columnarum, septem in capitello uno et septem in capitello altero. 
\verse Et fecit malogranatorum duos ordines per circuitum super sertum unum, ut tegerent capitella, quae erant super summitatem columnarum; eodem modo fecit et capitello secundo. 
\verse Capitella autem, quae erant super capita columnarum, quasi opere lilii fabricata erant in porticu, quattuor cubitorum. 
\verse Et rursum alia capitella in summitate duarum columnarum etiam desuper, iuxta alvum, quae erat super sertum. Malogranatorum autem ducentorum duo ordines erant in circuitu capitelli primi et eodem modo in circuitu capitelli secundi. 
\verse Et statuit duas columnas in porticum templi; cumque statuisset columnam dexteram, vocavit eam nomine Iachin, similiter erexit columnam sinistram et vocavit nomen eius Booz. 
\verse Et super capita columnarum opus in modum lilii posuit; per fectumque est opus columnarum. 
\verse Fecit quoque mare fusile decem cubitorum a labio usque ad labium, rotundum in circuitu, quinque cubitorum altitudo eius; et resticula triginta cubitorum cingebat illud per circuitum. 
\verse Et scalptura colocynthidum subter labium circuibat illud, duo ordines scalpturarum fusilium in una fusione cum mari. 
\verse Et stabat super duodecim boves, e quibus tres respiciebant ad aquilonem et tres ad occidentem et tres ad meridiem et tres ad orientem, et mare super eos desuper erat; quorum posteriora universa intrinsecus latitabant. 
\verse Grossitudo autem luteris habebat mensuram palmi, labiumque eius erat quasi labium calicis et folium repandi lilii; duo milia batos capiebat.
 \verse Et fecit bases decem aereas, quattuor cubitorum longitudinis bases singulas et quattuor cubitorum latitudinis et trium cubitorum altitudinis. 
\verse Hoc autem erat opus basium: limbos habebant, insuper et limbos inter columellas. 
 \verse Super limbos inter columellas erant leones et boves et cherubim, et super columellas similiter; supra et infra leones et boves erant coronae, opus malleatum. 
\verse Et quattuor rotae per bases singulas et axes aerei, et quattuor pedes et quasi umeruli subter luterem fusiles, contra singulos coronae. 
 \verse Et os eius erat rotundum, opus basis, unius cubiti et dimidii; etiam in ore eius variae caelaturae erant, limbi autem eius erant quadrati, non rotundi. 
 \verse Quattuor quoque rotae subter limbis erant, et fulcra rotarum cohaerebant basi; una rota habebat altitudinis cubitum et semis. 
\verse Tales autem rotae erant, quales solent in curru fieri, et fulcra earum et canthi et radii et modioli, omnia fusilia. 
\verse Nam et umeruli illi quattuor per singulos angulos basis unius ex ipsa basi fusiles et coniuncti erant. 
\verse In summitate autem basis erat quaedam rotunditas dimidii cubiti, et in summitate basis fulcra eius et limbi eius ex semetipsa. 
\verse Scalpsit quoque in tabulatis illis, fulcris eius et super limbos eius cherubim et leones et palmas secundum vacuum singulorum, et coronas per circuitum. 
\verse In hunc modum fecit decem bases, fusura una, et mensura scalpturaque consimili. 
\verse Fecit quoque decem luteres aereos; quadraginta batos capiebat luter unus, eratque quattuor cubitorum; singulosque luteres per singulas, id est decem bases posuit. 
\verse Et constituit decem bases, quinque ad dexteram partem templi et quinque ad sinistram; mare autem posuit ad dexteram partem templi contra orientem ad meridiem.
 \verse Fecit quoque Hiram lebetes et vatilla et phialas et perfecit omne opus regi Salomoni in templo Domini; 
\verse columnas duas et globos capitellorum super capita columnarum duos et serta duo, ut operirent duos globos, qui erant super capita columnarum; 
\verse et malogranata quadringenta in duobus sertis, duos versus malogranatorum in sertis singulis, ad operiendos globos capitellorum, qui erant super faciem columnarum; 
\verse et bases decem et luteres decem super bases 
\verse et mare unum et boves duodecim subter mare; 
\verse et lebetes et vatilla et phialas. Omnia vasa, quae fecit Hiram regi Salomoni in domo Domini, de aere polito erant. 
\verse In campestri regione Iordanis fudit ea rex in argillosa terra inter Succoth et Sarthan. 
\verse Et posuit Salomon omnia vasa; propter multitudinem autem nimiam ignorabatur pondus aeris.
 \verse Fecitque Salomon omnia vasa in domo Domini: altare aureum et mensam, super quam ponerentur panes propositionis, auream; 
\verse et candelabra, quinque ad dexteram et quinque ad sinistram contra Dabir, ex auro puro, et florem et lucernas desuper aureas; et forcipes aureos 
\verse et pateras et cultros et phialas et sartagines et turibula de auro purissimo; et cardines ostiorum domus interioris Sancti sanctorum et ostiorum domus templi ex auro.
 \verse Et perfecit omne opus, quod faciebat Salomon in domo Domini, et intulit Salomon, quae sanctificaverat David pater suus, argentum et aurum et vasa, reposuitque in thesauris domus Domini.
 
\begin{biblechapter}
\verse Tunc congregavit Salomon omnes maiores natu Israel — omnes principes tribuum, duces familiarum filiorum Israel ad regem Salomonem — in Ierusalem, ut deferrent arcam foederis Domini de civitate David, id est de Sion. 
\verse Convenitque ad regem Salomonem universus Israel in mense Ethanim in sollemnitate, ipse est mensis septimus. 
\verse Veneruntque cuncti senes Israel, et tulerunt sacerdotes arcam 
\verse et portaverunt arcam Domini et tabernaculum conventus et omnia vasa sanctuarii, quae erant in tabernaculo; et ferebant ea sacerdotes et Levitae. 
\verse Rex autem Salomon et universus coetus Israel, qui convenerat ad eum, cum illo ante arcam immolabant oves et boves absque aestimatione et numero. 
\verse Et intulerunt sacerdotes arcam foederis Domini in locum suum in Dabir templi, in sanctum sanctorum, subter alas cherubim; 
\verse siquidem cherubim expandebant alas super locum arcae et protegebant arcam et vectes eius desuper. 
\verse Cumque eminerent vectes et apparerent summitates eorum foris in sanctuario ante Dabir, non apparebant ultra extrinsecus; qui et fuerunt ibi usque in praesentem diem. 
\verse In arca autem non erat aliud nisi duae tabulae lapideae, quas posuerat in ea Moyses in Horeb, quando pepigit Dominus foedus cum filiis Israel, cum egrederentur de terra Aegypti.
 \verse Factum est autem cum exissent sacerdotes de sanctuario, nebula implevit domum Domini, 
\verse et non poterant sacerdotes stare et ministrare propter nebulam; impleverat enim gloria Domini domum Domini. 
\verse Tunc ait Salomon:
 “ Dominus dixit ut habitaret in nebula.
 \verse Aedificans aedificavi domum in habitaculum tuum,
 firmissimum solium tuum in sempiternum ”.
 \verse Convertitque rex faciem suam et benedixit omni ecclesiae Israel; omnis enim ecclesia Israel stabat. 
\verse Et ait: “ Benedictus Dominus, Deus Israel, qui locutus est ore suo ad David patrem meum et in manibus suis perfecit dicens: 
 \verse “A die qua eduxi populum meum Israel de Aegypto, non elegi civitatem de universis tribubus Israel, ut aedificaretur domus, et esset nomen meum ibi; sed elegi David, ut esset super populum meum Israel”. 
\verse Voluitque David pater meus aedificare domum nomini Domini, Dei Israel, 
\verse et ait Dominus ad David patrem meum: “Quod cogitasti in corde tuo aedificare domum nomini meo, bene fecisti hoc ipsum mente tractans; 
\verse verumtamen tu non aedificabis domum sed filius tuus, qui egredietur de lumbis tuis, ipse aedificabit domum nomini meo”. 
\verse Confirmavit Dominus sermonem suum, quem locutus est; stetique pro David patre meo et sedi super thronum Israel, sicut locutus est Dominus, et aedificavi domum nomini Domini, Dei Israel. 
\verse Et constitui ibi locum arcae, in qua foedus est Domini, quod percussit cum patribus nostris, quando eduxit eos de terra Aegypti ”.
 \verse Stetit autem Salomon ante altare Domini in conspectu omnis ecclesiae Israel et expandit manus suas in caelum 
\verse et ait: “ Domine, Deus Israel, non est similis tui Deus in caelo desuper et super terra deorsum, qui custodis pactum et misericordiam servis tuis, qui ambulant coram te in toto corde suo; 
\verse qui custodisti servo tuo David patri meo, quae locutus es ei; ore locutus es et manibus perfecisti, ut et haec dies probat. 
\verse Nunc igitur, Domine, Deus Israel, conserva famulo tuo David patri meo, quae locutus es ei dicens: “Non auferetur de te vir coram me, qui sedeat super thronum Israel, ita tamen, si custodierint filii tui viam suam, ut ambulent coram me, sicut tu ambulasti in conspectu meo”. 
\verse Et nunc, Domine, Deus Israel, firmentur verba tua, quae locutus es servo tuo David patri meo. 
\verse Ergone putandum est quod vere Deus habitet super terram? Si enim caelum et caeli caelorum te capere non possunt, quanto magis domus haec, quam aedificavi! 
\verse Sed respice ad orationem servi tui et ad preces eius, Domine Deus meus; audi clamorem et orationem, quam servus tuus orat coram te hodie, 
\verse ut sint oculi tui aperti super domum hanc nocte ac die, super locum, de quo dixisti: “Erit nomen meum ibi”, ut exaudias orationem, qua orat te servus tuus in loco isto, 
\verse ut exaudias deprecationem servi tui et populi tui Israel, quodcumque oraverint in loco isto, et exaudies in loco habitaculi tui in caelo et, cum exaudieris, propitius eris.
 \verse Si peccaverit homo in proximum suum et habuerit aliquod iuramentum, quo teneatur astrictus, et venerit propter iuramentum coram altari tuo in domum istam, 
\verse tu exaudies in caelo et facies et iudicabis servos tuos condemnans impium et reddens viam suam super caput eius iustificansque iustum et retribuens ei secundum iustitiam suam.
 \verse Si superatus fuerit populus tuus Israel ab inimicis suis, quia peccaturus est tibi, et agentes paenitentiam et confitentes nomini tuo venerint et oraverint et deprecati te fuerint in domo hac, 
\verse exaudi in caelo et dimitte peccatum populi tui Israel et reduces eos in terram, quam dedisti patribus eorum.
 \verse Si clausum fuerit caelum et non pluerit propter peccata eorum, et oraverint in loco isto confessi nomini tuo et a peccatis suis conversi propter afflictionem suam, 
\verse exaudi eos in caelo et dimitte peccata servorum tuorum et populi tui Israel et ostende eis viam bonam, per quam ambulent, et da pluviam super terram tuam, quam dedisti populo tuo in possessionem.
 \verse Fames si oborta fuerit in terra aut pestilentia aut uredo aut aurugo aut locusta vel bruchus, et afflixerit eum inimicus eius portas obsidens, omnis plaga, universa infirmitas, 
\verse cuncta oratio et deprecatio, quae acciderit omni homini de populo tuo Israel; si quis cognoverit plagam cordis sui et expanderit manus suas in domo hac, 
\verse tu audies in caelo in loco habitationis tuae et repropitiaberis et facies, ut des unicuique secundum omnes vias suas, sicut videris cor eius, quia tu nosti solus cor omnium filiorum hominum, 
\verse ut timeant te cunctis diebus, quibus vivunt super faciem terrae, quam dedisti patribus nostris.
 \verse Insuper et alienigena, qui non est de populo tuo Israel, cum venerit de terra longinqua propter nomen tuum 
\verse — audietur enim nomen tuum magnum et manus tua fortis et brachium tuum extentum ubique — cum venerit ergo et oraverit in hoc loco, 
\verse tu exaudies in caelo in loco habitationis tuae et facies omnia, pro quibus invocaverit te alienigena, ut sciant universi populi terrarum nomen tuum et timeant te, sicut populus tuus Israel, et probent quia nomen tuum invocatum est super domum hanc, quam aedificavi.
 \verse Si egressus fuerit populus tuus ad bellum contra inimicos suos per viam, quocumque miseris eos, et oraverint te contra viam civitatis, quam elegisti, et contra domum, quam aedificavi nomini tuo, 
\verse exaudies in caelo orationes eorum et preces eorum et facies iudicium eorum.
 \verse Quod si peccaverint tibi — non est enim homo qui non peccet — et iratus tradideris eos inimicis suis, et captivi ducti fuerint in terram inimicorum longe vel prope 
\verse et egerint paenitentiam in corde suo in loco captivitatis et conversi deprecati te fuerint in captivitate sua dicentes: “Peccavimus, inique egimus, impie gessimus”; 
\verse et reversi fuerint ad te in universo corde suo et tota anima sua in terra inimicorum suorum, ad quam captivi ducti sunt, et oraverint te contra viam terrae suae, quam dedisti patribus eorum, et civitatis, quam elegisti, et templi, quod aedificavi nomini tuo, 
\verse exaudies in caelo in firmamento solii tui orationes eorum et preces eorum et facies iudicium eorum; 
\verse et propitiaberis populo tuo, qui peccavit tibi, et omnibus iniquitatibus eorum, quibus praevaricati sunt in te, et dabis misericordiam coram eis, qui eos captivos habuerint, ut misereantur eis 
\verse — populus enim tuus est et hereditas tua, quos eduxisti de terra Aegypti de medio fornacis ferreae — 
\verse ut sint oculi tui aperti ad deprecationem servi tui et populi tui Israel, et exaudias eos in universis, pro quibus invocaverint te. 
\verse Tu enim separasti eos tibi in hereditatem de universis populis terrae, sicut locutus es per Moysen servum tuum, quando eduxisti patres nostros de Aegypto, Domine Deus ”.
 \verse Factum est autem cum complesset Salomon orans Dominum omnem orationem et deprecationem hanc, surrexit de conspectu altaris Domini; utrumque enim genu in terram fixerat et manus expanderat in caelum. 
\verse Stetit ergo et benedixit omni ecclesiae Israel voce magna dicens: 
\verse “ Benedictus Dominus, qui dedit requiem populo suo Israel iuxta omnia, quae locutus est; non cecidit ne unus quidem sermo ex omnibus bonis, quae locutus est per Moysen servum suum. 
\verse Sit Dominus Deus noster nobiscum, sicut fuit cum patribus nostris, non derelinquens nos neque proiciens, 
\verse sed inclinet corda nostra ad se, ut ambulemus in universis viis eius et custodiamus mandata eius et decreta et iudicia, quaecumque mandavit patribus nostris. 
\verse Et sint sermones mei isti, quibus deprecatus sum coram Domino, appropinquantes Domino Deo nostro die ac nocte, ut faciat iudicium servo suo et populo suo Israel per singulos dies, 
 \verse ut sciant omnes populi terrae quia Dominus ipse est Deus, et non est ultra absque eo. 
\verse Sit quoque cor vestrum perfectum cum Domino Deo nostro, ut ambuletis in decretis eius et custodiatis mandata eius sicut et hodie ”.
 \verse Igitur rex et omnis Israel cum eo immolabant victimas coram Domino. 
\verse Mactavitque Salomon hostias pacificas, quas immolavit Domino, boum viginti duo milia et ovium centum viginti milia. Et dedicaverunt templum Domini rex et omnes filii Israel. 
\verse In die illa sanctificavit rex medium atrii, quod erat ante domum Domini; fecit quippe holocaustum ibi et oblationem et adipem pacificorum, quoniam altare aereum, quod erat coram Domino, minus erat et capere non poterat holocaustum et oblationem et adipem pacificorum. 
\verse Fecit ergo Salomon in tempore illo festivitatem celebrem, et omnis Israel cum eo, ecclesia magna ab introitu Emath usque ad rivum Aegypti, coram Domino Deo nostro septem diebus. 
 \verse Et in die octava dimisit populos; qui benedicentes regi profecti sunt in tabernacula sua laetantes et alacri corde super omnibus bonis, quae fecerat Dominus David servo suo et Israel populo suo.
 
\begin{biblechapter}
\verse Factum est autem cum perfecisset Salomon aedificium domus Domini et aedificium regis et omne, quod optaverat et voluerat facere, 
\verse apparuit ei Dominus secundo, sicut apparuerat ei in Gabaon. 
\verse Dixitque Dominus ad eum: “ Exaudivi orationem tuam et deprecationem tuam, quam deprecatus es coram me; sanctificavi domum hanc, quam aedificasti, ut ponerem nomen meum ibi in sempiternum; et erunt oculi mei et cor meum ibi cunctis diebus. 
\verse Tu quoque, si ambulaveris coram me, sicut ambulavit David pater tuus in simplicitate cordis et in aequitate, et feceris omnia, quae praecepi tibi, et legitima mea et iudicia mea servaveris, 
\verse ponam thronum regni tui super Israel in sempiternum, sicut locutus sum David patri tuo dicens: Non auferetur de genere tuo vir de solio Israel. 
\verse Si autem aversione aversi fueritis vos et filii vestri non sequentes me nec custodientes mandata mea et decreta mea, quae proposui vobis, sed abieritis et colueritis deos alienos et adoraveritis eos, 
 \verse auferam Israel de superficie terrae, quam dedi eis, et templum, quod sanctificavi nomini meo, proiciam a conspectu meo; eritque Israel in proverbium et in fabulam cunctis populis, 
\verse et domus haec erit in ruinas. Omnis, qui transierit per eam, stupebit et sibilabit et dicet: “Quare fecit Dominus sic terrae huic et domui huic?”. 
\verse Et respondebunt: “Quia dereliquerunt Dominum Deum suum, qui eduxit patres eorum de terra Aegypti, et secuti sunt deos alienos et adoraverunt eos et coluerunt eos; idcirco induxit Dominus super eos omne malum hoc” ”.
 \verse Expletis autem annis viginti, postquam aedificaverat Salomon duas domos, id est domum Domini et domum regis 
\verse — Hiram rege Tyri praebente Salomoni ligna cedrina et abiegna et aurum iuxta omne quod opus habuerat — tunc dedit Salomon Hiram viginti oppida in terra Galilaeae. 
\verse Et egressus est Hiram de Tyro, ut videret oppida, quae dederat ei Salomon, et non placuerunt ei; 
\verse et ait: “ Haeccine sunt civitates, quas dedisti mihi, frater? ”. Et appellavit eas terram Chabul usque in diem hanc. 
\verse Misit quoque Hiram ad regem centum viginti talenta auri.
 \verse Haec est summa indictionis, quam constituit rex Salomon ad aedificandam domum Domini et domum suam et Mello et murum Ierusalem et Asor et Mageddo et Gazer. 
 \verse Pharao rex Aegypti ascendit et cepit Gazer succenditque eam igni et Chananaeum, qui habitabat in civitate, interfecit; et dedit eam in dotem filiae suae uxori Salomonis. 
\verse Aedificavit ergo Salomon Gazer et Bethoron inferiorem 
\verse et Baalath et Thamar in terra solitudinis 
\verse et omnes civitates horreorum, quae ad se pertinebant, et civitates curruum et civitates equorum et quodcumque ei placuit, ut aedificaret in Ierusalem et in Libano et in omni terra potestatis suae. 
\verse Universum populum, qui remanserat de Amorraeis et Hetthaeis et Pherezaeis et Hevaeis et Iebusaeis, qui non erant de filiis Israel, 
\verse horum filios, qui remanserant post eos in terra, quos scilicet non potuerant filii Israel exterminare, fecit Salomon tributarios usque in diem hanc. 
\verse De filiis autem Israel non constituit Salomon servire quemquam, sed erant viri bellatores et ministri eius et principes et pugnatores eius et praefecti curruum et equitum. 
\verse Erant autem principes eorum, qui super omnia opera Salomonis praepositi erant, quingenti quinquaginta; qui habebant subiectum populum et statutis operibus imperabant.
 \verse Filia autem pharaonis ascendit de civitate David in domum suam, quam aedificaverat ei; tunc aedificavit Mello.
 \verse Offerebat quoque Salomon tribus vicibus per annos singulos holocausta et pacificas victimas super altare, quod aedificaverat Domino, et adolebat coram Domino; perfectumque est templum.
 \verse Classem quoque fecit rex Salomon in Asiongaber, quae est iuxta Ailath in litore maris Rubri in terra Idumaea. 
\verse Misitque Hiram in classe illa servos suos viros nauticos gnaros maris cum servis Salomonis. 
\verse Qui, cum venissent in Ophir, sumptum inde aurum quadringentorum viginti talentorum detulerunt ad regem Salomonem.
 
\begin{biblechapter}
\verse Sed et regina Saba, audita fama Salomonis — in honorem nominis Domini — venit tentare eum in aenigmatibus. 
\verse Et ingressa Ierusalem multo cum comitatu et divitiis, camelis portantibus aromata et aurum infinitum nimis et gemmas pretiosas, venit ad Salomonem et locuta est ei universa, quae habebat in corde suo. 
\verse Et docuit eam Salomon omnia verba, quae proposuerat: non fuit sermo, qui regem posset latere, et non responderet ei. 
\verse Videns autem regina Saba omnem sapientiam Salomonis et domum, quam aedificaverat, 
\verse et cibos mensae eius et sessionem servorum et ordinem ministrantium vestesque eorum et pincernas et holocausta, quae offerebat in domo Domini, non habebat ultra spiritum 
\verse dixitque ad regem: “ Verus est sermo, quem audivi in terra mea super rebus tuis et super sapientia tua! 
\verse Et non credebam narrantibus mihi, donec ipsa veni et vidi oculis meis et probavi quod media pars mihi nuntiata non fuerit; maior est sapientia et bona tua quam rumor, quem audivi. 
\verse Beati viri tui et beati servi tui hi, qui stant coram te semper et audiunt sapientiam tuam! 
\verse Sit Dominus Deus tuus benedictus, cui placuisti, et posuit te super thronum Israel, eo quod dilexerit Dominus Israel in sempiternum et constituit te regem, ut faceres iudicium et iustitiam ”. 
\verse Dedit ergo regi centum viginti talenta auri et aromata multa nimis et gemmas pretiosas; non sunt allata ultra aromata tam multa quam ea, quae dedit regina Saba regi Salomoni.
 \verse Sed et classis Hiram, quae portabat aurum de Ophir, attulit ex Ophir ligna thyina multa nimis et gemmas pretiosas. 
\verse Fecitque rex de lignis thyinis fulcra domus Domini et domus regiae et citharas lyrasque cantoribus. Non sunt allata huiuscemodi ligna thyina neque visa usque in praesentem diem.
 \verse Rex autem Salomon dedit reginae Saba omnia, quae voluit et petivit ab eo, praeter ea, quae ultro obtulerat ei munere regio. Quae reversa est et abiit in terram suam cum servis suis.
 \verse Erat autem pondus auri, quod afferebatur Salomoni per annos singulos, sescentorum sexaginta sex talentorum auri, 
\verse praeter id, quod proveniebat ex tributis subiectorum et commercio negotiatorum et omnium regum Arabiae et ducum terrae.
 \verse Fecit quoque rex Salomon ducenta scuta de auro puro, sescentos auri siclos dedit in laminas scuti unius; 
\verse et trecentas peltas ex auro probato, tres minae auri unam peltam vestiebant; posuitque ea rex in domo Saltus Libani.
 \verse Fecit etiam rex Salomon thronum de ebore grandem et vestivit eum auro fulvo nimis. 
\verse Qui habebat sex gradus, et summitas throni rotunda erat in parte posteriori, et duae manus hinc atque inde tenentes sedile, et duo leones stabant iuxta manus; 
\verse et duodecim leunculi stantes super sex gradus hinc atque inde. Non est factum tale opus in universis regnis.
 \verse Sed et omnia vasa, quibus potabat rex Salomon, erant aurea, et universa supellex domus Saltus Libani de auro purissimo; non erat argentum nec alicuius pretii putabatur in diebus Salomonis, 
\verse quia classis Tharsis, quae regi erat, per mare cum classe Hiram semel per tres annos redibat deferens aurum et argentum et ebur et simias et pavos. 
\verse Magnificatus est ergo rex Salomon super omnes reges terrae divitiis et sapientia. 
\verse Et universa terra desiderabat vultum Salomonis, ut audiret sapientiam eius, quam dederat Deus in corde eius. 
\verse Et singuli deferebant ei munera, vasa argentea et aurea, vestes et arma bellica, aromata quoque et equos et mulos per annos singulos.
 \verse Congregavitque Salomon currus et equites, et facti sunt ei mille quadringenti currus et duodecim milia equitum; et disposuit eos per civitates quadrigarum et cum rege in Ierusalem. 
\verse Fecitque ut tanta esset abundantia argenti in Ierusalem quanta et lapidum; et cedrorum praebuit multitudinem quasi sycomoros, quae nascuntur in Sephela. 
\verse Et educebantur equi Salomoni de Aegypto et de Coa; negotiatores enim regis emebant de Coa statuto pretio. 
\verse Constabat autem et egrediebatur quadriga ex Aegypto sescentis siclis argenti, et equus centum quinquaginta; atque in hunc modum cunctis regibus Hetthaeorum et Syriae per manus suas venundabant.
 
\begin{biblechapter}
\verse Rex autem Salomon amavit mulieres alienigenas multas, filiam quoque pharaonis et Moabitidas et Ammonitidas, Idumaeas et Sidonias et Hetthaeas, 
\verse de gentibus, super quibus dixit Dominus filiis Israel: “ Non ingrediemini ad eas, neque de illis ingredientur ad vestras; certissime enim avertent corda vestra, ut sequamini deos earum ”. His itaque copulatus est Salomon amore; 
\verse fueruntque ei uxores quasi reginae septingentae et concubinae trecentae, et averterunt mulieres cor eius. 
\verse Cumque iam esset senex, depravatum est cor eius per mulieres, ut sequeretur deos alienos; nec erat cor eius perfectum cum Domino Deo suo sicut cor David patris eius, 
\verse sed colebat Salomon Astharthen, deam Sidoniorum, et Melchom idolum Ammonitarum. 
\verse Fecitque Salomon quod non placuerat coram Domino et non adimplevit ut sequeretur Dominum sicut David pater eius. 
\verse Tunc aedificavit Salomon fanum Chamos idolo Moab in monte, qui est contra Ierusalem, et Melchom idolo filiorum Ammon; 
\verse atque in hunc modum fecit universis uxoribus suis alienigenis, quae adolebant et immolabant diis suis.
 \verse Igitur iratus est Dominus Salomoni, quod aversa esset mens eius a Domino, Deo Israel, qui apparuerat ei bis 
\verse et praeceperat de verbo hoc, ne sequeretur deos alienos; et non custodivit, quae mandavit ei Dominus. 
\verse Dixit itaque Dominus Salomoni: “ Quia habuisti hoc apud te et non custodisti pactum meum et praecepta mea, quae mandavi tibi, disrumpens scindam regnum tuum a te et dabo illud servo tuo. 
\verse Verumtamen in diebus tuis non faciam propter David patrem tuum; de manu filii tui scindam illud. 
\verse Nec totum regnum auferam, sed tribum unam dabo filio tuo propter David servum meum et Ierusalem, quam elegi ”.
 \verse Suscitavit autem Dominus adversarium Salomoni Adad Idumaeum, qui erat de semine regio, in Edom. 
\verse Cum enim vicisset David Idumaeam, et ascendisset Ioab princeps militiae ad sepeliendum eos, qui fuerant interfecti, et occidisset omne masculinum in Idumaea 
\verse — sex enim mensibus ibi moratus est Ioab et omnis Israel, donec interimerent omne masculinum in Idumaea — 
\verse fugit Adad ipse et viri Idumaei de servis patris eius cum eo, ut ingrederetur Aegyptum; erat autem Adad puer parvulus. 
\verse Cumque surrexissent de Madian, venerunt in Pharan tuleruntque secum viros de Pharan et introierunt Aegyptum ad pharaonem regem Aegypti, qui dedit ei domum et cibos constituit et terram delegavit. 
 \verse Et invenit Adad gratiam coram pharao valde, in tantum ut daret ei uxorem sororem uxoris suae germanam Taphnes reginae. 
\verse Genuitque ei soror Taphnes Genubath filium et ablactavit eum Taphnes in domo pharaonis, eratque Genubath habitans apud pharaonem cum filiis eius. 
\verse Cumque audisset Adad in Aegypto dormisse David cum patribus suis et mortuum esse Ioab principem militiae, dixit pharaoni: “ Dimitte me, ut vadam in terram meam ”. 
\verse Dixitque ei pharao: “ Qua enim re apud me indiges, ut quaeras ire ad terram tuam? ”. At ille respondit: “ Nulla; sed obsecro, ut dimittas me ”.
 \verse Suscitavit quoque Deus Salomoni adversarium Razon filium Eliada, qui fugerat ab Adadezer rege Soba domino suo. 
\verse Et congregavit ad se viros et factus est princeps turmae, cum interficeret eos David; abieruntque Damascum et habitaverunt ibi et regnaverunt in Damasco. 
\verse Eratque adversarius Israeli cunctis diebus Salomonis; et hoc cum malo, quod erat Adad. Et detestatus est Israel regnavitque in Syria.
 \verse Ieroboam quoque filius Nabat, Ephrathaeus de Sareda, servus Salomonis, cuius mater erat nomine Sarva mulier vidua, levavit manum contra regem. 
\verse Et haec causa rebellionis adversus eum: Salomon aedificavit Mello et coaequavit voraginem civitatis David patris sui. 
\verse Erat autem Ieroboam vir fortis et strenuus; vidensque Salomon adulescentem industrium constituerat eum praefectum super labores universae domus Ioseph.
 \verse Factum est igitur in tempore illo, ut Ieroboam egrederetur de Ierusalem, et inveniret eum Ahias Silonites propheta in via opertus pallio novo; erant autem duo tantum in agro. 
\verse Apprehendensque Ahias pallium suum novum, quo coopertus erat, scidit in duodecim partes 
\verse et ait ad Ieroboam: “ Tolle tibi decem scissuras; haec enim dicit Dominus, Deus Israel: Ecce ego scindam regnum de manu Salomonis et dabo tibi decem tribus. 
\verse Porro una tribus remanebit ei propter servum meum David et Ierusalem civitatem, quam elegi ex omnibus tribubus Israel; 
\verse eo quod dereliquerint me et adoraverint Astharthen deam Sidoniorum et Chamos deum Moab et Melchom deum filiorum Ammon et non ambulaverint in viis meis, ut facerent iustitiam coram me et praecepta mea et iudicia sicut David pater eius. 
\verse Nec auferam omne regnum de manu eius, sed ducem ponam eum cunctis diebus vitae suae propter David servum meum, quem elegi, qui custodivit mandata mea et praecepta mea. 
\verse Auferam autem regnum de manu filii eius et dabo tibi decem tribus; 
\verse filio autem eius dabo tribum unam, ut remaneat lucerna David servo meo cunctis diebus coram me in Ierusalem civitate, quam elegi, ut esset nomen meum ibi. 
\verse Te autem assumam, et regnabis super omnia, quae desiderat anima tua, erisque rex super Israel. 
\verse Si igitur audieris omnia, quae praecepero tibi, et ambulaveris in viis meis et feceris, quod rectum est coram me custodiens mandata mea et praecepta mea, sicut fecit David servus meus, ero tecum et aedificabo tibi domum stabilem, quomodo aedificavi David, et tradam tibi Israel 
\verse et affligam semen David super hoc, verumtamen non cunctis diebus ”.
 \verse Voluit ergo Salomon interficere Ieroboam, qui surrexit et aufugit in Aegyptum ad Sesac regem Aegypti et fuit in Aegypto usque ad mortem Salomonis.
 \verse Reliqua autem gestorum Salomonis, omnia, quae fecit, et sapientia eius, ecce universa scripta sunt in libro gestorum Salomonis; 
\verse dies autem, quos regnavit Salomon in Ierusalem super omnem Israel, quadraginta anni sunt. 
\verse Dormivitque Salomon cum patribus suis et sepultus est in civitate David patris sui; regnavitque Roboam filius eius pro eo.
 
\begin{biblechapter}
\verse Venit autem Roboam in Sichem; illuc enim congregatus erat omnis Israel ad constituendum eum regem. 
\verse At Ieroboam filius Nabat, cum adhuc esset in Aegypto profugus a facie regis Salomonis, audito hoc nuntio, reversus est de Aegypto. 
\verse Miseruntque et vocaverunt eum. Venit ergo Ieroboam et omnis multitudo Israel, et locuti sunt ad Roboam dicentes: 
\verse “ Pater tuus durissimum iugum imposuit nobis; tu itaque nunc imminue paululum de imperio patris tui durissimo et de iugo gravissimo, quod imposuit nobis, et serviemus tibi ”. 
\verse Qui ait eis: “ Ite usque ad tertium diem et revertimini ad me ”.
 Cumque abisset populus, 
\verse iniit consilium rex Roboam cum senioribus, qui assistebant coram Salomone patre eius, cum adhuc viveret, et ait: “ Quod mihi datis consilium, ut respondeam populo huic? ”. 
\verse Qui dixerunt ei: “ Si hodie oboedieris populo huic et servieris et petitioni eorum cesseris locutusque fueris ad eos verba lenia, erunt tibi servi cunctis diebus ”. 
\verse Qui dereliquit consilium senum, quod dederant ei, et adhibuit adulescentes, qui nutriti fuerant cum eo et assistebant illi, 
\verse dixitque ad eos: “ Quod mihi datis consilium, ut respondeam populo huic, qui dixerunt mihi: “Levius fac iugum, quod imposuit pater tuus super nos”? ”. 
\verse Et dixerunt ei iuvenes, qui nutriti fuerant cum eo: “ Sic loquere populo huic, qui locuti sunt ad te dicentes: “Pater tuus aggravavit iugum nostrum, tu releva nos”; sic loqueris ad eos: Minimus digitus meus grossior est lumbis patris mei. 
\verse Et nunc, pater meus posuit super vos iugum grave, ego autem addam super iugum vestrum; pater meus cecidit vos flagellis, ego autem caedam scorpionibus ”.
 \verse Venit ergo Ieroboam et omnis populus ad Roboam die tertia, sicut locutus fuerat rex dicens: “ Revertimini ad me die tertia ”. 
\verse Responditque rex populo dura, derelicto consilio seniorum, quod ei dederant, 
\verse et locutus est eis secundum consilium iuvenum dicens:
 “ Pater meus aggravavit iugum vestrum,
 ego autem addam iugo vestro;
 pater meus cecidit vos flagellis,
 ego autem caedam vos scorpionibus ”.
 \verse Ergo non acquievit rex populo, quoniam dispositum erat a Domino, ut suscitaret verbum suum, quod locutus fuerat in manu Ahiae Silonitae ad Ieroboam filium Nabat.
 \verse Videns itaque omnis Israel quod noluisset eos audire rex, respondit ei dicens:
 “ Quae nobis pars in David,
 vel quae hereditas in filio Isai?
 Vade in tabernacula tua, Israel!
 Nunc vide domum tuam, David! ”. Et abiit Israel in tabernacula sua. 
\verse Super filios autem Israel, quicumque habitabant in civitatibus Iudae, regnavit Roboam. 
\verse Misit rex Roboam Adoniram, qui erat super servitutem; et lapidavit eum omnis Israel, et mortuus est. Porro rex Roboam festinus ascendit currum et fugit in Ierusalem. 
\verse Recessitque Israel a domo David usque in praesentem diem.
 \verse Factum est autem cum audisset omnis Israel quod reversus esset Ieroboam, miserunt et vocaverunt eum, congregato coetu, et constituerunt eum regem super omnem Israel; nec secutus est quisquam domum David praeter tribum Iudae solam.
 \verse Venit autem Roboam Ierusalem et congrcgavit universam domum Iudae et tribum Beniamin, centum octoginta milia electorum virorum bellatorum, ut pugnaret contra domum Israel et reduceret regnum Roboam filio Salomonis. 
\verse Factus est vero sermo Domini ad Semeiam virum Dei dicens: 
\verse “ Loquere ad Roboam filium Salomonis regem Iudae et ad omnem domum Iudae et Beniamin et reliquos de populo dicens: 
\verse Haec dicit Dominus: Non ascendetis neque bellabitis contra fratres vestros, filios Israel; revertatur vir in domum suam; a me enim factum est hoc ”. Audierunt sermonem Domini et reversi sunt de itinere, sicut eis praeceperat Dominus.
 \verse Aedificavit autem Ieroboam Sichem in monte Ephraim et habitavit ibi; et egressus inde aedificavit Phanuel.
 \verse Dixitque Ieroboam in corde suo: “ Nunc revertetur regnum ad domum David, 
 \verse si ascenderit populus iste, ut faciat sacrificia in domo Domini in Ierusalem, et convertetur cor populi huius ad dominum suum Roboam regem Iudae, interficientque me et revertentur ad Roboam regem Iudae ”. 
\verse Et excogitato consilio, fecit rex duos vitulos aureos et dixit ad populum: “ Nolite ultra ascendere in Ierusalem! Ecce dii tui, Israel, qui te eduxerunt de terra Aegypti ”. 
\verse Posuitque unum in Bethel et alterum donavit in Dan; 
\verse et factum est hoc in peccatum: ibat enim populus coram uno usque in Dan. 
\verse Et fecit fana in excelsis et sacerdotes de extremis populi, qui non erant de filiis Levi. 
 \verse Constituitque diem sollemnem in mense octavo, quinta decima die mensis, in similitudinem sollemnitatis, quae celebratur in Iuda. Et ascendit altare; sic fecit in Bethel, ut immolaret vitulis, quos fabricatus erat; constituitque in Bethel sacerdotes excelsorum, quae fecerat. 
\verse Et ascendit super altare, quod exstruxerat in Bethel, quinta decima die mensis octavi, quem finxerat de corde suo; et fecit sollemnitatem filiis Israel et ascendit super altare, ut adoleret.
 
\begin{biblechapter}
\verse Et ecce vir Dei venit de Iuda in sermone Domini in Bethel, Ieroboam stante super altare ad adolendum; 
\verse et exclamavit contra altare in sermone Domini et ait: “ Altare, altare, haec dicit Dominus: Ecce filius nascetur domui David, Iosias nomine, et immolabit super te sacerdotes excelsorum, qui nunc in te immolant, et ossa hominum super te incendent ”. 
\verse Deditque in illa die signum dicens: “ Hoc erit signum, quod locutus est Dominus: ecce altare scindetur, et effundetur cinis, qui in eo est ”.
 \verse Cumque audisset rex sermonem hominis Dei, quem inclamaverat contra altare in Bethel, extendit manum suam de altari dicens: “ Apprehendite eum! ”. Et exaruit manus eius, quam extenderat contra eum, nec valuit retrahere eam ad se. 
\verse Altare quoque scissum est, et effusus est cinis de altari iuxta signum, quod praedixerat vir Dei in sermone Domini. 
\verse Et ait rex ad virum Dei: “ Deprecare faciem Domini Dei tui et ora pro me, ut restituatur manus mea mihi ”. Oravit vir Dei faciem Domini, et reversa est manus regis ad eum et facta est sicut prius fuerat.
 \verse Locutus est autem rex ad virum Dei: “ Veni mecum domum, ut prandeas, et dabo tibi munera ”. 
\verse Responditque vir Dei ad regem: “ Si dederis mihi mediam partem domus tuae, non veniam tecum nec comedam panem neque bibam aquam in loco isto; 
\verse sic enim mandatum est mihi in sermone Domini praecipientis: “Non comedes panem neque bibes aquam nec reverteris per viam, qua venisti” ”. 
\verse Abiit ergo per aliam viam et non est reversus per iter, quo venerat in Bethel.
 \verse Prophetes autem quidam senex habitabat in Bethel; ad quem venerunt filii sui et narraverunt ei omnia opera, quae fecerat vir Dei illa die in Bethel, et verba, quae locutus fuerat ad regem, narraverunt quoque patri suo. 
\verse Et dixit eis pater eorum: “ Per quam viam abiit? ”. Ostenderunt ei filii sui viam, per quam abierat vir Dei, qui venerat de Iuda. 
\verse Et ait filiis suis: “ Sternite mihi asinum ”. Qui cum stravissent, ascendit 
\verse et abiit post virum Dei et invenit eum sedentem subtus terebinthum et ait illi: “ Tune es vir Dei, qui venisti de Iuda?”. Respondit ille: “ Ego sum ”. 
\verse Dixit ad eum: “ Veni mecum domum, ut comedas panem ”. 
\verse Qui ait: “ Non possum reverti neque venire tecum nec comedam panem neque bibam aquam in loco isto; 
\verse sic enim dictum est mihi in sermone Domini: “Non comedes panem et non bibes ibi aquam nec reverteris per viam, qua ieris” ”. 
\verse Qui ait illi: “ Et ego propheta sum similis tui; et angelus locutus est mihi in sermone Domini dicens: “Reduc eum tecum in domum tuam, et comedat panem et bibat aquam” ”. Fefellit eum 
\verse et reduxit secum; comedit ergo panem in domo eius et bibit aquam.
 \verse Cumque sederent ad mensam, factus est sermo Domini ad prophetam, qui reduxerat eum, 
\verse et exclamavit ad virum Dei, qui venerat de Iuda, dicens: “ Haec dicit Dominus: Quia non oboediens fuisti ori Domini et non custodisti mandatum, quod praecepit tibi Dominus Deus tuus, 
\verse et reversus es et comedisti panem et bibisti aquam in loco, in quo praecepit tibi, ne comederes panem neque biberes aquam, non inferetur cadaver tuum in sepulcrum patrum tuorum ”.
 \verse Cumque comedisset panem et bibisset, stravit sibi asinum prophetae, qui reduxerat eum; 
\verse et, cum abisset, invenit eum leo in via et occidit, et erat cadaver eius proiectum in itinere; asinus autem stabat iuxta illum, et leo stabat iuxta cadaver. 
\verse Et ecce viri transeuntes viderunt cadaver proiectum in via et leonem stantem iuxta cadaver; et venerunt et divulgaverunt in civitate, in qua prophetes ille senex habitabat. 
\verse Quod cum audisset propheta ille, qui reduxerat eum de via, ait: “ Vir Dei est, qui inoboediens fuit ori Domini, et tradidit eum Dominus leoni; et confregit eum et occidit iuxta verbum Domini, quod locutus est ei ”. 
\verse Dixitque ad filios suos: “ Sternite mihi asinum! ”. Qui cum stravissent, 
\verse et ille abisset, invenit cadaver eius proiectum in via et asinum et leonem stantes iuxta cadaver; non comedit leo de cadavere nec laesit asinum.
 \verse Tulit ergo prophetes cadaver viri Dei et posuit illud super asinum et reversus intulit in civitatem prophetae senis, ut plangerent eum et sepelirent. 
\verse Et posuit cadaver eius in sepulcro suo, et planxerunt eum: “ Heu, heu, mi frater! ”. 
\verse Cumque sepelissent eum, dixit ad filios suos: “ Cum mortuus fuero, sepelite me in sepulcro, in quo vir Dei sepultus est; iuxta ossa eius ponite ossa mea. 
\verse Profecto enim veniet sermo, quem praedixit in sermone Domini contra altare, quod est in Bethel, et contra omnia fana excelsorum, quae sunt in urbibus Samariae ”.
 \verse Post haec non est reversus Ieroboam de via sua pessima, sed iterum faciebat de novissimis populi sacerdotes excelsorum; quicumque volebat, implebat eius manum, ut fieret sacerdos excelsorum. 
\verse Et propter hanc causam peccavit domus Ieroboam, et eversa est et deleta de superficie terrae.
 
\begin{biblechapter}
 \verse In tempore illo aegrotavit Abia filius Ieroboam, 
 \verse dixitque Ieroboam uxori suae: “ Surge et commuta habitum, ne cognoscaris quod sis uxor Ieroboam, et vade in Silo, ubi est Ahias propheta, qui locutus est mihi quod regnaturus essem super populum hunc. 
\verse Tolle quoque in manu tua decem panes et crustula et vas mellis et vade ad illum: ipse indicabit tibi quid eventurum sit puero ”. 
 \verse Fecit, ut dixerat, uxor Ieroboam et consurgens abiit in Silo et venit in domum Ahiae; at ille non poterat videre, quia caligaverant oculi eius prae senectute.
 \verse Dixerat autem Dominus ad Ahiam: “ Ecce uxor Ieroboam ingredietur, ut consulat te super filio suo, qui aegrotat; haec et haec loqueris ei. Cum intret, simulabit se peregrinam esse ”.
 \verse Cum ergo audiret Ahias sonitum pedum eius introeuntis per ostium, ait: “ Ingredere, uxor Ieroboam. Quare aliam te esse simulas? Ego autem missus sum ad te durus nuntius. 
\verse Vade et dic Ieroboam: “Haec dicit Dominus, Deus Israel: Quia exaltavi te de medio populi et dedi te ducem super populum meum Israel 
\verse et scidi regnum a domo David et dedi illud tibi, et non fuisti sicut servus meus David, qui custodivit mandata mea et secutus est me in toto corde suo faciens quod placitum esset in conspectu meo, 
 \verse sed operatus es mala super omnes, qui fuerunt ante te, et fecisti tibi deos alienos et conflatiles, ut me ad iracundiam provocares, me autem proiecisti post tergum tuum: 
\verse idcirco ecce ego inducam mala super domum Ieroboam et percutiam de Ieroboam quidquid masculini sexus, impuberem et puberem in Israel; et mundabo reliquias domus Ieroboam, sicut mundari solet fimus usque ad purum. 
 \verse Qui mortui fuerint de Ieroboam in civitate, comedent eos canes; qui autem mortui fuerint in agro, vorabunt eos aves caeli, quia Dominus locutus est. 
 \verse Tu igitur surge et vade in domum tuam, et in ipso introitu pedum tuorum in urbem morietur puer, 
\verse et planget eum omnis Israel et sepeliet; iste enim solus inferetur de Ieroboam in sepulcrum, quia inventum est in eo, quod bonum erat Domino, Deo Israel, in domo Ieroboam. 
\verse Constituet autem sibi Dominus regem super Israel, qui percutiat domum Ieroboam. 
\verse Et percutiet Dominus Israel, ut moveatur sicut arundo in aqua, et evellet Israel de terra bona hac, quam dedit patribus eorum; et ventilabit eos trans Flumen, quia fecerunt sibi palos, ut irritarent Dominum. 
\verse Et tradet Dominus Israel propter peccata Ieroboam, qui peccavit et peccare fecit Israel” ”.
 \verse Surrexit itaque uxor Ieroboam et abiit et venit in Thersa; cumque illa ingrederetur limen domus, puer mortuus est. 
\verse Et sepelierunt eum, et planxit illum omnis Israel iuxta sermonem Domini, quem locutus est in manu servi sui Ahiae prophetae.
 \verse Reliqua autem gestorum Ieroboam, quomodo pugnaverit et quomodo regnaverit, ecce scripta sunt in libro annalium regum Israel. 
\verse Dies autem, quibus regnavit Ieroboam, viginti duo anni sunt; et dormivit cum patribus suis. Regnavitque Nadab filius eius pro eo.
 \verse Porro Roboam filius Salomonis regnavit in Iuda. Quadraginta et unius anni erat Roboam, cum regnare coepisset, et decem et septem annos regnavit in Ierusalem civitate, quam elegit Dominus, ut poneret nomen suum ibi ex omnibus tribubus Israel. Nomen autem matris eius Naama Ammanites.
 \verse Et fecit Iuda malum coram Domino, et irritaverunt eum super omnibus, quae fecerant patres eorum in peccatis suis, quae peccaverant; 
\verse aedificaverunt enim et ipsi sibi excelsa et lapides et palos super omnem collem excelsum et subter omnem arborem frondosam. 
\verse Sed et prostibula fuerunt in terra; feceruntque omnes abominationes gentium, quas attrivit Dominus ante faciem filiorum Israel.
 \verse In quinto autem anno regni Roboam ascendit Sesac rex Aegypti in Ierusalem 
 \verse et tulit thesauros domus Domini et thesauros regios et universa diripuit, scuta quoque aurea omnia, quae fecerat Salomon. 
\verse Pro quibus fecit rex Roboam scuta aerea et tradidit ea in manu ducum cursorum, qui excubabant ante ostium domus regis. 
\verse Cumque ingrederetur rex in domum Domini, portabant ea cursores et postea reportabant ad armamentarium cursorum.
 \verse Reliqua autem gestorum Roboam et omnia, quae fecit, ecce scripta sunt in libro annalium regum Iudae. 
\verse Fuitque bellum inter Roboam et Ieroboam cunctis diebus. 
\verse Dormivit itaque Roboam cum patribus suis et sepultus est cum eis in civitate David; nomen autem matris eius Naama Ammanites. Et regnavit Abiam filius eius pro eo.
 
\begin{biblechapter}
\verse Igitur in octavo decimo anno regni Ieroboam filii Nabat regnavit Abiam super Iudam. 
\verse Tribus annis regnavit in Ierusalem; nomen matris eius Maacha filia Abessalom. 
\verse Ambulavitque in omnibus peccatis patris sui, quae fecerat ante eum; nec erat cor eius perfectum cum Domino Deo suo sicut cor David patris eius. 
\verse Sed propter David dedit ei Dominus Deus suus lucernam in Ierusalem, ut suscitaret filium eius post eum et statueret Ierusalem; 
\verse eo quod fecisset David rectum in oculis Domini et non declinasset ab omnibus, quae praeceperat ei, cunctis diebus vitae suae, excepta re Uriae Hetthaei. ( \verse )
 \verse Reliqua autem gestorum Abiam et omnia, quae fecit, nonne haec scripta sunt in libro annalium regum Iudae? Fuitque bellum inter Abiam et inter Ieroboam. 
\verse Et dormivit Abiam cum patribus suis, et sepelierunt eum in civitate David; regnavitque Asa filius eius pro eo.
 \verse In anno ergo vicesimo Ieroboam regis Israel regnavit Asa rex Iudae 
\verse et quadraginta et uno anno regnavit in Ierusalem. Nomen matris eius Maacha filia Abessalom.
 \verse Et fecit Asa rectum ante conspectum Domini sicut David pater eius. 
\verse Et abstulit prostibula de terra purgavitque universas sordes idolorum, quae fecerant patres eius. 
\verse Insuper et Maacham matrem suam amovit, ne esset domina, eo quod fecisset abominationem Aserae; confregitque Asa simulacrum turpissimum et combussit in torrente Cedron. 
\verse Excelsa autem non abstulit; verumtamen cor Asa perfectum erat coram Domino cunctis diebus suis. 
\verse Et intulit ea, quae sanctificaverat pater suus et quae ipse voverat, in domum Domini, argentum et aurum et vasa.
 \verse Bellum autem erat inter Asa et Baasa regem lsrael cunctis diebus eorum. 
 \verse Ascendit quoque Baasa rex Israel in Iudam et aedificavit Rama, ut non posset quispiam egredi vel ingredi de parte Asa regis Iudae. 
\verse Tollens itaque Asa omne argentum et aurum, quod remanserat in thesauris domus Domini et in thesauris domus regiae, dedit illud in manu servorum suorum et misit ad Benadad filium Tabremmon filii Hezion regem Syriae, qui habitabat in Damasco, dicens: 
 \verse “ Foedus est inter me et te et inter patrem meum et patrem tuum; ideo misi tibi munera, argentum et aurum, et peto, ut irritum facias foedus, quod habes cum Baasa rege Israel, et recedat a me ”. 
\verse Acquiescens Benadad regi Asa misit principes exercituum suorum in civitates Israel, et percusserunt Ahion et Dan et Abelbethmaacha et universam Chenereth cum omni terra Nephthali. 
\verse Quod cum audisset Baasa, cessavit aedificare Rama et reversus est in Thersa. 
 \verse Rex autem Asa convocavit omnem Iudam, nullo excusato; et tulerunt lapides Rama et ligna eius, quibus aedificaverat Baasa, et exstruxit de eis rex Asa Gabaa Beniamin et Maspha.
 \verse Reliqua autem omnium gestorum Asa et universa fortitudo eius et cuncta, quae fecit, et civitates, quas exstruxit, nonne haec scripta sunt in libro annalium regum Iudae? Verumtamen in tempore senectutis suae doluit pedes; 
\verse et dormivit cum patribus suis et sepultus est cum eis in civitate David patris sui. Regnavitque Iosaphat filius eius pro eo.
 \verse Nadab vero filius Ieroboam regnavit super Israel anno secundo Asa regis Iudae; regnavitque super Israel duobus annis. 
\verse Et fecit, quod malum est in conspectu Domini, et ambulavit in viis patris sui et in peccato eius, quo peccare fecit Israel. 
\verse Insidiatus est autem ei Baasa filius Ahiae de domo Issachar et percussit eum in Gebbethon, quae est urbs Philisthinorum; siquidem Nadab et omnis Israel obsidebant Gebbethon. 
\verse Interfecit igitur illum Baasa in anno tertio Asa regis Iudae et regnavit pro eo. 
\verse Cumque regnasset, percussit omnem domum Ieroboam; non dimisit ne unam quidem animam de semine eius, donec deleret eam iuxta verbum Domini, quod locutus fuerat in manu servi sui Ahiae Silonitis 
\verse propter peccata Ieroboam, quae peccaverat et quibus peccare fecerat Israel, et propter delictum, quo irritaverat Dominum, Deum Israel.
 \verse Reliqua autem gestorum Nadab et omnia, quae fortiter operatus est, nonne haec scripta sunt in libro annalium regum Israel? 
\verse Fuitque bellum inter Asa et Baasa regem Israel cunctis diebus eorum.
 \verse Anno tertio Asa regis Iudae regnavit Baasa filius Ahiae super omnem Israel in Thersa viginti quattuor annis; 
\verse et fecit malum coram Domino ambulavitque in via Ieroboam et in peccato eius, quo peccare fecit Israel.
 
\begin{biblechapter}
\verse Factus est autem sermo Domini ad Iehu filium Hanani contra Baasa dicens: 
\verse “ Pro eo quod exaltavi te de pulvere et posui te ducem super populum meum Israel, tu autem ambulasti in via Ieroboam et peccare fecisti populum meum Israel, ut me irritares in peccatis eorum, 
\verse ecce ego demetam posteriora Baasa et posteriora domus eius et faciam domum tuam sicut domum Ieroboam filii Nabat. 
\verse Qui mortuus fuerit de Baasa in civitate, comedent eum canes; et, qui mortuus fuerit ex eo in agro, comedent eum volucres caeli”.
 \verse Reliqua autem gestorum Baasa et quaecumque fecit et fortitudo eius, nonne haec scripta sunt in libro annalium regum Israel? 
\verse Dormivit ergo Baasa cum patribus suis sepultusque est in Thersa; et regnavit Ela filius eius pro eo.
 \verse Sed et in manu Iehu filii Hanani prophetae verbum Domini factum est ad Baasa et ad domum eius propter omne malum, quod fecerat coram Domino ad irritandum eum in operibus manuum suarum, ut fieret sicut domus Ieroboam, eo quod percussisset eam.
 \verse Anno vicesimo sexto Asa regis Iudae regnavit Ela filius Baasa super Israel in Thersa duobus annis. 
\verse Et rebellavit contra eum servus suus Zamri dux mediae partis curruum. Erat autem Ela in Thersa bibens et temulentus in domo Arsa praefecti domus in Thersa; 
\verse irruens ergo Zamri percussit et occidit eum anno vicesimo septimo Asa regis Iudae et regnavit pro eo. 
\verse Cumque regnasset et sedisset super solium eius, percussit omnem domum Baasa et non dereliquit ex eo quidquid masculini sexus et propinquos et amicos eius. 
\verse Delevitque Zamri omnem domum Baasa iuxta verbum Domini, quod locutus fuerat ad Baasa in manu Iehu prophetae, 
\verse propter universa peccata Baasa et peccata Ela filii eius, qui peccaverunt et peccare fecerunt Israel provocantes Dominum, Deum Israel, in vanitatibus suis.
 \verse Reliqua autem gestorum Ela et omnia, quae fecit, nonne haec scripta sunt in libro annalium regum Israel?
 \verse Anno vicesimo septimo Asa regis Iudae regnavit Zamri septem diebus in Thersa. Porro exercitus obsidebat Gebbethon urbem Philisthinorum. 
\verse Cumque audisset rebellasse Zamri et occidisse regem, fecit sibi regem omnis Israel Amri, qui erat princeps militiae super Israel in die illa in castris. 
\verse Ascendit ergo Amri et omnis Israel cum eo de Gebbethon, et obsidebant Thersa; 
\verse videns autem Zamri quod expugnanda esset civitas, ingressus est palatium et succendit super se domum regiam et mortuus est igne 
\verse in peccatis suis, quae peccaverat faciens malum coram Domino et ambulans in via Ieroboam et in peccato eius, quo fecit peccare Israel.
 \verse Reliqua autem gestorum Zamri et rebellio, quam fecit, nonne haec scripta sunt in libro annalium regum Israel?
 \verse Tunc divisus est populus Israel in duas partes: media pars populi sequebatur Thebni filium Gineth, ut constitueret eum regem, et media pars Amri. 
\verse Praevaluit autem populus, qui erat cum Amri, populo, qui sequebatur Thebni filium Gineth; mortuusque est Thebni, et regnavit Amri.
 \verse Anno tricesimo primo Asa regis Iudae regnavit Amri super Israel duodecim annis; in Thersa regnavit sex annis. 
\verse Emitque montem Samariae a Somer duobus talentis argenti et aedificavit eum et vocavit nomen civitatis, quam exstruxerat, nomine Somer domini montis Samariae. 
\verse Fecit autem Amri malum in conspectu Domini et operatus est nequiter super omnes, qui fuerunt ante eum; 
 \verse ambulavitque in omni via Ieroboam filii Nabat et in peccato eius, quo peccare fecerat Israel, ut irritaret Dominum, Deum Israel, in vanitatibus suis.
 \verse Reliqua autem gestorum Amri et proelia eius, quae fortiter gessit, nonne haec scripta sunt in libro annalium regum Israel? 
\verse Et dormivit Amri cum patribus suis et sepultus est in Samaria; regnavitque Achab filius eius pro eo.
 \verse Achab vero filius Amri regnavit super Israel anno tricesimo octavo Asa regis Iudae; et regnavit Achab filius Amri super Israel in Samaria viginti et duobus annis. 
\verse Et fecit Achab filius Amri malum in conspectu Domini super omnes, qui fuerunt ante eum. 
\verse Nec suffecit ei, ut ambularet in peccatis Ieroboam filii Nabat; insuper duxit uxorem Iezabel filiam Ethbaal regis Sidoniorum et abiit et servivit Baal et adoravit eum. 
\verse Et posuit aram Baal in templo Baal, quod aedificaverat in Samaria, 
\verse et fecit Achab palum. Et addidit Achab in opere suo irritans Dominum, Deum Israel, super omnes reges Israel, qui fuerant ante eum. 
\verse In diebus eius aedificavit Hiel de Bethel Iericho; in Abiram primitivo suo fundavit eam et in Segub novissimo suo posuit portas eius, iuxta verbum Domini, quod locutus fuerat in manu Iosue filii Nun.
 
\begin{biblechapter}
\verse Et dixit Elias Thesbites de Thesbi in Galaad ad Achab: “ Vivit Dominus, Deus Israel, in cuius conspectu sto. Non erit annis his ros et pluvia, nisi iuxta oris mei verba! ”.
 \verse Et factum est verbum Domini ad eum dicens: 
\verse “ Recede hinc et vade contra orientem et abscondere in torrente Charith, qui est contra Iordanem, 
\verse et ibi de torrente bibes; corvisque praecepi, ut pascant te ibi ”. 
\verse Abiit ergo et fecit iuxta verbum Domini; cumque abisset, sedit in torrente Charith, qui est contra Iordanem. 
\verse Corvi quoque deferebant ei panem et carnes mane, similiter panem et carnes vesperi; et bibebat de torrente. 
\verse Post dies autem siccatus est torrens; non enim pluerat super terram.
 \verse Factus est igitur sermo Domini ad eum dicens: 
\verse “ Surge et vade in Sarepta Sidoniorum et manebis ibi; praecepi enim ibi mulieri viduae, ut pascat te ”. 
\verse Surrexit et abiit Sareptam. Cumque venisset ad portam civitatis, apparuit ei mulier vidua colligens ligna; et vocavit eam dixitque: “ Da mihi paululum aquae in vase, ut bibam ”. 
\verse Cumque illa pergeret, ut afferret, clamavit post tergum eius dicens: “ Affer mihi, obsecro, et buccellam panis in manu tua ”. 
\verse Quae respondit: “ Vivit Dominus Deus tuus, non habeo panem, nisi quantum pugillus capere potest farinae in hydria et paululum olei in lecytho. En colligo duo ligna, ut ingrediar et faciam illud mihi et filio meo, ut comedamus et moriamur ”.
 \verse Ad quam Elias ait: “ Noli timere, sed vade et fac, sicut dixisti; verumtamen mihi primum fac de ipsa farinula subcinericium panem parvulum et affer ad me; tibi autem et filio tuo facies postea. 
\verse Haec autem dicit Dominus, Deus Israel: “Hydria farinae non deficiet, nec lecythus olei minuetur usque ad diem, in qua daturus est Dominus pluviam super faciem terrae” ”. 
\verse Quae abiit et fecit iuxta verbum Eliae et comedit illa et ipse et domus eius per dies. 
\verse Hydria farinae non defecit, et lecythus olei non est imminutus iuxta verbum Domini, quod locutus fuerat in manu Eliae.
 \verse Factum est autem post haec, aegrotavit filius mulieris matris familiae; et erat languor fortis nimis, ita ut non remaneret in eo halitus. 
\verse Dixit ergo ad Eliam: “ Quid mihi et tibi, vir Dei? Ingressus es ad me, ut rememorarentur iniquitates meae, et interficeres filium meum? ”. 
\verse Et ait ad eam: “ Da mihi filium tuum ”. Tulitque eum de sinu illius et portavit in cenaculum, ubi ipse manebat, et posuit super lectulum suum; 
\verse clamavitque ad Dominum et dixit: “ Domine Deus meus, etiamne viduam, apud quam ego ut hospes habito, afflixisti, ut interficeres filium eius? ”. 
\verse Et expandit se atque mensus est super puerum tribus vicibus et clamavit ad Dominum et ait: “ Domine Deus meus, revertatur, oro, anima pueri huius in viscera eius ”. 
\verse Et exaudivit Dominus vocem Eliae, et reversa est anima pueri intra eum, et revixit. 
\verse Tulitque Elias puerum et deposuit eum de cenaculo in inferiorem domum et tradidit matri suae et ait illi: “ En vivit filius tuus ”. 
\verse Dixitque mulier ad Eliam: “ Nunc in isto cognovi quoniam vir Dei es tu, et verbum Domini in ore tuo verum est ”.
 
\begin{biblechapter}
\verse Post dies multos factum est verbum Domini ad Eliam in anno tertio dicens: “ Vade et ostende te Achab, ut dem pluviam super faciem terrae ”. 
\verse Ivit ergo Elias, ut ostenderet se Achab.
 Erat autem fames vehemens in Samaria. 
\verse Vocavitque Achab Abdiam dispensatorem domus suae. Abdias autem timebat Dominum valde; 
\verse nam, cum interficeret Iezabel prophetas Domini, tulit ille centum prophetas et abscondit eos quinquagenos et quinquagenos in speluncis et pavit eos pane et aqua. 
\verse Dixit ergo Achab ad Abdiam: “ Vade in terra ad universos fontes aquarum et in cunctas valles, si forte invenire possimus herbam, ut salvemus equos et mulos et nullum de iumentis interficere debeamus ”. 
\verse Diviseruntque sibi regiones, ut circuirent eas: Achab ibat per viam unam, et Abdias per viam alteram seorsum.
 \verse Cumque esset Abdias in via, Elias occurrit ei; qui cum cognovisset eum, cecidit super faciem suam et ait: “ Num tu es, domine mi, Elias? ”. 
\verse Cui ille respondit: “ Ego. Vade, dic domino tuo: “Adest Elias” ”. 
\verse Et ille: “ Quid peccavi, inquit, quoniam trades me servum tuum in manu Achab, ut interficiat me? 
\verse Vivit Dominus Deus tuus, non est gens aut regnum, quo non miserit dominus meus te requirens et, respondentibus cunctis: “Non est hic”, adiuravit regna singula et gentes, eo quod minime reperireris. 
\verse Et nunc dicis mihi: “Vade et dic domino tuo: Adest Elias”. 
\verse Cumque recessero a te, spiritus Domini asportabit te in locum, quem ego ignoro; et ingressus nuntiabo Achab, et non inveniet te et interficiet me. Servus autem tuus timet Dominum ab infantia sua. 
\verse Numquid non indicatum est domino meo quid fecerim, cum interficeret Iezabel prophetas Domini: quod absconderim de prophetis Domini centum viros, quinquagenos et quinquagenos in speluncis et paverim eos pane et aqua? 
\verse Et nunc tu dicis: “Vade et dic domino tuo: Adest Elias”, ut interficiat me ”. 
\verse Dixit Elias: “ Vivit Dominus exercituum ante cuius vultum sto: hodie apparebo ei ”.
 \verse Abiit ergo Abdias in occursum Achab et indicavit ei.
 Venitque Achab in occursum Eliae 
\verse et, cum vidisset eum, ait: “ Tune es, qui conturbas Israel? ”. 
\verse Et ille ait: “ Non turbavi Israel, sed tu et domus patris tui, qui dereliquistis mandata Domini, et secutus es Baalim. 
\verse Verumtamen nunc mitte et congrega ad me universum Israel in monte Carmeli et prophetas Baal quadringentos quinquaginta prophetasque Aserae quadringentos, qui comedunt de mensa Iezabel ”.
 \verse Misit Achab ad omnes filios Israel et congregavit prophetas in monte Carmeli. 
\verse Accedens autem Elias ad omnem populum ait: “ Usquequo claudicatis in duas partes? Si Dominus est Deus, sequimini eum; si autem Baal, sequimini illum ”. Et non respondit ei populus verbum. 
\verse Et ait rursus Elias ad populum: “ Ego remansi propheta Domini solus; prophetae autem Baal quadringenti et quinquaginta viri sunt. 
\verse Dentur nobis duo boves, et illi eligant sibi bovem unum et in frusta caedentes ponant super ligna; ignem autem non supponant. Et ego faciam bovem alterum et imponam super ligna; ignemque non supponam. 
\verse Invocate nomen dei vestri, et ego invocabo nomen Domini; et Deus, qui exaudierit per ignem, ipse est Deus! ”. Respondens omnis populus ait: “ Optima propositio ”.
 \verse Dixit ergo Elias prophetis Baal: “ Eligite vobis bovem unum et facite primi, quia vos plures estis; et invocate nomen dei vestri ignemque non supponatis ”. 
 \verse Qui cum tulissent bovem, quem dederat eis, fecerunt et invocabant nomen Baal de mane usque ad meridiem dicentes: “ Baal, exaudi nos! ”. Et non erat vox, nec qui responderet. Saliebantque in circuitu altaris, quod fecerant. 
\verse Cumque esset iam meridies, illudebat eis Elias dicens: “ Clamate voce maiore; deus enim est et forsitan occupatus est aut secessit aut in itinere aut certe dormit, ut excitetur ”. 
\verse Clamabant ergo voce magna et incidebant se iuxta ritum suum cultris et lanceolis, donec perfunderentur sanguine.
 \verse Postquam autem transiit meridies, et, illis prophetantibus, venerat tempus, quo sacrificium offerri solet, nec audiebatur vox, neque aliquis respondebat nec attendebat orantes, 
\verse dixit Elias omni populo: “ Venite ad me ”. Et, accedente ad se populo, curavit altare Domini, quod destructum fuerat; 
\verse et tulit duodecim lapides iuxta numerum tribuum filiorum Iacob, ad quem factus est sermo Domini dicens: “ Israel erit nomen tuum ”. 
\verse Et aedificavit lapidibus altare in nomine Domini fecitque aquaeductum quasi pro duobus satis in circuitu altaris 
\verse et composuit ligna divisitque per membra bovem et posuit super ligna 
\verse et ait: “ Implete quattuor hydrias aqua et fundite super holocaustum et super ligna ”. Rursumque dixit: “ Etiam secundo hoc facite ”. Qui cum fecissent et secundo, ait: “ Etiam tertio idipsum facite ”. Feceruntque et tertio, 
\verse et currebant aquae circum altare, et fossa aquaeductus repleta est.
 \verse Cumque iam tempus esset, ut offerretur sacrificium, accedens Elias propheta ait: “ Domine, Deus Abraham, Isaac et Israel, hodie ostende quia tu es Deus in Israel, et ego servus tuus et iuxta praeceptum tuum feci omnia haec. 
\verse Exaudi me, Domine, exaudi me, ut discat populus iste quia tu, Domine, es Deus et tu convertisti cor eorum iterum! ”.
 \verse Cecidit autem ignis Domini et voravit holocaustum et ligna et lapides, pulverem quoque et aquam, quae erat in aquaeductu lambens. 
\verse Quod cum vidisset omnis populus, cecidit in faciem suam et ait: “ Dominus ipse est Deus, Dominus ipse est Deus! ”. 
\verse Dixitque Elias ad eos: “ Apprehendite prophetas Baal, et ne unus quidem effugiat ex eis! ”. Quos cum comprehendissent, duxit eos Elias ad torrentem Cison et interfecit eos ibi.
 \verse Et ait Elias ad Achab: “ Ascende, comede et bibe, quia sonus multae pluviae est ”. 
\verse Ascendit Achab, ut comederet et biberet. Elias autem ascendit in verticem Carmeli et pronus in terram posuit faciem inter genua sua 
\verse et dixit ad puerum suum: “ Ascende et prospice contra mare ”. Qui, cum ascen disset et contemplatus esset, ait: “ Non est quidquam ”. Et rursum ait illi: “ Revertere septem vicibus ”. 
\verse In septima autem vice dixit: “ Ecce nubecula parva quasi manus hominis ascendit de mari ”. Et ait: “ Ascende et dic Achab: Iunge et descende, ne occupet te pluvia! ”. 
\verse Et factum est interea: ecce caeli contenebrati sunt, et nubes et ventus, et facta est pluvia grandis. Ascendens itaque Achab abiit in Iezrahel. 
\verse Et manus Domini facta est super Eliam; accinctisque lumbis, currebat ante Achab, donec veniret in Iezrahel.
 
\begin{biblechapter}
\verse Nuntiavit autem Achab Iezabel omnia, quae fecerat Elias, et quomodo occidisset universos prophetas gladio. 
\verse Misitque Iezabel nuntium ad Eliam dicens: “ Haec mihi faciant dii et haec addant, nisi hac hora cras posuero animam tuam sicut animam unius ex illis ”. 
\verse Timuit ergo Elias et surgens abiit, ut animam suam salvaret, venitque in Bersabee Iudae et dimisit ibi puerum suum. 
\verse Et perrexit in desertum via unius diei; cumque venisset et sederet subter unam iuniperum, petivit animae suae, ut moreretur, et ait: “ Sufficit mihi, Domine! Tolle animam meam; neque enim melior sum quam patres mei ”. 
\verse Proiecitque se et obdormivit in umbra iuniperi; et ecce angelus tetigit eum et dixit illi: “ Surge, comede! ”. 
\verse Respexit, et ecce ad caput suum subcinericius panis et vas aquae; comedit ergo et bibit et rursum obdormivit. 
 \verse Reversusque est angelus Domini secundo et tetigit eum dixitque illi: “ Surge, comede! Grandis enim tibi restat via ”. 
\verse Qui, cum surrexisset, comedit et bibit et ambulavit in fortitudine cibi illius quadraginta diebus et quadraginta noctibus usque ad montem Dei Horeb.
 \verse Cumque venisset illuc, mansit in spelunca. Et ecce sermo Domini ad eum dixitque illi: “ Quid hic agis, Elia? ”. 
\verse At ille respondit: “ Zelo zelatus sum pro Domino, Deo exercituum, quia dereliquerunt pactum tuum filii Israel, altaria tua destruxerunt et prophetas tuos occiderunt gladio; et derelictus sum ego solus, et quaerunt animam meam, ut auferant eam ”. 
\verse Et ait ei: “ Egredere et sta in monte coram Domino ”. Et ecce Dominus transit, et ventus grandis et fortis subvertens montes et conterens petras ante Dominum; non in vento Dominus. Et post ventum, commotio; non in commotione Dominus. 
\verse Et post commotionem, ignis; non in igne Dominus. Et post ignem, sibilus aurae tenuis. 
\verse Quod cum audisset Elias, operuit vultum suum pallio et egressus stetit in ostio speluncae; et ecce vox ad eum dicens: “ Quid agis hic, Elia? ”. 
 \verse Et ille respondit: “ Zelo zelatus sum pro Domino, Deo exercituum, quia dereliquerunt pactum tuum filii Israel, altaria tua destruxerunt et prophetas tuos occiderunt gladio; et derelictus sum ego solus, et quaerunt animam meam, ut auferant eam ”.
 \verse Et ait Dominus ad eum: “ Vade et revertere in viam tuam per desertum in Damascum. Cumque perveneris, unges Hazael regem super Syriam; 
\verse et Iehu filium Namsi unges regem super Israel; Eliseum autem filium Saphat, qui est de Abelmehula, unges prophetam pro te. 
\verse Et erit: quicumque fugerit gladium Hazael, occidet eum Iehu; et, qui fugerit gladium Iehu, interficiet eum Eliseus. 
\verse Et relinquam mihi in Israel septem milia: universorum genua, quae non sunt incurvata ante Baal, et omne os, quod non osculatum est eum ”.
 \verse Profectus ergo inde repperit Eliseum filium Saphat arantem duodecim iugis boum; et ipse cum duodecimo erat. Cumque venisset Elias ad eum, misit pallium suum super illum, 
\verse qui statim, relictis bobus, cucurrit post Eliam et ait: “ Osculer, oro, patrem meum et matrem meam, et sic sequar te ”. Dixitque ei: “ Vade et revertere; quid enim feci tibi? ”.
 \verse Reversus autem ab eo tulit par boum et mactavit illud et in iugo boum coxit carnes et dedit populo, et comederunt. Consurgensque abiit et secutus est Eliam et ministrabat ei.
 
\begin{biblechapter}
\verse Porro Benadad rex Syriae congregavit omnem exercitum suum et triginta duos reges secum et equos et currus et ascendens pugnabat contra Samariam et obsidebat eam. 
\verse Mittensque nuntios ad Achab regem Israel in civitatem 
\verse ait: “ Haec dicit Benadad: Argentum tuum et aurum tuum meum est, et uxores tuae et filii tui optimi mei sunt ”. 
\verse Responditque rex Israel: “ Iuxta verbum tuum, domine mi rex; tuus sum ego et omnia mea ”. 
\verse Revertentesque nuntii dixerunt: “ Haec dicit Benadad: Quia misi ad te dicens: “Argentum tuum et aurum tuum et uxores tuas et filios tuos dabis mihi”, 
\verse profecto cras hac eadem hora mittam servos meos ad te, et scrutabuntur domum tuam et domum servorum tuorum; et omne, quod oculis tuis pretiosum est, ponent in manibus suis et auferent ”.
 \verse Vocavit autem rex Israel omnes seniores terrae et ait: “ Animadvertite et videte quoniam insidietur nobis; misit enim ad me pro uxoribus meis et filiis et pro argento et auro, et non abnui ”. 
\verse Dixeruntque omnes maiores natu et universus populus ad eum: “ Non audias neque acquiescas illi ”. 
\verse Respondit itaque nuntiis Benadad: “ Dicite domino meo regi: Omnia, propter quae misisti ad me servum tuum initio, faciam; hanc autem rem facere non possum ”. Reversique nuntii rettulerunt ei. 
\verse Qui remisit et ait: “ Haec faciant mihi dii et haec addant, si suffecerit pulvis Samariae pugillis omnis populi, qui sequitur me ”. 
\verse Et respondens rex Israel ait: “ Dicite ei: Ne glorietur accinctus aeque ut discinctus ”. 
\verse Factum est autem, cum audisset verbum istud, bibebat ipse et reges in umbraculis et ait servis suis: “ Circumdate civitatem! ”. Et circumdederunt eam.
 \verse Et ecce propheta unus accedens ad Achab regem Israel ait: “ Haec dicit Dominus: Certe vidisti omnem multitudinem hanc nimiam. Ecce ego tradam eam in manu tua hodie, ut scias quia ego sum Dominus ”. 
\verse Et ait Achab: “ Per quem? ”. Dixitque ei: “ Haec dicit Dominus: Per pedisequos principum provinciarum ”. Et ait: “ Quis incipiet proeliari? ”. Et ille dixit: “ Tu ”.
 \verse Recensuit ergo pueros principum provinciarum et repperit numerum ducentorum triginta duorum; et post eos recensuit populum, omnes filios Israel, septem milia. 
\verse Et egressi sunt meridie. Benadad autem bibebat temulentus in umbraculis ipse et reges triginta duo cum eo, qui ad auxilium eius venerant. 
 \verse Egressi sunt autem pueri principum provinciarum in prima fronte. Misit itaque Benadad, qui nuntiaverunt ei dicentes: “ Viri egressi sunt de Samaria ”. 
\verse At ille ait: “ Sive pro pace veniunt, apprehendite eos vivos; sive ut proelientur, vivos eos capite ”. 
\verse Egressi erant ergo ex urbe pueri principum provinciarum, ac reliquus exercitus sequebatur, 
\verse et percussit unusquisque virum, qui contra se venerat; fugeruntque Syri, et persecutus est eos Israel. Fugit quoque Benadad rex Syriae in equo cum equitibus. 
\verse Necnon egressus rex Israel percussit equos et currus et percussit Syriam plaga magna.
 \verse Accedens autem propheta ad regem Israel dixit ei: “ Vade et confortare et scito et vide quid facias; vertente enim anno rex Syriae ascendet contra te ”.
 \verse Servi vero regis Syriae dixerunt ei: “ Deus montium est Deus eorum, ideo superaverunt nos; sed pugnemus contra eos in campestribus et obtinebimus eos. 
 \verse Fac ergo hoc: Amove reges singulos a loco suo et pone principes pro eis; 
 \verse et instaura numerum militum, qui ceciderunt de tuis, et equos secundum equos pristinos et currus secundum currus, quos ante habuisti, et pugnabimus contra eos in campestribus: et videbis quod obtinebimus eos ”. Credidit consilio eorum et fecit ita.
 \verse Igitur vertente anno recensuit Benadad Syros et ascendit in Aphec, ut pugnaret contra Israel. 
\verse Porro filii Israel recensiti sunt et, acceptis cibariis, profecti ex adverso castraque metati sunt contra eos, quasi duo parvi greges caprarum; Syri autem repleverunt terram.
 \verse Et accedens vir Dei dixit ad regem Israel: “ Haec dicit Dominus: Quia dixerunt Syri: “Deus montium est Dominus et non est Deus vallium”, dabo omnem multitudinem hanc grandem in manu tua, et scietis quia ego Dominus ”. 
\verse Dirigebant septem diebus ex adverso hi atque illi acies, septima autem die commissum est bellum; percusseruntque filii Israel de Syris centum milia peditum in die una. 
\verse Fugerunt autem, qui remanserant in Aphec, in civitatem, et cecidit murus super viginti septem milia hominum, qui remanserant.
 Porro Benadad fugiens ingressus est civitatem in cubiculum, quod erat intra cubiculum. 
\verse Dixeruntque ei servi sui: “ Ecce audivimus quod reges domus Israel clementes sint; ponamus itaque saccos in lumbis nostris et funiculos in capitibus nostris et egrediamur ad regem Israel; forsitan salvabit animam tuam ”. 
\verse Accinxerunt saccis lumbos suos et posuerunt funes in capitibus suis veneruntque ad regem Israel et dixerunt: “ Servus tuus Benadad dicit: “Vivat, oro te, anima mea” ”. Et ille ait: “ Si adhuc vivit, frater meus est ”. 
\verse Quod acceperunt viri pro omine et festinantes rapuerunt verbum ex ore eius atque dixerunt: “ Frater tuus Benadad ”. Et dixit eis: “ Ite et adducite eum ”. Egressus est ergo ad eum Benadad, et levavit eum in currum suum. 
\verse Qui dixit ei: “ Civitates, quas tulit pater meus a patre tuo, reddam; et plateas fac tibi in Damasco, sicut fecit pater meus in Samaria ”. Achab: “ Ego autem, inquit, foederatum te dimittam ”. Et pepigit ei foedus et dimisit eum.
 \verse Tunc vir quidam de filiis prophetarum dixit ad socium suum in sermone Domini: “ Percute me! ”. At ille noluit percutere. 
\verse Cui ait: “ Quia noluisti audire vocem Domini, ecce recedes a me, et percutiet te leo ”. Cumque paululum recessisset ab eo, invenit eum leo atque percussit. 
\verse Sed et alterum inveniens virum dixit ad eum: “ Percute me! ”. Qui percussit eum et vulneravit. 
 \verse Abiit ergo propheta et occurrit regi in via et mutavit aspectum ponens fasciam super oculos suos. 
\verse Cumque rex transiret, clamavit ad regem et ait: “ Servus tuus egressus est ad proeliandum comminus; cumque fugisset vir unus, adduxit eum quidam ad me et ait: “Custodi virum istum! Qui si lapsus fuerit, erit anima tua pro anima eius, aut talentum argenti appendes”. 
\verse Dum autem ego turbatus huc illucque me verterem, subito non comparuit ”. Et ait rex Israel ad eum: “ Hoc est iudicium tuum, quod ipse decrevisti ”. 
\verse At ille statim abstulit fasciam de oculis suis, et cognovit eum rex Israel quod esset de prophetis. 
\verse Qui ait ad eum: “ Haec dicit Dominus: Quia dimisisti de manu tua virum, quem morti devoveram, erit anima tua pro anima eius, et populus tuus pro populo eius ”. 
\verse Reversus est igitur rex Israel in domum suam tristis et indignans venitque in Samariam.
 
\begin{biblechapter}
\verse Postea autem factum est hoc. Vinea erat Naboth Iez rahelitae, quae erat in Iezrahel iuxta palatium Achab regis Samariae. 
\verse Locutus est ergo Achab ad Naboth dicens: “ Da mihi vineam tuam, ut faciam mihi hortum holerum, quia vicina est et prope domum meam. Daboque tibi pro ea vineam meliorem aut, si tibi commodius putas, argenti pretium quanto digna est ”. 
\verse Cui respondit Naboth: “ Propitius mihi sit Dominus, ne dem hereditatem patrum meorum tibi ”.
 \verse Venit ergo Achab in domum suam tristis et indignans super verbo, quod locutus fuerat ad eum Naboth Iezrahelites dicens: “ Non dabo tibi hereditatem patrum meorum ”. Et proiciens se in lectulum suum avertit faciem ad parietem et non comedit panem. 
\verse Ingressa est autem ad eum Iezabel uxor sua dixitque ei: “ Quid est hoc, unde anima tua contristata est? Et quare non comedis panem? ”. 
 \verse Qui respondit ei: “ Quia locutus sum Naboth Iezrahelitae et dixi ei: Da mihi vineam tuam, accepta pecunia; aut, si tibi placet, dabo tibi vineam pro ea. Et ille ait: “Non dabo tibi vineam meam” ”. 
\verse Dixit ergo ad eum Iezabel uxor eius: “ Grandis auctoritatis es et bene regis regnum Israel! Surge et comede panem et aequo esto animo; ego dabo tibi vineam Naboth Iezrahelitae ”.
 \verse Scripsit itaque litteras ex nomine Achab et signavit eas anulo eius et misit ad maiores natu et ad optimates, qui erant in civitate eius et habitabant cum Naboth. 
\verse Litterarum autem haec erat sententia: “ Praedicate ieiunium et sedere facite Naboth in capite populi 
\verse et submittite duos viros filios Belial contra eum, et testimonium dicant: “Maledixisti Deum et regem”; et educite eum et lapidate, sicque moriatur ”. 
\verse Fecerunt ergo cives eius maiores natu et optimates, qui habitabant cum eo in urbe, sicut praeceperat eis Iezabel et sicut scriptum erat in litteris, quas miserat ad eos. 
\verse Praedicaverunt ieiunium et sedere fecerunt Naboth in capite populi; 
\verse et ingressi duo viri filii Belial sederunt contra eum et illi, ut viri diabolici, dixerunt contra eum testimonium coram multitudine: “ Maledixit Naboth Deum et regem ”. Quam ob rem eduxerunt eum extra civitatem et lapidibus interfecerunt; 
 \verse miseruntque ad Iezabel dicentes: “ Lapidatus est Naboth et mortuus est ”.
 \verse Factum est autem cum audisset Iezabel lapidatum Naboth et mortuum, locuta est ad Achab: “ Surge, posside vineam Naboth Iezrahelitae, qui noluit tibi acquiescere et dare eam, accepta pecunia; non enim vivit Naboth, sed mortuus est ”. 
\verse Quod cum audisset Achab, mortuum videlicet Naboth, surrexit et descendebat in vineam Naboth Iezrahelitae, ut possideret eam.
 \verse Factus est igitur sermo Domini ad Eliam Thesbiten dicens: 
\verse “ Surge et descende in occursum Achab regis Israel, qui est in Samaria; ecce est in vinea Naboth, ad quam descendit, ut possideat eam. 
\verse Et loqueris ad eum dicens: Haec dicit Dominus: Occidisti, insuper et possedisti! Et post haec addes: Haec dicit Dominus: In loco, in quo linxerunt canes sanguinem Naboth, lambent tuum quoque sanguinem ”. 
\verse Et ait Achab ad Eliam: “ Num invenisti me, inimice mi? ”. Qui dixit: “ Inveni, eo quod venumdatus sis, ut faceres malum in conspectu Domini. 
\verse Ecce ego inducam super te malum et demetam posteriora tua et interficiam de Achab quidquid masculini sexus sive impuberem sive puberem in Israel. 
\verse Et dabo domum tuam sicut domum Ieroboam filii Nabat et sicut domum Baasa filii Ahia, quia egisti, ut me ad iracundiam provocares, et peccare fecisti Israel. 
\verse Sed et de Iezabel locutus est Dominus dicens: Canes comedent Iezabel in agro Iezrahel. 
\verse Qui de Achab mortuus fuerit in civitate, comedent eum canes; qui autem mortuus fuerit in agro, comedent eum volucres caeli ”.
 \verse Igitur non fuit alter talis sicut Achab, qui venumdatus est, ut faceret malum in conspectu Domini; concitavit enim eum Iezabel uxor sua, 
\verse et abominabilis effectus est, in tantum ut sequeretur idola secundum omnia, quae fecerant Amorraei, quos consumpsit Dominus a facie filiorum Israel.
 \verse Itaque cum audisset Achab sermones istos, scidit vestem suam et operuit cilicio carnem suam ieiunavitque et dormivit in sacco et ambulabat demisso capite. 
\verse Factus est autem sermo Domini ad Eliam Thesbiten dicens: 
\verse Nonne vidisti humiliatum Achab coram me? Quia igitur humiliatus est mei causa, non inducam malum in diebus eius, sed in diebus filii sui inferam malum domui eius ”.
 
\begin{biblechapter}
\verse Transierunt igitur tres anni absque bello inter Syriam et Israel. 
\verse In anno autem tertio descendit Iosaphat rex ludae ad regem Israel, 
\verse dixitque rex Israel ad servos suos: “ Ignoratis quod nostra sit Ramoth Galaad et neglegimus tollere eam de manu regis Syriae? ”. 
\verse Et ait ad Iosaphat: “ Veniesne mecum ad proeliandum in Ramoth Galaad? ”.
 \verse Dixitque Iosaphat ad regem Israel: “ Sicut ego sum, ita et tu; populus meus et populus tuus unum sunt, et equites mei et equites tui ”. Dixitque Iosaphat ad regem Israel: “ Quaere, oro te, hodie sermonem Domini ”. 
\verse Congregavit ergo rex Israel prophetas quadringentos circiter viros et ait ad eos: “ Ire debeo in Ramoth Galaad ad bellandum, an quiescere? ”. Qui responderunt: “ Ascende, et dabit Dominus in manu regis ”. 
\verse Dixit autem Iosaphat: “ Non est hic et alius propheta Domini, ut interrogemus per eum? ”. 
\verse Et ait rex Israel ad Iosaphat: “ Remansit vir unus, per quem possimus interrogare Dominum; sed ego odi eum, quia non prophetat mihi bonum sed malum: Michaeas filius Iemla ”. Cui Iosaphat ait: “ Ne loquaris ita, rex ”. 
\verse Vocavit ergo rex Israel eunuchum quendam et dixit ei: “ Festina adducere Michaeam filium Iemla ”.
 \verse Rex autem Israel et Iosaphat rex Iudae sedebat unusquisque in solio suo vestiti cultu regio in area iuxta ostium portae Samariae; et universi prophetae prophetabant in conspectu eorum. 
\verse Fecit quoque sibi Sedecias filius Chanaana cornua ferrea et ait: “ Haec dicit Dominus: His ventilabis Syriam, donec deleas eam ”. 
\verse Omnesque prophetae similiter prophetabant dicentes: “ Ascende in Ramoth Galaad et vade prospere, et tradet Dominus in manu regis ”.
 \verse Nuntius vero, qui ierat ut vocaret Michaeam, locutus est ad eum dicens: “ Ecce sermones prophetarum ore uno regi bona praedicant; sit ergo sermo tuus similis eorum, et loquere bona ”. 
\verse Cui Michaeas ait: “ Vivit Dominus quia, quodcumque dixerit mihi Dominus, hoc loquar! ”.
 \verse Venit itaque ad regem, et ait illi rex: “ Michaea, ire debemus in Ramoth Galaad ad proeliandum, an cessare? ”. Cui ille respondit: “ Ascende et vade prospere, et tradet Dominus in manu regis ”. 
\verse Dixit autem rex ad eum: “ Iterum atque iterum adiuro te, ut non loquaris mihi, nisi quod verum est in nomine Domini ”. 
\verse Et ille ait:
 “ Vidi cunctum Israel
 dispersum in montibus
 quasi oves non habentes pastorem. Et ait Dominus: “Non habent isti dominum; revertatur unusquisque in domum suam in pace” ”.
 \verse Dixit ergo rex Israel ad Iosaphat: “ Numquid non dixi tibi quia non prophetat mihi bonum sed semper malum? ”.
 \verse Ille vero addens ait: “ Propterea audi sermonem Domini: Vidi Dominum sedentem super solium suum et omnem exercitum caeli assistentem ei a dextris et a sinistris. 
\verse Et ait Dominus: “Quis decipiet Achab, ut ascendat et cadat in Ramoth Galaad?”. Et dixit unus verba huiuscemodi et alius aliter. 
\verse Egressus est autem spiritus et stetit coram Domino et ait: “Ego decipiam illum”. Cui locutus est Dominus: “In quo?”. 
\verse Et ille ait: “Egrediar et ero spiritus mendax in ore omnium prophetarum eius”. Et dixit Dominus: “Decipies et praevalebis; egredere et fac ita”. 
\verse Nunc igitur ecce dedit Dominus spiritum mendacii in ore omnium prophetarum tuorum, qui hic sunt, et Dominus locutus est contra te malum ”.
 \verse Accessit autem Sedecias filius Chanaana et percussit Michaeam in maxillam et dixit: “ Quomodo transivit spiritus Domini a me, ut loqueretur tibi? ”. 
\verse Et ait Michaeas: “ Visurus es in die illa, quando ingredieris cubiculum intra cubiculum, ut abscondaris ”. 
\verse Et ait rex Israel: “ Tolle Michaeam, et maneat apud Amon principem civitatis et apud Ioas filium regis, 
\verse et dic eis: “Haec dicit rex: Mittite virum istum in carcerem et sustentate eum pane tribulationis et aqua angustiae, donec revertar in pace” ”. 
\verse Dixitque Michaeas: “ Si reversus fueris in pace, non est locutus Dominus in me ”. Et ait: “ Audite, populi omnes! ”.
 \verse Ascendit itaque rex Israel et Iosaphat rex Iudae in Ramoth Galaad. 
\verse Dixitque rex Israel ad Iosaphat: “ Mutato aspectu ineundum est proelium; tu autem induere vestibus tuis ”. Porro rex Israel mutavit aspectum et ingressus est bellum. 
\verse Rex autem Syriae praeceperat principibus curruum triginta duobus dicens: “ Non pugnabitis contra minorem et maiorem quempiam, nisi contra regem Israel solum ”. 
\verse Cum ergo vidissent principes curruum Iosaphat, suspicati sunt quod ipse esset rex Israel et impetu facto pugnabant contra eum. Et exclamavit Iosaphat; 
\verse intellexeruntque principes curruum quod non esset rex Israel et cessaverunt ab eo.
 \verse Vir autem quidam tetendit arcum in incertum sagittam dirigens et percussit regem Israel inter iuncturas et loricam. At ille dixit aurigae suo: “ Verte manum tuam et eice me de exercitu, quia graviter vulneratus sum ”. 
\verse Aggravatum est ergo proelium in die illa; et rex Israel stabat in curru suo contra Syros et mortuus est vespere: fluebat autem sanguis plagae in sinum currus. 
\verse Et clamor insonuit in universo exercitu ad solis occasum: “ Unusquisque revertatur in civitatem et in terram suam! ”.
 \verse Mortuus est igitur rex et perlatus est Samariam; sepelieruntque regem in Samaria. 
\verse Et laverunt currum eius in piscina Samariae; et linxerunt canes sanguinem eius, et scorta laverunt se iuxta verbum Domini, quod locutus fuerat.
 \verse Reliqua vero gestorum Achab et universa, quae fecit, et domus eburnea, quam aedificavit, cunctaeque urbes, quas exstruxit, nonne haec scripta sunt in libro annalium regum Israel? 
\verse Dormivit ergo Achab cum patribus suis; et regnavit Ochozias filius eius pro eo.
 \verse Iosaphat vero filius Asa regnare coeperat super Iudam anno quarto Achab regis Israel; 
\verse triginta quinque annorum erat, cum regnare coepisset, et viginti quinque annos regnavit in Ierusalem. Nomen matris eius Azuba filia Selachi. 
 \verse Et ambulavit in omni via Asa patris sui et non declinavit ex ea; fecitque, quod rectum erat in conspectu Domini. 
\verse Verumtamen excelsa non abstulit; adhuc enim populus sacrificabat et adolebat in excelsis. 
\verse Pacemque fecit Iosaphat cum rege Israel.
 \verse Reliqua autem gestorum Iosaphat et opera eius, quae fortiter gessit, et proelia, nonne haec scripta sunt in libro annalium regum Iudae?
 \verse Sed et reliquias prostibulorum, qui remanserant in diebus Asa patris eius, abstulit de terra. 
\verse Nec erat tunc rex in Edom sed praefectus regius. 
 \verse Rex vero Iosaphat fecerat naves Tharsis, quae navigarent in Ophir propter aurum; et ire non potuerunt, quia confractae sunt in Asiongaber. 
\verse Tunc ait Ochozias filius Achab ad Iosaphat: “ Vadant servi mei cum servis tuis in navibus ”. Et noluit Iosaphat.
 \verse Dormivitque Iosaphat cum patribus suis et sepultus est cum eis in civitate David patris sui; regnavitque Ioram filius eius pro eo.
 \verse Ochozias autem filius Achab regnare coeperat super Israel in Samaria anno septimo decimo Iosaphat regis Iudae regnavitque super Israel duobus annis. 
 \verse Et fecit malum in conspectu Domini et ambulavit in via patris sui et matris suae et in via Ieroboam filii Nabat, qui peccare fecit Israel; 
\verse servivit quoque Baal et adoravit eum et irritavit Dominum, Deum Israel, iuxta omnia, quae fecerat pater eius.
\end{biblechapter}
