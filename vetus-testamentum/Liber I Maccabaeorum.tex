\biblebook{Liber I Maccabaeorum}

\begin{biblechapter}   
\verse Et factum est, postquam percussit Alexander Philippi Macedo, qui prius regnavit in Graecia, egressus de terra Cetthim, Darium regem Persarum et Medorum, 
\verse constituit proelia multa et obtinuit munitiones et interfecit reges terrae; 
\verse et pertransiit usque ad fines terrae et accepit spolia multitudinis gentium, et siluit terra in conspectu eius, et exaltatum est et elevatum est cor eius. 
\verse Et congregavit virtutem fortem nimis et obtinuit regiones gentium et tyrannos, et facti sunt illi in tributum. 
\verse Et post haec decidit in lectum et cognovit quia moreretur; 
\verse et vocavit pueros suos nobiles, qui secum erant nutriti a iuventute, et divisit illis regnum suum, cum adhuc viveret. 
\verse Et regnavit Alexander annis duodecim et mortuus est. 
\verse Et obtinuerunt pueri eius regnum, unusquisque in loco suo; 
\verse et imposuerunt omnes sibi diademata post mortem eius, et filii eorum post eos annis multis. Et multiplicata sunt mala in terra. 
\verse Et exiit ex eis radix peccatrix, Antiochus Epiphanes filius Antiochi regis, qui fuerat Romae obses, et regnavit in anno centesimo tricesimo septimo regni Graecorum. 
\verse In diebus illis exierunt ex Israel filii iniqui et suaserunt multis dicentes: “Eamus et disponamus testamentum cum gentibus, quae circa nos sunt, quia, ex quo recessimus ab eis, invenerunt nos multa mala". 
\verse Et bonus visus est sermo in oculis eorum; 
\verse et destinaverunt aliqui de populo et abierunt ad regem, et dedit illis potestatem, ut facerent iustitias gentium.  
\verse Et aedificaverunt gymnasium in Hierosolymis secundum leges nationum; 
\verse et fecerunt sibi praeputia et recesserunt a testamento sancto et iuncti sunt nationibus et venumdati sunt, ut facerent malum. 
\verse Et paratum est regnum in conspectu Antiochi, et coepit regnare in terra Aegypti, ut regnaret super duo regna. 
\verse Et intravit in Aegyptum in multitudine gravi, in curribus et elephantis et equitibus et navium multitudine;  
\verse et constituit bellum adversus Ptolemaeum regem Aegypti, et veritus est Ptolemaeus a facie eius et fugit, et ceciderunt vulnerati multi. 
\verse Et comprehenderunt civitates munitas in terra Aegypti, et accepit spolia terrae Aegypti. 
\verse Et reversus est Antiochus, postquam percussit Aegyptum in centesimo et quadragesimo tertio anno, et ascendit ad Israel et ad Hierosolyma in multitudine gravi 
\verse et intravit in sanctificationem cum superbia et accepit altare aureum et candelabrum luminis et universa vasa eius 
\verse et mensam propositionis et libatoria et phialas et pateras aureas et velum et coronas et ornamentum aureum, quod in facie templi erat; et comminuit omnia.  
\verse Et accepit argentum et aurum et vasa concupiscibilia et accepit thesauros occultos, quos invenit; 
\verse et, sublatis omnibus, abiit in terram suam et fecit caedem hominum et locutus est superbia magna. 
\verse Et factus est planctus magnus in Israel et in omni loco eorum; 
\verse et ingemuerunt principes et seniores, virgines et iuvenes infirmati sunt, et speciositas mulierum immutata est, 
\verse omnis maritus sumpsit lamentum, et, quae sedebat in toro maritali, lugebat; 
\verse et commota est terra super habitantes in ea, et universa domus Iacob induit confusionem. 
\verse Et post duos annos dierum misit rex principem tributorum in civitates Iudae et venit Ierusalem cum turba magna; 
\verse et locutus est ad eos verba pacifica in dolo, et crediderunt ei. Et irruit super civitatem repente et percussit eam plaga magna et perdidit populum multum ex Israel. 
\verse Et accepit spolia civitatis et succendit eam igne et destruxit domos eius et muros eius in circuitu; 
\verse et captivas duxerunt mulieres et natos et pecora possederunt.  
\verse Et aedificaverunt civitatem David muro magno et firmo et turribus firmis; et facta est illis in arcem. 
\verse Et posuerunt illic gentem peccatricem, viros iniquos, et convaluerunt in ea; 
\verse et posuerunt arma et escas et, congregatis spoliis Ierusalem, reposuerunt illic; et facti sunt in laqueum magnum. 
\verse Et factum est hoc ad insidias sanctificationi et in diabolum malum in Israel semper; 
\verse et effuderunt sanguinem innocentem per circuitum sanctificationis et contaminaverunt sanctificationem. 
\verse Et fugerunt habitatores Ierusalem propter eos, et facta est habitatio exterorum, et facta est extera semini suo; et nati eius reliquerunt eam. 
\verse Sanctificatio eius desolata est sicut solitudo, dies festi eius conversi sunt in luctum, sabbata eius in opprobrium, honor eius in nihilum. 
\verse Secundum gloriam eius multiplicata est ignominia eius, et sublimitas eius conversa est in luctum. 
\verse Et scripsit rex Antiochus omni regno suo, ut essent universi populus unus,  
\verse et relinqueret unusquisque legem suam. Et receperunt omnes gentes secundum verbum regis Antiochi; 
\verse et multi ex Israel consenserunt cultui eius et sacrificaverunt idolis et coinquinaverunt sabbatum. 
\verse Et misit rex libros per manus nuntiorum in Ierusalem et in civitates Iudae, ut sequerentur leges gentium terrae, 
\verse et prohibere holocausta et sacrificia et placationes fieri in templo Dei et contaminare sabbata et dies sollemnes 
\verse et polluere sancta et sanctos, 
\verse instruere aras et templa et idola et immolare porcina et pecora communia  
\verse et relinquere filios suos incircumcisos et polluere animas eorum in omni immundo et abominatione, 
\verse ita ut obliviscerentur legem et immutarent omnes iustificationes; 
\verse et, quicumque non fecerit secundum verbum regis Antiochi, morietur. 
\verse Secundum omnia verba haec scripsit omni regno suo et praeposuit consideratores super omnem populum et mandavit civitatibus Iudae immolare per civitatem et civitatem. 
\verse Et congregati sunt multi de populo ad eos, omnes, qui dereliquerant legem Domini, et fecerunt mala in terra; 
\verse et posuerunt Israel in abditis et in absconditis fugitivorum locis. 
\verse Die quinta decima mensis Casleu, quinto et quadragesimo et centesimo anno, aedificavit abominationem desolationis super altare; et per civitates Iudae in circuitu aedificaverunt aras 
\verse et ante ianuas domorum et in plateis sacrificabant. 
\verse Et libros legis, quos invenerunt, combusserunt igne scindentes eos; 
\verse et ubicumque inveniebatur apud aliquem liber testamenti, et si quis consentiebat legi, constitutio regis interficiebat eum. 
\verse In virtute sua faciebant haec Israeli, omnibus, qui inveniebantur in omni mense et mense in civitatibus. 
\verse Et quinta et vicesima die mensis sacrificabant super aram, quae erat super altare; 
\verse et mulieres, quae circumciderant filios suos, interficiebant secundum iussum 
\verse — et suspendebant infantes a cervicibus eorum — et domos eorum et eos, qui circumciderant illos. 
\verse Et multi in Israel obtinuerunt et confortati sunt apud se, ut non manducarent immunda, 
\verse et elegerunt mori, ut non polluerentur escis et non profanarent testamentum sanctum, et moriebantur. 
\verse Et facta est ira magna super Israel valde. 
\end{biblechapter}

\begin{biblechapter}  
\verse In diebus illis surrexit Matthathias filius Ioannis filii Simeonis sacerdos ex filiis Ioarib ab Ierusalem et consedit in Modin. 
\verse Et habebat filios quinque: Ioannem, qui cognominabatur Gaddis, 
\verse et Simonem, qui vocabatur Thasi, 
\verse et Iudam, qui vocabatur Maccabaeus, 
\verse et Eleazarum, qui vocabatur Abaran, et Ionathan, qui vocabatur Apphus. 
\verse Et vidit blasphemias, quae fiebant in Iuda et in Ierusalem, 
\verse et dixit: “Vae mihi! Ut quid natus sum videre contritionem populi mei et contritionem civitatis sanctae? Et sederunt illic, cum daretur ea in manibus inimicorum, sanctificatio in manu extraneorum. 
\verse Factum est templum eius sicut homo ignobilis, 
\verse vasa gloriae eius captiva abducta sunt, trucidati sunt parvuli eius in plateis eius, iuvenes eius in gladio inimicorum. 
\verse Quae gens non hereditavit regnum eius et non obtinuit spolia eius? 
\verse Omnis ornatus eius ablatus est; quae erat libera, facta est ancilla. 
\verse Et ecce sancta nostra et pulchritudo nostra et gloria nostra desolata est, et polluerunt ea gentes.  
\verse Ut quid nobis adhuc vita?". 
\verse Et scidit vestimenta sua Matthathias et filii eius et operuerunt se ciliciis et planxerunt valde. 
\verse Et venerunt, qui ex rege compellebant discessionem, in civitatem Modin, ut sacrificarent. 
\verse Et multi de Israel accesserunt ad eos, et Matthathias et filii eius congregati sunt. 
\verse Et responderunt, qui missi erant a rege, et dixerunt Matthathiae: “Princeps et nobilis et magnus es in hac civitate et confirmatus filiis et fratribus. 
\verse Nunc accede primus et fac iussum regis, sicut fecerunt omnes gentes et viri Iudae et qui remanserunt in Ierusalem, et eris tu et filii tui inter amicos regis et tu et filii tui glorificabimini et argento et auro et muneribus multis". 
\verse Et respondit Matthathias et dixit magna voce: “Etsi omnes gentes, quae in domo regni sunt, regi oboediunt, ut discedat unusquisque ab officio patrum suorum, et consentiunt mandatis eius,  
\verse et ego et filii mei et fratres mei ibimus in testamento patrum nostrorum.  
\verse Propitius sit nobis Dominus, ne derelinquamus legem et iustificationes.  
\verse Non audiemus verba regis, ut praetereamus officium nostrum dextra vel sinistra". 
\verse Et, ut cessavit loqui verba haec, accessit quidam Iudaeus in omnium oculis sacrificare super aram in Modin secundum iussum regis. 
\verse Et vidit Matthathias et zelatus est, et contremuerunt renes eius; et attulit iram secundum iudicium et insiliens trucidavit eum super aram. 
\verse Et virum regis, qui cogebat immolare, occidit in ipso tempore et aram destruxit; 
\verse et zelatus est legem, sicut fecit Phinees Zambri filio Salom. 
\verse Et exclamavit Matthathias voce magna in civitate dicens: “Omnis, qui zelum habet legis statuens testamentum, exeat post me". 
\verse Et fugit ipse et filii eius in montes, et reliquerunt quaecumque habebant in civitate. 
\verse Tunc descenderunt multi quaerentes iustitiam et iudicium in desertum, ut sederent ibi, 
\verse ipsi et filii eorum et mulieres eorum et pecora eorum, quoniam induraverant super eos mala. 
\verse Et renuntiatum est viris regis et exercitui, qui erat in Ierusalem civitate David, quoniam descenderunt viri quidam, qui dissipaverant mandatum regis, in loca occulta in deserto. 
\verse Et cucurrerunt post illos multi, et deprehendentes eos applicaverunt contra eos et constituerunt adversus eos proelium in die sabbatorum 
\verse et dixerunt ad eos: “Usque hoc nunc! Exite et facite secundum verbum regis et vivetis". 
\verse Et dixerunt: “Non exibimus neque faciemus verbum regis, ut polluamus diem sabbatorum". 
\verse Et concitaverunt adversus eos proelium. 
\verse Et non responderunt eis nec lapidem miserunt in eos nec oppilaverunt loca occulta 
\verse dicentes: “Moriamur omnes in simplicitate nostra, et testes erunt super nos caelum et terra quod iniuste perditis nos". 
\verse Et insurrexerunt in eos in bello sabbatis; et mortui sunt ipsi et uxores eorum et filii eorum et pecora eorum usque ad mille animas hominum. 
\verse Et cognovit Matthathias et amici eius et luctum habuerunt super eos valde;  
\verse et dixit vir proximo suo: “Si omnes fecerimus, sicut fratres nostri fecerunt, et non pugnaverimus adversus gentes pro animabus nostris et iustificationibus nostris, nunc citius disperdent nos a terra". 
\verse Et cogitaverunt in die illa dicentes: “Omnis homo, quicumque venerit ad nos in bello die sabbatorum, pugnemus adversus eum et non moriemur omnes, sicut mortui sunt fratres nostri in occultis". 
\verse Tunc congregata est ad eos synagoga Asidaeorum fortis viribus ex Israel, omnis voluntarius in lege; 
\verse et omnes, qui fugiebant a malis, additi sunt ad eos et facti sunt illis ad firmamentum.  
\verse Et constituerunt exercitum et percusserunt peccatores in ira sua et viros iniquos in indignatione sua; et ceteri fugerunt ad nationes, ut se liberarent.  
\verse Et circuivit Matthathias et amici eius, et destruxerunt aras; 
\verse et circumciderunt pueros incircumcisos, quotquot invenerunt in finibus Israel, in fortitudine. 
\verse Et persecuti sunt filios superbiae, et prosperatum est opus in manu eorum; 
\verse et obtinuerunt legem de manu gentium et de manu regum et non dederunt cornu peccatori. 
\verse Et appropinquaverunt dies Matthathiae moriendi, et dixit filiis suis: “Nunc confirmata est superbia et castigatio et tempus eversionis et ira indignationis.  
\verse Nunc, o filii, aemulatores estote legis; et date animas vestras pro testamento patrum vestrorum. 
\verse Et mementote operum patrum, quae fecerunt in generationibus suis, et accipietis gloriam magnam et nomen aeternum. 
\verse Abraham, nonne in tentatione inventus est fidelis, et reputatum est ei ad iustitiam? 
\verse Ioseph in tempore angustiae suae custodivit mandatum et factus est dominus Aegypti. 
\verse Phinees pater noster zelando zelum Dei accepit testamentum sacerdotii aeterni. 
\verse Iosue, dum implet verbum, factus est iudex in Israel. 
\verse Chaleb, dum testificatur in ecclesia, accepit hereditatem. 
\verse David in sua misericordia consecutus est sedem regni in saecula. 
\verse Elias, dum zelat zelum legis, receptus est in caelum. 
\verse Ananias et Azarias et Misael credentes liberati sunt de flamma. 
\verse Daniel in sua simplicitate liberatus est de ore leonum. 
\verse Et ita cogitate per generationem et generationem, quia omnes, qui sperant in eum, non infirmabuntur.  
\verse Et a verbis viri peccatoris ne timueritis, quia gloria eius in stercora et in vermes; 
\verse hodie extolletur et cras non invenietur, quia conversus est in pulverem suam, et cogitatio eius peribit. 
\verse Filii, confortamini et viriliter agite in lege, quia in ipsa gloriosi eritis. 
\verse Et ecce Simon frater vester, scio quod vir consilii est: ipsum audite semper; ipse erit vobis pater. 
\verse Et Iudas Maccabaeus fortis viribus a iuventute sua erit vobis princeps militiae; et ipse pugnabit bellum populi.  
\verse Et adducetis ad vos omnes factores legis et vindicate vindictam populi vestri; 
\verse retribuite retributionem gentibus et intendite in praeceptum legis". 
\verse Et benedixit eos et appositus est ad patres suos. 
\verse Et defunctus est anno centesimo et quadragesimo sexto; et sepultus est in sepulcris patrum suorum in Modin, et planxerunt eum omnis Israel planctu magno. 
\end{biblechapter}

\begin{biblechapter}  
\verse Et surrexit Iudas, qui vocabatur Maccabaeus, filius eius pro eo. 
\verse Et adiuvabant eum omnes fratres eius et universi, qui se coniunxerant patri eius; et proeliabantur proelium Israel cum laetitia. 
\verse Et dilatavit gloriam populo suo et induit se loricam sicut gigas et succinxit se arma bellica sua et proelia constituit protegens castra gladio. 
\verse Similis factus est leoni in operibus suis et sicut catulus leonis rugiens in venationem; 
\verse et persecutus est iniquos perscrutans et eos, qui conturbabant populum suum, succendit. 
\verse Et subducti sunt iniqui prae timore eius, et omnes operarii iniquitatis conturbati sunt; et prosperata est salus in manu eius. 
\verse Et exacerbabat reges multos; et laetificabat Iacob in operibus suis, et in saeculum memoria eius in benedictionem. 
\verse Et perambulavit in civitatibus Iudae et disperdidit impios ex ea et avertit iram ab Israel; 
\verse et nominatus est usque ad ultimum terrae et congregavit pereuntes. 
\verse Et congregavit Apollonius gentes et a Samaria virtutem magnam ad bellandum contra Israel. 
\verse Et cognovit Iudas et exiit obviam illi; et percussit eum et occidit, et ceciderunt vulnerati multi, et reliqui fugerunt. 
\verse Et acceperunt spolia eorum, et gladium Apollonii accepit Iudas; et erat pugnans in eo omnibus diebus. 
\verse Et audivit Seron, princeps exercitus Syriae, quod congregavit Iudas congregationem et convocationem fidelium secum et egredientium in proelium, 
\verse et ait: “Faciam mihi nomen et glorificabor in regno et debellabo Iudam et eos, qui cum ipso sunt, qui spernunt verbum regis". 
\verse Et accessit, et ascendit cum eo exercitus impiorum fortis auxiliari ei, ut faceret vindictam in filios Israel. 
\verse Et appropinquavit usque ad ascensum Bethoron, et exivit Iudas obviam illi cum paucis. 
\verse Ut autem viderunt exercitum venientem sibi obviam dixerunt Iudae: “Quomodo poterimus pauci pugnare contra multitudinem tantam, fortem, et nos fatigati sumus ieiunio hodie?". 
\verse Et ait Iudas: “Facile est concludi multos in manibus paucorum; et non est differentia in conspectu caeli liberare in multis aut in paucis, 
\verse quoniam non in multitudine exercitus victoria belli, sed de caelo fortitudo est.  
\verse Ipsi veniunt ad nos in multitudine contumeliae et iniquitatis, ut disperdant nos et uxores nostras et filios nostros et ut spolient nos; 
\verse nos vero pugnamus pro animabus nostris et pro legitimis nostris, 
\verse et ipse Dominus conteret eos ante faciem nostram; vos autem ne timueritis eos". 
\verse Ut cessavit autem loqui, insiluit in eos subito; et contritus est Seron et exercitus eius in conspectu ipsius. 
\verse Et persequebantur eum in descensu Bethoron usque ad campum, et ceciderunt ex eis octingenti viri; reliqui autem fugerunt in terram Philisthim. 
\verse Et coepit timor Iudae ac fratrum eius, et formido cecidit super gentes in circuitu eorum; 
\verse et pervenit ad regem nomen eius, et de proeliis Iudae narrabant omnes gentes. 
\verse Ut audivit autem Antiochus sermones istos, iratus est animo; et misit et congregavit exercitum universi regni sui, castra fortia valde. 
\verse Et aperuit aerarium suum et dedit stipendia exercitui in annum et mandavit illis, ut essent parati ad omnia. 
\verse Et vidit quod defecit pecunia de thesauris suis, et tributa regionis modica propter dissensionem et plagam, quam fecit in terra, ut tolleret legitima, quae erant a primis diebus; 
\verse et timuit, ne non haberet ut semel et bis in sumptus et donaria, quae dederat antea larga manu, et abundaverat super reges, qui ante eum fuerant. 
\verse Et consternatus erat animo valde et cogitavit ire in Persidem et accipere tributa regionum et congregare argentum multum. 
\verse Et reliquit Lysiam hominem nobilem de genere regali super negotia regia a flumine Euphrate usque ad fines Aegypti 
\verse et ut nutriret Antiochum filium suum, donec rediret. 
\verse Et tradidit ei dimidium exercitum et elephantos et mandavit ei de omnibus, quae volebat, et de inhabitantibus Iudaeam et Ierusalem, 
\verse ut mitteret ad eos exercitum ad conterendam et exstirpandam virtutem Israel et reliquias Ierusalem et auferendam memoriam eorum de loco, 
\verse et ut constitueret habitatores filios alienigenas in omnibus finibus eorum et sorte distribueret terram eorum. 
\verse Et rex assumpsit dimidium exercitum residuum et exivit ab Antiochia de civitate regni sui anno centesimo et quadragesimo septimo et transfretavit Euphratem flumen et perambulabat superiores regiones. 
\verse Et elegit Lysias Ptolemaeum filium Dorymeni et Nicanorem et Gorgiam viros potentes ex amicis regis 
\verse et misit cum eis quadraginta milia virorum et septem milia equitum, ut venirent in terram Iudae et disperderent eam secundum verbum regis. 
\verse Et processerunt cum universa virtute sua; et venerunt et applicaverunt prope Emmaus in terra campestri. 
\verse Et audierunt mercatores regionis nomen eorum et acceperunt argentum et aurum multum valde et compedes et venerunt in castra, ut acciperent filios Israel in servos, et additi sunt ad eos exercitus Syriae et terrae alienigenarum. 
\verse Et vidit Iudas et fratres eius quia multiplicata sunt mala, et exercitus applicabant ad fines eorum, et cognoverunt verba regis, quae mandavit facere populo in interitum et consummationem. 
\verse Et dixerunt unusquisque ad proximum suum: “Erigamus deiectionem populi nostri et pugnemus pro populo nostro et sanctis nostris". 
\verse Et congregatus est conventus, ut essent parati in proelium et ut orarent et peterent misericordiam et miserationes.  
\verse Et Ierusalem non habitabatur sicut desertum; non erat qui ingrederetur et egrederetur de natis eius, et sanctum conculcabatur, et filii alienigenarum erant in arce; ibi erat habitatio gentibus. Et ablata est voluptas a Iacob, et defecit tibia et cithara. 
\verse Et congregati sunt et venerunt in Maspha contra Ierusalem, quia in Maspha erat antea locus orationis Israeli; 
\verse et ieiunaverunt illa die et induerunt se ciliciis et cinerem imposuerunt capiti suo et disciderunt vestimenta sua  
\verse et expanderunt librum Legis, de quibus scrutabantur gentes similitudines simulacrorum suorum; 
\verse et attulerunt ornamenta sacerdotalia et primitias et decimas et suscitaverunt nazaraeos, qui impleverunt dies, 
\verse et clamaverunt voce in caelum dicentes: “Quid faciemus istis et quo eos ducemus? 
\verse Et sancta tua conculcata sunt et contaminata, et sacerdotes tui in luctu et humiliatione; 
\verse et ecce nationes convenerunt adversum nos, ut nos disperdant: tu scis, quae cogitant in nos. 
\verse Quomodo poterimus subsistere ante faciem eorum, nisi tu adiuves nos?”. 
\verse Et tubis bucinaverunt et clamaverunt voce magna. 
\verse Et post haec constituit Iudas duces populi, tribunos et centuriones et pentacontarchos et decuriones 
\verse et dixit his, qui aedificabant domos et sponsabant uxores et plantabant vineas, et formidolosis, ut redirent unusquisque in domum suam secundum legem. 
\verse Et moverunt castra et collocaverunt ad austrum Emmaum. 
\verse Et ait Iudas: “Accingimini et estote filii potentes et estote parati in mane, ut pugnetis adversus nationes has, quae convenerunt adversus nos disperdere nos et sancta nostra; 
\verse quoniam melius est nos mori in bello quam respicere mala gentis nostrae et sanctorum. 
\verse Sicut autem fuerit voluntas in caelo, sic faciet". 
\end{biblechapter}

\begin{biblechapter}  
\verse Et assumpsit Gorgias quinque milia virorum et mille equites electos, et moverunt castra nocte, 
\verse ut applicarent ad castra Iudaeorum et percuterent eos subito; et, qui erant ex arce, erant illis duces. 
\verse Et audivit Iudas et surrexit ipse et potentes percutere exercitum regis, qui erat in Emmaus, 
\verse dum adhuc dispersus esset exercitus a castris. 
\verse Et venit Gorgias in castra Iudae noctu et neminem invenit; et quaerebat eos in montibus, quoniam dixit: “Fugiunt hi a nobis". 
\verse Et simul ut dies factus est, apparuit Iudas in campo cum tribus milibus virorum, tantum quod tegumenta et gladios non habebant ut volebant. 
\verse Et viderunt castra gentium valida et loricatos et equitatus in circuitu eorum, et hi docti ad proelium. 
\verse Et ait Iudas viris, qui secum erant: “Ne timueritis multitudinem eorum et impetum eorum ne formidetis; 
\verse mementote qualiter salvi facti sunt patres nostri in mari Rubro, cum sequeretur eos pharao cum exercitu. 
\verse Et nunc clamemus in caelum, si miserebitur nostri et memor erit testamenti patrum nostrorum et conteret exercitum istum ante faciem nostram hodie; 
\verse et scient omnes gentes quia est, qui redimat et liberet Israel". 
\verse Et levaverunt alienigenae oculos suos et viderunt eos venientes ex adverso  
\verse et exierunt de castris in proelium. Et tuba cecinerunt hi, qui erant cum Iuda, 
\verse et commiserunt bellum; et contritae sunt gentes et fugerunt in campum.  
\verse Novissimi autem omnes ceciderunt in gladio, et persecuti sunt eos usque Gazeron et usque in campos Idumaeae et Azoti et Iamniae; et ceciderunt ex illis usque ad tria milia virorum. 
\verse Et reversus est Iudas et exercitus eius a persecutione eorum; 
\verse dixitque ad populum: “Non concupiscatis spolia, quia bellum contra nos est, 
\verse et Gorgias et exercitus eius prope nos in monte, sed state nunc contra inimicos nostros et expugnate eos et post haec accipite spolia confidenter". 
\verse Et, adhuc loquente Iuda haec, apparuit pars quaedam prospiciens de monte. 
\verse Et vidit quod in fugam conversi sunt sui, et succenderunt castra; fumus enim, qui videbatur, declarabat, quod factum est.  
\verse Quibus illi conspectis, timuerunt valde; aspicientes vero et Iudae exercitum in campo paratum ad proelium 
\verse fugerunt omnes in terram alienigenarum.  
\verse Et Iudas reversus est ad spolia castrorum; et acceperunt aurum multum et argentum et hyacinthum et purpuram marinam et opes magnas. 
\verse Et conversi hymnum canebant et benedicebant in caelum: “Quoniam bonum, quoniam in saeculum misericordia eius". 
\verse Et facta est salus magna in Israel in die illa. 
\verse Quicumque autem alienigenarum evaserunt, venerunt et nuntiaverunt Lysiae universa, quae acciderant; 
\verse quibus ille auditis, consternatus est et animo deficiebat, quod non, qualia voluit, talia contigerant in Israel, et, qualia mandaverat ei rex, evenerant. 
\verse Et sequenti anno congregavit virorum electorum sexaginta milia et equitum quinque milia, ut debellaret eos. 
\verse Et venerunt in Idumaeam et castra posuerunt in Bethsuris; et occurrit illis Iudas cum decem milibus viris. 
\verse Et vidit exercitum fortem et oravit et dixit: “Benedictus es, Salvator Israel, qui contrivisti impetum potentis in manu servi tui David et tradidisti castra alienigenarum in manu Ionathae filii Saul et armigeri eius. 
\verse Conclude exercitum istum in manu populi tui Israel, et confundantur in exercitu suo et equitibus suis. 
\verse Da illis formidinem et tabefac audaciam virtutis eorum, et commoveantur contritione sua; 
\verse deice illos gladio diligentium te, et collaudent te omnes, qui noverunt nomen tuum, in hymnis". 
\verse Et commiserunt invicem proelium, et ceciderunt de exercitu Lysiae quinque milia virorum et prociderunt ex adverso eorum. 
\verse Videns autem Lysias factam eversionem exercitus sui, Iudae vero audaciam et quemadmodum parati sunt aut vivere aut mori fortiter, abiit Antiochiam et colligebat externos, ut multo numero rursus venirent in Iudaeam. 
\verse Dixit autem Iudas et fratres eius: “Ecce contriti sunt inimici nostri; ascendamus mundare sancta et renovare".  
\verse Et congregatus est omnis exercitus, et ascenderunt in montem Sion. 
\verse Et viderunt sanctificationem desertam et altare profanatum et portas exustas et in atriis virgulta nata, sicut in saltu vel in uno ex montibus, et pastophoria diruta. 
\verse Et sciderunt vestimenta sua et planxerunt planctu magno et imposuerunt cinerem 
\verse et ceciderunt in faciem super terram et exclamaverunt tubis signorum et clamaverunt in caelum. 
\verse Tunc ordinavit Iudas viros, ut pugnarent adversus eos, qui erant in arce, donec mundaret sancta. 
\verse Et elegit sacerdotes sine macula voluntatem habentes in lege; 
\verse et mundaverunt sancta et tulerunt lapides contaminationis in locum immundum. 
\verse Et cogitaverunt de altari holocaustorum, quod profanatum erat, quid de eo facerent. 
\verse Et incidit illis consilium bonum, ut destruerent illud, ne umquam illis esset in opprobrium, quia contaminaverunt illud gentes; et demoliti sunt altare  
\verse et reposuerunt lapides in monte domus in loco apto, quoadusque veniret propheta, ut responderet de eis. 
\verse Et acceperunt lapides integros secundum legem et aedificaverunt altare novum secundum illud, quod fuit prius. 
\verse Et aedificaverunt sancta et, quae intra domum erant, et atria sanctificaverunt.  
\verse Et fecerunt vasa sancta nova et intulerunt candelabrum et altare incensorum et mensam in templum. 
\verse Et incenderunt super altare et accenderunt lucernas, quae super candelabrum erant et lucebant in templo, 
\verse et posuerunt super mensam panes et appenderunt vela et consummaverunt omnia opera, quae fecerant. 
\verse Et ante lucem surrexerunt quinta et vicesima die mensis noni — hic est mensis Casleu — centesimi quadragesimi octavi anni 
\verse et obtulerunt sacrificium secundum legem super altare holocaustorum novum, quod fecerunt; 
\verse secundum tempus et secundum diem, in qua contaminaverunt illud gentes, in ipsa renovatum est in canticis et citharis et cinyris et cymbalis.  
\verse Et cecidit omnis populus in faciem, et adoraverunt et benedixerunt in caelum eum, qui prosperavit eis; 
\verse et fecerunt dedicationem altaris diebus octo et obtulerunt holocausta cum laetitia et sacrificaverunt sacrificium salutaris et laudis 
\verse et ornaverunt faciem templi coronis aureis et scutulis et dedicaverunt portas et pastophoria et imposuerunt eis ianuas. 
\verse Et facta est laetitia in populo magna valde, et aversum est opprobrium gentium. 
\verse Et statuit Iudas et fratres eius et universa ecclesia Israel, ut agantur dies dedicationis altaris in temporibus suis ab anno in annum per dies octo, a quinta et vicesima die mensis Casleu, cum laetitia et gaudio. 
\verse Et aedificaverunt in tempore illo montem Sion per circuitum muros altos et turres firmas, ne quando venirent gentes et conculcarent ea, sicut antea fecerunt. 
\verse Et collocavit illic exercitum, ut servarent eum, et munivit eum ad custodiendam Bethsuram, ut haberet populus munitionem contra faciem Idumaeae. 
\end{biblechapter}

\begin{biblechapter}  
\verse Et factum est, ut audierunt gentes in circuitu quia aedificatum est altare, et dedicatum est sanctuarium sicut prius, iratae sunt valde 
\verse et cogitabant tollere genus Iacob, qui erant inter eos, et coeperunt occidere in populo et persequi. 
\verse Et bellabat Iudas adversus filios Esau in Idumaea, in Acrabattane, quia circumsedebant Israel; et percussit eos plaga magna, compressit eos et cepit spolia eorum. 
\verse Et recordatus est malitiam filiorum Bean, qui erant populo in laqueum et in scandalum insidiantes eis in viis. 
\verse Et conclusi sunt ab eo in turribus, et applicuit ad eos et anathematizavit eos et incendit turres eius igne cum omnibus, qui intus erant. 
\verse Et transivit ad filios Ammon et invenit manum fortem et populum copiosum et Timotheum ducem ipsorum. 
\verse Et commisit cum eis proelia multa, et contriti sunt in conspectu eius, et percussit eos. 
\verse Et cepit Iazer et filias eius et reversus est in Iudaeam. 
\verse Et congregatae sunt gentes, quae sunt in Galaad, adversus Israel, qui erant in finibus eorum, ut tollerent eos; et fugerunt in Datheman munitionem 
\verse et miserunt litteras ad Iudam et fratres eius dicentes: “Congregatae sunt adversum nos gentes per circuitum, ut nos auferant, 
\verse et parant venire et occupare munitionem, in quam confugimus, et Timotheus est dux exercitus eorum. 
\verse Nunc ergo veni et eripe nos de manibus eorum, quia cecidit multitudo de nobis, 
\verse et omnes fratres nostri, qui erant in locis Tubin, interfecti sunt, et captivas duxerunt uxores eorum et natos et sarcinas et peremerunt illic fere mille viros". 
\verse Adhuc epistulae legebantur, et ecce alii nuntii venerunt de Galilaea, conscissis tunicis, nuntiantes secundum verba haec, 
\verse dicentes convenisse adversum se a Ptolemaida et Tyro et Sidone et omnem Galilaeam alienigenarum, ut nos consumant. 
\verse Ut audivit autem Iudas et populus sermones istos, convenit ecclesia magna cogitare quid facerent fratribus suis, qui in tribulatione erant et expugnabantur ab eis. 
\verse Dixitque Iudas Simoni fratri suo: “Elige tibi viros et vade et libera fratres tuos, qui sunt in Galilaea; ego autem et frater meus Ionathas ibimus in Galaaditim". 
\verse Et reliquit Iosephum filium Zachariae et Azariam ducem populi cum residuo exercitu in Iudaea ad custodiam.  
\verse Et praecepit illis dicens: “Praeestote populo huic et nolite bellum committere adversum gentes, donec revertamur". 
\verse Et partiti sunt Simoni virorum tria milia, ut iret in Galilaeam, Iudae autem octo milia in Galaaditim.  
\verse Et abiit Simon in Galilaeam et commisit proelia multa cum gentibus; et contritae sunt gentes a facie eius, 
\verse et persecutus est eos usque ad portam Ptolemaidis, et ceciderunt de gentibus fere tria milia virorum, et accepit spolia eorum. 
\verse Et assumpsit eos, qui erant de Galilaea et in Arbattis, cum uxoribus et natis et omnibus, quae erant illis, et adduxit in Iudaeam cum laetitia magna. 
\verse Et Iudas Maccabaeus et Ionathas frater eius transierunt Iordanem et abierunt viam trium dierum in deserto; 
\verse et occurrerunt Nabathaeis et obviaverunt eis pacifice et narraverunt eis omnia, quae acciderant fratribus eorum in Galaaditide, 
\verse et quia multi ex eis comprehensi sunt in Bosora et Bosor in Alimis, Chaspho, Maced et Carnain; hae omnes civitates munitae et magnae. 
\verse Et in ceteris civitatibus Galaaditidis tenentur comprehensi; in crastinum constituerunt admovere ad munitiones et comprehendere et tollere omnes eos in una die. 
\verse Et convertit Iudas et exercitus eius viam in desertum Bosora repente et occupavit civitatem et occidit omnem masculum in ore gladii et accepit omnia spolia eorum et succendit eam igne; 
\verse et profectus est inde nocte, et ibant usque ad munitionem. 
\verse Et factum est diluculo, cum levassent oculos suos, ecce populus multus, cuius non erat numerus, portantes scalas et machinas, ut comprehenderent munitionem, et expugnabant eos. 
\verse Et vidit Iudas quia coepit bellum, et clamor civitatis ascendit ad caelum sicut tuba et clamor magnus; 
\verse et dixit viris exercitus: “Pugnate hodie pro fratribus nostris". 
\verse Et exiit tribus ordinibus post eos, et exclamaverunt tubis et clamaverunt in oratione. 
\verse Et cognoverunt castra Timothei quia Maccabaeus est, et refugerunt a facie eius; et percussit eos plaga magna, et ceciderunt ex eis in die illa fere octo milia virorum. 
\verse Et divertit Iudas in Maspha et expugnavit et cepit eam. Et occidit omnem masculinum eius et sumpsit spolia eius et succendit eam igne. 
\verse Inde perrexit et cepit Chaspho, Maced et Bosor et reliquas civitates Galaaditidis. 
\verse Post haec autem verba congregavit Timotheus exercitum alium et castra posuit contra Raphon trans torrentem. 
\verse Et misit Iudas speculari exercitum, et renuntiaverunt ei dicentes: “Convenerunt ad eum omnes gentes, quae in circuitu nostro sunt, exercitus multus nimis; 
\verse et Arabas conduxit in auxilium sibi, et castra posuerunt trans torrentem, parati ad te venire in proelium". Et abiit Iudas obviam illis. 
\verse Et ait Timotheus principibus exercitus sui: “Cum appropinquaverit Iudas et exercitus eius ad torrentem aquae, si transierit ad nos prior, non poterimus sustinere eum, quia potens poterit adversum nos; 
\verse si vero timuerit transire et posuerit castra ultra flumen, transfretemus ad eos et poterimus adversus illum". 
\verse Ut autem appropinquavit Iudas ad torrentem aquae, statuit scribas populi secus torrentem et mandavit eis dicens: “Neminem hominum reliqueritis, sed veniant omnes in proelium". 
\verse Et transfretavit ad illos prior, et omnis populus post eum. Et contritae sunt omnes gentes a facie eorum et proiecerunt arma sua et fugerunt ad fanum in Carnain. 
\verse Et occupaverunt ipsam civitatem et fanum succenderunt igne cum omnibus, qui erant in ipso; et oppressa est Carnain et non potuit sustinere contra faciem Iudae. 
\verse Et congregavit Iudas universum Israel, qui erant in Galaaditide, a minimo usque ad maximum et uxores eorum et natos et sarcinas, exercitum magnum valde, ut venirent in terram Iudae. 
\verse Et venerunt usque Ephron. Et haec civitas magna in via, munita valde; non erat declinare ab ea dextera vel sinistra, sed per mediam iter erat. 
\verse Et incluserunt se, qui erant in civitate, et obstruxerunt portas lapidibus. Et misit ad eos Iudas verbis pacificis 
\verse dicens: “Transeamus per terram vestram, ut eamus in terram nostram, et nemo vobis nocebit; tantum pedibus transibimus". Et nolebant eis aperire. 
\verse Et praecepit Iudas praedicare in castris, ut applicarent se unusquisque in quo erat loco. 
\verse Et applicuerunt se viri virtutis, et oppugnavit civitatem illam tota die et tota nocte; et tradita est civitas in manu eius. 
\verse Et peremit omnem masculinum in ore gladii et eradicavit eam et accepit spolia eius et transivit per civitatem super interfectos. 
\verse Et transgressi sunt Iordanem in campum magnum contra faciem Bethsan. 
\verse Et erat Iudas congregans extremos et exhortabatur populum per totam viam, donec veniret in terram Iudae. 
\verse Et ascenderunt in montem Sion cum laetitia et gaudio et obtulerunt holocausta, quod nemo ex eis cecidisset, donec reverterentur in pace. 
\verse Et in diebus, quibus erat Iudas et Ionathas in terra Galaad, et Simon frater eius in Galilaea contra faciem Ptolemaidis, 
\verse audivit Iosephus Zachariae filius et Azarias princeps virtutis res bene gestas et proelia, quae fecerunt,  
\verse et dixerunt: “Faciamus et ipsi nobis nomen et eamus pugnare adversus gentes, quae in circuitu nostro sunt". 
\verse Et nuntiaverunt his, qui erant de exercitu suo, et abierunt ad Iamniam. 
\verse Et exivit Gorgias de civitate et viri eius obviam illis in pugnam; 
\verse et fugati sunt Iosephus et Azarias et impulsi sunt usque in fines Iudaeae, et ceciderunt illo die de populo Israel ad duo milia viri; et facta est fuga magna in populo, 
\verse quia non audie runt Iudam et fratres eius existimantes fortiter se facturos. 
\verse Ipsi autem non erant de semine virorum illorum, per quorum manum salus data est Israel. 
\verse Et vir Iudas et fratres eius magnificati sunt valde in conspectu omnis Israel et gentium omnium, ubi audiebatur nomen eorum; 
\verse et conveniebant ad eos fausta acclamantes. 
\verse Et exivit Iudas et fratres eius et expugnabant filios Esau in terra, quae ad austrum est. Et percussit Hebron et filias eius et destruxit munitiones eius et turres eius succendit in circuitu. 
\verse Et movit castra, ut iret in terram alienigenarum, et perambulabat Maresam. 
\verse In die illa ceciderunt sacerdotes in bello, dum volunt fortiter facere, dum sine consilio exeunt in proelium. 
\verse Et declinavit Iudas in Azotum terram alienigenarum et diruit aras eorum et sculptilia deorum ipsorum succendit igne et expoliavit exuvias civitatum. Et reversus est in terram Iudae. 
\end{biblechapter}

\begin{biblechapter}  
\verse Et rex Antiochus perambulabat superiores regiones et audivit esse Elymaida in Perside civitatem gloriosam divitiis argento et auro 
\verse templumque in ea locuples valde et illic velamina aurea et loricae et scuta, quae reliquit ibi Alexander Philippi rex Macedo, qui regnavit primus in Graecia.  
\verse Et venit et quaerebat capere civitatem et depraedari eam et non potuit, quoniam innotuit sermo his, qui erant in civitate. 
\verse Et restiterunt ei in proelium. Et fugit inde et abiit cum tristitia magna, ut reverteretur in Babyloniam. 
\verse Et venit, qui nuntiaret ei in Perside quia fugata sunt castra, quae iverant in terram Iudae, 
\verse et quia abiit Lysias cum virtute forti in primis et fugatus est a facie eorum, et invaluerunt armis et viribus et spoliis multis, quae ceperunt de castris quae exciderunt, 
\verse et quia diruerunt abominationem, quam aedificaverat super altare, quod erat in Ierusalem, et sanctificationem sicut prius circumdederunt muris excelsis et Bethsuram civitatem eius. 
\verse Et factum est, ut audivit rex sermones istos, expavit et commotus est valde et decidit in lectum et incidit in languorem prae tristitia, quia non factum est ei, sicut cogitabat. 
\verse Et erat illic per dies multos, quia renovata est in eo tristitia magna, et arbitratus est se mori. 
\verse Et vocavit omnes amicos suos et dixit illis: “Recessit somnus ab oculis meis, et concidi corde prae sollicitudine 
\verse et dixi in corde meo: Quousque tribulationis deveni et tempestatis magnae, in qua nunc sum? Quia iucundus eram et dilectus in potestate mea! 
\verse Nunc vero reminiscor malorum, quae feci in Ierusalem, unde et abstuli omnia vasa aurea et argentea, quae erant in ea, et misi auferre habitantes Iudam sine causa. 
\verse Cognovi quia propterea invenerunt me mala ista; et ecce pereo tristitia magna in terra aliena". 
\verse Et vocavit Philippum, unum de amicis suis, et praeposuit eum super universum regnum suum; 
\verse et dedit ei diadema et stolam suam et anulum, ut adduceret Antiochum filium suum et nutriret eum, ut regnaret. 
\verse Et mortuus est illic Antiochus rex anno centesimo quadragesimo nono. 
\verse Et cognovit Lysias quoniam mortuus est rex, et constituit regnare Antiochum filium eius pro eo, quem nutrivit adulescentiorem; et vocavit nomen eius Eupatorem. 
\verse Et hi, qui erant in arce, concluserant Israel in circuitu sanctorum et quaerebant eis mala semper et firmamentum gentium. 
\verse Et cogitavit Iudas disperdere eos et convocavit universum populum, ut obsiderent eos. 
\verse Et convenerunt simul et obsederunt eos anno centesimo quinquagesimo et fecerunt ballistas et machinas. 
\verse Et exierunt quidam ex eis, qui obsidebantur, et adiunxerunt se illis aliqui impii ex Israel 
\verse et abierunt ad regem et dixerunt: “Quousque non facis iudicium et vindicabis fratres nostros? 
\verse Nos decrevimus servire patri tuo et ambulare in praeceptis eius et obsequi edictis eius; 
\verse et filii populi nostri propter hoc obsederunt arcem et alienabant se a nobis, et, quicumque inveniebantur ex nobis, interficiebantur, et hereditates nostrae diripiebantur. 
\verse Et non ad nos tantum extenderunt manum sed et in omnes fines suos; 
\verse et ecce applicuerunt hodie ad arcem in Ierusalem occupare eam et sancta et Bethsuram munierunt. 
\verse Et, nisi praeveneris eos velocius, maiora quam haec facient, et non poteris continere eos". 
\verse Et iratus est rex, ut audivit, et convocavit omnes amicos suos et principes exercitus sui et eos, qui super vehicula erant; 
\verse sed et de regnis aliis et de insulis maritimis venerunt ad eum exercitus conducticii. 
\verse Et erat numerus exercitus eius centum milia peditum et viginti milia equitum, et elephanti triginta duo docti ad proelium. 
\verse Et venerunt per Idumaeam et applicuerunt ad Bethsuram. Et pugnaverunt dies multos et fecerunt machinas et exierunt et succenderunt eas igne et pugnaverunt viriliter. 
\verse Et recessit Iudas ab arce et movit castra ad Bethzacharam contra castra regis. 
\verse Et surrexit rex ante lucem et excitavit exercitum in impetu suo contra viam Bethzacharam, et comparaverunt se exercitus in proelium et tubis cecinerunt  
\verse et elephantis ostenderunt sanguinem uvae et mori ad acuendos eos in proelium.  
\verse Et diviserunt bestias per legiones, et astiterunt singulis elephantis mille viri in loricis concatenatis, et galeae aereae in capitibus eorum, et quingenti equites ordinati unicuique bestiae electi. 
\verse Hi ante tempus, ubicumque erat bestia, ibi erant et, quocumque ibat, ibant; non discedebant ab ea. 
\verse Et turres ligneae super eos firmae, protectae super singulas bestias, praecinctae super eas machinis, et super singulas viri virtutis quattuor, qui pugnabant desuper, et Indus eius. 
\verse Et residuos equites hinc et inde statuit in duas partes exercitus excitaturos et protecturos in legionibus. 
\verse Et, ut refulsit sol in clipeos aureos et aereos, resplenduerunt montes ab eis et resplenduerunt sicut lampades ignis. 
\verse Et distincta est pars exercitus regis super montes excelsos, et quidam per loca humilia; et ibant caute et ordinate. 
\verse Et commovebantur omnes audientes vocem multitudinis et incessum turbae et collisionem armorum; erat enim exercitus magnus valde et fortis.  
\verse Et appropiavit Iudas et exercitus eius in proelium, et ceciderunt de exercitu regis sescenti viri. 
\verse Et vidit Eleazar Abaran unam de bestiis loricatam loricis regis, et erat eminens super ceteras bestias, et visum est ei quod in ea esset rex; 
\verse et dedit se, ut liberaret populum suum et acquireret sibi nomen aeternum. 
\verse Et cucurrit ad eam audacter, in medio legionis interficiens a dextris et a sinistris, et findebantur ab eo huc atque illuc; 
\verse et ivit sub elephantum et supposuit se ei et occidit eum; et cecidit in terram super ipsum, et mortuus est illic. 
\verse Et videntes virtutem regni et impetum exercituum diverterunt se ab eis.  
\verse Qui autem erant de castris regis, ascenderunt obviam illis in Ierusalem, et applicuit rex ad Iudaeam et montem Sion. 
\verse Et fecit pacem cum his, qui erant in Bethsura; et exierunt de civitate, quia non erant eis ibi alimenta, eo quod conclusi essent in ea, quia sabbatum erat terrae. 
\verse Et comprehendit rex Bethsuram et constituit illic custodiam servare eam. 
\verse Et applicuit castra ad locum sanctificationis dies multos; et statuit illic ballistas et machinas et ignis iacula et tormenta ad lapides iactandos et scorpios ad mittendas sagittas et fundibula. 
\verse Fecerunt autem et ipsi machinas adversus machinas eorum et pugnaverunt dies multos; 
\verse escae autem non erant in horreis, eo quod septimus annus esset, et, qui evaserant in Iudaeam de gentibus, consumpserant reliquias repositionis. 
\verse Et remanserunt in sanctis viri pauci, quoniam obtinuerat eos fames, et dispersi sunt unusquisque in locum suum. 
\verse Et audivit Lysias quod Philippus, quem constituerat rex Antiochus, cum adhuc viveret, ut nutriret Antiochum filium suum ut regnaret, 
\verse reversus esset a Perside et Media, et exercitus, qui abierat cum ipso, et quia quaerebat suscipere regni negotia. 
\verse Festinavit et significavit ire dixitque ad regem et duces exercitus et viros: “Deficimus cotidie, et esca nobis modica est, et locus, quem obsidemus, est munitus, et incumbunt nobis negotia regni. 
\verse Nunc itaque demus dextras hominibus istis et faciamus cum illis pacem et cum omni gente eorum 
\verse et constituamus illis, ut ambulent in legitimis suis sicut prius; propter legitima enim ipsorum, quae dispersimus, irati sunt et fecerunt omnia haec". 
\verse Et placuit sermo in conspectu regis et principum, et misit ad eos pacem facere, et receperunt illam. 
\verse Et iuravit illis rex et principes. His condicionibus exierunt de munitione. 
\verse Et intravit rex in montem Sion et vidit munitionem loci et rupit iuramentum, quod iuravit, et mandavit destruere murum in gyro. 
\verse Et discessit festinanter et reversus est Antiochiam; et invenit Philippum dominantem civitati et pugnavit adversus eum et occupavit civitatem per vim. 
\end{biblechapter}

\begin{biblechapter}  
\verse Anno centesimo quinquagesimo primo exiit Demetrius Seleuci filius a Roma; et ascendit cum paucis viris in civitatem maritimam et regnavit illic. 
\verse Et factum est, ut ingressus est domum regni patrum suorum, comprehendit exercitus Antiochum et Lysiam, ut adduceret eos ad eum. 
\verse Et res ei innotuit, et ait: “Nolite mihi ostendere facies eorum". 
\verse Et occidit eos exercitus, et sedit Demetrius super sedem regni sui. 
\verse Et venerunt ad eum viri iniqui et impii ex Israel, et Alcimus dux erat eorum, qui volebat fieri sacerdos; 
\verse et accusaverunt populum apud regem dicentes: “Perdidit Iudas et fratres eius omnes amicos tuos et nos dispersit de terra nostra; 
\verse nunc ergo mitte virum, cui credis, ut eat et videat exterminium omne, quod fecit nobis et regioni regis, et puniat eos et omnes adiutores eorum". 
\verse Et elegit rex ex amicis suis Bacchidem, qui dominabatur trans flumen, magnum in regno et fidelem regi. Et misit eum 
\verse et Alcimum impium et constituit ei sacerdotium et mandavit ei facere ultionem in filios Israel. 
\verse Et surrexerunt et venerunt cum exercitu magno in terram Iudae; et misit nuntios ad Iudam et ad fratres eius verbis pacificis in dolo. 
\verse Et non intenderunt sermonibus eorum; viderunt enim quia venerunt cum exercitu magno. 
\verse Et convenerunt ad Alcimum et Bacchidem congregatio scribarum requirere iusta;  
\verse et primi Asidaei erant in filiis Israel et exquirebant ab eis pacem. 
\verse Dixerunt enim: “Homo sacerdos de semine Aaron venit et non decipiet nos".  
\verse Et locutus est cum eis verba pacifica et iuravit illis dicens: “Non inferemus vobis malum neque amicis vestris". 
\verse Et crediderunt ei. Et comprehendit ex eis sexaginta viros et occidit eos in una die, secundum verbum quod scripsit: 
\verse “Carnes sanctorum tuorum et sanguinem ipsorum effuderunt in circuitu Ierusalem, et non erat qui sepeliret". 
\verse Et incubuit timor eorum et tremor in omnem populum, quia dixerunt: “Non est in eis veritas et iudicium; transgressi sunt enim constitutum et iusiurandum, quod iuraverunt". 
\verse Et movit Bacchides ab Ierusalem et applicuit in Bethzaith; et misit et comprehendit multos ex eis, qui ad se refugerant, et quosdam de populo mactavit in puteum magnum. 
\verse Et commisit regionem Alcimo et reliquit cum eo auxilium in adiutorium ipsi; et abiit Bacchides ad regem. 
\verse Et contendebat Alcimus pro principatu sacerdotii sui; 
\verse et convenerunt ad eum omnes, qui perturbabant populum suum, et obtinuerunt terram Iudae et fecerunt plagam magnam in Israel. 
\verse Et vidit Iudas omnem malitiam, quam fecit Alcimus et, qui cum eo erant, filiis Israel multo plus quam gentes; 
\verse et exiit in omnes fines Iudaeae in circuitu et fecit vindictam in viros desertores, et cessaverunt ultra exire in regionem. 
\verse Ut vidit autem Alcimus quod praevaluit Iudas et, qui cum eo erant, et cognovit quia non potest sustinere eos; et regressus est ad regem et accusavit eos criminibus. 
\verse Et misit rex Nicanorem unum ex principibus suis nobilioribus, qui oderat et inimicitias exercebat contra Israel; et mandavit ei evertere populum. 
\verse Et venit Nicanor in Ierusalem cum exercitu magno et misit ad Iudam et ad fratres eius verbis pacificis cum dolo 
\verse dicens: “Non sit pugna inter me et vos; veniam cum viris paucis, ut videam facies vestras cum pace". 
\verse Et venit ad Iudam, et salutaverunt se invicem pacifice, et hostes parati erant rapere Iudam. 
\verse Et innotuit sermo Iudae quoniam cum dolo venerat ad eum, et conterritus est ab eo et amplius noluit videre faciem eius. 
\verse Et cognovit Nicanor quoniam denudatum est consilium eius et exivit obviam Iudae in pugnam iuxta Chapharsalama; 
\verse et ceciderunt de Nicanoris exercitu fere quingenti viri, et fugerunt in civitatem David. 
\verse Et post haec verba ascendit Nicanor in montem Sion; et exierunt quidam ex sacerdotibus de sanctis et quidam ex senioribus populi salutare eum pacifice et demonstrare ei holocaustum, quod offerebatur pro rege. 
\verse Et irridens sprevit eos et polluit eos et locutus est superbe 
\verse et iuravit cum ira dicens: “Nisi traditus fuerit Iudas et exercitus eius in manus meas continuo, et erit, si regressus fuero in pace, succendam domum istam". Et exiit cum ira magna. 
\verse Et intraverunt sacerdotes et steterunt ante faciem altaris et templi et flentes dixerunt:  
\verse “Tu elegisti domum istam ad invocandum nomen tuum super eam, ut esset domus orationis et obsecrationis populo tuo; 
\verse fac vindictam in homine isto et exercitu eius, et cadant in gladio. Memento blasphemias eorum et ne dederis eis mansionem". 
\verse Et exiit Nicanor ab Ierusalem et applicuit ad Bethoron; et occurrit illi exercitus Syriae. 
\verse Et Iudas applicuit in Hadasa cum tribus milibus viris. Et oravit Iudas et dixit: 
\verse “Qui missi erant a rege, cum male locuti sunt; exiit angelus et percussit in eis centum octoginta quinque milia. 
\verse Sic contere exercitum istum in conspectu nostro hodie, et sciant ceteri quia male locutus est super sancta tua, et iudica illum secundum malitiam illius". 
\verse Et commiserunt exercitus proelium tertia decima die mensis Adar; et contrita sunt castra Nicanoris, et cecidit ipse primus in proelio. 
\verse Ut autem vidit exercitus eius quia cecidit Nicanor, proiecerunt arma et fugerunt. 
\verse Et persecuti sunt eos viam unius diei ab Hadasa usquequo veniatur in Gazara et tubis cecinerunt post eos cum significationibus. 
\verse Et exierunt de omnibus castellis Iudaeae in circuitu et ventilabant eos et conversi sunt ad eos; et ceciderunt omnes gladio, et non est relictus ex eis nec unus. 
\verse Et acceperunt spolia eorum et praedam et caput Nicanoris amputaverunt et dexteram eius, quam extenderat superbe, et attulerunt et suspenderunt contra Ierusalem.  
\verse Et laetatus est populus valde; et egerunt diem illam in laetitia magna 
\verse et constituerunt agere omnibus annis diem istam tertia decima die Adar. 
\verse Et siluit terra Iudae dies paucos. 
\end{biblechapter}

\begin{biblechapter}  
\verse Et audivit Iudas nomen Roma norum quia sunt potentes viribus et consentiunt omnibus, quae postulantur ab eis, et, quicumque accesserint ad eos, statuerunt cum eis amicitiam, 
\verse et quia sunt potentes viribus. Et narraverunt proelia eorum et virtutes bonas, quas fecerunt in Galatia, quia obtinuerunt eos et duxerunt eos sub tributum, 
\verse et quanta fecerunt in regione Hispaniae, quod in potestatem redegerunt metalla argenti et auri, quae illic sunt; 
\verse et possederunt omnem locum consilio suo et patientia — et locus erat longe distans ab eis — et reges, qui supervenerant eis ab extremis terrae, donec contriverunt eos et percusserunt eos plaga magna; ceteri autem dant eis tributum omnibus annis; 
\verse et Philippum et Persea Citiorum regem et, quotquot adversum eos arma tulerant, contriverunt in bello et obtinuerunt eos;  
\verse et Antiochum magnum regem Asiae, qui eis pugnam intulerat habens centum viginti elephantos et equitatum et currus et exercitum magnum valde, contritum ab eis, 
\verse et ceperunt eum vivum et statuerunt, ut eis daret ipse et, qui regnarent post ipsum, tributum magnum et daret obsides et constitutum; 
\verse et regionem Indorum et Mediam et Lydiam et de optimis regionibus eorum et acceptas eas ab illo dederunt Eumeni regi; 
\verse et quia, qui erant de Hellade, voluerunt ire et tollere eos, et innotuit sermo his, 
\verse et miserunt ad eos ducem unum et pugnaverunt contra illos et ceciderunt ex eis multi et captivas duxerunt uxores eorum et filios et diripuerunt eos et terram eorum possederunt et destruxerunt munitiones eorum et in servitutem illos redegerunt usque in hunc diem; 
\verse et residua regna et insulas, quae aliquando restiterant illis, exterminaverunt et in potestatem redegerunt; 
\verse cum amicis autem suis et, qui in ipsis requiem habebant, conservaverunt amicitiam; et obtinuerunt reges, qui prope et qui longe erant; et, quicumque audiebant nomen eorum, timebant eos;  
\verse quibus vero vellent auxilio esse et regnare, regnabant; quos autem vellent, amovebant; et exaltati sunt valde. 
\verse Et in omnibus istis nemo portabat diadema nec induebatur purpura, ut magnificaretur in ea; 
\verse et curiam fecerunt sibi, et cotidie consulebant trecenti et viginti consulentes semper de multitudine, ut quiete agerent; 
\verse et committunt uni homini regnare eis per singulos annos et dominari universae terrae suae, et omnes oboediunt uni, et non est invidia neque zelus inter eos. 
\verse Et elegit Iudas Eupolemum filium Ioannis filii Accos et Iasonem filium Eleazari et misit eos Romam constituere cum illis amicitiam et societatem 
\verse et ut auferrent ab eis iugum, quia viderunt quod regnum Graecorum in servitutem premeret Israel. 
\verse Et abierunt Romam — et via multa valde — et introierunt curiam et responderunt et dixerunt: 
\verse “Iudas Maccabaeus et fratres eius et populus Iudaeorum miserunt nos ad vos statuere vobiscum societatem et pacem et conscribere nos socios et amicos vestros". 
\verse Et placuit sermo in conspectu eorum. 
\verse Et hoc est rescriptum epistulae, quam rescripserunt in tabulis aereis et miserunt in Ierusalem, ut esset apud eos ibi memoriale pacis et societatis: 
\verse “Bene sit Romanis et genti Iudaeorum in mari et in terra in aeternum, gladiusque et hostis procul sit ab eis. 
\verse Quod si institerit bellum Romanis prius aut omnibus sociis eorum in omni dominatione eorum, 
\verse auxilium feret gens Iudaeorum, prout tempus dictaverit illis, corde pleno 
\verse et proeliantibus non dabunt neque subministrabunt triticum, arma, argentum, naves, sicut placuit Romae; et custodient mandata eorum, nihil accipientes. 
\verse Similiter autem et si genti Iudaeorum prius acciderit bellum, adiuvabunt Romani ex animo, prout eis tempus permiserit; 
\verse et adiuvantibus non dabitur triticum, arma, argentum, naves, sicut placuit Romae; et custodient mandata haec absque dolo. 
\verse Secundum haec verba ita constituerunt Romani populo Iudaeorum. 
\verse Quod si post haec verba cogitaverint hi aut illi addere aut demere, facient ex proposito suo; et, quaecumque addiderint vel dempserint, rata erunt. 
\verse Et de malis, quae Demetrius rex fecit in eos, scripsimus ei dicentes: “Quare gravasti iugum tuum super amicos nostros, socios Iudaeos?  
\verse Si ergo iterum adierint nos adversum te, faciemus illis iudicium et pugnabimus tecum mari terraque”". 
\end{biblechapter}

\begin{biblechapter}  
\verse Et audivit Demetrius quia cecidit Nicanor et exercitus eius in proelio et apposuit Bacchidem et Alcimum rursum mittere in terram Iudaeae et dextrum cornu cum illis. 
\verse Et abierunt viam, quae ducit in Galgala, et castra posuerunt in Masaloth, quae est in Arbelis, et occupaverunt eam et peremerunt animas hominum multas. 
\verse Et mense primo anni centesimi et quinquagesimi secundi applicuerunt ad Ierusalem; 
\verse et surrexerunt et abierunt in Bereth in viginti milibus virorum et duobus milibus equitum. 
\verse Et Iudas posuerat castra in Elasa, et tria milia viri electi cum eo; 
\verse et viderunt multitudinem exercitus quia multi sunt et timuerunt valde; et multi subtraxerunt se de castris, et non remanserunt ex eis nisi octingenti viri. 
\verse Et vidit Iudas quod defluxit exercitus suus, et bellum perurgebat eum; et confractus est corde, quia non habebat tempus congregandi eos, 
\verse et dissolutus est. Et dixit his, qui residui erant: “Surgamus et ascendamus ad adversarios nostros, si poterimus pugnare adversus eos". 
\verse Et avertebant eum dicentes: “Non poterimus, sed liberemus animas nostras modo et revertamur nos et fratres nostri et pugnabimus adversus eos; nos autem pauci sumus". 
\verse Et ait Iudas: “Absit istam rem facere, ut fugiamus ab eis; et si appropiavit tempus nostrum, et moriamur in virtute propter fratres nostros et non inferamus crimen gloriae nostrae". 
\verse Et movit exercitus de castris, et steterunt illis obviam; et divisi sunt equites in duas partes, et fundibularii et sagittarii praeibant exercitum, et primi certaminis omnes potentes. 
\verse Bacchides autem erat in dextro cornu; et proximavit legio ex duabus partibus, et clamabant tubis; et clamaverunt hi, qui erant ex parte Iudae, etiam ipsi in tubis; 
\verse et commota est terra a voce exercituum; et commissum est proelium a mane usque ad vesperam. 
\verse Et vidit Iudas quod Bacchides et firmior pars exercitus erat in dextris, et convenerunt cum ipso omnes constantes corde; 
\verse et contrita est dextera pars ab eis, et persecutus est eos usque ad montem Azoti. 
\verse Et, qui in sinistro cornu erant, viderunt quod contritum est dextrum cornu, et secuti sunt post Iudam et eos, qui cum ipso erant, a tergo. 
\verse Et ingravatum est proelium, et ceciderunt vulnerati multi ex his et ex illis; 
\verse et Iudas cecidit, et ceteri fugerunt. 
\verse Et Ionathas et Simon tulerunt Iudam fratrem suum et sepelierunt eum in sepulcro patrum suorum in Modin. 
\verse Et fleverunt eum et planxerunt omnis populus Israel planctu magno et lugebant dies multos 
\verse et dixerunt: “Quomodo cecidit potens, qui salvum faciebat populum Israel!". 
\verse Et cetera verborum Iudae et bellorum et virtutum, quas fecit, et magnitudinis eius non sunt descripta; multa enim erant valde. 
\verse Et factum est, post obitum Iudae emerserunt iniqui in omnibus finibus Israel, et exorti sunt omnes, qui operabantur iniquitatem. 
\verse In diebus illis facta est fames magna valde, et tradidit se regio cum ipsis. 
\verse Et elegit Bacchides viros impios et constituit eos dominos regionis; 
\verse et exquirebant et perscrutabantur amicos Iudae et adducebant eos ad Bacchidem, et vindicabat in illos et illudebat. 
\verse Et facta est tribulatio magna in Israel, qualis non fuit ex die, qua non est visus propheta illis. 
\verse Et congregati sunt omnes amici Iudae et dixerunt Ionathae: 
\verse “Ex quo frater tuus Iudas defunctus est, et vir similis ei non est, qui exeat contra inimicos et Bacchidem et eos, qui inimici sunt gentis nostrae; 
\verse nunc itaque te hodie elegimus esse pro eo nobis in principem et ducem ad bellandum bellum nostrum". 
\verse Et suscepit Ionathas tempore illo principatum et surrexit loco Iudae fratris sui. 
\verse Et cognovit Bacchides et quaerebat eum occidere; 
\verse et cognovit Ionathas et Simon frater eius et omnes, qui cum eo erant, et fugerunt in desertum Thecue et consederunt ad aquam lacus Asphar. 
\verse Et cognovit Bacchides die sabbatorum et venit ipse et omnis exercitus eius trans Iordanem. 
\verse Et Ionathas misit fratrem suum ducem populi et rogavit Nabathaeos amicos suos, ut commodarent illis apparatum suum, qui erat copiosus. 
\verse Et exierunt filii Iambri ex Medaba et comprehenderunt Ioannem et omnia, quae habebat, et abierunt habentes ea. 
\verse Post haec verba renuntiatum est Ionathae et Simoni fratri eius quia filii Iambri faciunt nuptias magnas et ducunt sponsam ex Nadabath filiam unius de magnis principibus Chanaan cum ambitione magna. 
\verse Et recordati sunt sanguinis Ioannis fratris sui et ascenderunt et absconderunt se sub tegumento montis; 
\verse et elevaverunt oculos suos et viderunt: et ecce tumultus et apparatus multus, et sponsus processit et amici eius et fratres eius obviam illis cum tympanis et musicis et armis multis. 
\verse Et surrexerunt ad eos ex insidiis et occiderunt eos, et ceciderunt vulnerati multi; et residui fugerunt in montes, et acceperunt omnia spolia eorum. 
\verse Et conversae sunt nuptiae in luctum, et vox musicorum ipsorum in lamentum. 
\verse Et vindicaverunt vindictam sanguinis fratris sui et reversi sunt ad ripam Iordanis. 
\verse Et audivit Bacchides et venit die sabbatorum usque ad oram Iordanis in virtute magna. 
\verse Et dixit ad suos Ionathas: “Surgamus et pugnemus pro animabus nostris; non est enim hodie sicut heri et nudiustertius: 
\verse ecce enim bellum ex adverso nostrum, aqua vero Iordanis hinc et inde, et paludes et saltus, et non est locus divertendi. 
\verse Nunc ergo clamate in caelum, ut liberemini de manu inimicorum vestrorum". Et commissum est bellum. 
\verse Et extendit Ionathas manum suam percutere Bacchidem, et divertit ab eo retro.  
\verse Et dissiliit Ionathas et, qui cum eo erant, in Iordanem et transnataverunt in ulteriora; et non transierunt ad eos Iordanem. 
\verse Et ceciderunt de parte Bacchidis die illa mille viri. Et reversi sunt in Ierusalem. 
\verse Et aedificaverunt civitates munitas in Iudaea: munitionem, quae erat in Iericho, et Emmaus et Bethoron et Bethel et Thamnata et Pharathon et Tephon muris excelsis et portis et seris; 
\verse et posuit custodiam in eis, ut inimicitias exercerent in Israel. 
\verse Et munivit civitatem Bethsuram et Gazaram et arcem et posuit in eis auxilia et apparatum escarum. 
\verse Et accepit filios principum regionis obsides et posuit eos in arce in Ierusalem in custodia. 
\verse Et anno centesimo quinquagesimo tertio, mense secundo, praecepit Alcimus destrui murum atrii sanctorum interioris et destruxit opera prophetarum. Et coepit destruere. 
\verse In tempore illo percussus est Alcimus, et impedita sunt opera illius; et occlusum est os eius, et dissolutus est nec ultra poterat loqui verbum et mandare de domo sua; 
\verse et mortuus est Alcimus in tempore illo cum tormento magno. 
\verse Et vidit Bacchides quoniam mortuus est Alcimus et reversus est ad regem; et siluit terra Iudae annis duobus. 
\verse Et cogitaverunt omnes iniqui dicentes: “Ecce Ionathas et, qui cum eo sunt, in silentio habitant confidentes; nunc ergo adducamus Bacchidem, et comprehendet eos omnes una nocte". 
\verse Et abierunt et consilium ei dederunt. 
\verse Et surrexit, ut veniret cum exercitu multo, et misit occulte epistulas sociis suis, qui erant in Iudaea, ut comprehenderent Ionathan et eos, qui cum eo erant; et non potuerunt, quia innotuit eis consilium eorum. 
\verse Et apprehenderunt de viris regionis, qui principes erant malitiae, quinquaginta viros et occiderunt eos. 
\verse Et secessit Ionathas et Simon et, qui cum eo erant, in Bethbasi, quae est in deserto; et exstruxit diruta eius, et firmaverunt eam. 
\verse Et cognovit Bacchides et congregavit universam multitudinem suam et his, qui de Iudaea erant, denuntiavit; 
\verse et venit et castra posuit ad Bethbasi et oppugnavit eam dies multos et fecit machinas. 
\verse Et reliquit Ionathas Simonem fratrem suum in civitate et exiit in regionem et venit cum numero;  
\verse et percussit Odomera et fratres eius et filios Phasiron in tabernaculo ipsorum et coepit caedere et crescere in virtutibus. 
\verse Simon vero et, qui cum ipso erant, exierunt de civitate et succenderunt machinas 
\verse et pugnaverunt contra Bacchidem, et contritus est ab eis, et afflixerunt eum valde, quoniam consilium eius et adventus eius erat inanis. 
\verse Et iratus est animo contra viros iniquos, qui ei consilium dederant, ut veniret in regionem, et multos ex eis occiderunt; et cogitaverunt abire in regionem eius. 
\verse Et cognovit Ionathas et misit ad eum legatos componere pacem cum ipso et reddere ei captivitatem. 
\verse Et accepit et fecit secundum verba eius et iuravit se nihil facturum ei mali omnibus diebus vitae eius; 
\verse et reddidit ei captivitatem, quam prius erat praedatus de terra Iudae, et conversus abiit in terram suam et non apposuit amplius venire in fines eius. 
\verse Et cessavit gladius ex Israel, et habitavit Ionathas in Machmas; et coepit Ionathas ibi iudicare populum et exterminabat impios ex Israel. 
\end{biblechapter}

\begin{biblechapter}  
\verse Et anno centesimo sexagesimo ascendit Alexander Antiochi filius, Epiphanes, et occupavit Ptolemaidam; et receperunt eum, et regnavit illic. 
\verse Et audivit Demetrius rex et congregavit exercitum copiosum valde et exivit obviam illi in proelium. 
\verse Et misit Demetrius epistulam ad Ionathan verbis pacificis, ut magnificaret eum. 
\verse Dixit enim: “Anticipemus facere pacem cum eo, priusquam faciat cum Alexandro adversum nos; 
\verse recordabitur enim omnium malorum, quae consummavimus in eum et in fratrem eius et in gentem eius". 
\verse Et dedit ei potestatem congregare exercitum et fabrificare arma et esse ipsum socium eius; et obsides, qui erant in arce, dixit tradi ei. 
\verse Et venit Ionathas in Ierusalem et legit epistulas in auditu omnis populi et eorum, qui in arce erant; 
\verse et timuerunt timore magno, cum audirent quoniam dedit ei rex potestatem congregandi exercitum. 
\verse Et tradiderunt, qui erant in arce, Ionathae obsides, et reddidit eos parentibus ipsorum. 
\verse Et habitavit Ionathas in Ierusalem et coepit aedificare et innovare civitatem; 
\verse et dixit facientibus opera, ut exstruerent muros et montem Sion in circuitu lapidibus quadratis ad munitionem: et ita fecerunt. 
\verse Et fugerunt alienigenae, qui erant in munitionibus, quas Bacchides aedificaverat, 
\verse et reliquit unusquisque locum suum et abiit in terram suam; 
\verse tantum in Bethsura remanserunt aliqui ex his, qui reliquerant legem et praecepta; erat enim ad refugium. 
\verse Et audivit Alexander rex promissa, quae misit Demetrius Ionathae, et narraverunt ei proelia et virtutes, quas ipse fecit et fratres eius, et labores, quos habuerunt. 
\verse Et ait: “Numquid inveniemus aliquem virum talem? Et nunc faciemus eum amicum et socium nostrum". 
\verse Et scripsit epistulam et misit ei secundum haec verba dicens: 
\verse “Rex Alexander fratri Ionathae salutem.  
\verse Audivimus de te quod vir potens es viribus et aptus es, ut sis amicus noster;  
\verse et nunc constituimus te hodie summum sacerdotem gentis tuae, et ut amicus voceris regis — et misit ei purpuram et coronam auream — et, quae nostra sunt, sentias nobiscum et conserves amicitias ad nos". 
\verse Et induit se Ionathas stola sancta septimo mense, anno centesimo sexagesimo, in die sollemni Scenopegiae; et congregavit exercitum et fecit arma copiosa. 
\verse Et audivit Demetrius verba ista et contristatus est nimis et ait: 
\verse “Quid hoc fecimus quod praeoccupavit nos Alexander apprehendere amicitiam Iudaeorum ad firmamentum? 
\verse Scribam et ego illis verba deprecatoria et dignitates et dona, ut sint mecum in adiutorium". 
\verse Et misit eis secundum haec verba: “Rex Demetrius genti Iudaeorum salutem. 
\verse Quoniam servastis ad nos pactum et mansistis in amicitia nostra et non accessistis ad inimicos nostros, audivimus et gavisi sumus. 
\verse Et nunc perseverate adhuc conservare ad nos fidem, et retribuemus vobis bona pro his, quae facitis nobiscum, 
\verse et remittemus vobis praestationes multas et dabimus vobis donationes. 
\verse Et nunc absolvo vos et remitto Iudaeos a tributis et pretio salis et a coronis;  
\verse et pro tertia parte seminis et pro dimidia parte fructus ligni, quod debetur mihi accipere, remitto ex hodierno die et deinceps accipere a terra Iudae et a tribus regionibus, quae additae sunt illi ex Samaritide et Galilaea, ex hodierna die et in totum tempus; 
\verse et Ierusalem sit sancta et libera et fines eius et decimae et tributa.  
\verse Remitto etiam potestatem arcis, quae est in Ierusalem, et do eam summo sacerdoti, ut constituat in ea viros, quoscumque ipse elegerit, qui custodiant eam. 
\verse Et omnem animam Iudaeorum, quae captiva est a terra Iudae in omni regno meo, relinquo liberam gratis, et omnes a tributis solvantur etiam pecorum suorum. 
\verse Et omnes dies sollemnes et sabbata et neomeniae et dies decreti et tres dies ante diem sollemnem et tres dies post diem sollemnem sint omnes immunitatis et remissionis omnibus Iudaeis, qui sunt in regno meo; 
\verse et nemo habebit potestatem agere et perturbare aliquem illorum de omni causa.  
\verse Et ascribantur ex Iudaeis in exercitu regis ad triginta milia virorum, et dabuntur illis copiae, ut oportet omnibus exercitibus regis. 
\verse Et ex eis constituentur, qui sint in munitionibus regis magnis, et ex his constituentur super negotia regni, quae aguntur ex fide, et praepositi eorum et principes sint ex eis et ambulent in legibus suis, sicut praecepit rex in terra Iudae. 
\verse Et tres regiones, quae additae sunt Iudaeae ex regione Samariae, addatur Iudaeae reputari, ut sint sub uno et non oboediant alii potestati, nisi summi sacerdotis. 
\verse Ptolemaida et confines eius dedi donum sanctis, quae sunt in Ierusalem, ad necessarios sumptus sanctis. 
\verse Et ego do singulis annis quindecim milia siclorum argenti de rationibus regis ex locis, quae me contingunt; 
\verse et omne, quod reliquum fuerit, quod non reddiderant, qui super negotia erant, annis prioribus, ex hoc dabunt in opera domus. 
\verse Et super haec quinque milia siclorum argenti, quanta accipiebant de sanctorum ratione per singulos annos, et haec remittuntur eo quod ipsa ad sacerdotes pertineant, qui ministerio funguntur. 
\verse Et quicumque confugerint in templum, quod est Hierosolymis, et in omnibus finibus eius debentes regalia et quamlibet rem, dimittantur, et universa, quae sunt eis in regno meo. 
\verse Et ad aedificanda vel restauranda opera sanctorum sumptus dabitur de ratione regis;  
\verse et ad exstruendos muros Ierusalem et communiendum in circuitu sumptus dabitur de ratione regis et ad construendos muros in Iudaea". 
\verse Ut audivit autem Ionathas et populus sermones istos, non crediderunt eis nec receperunt, quia recordati sunt malitiae magnae, quam fecerat in Israel et tribulaverat eos valde. 
\verse Et complacuit eis in Alexandro, quia ipse fuerat eis princeps sermonum pacis, et ipsi auxilium ferebant omnibus diebus. 
\verse Et congregavit rex Alexander exercitum magnum et admovit castra contra Demetrium. 
\verse Et commiserunt proelium duo reges, et fugit exercitus Alexandri, et insecutus est eum Demetrius et praevaluit adversus eos; 
\verse et confirmavit proelium nimis, donec occidit sol, et cecidit Demetrius in die illa. 
\verse Et misit Alexander ad Ptolemaeum regem Aegypti legatos secundum haec verba dicens: 
\verse “Quoniam regressus sum in regnum meum et sedi in sede patrum meorum et obtinui principatum et contrivi Demetrium et possedi regionem nostram 
\verse et commisi pugnam cum eo, et contritus est ipse et castra eius a nobis, et sedimus in sede regni eius; 
\verse et nunc statuamus ad invicem amicitiam, et da mihi filiam tuam uxorem, et ego ero gener tuus et dabo tibi dona et ipsi digna te". 
\verse Et respondit rex Ptolemaeus dicens: “Felix dies, in qua reversus es ad terram patrum tuorum et sedisti in sede regni eorum! 
\verse Et nunc faciam tibi, quae scripsisti, sed occurre in Ptolemaidam, ut videamus invicem nos, et socer fiam tibi, sicut dixisti". 
\verse Et exivit Ptolemaeus de Aegypto, ipse et Cleopatra filia eius, et venit Ptolemaidam anno centesimo sexagesimo secundo. 
\verse Et occurrit ei Alexander rex, et dedit ei Cleopatram filiam suam et fecit nuptias eius Ptolemaidae sicut reges in magna gloria. 
\verse Et scripsit rex Alexander Ionathae, ut veniret obviam sibi. 
\verse Et abiit cum gloria Ptolemaidam et occurrit ibi duobus regibus et dedit illis argentum multum et aurum et dona et invenit gratiam in conspectu eorum. 
\verse Et convenerunt adversus eum viri pestilentes ex Israel, viri iniqui interpellantes adversus eum; et non intendit ad eos rex. 
\verse Et iussit rex, et exspoliaverunt Ionathan vestibus suis et induerunt eum purpura; et ita fecerunt. Et collocavit eum rex sedere secum. 
\verse Dixitque principibus suis: “Exite cum eo in medium civitatis et praedicate, ut nemo adversus eum interpellet de ullo negotio, nec quisquam ei molestus sit de ulla ratione". 
\verse Et factum est, ut viderunt, qui interpellabant gloriam eius, quae praedicabatur, et opertum eum purpura, fugerunt omnes. 
\verse Et magnificavit eum rex et scripsit eum inter primos amicos et posuit eum ducem et participem principatus. 
\verse Et reversus est Ionathas in Ierusalem cum pace et laetitia. 
\verse In anno centesimo sexagesimo quinto venit Demetrius filius Demetrii a Creta in terram patrum suorum; 
\verse et audivit Alexander rex et contristatus est valde et reversus est Antiochiam. 
\verse Et constituit Demetrius rex Apollonium, qui praeerat Coelesyriae, et congregavit exercitum magnum; et accessit ad Iamniam et misit ad Ionathan summum sacerdotem 
\verse dicens: “Tu omnino solus resistis nobis; ego autem factus sum in derisum et in opprobrium propter te; et quare tu potestatem adversum nos exerces in montibus? 
\verse Nunc ergo si confidis in virtutibus tuis, descende ad nos in campum, et comparemus illic invicem, quia mecum est virtus civitatum. 
\verse Interroga et disce quis sum ego et ceteri, qui auxilio sunt nobis, et dicunt: “Non potest stare pes vester ante faciem nostram, quia bis in fugam conversi sunt patres tui in terra sua”. 
\verse Et nunc non poteris sustinere equitatum et exercitum talem in campo, ubi non est lapis neque silex neque locus fugiendi". 
\verse Ut audivit autem Ionathas sermones Apollonii, motus est animo et elegit decem milia virorum et exiit ab Ierusalem, et occurrit ei Simon frater eius in adiutorium. 
\verse Et applicuit castra in Ioppen; et excluserunt eum, qui erant de civitate, quia custodia Apollonii in Ioppe erat, et oppugnaverunt eam. 
\verse Et exterriti, qui erant in civitate, aperuerunt ei, et obtinuit Ionathas Ioppen.  
\verse Et audivit Apollonius et admovit tria milia equitum et exercitum multum. Et abiit Azotum tamquam iter faciens et statim exiit in campum, eo quod haberet multitudinem equitum et confideret in eis. 
\verse Et insecutus est eum Ionathas in Azotum, et exercitus commiserunt proelium. 
\verse Et reliquit Apollonius mille equites post eos occulte. 
\verse Et cognovit Ionathas quoniam insidiae sunt post se; et circuierunt castra eius et iecerunt iacula in populum a mane usque ad vesperam. 
\verse Populus autem stabat, sicut praeceperat Ionathas, et laboraverunt equi eorum. 
\verse Et eiecit Simon exercitum suum et commisit contra legionem; equites enim fatigati erant. Et contriti sunt ab eo et fugerunt, 
\verse et equi dispersi sunt in campo et fugerunt in Azotum et intraverunt in Bethdagon idolum suum, ut ibi se liberarent. 
\verse Et succendit Ionathas Azotum et civitates, quae erant in circuitu eius, et accepit spolia eorum et templum Dagon et omnes, qui fugerunt in illud, succendit igne. 
\verse Et fuerunt, qui ceciderunt gladio cum his qui succensi sunt, fere octo milia virorum. 
\verse Et movit inde Ionathas castra et applicuit ad Ascalonem, et exierunt de civitate obviam illi in magna gloria. 
\verse Et reversus est Ionathas in Ierusalem cum suis habentibus spolia multa. 
\verse Et factum est, ut audivit Alexander rex sermones istos, et addidit adhuc glorificare Ionathan  
\verse et misit ei fibulam auream, sicut consuetudo est dari cognatis regum, et dedit ei Accaron et omnes fines eius in possessionem. 
\end{biblechapter}

\begin{biblechapter}  
\verse Et rex Aegypti congregavit exercitum sicut arena, quae est circa oram maris, et naves multas et quaerebat obtinere regnum Alexandri dolo et addere illud regno suo. 
\verse Et exiit in Syriam verbis pacificis, et aperiebant ei civitates et occurrebant ei, quia mandaverat Alexander rex exire ei obviam, eo quod socer suus esset. 
\verse Cum autem introiret civitatem, Ptolemaeus ponebat custodias militum in singulis civitatibus. 
\verse Et, ut appropiavit Azoto, ostenderunt ei templum Dagon succensum et Azotum et suburbana eius demolita et corpora proiecta et combustos, quos combusserat in bello; fecerant enim tumulos eorum in via eius. 
\verse Et narraverunt regi, quae fecit Ionathas, ut vituperarent eum; et tacuit rex. 
\verse Et occurrit Ionathas regi in Ioppen cum gloria, et invicem se salutaverunt et dormierunt illic. 
\verse Et abiit Ionathas cum rege usque ad fluvium, qui vocatur Eleutherus, et reversus est in Ierusalem.  
\verse Rex autem Ptolemaeus obtinuit dominium civitatum maritimarum usque Seleuciam maritimam et cogitabat in Alexandrum consilia mala. 
\verse Et misit legatos ad Demetrium dicens: “Veni, componamus inter nos pactum, et dabo tibi filiam meam, quam habet Alexander, et regnabis in regno patris tui; 
\verse paenitet enim me quod dederim illi filiam meam: quaesivit enim me occidere". 
\verse Et vituperavit eum, propterea quod concupisceret regnum eius. 
\verse Et abstulit filiam suam et dedit eam Demetrio et alienavit se ab Alexandro, et manifestatae sunt inimicitiae eorum. 
\verse Et intravit Ptolemaeus Antiochiam et imposuit duo diademata capiti suo, Aegypti et Asiae. 
\verse Alexander autem rex erat in Cilicia illis temporibus, quia rebellabant, qui erant de locis illis; 
\verse et audivit Alexander et venit ad eum in bello. Et produxit Ptolemaeus exercitum et occurrit ei in manu valida et fugavit eum. 
\verse Et fugit Alexander in Arabiam, ut ibi protegeretur; rex autem Ptolemaeus exaltatus est. 
\verse Et abstulit Zabdiel Arabs caput Alexandri et misit Ptolemaeo. 
\verse Et rex Ptolemaeus mortuus est in die tertia; et qui erant in munitionibus, perierunt ab his, qui erant in munitionibus. 
\verse Et regnavit Demetrius anno centesimo sexagesimo septimo. 
\verse In diebus illis congregavit Ionathas eos, qui erant de Iudaea, ut expugnarent arcem, quae est in Ierusalem; et fecerunt contra eam machinas multas. 
\verse Et abierunt quidam, qui oderant gentem suam, viri iniqui ad regem et renuntiaverunt ei quod Ionathas obsideret arcem. 
\verse Et audiens iratus est et statim, ut audivit, movit castra et venit ad Ptolemaidam et scripsit Ionathae, ne obsideret arcem et ut occurreret sibi ad colloquium in Ptolemaidam festinato. 
\verse Ut audivit autem Ionathas, iussit obsidere et elegit de senioribus Israel et de sacerdotibus et dedit se periculo 
\verse et accepit argentum et aurum et vestem et alia xenia multa et abiit ad regem in Ptolemaidam et invenit gratiam in conspectu eius. 
\verse Et interpellabant adversus eum quidam iniqui ex gente sua. 
\verse Et fecit ei rex, sicut fecerant ei qui ante eum fuerant, et exaltavit eum in conspectu omnium amicorum suorum 
\verse et statuit ei principatum sacerdotii et, quaecumque alia habuit prius pretiosa, et fecit eum principem primorum amicorum. 
\verse Et postulavit Ionathas a rege, ut immunem faceret Iudaeam et tres toparchias et Samaritidem, et promisit ei talenta trecenta. 
\verse Et consensit rex et scripsit Ionathae epistulas de his omnibus hunc modum continentes: 
\verse “Rex Demetrius fratri Ionathae salutem et genti Iudaeorum. 
\verse Exemplum epistulae, quam scripsimus Lastheni parenti nostro de vobis, scripsimus et ad vos, ut sciretis: 
\verse “Rex Demetrius Lastheni patri salutem. 
\verse Genti Iudaeorum amicis nostris et conservantibus, quae iusta sunt apud nos, decrevimus benefacere propter benignitatem ipsorum, quam erga nos habent. 
\verse Statuimus ergo illis fines Iudaeae et tres regiones, Apherema et Lydda et Ramathaim, quae additae sunt Iudaeae ex Samaritide, et omnia confinia earum, omnibus sacrificantibus in Hierosolymis, pro regalibus, quae ab eis prius accipiebat rex per singulos annos de fructibus terrae et pomorum; 
\verse et alia, quae ad nos pertinent ex hoc tempore decimarum et tributorum pertinentium ad nos, et salis stagna et pertinentes ad nos coronas, omnia ipsis concedimus. 
\verse Et nihil horum irritum erit ex hoc et in omne tempus. 
\verse Nunc ergo curate facere horum exemplum, et detur Ionathae et ponatur in monte sancto in loco celebri”". 
\verse Et videns Demetrius rex quod siluit terra in conspectu suo, et nihil ei resistit, dimisit totum exercitum suum, unumquemque in locum suum, excepto peregrino exercitu, quem contraxit ab insulis gentium; et inimici erant ei omnes exercitus patrum eius. 
\verse Tryphon autem erat quidam partium Alexandri prius et vidit quoniam omnis exercitus murmurabat contra Demetrium et ivit ad Imalcue Arabem, qui nutriebat Antiochum filium Alexandri; 
\verse et assidebat ei, ut traderet eum ipsi, ut regnaret loco patris sui. Et enuntiavit ei quanta constituerat Demetrius et inimicitias exercituum eius adversus illum; et mansit ibi diebus multis. 
\verse Et misit Ionathas ad Demetrium regem, ut eiceret eos, qui in arce erant in Ierusalem et qui in praesidiis erant, quia impugnabant Israel. 
\verse Et misit Demetrius ad Ionathan dicens: “Non haec tantum faciam tibi et genti tuae, sed gloria illustrabo te et gentem tuam, cum fuerit opportunum; 
\verse nunc ergo recte feceris, si miseris in auxilium mihi viros, quia discessit omnis exercitus meus". 
\verse Et misit ei Ionathas tria milia virorum fortium Antiochiam, et venerunt ad regem, et delectatus est rex in adventu eorum. 
\verse Et convenerunt, qui erant de civitate, centum viginti milia virorum, et volebant interficere regem; 
\verse et fugit rex in aulam, et occupaverunt, qui erant de civitate, itinera civitatis et coeperunt pugnare. 
\verse Et vocavit rex Iudaeos in auxilium, et convenerunt omnes simul ad eum et dispersi sunt per civitatem; et occiderunt in illa die centum milia hominum. 
\verse Et succenderunt civitatem et ceperunt spolia multa in die illa et liberaverunt regem. 
\verse Et viderunt, qui erant de civitate, quod obtinuissent Iudaei civitatem, sicut volebant, et infirmati sunt mente sua et clamaverunt ad regem cum precibus dicentes: 
\verse “Da nobis dextras, et cessent Iudaei oppugnare nos et civitatem". 
\verse Et proiecerunt arma sua et fecerunt pacem; et glorificati sunt Iudaei in conspectu regis et in conspectu omnium, qui erant in regno eius, et nominati sunt in regno et regressi sunt in Ierusalem habentes spolia multa. 
\verse Et sedit Demetrius rex in sede regni sui, et siluit terra in conspectu eius. 
\verse Et mentitus est omnia, quaecumque dixit, et abalienavit se a Ionatha et non retribuit ei secundum beneficia, quae sibi tribuerat, et vexabat eum valde. 
\verse Post haec autem reversus est Tryphon, et Antiochus cum eo puer adulescentior; et regnavit et imposuit sibi diadema. 
\verse Et congregati sunt ad eum omnes exercitus, quos disperserat Demetrius, et pugnaverunt contra eum, et fugit et terga vertit. 
\verse Et accepit Tryphon bestias et obtinuit Antiochiam. 
\verse Et scripsit Antiochus adulescentior Ionathae dicens: “Constituo tibi summum sacerdotium et constituo te super quattuor regiones et ut sis de amicis regis".  
\verse Et misit illi vasa aurea et ministerium et dedit ei potestatem bibendi in auro et esse in purpura et habere fibulam auream. 
\verse Et Simonem fratrem eius constituit ducem a descensu Tyri usque ad fines Aegypti. 
\verse Et exiit Ionathas et perambulabat trans flumen et in civitatibus, et congregatus est ad eum omnis exercitus Syriae in auxilium; et venit Ascalonem, et occurrerunt ei honorifice de civitate. 
\verse Et abiit inde Gazam, et concluserunt, qui erant Gazae; et obsedit eam et succendit, quae erant in circuitu civitatis, et praedatus est ea. 
\verse Et rogaverunt Gazenses Ionathan, et dedit illis dexteram et accepit filios principum eorum obsides et misit illos in Ierusalem; et perambulavit regionem usque Damascum. 
\verse Et audivit Ionathas quod aderant principes Demetrii in Cades, quae est in Galilaea, cum exercitu multo volentes eum removere a negotio; 
\verse et occurrit illis, fratrem autem suum Simonem reliquit in regione. 
\verse Et applicuit Simon ad Bethsuram et expugnabat eam diebus multis et conclusit eos. 
\verse Et postulaverunt ab eo dextras accipere, et dedit illis; et eiecit eos inde et cepit civitatem et posuit in ea praesidium. 
\verse Et Ionathas et castra eius applicuerunt ad aquam Gennesar et ante lucem vigilaverunt in campo Asor. 
\verse Et ecce castra alienigenarum occurrebant ei in campo et tendebant ei insidias in montibus; ipsi autem occurrerunt ex adverso. 
\verse Insidiae vero exsurrexerunt de locis suis et commiserunt proelium. 
\verse Et fugerunt, qui erant ex parte Ionathae omnes, et nemo relictus est ex eis, nisi Matthathias filius Absalomi et Iudas filius Chalphi princeps militiae exercitus. 
\verse Et scidit Ionathas vestimenta sua et posuit terram in capite suo et oravit. 
\verse Et reversus est ad eos in proelium et convertit eos in fugam; et fugerunt. 
\verse Et viderunt, qui fugiebant partis illius, et reversi sunt ad eum et insequebantur cum eo usque Cades, usque ad castra ipsorum, et applicuerunt illic. 
\verse Et ceciderunt de alienigenis in die illa tria milia virorum; et reversus est Ionathas in Ierusalem. 
\end{biblechapter}

\begin{biblechapter}  
\verse Et vidit Ionathas quia tempus eum adiuvat; et elegit viros et misit Romam statuere et renovare cum eis amicitiam; 
\verse et ad Spartiatas et ad alia loca misit epistulas secundum eadem. 
\verse Et abierunt Romam et intraverunt curiam et dixerunt: “Ionathas summus sacerdos et gens Iudaeorum miserunt nos renovare amicitiam et societatem secundum pristinum”. 
\verse Et dederunt illis epistulas ad ipsos per loca, ut deducerent eos in terram Iudae cum pace. 
\verse Et hoc est exemplum epistularum, quas scripsit Ionathas Spartiatis: 
\verse “Ionathas summus sacerdos et seniores gentis et sacerdotes et reliquus populus Iudaeorum Spartiatis fratribus salutem. 
\verse Iampridem missae erant epistulae ad Oniam summum sacerdotem ab Ario, qui regnabat apud vos, quoniam estis fratres nostri, sicut rescriptum continet, quod subiectum est. 
\verse Et suscepit Onias virum, qui missus fuerat, cum honore; et accepit epistulas, in quibus significabatur de societate et amicitia. 
\verse Nos igitur, cum nullo horum indigeremus, exhortationem habentes sanctos libros, qui sunt in manibus nostris,  
\verse tentavimus mittere ad vos renovare fraternitatem et amicitiam, ne forte alieni efficiamur a vobis; multa enim tempora transierunt, ex quo misistis ad nos. 
\verse Nos ergo in omni tempore sine intermissione in diebus sollemnibus et ceteris, quibus oportet, memores sumus vestri in sacrificiis, quae offerimus, et in obsecrationibus, sicut fas est et decet meminisse fratrum. 
\verse Laetamur itaque de gloria vestra. 
\verse Nos autem circumdederunt multae tribulationes et multa proelia; et impugnaverunt nos reges, qui sunt in circuitu nostro. 
\verse Noluimus ergo vobis molesti esse et ceteris sociis et amicis nostris in his proeliis; 
\verse habemus enim de caelo auxilium, quod nos adiuvat, et liberati sumus nos ab inimicis nostris, et humiliati sunt inimici nostri. 
\verse Elegimus itaque Numenium Antiochi filium et Antipatrem Iasonis filium et misimus ad Romanos renovare cum eis amicitiam et societatem pristinam; 
\verse mandavimus itaque eis, ut veniant etiam ad vos et salutent vos et reddant vobis epistulas nostras de innovatione et fraternitate nostra. 
\verse Et nunc bene facietis respondentes nobis ad haec". 
\verse Et hoc est rescriptum epistularum, quas miserunt Oniae: 
\verse “Arius rex Spartiatarum Oniae sacerdoti magno salutem. 
\verse Inventum est in scriptura de Spartiatis et Iudaeis quoniam sunt fratres et quod sunt de genere Abraham.  
\verse Et nunc, ex quo haec cognovimus, bene facietis scribentes nobis de pace vestra. 
\verse Sed et nos rescribimus vobis. Pecora vestra et possessiones vestrae nostrae sunt; et, quae nostra, vestra sunt. Mandamus itaque, ut annuntient vobis secundum haec". 
\verse Et audivit Ionathas quoniam regressi sunt principes Demetrii cum exercitu multo supra quam prius pugnare adversus eum. 
\verse Et exiit ab Ierusalem et occurrit eis in Amathitem regionem; non enim dedit eis spatium, ut ingrederentur regionem eius. 
\verse Et misit speculatores in castra eorum, et reversi renuntiaverunt ei quod ita constituunt supervenire illis nocte. 
\verse Cum occidisset autem sol, praecepit Ionathas suis vigilare et esse in armis paratos ad pugnam tota nocte; et emisit custodes per circuitum castrorum. 
\verse Et audierunt adversarii quod paratus est Ionathas cum suis in bellum, et timuerunt et formidaverunt in corde suo et accenderunt focos in castris suis. 
\verse Ionathas autem et, qui cum eo erant, non cognoverunt usque mane; videbant enim luminaria ardentia. 
\verse Et insecutus est eos Ionathas et non comprehendit eos; transierant enim flumen Eleutherum. 
\verse Et divertit Ionathas ad Arabas, qui vocantur Zabadaei, et percussit eos et accepit spolia eorum. 
\verse Et iunxit et venit Damascum et perambulabat omnem regionem illam. 
\verse Simon autem exiit et venit usque ad Ascalonem et ad proxima praesidia et declinavit in Ioppen et occupavit eam; 
\verse audivit enim quod vellent praesidium tradere partibus Demetrii, et posuit ibi custodes, ut custodirent eam. 
\verse Et reversus est Ionathas et convocavit seniores populi et cogitavit cum eis aedificare praesidia in Iudaea 
\verse et altius extollere muros Ierusalem et exaltare altitudinem magnam inter medium arcis et civitatis, ut separaret eam a civitate, ut esset ipsa singulariter, ut neque emant neque vendant. 
\verse Et convenerunt, ut aedificarent civitatem; et cecidit de muro, qui erat super torrentem a subsolano, et reparavit eum, qui vocatur Chaphenatha. 
\verse Et Simon aedificavit Adida in Sephela et munivit eam et imposuit portas et seras. 
\verse Et quaesivit Tryphon regnare Asiae et imponere sibi diadema et extendere manum in Antiochum regem. 
\verse Et veritus est, ne forte non permitteret eum Ionathas et ne forte pugnaret adversus eum, et quaerebat comprehendere eum et occidere et exsurgens abiit in Bethsan. 
\verse Et exivit Ionathas obviam illi cum quadraginta milibus virorum electorum in proelium et venit Bethsan. 
\verse Et vidit Tryphon quia venit cum exercitu multo et extendere in eum manus timuit  
\verse et excepit eum cum honore et commendavit eum omnibus amicis suis et dedit ei munera et praecepit exercitibus suis, ut oboedirent ei sicut sibi. 
\verse Et dixit Ionathae: “Ut quid vexasti universum populum, cum bellum nobis non sit?  
\verse Et nunc remitte eos in domos suas. Elige autem tibi viros paucos, qui tecum sint, et veni mecum Ptolemaidam, et tradam eam tibi et reliqua praesidia et reliquum exercitum et universos praepositos negotiis et conversus abibo; propterea enim veni". 
\verse Et credidit ei et fecit, sicut dixit, et dimisit exercitum, et abierunt in terram Iudae. 
\verse Retinuit autem secum tria milia virorum, ex quibus remisit in Galilaeam duo milia; mille autem venerunt cum eo.  
\verse Ut autem intravit Ptolemaidam Ionathas, clauserunt portas Ptolemenses et comprehenderunt eum; et omnes, qui cum eo intraverant, gladio interfecerunt.  
\verse Et misit Tryphon exercitum et equites in Galilaeam et in campum magnum, ut perderent omnes socios Ionathae. 
\verse At illi, cum cognovissent quia comprehensus est Ionathas et periit, et omnes, qui cum eo erant, hortati sunt semetipsos et ibant conglobati parati in proelium. 
\verse Et videntes hi, qui insecuti fuerant, quia pro anima res est illis, reversi sunt; 
\verse illi autem venerunt omnes cum pace in terram Iudae et planxerunt Ionathan et eos, qui cum ipso fuerant, et timuerunt valde; et luxit Israel luctu magno. 
\verse Et quaesierunt omnes gentes, quae erant in circuitu eorum, conterere eos; dixerunt enim: 
\verse “Non habent principem et adiuvantem; nunc ergo expugnemus illos et tollamus de hominibus memoriam eorum". 
\end{biblechapter}

\begin{biblechapter}  
\verse Et audivit Simon quod con gregavit Tryphon exercitum copiosum, ut veniret in terram Iudae et attereret eam. 
\verse Videns quia in tremore populus est et in timore, ascendit Ierusalem et congregavit populum 
\verse et exhortatus est eos et dixit illis: “Vos scitis quanta ego et fratres mei et domus patris mei fecimus pro legibus et pro sanctis proelia et angustias quales vidimus. 
\verse Horum gratia perierunt fratres mei omnes propter Israel, et relictus sum ego solus.  
\verse Et nunc non mihi contingat parcere animae meae in omni tempore tribulationis; non enim melior sum fratribus meis. 
\verse Vindicabo tamen gentem meam et sancta, uxores quoque et natos vestros, quia congregatae sunt universae gentes conterere nos inimicitiae gratia". 
\verse Et accensus est spiritus populi, simul ut audivit sermones istos, 
\verse et responderunt voce magna dicentes: “Tu es dux noster loco Iudae et Ionathae fratris tui; 
\verse pugna proelium nostrum, et omnia, quaecumque dixeris nobis, faciemus". 
\verse Et congregans omnes viros bellatores acceleravit consummare universos muros Ierusalem et munivit eam in gyro. 
\verse Et misit Ionathan filium Absalomi et cum eo exercitum magnum in Ioppen et, eiectis his, qui erant in ea, remansit illic in ea. 
\verse Et movit Tryphon a Ptolemaida cum exercitu multo, ut veniret in terram Iudae, et Ionathas cum eo in custodia. 
\verse Simon autem applicuit in Adidis contra faciem campi. 
\verse Et ut cognovit Tryphon quia surrexit Simon loco fratris sui Ionathae et quia commissurus esset cum eo proelium, misit ad eum legatos 
\verse dicens: “Pro argento, quod debebat frater tuus Ionathas fisco regis propter negotia, quae habuit, detinuimus eum; 
\verse et nunc mitte argenti talenta centum et duos filios eius obsides, ut non dimissus fugiat a nobis, et remittemus eum". 
\verse Et cognovit Simon quia cum dolo loquuntur secum; misit tamen accipere argentum et pueros, ne inimicitiam magnam sumeret ad populum,  
\verse qui dicerent: “Quia non misit ei argentum et pueros, periit". 
\verse Et misit pueros et centum talenta. Et mentitus est et non dimisit Ionathan; 
\verse et post haec venit Tryphon intrare in regionem, ut contereret eam, et gyraverunt per viam, quae ducit Adoram. Et Simon et castra eius ambulabant in omnem locum, quocumque ibant. 
\verse Qui autem in arce erant, miserunt ad Tryphonem legatos urgentes eum, ut veniret ad eos per desertum et mitteret illis alimonias. 
\verse Et paravit Tryphon omnem equitatum suum, ut veniret; et in illa nocte fuit nix multa valde, et non venit propter nivem et discessit et abiit in Galaaditim. 
\verse Et cum appropinquasset Bascaman, occidit Ionathan, et sepultus est illic;  
\verse et convertit Tryphon et abiit in terram suam. 
\verse Et misit Simon et accepit ossa Ionathae fratris sui et sepelivit eum in Modin civitate patrum eius. 
\verse Et planxerunt eum omnis Israel planctu magno et luxerunt eum dies multos. 
\verse Et aedificavit Simon super sepulcrum patris sui et fratrum suorum et exaltavit illud visu, lapide polito retro et ante; 
\verse et statuit septem pyramidas, unam contra unam, patri et matri et quattuor fratribus. 
\verse Et his fecit machinamenta circumponens columnas magnas et super columnas arma ad memoriam aeternam et iuxta arma naves sculptas, quae viderentur ab omnibus navigantibus mare. 
\verse Hoc est sepulcrum, quod fecit in Modin, usque in hunc diem. 
\verse Tryphon autem iter faciebat dolo cum Antiocho rege adulescentiore et occidit eum 
\verse et regnavit loco eius et imposuit sibi diadema Asiae et fecit plagam magnam in terra. 
\verse Et aedificavit Simon praesidia Iudaeae muniens ea turribus excelsis et muris magnis et portis et seris; et posuit alimenta in munitionibus. 
\verse Et elegit Simon viros et misit ad Demetrium regem, ut faceret remissionem regioni, quia actus omnes Tryphonis fuerant rapinae. 
\verse Et Demetrius rex ad verba ista respondit ei et scripsit epistulam talem:  
\verse “Rex Demetrius Simoni summo sacerdoti et amico regum et senioribus et genti Iudaeorum salutem. 
\verse Coronam auream et baen, quam misistis, suscepimus; et parati sumus facere vobiscum pacem magnam et scribere praepositis regis remittere vobis, quae indulsimus. 
\verse Quaecumque enim constituimus vobis, constant; munitiones, quas aedificastis, vobis sint. 
\verse Remittimus quoque ignorantias et peccata usque in hodiernum diem et coronam, quam debebatis; et, si quid aliud erat tributarium in Ierusalem, iam non sit tributarium. 
\verse Et, si qui ex vobis apti sunt conscribi inter nostros, conscribantur, et sit inter nos pax". 
\verse Anno centesimo septuagesimo ablatum est iugum gentium ab Israel. 
\verse Et coepit populus Israel scribere in conscriptionibus et commutationibus: “Anno primo sub Simone summo sacerdote magno duce et principe Iudaeorum". 
\verse In diebus illis applicuit Simon ad Gazaram et circumdedit eam castris et fecit machinam et applicuit ad civitatem et percussit turrim unam et comprehendit. 
\verse Et eruperant, qui erant intra machinam, in civitatem; et factus est motus magnus in civitate. 
\verse Et ascenderunt, qui erant in civitate, cum uxoribus et filiis supra murum, scissis tunicis suis, et clamaverunt voce magna postulantes a Simone dextras sibi dari 
\verse et dixerunt: “Non nobis reddas secundum malitias nostras sed secundum misericordiam tuam". 
\verse Et consensit illis Simon et non debellavit eos; eiecit tamen eos de civitate et mundavit aedes, in quibus fuerant simulacra, et tunc intravit in eam canens et benedicens. 
\verse Et, eiecta ab ea omni immunditia, collocavit in ea viros, qui legem facerent, et munivit eam et aedificavit sibi habitationem. 
\verse Qui autem erant in arce Ierusalem, prohibebantur egredi et ingredi regionem et emere ac vendere; et esurierunt valde, et multi ex eis fame perierunt. 
\verse Et clamaverunt ad Simonem, ut dextras acciperent, et dedit illis et eiecit eos inde et mundavit arcem a contaminationibus. 
\verse Et intraverunt in eam tertia et vicesima die secundi mensis, anno centesimo septuagesimo primo, cum laude et ramis palmarum et cinyris et cymbalis et nablis et hymnis et canticis, quia contritus est inimicus magnus ex Israel. 
\verse Et constituit, ut omnibus annis agerentur dies hi cum laetitia. 
\verse Et munivit montem templi, qui erat secus arcem, et habitavit ibi ipse et qui cum eo erant. 
\verse Et vidit Simon Ioannem filium suum quod vir esset et posuit eum ducem virtutum universarum; et habitavit in Gazaris. 
\end{biblechapter}

\begin{biblechapter}  
\verse Anno centesimo septuagesimo secundo congregavit rex Demetrius exercitum suum et abiit in Mediam ad contrahenda sibi auxilia, ut expugnaret Tryphonem.  
\verse Et audivit Arsaces rex Persidis et Mediae, quia intravit Demetrius confines suos, et misit unum de principibus suis, ut comprehenderet eum vivum. 
\verse Et abiit et percussit castra Demetrii et comprehendit eum et duxit eum ad Arsacem et posuit eum in custodiam. 
\verse Et siluit terra Iudae omnibus diebus Simonis; et quaesivit bona genti suae, et placuit illis potestas eius, et gloria eius omnibus diebus. 
\verse Et cum omni gloria sua accepit Ioppen in portum et fecit introitum insulis maris. 
\verse Et dilatavit fines gentis suae et obtinuit regionem. 
\verse Et congregavit captivitatem multam et dominatus est Gazarae et Bethsurae et arci; et abstulit immunditias ex ea, et non erat qui resisteret ei. 
\verse Et unusquisque colebat terram suam cum pace; et terra dabat fructus suos, et ligna camporum fructum suum. 
\verse Seniores in plateis sedebant, omnes de bonis communiter tractabant, et iuvenes induebant se gloriam et stolas belli. 
\verse Et civitatibus tribuebat alimonias et constituebat eas, ut essent vasa munitionis, quoadusque nominatum est nomen gloriae eius usque ad extremum terrae. 
\verse Fecit pacem super terram, et laetatus est Israel laetitia magna. 
\verse Et sedit unusquisque sub vite sua et sub ficulnea sua, et non erat qui eos terreret. 
\verse Defecit impugnans eos super terram; reges contriti sunt in diebus illis. 
\verse Et confirmavit omnes humiles populi sui et legem exquisivit et abstulit omnem iniquum et malum. 
\verse Sancta glorificavit et multiplicavit vasa sanctorum. 
\verse Et auditum est Romae quia defunctus esset Ionathas, et usque in Spartiatas, et contristati sunt valde. 
\verse Ut audierunt autem quod Simon frater eius factus esset summus sacerdos loco eius, et ipse obtineret regionem et civitates in ea, 
\verse scripserunt ad eum in tabulis aereis, ut renovarent cum eo amicitias et societatem, quam fecerant cum Iuda et cum Ionatha fratribus eius; 
\verse et lectae sunt in conspectu ecclesiae in Ierusalem. Et hoc exemplum epistularum, quas Spartiatae miserunt: 
\verse “Spartianorum principes et civitas Simoni sacerdoti magno et senioribus et sacerdotibus et reliquo populo Iudaeorum fratribus salutem. 
\verse Legati, qui missi sunt ad populum nostrum, nuntiaverunt nobis de vestra gloria et honore, et gavisi sumus in introitu eorum  
\verse et scripsimus, quae ab eis erant dicta in conciliis populi sic: “Numenius Antiochi et Antipater Iasonis filius, legati Iudaeorum, venerunt ad nos renovantes nobiscum amicitiam”. 
\verse Et placuit populo excipere viros gloriose et ponere exemplum sermonum eorum in segregatis populi libris, ut sit ad memoriam populo Spartiatarum. Exemplum autem horum scripsimus Simoni magno sacerdoti". 
\verse Post haec autem misit Simon Numenium Romam habentem clipeum aureum magnum pondo minarum mille ad statuendam cum eis societatem. 
\verse Cum autem audisset populus sermones istos, dixerunt: “Quam gratiarum actionem reddemus Simoni et filiis eius? 
\verse Invaluit enim ipse et fratres eius et domus patris eius et expugnavit inimicos Israel ab eis; et statuerunt ei libertatem". Et descripserunt in tabulis aereis et posuerunt in titulis in monte Sion. 
\verse Et hoc est exemplum scripturae: “Octava decima die Elul, anno centesimo septuagesimo secundo, anno tertio sub Simone sacerdote magno, in Asaramel, 
\verse in conventu magno sacerdotum et populi et principum gentis et seniorum regionis nota facta sunt nobis haec: 
\verse Quoniam frequenter facta sunt proelia in regione, Simon autem filius Matthathiae, filius ex filiis Ioarib, et fratres eius dederunt se periculo et restiterunt adversariis gentis suae, ut starent sancta ipsorum et lex; et gloria magna glorificaverunt gentem suam. 
\verse Et congregavit Ionathas gentem suam et factus est illis sacerdos magnus et appositus est ad populum suum. 
\verse Et voluerunt inimici eorum calcare et atterere regionem ipsorum et extendere manus in sancta eorum. 
\verse Tunc restitit Simon et pugnavit pro gente sua et erogavit pecunias multas et armavit viros virtutis gentis suae et dedit illis stipendia. 
\verse Et munivit civitates Iudaeae et Bethsuram, quae erat in finibus Iudaeae, ubi erant arma hostium antea, et posuit illic praesidium viros Iudaeos; 
\verse et Ioppen munivit, quae erat ad mare, et Gazaram, quae est in finibus Azoti, in qua hostes antea habitabant, et collocavit illic Iudaeos et, quaecumque apta erant ad correptionem eorum, posuit in eis. 
\verse Et vidit populus fidem Simonis et gloriam, quam cogitabat facere genti suae; et posuerunt eum ducem suum et principem sacerdotum, eo quod ipse fecerat haec omnia et iustitiam et fidem, quam conservavit genti suae, et exquisivit omni modo exaltare populum suum.  
\verse Et in diebus eius prosperatum est in manibus eius, ut tollerentur gentes de regione ipsorum et, qui in civitate David erant in Ierusalem, qui fecerant sibi arcem, de qua procedebant et contaminabant omnia, quae in circuitu sanctorum sunt, et inferebant plagam magnam castitati; 
\verse et collocavit in ea viros Iudaeos et munivit eam ad tutamentum regionis et civitatis et exaltavit muros Ierusalem. 
\verse Et rex Demetrius statuit illi summum sacerdotium secundum haec  
\verse et fecit eum amicum suum et glorificavit eum gloria magna. 
\verse Audivit enim quod appellati sunt Iudaei a Romanis amici et socii et fratres, et quia susceperunt legatos Simonis gloriose; 
\verse et quia Iudaei et sacerdotes consenserunt eum esse ducem suum et summum sacerdotem in aeternum, donec surgat propheta fidelis, 
\verse et ut sit super eos dux, et ut cura esset illi pro sanctis, ut constitueret per eum super opera eorum et super regionem et super arma et super praesidia; 
\verse et cura sit illi de sanctis, et ut audiatur ab omnibus; et scribantur in nomine eius omnes conscriptiones in regione, et ut operiatur purpura et aurum portet; 
\verse et ne liceat ulli ex populo et ex sacerdotibus irritum facere aliquid horum et contradicere his, quae ab eo dicuntur, aut convocare conventum in regione sine ipso et vestiri purpura et uti fibula aurea; 
\verse qui autem fecerit extra haec aut irritum fecerit aliquid horum, reus erit. 
\verse Et complacuit omni populo statuere Simoni facere secundum verba ista. 
\verse Et suscepit Simon et placuit ei, ut summo sacerdotio fungeretur et esset dux et princeps gentis Iudaeorum et sacerdotum et praeesset omnibus". 
\verse Et scripturam istam dixerunt ponere in tabulis aereis et ponere eas in peribolo sanctorum in loco celebri; 
\verse exemplum autem eorum ponere in aerario, ut habeat Simon et filii eius. 
\end{biblechapter}

\begin{biblechapter}  
\verse Et misit rex Antiochus filius Demetrii epistulas ab insulis maris Simoni sacerdoti et principi gentis Iudaeorum et universae genti, 
\verse et erant continentes hunc modum: “Rex Antiochus Simoni sacerdoti magno et gentis principi et genti Iudaeorum salutem. 
\verse Quoniam quidam pestilentes obtinuerunt regnum patrum nostrorum, volo autem vindicare regnum, ut restituam illud, sicut erat antea, delectumque feci multitudinis exercitus et feci naves bellicas; 
\verse volo autem procedere per regionem, ut ulciscar in eos, qui corruperunt regionem nostram et qui desolaverunt civitates multas in regno meo.  
\verse Nunc ergo statuo tibi omnes oblationes, quas remiserunt tibi ante me reges, et quaecumque alia dona remiserunt tibi. 
\verse Et permisi tibi facere monetam propriam numisma regioni tuae; 
\verse Ierusalem autem et sancta esse libera, et omnia arma, quae fabricatus es, et praesidia, quae construxisti, quae tenes, maneant tibi; 
\verse et omne debitum regis, et quae futura sunt regi ex hoc et in totum tempus, remittantur tibi. 
\verse Cum autem obtinuerimus regnum nostrum, glorificabimus te et gentem tuam et templum gloria magna, ita ut manifestetur gloria vestra in universa terra". 
\verse Anno centesimo septuagesimo quarto exiit Antiochus in terram patrum suorum, et convenerunt ad eum omnes exercitus, ita ut pauci relicti essent cum Tryphone. 
\verse Et insecutus est eum Antiochus rex, et venit Doram fugiens, quae est ad mare; 
\verse sciebat enim quod congregata sunt mala in eum, et reliquit eum exercitus.  
\verse Et applicuit Antiochus ad Doram cum centum viginti milibus virorum belligeratorum et octo milibus equitum 
\verse et circuivit civitatem, et naves a mari accesserunt; et vexabat civitatem a terra et mari et neminem sinebat ingredi vel egredi. 
\verse Venit autem Numenius et, qui cum eo fuerant, ab urbe Roma habentes epistulas regibus et regionibus scriptas, in quibus continebantur haec: 
\verse “Lucius consul Romanorum Ptolemaeo regi salutem. 
\verse Legati Iudaeorum venerunt ad nos amici nostri et socii renovantes pristinam amicitiam et societatem, missi a Simone principe sacerdotum et populo Iudaeorum. 
\verse Attulerunt autem et clipeum aureum minarum mille. 
\verse Placuit itaque nobis scribere regibus et regionibus, ut non exquirant illis mala neque impugnent eos et civitates eorum et regionem eorum et ut non ferant auxilium pugnantibus adversus eos. 
\verse Visum autem est nobis accipere ab eis clipeum. 
\verse Si qui ergo pestilentes refugerunt de regione ipsorum ad vos, tradite eos Simoni principi sacerdotum, ut vindicet in eos secundum legem suam". 
\verse Haec eadem scripsit Demetrio regi et Attalo et Ariarathae et Arsacae 
\verse et in omnes regiones et Sampsacae et Spartiatis et in Delum et in Myndum et in Sicyonem et in Carida et in Samum et in Pamphyliam et in Lyciam et in Alicarnassum et in Rhodum et in Phaselidam et in Cho et in Siden et in Aradon et in Gortynam et Cnidum et Cyprum et Cyrenen.  
\verse Exemplum autem eorum scripserunt Simoni principi sacerdotum et populo Iudaeorum. 
\verse Antiochus autem rex applicuit castra in Doram in secunda die admovens ei semper manus et machinas faciens et conclusit Tryphonem, ne exiret aut introiret. 
\verse Et misit ad eum Simon duo milia virorum electorum in auxilium et argentum et aurum et vasa copiosa. 
\verse Et noluit ea accipere, sed rupit omnia, quae pactus est cum eo antea, et alienavit se ab eo. 
\verse Et misit ad eum Athenobium unum de amicis suis, ut tractaret cum ipso dicens: “Vos tenetis Ioppen et Gazaram et arcem, quae est in Ierusalem, civitates regni mei; 
\verse fines earum desolastis et fecistis plagam magnam in terra et dominati estis per loca multa in regno meo. 
\verse Nunc ergo tradite civitates, quas occupastis, et tributa locorum, in quibus dominati estis extra fines Iudaeae; 
\verse sin autem, date pro illis quingenta talenta argenti, et exterminii, quod exterminastis, et tributorum civitatum alia talenta quingenta; sin autem, veniemus et expugnabimus vos”. 
\verse Et venit Athenobius amicus regis in Ierusalem et vidit gloriam Simonis et claritatem in auro et argento et apparatum copiosum et obstupuit et rettulit ei verba regis. 
\verse Et respondit ei Simon et dixit ei: “Neque alienam terram sumpsimus neque aliena detinemus sed hereditatem patrum nostrorum, quae iniuste ab inimicis nostris aliquo tempore possessa est. 
\verse Nos vero tempus habentes vindicamus hereditatem patrum nostrorum; 
\verse nam de Ioppe et Gazara, quae expostulas, ipsae faciebant in populo plagam magnam et in regione nostra: horum damus talenta centum". Et non respondit ei verbum. 
\verse Reversus autem cum ira ad regem renuntiavit ei verba ista et gloriam Simonis et universa, quae vidit; et iratus est rex ira magna. 
\verse Tryphon autem ascendit in navem et fugit in Orthosiam. 
\verse Et constituit rex Cendebaeum ducem maritimum et exercitum peditum et equitum dedit illi; 
\verse et mandavit illi movere castra contra faciem Iudaeae et mandavit ei aedificare Cedron et obstruere portas civitatis et ut debellaret populum. Rex autem persequebatur Tryphonem. 
\verse Et pervenit Cendebaeus Iamniam et coepit irritare plebem et conculcare Iudaeam et captivare populum et interficere. 
\verse Et aedificavit Cedron et collocavit illic equites et exercitum, ut egressi perambularent vias Iudaeae, sicut constituit ei rex. 
\end{biblechapter}

\begin{biblechapter}  
\verse Et ascendit Ioannes de Gazaris et nuntiavit Simoni patri suo, quae fecit Cendebaeus. 
\verse Et vocavit Simon duos filios seniores, Iudam et Ioannem, et ait illis: “Ego et fratres mei et domus patris mei expugnavimus hostes Israel ab adulescentia usque in hunc diem, et prosperatum est in manibus nostris liberare Israel saepius. 
\verse Nunc autem senui, vos autem in misericordia sufficientes estis in annis; estote loco meo et fratris mei et egressi pugnate pro gente nostra; auxilium vero de caelo vobiscum sit". 
\verse Et elegit de regione viginti milia virorum belligeratorum et equites; et profecti sunt ad Cendebaeum et dormierunt in Modin 
\verse et surrexerunt mane et abierunt in campum. Et ecce exercitus copiosus in obviam illis peditum et equitum, et fluvius torrens erat inter medium ipsorum. 
\verse Et admovit castra contra faciem eorum ipse et populus eius et vidit populum trepidantem ad transfretandum torrentem; et transfretavit primus, et viderunt eum viri et transierunt post eum. 
\verse Et divisit populum, et equites in medio peditum; erat autem equitatus adversariorum copiosus nimis. 
\verse Et exclamaverunt tubis, et in fugam conversus est Cendebaeus et castra eius, et ceciderunt ex eis multi vulnerati; residui autem in munitionem fugerunt. 
\verse Tunc vulneratus est Iudas frater Ioannis; Ioannes autem insecutus est eos, donec venit Cedron, quam aedificavit.  
\verse Et fugerunt usque ad turres, quae erant in agris Azoti, et succendit eas igni; et ceciderunt ex illis duo milia virorum. Et reversus est in Iudaeam in pace. 
\verse Et Ptolemaeus filius Abubi constitutus erat dux in campo Iericho et habebat argentum et aurum multum; 
\verse erat enim gener summi sacerdotis. 
\verse Et exaltatum est cor eius, et volebat obtinere regionem et cogitabat dolum adversus Simonem et filios eius, ut tolleret eos. 
\verse Simon autem perambulans civitates, quae erant in regione, et sollicitudinem gerens earum descendit in Iericho ipse et Matthathias et Iudas filii eius, anno centesimo septuagesimo septimo, mense undecimo, hic est mensis Sabath. 
\verse Et suscepit eos filius Abubi in munitiunculam, quae vocatur Doc, cum dolo, quam aedificavit; et fecit eis convivium magnum et abscondit illic viros. 
\verse Et, cum inebriatus esset Simon et filii eius, surrexit Ptolemaeus cum suis et sumpserunt arma sua et intraverunt in convivium et occiderunt eum et duos filios eius et quosdam pueros eius. 
\verse Et fecit deceptionem magnam et reddidit mala pro bonis. 
\verse Et scripsit haec Ptolemaeus et misit regi, ut mitteret ei exercitum in auxilium et traderet ei civitates et regionem. 
\verse Et misit alios in Gazaram tollere Ioannem; et tribunis misit epistulas, ut venirent ad se, et daret eis argentum et aurum et dona. 
\verse Et alios misit occupare Ierusalem et montem templi.  
\verse Et praecurrens quidam nuntiavit Ioanni in Gazara quia periit pater eius et fratres eius et quia: “Misit te quoque interfici". 
\verse Ut audivit autem, vehementer expavit et comprehendit viros, qui venerant perdere eum, et occidit eos; cognovit enim quia quaerebant eum perdere. 
\verse Et cetera sermonum Ioannis et bellorum eius et bonarum virtutum, quibus fortiter gessit, et aedificii murorum, quos exstruxit, et rerum gestarum eius, 
\verse ecce haec scripta sunt in libro dierum sacerdotii eius, ex quo factus est princeps sacerdotum post patrem suum.
\end{biblechapter}
