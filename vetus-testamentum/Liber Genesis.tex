\biblebook{Liber Genesis}

\begin{biblechapter}  
\verse In principio creavit Deus caelum et terram. 
\verse Terra autem erat inanis et vacua, et tenebrae super faciem abyssi, et spiritus Dei ferebatur super aquas. 
\verse Dixitque Deus: “Fiat lux”. Et facta est lux. 
\verse Et vidit Deus lucem quod esset bona et divisit Deus lucem ac tenebras. 
\verse Appellavitque Deus lucem Diem et tenebras Noctem. Factumque est vespere et mane, dies unus. 
\verse Dixit quoque Deus: “Fiat firmamentum in medio aquarum et dividat aquas ab aquis”. 
\verse Et fecit Deus firmamentum divisitque aquas, quae erant sub firmamento, ab his, quae erant super firmamentum. Et factum est ita. 
\verse Vocavitque Deus firmamentum Caelum. Et factum est vespere et mane, dies secundus. 
\verse Dixit vero Deus: “Congregentur aquae, quae sub caelo sunt, in locum unum, et appareat arida”. Factumque est ita. 
\verse Et vocavit Deus aridam Terram congregationesque aquarum appellavit Maria. Et vidit Deus quod esset bonum.  
\verse Et ait Deus: “Germinet terra herbam virentem et herbam facientem semen et lignum pomiferum faciens fructum iuxta genus suum, cuius semen in semetipso sit super terram”. Et factum est ita. 
\verse Et protulit terra herbam virentem et herbam afferentem semen iuxta genus suum lignumque faciens fructum, qui habet in semetipso sementem secundum speciem suam. Et vidit Deus quod esset bonum. 
\verse Et factum est vespere et mane, dies tertius. 
\verse Dixit autem Deus: “Fiant luminaria in firmamento caeli, ut dividant diem ac noctem et sint in signa et tempora et dies et annos, 
\verse ut luceant in firmamento caeli et illuminent terram. Et factum est ita. 
\verse Fecitque Deus duo magna luminaria: luminare maius, ut praeesset diei, et luminare minus, ut praeesset nocti, et stellas. 
\verse Et posuit eas Deus in firmamento caeli, ut lucerent super terram 
\verse et praeessent diei ac nocti et dividerent lucem ac tenebras. Et vidit Deus quod esset bonum. 
\verse Et factum est vespere et mane, dies quartus. 
\verse Dixit etiam Deus: “Pullulent aquae reptile animae viventis, et volatile volet super terram sub firmamento caeli”. 
\verse Creavitque Deus cete grandia et omnem animam viventem atque motabilem, quam pullulant aquae secundum species suas, et omne volatile secundum genus suum. Et vidit Deus quod esset bonum; 
\verse benedixitque eis Deus dicens: “Crescite et multiplicamini et replete aquas maris, avesque multiplicentur super terram". 
\verse Et factum est vespere et mane, dies quintus. 
\verse Dixit quoque Deus: “Producat terra animam viventem in genere suo, iumenta et reptilia et bestias terrae secundum species suas”. Factumque est ita. 
\verse Et fecit Deus bestias terrae iuxta species suas et iumenta secundum species suas et omne reptile terrae in genere suo. Et vidit Deus quod esset bonum. 
\verse Et ait Deus: “Faciamus hominem ad imaginem et similitudinem nostram; et praesint piscibus maris et volatilibus caeli et bestiis universaeque terrae omnique reptili, quod movetur in terra”. 
\verse Et creavit Deus hominem ad imaginem suam; ad imaginem Dei creavit illum; masculum et feminam creavit eos. 
\verse Benedixitque illis Deus et ait illis Deus: “Crescite et multiplicamini et replete terram et subicite eam et dominamini piscibus maris et volatilibus caeli et universis animantibus, quae moventur super terram”. 
\verse Dixitque Deus: “Ecce dedi vobis omnem herbam afferentem semen super terram et universa ligna, quae habent in semetipsis fructum ligni portantem sementem, ut sint vobis in escam 
\verse et cunctis animantibus terrae omnique volucri caeli et universis, quae moventur in terra et in quibus est anima vivens, omnem herbam virentem ad vescendum”. Et factum est ita. 
\verse Viditque Deus cuncta, quae fecit, et ecce erant valde bona. Et factum est vespere et mane, dies sextus. 
\end{biblechapter}

\begin{biblechapter}  
\verse Igitur perfecti sunt caeli et terra et omnis exercitus eorum. 
\verse Complevitque Deus die septimo opus suum, quod fecerat, et requievit die septimo ab universo opere, quod patrarat. 
\verse Et benedixit Deus diei septimo et sanctificavit illum, quia in ipso requieverat ab omni opere suo, quod creavit Deus, ut faceret. 
\verse Istae sunt generationes caeli et terrae, quando creata sunt. In die quo fecit Dominus Deus terram et caelum ­ 
\verse omne virgultum agri, antequam oriretur in terra, omnisque herba regionis, priusquam germinaret; non enim pluerat Dominus Deus super terram, et homo non erat, qui operaretur humum,  
\verse sed fons ascendebat e terra irrigans universam superficiem terrae ­ 
\verse tunc formavit Dominus Deus hominem pulverem de humo et inspiravit in nares eius spiraculum vitae, et factus est homo in animam viventem. 
\verse Et plantavit Dominus Deus paradisum in Eden ad orientem, in quo posuit hominem, quem formaverat. 
\verse Produxitque Dominus Deus de humo omne lignum pulchrum visu et ad vescendum suave, lignum etiam vitae in medio paradisi lignumque scientiae boni et mali. 
\verse Et fluvius egrediebatur ex Eden ad irrigandum paradisum, qui inde dividitur in quattuor capita. 
\verse Nomen uni Phison: ipse est, qui circuit omnem terram Hevila, ubi est aurum; 
\verse et aurum terrae illius optimum est; ibi invenitur bdellium et lapis onychinus. 
\verse Et nomen fluvio secundo Geon: ipse est, qui circuit omnem terram Aethiopiae. 
\verse Nomen vero fluminis tertii Tigris: ipse vadit ad orientem Assyriae. Fluvius autem quartus ipse est Euphrates. 
\verse Tulit ergo Dominus Deus hominem et posuit eum in paradiso Eden, ut operaretur et custodiret illum; 
\verse praecepitque Dominus Deus homini dicens: “Ex omni ligno paradisi comede; 
\verse de ligno autem scientiae boni et mali ne comedas; in quocumque enim die comederis ex eo, morte morieris”. 
\verse Dixit quoque Dominus Deus: “Non est bonum esse hominem solum; faciam ei adiutorium simile sui”. 
\verse Formatis igitur Dominus Deus de humo cunctis animantibus agri et universis volatilibus caeli, adduxit ea ad Adam, ut videret quid vocaret ea; omne enim, quod vocavit Adam animae viventis, ipsum est nomen eius. 
\verse Appellavitque Adam nominibus suis cuncta pecora et universa volatilia caeli et omnes bestias agri; Adae vero non inveniebatur adiutor similis eius. 
\verse Immisit ergo Dominus Deus soporem in Adam. Cumque obdormisset, tulit unam de costis eius et replevit carnem pro ea; 
\verse et aedificavit Dominus Deus costam, quam tulerat de Adam, in mulierem et adduxit eam ad Adam. 
\verse Dixitque Adam: “Haec nunc os ex ossibus meis et caro de carne mea! Haec vocabitur Virago, quoniam de viro sumpta est haec”. 
\verse Quam ob rem relinquet vir patrem suum et matrem et adhaerebit uxori suae; et erunt in carnem unam. 
\verse Erant autem uterque nudi, Adam scilicet et uxor eius, et non erubescebant. 
\end{biblechapter}

\begin{biblechapter}  
\verse Et serpens erat callidior cunctis animantibus agri, quae fecerat Dominus Deus. Qui dixit ad mulierem: “Verene praecepit vobis Deus, ut non comederetis de omni ligno paradisi?”. 
\verse Cui respondit mulier: “De fructu lignorum, quae sunt in paradiso, vescimur; 
\verse de fructu vero ligni, quod est in medio paradisi, praecepit nobis Deus, ne comederemus et ne tangeremus illud, ne moriamur”. 
\verse Dixit autem serpens ad mulierem: “Nequaquam morte moriemini!  
\verse Scit enim Deus quod in quocumque die comederitis ex eo, aperientur oculi vestri, et eritis sicut Deus scientes bonum et malum”. 
\verse Vidit igitur mulier quod bonum esset lignum ad vescendum et pulchrum oculis et desiderabile esset lignum ad intellegendum; et tulit de fructu illius et comedit deditque etiam viro suo secum, qui comedit. 
\verse Et aperti sunt oculi amborum. Cumque cognovissent esse se nudos, consuerunt folia ficus et fecerunt sibi perizomata. 
\verse Et cum audissent vocem Domini Dei deambulantis in paradiso ad auram post meridiem, abscondit se Adam et uxor eius a facie Domini Dei in medio ligni paradisi. 
\verse Vocavitque Dominus Deus Adam et dixit ei: “Ubi es?”. 
\verse Qui ait: “Vocem tuam audivi in paradiso et timui eo quod nudus essem et abscondi me”. 
\verse Cui dixit: “Quis enim indicavit tibi quod nudus esses, nisi quod ex ligno, de quo tibi praeceperam, ne comederes, comedisti?”.  
\verse Dixitque Adam: “Mulier, quam dedisti sociam mihi, ipsa dedit mihi de ligno, et comedi”.  
\verse Et dixit Dominus Deus ad mulierem: “Quid hoc fecisti?”. Quae respondit: “Serpens decepit me, et comedi”. 
\verse Et ait Dominus Deus ad serpentem: “Quia fecisti hoc, maledictus es inter omnia pecora et omnes bestias agri! Super pectus tuum gradieris et pulverem comedes cunctis diebus vitae tuae. 
\verse Inimicitias ponam inter te et mulierem et semen tuum et semen illius; ipsum conteret caput tuum, et tu conteres calcaneum eius”. 
\verse Mulieri dixit: “Multiplicabo aerumnas tuas et conceptus tuos: in dolore paries filios, et ad virum tuum erit appetitus tuus, ipse autem dominabitur tui”. 
\verse Adae vero dixit: “Quia audisti vocem uxoris tuae et comedisti de ligno, ex quo praeceperam tibi, ne comederes, maledicta humus propter te! In laboribus comedes ex ea cunctis diebus vitae tuae. 
\verse Spinas et tribulos germinabit tibi, et comedes herbas terrae; 
\verse in sudore vultus tui vesceris pane, donec revertaris ad humum, de qua sumptus es, quia pulvis es et in pulverem reverteris”. 
\verse Et vocavit Adam nomen uxoris suae Eva, eo quod mater esset cunctorum viventium. 
\verse Fecit quoque Dominus Deus Adae et uxori eius tunicas pelliceas et induit eos. 
\verse Et ait Dominus Deus: “Ecce homo factus est quasi unus ex nobis, ut sciat bonum et malum; nunc ergo, ne mittat manum suam et sumat etiam de ligno vitae et comedat et vivat in aeternum!”.
\verse Emisit eum Dominus Deus de paradiso Eden, ut operaretur humum, de qua sumptus est. 
\verse Eiecitque hominem et collocavit ad orientem paradisi Eden cherubim et flammeum gladium atque versatilem ad custodiendam viam ligni vitae. 
\end{biblechapter}

\begin{biblechapter}  
\verse Adam vero cognovit Evam uxorem suam, quae concepit et peperit Cain dicens: “Acquisivi virum per Dominum”. 
\verse Rursusque peperit fratrem eius Abel. Et fuit Abel pastor ovium et Cain agricola. 
\verse Factum est autem post aliquot dies ut offerret Cain de fructibus agri munus Domino. 
\verse Abel quoque obtulit de primogenitis gregis sui et de adipibus eorum. Et respexit Dominus ad Abel et ad munus eius, 
\verse ad Cain vero et ad munus illius non respexit. Iratusque est Cain vehementer, et concidit vultus eius. 
\verse Dixitque Dominus ad eum: “Quare iratus es, et cur concidit facies tua? 
\verse Nonne si bene egeris, vultum attolles? Sin autem male, in foribus peccatum insidiabitur, et ad te erit appetitus eius, tu autem dominaberis illius”. 
\verse Dixitque Cain ad Abel fratrem suum: “Egrediamur foras”. Cumque essent in agro, consurrexit Cain adversus Abel fratrem suum et interfecit eum. 
\verse Et ait Dominus ad Cain: “Ubi est Abel frater tuus?”. Qui respondit: “Nescio. Num custos fratris mei sum ego?”. 
\verse Dixitque ad eum: “Quid fecisti? Vox sanguinis fratris tui clamat ad me de agro. 
\verse Nunc igitur maledictus eris procul ab agro, qui aperuit os suum et suscepit sanguinem fratris tui de manu tua! 
\verse Cum operatus fueris eum, amplius non dabit tibi fructus suos; vagus et profugus eris super terram”. 
\verse Dixitque Cain ad Dominum: “Maior est poena mea quam ut portem eam. 
\verse Ecce eicis me hodie a facie agri, et a facie tua abscondar et ero vagus et profugus in terra; omnis igitur, qui invenerit me, occidet me”. 
\verse Dixitque ei Dominus: “Nequaquam ita fiet, sed omnis qui occiderit Cain, septuplum punietur!”. Posuitque Dominus Cain signum, ut non eum interficeret omnis qui invenisset eum. 
\verse Egressusque Cain a facie Domini habitavit in terra Nod ad orientalem plagam Eden. 
\verse Cognovit autem Cain uxorem suam, quae concepit et peperit Henoch. Et aedificavit civitatem vocavitque nomen eius ex nomine filii sui Henoch. 
\verse Porro Henoch genuit Irad, et Irad genuit Maviael, et Maviael genuit Mathusael, et Mathusael genuit Lamech. 
\verse Qui accepit uxores duas: nomen uni Ada et nomen alteri Sella. 
\verse Genuitque Ada Iabel, qui fuit pater habitantium in tentoriis atque pastorum. 
\verse Et nomen fratris eius Iubal; ipse fuit pater omnium canentium cithara et organo. 
\verse Sella quoque genuit Tubalcain, qui fuit malleator et faber in cuncta opera aeris et ferri. Soror vero Tubalcain Noema. 
\verse Dixitque Lamech uxoribus suis: “Ada et Sella, audite vocem meam; uxores Lamech, auscultate sermonem meum: occidi virum pro vulnere meo et adulescentulum pro livore meo; 
\verse septuplum ultio dabitur de Cain, de Lamech vero septuagies septies”. 
\verse Cognovit quoque Adam uxorem suam, et peperit filium vocavitque nomen eius Seth dicens: “Posuit mihi Deus semen aliud pro Abel, quem occidit Cain”. 
\verse Sed et Seth natus est filius, quem vocavit Enos. Tunc coeperunt invocare nomen Domini. 
\end{biblechapter}

\begin{biblechapter}  
\verse Hic est liber generationis Adam. In die qua creavit Deus hominem, ad similitudinem Dei fecit illum. 
\verse Masculum et feminam creavit eos et benedixit illis; et vocavit nomen eorum Adam in die, quo creati sunt. 
\verse Vixit autem Adam centum triginta annis et genuit ad similitudinem et imaginem suam vocavitque nomen eius Seth. 
\verse Et facti sunt dies Adam, postquam genuit Seth, octingenti anni, genuitque filios et filias. 
\verse Et factum est omne tempus, quod vixit Adam, anni nongenti triginta, et mortuus est. 
\verse Vixit quoque Seth centum quinque annos et genuit Enos. 
\verse Vixitque Seth, postquam genuit Enos, octingentis septem annis genuitque filios et filias. 
\verse Et facti sunt omnes dies Seth nongentorum duodecim annorum, et mortuus est. 
\verse Vixit vero Enos nonaginta annis et genuit Cainan. 
\verse Et vixit Enos, postquam genuit Cainan, octingentis quindecim annis et genuit filios et filias.  
\verse Factique sunt omnes dies Enos nongentorum quinque annorum, et mortuus est. 
\verse Vixit quoque Cainan septuaginta annis et genuit Malaleel. 
\verse Et vixit Cainan, postquam genuit Malaleel, octingentos quadraginta annos genuitque filios et filias. 
\verse Et facti sunt omnes dies Cainan nongenti decem anni, et mortuus est. 
\verse Vixit autem Malaleel sexaginta quinque annos et genuit Iared. 
\verse Et vixit Malaleel, postquam genuit Iared, octingentis triginta annis et genuit filios et filias. 
\verse Et facti sunt omnes dies Malaleel octingenti nonaginta quinque anni, et mortuus est. 
\verse Vixitque Iared centum sexaginta duobus annis et genuit Henoch. 
\verse Et vixit Iared, postquam genuit Henoch, octingentos annos et genuit filios et filias. 
\verse Et facti sunt omnes dies Iared nongenti sexaginta duo anni, et mortuus est. 
\verse Porro Henoch vixit sexaginta quinque annis et genuit Mathusalam. 
\verse Et ambulavit Henoch cum Deo, postquam genuit Mathusalam, trecentis annis et genuit filios et filias. 
\verse Et facti sunt omnes dies Henoch trecenti sexaginta quinque anni, 
\verse ambulavitque cum Deo et non apparuit, quia tulit eum Deus. 
\verse Vixit quoque Mathusala centum octoginta septem annos et genuit Lamech. 
\verse Et vixit Mathusala, postquam genuit Lamech, septingentos octoginta duos annos et genuit filios et filias. 
\verse Et facti sunt omnes dies Mathusalae nongenti sexaginta novem anni, et mortuus est. 
\verse Vixit autem Lamech centum octoginta duobus annis et genuit filium 
\verse vocavitque nomen eius Noe dicens: “Iste consolabitur nos ab operibus nostris et labore manuum nostrarum in agro, cui maledixit Dominus”. 
\verse Vixitque Lamech, postquam genuit Noe, quingentos nonaginta quinque annos et genuit filios et filias. 
\verse Et facti sunt omnes dies Lamech septingenti septuaginta septem anni, et mortuus est. 
\verse Noe vero, cum quingentorum esset annorum, genuit Sem, Cham et Iapheth. 
\end{biblechapter}

\begin{biblechapter}  
\verse Cumque coepissent homines multiplicari super terram et filias procreassent, 
\verse videntes filii Dei filias hominum quod essent pulchrae, acceperunt sibi uxores ex omnibus, quas elegerant. 
\verse Dixitque Deus: “Non permanebit spiritus meus in homine in aeternum, quia caro est; eruntque dies illius centum viginti annorum”. 
\verse Gigantes erant super terram in diebus illis et etiam postquam ingressi sunt filii Dei ad filias hominum, illaeque eis genuerunt: isti sunt potentes a saeculo viri famosi. 
\verse Videns autem Dominus quod multa malitia hominum esset in terra, et cuncta cogitatio cordis eorum non intenta esset nisi ad malum omni tempore, 
\verse paenituit Dominum quod hominem fecisset in terra. Et tactus dolore cordis intrinsecus: 
\verse “Delebo, inquit, hominem, quem creavi, a facie terrae, ab homine usque ad pecus, usque ad reptile et usque ad volucres caeli; paenitet enim me fecisse eos”. 
\verse Noe vero invenit gratiam coram Domino. 
\verse Hae sunt generationes Noe: Noe vir iustus atque perfectus fuit in generatione sua; cum Deo ambulavit. 
\verse Et genuit tres filios: Sem, Cham et Iapheth. 
\verse Corrupta est autem terra coram Deo et repleta est iniquitate. 
\verse Cumque vidisset Deus terram esse corruptam ­ omnis quippe caro corruperat viam suam super terram ­ 
\verse dixit ad Noe: “Finis universae carnis venit coram me; repleta est enim terra iniquitate a facie eorum, et ecce ego disperdam eos de terra. 
\verse Fac tibi arcam de lignis cupressinis; mansiunculas in arca facies et bitumine linies eam intrinsecus et extrinsecus. 
\verse Et sic facies eam: trecentorum cubitorum erit longitudo arcae, quinquaginta cubitorum latitudo et triginta cubitorum altitudo illius. 
\verse Fenestram in arca facies et cubito consummabis summitatem eius. Ostium autem arcae pones ex latere; tabulatum inferius, medium et superius facies in ea. 
\verse Ecce ego adducam diluvii aquas super terram, ut interficiam omnem carnem, in qua spiritus vitae est subter caelum: universa, quae in terra sunt, consumentur. 
\verse Ponamque foedus meum tecum; et ingredieris arcam tu et filii tui, uxor tua et uxores filiorum tuorum tecum. 
\verse Et ex cunctis animantibus universae carnis bina induces in arcam, ut vivant tecum, masculini sexus et feminini.  
\verse De volucribus iuxta genus suum et de iumentis in genere suo et ex omni reptili terrae secundum genus suum: bina de omnibus ingredientur ad te, ut possint vivere. 
\verse Tu autem tolle tecum ex omnibus escis, quae mandi possunt, et comportabis apud te; et erunt tam tibi quam illis in cibum”. 
\verse Fecit ergo Noe omnia, quae praeceperat illi Deus; sic fecit. 
\end{biblechapter}

\begin{biblechapter}  
\verse Dixitque Dominus ad Noe: “Ingredere tu et omnis domus tua arcam; te enim vidi iustum coram me in generatione hac. 
\verse Ex omnibus pecoribus mundis tolle septena septena, masculum et feminam; de pecoribus vero non mundis duo duo, masculum et feminam. 
\verse Sed et de volatilibus caeli septena septena, masculum et feminam, ut salvetur semen super faciem universae terrae. 
\verse Adhuc enim et post dies septem ego pluam super terram quadraginta diebus et quadraginta noctibus et delebo omnem substantiam, quam feci, de superficie terrae”. 
\verse Fecit ergo Noe omnia, quae mandaverat ei Dominus. 
\verse Eratque Noe sescentorum annorum, quando diluvii aquae inundaverunt super terram. 
\verse Et ingressus est Noe et filii eius, uxor eius et uxores filiorum eius cum eo in arcam propter aquas diluvii. 
\verse De pecoribus mundis et immundis et de volucribus et ex omni, quod movetur super terram, 
\verse duo et duo ingressa sunt ad Noe in arcam, masculus et femina, sicut praeceperat Deus Noe. 
\verse Cumque transissent septem dies, aquae diluvii inundaverunt super terram. 
\verse Anno sescentesimo vitae Noe, mense secundo, septimo decimo die mensis rupti sunt omnes fontes abyssi magnae, et cataractae caeli apertae sunt; 
\verse et facta est pluvia super terram quadraginta diebus et quadraginta noctibus. 
\verse In articulo diei illius ingressus est Noe et Sem et Cham et Iapheth filii eius, uxor illius et tres uxores filiorum eius cum eis in arcam. 
\verse Ipsi et omne animal secundum genus suum, universaque iumenta in genere suo, et omne reptile, quod movetur super terram in genere suo, cunctumque volatile secundum genus suum, universae aves omnesque volucres 
\verse ingressae sunt ad Noe in arcam, bina et bina ex omni carne, in qua erat spiritus vitae. 
\verse Et quae ingressa sunt, masculus et femina ex omni carne introierunt, sicut praeceperat ei Deus; et inclusit eum Dominus de foris. 
\verse Factumque est diluvium quadraginta diebus super terram, et multiplicatae sunt aquae et elevaverunt arcam in sublime a terra. 
\verse Vehementer enim inundaverunt et omnia repleverunt in superficie terrae; porro arca ferebatur super aquas. 
\verse Et aquae praevaluerunt nimis super terram, opertique sunt omnes montes excelsi sub universo caelo. 
\verse Quindecim cubitis altior fuit aqua super montes, quos operuerat. 
\verse Consumptaque est omnis caro, quae movebatur super terram, volucrum, pecorum, bestiarum omniumque reptilium, quae reptant super terram, et universi homines:  
\verse cuncta, in quibus spiraculum vitae in terra, mortua sunt. 
\verse Et delevit omnem substantiam, quae erat super terram, ab homine usque ad pecus, usque ad reptile et usque ad volucres caeli; et deleta sunt de terra. Remansit autem solus Noe et qui cum eo erant in arca. 
\verse Obtinueruntque aquae terram centum quinquaginta diebus. 
\end{biblechapter}

\begin{biblechapter}  
\verse Recordatus autem Deus Noe cunctorumque animantium et omnium iumentorum, quae erant cum eo in arca, adduxit spiritum super terram, et imminutae sunt aquae. 
\verse Et clausi sunt fontes abyssi et cataractae caeli, et prohibitae sunt pluviae de caelo. 
\verse Reversaeque sunt aquae de terra euntes et redeuntes et coeperunt minui post centum quinquaginta dies. 
\verse Requievitque arca mense septimo, decima septima die mensis super montes Ararat. 
\verse At vero aquae ibant et decrescebant usque ad decimum mensem; decimo enim mense, prima die mensis, apparuerunt cacumina montium. 
\verse Cumque transissent quadraginta dies, aperiens Noe fenestram arcae, quam fecerat, dimisit corvum; 
\verse qui egrediebatur exiens et rediens, donec siccarentur aquae super terram. 
\verse Emisit quoque columbam a se, ut videret si iam cessassent aquae super faciem terrae. 
\verse Quae, cum non invenisset, ubi requiesceret pes eius, reversa est ad eum in arcam; aquae enim erant super universam terram. Extenditque manum et apprehensam intulit in arcam. 
\verse Exspectatis autem ultra septem diebus aliis, rursum dimisit columbam ex arca.  
\verse At illa venit ad eum ad vesperam portans ramum olivae virentibus foliis in ore suo. Intellexit ergo Noe quod cessassent aquae super terram. 
\verse Exspectavitque nihilominus septem alios dies; et emisit columbam, quae non est reversa ultra ad eum. 
\verse Igitur sescentesimo primo anno, primo mense, prima die mensis, siccatae sunt aquae super terram; et aperiens Noe tectum arcae, et ecce aspexit viditque quod exsiccata erat superficies terrae. 
\verse Mense secundo, septima et vicesima die mensis, arefacta est terra. 
\verse Locutus est autem Deus ad Noe dicens: 
\verse “Egredere de arca tu et uxor tua, filii tui et uxores filiorum tuorum tecum. 
\verse Cuncta animantia, quae sunt apud te ex omni carne, tam in volatilibus quam in pecoribus et in universis reptilibus, quae reptant super terram, educ tecum, ut pullulent super terram et crescant et multiplicentur super eam”. 
\verse Egressus est ergo Noe et filii eius, uxor illius et uxores filiorum eius cum eo. 
\verse Sed et omnia animantia, iumenta, volatilia et reptilia, quae reptant super terram, secundum genus suum egressa sunt de arca. 
\verse Aedificavit autem Noe altare Domino; et tollens de cunctis pecoribus mundis et volucribus mundis obtulit holocausta super altare. 
\verse Odoratusque est Dominus odorem suavitatis et locutus est Dominus ad cor suum: “Nequaquam ultra maledicam terrae propter homines, quia cogitatio humani cordis in malum prona est ab adulescentia sua. Non igitur ultra percutiam omnem animam viventem, sicut feci. 
\verse Cunctis diebus terrae, sementis et messis, frigus et aestus, aestas et hiems, dies et nox non requiescent”. 
\end{biblechapter}

\begin{biblechapter}  
\verse Benedixitque Deus Noe et filiis eius et dixit ad eos: “Crescite et multiplicamini et implete terram.
\verse Et terror vester ac tremor sit super cuncta animalia terrae et super omnes volucres caeli cum universis, quae moventur super terram; omnes pisces maris manui vestrae traditi sunt. 
\verse Omne, quod movetur et vivit, erit vobis in cibum; quasi holera virentia tradidi vobis omnia, 
\verse excepto quod carnem cum anima, quae est in sanguine, non comedetis. 
\verse Sanguinem enim animarum vestrarum requiram de manu cunctarum bestiarum; et de manu hominis, de manu viri fratris eius requiram animam hominis. 
\verse Quicumque effuderit humanum sanguinem, per hominem fundetur sanguis illius; ad imaginem quippe Dei factus est homo. 
\verse Vos autem crescite et multiplicamini et pullulate super terram et dominamini ei”. 
\verse Haec quoque dixit Deus ad Noe et ad filios eius cum eo: 
\verse “Ecce ego statuam pactum meum vobiscum et cum semine vestro post vos 
\verse et ad omnem animam viventem, quae est vobiscum tam in volucribus quam in iumentis et in omnibus bestiis terrae, quae sunt vobiscum, cunctis, quae egressa sunt de arca, universis bestiis terrae. 
\verse Statuam pactum meum vobiscum; et nequaquam ultra interficietur omnis caro aquis diluvii, neque erit deinceps diluvium dissipans terram”. 
\verse Dixitque Deus: “Hoc signum foederis, quod do inter me et vos et ad omnem animam viventem, quae est vobiscum, in generationes sempiternas: 
\verse arcum meum ponam in nubibus, et erit signum foederis inter me et inter terram. 
\verse Cumque obduxero nubibus caelum, apparebit arcus meus in nubibus, 
\verse et recordabor foederis mei vobiscum et cum omni anima vivente, quae carnem vegetat; et non erunt ultra aquae diluvii ad delendum universam carnem. 
\verse Eritque arcus in nubibus, et videbo illum et recordabor foederis sempiterni, quod pactum est inter Deum et omnem animam viventem universae carnis, quae est super terram”. 
\verse Dixitque Deus ad Noe: “Hoc erit signum foederis, quod constitui inter me et omnem carnem super terram”. 
\verse Erant ergo filii Noe, qui egressi sunt de arca, Sem, Cham et Iapheth. Porro Cham ipse est pater Chanaan. 
\verse Tres isti filii sunt Noe, et ab his disseminatum est omne hominum genus super universam terram.
\verse Coepitque Noe agricola plantare vineam; 
\verse bibensque vinum inebriatus est et nudatus in tabernaculo suo. 
\verse Quod cum vidisset Cham pater Chanaan, verenda scilicet patris sui esse nudata, nuntiavit duobus fratribus suis foras. 
\verse At vero Sem et Iapheth pallium imposuerunt umeris suis et incedentes retrorsum operuerunt verecunda patris sui, faciesque eorum aversae erant, et patris virilia non viderunt. 
\verse Evigilans autem Noe ex vino, cum didicisset, quae fecerat ei filius suus minor, 
\verse ait: Maledictus Chanaan! Servus servorum erit fratribus suis”. 
\verse Dixitque: Benedictus Dominus Deus Sem! Sitque Chanaan servus eius. 
\verse Dilatet Deus Iapheth, et habitet in tabernaculis Sem, sitque Chanaan servus eius”. 
\verse Vixit autem Noe post diluvium trecentis quinquaginta annis. 
\verse Et impleti sunt omnes dies eius nongentorum quinquaginta annorum, et mortuus est. 
\end{biblechapter}

\begin{biblechapter}  
\verse Hae sunt generationes filiorum Noe, Sem, Cham et Iapheth; natique sunt eis filii post diluvium.
\verse Filii Iapheth: Gomer et Magog et Madai et Iavan et Thubal et Mosoch et Thiras. 
\verse Porro filii Gomer: Aschenez et Riphath et Thogorma. 
\verse Filii autem Iavan: Elisa et Tharsis, Cetthim et Rodanim. 
\verse Ab his divisae sunt insulae gentium in regionibus suis, unusquisque secundum linguam suam et familias suas in nationibus suis. 
\verse Filii autem Cham: Chus et Mesraim et Phut et Chanaan. 
\verse Filii Chus: Saba et Hevila et Sabatha et Regma et Sabathacha. Filii Regma: Saba et Dedan. 
\verse Porro Chus genuit Nemrod: ipse coepit esse potens in terra 
\verse et erat robustus venator coram Domino. Ob hoc exivit proverbium: “Quasi Nemrod robustus venator coram Domino”. 
\verse Fuit autem principium regni eius Babylon et Arach et Achad et Chalanne in terra Sennaar. 
\verse De terra illa egressus est in Assyriam et aedificavit Nineven et Rohobothir et Chale, 
\verse Resen quoque inter Nineven et Chale; haec est civitas magna. 
\verse At vero Mesraim genuit Ludim et Anamim et Laabim, Nephthuim 
\verse et Phetrusim et Chasluim et Caphtorim, de quibus egressi sunt Philisthim. 
\verse Chanaan autem genuit Sidonem primogenitum suum, Hetthaeum 
\verse et Iebusaeum et Amorraeum, Gergesaeum, 
\verse Hevaeum et Aracaeum, Sinaeum 
\verse et Aradium, Samaraeum et Emathaeum; et post haec disseminati sunt populi Chananaeorum. 
\verse Factique sunt termini Chanaan venientibus a Sidone Geraram usque Gazam, donec ingrediaris Sodomam et Gomorram et Adamam et Seboim usque Lesa. 
\verse Hi sunt filii Cham in cognationibus et linguis terrisque et gentibus suis. 
\verse De Sem quoque nati sunt, patre omnium filiorum Heber, fratre Iapheth maiore.  
\verse Filii Sem: Elam et Assur et Arphaxad et Lud et Aram. 
\verse Filii Aram: Us et Hul et Gether et Mes. 
\verse At vero Arphaxad genuit Sala, de quo ortus est Heber. 
\verse Natique sunt Heber filii duo: nomen uni Phaleg, eo quod in diebus eius divisa sit terra, et nomen fratris eius Iectan. 
\verse Qui Iectan genuit Elmodad et Saleph et Asarmoth, Iare 
\verse et Adoram et Uzal et Decla 
\verse et Ebal et Abimael, Saba 
\verse et Ophir et Hevila et Iobab. Omnes isti filii Iectan; 
\verse et facta est habitatio eorum de Messa pergentibus usque Sephar montem orientalem. 
\verse Isti filii Sem secundum cognationes et linguas et regiones in gentibus suis. 
\verse Hae familiae filiorum Noe iuxta generationes et nationes suas. Ab his divisae sunt gentes in terra post diluvium. 
\end{biblechapter}

\begin{biblechapter}  
\verse Erat autem universa terra labii unius et sermonum eorundem. 
\verse Cumque proficiscerentur de oriente, invenerunt campum in terra Sennaar et habitaverunt in eo. 
\verse Dixitque alter ad proximum suum: “Venite, faciamus lateres et coquamus eos igni”. Habueruntque lateres pro saxis et bitumen pro caemento.  
\verse Et dixerunt: “Venite, faciamus nobis civitatem et turrim, cuius culmen pertingat ad caelum, et faciamus nobis nomen, ne dividamur super faciem universae terrae”. 
\verse Descendit autem Dominus, ut videret civitatem et turrim, quam aedificaverunt filii hominum, 
\verse et dixit Dominus: “Ecce unus est populus et unum labium omnibus; et hoc est initium operationis eorum, nec eis erit deinceps difficile, quidquid cogitaverint facere. 
\verse Venite igitur, descendamus et confundamus ibi linguam eorum, ut non intellegat unusquisque vocem proximi sui”. 
\verse Atque ita divisit eos Dominus ex illo loco super faciem universae terrae, et cessaverunt aedificare civitatem. 
\verse Et idcirco vocatum est nomen eius Babel, quia ibi confusum est labium universae terrae, et inde dispersit eos Dominus super faciem universae terrae. 
\verse Hae sunt generationes Sem. Sem centum erat annorum, quando genuit Arphaxad biennio post diluvium; 
\verse vixitque Sem, postquam genuit Arphaxad, quingentos annos et genuit filios et filias. 
\verse Porro Arphaxad vixit triginta quinque annos et genuit Sala. 
\verse Vixitque Arphaxad, postquam genuit Sala, quadringentis tribus annis et genuit filios et filias.
\verse Sala quoque vixit triginta annis et genuit Heber. 
\verse Vixitque Sala, postquam genuit Heber, quadringentis tribus annis et genuit filios et filias.
\verse Vixit autem Heber triginta quattuor annis et genuit Phaleg. 
\verse Et vixit Heber, postquam genuit Phaleg, quadringentis triginta annis et genuit filios et filias.
\verse Vixit quoque Phaleg triginta annis et genuit Reu. 
\verse Vixitque Phaleg, postquam genuit Reu, ducentis novem annis et genuit filios et filias. 
\verse Vixit autem Reu triginta duobus annis et genuit Seruch. 
\verse Vixitque Reu, postquam genuit Seruch, ducentis septem annis et genuit filios et filias. 
\verse Vixit vero Seruch triginta annis et genuit Nachor. 
\verse Vixitque Seruch, postquam genuit Nachor, ducentos annos et genuit filios et filias. 
\verse Vixit autem Nachor viginti novem annis et genuit Thare. 
\verse Vixitque Nachor, postquam genuit Thare, centum decem et novem annos et genuit filios et filias. 
\verse Vixitque Thare septuaginta annis et genuit Abram, Nachor et Aran. 
\verse Hae sunt autem generationes Thare. Thare genuit Abram, Nachor et Aran. Porro Aran genuit Lot; 
\verse mortuusque est Aran ante Thare patrem suum in terra nativitatis suae in Ur Chaldaeorum. 
\verse Duxerunt autem Abram et Nachor uxores: nomen uxoris Abram Sarai, et nomen uxoris Nachor Melcha, filia Aran patris Melchae et patris Ieschae. 
\verse Erat autem Sarai sterilis nec habebat liberos. 
\verse Tulitque Thare Abram filium suum et Lot filium Aran filium filii sui et Sarai nurum suam, uxorem Abram filii sui, et eduxit eos de Ur Chaldaeorum, ut irent in terram Chanaan. Veneruntque usque Charran et habitaverunt ibi. 
\verse Et facti sunt dies Thare ducentorum quinque annorum, et mortuus est in Charran. 
\end{biblechapter}

\begin{biblechapter}  
\verse Dixit autem Dominus ad Abram: “Egredere de terra tua et de cognatione tua et de domo patris tui in terram, quam monstrabo tibi. 
\verse Faciamque te in gentem magnam et benedicam tibi et magnificabo nomen tuum, erisque in benedictionem. 
\verse Benedicam benedicentibus tibi et maledicentibus tibi maledicam, atque in te benedicentur universae cognationes terrae!”. 
\verse Egressus est itaque Abram, sicut praeceperat ei Dominus, et ivit cum eo Lot. Septuaginta quinque annorum erat Abram, cum egrederetur de Charran. 
\verse Tulitque Sarai uxorem suam et Lot filium fratris sui universamque substantiam, quam acquisiverant, et animas, quas fecerant in Charran, et egressi sunt, ut irent in terram Chanaan; et venerunt in terram Chanaan. 
\verse Pertransivit Abram terram usque ad locum Sichem, usque ad Quercum Moreh. Chananaeus autem tunc erat in terra. 
\verse Apparuit autem Dominus Abram et dixit ei: “Semini tuo dabo terram hanc”. Qui aedificavit ibi altare Domino, qui apparuerat ei. 
\verse Et inde transgrediens ad montem, qui erat contra orientem Bethel, tetendit ibi tabernaculum suum ab occidente habens Bethel et ab oriente Hai; aedificavit quoque ibi altare Domino et invocavit nomen Domini. 
\verse Perrexitque Abram de mansione in mansionem usque ad Nageb. 
\verse Facta est autem fames in terra; descenditque Abram in Aegyptum, ut peregrinaretur ibi; praevaluerat enim fames in terra. 
\verse Cumque prope esset, ut ingrederetur Aegyptum, dixit Sarai uxori suae: “Novi quod pulchra sis mulier  
\verse et quod, cum viderint te Aegyptii, dicturi sunt: "Uxor ipsius est"; et interficient me et te reservabunt. 
\verse Dic ergo, obsecro te, quod soror mea sis, ut bene sit mihi propter te, et vivat anima mea ob gratiam tui”. 
\verse Cum itaque ingressus esset Abram Aegyptum, viderunt Aegyptii mulierem quod esset pulchra nimis, 
\verse et viderunt eam principes pharaonis et laudaverunt eam apud illum; et sublata est mulier in domum pharaonis. 
\verse Abram vero bene usus est propter illam; fueruntque ei oves et boves et asini et servi et famulae et asinae et cameli. 
\verse Flagellavit autem Dominus pharaonem plagis maximis et domum eius propter Sarai uxorem Abram. 
\verse Vocavitque pharao Abram et dixit ei: “Quidnam est hoc quod fecisti mihi? Quare non indicasti mihi quod uxor tua esset? 
\verse Quam ob causam dixisti esse sororem tuam, ut tollerem eam mihi in uxorem? Nunc igitur, ecce coniux tua: accipe eam et vade!”. 
\verse Praecepitque pharao super Abram viris; et deduxerunt eum et uxorem illius et omnia, quae habebat. 
\end{biblechapter}

\begin{biblechapter}  
\verse Ascendit ergo Abram de Aegypto ipse et uxor eius et omnia, quae habebat, et Lot cum eo ad Nageb. 
\verse Abram autem erat dives valde in pecoribus, argento et auro. 
\verse Et profectus est de mansione in mansionem a Nageb in Bethel usque ad locum, ubi prius fixerat tabernaculum inter Bethel et Hai, 
\verse in loco altaris, quod fecerat prius, et invocavit ibi nomen Domini. 
\verse Sed et Lot, qui ibat cum Abram, fuerunt greges ovium et armenta et tabernacula; 
\verse nec poterat eos capere terra, ut habitarent simul: erat quippe substantia eorum multa, et nequibant habitare communiter. 
\verse Unde et facta est rixa inter pastores gregum Abram et pastores gregum Lot. Eo autem tempore Chananaeus et Pherezaeus habitabant in illa terra. 
\verse Dixit ergo Abram ad Lot: “Ne, quaeso, sit iurgium inter me et te et inter pastores meos et pastores tuos: fratres enim sumus. 
\verse Nonne universa terra coram te est? Recede a me, obsecro: si ad sinistram ieris, ego dexteram tenebo; si tu dexteram elegeris, ego ad sinistram pergam”. 
\verse Elevatis itaque Lot oculis, vidit omnem circa regionem Iordanis, quae universa irrigabatur, antequam subverteret Dominus Sodomam et Gomorram, sicut paradisus Domini et sicut Aegyptus usque in Segor. 
\verse Elegitque sibi Lot omnem regionem circa Iordanem et recessit ad orientem; divisique sunt alterutrum a fratre suo. 
\verse Abram habitavit in terra Chanaan; Lot vero moratus est in oppidis, quae erant circa Iordanem, et tabernacula movit usque ad Sodomam.  
\verse Homines autem Sodomitae pessimi erant et peccatores coram Domino nimis. 
\verse Dixitque Dominus ad Abram, postquam divisus est Lot ab eo: “Leva oculos tuos et vide a loco, in quo nunc es, ad aquilonem et ad meridiem, ad orientem et ad occidentem: 
\verse omnem terram, quam conspicis, tibi dabo et semini tuo usque in sempiternum; 
\verse faciamque semen tuum sicut pulverem terrae: si quis potest hominum numerare pulverem terrae, semen quoque tuum numerare poterit.  
\verse Surge et perambula terram in longitudine et in latitudine sua, quia tibi daturus sum eam”. 
\verse Movens igitur tabernaculum suum, Abram venit et habitavit iuxta Quercus Mambre, quae sunt in Hebron, aedificavitque ibi altare Domino. 
\end{biblechapter}

\begin{biblechapter}  
\verse Factum est autem in illo tempore, ut Amraphel rex Sennaar et Arioch rex Ellasar et Chodorlahomor rex Elam et Thadal rex gentium 
\verse inirent bellum contra Bara regem Sodomae et contra Bersa regem Gomorrae et contra Sennaab regem Adamae et contra Semeber regem Seboim contraque regem Belae; ipsa est Segor. 
\verse Omnes hi convenerunt in vallem Siddim, quae nunc est mare Salis. 
\verse Duodecim annis servierant Chodorlahomor et tertio decimo anno recesserunt ab eo. 
\verse Igitur anno quarto decimo venit Chodorlahomor et reges, qui erant cum eo, percusseruntque Raphaim in Astharothcarnaim et Zuzim in Ham et Emim in Savecariathaim 
\verse et Chorraeos in montibus Seir usque ad Elpharan, quae est in deserto. 
\verse Reversique sunt et venerunt ad fontem Mesphat; ipsa est Cades. Et percusserunt omnem regionem Amalecitarum et etiam Amorraeum, qui habitabat in Asasonthamar. 
\verse Et egressi sunt rex Sodomae et rex Gomorrae rexque Adamae et rex Seboim necnon et rex Belae, quae est Segor; et direxerunt contra eos aciem in valle Siddim, 
\verse scilicet adversus Chodorlahomor regem Elam et Thadal regem gentium et Amraphel regem Sennaar et Arioch regem Ellasar: quattuor reges adversus quinque. 
\verse Vallis autem Siddim habebat puteos multos bituminis. Itaque rex Sodomae et Gomorrae terga verterunt cecideruntque illuc; et, qui remanserant, fugerunt ad montem. 
\verse Tulerunt autem omnem substantiam Sodomae et Gomorrae et universa, quae ad cibum pertinent, et abierunt; 
\verse ceperunt et Lot et substantiam eius, filium fratris Abram, qui habitabat in Sodoma. 
\verse Et ecce unus, qui evaserat, nuntiavit Abram Hebraeo, qui habitabat iuxta Quercus Mambre Amorraei fratris Eschol et fratris Aner; hi enim pepigerant foedus cum Abram. 
\verse Quod cum audisset Abram, captum videlicet Lot fratrem suum, numeravit expeditos vernaculos suos trecentos decem et octo et persecutus est usque Dan; 
\verse et, divisis sociis, irruit super eos nocte percussitque eos et persecutus est eos usque Hoba, quae est ad laevam Damasci; 
\verse reduxitque omnem substantiam, necnon et Lot fratrem suum cum substantia illius, mulieres quoque et populum. 
\verse Egressus est autem rex Sodomae in occursum eius, postquam reversus est a caede Chodorlahomor et regum, qui cum eo erant, in vallem Save, quae est vallis Regis. 
\verse At vero Melchisedech rex Salem proferens panem et vinum ­ erat enim sacerdos Dei altissimi ­ 
\verse benedixit ei et ait: “Benedictus Abram a Deo excelso, qui creavit caelum et terram 
\verse et benedictus Deus excelsus, qui tradidit hostes tuos in manus tuas”. Et dedit ei decimas ex omnibus. 
\verse Dixit autem rex Sodomae ad Abram: “Da mihi animas; substantiam tolle tibi”. 
\verse Qui respondit ei: “Levo manum meam ad Dominum, Deum excelsum, creatorem caeli et terrae,
\verse a filo subteminis usque ad corrigiam caligae non accipiam ex omnibus, quae tua sunt, ne dicas: "Ego ditavi Abram"; 
\verse exceptis his, quae comederunt iuvenes, et partibus virorum, qui venerunt mecum, Aner, Eschol et Mambre: isti accipient partes suas”. 
\end{biblechapter}

\begin{biblechapter}  
\verse His itaque transactis, factus est sermo Domini ad Abram per visionem dicens: “Noli timere, Abram! Ego protector tuus sum, et merces tua magna erit nimis”. 
\verse Dixitque Abram: “Domine Deus, quid dabis mihi? Ego vadam absque liberis, et heres domus meae erit Damascenus Eliezer". 
\verse Addiditque Abram: “En mihi non dedisti semen, et ecce vernaculus meus heres meus erit”. 
\verse Sed ecce sermo Domini factus est ad eum: “Non erit hic heres tuus, sed qui egredietur de visceribus tuis, ipsum habebis heredem". 
\verse Eduxitque eum foras et ait illi: “Suspice caelum et numera stellas, si potes”. Et dixit ei: “Sic erit semen tuum”. 
\verse Credidit Domino, et reputatum est ei ad iustitiam. 
\verse Dixitque ad eum: “Ego Dominus, qui eduxi te de Ur Chaldaeorum, ut darem tibi terram istam, et possideres eam”. 
\verse Et ille ait: “Domine Deus, unde scire possum quod possessurus sim eam?”. 
\verse Respondens Dominus: “Sume, inquit, mihi vitulam triennem et capram trimam et arietem annorum trium, turturem quoque et columbam”. 
\verse Qui tollens universa haec divisit ea per medium et utrasque partes contra se altrinsecus posuit; aves autem non divisit. 
\verse Descenderuntque volucres super cadavera, et abigebat eas Abram. 
\verse Cumque sol occumberet, sopor irruit super Abram, et ecce horror magnus et tenebrosus invasit eum. 
\verse Dictumque est ad eum: “Scito praenoscens quod peregrinum futurum sit semen tuum in terra non sua, et subicient eos servituti et affligent quadringentis annis. 
\verse Verumtamen et gentem, cui servituri sunt, ego iudicabo, et post haec egredientur cum magna substantia. 
\verse Tu autem ibis ad patres tuos in pace, sepultus in senectute bona. 
\verse Generatione autem quarta revertentur huc; necdum enim completae sunt iniquitates Amorraeorum usque ad praesens tempus”. 
\verse Cum ergo occubuisset sol, facta est caligo tenebrosa, et apparuit clibanus fumans et lampas ignis transiens inter divisiones illas. 
\verse In illo die pepigit Dominus cum Abram foedus dicens: “Semini tuo dabo terram hanc a fluvio Aegypti usque ad magnum fluvium Euphraten, 
\verse Cinaeos et Cenezaeos, Cedmonaeos 
\verse et Hetthaeos et Pherezaeos, Raphaim quoque 
\verse et Amorraeos et Chananaeos et Gergesaeos et Iebusaeos”. 
\end{biblechapter}

\begin{biblechapter}  
\verse Sarai autem uxor Abram non genuerat ei liberos; sed habens ancillam Aegyptiam nomine Agar, 
\verse dixit marito suo: “Ecce conclusit me Dominus, ne parerem; ingredere ad ancillam meam, si forte saltem ex illa suscipiam filios”. Cumque ille acquiesceret deprecanti, 
\verse tulit Agar Aegyptiam ancillam suam post annos decem quam habitare coeperant in terra Chanaan, et dedit eam viro suo uxorem. 
\verse Qui ingressus est ad eam. At illa concepisse se videns despexit dominam suam. 
\verse Dixitque Sarai ad Abram: “Inique agis contra me; ego dedi ancillam meam in sinum tuum, quae videns quod conceperit, despectui me habet. Iudicet Dominus inter me et te”. 
\verse Cui respondens Abram: “Ecce, ait, ancilla tua in manu tua est; utere ea, ut libet”. Affligente igitur eam Sarai, aufugit ab ea. 
\verse Cumque invenisset illam angelus Domini iuxta fontem aquae in deserto, ad fontem in via Sur, 
\verse dixit: “Agar, ancilla Sarai, unde venis et quo vadis?". Quae respondit: “A facie Sarai dominae meae ego fugio". 
\verse Dixitque ei angelus Domini: “Revertere ad dominam tuam et humiliare sub manibus ipsius".  
\verse Et dixit ei angelus Domini: “Multiplicans multiplicabo semen tuum, et non numerabitur prae multitudine".  
\verse Et dixit ei angelus Domini: “Ecce, concepisti et paries filium vocabisque nomen eius Ismael, eo quod audierit Dominus afflictionem tuam. 
\verse Hic erit homo onagro similis; manus eius contra omnes, et manus omnium contra eum; et e regione universorum fratrum suorum figet tabernacula". 
\verse Vocavit autem nomen Domini, qui loquebatur ad eam: “Tu Deus, qui vidisti me". Dixit enim: “Profecto hic vidi posteriora videntis me".  
\verse Propterea appellatur puteus ille Lahairoi (id est Viventis et Videntis me); ipse est inter Cades et Barad. 
\verse Peperitque Agar Abrae filium; qui vocavit nomen filii sui, quem pepererat Agar, Ismael.  
\verse Octoginta et sex annorum erat Abram, quando peperit ei Agar Ismaelem. 
\end{biblechapter}

\begin{biblechapter}  
\verse Postquam Abram nonaginta et novem annorum factus est, apparuit ei Dominus dixitque ad eum: “Ego Deus omnipotens, ambula coram me et esto perfectus. 
\verse Ponamque foedus meum inter me et te et multiplicabo te vehementer nimis”. 
\verse Cecidit Abram pronus in faciem. 
\verse Dixitque ei Deus: “Ecce pactum meum tecum. Erisque pater multarum gentium, 
\verse nec ultra vocabitur nomen tuum Abram, sed Abraham erit nomen tuum, quia patrem multarum gentium constitui te. 
\verse Faciamque te crescere vehementissime et ponam te in gentes; regesque ex te egredientur. 
\verse Et statuam pactum meum inter me et te et inter semen tuum post te in generationibus suis foedere sempiterno, ut sim Deus tuus et seminis tui post te. 
\verse Daboque tibi et semini tuo post te terram peregrinationis tuae, omnem terram Chanaan in possessionem aeternam; eroque Deus eorum”. 
\verse Dixit iterum Deus ad Abraham: “Tu autem pactum meum custodies, et semen tuum post te in generationibus suis. 
\verse Hoc est pactum meum, quod observabitis, inter me et vos et semen tuum post te. Circumcidetur ex vobis omne masculinum,  
\verse et circumcidetis carnem praeputii vestri, ut sit in signum foederis inter me et vos. 
\verse Infans octo dierum circumcidetur in vobis: omne masculinum in generationibus vestris, tam vernaculus quam empticius ex omnibus alienigenis, quicumque non fuerit de stirpe vestra. 
\verse Circumcidetur vernaculus et empticius, eritque pactum meum in carne vestra in foedus aeternum. 
\verse Masculus, cuius praeputii caro circumcisa non fuerit, delebitur anima illa de populo suo; pactum meum irritum fecit”. 
\verse Dixit quoque Deus ad Abraham: “Sarai uxorem tuam non vocabis nomen eius Sarai, sed Sara erit nomen eius. 
\verse Et benedicam ei; et ex illa quoque dabo tibi filium. Benedicturus sum eam, eritque in nationes; reges populorum orientur ex ea”. 
\verse Cecidit Abraham in faciem suam et risit dicens in corde suo: “Putasne centenario nascetur filius? Et Sara nonagenaria pariet?”. 
\verse Dixitque ad Deum: “Utinam Ismael vivat coram te”. 
\verse Et ait Deus: “Sara uxor tua pariet tibi filium, vocabisque nomen eius Isaac; et constituam pactum meum illi in foedus sempiternum et semini eius post eum. 
\verse Super Ismael quoque exaudivi te: ecce benedicam ei et crescere faciam et multiplicabo eum vehementissime; duodecim duces generabit, et faciam illum in gentem magnam.  
\verse Pactum vero meum statuam ad Isaac, quem pariet tibi Sara tempore isto in anno altero”. 
\verse Cumque cessasset loqui cum eo, ascendit Deus ab Abraham. 
\verse Tulit ergo Abraham Ismael filium suum et omnes vernaculos domus suae universosque, quos emerat: cunctos mares ex omnibus viris domus suae; et circumcidit carnem praeputii eorum statim in ipsa die, sicut praeceperat ei Deus. 
\verse Abraham nonaginta novem erat annorum, quando circumcisus est in carne praeputii sui; 
\verse et Ismael filius eius tredecim annos impleverat tempore circumcisionis suae. 
\verse Eadem die circumcisus est Abraham et Ismael filius eius; 
\verse et omnes viri domus illius, tam vernaculi quam empticii ex alienigenis, circumcisi sunt cum eo. 
\end{biblechapter}

\begin{biblechapter}  
\verse Apparuit autem ei Dominus iuxta Quercus Mambre sedenti in ostio tabernaculi sui in ipso fervore diei. 
\verse Cumque elevasset oculos, apparuerunt ei tres viri stantes prope eum. Quos cum vidisset, cucurrit in occursum eorum de ostio tabernaculi et adoravit in terram 
\verse et dixit: “Domine mi, si inveni gratiam in oculis tuis, ne transeas servum tuum; 
\verse afferatur pauxillum aquae, et lavate pedes vestros et requiescite sub arbore. 
\verse Ponamque buccellam panis, et confortate cor vestrum, postea transibitis; idcirco enim declinastis ad servum vestrum”. Qui dixerunt: “Fac ut locutus es”. 
\verse Festinavit Abraham in tabernaculum ad Saram dixitque: “Accelera, tria sata similae commisce et fac subcinericios panes”. 
\verse Ipse vero ad armentum cucurrit et tulit inde vitulum tenerrimum et optimum deditque puero; qui festinavit et coxit illum. 
\verse Tulit quoque butyrum et lac et vitulum, quem coxerat, et posuit coram eis. Ipse vero stabat iuxta eos sub arbore; et comederunt. 
\verse Dixeruntque ad eum: “Ubi est Sara uxor tua?”. Ille respondit: “Ecce in tabernaculo est”. 
\verse Cui dixit: “Revertens veniam ad te tempore isto, et habebit filium Sara uxor tua”. Quo audito, Sara risit ad ostium tabernaculi, quod erat post eum. 
\verse Erant autem ambo senes provectaeque aetatis, et desierant Sarae fieri muliebria. 
\verse Quae risit occulte dicens: “Postquam consenui, et dominus meus vetulus est, voluptas mihi erit?”. 
\verse Dixit autem Dominus ad Abraham: “Quare risit Sara dicens: "Num vere paritura sum anus?"”. 
\verse Numquid Domino est quidquam difficile? Revertar ad te hoc eodem tempore, et habebit Sara filium". 
\verse Negavit Sara dicens: “Non risi”, timore perterrita. Ille autem dixit: “Non; sed risisti”. 
\verse Cum ergo surrexissent inde viri, direxerunt oculos contra Sodomam; et Abraham simul gradiebatur deducens eos.  
\verse Dixitque Dominus: “Num celare potero Abraham, quae gesturus sum, 
\verse cum futurus sit in gentem magnam ac robustissimam, et benedicendae sint in illo omnes nationes terrae? 
\verse Nam elegi eum, ut praecipiat filiis suis et domui suae post se, ut custodiant viam Domini et faciant iustitiam et iudicium, ut adducat Dominus super Abraham omnia, quae locutus est ad eum”.
\verse Dixit itaque Dominus: “Clamor contra Sodomam et Gomorram multiplicatus est, et peccatum eorum aggravatum est nimis. 
\verse Descendam et videbo utrum clamorem, qui venit ad me, opere compleverint an non; sciam”. 
\verse Converteruntque se inde viri et abierunt Sodomam; Abraham vero adhuc stabat coram Domino.
\verse Et appropinquans ait: “Numquid vere perdes iustum cum impio?  
\verse Si forte fuerint quinquaginta iusti in civitate, vere perdes et non parces loco illi propter quinquaginta iustos, si fuerint in eo? 
\verse Absit a te, ut rem hanc facias et occidas iustum cum impio, fiatque iustus sicut impius; absit a te. Nonne iudex universae terrae faciet iudicium?”. 
\verse Dixitque Dominus: “Si invenero Sodomae quinquaginta iustos in medio civitatis, dimittam omni loco propter eos”. 
\verse Respondensque Abraham ait: “Ecce coepi loqui ad Dominum meum, cum sim pulvis et cinis. 
\verse Quid, si forte minus quinquaginta iustis quinque fuerint? Delebis propter quinque universam urbem?”. Et ait: “Non delebo, si invenero ibi quadraginta quinque”. 
\verse Rursumque locutus est ad eum: “Si forte inventi fuerint ibi quadraginta?”. Ait: “Non percutiam propter quadraginta”.  
\verse “Ne, quaeso, inquit, indignetur Dominus meus, si loquar. Si forte ibi inventi fuerint triginta?”. Respondit: “Non faciam, si invenero ibi triginta”.  
\verse “Ecce, ait, coepi loqui ad Dominum meum. Si forte inventi fuerint ibi viginti?". Dixit: “Non interficiam propter viginti”.  
\verse “Obsecro, inquit, ne irascatur Dominus meus, si loquar adhuc semel. Si forte inventi fuerint ibi decem?”. Dixit: “Non delebo propter decem”. 
\verse Abiit Dominus, postquam cessavit loqui ad Abraham; et ille reversus est in locum suum. 
\end{biblechapter}

\begin{biblechapter}  
\verse Veneruntque duo angeli Sodomam vespere, sedente Lot in foribus civitatis. Qui cum vidisset eos, surrexit et ivit obviam eis adoravitque pronus in terram  
\verse et dixit: “Obsecro, domini mei, declinate in domum pueri vestri et pernoctate; lavate pedes vestros et mane proficiscemini in viam vestram”. Qui dixerunt: “Minime, sed in platea pernoctabimus”.  
\verse Compulit illos oppido, et diverterunt ad eum. Ingressisque domum illius fecit convivium et coxit azyma, et comederunt. 
\verse Prius autem quam irent cubitum, viri civitatis, viri Sodomae, vallaverunt domum a iuvene usque ad senem, omnis populus simul. 
\verse Vocaveruntque Lot et dixerunt ei: “Ubi sunt viri, qui introierunt ad te nocte? Educ illos ad nos, ut cognoscamus eos”. 
\verse Egressus ad eos Lot post tergum occludens ostium ait: 
\verse “Nolite, quaeso, fratres mei, nolite malum hoc facere. 
\verse Ecce, habeo duas filias, quae necdum cognoverunt virum; educam eas ad vos, et facite eis sicut placuerit vobis, dummodo viris istis nihil faciatis; ideo enim ingressi sunt sub umbra tecti mei”. 
\verse At illi dixerunt: “Recede illuc”. Et rursus: “Unus ingressus est, inquiunt, ut advena et vult iudicare? Te ergo ipsum magis quam hos affligemus”. Vimque faciebant Lot vehementissime, iamque prope erat, ut effringerent fores. 
\verse Et ecce miserunt manum viri et introduxerunt ad se Lot clauseruntque ostium; 
\verse et eos, qui foris erant, percusserunt caecitate a minimo usque ad maximum, ita ut ostium invenire non possent. 
\verse Dixerunt autem viri ad Lot: “Habes hic quempiam tuorum? Generum et filios et filias et omnes, qui tui sunt in urbe, educ de loco hoc: 
\verse delebimus enim locum istum, eo quod increverit clamor contra eos coram Domino, qui misit nos, ut perdamus eam”. 
\verse Egressus itaque Lot locutus est ad generos suos, qui accepturi erant filias eius, et dixit: “Surgite, egredimini de loco isto, quia delebit Dominus civitatem”. Et visus est eis quasi ludens loqui. 
\verse Cumque esset mane, cogebant eum angeli dicentes: “Surge, tolle uxorem tuam et duas filias, quas habes hic, ne pereas in scelere civitatis”. 
\verse Tardante illo, apprehenderunt viri manum eius et manum uxoris ac duarum filiarum eius, eo quod parceret Dominus illi. 
\verse Et eduxerunt eum posueruntque extra civitatem. Ibi locutus est: “Salvare, agitur de vita tua; noli respicere post tergum, nec stes in omni circa regione; sed in monte salvum te fac, ne pereas”.  
\verse Dixitque Lot ad eos: “Non, quaeso, Domine. 
\verse Ecce invenit servus tuus gratiam coram te, et magnificasti misericordiam tuam, quam fecisti mecum, ut salvares animam meam; nec possum in monte salvari, ne forte apprehendat me malum et moriar. 
\verse Ecce, civitas haec iuxta, ad quam possum fugere, parva, et salvabor in ea ­ numquid non modica est? ­ et vivet anima mea”. 
\verse Dixitque ad eum: “Ecce, etiam in hoc suscepi preces tuas, ut non subvertam urbem, pro qua locutus es. 
\verse Festina et salvare ibi, quia non potero facere quidquam, donec ingrediaris illuc”. Idcirco vocatum est nomen urbis illius Segor. 
\verse Sol egressus est super terram, et Lot ingressus est Segor. 
\verse Igitur Dominus pluit super Sodomam et Gomorram sulphur et ignem a Domino de caelo  
\verse et subvertit civitates has et omnem circa regionem, universos habitatores urbium et cuncta terrae virentia. 
\verse Respiciensque uxor eius post se versa est in statuam salis. 
\verse Abraham autem consurgens mane venit ad locum, ubi steterat prius cum Domino, 
\verse intuitus est Sodomam et Gomorram et universam terram regionis illius; viditque ascendentem favillam de terra quasi fornacis fumum. 
\verse Cum enim subverteret Deus civitates regionis illius, recordatus Abrahae liberavit Lot de subversione urbium, in quibus habitaverat. 
\verse Ascenditque Lot de Segor et mansit in monte, duae quoque filiae eius cum eo; timuerat enim manere in Segor. Et mansit in spelunca ipse et duae filiae eius. 
\verse Dixitque maior ad minorem: “Pater noster senex est, et nullus virorum remansit in terra, qui possit ingredi ad nos iuxta morem universae terrae.  
\verse Veni, inebriemus patrem nostrum vino dormiamusque cum eo, ut servare possimus ex patre nostro semen”. 
\verse Dederunt itaque patri suo bibere vinum nocte illa, et ingressa est maior dormivitque cum patre; at ille non sensit, nec quando accubuit filia nec quando surrexit. 
\verse Altera quoque die dixit maior ad minorem: “Ecce, dormivi heri cum patre meo; demus ei bibere vinum etiam hac nocte, et ingressa dormies cum eo, ut salvemus semen de patre nostro”. 
\verse Dederunt et illa nocte patri suo bibere vinum, ingressaque minor filia dormivit cum eo; et ne tunc quidem sensit, quando illa concubuerit vel quando surrexerit.  
\verse Conceperunt ergo duae filiae Lot de patre suo. 
\verse Peperitque maior filium et vocavit nomen eius Moab; ipse est pater Moabitarum usque in praesentem diem.  
\verse Minor quoque peperit filium et vocavit nomen eius Benammi (id est Filius populi mei); ipse est pater Ammonitarum usque hodie. 
\end{biblechapter}

\begin{biblechapter}  
\verse Profectus inde Abraham in terram Nageb, habitavit inter Cades et Sur et peregrinatus est in Geraris. 
\verse Dixitque de Sara uxore sua: “Soror mea est”. Misit ergo Abimelech rex Gerarae et tulit eam. 
\verse Venit autem Deus ad Abimelech per somnium nocte et ait illi: “En morieris propter mulierem, quam tulisti; habet enim virum”. 
\verse Abimelech vero non tetigerat eam. Et ait: “Domine, num gentem etiam iustam interficies? 
\verse Nonne ipse dixit mihi: "Soror mea est", et ipsa quoque ait: "Frater meus est"? In simplicitate cordis mei et munditia manuum mearum feci hoc”. 
\verse Dixitque ad eum Deus per somnium: “Et ego scio quod simplici corde feceris; et ideo custodivi te, ne peccares in me, et non dimisi, ut tangeres eam. 
\verse Nunc igitur redde viro suo uxorem, quia propheta est; et orabit pro te, et vives. Si autem nolueris reddere, scito quod morte morieris tu et omnia, quae tua sunt”. 
\verse Statimque de nocte consurgens Abimelech vocavit omnes servos suos et locutus est universa verba haec in auribus eorum; timueruntque viri valde. 
\verse Vocavit autem Abimelech etiam Abraham et dixit ei: “Quid fecisti nobis? Quid peccavi in te, quia induxisti super me et super regnum meum peccatum grande? Quae non debuisti facere, fecisti mihi”. 
\verse Rursusque ait: “Quid vidisti, ut hoc faceres?”. 
\verse Respondit Abraham: “Cogitavi mecum: Certe non est timor Dei in loco isto, et interficient me propter uxorem meam. 
\verse Alias autem et vere soror mea est, filia patris mei et non filia matris meae, et duxi eam in uxorem. 
\verse Cum autem vagari me faceret Deus de domo patris mei, dixi ad eam: Hanc misericordiam facies mecum: in omni loco, ad quem ingrediemur, dices quod frater tuus sim". 
\verse Tulit igitur Abimelech oves et boves et servos et ancillas et dedit Abraham; reddiditque illi Saram uxorem suam 
\verse et ait: “Ecce terra mea coram te; ubicumque tibi placuerit, habita”. 
\verse Sarae autem dixit: “Ecce mille argenteos dedi fratri tuo; ecce hoc erit tibi in velamen oculorum ad omnes, qui tecum sunt, et apud omnes iustificaberis”. 
\verse Orante autem Abraham, sanavit Deus Abimelech et uxorem ancillasque eius et pepererunt; 
\verse concluserat enim Dominus omnem vulvam domus Abimelech propter Saram uxorem Abraham. 
\end{biblechapter}

\begin{biblechapter}  
\verse Visitavit autem Dominus Saram, sicut promiserat, et implevit Sarae, quae locutus est; 
\verse concepitque et peperit Abrahae filium in senectute eius tempore, quo praedixerat ei Deus. 
\verse Vocavitque Abraham nomen filii sui, quem genuit ei Sara, Isaac 
\verse et circumcidit eum octavo die, sicut praeceperat ei Deus. 
\verse Cum Abraham centum esset annorum, natus est ei Isaac filius eius.  
\verse Dixitque Sara: “Risum fecit mihi Deus; quicumque audierit, corridebit mihi ?. 
\verse Rursumque ait: ? Quis auditurum crederet Abraham quod Sara lactaret filios, quia peperit ei filium iam seni?”. 
\verse Crevit igitur puer et ablactatus est. Fecitque Abraham grande convivium in die ablactationis eius. 
\verse Cumque vidisset Sara filium Agar Aegyptiae iocantem cum Isaac filio suo, dixit ad Abraham: 
\verse “Eice ancillam hanc et filium eius; non enim erit heres filius ancillae cum filio meo Isaac”. 
\verse Dure accepit hoc Abraham propter filium suum. 
\verse Cui dixit Deus: “Non tibi videatur asperum super puero et super ancilla tua; omnia, quae dixerit tibi Sara, audi vocem eius, quia in Isaac vocabitur tibi semen. 
\verse Sed et filium ancillae faciam in gentem magnam, quia semen tuum est”. 
\verse Surrexit itaque Abraham mane et tollens panem et utrem aquae imposuit scapulae eius tradiditque puerum et dimisit eam. Quae cum abisset, errabat in deserto Bersabee. 
\verse Cumque consumpta esset aqua in utre, abiecit puerum subter unum arbustum 
\verse et abiit; seditque e regione procul, quantum potest arcus iacere. Dixit enim: “Non videbo morientem puerum”. Et sedens contra levavit vocem suam et flevit. 
\verse Exaudivit autem Deus vocem pueri; vocavitque angelus Dei Agar de caelo dicens: “Quid tibi, Agar? Noli timere; exaudivit enim Deus vocem pueri de loco, in quo est. 
\verse Surge, tolle puerum et tene illum manu tua, quia in gentem magnam faciam eum”. 
\verse Aperuitque Deus oculos eius; quae videns puteum aquae abiit et implevit utrem deditque puero bibere. 
\verse Et fuit Deus cum eo; qui crevit et moratus est in solitudine factusque est iuvenis sagittarius. 
\verse Habitavitque in deserto Pharan; et accepit illi mater sua uxorem de terra Aegypti. 
\verse Eodem tempore dixit Abimelech et Phicol princeps exercitus eius ad Abraham: “Deus tecum est in universis, quae agis. 
\verse Iura ergo per Deum, ne noceas mihi et posteris meis stirpique meae; sed iuxta fidem, quam feci tibi, facies mihi et terrae, in qua versatus es advena”. 
\verse Dixitque Abraham: “Ego iurabo”. 
\verse Et increpavit Abraham Abimelech propter puteum aquae, quem vi abstulerant servi eius. 
\verse Responditque Abimelech: “Nescivi quis fecerit hanc rem; sed et tu non indicasti mihi, et ego non audivi praeter hodie”. 
\verse Tulit itaque Abraham oves et boves et dedit Abimelech; percusseruntque ambo foedus. 
\verse Et statuit Abraham septem agnas gregis seorsum. 
\verse Cui dixit Abimelech: “Quid sibi volunt septem agnae istae, quas stare fecisti seorsum?”. 
\verse At ille: “Septem, inquit, agnas accipies de manu mea, ut sint in testimonium mihi, quoniam ego fodi puteum istum”. 
\verse Idcirco vocatus est locus ille Bersabee, quia ibi uterque iuraverunt. 
\verse Et inierunt foedus in Bersabee. 
\verse Surrexit autem Abimelech et Phicol princeps militiae eius reversique sunt in terram Philisthim. Abraham vero plantavit nemus in Bersabee et invocavit ibi nomen Domini, Dei aeterni. 
\verse Et fuit colonus in terra Philisthim diebus multis. 
\end{biblechapter}

\begin{biblechapter}  
\verse Quae postquam gesta sunt, tentavit Deus Abraham et dixit ad eum: “Abraham”. Ille respondit: “Adsum”. 
\verse Ait: “Tolle filium tuum unigenitum, quem diligis, Isaac et vade in terram Moria; atque offer eum ibi in holocaustum super unum montium, quem monstravero tibi”. 
\verse Igitur Abraham de nocte consurgens stravit asinum suum ducens secum duos iuvenes suos et Isaac filium suum. Cumque concidisset ligna in holocaustum, surrexit et abiit ad locum, quem praeceperat ei Deus. 
\verse Die autem tertio, elevatis oculis, vidit locum procul 
\verse dixitque ad pueros suos: “Exspectate hic cum asino. Ego et puer illuc usque properantes, postquam adoraverimus, revertemur ad vos”. 
\verse Tulit quoque ligna holocausti et imposuit super Isaac filium suum; ipse vero portabat in manibus ignem et cultrum. Cumque duo pergerent simul, 
\verse dixit Isaac Abrahae patri suo: “Pater mi”. Ille respondit: “Quid vis, fili?”. “Ecce, inquit, ignis et ligna; ubi est victima holocausti?”. 
\verse Dixit Abraham: “Deus providebit sibi victimam holocausti, fili mi”. Pergebant ambo pariter; 
\verse et venerunt ad locum, quem ostenderat ei Deus, in quo aedificavit Abraham altare et desuper ligna composuit. Cumque colligasset Isaac filium suum, posuit eum in altari super struem lignorum 
\verse extenditque Abraham manum et arripuit cultrum, ut immolaret filium suum. 
\verse Et ecce angelus Domini de caelo clamavit: “Abraham, Abraham”. Qui respondit: “Adsum”.  
\verse Dixitque: “Non extendas manum tuam super puerum neque facias illi quidquam. Nunc cognovi quod times Deum et non pepercisti filio tuo unigenito propter me”.  
\verse Levavit Abraham oculos suos viditque arietem unum inter vepres haerentem cornibus; quem assumens obtulit holocaustum pro filio. 
\verse Appellavitque nomen loci illius: “Dominus videt”. Unde usque hodie dicitur: “In monte Dominus videtur”. 
\verse Vocavit autem angelus Domini Abraham secundo de caelo et dixit: 
\verse “Per memetipsum iuravi, dicit Dominus: quia fecisti hanc rem et non pepercisti filio tuo unigenito, 
\verse benedicam tibi et multiplicabo semen tuum sicut stellas caeli et velut arenam, quae est in litore maris. Possidebit semen tuum portas inimicorum suorum, 
\verse et benedicentur in semine tuo omnes gentes terrae, quia oboedisti voci meae”. 
\verse Reversus est Abraham ad pueros suos, et surrexerunt abieruntque Bersabee simul, et habitavit Abraham in Bersabee. 
\verse His ita gestis, nuntiatum est Abrahae quod Melcha quoque genuisset filios Nachor fratri suo: 
\verse Us primogenitum et Buz fratrem eius et Camuel patrem Aram 
\verse et Cased et Azau, Pheldas quoque et Iedlaph 
\verse ac Bathuel, de quo nata est Rebecca. Octo istos genuit Melcha Nachor fratri Abrahae.  
\verse Concubina vero illius, nomine Reuma, peperit Tabee et Gaham et Tahas et Maacha. 
\end{biblechapter}

\begin{biblechapter}  
\verse Vixit autem Sara centum viginti septem annis 
\verse et mortua est in Cariatharbe, quae est Hebron, in terra Chanaan; venitque Abraham, ut plangeret et fleret eam. 
\verse Cumque surrexisset ab officio funeris, locutus est ad filios Heth dicens: 
\verse “Advena sum et inquilinus apud vos; date mihi possessionem sepulcri vobiscum, ut sepeliam mortuum meum”. 
\verse Responderunt filii Heth dicentes: 
\verse “Audi nos, domine, princeps Dei es apud nos: in nobilissimo sepulcrorum nostrorum sepeli mortuum tuum; nullusque te prohibebit, quin in sepulcro eius sepelias mortuum tuum”. 
\verse Surrexit Abraham et adoravit populum terrae, filios videlicet Heth, 
\verse dixitque ad eos: “Si placet animae vestrae, ut sepeliam mortuum meum, audite me et intercedite pro me apud Ephron filium Seor, 
\verse ut det mihi speluncam Machpela, quam habet in extrema parte agri sui. Pecunia digna tradat eam mihi coram vobis in possessionem sepulcri”. 
\verse Sedebat autem Ephron in medio filiorum Heth. Responditque Ephron Hetthaeus ad Abraham, filiis Heth audientibus cunctis, qui ingrediebantur portam civitatis illius, dicens: 
\verse “Nequaquam ita fiat, domine mi, ausculta me. Agrum do tibi et speluncam, quae in eo est, praesentibus filiis populi mei; sepeli mortuum tuum”. 
\verse Adoravit Abraham coram populo terrae 
\verse et locutus est ad Ephron, audiente populo terrae: “Quaeso, ut audias me. Dabo pecuniam pro agro; suscipe eam, et sic sepeliam mortuum meum in eo”. 
\verse Respondit Ephron ad Abraham dicens ei: 
\verse “Domine mi, audi me. Terra quadringentorum siclorum argenti inter me et te quid est hoc? Sepeli mortuum tuum”. 
\verse Auscultavit Abraham Ephron et appendit pecuniam, quam Ephron postulaverat, audientibus filiis Heth, quadringentos siclos argenti, sicut mos erat apud negotiatores. 
\verse Confirmatusque est ager Ephronis, qui erat in Machpela respiciens Mambre, tam ipse quam spelunca in eo et omnes arbores eius in cunctis terminis eius per circuitum, 
\verse Abrahae in possessionem, videntibus filiis Heth cunctis, qui intrabant portam civitatis illius. 
\verse Deinde sepelivit Abraham Saram uxorem suam in spelunca agri Machpela, qui respiciebat Mambre ­ haec est Hebron ­ in terra Chanaan. 
\verse Et confirmatus est ager et antrum, quod erat in eo, Abrahae in possessionem sepulcri a filiis Heth. 
\end{biblechapter}

\begin{biblechapter}  
\verse Erat autem Abraham senex dierumque multorum; et Dominus in cunctis benedixerat ei. 
\verse Dixitque Abraham ad servum seniorem domus suae, qui praeerat omnibus, quae habebat: “Pone manum tuam subter femur meum, 
\verse ut adiurem te per Dominum, Deum caeli et Deum terrae, ut non accipias uxorem filio meo de filiabus Chananaeorum, inter quos habito; 
\verse sed ad terram et cognationem meam proficiscaris et inde accipias uxorem filio meo Isaac”. 
\verse Respondit servus: “Si noluerit mulier venire mecum in terram hanc, num reducere debeo filium tuum ad terram, a quo tu egressus es?”. 
\verse Dixit Abraham: “Cave, ne quando reducas illuc filium meum. 
\verse Dominus, Deus caeli, qui tulit me de domo patris mei et de terra nativitatis meae, qui locutus est mihi et iuravit mihi dicens: "Semini tuo dabo terram hanc", ipse mittet angelum suum coram te, et accipies inde uxorem filio meo. 
\verse Sin autem noluerit mulier sequi te, non teneberis iuramento; filium tantum meum ne reducas illuc”.  
\verse Posuit ergo servus manum sub femore Abraham domini sui et iuravit illi super hac re. 
\verse Tulitque servus decem camelos de grege domini sui et abiit ex omnibus bonis eius portans secum; profectusque perrexit in Aram Naharaim ad urbem Nachor. 
\verse Cumque camelos fecisset accumbere extra oppidum iuxta puteum aquae vespere, tempore quo solent mulieres egredi ad hauriendam aquam, dixit: 
\verse “Domine,Deus domini mei Abraham, occurre obsecro mihi hodie et fac misericordiam cum domino meo Abraham. 
\verse Ecce ego sto prope fontem aquae, et filiae habitatorum huius civitatis egredientur ad hauriendam aquam. 
\verse Igitur puella, cui ego dixero: "Inclina hydriam tuam, ut bibam", et illa responderit: "Bibe, quin et camelis tuis dabo potum", ipsa est, quam praeparasti servo tuo Isaac, et per hoc intellegam quod feceris misericordiam cum domino meo”. 
\verse Necdum intra se verba compleverat, et ecce Rebecca egrediebatur filia Bathuel filii Melchae uxoris Nachor fratris Abraham habens hydriam in scapula: 
\verse puella decora nimis, virgo et incognita viro. Descendit ad fontem et implevit hydriam ac revertebatur. 
\verse Occurritque ei servus et ait: “Pauxillum mihi ad sorbendum praebe aquae de hydria tua”. 
\verse Quae respondit: “Bibe, domine mi”. Celeriterque deposuit hydriam super ulnam suam et dedit ei potum. 
\verse Cumque ille bibisset, adiecit: “Quin et camelis tuis hauriam aquam, donec cuncti bibant”. 
\verse Effundensque hydriam in canalibus recurrit ad puteum, ut hauriret aquam; et haustam omnibus camelis dedit. 
\verse Ille autem contemplabatur eam tacitus, scire volens utrum prosperum fecisset iter suum Dominus an non. 
\verse Postquam ergo biberunt cameli, protulit vir anulum aureum pondo dimidii sicli pro naribus et duas armillas pro manibus eius pondo siclorum decem; 
\verse dixitque: “Cuius es filia? lndica mihi. Est in domo patris tui locus nobis ad pernoctandum?”. 
\verse Quae respondit: “Filia Bathuelis sum filii Melchae, quem peperit Nachor”. 
\verse Et addidit dicens: “Palearum quoque et pabuli plurimum est apud nos et locus ad pernoctandum”. 
\verse Inclinavit se homo et adoravit Dominum 
\verse dicens: “Benedictus Dominus, Deus domini mei Abraham, qui non abstulit misericordiam et veritatem suam a domino meo et recto itinere me perduxit in domum fratris domini mei”. 
\verse Cucurrit itaque puella et nuntiavit in domum matris suae omnia, quae evenerant. 
\verse Habebat autem Rebecca fratrem nomine Laban, qui festinus egressus est ad hominem, ubi erat fons. 
\verse Cumque vidisset anulum in naribus et armillas in manibus sororis suae et audisset cuncta verba referentis: “Haec locutus est mihi homo”, venit ad virum, qui stabat iuxta camelos et prope fontem aquae; 
\verse dixitque ad eum: “Ingredere, benedicte Domini, cur foris stas? Praeparavi domum et locum camelis”. 
\verse Et introduxit eum in hospitium ac destravit camelos; deditque paleas et pabulum camelis et aquam ad lavandos pedes eius et virorum, qui venerant cum eo. 
\verse Et apposuit in conspectu eius panem. Qui ait: “Non comedam, donec loquar sermones meos”. Respondit: “Loquere”. 
\verse At ille: “Servus, inquit, Abraham sum; 
\verse et Dominus benedixit domino meo valde, magnificatusque est; et dedit ei oves et boves, argentum et aurum, servos et ancillas, camelos et asinos. 
\verse Et peperit Sara uxor domini mei filium domino meo in senectute sua; deditque illi omnia, quae habuerat. 
\verse Et adiuravit me dominus meus dicens: "Non accipies uxorem filio meo de filiabus Chananaeorum, in quorum terra habito; 
\verse sed ad domum patris mei perges et de cognatione mea accipies uxorem filio meo". 
\verse Ego vero respondi domino meo: Quid, si noluerit venire mecum mulier? 
\verse "Dominus, ait, in cuius conspectu ambulo, mittet angelum suum tecum et diriget viam tuam; accipiesque uxorem filio meo de cognatione mea et de domo patris mei. 
\verse Innocens eris a maledictione mea, cum veneris ad propinquos meos, et non dederint tibi; tunc innocens eris a maledictione mea". 
\verse Veni ergo hodie ad fontem et dixi: Domine, Deus domini mei Abraham, si direxisti viam meam, in qua nunc ambulo, 
\verse ecce sto iuxta fontem aquae; et virgo, quae egredietur ad hauriendam aquam, audierit a me: "Da mihi pauxillum aquae ad bibendum ex hydria tua"; 
\verse et dixerit mihi: "Et tu bibe, et camelis tuis hauriam", ipsa est mulier, quam praeparavit Dominus filio domini mei. 
\verse Dum haec tacitus mecum volverem, apparuit Rebecca veniens cum hydria, quam portabat in scapula; descenditque ad fontem et hausit aquam. Et aio ad eam: Da mihi paululum bibere. 
\verse Quae festina deposuit hydriam de umero et dixit mihi: "Et tu bibe, et camelis tuis potum tribuam". Bibi, et adaquavit camelos.  
\verse Interrogavique eam et dixi: Cuius es filia? Quae respondit: "Filia Bathuelis sum filii Nachor, quem peperit illi Melcha". Suspendi itaque anulum in naribus eius et armillas posui in manibus eius. 
\verse Pronusque adoravi Dominum benedicens Domino, Deo domini mei Abraham, qui perduxit me recto itinere, ut sumerem filiam fratris domini mei filio eius.  
\verse Quam ob rem, si facitis misericordiam et veritatem cum domino meo, indicate mihi; sin autem aliud placet, et hoc dicite mihi, ut vadam ad dexteram sive ad sinistram”. 
\verse Responderunt Laban et Bathuel: “A Domino egressus est sermo; non possumus extra placitum eius quidquam aliud loqui tecum. 
\verse En Rebecca coram te est; tolle eam et proficiscere, et sit uxor filii domini tui, sicut locutus est Dominus”. 
\verse Quod cum audisset puer Abraham, procidens adoravit in terram Dominum. 
\verse Prolatisque vasis argenteis et aureis ac vestibus, dedit ea Rebeccae; res pretiosas dedit fratri eius et matri. 
\verse Tunc comederunt et biberunt ipse et viri, qui erant cum eo, et pernoctaverunt ibi. Surgens autem mane locutus est puer: “Dimittite me, ut vadam ad dominum meum”.  
\verse Responderuntque frater eius et mater: “Maneat puella saltem decem dies apud nos et postea proficiscetur”. 
\verse “Nolite, ait, me retinere, quia Dominus direxit viam meam; dimittite me, ut pergam ad dominum meum”. 
\verse Dixerunt: “Vocemus puellam et quaeramus ipsius voluntatem”. 
\verse Cumque vocata venisset, sciscitati sunt: “Vis ire cum homine isto?”. Quae ait: “Vadam”. 
\verse Dimiserunt ergo Rebeccam sororem eorum et nutricem illius servumque Abraham et comites eius, 
\verse imprecantes prospera sorori suae atque dicentes: “Soror nostra es, crescas in mille milia, et possideat semen tuum portas inimicorum suorum!”. 
\verse Igitur surrexit Rebecca et puellae illius et, ascensis camelis, secutae sunt virum; sumpsitque servus Rebeccam et abiit. 
\verse Isaac autem venerat a regione putei Lahairoi et habitabat in terra Nageb.  
\verse Et egressus est Isaac ad lamentandum in agro, inclinata iam die. Cumque levasset oculos, vidit camelos venientes. 
\verse Rebecca quoque levavit oculos et vidit Isaac; descenditque de camelo 
\verse et ait ad puerum: “Quis est ille homo, qui venit per agrum in occursum nobis?”. Dixitque ei: “Ipse est dominus meus”. At illa tollens cito velum operuit se. 
\verse Servus autem cuncta, quae gesserat, narravit Isaac; 
\verse qui introduxit eam in tabernaculum Sarae matris suae et accepit Rebeccam uxorem; et dilexit eam et consolatus est a morte matris suae. 
\end{biblechapter}

\begin{biblechapter}  
\verse Abraham vero aliam duxit uxorem nomine Ceturam, 
\verse quae peperit ei Zamran et Iecsan et Madan et Madian et Iesboc et Sue.  
\verse Iecsan quoque genuit Saba et Dedan. Filii Dedan fuerunt Assurim et Latusim et Loommim. 
\verse At vero ex Madian ortus est Epha et Opher et Henoch et Abida et Eldaa. Omnes hi filii Ceturae. 
\verse Deditque Abraham cuncta, quae possederat, Isaac; 
\verse filiis autem concubinarum suarum largitus est munera et separavit eos ab Isaac filio suo, dum adhuc ipse viveret, ad plagam orientalem. 
\verse Fuerunt autem dies vitae Abrahae centum septuaginta quinque anni. 
\verse Et deficiens mortuus est Abraham in senectute bona provectaeque aetatis et plenus dierum congregatusque est ad populum suum. 
\verse Et sepelierunt eum Isaac et Ismael filii sui in spelunca Machpela, quae sita est in agro Ephron filii Seor Hetthaei e regione Mambre, 
\verse quem emerat a filiis Heth. Ibi sepultus est ipse et Sara uxor eius. 
\verse Et post obitum illius benedixit Deus Isaac filio eius, qui habitabat iuxta puteum Lahairoi. 
\verse Hae sunt generationes Ismael filii Abrahae, quem peperit ei Agar Aegyptia famula Sarae. 
\verse Et haec nomina filiorum Ismael in vocabulis et generationibus suis: primogenitus Ismaelis Nabaioth, dein Cedar et Adbeel et Mabsam, 
\verse Masma quoque et Duma et Massa, 
\verse Hadad et Thema, Iethur et Naphis et Cedma. 
\verse Isti sunt filii Ismaelis, et haec nomina eorum per vicos et mansiones eorum: duodecim principes tribuum suarum. 
\verse Et facti sunt anni vitae Ismaelis centum triginta septem; deficiens mortuus est et appositus ad populum suum. 
\verse Habitaverunt autem ab Hevila usque Sur, quae respicit Aegyptum introeuntibus Assyriam. In faciem cunctorum fratrum suorum obiit. 
\verse Hae sunt generationes Isaac filii Abraham: Abraham genuit Isaac; 
\verse qui, cum quadraginta esset annorum, duxit uxorem Rebeccam filiam Bathuelis Aramaei de Paddanaram, sororem Laban Aramaei. 
\verse Deprecatusque est Isaac Dominum pro uxore sua, eo quod esset sterilis. Qui exaudivit eum et dedit conceptum Rebeccae. 
\verse Sed collidebantur in utero eius parvuli. Quae ait: “Si sic est, cur mihi?”. Perrexitque, ut consuleret Dominum. 
\verse Qui respondens ait: “Duae gentes sunt in utero tuo, et duo populi ex ventre tuo dividentur; populusque populum superabit, et maior serviet minori”. 
\verse Iam tempus pariendi venerat, et ecce gemini in utero eius. 
\verse Qui primus egressus est rufus erat et totus quasi pallium pilosum; vocatumque est nomen eius Esau. Postea frater eius egrediens plantam Esau tenebat manu, et idcirco appellatum est nomen eius Iacob. 
\verse Sexagenarius erat Isaac, quando nati sunt parvuli. 
\verse Quibus adultis, factus est Esau vir gnarus venandi et homo agrestis; Iacob autem vir compositus et habitans in tabernaculis. 
\verse Isaac amabat Esau, eo quod de venationibus illius libenter vesceretur; et Rebecca diligebat Iacob.  
\verse Coxit autem Iacob pulmentum; ad quem, cum venisset Esau de agro lassus,  
\verse ait: “Da mihi de coctione hac rufa, quia oppido lassus sum”. Quam ob causam vocatum est nomen eius Edom (id est Rufus). 
\verse Cui dixit Iacob: “Vende mihi prius primogenita tua”. 
\verse Ille respondit: “En morior; quid mihi proderunt primogenita?”. 
\verse Ait Iacob: “Iura ergo mihi”. Iuravit et vendidit primogenita. 
\verse Et sic, accepto pane et lentis edulio, comedit et bibit; surrexit et abiit parvipendens quod primogenita vendidisset. 
\end{biblechapter}

\begin{biblechapter}  
\verse Orta autem fame super terram post eam sterilitatem, quae acciderat in diebus Abraham, abiit Isaac ad Abimelech regem Philisthim in Gerara. 
\verse Apparuitque ei Dominus et ait: “Ne descendas in Aegyptum, sed habita in terra, quam dixero tibi, 
\verse et peregrinare in ea; eroque tecum et benedicam tibi. Tibi enim et semini tuo dabo universas regiones has complens iuramentum, quod spopondi Abraham patri tuo, 
\verse et multiplicabo semen tuum sicut stellas caeli daboque posteris tuis universas regiones has; et benedicentur in semine tuo omnes gentes terrae, 
\verse eo quod oboedierit Abraham voci meae et custodierit praecepta et mandata mea et iustificationes legesque servaverit”. 
\verse Mansit itaque Isaac in Geraris. 
\verse Qui, cum interrogaretur a viris loci illius super uxore sua, respondit: “Soror mea est”. Timuerat enim confiteri quod sibi esset sociata coniugio, reputans ne forte interficerent eum propter illius pulchritudinem. 
\verse Cumque pertransissent dies plurimi et ibidem moraretur, prospiciens Abimelech rex Philisthim per fenestram vidit eum iocantem cum Rebecca uxore sua. 
\verse Et, accersito eo, ait: “Perspicuum est quod uxor tua sit; cur mentitus es eam sororem tuam esse?”. Respondit: “Timui, ne morerer propter eam”. 
\verse Dixitque Abimelech: “Quare hoc fecisti nobis? Potuit coire quispiam de populo cum uxore tua, et induxeras super nos grande peccatum”. Praecepitque omni populo dicens:  
\verse “Qui tetigerit hominem hunc et uxorem eius, morte morietur”. 
\verse Sevit autem Isaac in terra illa et invenit in ipso anno centuplum; benedixitque ei Dominus. 
\verse Et locupletatus est homo et ibat proficiens atque succrescens, donec magnus vehementer effectus est; 
\verse habuitque possessionem ovium et armentorum et familiae plurimum. Ob haec invidentes ei Philisthim 
\verse omnes puteos, quos foderant servi patris illius in diebus Abraham, obstruxerunt implentes humo, 
\verse in tantum ut ipse Abimelech diceret ad Isaac: “Recede a nobis, quoniam potentior nostri factus es valde". 
\verse Et ille discedens tentoria fixit ad torrentem Gerarae habitavitque ibi.  
\verse Rursum fodit puteos, quos foderant in diebus patris sui Abraham et quos, illo mortuo, obstruxerant Philisthim. Appellavitque eos eisdem nominibus, quibus ante pater vocaverat. 
\verse Foderunt servi Isaac in torrente et reppererunt ibi puteum aquae vivae. 
\verse Sed et ibi iurgium fuit pastorum Gerarae adversus pastores Isaac dicentium: “Nostra est aqua!". Quam ob rem nomen putei vocavit Esec (id est Iurgium), quia iurgati sunt cum eo. 
\verse Foderunt autem et alium puteum, et pro illo quoque rixati sunt; appellavitque eum Sitna (id est Inimicitias). 
\verse Profectus inde fodit alium puteum, pro quo non contenderunt; itaque vocavit nomen eius Rehoboth (id est Latitudinem) dicens: “Nunc dilatavit nos Dominus, et crescemus in terra". 
\verse Ascendit autem ex illo loco in Bersabee, 
\verse ubi apparuit ei Dominus in ipsa nocte dicens: “Ego sum Deus Abraham patris tui. Noli timere, quia tecum sum; benedicam tibi et multiplicabo semen tuum propter servum meum Abraham". 
\verse Itaque aedificavit ibi altare et, invocato nomine Domini, extendit tabernaculum, et servi Isaac foderunt ibi puteum. 
\verse Abimelech autem venit ad eum de Geraris et Ochozath amicus illius et Phicol dux militum, 
\verse et locutus est eis Isaac: “Quid venistis ad me hominem, quem odistis et expulistis a vobis?". 
\verse Qui responderunt: “Vidimus tecum esse Dominum et idcirco diximus: Sit iuramentum inter nos et te, et ineamus tecum foedus, 
\verse ut non facias nobis quidquam mali, sicut et nos non attigimus te et nihil fecimus tibi nisi bonum et cum pace dimisimus te. Tu es enim benedictus Domini". 
\verse Fecit ergo eis convivium, et comederunt et biberunt. 
\verse Surgentesque mane iuraverunt sibi mutuo. Dimisitque eos Isaac, et profecti sunt ab eo cum pace. 
\verse Ecce autem venerunt in ipso die servi Isaac annuntiantes ei de puteo, quem foderant, atque dicentes: “Invenimus aquam". 
\verse Unde appellavit eum Sabee (quod significat Abundantiam); et nomen urbi impositum est Bersabee usque in praesentem diem. 
\verse Esau vero quadragenarius duxit uxores Iudith filiam Beeri Hetthaei et Basemath filiam Elon Hetthaei. 
\verse Quae ambae offenderant animum Isaac et Rebeccae. 
\end{biblechapter}

\begin{biblechapter}  
\verse Senuit autem Isaac, et caligaverunt oculi eius, et videre non poterat. Vocavitque Esau filium suum maiorem et dixit ei: “Fili mi". Qui respondit: “Adsum". 
\verse Cui pater: “Vides, inquit, quod senuerim et ignorem diem mortis meae; 
\verse sume arma tua, pharetram et arcum, et egredere in agrum. Cumque venatu aliquid apprehenderis, 
\verse fac mihi inde pulmentum, sicut velle me nosti, et affer, ut comedam; et benedicat tibi anima mea, antequam moriar". 
\verse Rebecca autem audierat Isaac loquentem cum Esau filio suo. Esau ergo abiit in agrum, ut venationem caperet et offerret eam. 
\verse Rebecca autem dixit filio suo Iacob: “Ecce, audivi patrem tuum loquentem cum Esau fratre tuo et dicentem ei: 
\verse "Affer mihi venationem tuam et fac cibos, ut comedam et benedicam tibi coram Domino, antequam moriar". 
\verse Nunc ergo, fili mi, audi vocem meam in eo, quod praecipio tibi. 
\verse Pergens ad gregem affer mihi duos haedos optimos, ut faciam ex eis escas patri tuo, quibus libenter vescitur. 
\verse Quas cum intuleris patri tuo, et comederit, benedicat tibi, priusquam moriatur". 
\verse Cui ille respondit: “Nosti quod Esau frater meus homo pilosus sit, et ego lenis. 
\verse Si attrectaverit me pater meus et senserit, timeo, ne putet me sibi voluisse illudere; et inducam super me maledictionem pro benedictione”.  
\verse Ad quem mater: “In me sit, ait, ista maledictio, fili mi; tantum audi vocem meam et perge afferque, quae dixi". 
\verse Abiit et attulit deditque matri. Paravit illa cibos, sicut noverat velle patrem illius. 
\verse Et vestibus Esau valde bonis, quas apud se habebat domi, induit eum 
\verse pelliculasque haedorum circumdedit manibus et colli nuda protexit; 
\verse dedit pulmentum optimum et panes, quos coxerat, in manus filii sui Iacob. 
\verse Qui ingressus ad patrem suum dixit: “Pater mi". At ille respondit: “Audio. Quis es tu, fili mi?". 
\verse Dixitque Iacob ad patrem suum: “Ego sum Esau primogenitus tuus. Feci sicut praecepisti mihi; surge, sede et comede de venatione mea, ut benedicat mihi anima tua". 
\verse Rursum Isaac ad filium suum: “Quomodo, inquit, tam cito invenire potuisti, fili mi?". Qui respondit: “Voluntas Domini Dei tui fuit, ut occurreret mihi". 
\verse Dixitque Isaac ad Iacob: “Accede huc, ut tangam te, fili mi, et probem, utrum tu sis filius meus Esau an non". 
\verse Accessit ille ad patrem, et, palpato eo, dixit Isaac: “Vox quidem, vox Iacob est, sed manus, manus sunt Esau". 
\verse Et non cognovit eum, quia pilosae manus similitudinem maioris expresserant. Benedixit ergo illi. 
\verse Ait: “Tu es filius meus Esau?". Respondit: “Ego sum". 
\verse At ille: “Affer, inquit, mihi, et comedam de venatione tua, fili mi, ut benedicat tibi anima mea". Quos cum oblatos comedisset, obtulit ei etiam vinum. Quo hausto, 
\verse dixit ad eum Isaac pater eius: “Accede ad me et da mihi osculum, fili mi". 
\verse Accessit et osculatus est eum. Statimque, ut sensit vestimentorum illius fragrantiam, benedicens illi ait: “Ecce odor filii mei sicut odor agri pleni, cui benedixit Dominus. 
\verse Det tibi Deus de rore caeli et de pinguedine terrae et abundantiam frumenti et vini. 
\verse Et serviant tibi populi, et adorent te nationes; esto dominus fratrum tuorum, et incurventur ante te filii matris tuae. Qui maledixerit tibi, sit maledictus; et, qui benedixerit tibi, sit benedictus!". 
\verse Vix Isaac benedictionem Iacob finierat, et Iacob egressus erat a patre suo Isaac, venit Esau frater eius 
\verse coctosque de venatione cibos intulit patri dicens: “Surge, pater mi, et comede de venatione filii tui, ut benedicat mihi anima tua". 
\verse Dixitque illi Isaac pater eius: “Quis enim es tu?". Qui respondit: “Ego sum filius tuus primogenitus Esau". 
\verse Expavit Isaac stupore vehementi ultra modum et ait: “Quis igitur ille est, qui dudum captam venationem attulit mihi, et comedi ex omnibus, priusquam tu venires? Benedixique ei, et erit benedictus!". 
\verse Auditis Esau sermonibus patris, irrugiit clamore magno et amaro ultra modum et ait patri suo: “Benedic etiam mihi, pater mi!". 
\verse Qui ait: “Venit germanus tuus fraudulenter et accepit benedictionem tuam". 
\verse At ille subiunxit: “Iuste vocatum est nomen eius Iacob; supplantavit enim me en altera vice: primogenita mea ante tulit et nunc secundo surripuit benedictionem meam". Rursumque ait: “Numquid non reservasti mihi benedictionem?". 
\verse Respondit Isaac: “Ecce, dominum tuum illum constitui et omnes fratres eius servituti illius subiugavi; frumento et vino stabilivi eum. Et tibi post haec, fili mi, ultra quid faciam?". 
\verse Dixitque Esau ad patrem suum: “Num unam tantum benedictionem habes, pater mi? Mihi quoque obsecro, ut benedicas!". Cumque eiulatu magno fleret, 
\verse motus Isaac dixit ad eum: “Ecce, procul a pinguedine terrae erit habitatio tua et procul a rore caeli desuper. 
\verse De gladio tuo vives et fratri tuo servies. Tempusque veniet, cum excutias et solvas iugum eius de cervicibus tuis". 
\verse Oderat ergo Esau Iacob pro benedictione, qua benedixerat ei pater, dixitque in corde suo: “Appropinquabunt dies luctus patris mei, et occidam Iacob fratrem meum". 
\verse Nuntiata sunt Rebeccae verba Esau filii eius maioris, quae mittens et vocans Iacob filium suum minorem dixit ad eum: “Ecce, Esau frater tuus minatur, ut occidat te. 
\verse Nunc ergo, fili mi, audi vocem meam et consurgens fuge ad Laban fratrem meum in Charran; 
\verse habitabisque cum eo dies paucos, donec requiescat furor fratris tui, 
\verse et cesset indignatio eius, obliviscaturque eorum, quae fecisti in eum. Postea mittam et adducam te inde huc. Cur utroque orbabor filio in uno die?". 
\verse Dixit quoque Rebecca ad Isaac: “Taedet me vitae meae propter filias Heth; si acceperit Iacob uxorem de filiabus Heth sicut istis de filiabus terrae, nolo vivere". 
\end{biblechapter}

\begin{biblechapter}  
\verse Vocavit itaque Isaac Iacob et benedixit eum praecepit que ei dicens: “Noli accipere coniugem de filiabus Chanaan; 
\verse surge, vade in Paddanaram ad domum Bathuel patris matris tuae et accipe tibi inde uxorem de filiabus Laban avunculi tui. 
\verse Deus autem omnipotens benedicat tibi et crescere te faciat atque multiplicet, ut sis in multitudinem populorum; 
\verse et det tibi benedictiones Abraham tibi et semini tuo tecum, ut possideas terram peregrinationis tuae, quam pollicitus est Deus avo tuo". 
\verse Cumque dimisisset eum Isaac, profectus est in Paddanaram ad Laban filium Bathuel Aramaei fratrem Rebeccae matris Iacob et Esau. 
\verse Videns autem Esau quod benedixisset pater suus Iacob et misisset eum in Paddanaram, ut inde uxorem duceret, et quod post benedictionem praecepisset ei dicens: “Non accipies uxorem de filiabus Chanaan", 
\verse quodque oboediens Iacob parentibus suis isset in Paddanaram; 
\verse probans quoque quod non libenter aspiceret filias Chanaan pater suus, 
\verse ivit ad Ismaelem et duxit uxorem, absque iis, quas habebat, Mahalath filiam Ismael filii Abraham sororem Nabaioth. 
\verse Igitur egressus Iacob de Bersabee pergebat Charran. 
\verse Cumque venisset ad quendam locum et vellet in eo requiescere post solis occubitum, tulit de lapidibus, qui iacebant, et supponens capiti suo dormivit in eodem loco. 
\verse Viditque in somnio scalam stantem super terram et cacumen illius tangens caelum, angelos quoque Dei ascendentes et descendentes per eam 
\verse et Dominum innixum scalae dicentem sibi: “Ego sum Dominus, Deus Abraham patris tui et Deus Isaac. Terram, in qua dormis, tibi dabo et semini tuo. 
\verse Eritque semen tuum quasi pulvis terrae; dilataberis ad occidentem et orientem et septentrionem et meridiem; et benedicentur in te et in semine tuo cunctae tribus terrae. 
\verse Et ecce, ego tecum sum et custodiam te, quocumque perrexeris, et reducam te in terram hanc; nec dimittam te, nisi complevero quae dixi tibi". 
\verse Cumque evigilasset Iacob de somno, ait: “Vere Dominus est in loco isto, et ego nesciebam". 
\verse Pavensque: “Quam terribilis est, inquit, locus iste! Non est hic aliud nisi domus Dei et porta caeli". 
\verse Surgens ergo Iacob mane tulit lapidem, quem supposuerat capiti suo, et erexit in titulum fundens oleum desuper. 
\verse Appellavitque nomen loci illius Bethel; prius autem urbs vocabatur Luza. 
\verse Vovit Iacob etiam votum dicens: “Si fuerit Deus mecum et custodierit me in via hac, per quam ambulo, et dederit mihi panem ad vescendum et vestimentum ad induendum, 
\verse reversusque fuero prospere ad domum patris mei, erit mihi Dominus in Deum, 
\verse et lapis iste, quem erexi in titulum, erit domus Dei; cunctorumque, quae dederis mihi, decimas offeram tibi". 
\end{biblechapter}

\begin{biblechapter}  
\verse Profectus ergo Iacob venit in terram orientalium. 
\verse Et vidit puteum in agro, tres quoque greges ovium accubantes iuxta eum; nam ex illo adaquabantur pecora, et os eius grandi lapide claudebatur. 
\verse Morisque erat, ut, cunctis ovibus congregatis, devolverent lapidem et, refectis gregibus, rursum super os putei ponerent. 
\verse Dixitque ad pastores: “Fratres, unde estis?". Qui responderunt: “De Charran". 
\verse Quos interrogans: “Numquid, ait, nostis Laban filium Nachor?". Dixerunt: “Novimus". 
\verse “Sanusne est?", inquit. “Valet, inquiunt, et ecce Rachel filia eius venit cum grege". 
\verse Dixitque: “Adhuc multum diei superest, nec est tempus, ut congregentur greges; date potum ovibus et sic ad pastum eas reducite". 
\verse Qui responderunt: “Non possumus, donec omnia pecora congregentur et amoveamus lapidem de ore putei, ut adaquemus greges". 
\verse Adhuc loquebatur cum eis, et ecce Rachel veniebat cum ovibus patris sui; nam gregem ipsa pascebat. 
\verse Cum vidisset Iacob Rachel filiam Laban avunculi sui ovesque Laban avunculi sui, accedens amovit lapidem de ore putei 
\verse et adaquavit gregem Laban avunculi sui. Tunc Iacob osculatus est Rachel et elevata voce flevit; 
\verse et indicavit ei quod frater esset patris eius et filius Rebeccae. At illa festinans nuntiavit patri suo. 
\verse Qui cum audisset venisse Iacob filium sororis suae, cucurrit obviam ei; complexusque eum et in oscula ruens duxit in domum suam. Auditis autem omnibus, quae evenerant, 
\verse respondit: “Vere os meum es et caro mea!". Et, postquam Iacob habitavit apud eum per dies mensis unius, 
\verse dixit ei Laban: “Num, quia frater meus es, gratis servies mihi? Dic quid mercedis accipias". 
\verse Habebat vero filias duas: nomen maioris Lia, minor vero appellabatur Rachel; 
\verse sed Lia lippis erat oculis, Rachel decora et venusto aspectu. 
\verse Quam diligens Iacob ait: “Serviam tibi pro Rachel filia tua minore septem annis". 
\verse Respondit Laban: “Melius est, ut tibi eam dem quam alteri viro; mane apud me". 
\verse Servivit igitur Iacob pro Rachel septem annis, et videbantur illi pauci dies prae amoris magnitudine. 
\verse Dixitque ad Laban: “Da mihi uxorem meam, quia iam tempus expletum est, ut ingrediar ad eam". 
\verse Qui, vocatis omnibus viris loci ad convivium, fecit nuptias. 
\verse Et vespere sumpsit Liam filiam suam et introduxit ad eum, et venit ad eam. 
\verse Et dedit Laban ancillam filiae Zelpham nomine. Facto mane, vidit, et ecce erat Lia. 
\verse Et dixit ad socerum suum: “Quid hoc fecisti mihi? Nonne pro Rachel servivi tibi? Quare imposuisti mihi?". 
\verse Respondit Laban: “Non est in loco nostro consuetudinis, ut minorem ante maiorem tradamus ad nuptias. 
\verse Imple hebdomadam hanc, et alteram quoque dabo tibi pro opere, quo serviturus es mihi septem annis aliis". 
\verse Acquievit placito et, hebdomada transacta, dedit ei Laban filiam suam Rachel uxorem, 
\verse cui servam Bilham tradidit. 
\verse Et ingressus etiam ad Rachel amavit eam plus quam Liam serviens apud eum septem annis aliis. 
\verse Videns autem Dominus quod despiceret Liam, aperuit vulvam eius, Rachel sterili permanente. 
\verse Et concepit Lia et genuit filium vocavitque nomen eius Ruben dicens: “Vidit Dominus humilitatem meam; nunc amabit me vir meus". 
\verse Rursumque concepit et peperit filium et ait: “Quoniam audivit me Dominus haberi contemptui, dedit etiam istum mihi"; vocavitque nomen illius Simeon. 
\verse Concepit tertio et genuit alium filium dixitque: “Nunc quoque copulabitur mihi maritus meus, eo quod pepererim ei tres filios"; et idcirco appellavit nomen eius Levi. 
\verse Quarto concepit et peperit filium et ait: “Modo confitebor Domino"; et ob hoc vocavit eum Iudam. Cessavitque parere. 
\end{biblechapter}

\begin{biblechapter}  
\verse Cernens autem Rachel quod infecunda esset, invidit sorori et ait marito suo: “Da mihi liberos, alioquin moriar". 
\verse Cui iratus respondit Iacob: “Num pro Deo ego sum, qui privavit te fructu ventris?". 
\verse At illa: “Ecce, inquit, famula mea Bilha; ingredere ad illam, ut pariat super genua mea, et habeam ex illa et ego filios". 
\verse Deditque illi Bilham famulam suam in coniugium. Quae, 
\verse ingresso ad se Iacob, concepit et peperit filium. 
\verse Dixitque Rachel: “Iudicavit mihi Deus et exaudivit vocem quoque meam dans mihi filium"; et idcirco appellavit nomen illius Dan. 
\verse Rursumque Bilha famula Rachel concepit et peperit Iacob alterum filium, et  
\verse ait Rachel: “Certamina Dei certavi cum sorore mea et invalui"; vocavitque eum Nephthali. 
\verse Sentiens Lia quod parere desisset, sumpsit Zelpham ancillam suam et tradidit eam Iacob in uxorem. 
\verse Quae peperit Iacob filium. 
\verse Dixitque Lia: “Feliciter!"; et idcirco vocavit nomen eius Gad. 
\verse Peperit quoque Zelpha ancilla Liae Iacob alterum filium. 
\verse Dixitque Lia: “Pro beatitudine mea! Beatam quippe me dicent mulieres"; propterea appellavit eum Aser. 
\verse Egressus autem Ruben tempore messis triticeae, repperit in agro mandragoras, quas Liae matri suae detulit. Dixitque Rachel: “Da mihi partem de mandragoris filii tui". 
\verse Illa respondit: “Parumne tibi videtur, quod praeripueris maritum mihi, ut etiam mandragoras filii mei auferas?". Ait Rachel: “Dormiat ergo tecum hac nocte pro mandragoris filii tui". 
\verse Redeuntique ad vesperam Iacob de agro egressa est in occursum eius Lia et: “Ad me, inquit, intrabis, quia mercede conduxi te pro mandragoris filii mei". Dormivitque cum ea nocte illa. 
\verse Et exaudivit Deus Liam, concepitque et peperit Iacob filium quintum  
\verse et ait: “Dedit Deus mercedem mihi, quia dedi ancillam meam viro meo"; appellavitque nomen illius Issachar. 
\verse Rursum Lia concepit et peperit Iacob sextum filium 
\verse et ait: “Donavit me Deus dono bono; hac vice honorabit me maritus meus, eo quod genuerim ei sex filios"; et idcirco appellavit nomen eius Zabulon. 
\verse Post quem peperit filiam nomine Dinam. 
\verse Recordatus quoque Deus Rachelis exaudivit eam Deus et aperuit vulvam illius. 
\verse Quae concepit et peperit filium dicens: “Abstulit Deus opprobrium meum"; 
\verse et vocavit nomen illius Ioseph dicens: “Addat mihi Dominus filium alterum!". 
\verse Nato autem Ioseph, dixit Iacob ad Laban: “Dimitte me, ut revertar in patriam et ad terram meam. 
\verse Da mihi uxores et liberos meos, pro quibus servivi tibi, ut abeam; tu nosti servitutem, qua servivi tibi". 
\verse Ait illi Laban: “Inveniam gratiam in conspectu tuo; augurio didici, quia benedixerit mihi Deus propter te. 
\verse Constitue mercedem tuam, quam dem tibi". 
\verse At ille respondit: “Tu nosti quomodo servierim tibi et quanti in manibus meis facti sint greges tui. 
\verse Modicum habuisti, antequam venirem ad te, et nunc multiplicatum est vehementer, benedixitque tibi Dominus ad introitum meum. Nunc autem quando providebo etiam domui meae?". 
\verse Dixitque Laban: “Quid tibi dabo?". At ille ait: “Nihil mihi dabis; si feceris, quod postulo, iterum pascam et custodiam pecora tua. 
\verse Gyrabo omnes greges tuos hodie; separa cuncta pecora varia et maculosa et, quodcumque furvum in ovibus et maculosum variumque in capris fuerit, erit merces mea. 
\verse Respondebitque mihi cras iustitia mea; quando veneris, ut inspicias mercedem meam, omnia, quae non fuerint varia et maculosa in capris et furva in ovibus, furti me arguent". 
\verse Dixit Laban: “Gratum habeo, quod petis!". 
\verse Et separavit in die illo hircos striatos atque maculosos et omnes capras varias et maculosas, omne, in quo album erat, et omne furvum in ovibus, et tradidit in manu filiorum suorum. 
\verse Et posuit spatium itineris trium dierum inter se et Iacob, qui pascebat reliquos greges Laban. 
\verse Tollens ergo Iacob virgas virides populeas et amygdalinas et ex platanis, ex parte ita decorticavit eas, ut in his, quae spoliata fuerant, candor appareret. 
\verse Posuitque virgas, quas ex parte decorticaverat, in canalibus, ubi effundebatur aqua, ut, cum venissent greges ad bibendum, ante oculos haberent virgas et in aspectu earum conciperent. 
\verse Factumque est ut in ipso calore coitus greges intuerentur virgas et parerent striata et varia et maculosa. 
\verse Agnos autem segregavit Iacob et posuit gregem ex adverso striatorum et omnium furvorum in grege Laban et constituit sibi greges seorsum neque statuit eos cum grege Laban. 
\verse Quotiescumque igitur calefiebant pecora robusta, ponebat Iacob virgas in canalibus aquarum ante oculos pecorum, ut in earum contemplatione conciperent. 
\verse Quando vero pecora debilia erant, non ponebat eas. Factaque sunt debilia Laban et robusta Iacob; 
\verse ditatusque est homo ultra modum et habuit greges multos, ancillas et servos, camelos et asinos. 
\end{biblechapter}

\begin{biblechapter}  
\verse Postquam autem audivit verba filiorum Laban dicentium: “Tulit Iacob omnia, quae fuerunt patris nostri, et de patris nostri facultate acquisivit has divitias", 
\verse animadvertit quoque faciem Laban quod non esset erga se sicut heri et nudiustertius. 
\verse Et dixit Dominus ad Iacob: “Revertere in terram patrum tuorum et ad cognationem tuam, eroque tecum". 
\verse Misit Iacob et vocavit Rachel et Liam in agrum, ubi pascebat greges, 
\verse dixitque eis: “Video faciem patris vestri quod non sit erga me sicut heri et nudiustertius; Deus autem patris mei fuit mecum, 
\verse et ipsae nostis quod totis viribus meis servierim patri vestro. 
\verse Sed pater vester circumvenit me et mutavit mercedem meam decem vicibus; et tamen non dimisit eum Deus, ut noceret mihi. 
\verse Si quando dixit: "Variae erunt mercedes tuae", pariebant omnes oves varios fetus. Quando vero e contrario ait: "Striata quaeque accipies pro mercede", omnes greges striata pepererunt. 
\verse Tulitque Deus substantiam patris vestri et dedit mihi. 
\verse Postquam enim conceptus gregis tempus advenerat, levavi oculos meos et vidi in somnis ascendentes mares super feminas, striatos et varios et respersos. 
\verse Dixitque angelus Dei ad me in somnis: "Iacob". Et ego respondi: Adsum. 
\verse Qui ait: "Leva oculos tuos et vide universos masculos ascendentes super feminas, striatos et varios atque respersos. Vidi enim omnia, quae fecit tibi Laban.  
\verse Ego sum Deus Bethel, ubi unxisti lapidem et votum vovisti mihi. Nunc ergo surge et egredere de terra hac revertens in terram nativitatis tuae"". 
\verse Responderunt ei Rachel et Lia: “Numquid habemus adhuc partem et hereditatem in domo patris nostri? 
\verse Nonne quasi alienas reputavit nos et vendidit nos comeditque pretium nostrum? 
\verse Sed omnes opes, quas tulit Deus patri nostro, nobis abstulit ac filiis nostris; unde omnia, quae praecepit tibi Deus, fac". 
\verse Surrexit itaque Iacob et imposuit liberos suos ac coniuges suas super camelos. 
\verse Tulitque omnes greges suos et omnem substantiam suam, quidquid in Paddanaram acquisierat, ut iret ad Isaac patrem suum in terram Chanaan. 
\verse Eo tempore Laban ierat ad tondendas oves, et Rachel furata est theraphim patris sui. 
\verse Iacob autem decepit cor Laban, non indicans ei quod fugeret. 
\verse Cumque fugisset cum omnibus, quae possidebat, et, amne transmisso, pergeret contra montem Galaad, 
\verse nuntiatum est Laban die tertio quod fugisset Iacob.  
\verse Qui, assumptis fratribus suis, persecutus est eum diebus septem et comprehendit eum in monte Galaad. 
\verse Venit autem Deus ad Laban Aramaeum per somnium noctis dixitque ei: "Cave, ne quidquam loquaris contra Iacob!". 
\verse Iamque Iacob extenderat in monte tabernaculum, cum Laban, consecutus eum cum fratribus suis, in eodem monte Galaad fixit tentorium. 
\verse Et dixit ad Iacob: “Quare ita egisti et decepisti cor meum, abigens filias meas quasi captivas gladio? 
\verse Cur clam fugisti et decepisti me, non indicans mihi, ut prosequerer te cum gaudio et canticis et tympanis et citharis? 
\verse Non es passus, ut oscularer filios meos ac filias; stulte operatus es. Et nunc 
\verse valet quidem manus mea reddere tibi malum, sed Deus patris vestri heri dixit mihi: “Cave, ne loquaris contra Iacob quidquam!”. 
\verse Esto, profectus es, quia desiderio tibi erat domus patris tui; cur furatus es deos meos?". 
\verse Respondit Iacob: “Quia timui. Dixi enim, ne forte violenter auferres filias tuas a me. 
\verse Apud quemcumque inveneris deos tuos, non vivat! Coram fratribus nostris scrutare, quidquid tuorum apud me inveneris, et aufer". Ignorabat enim Iacob quod Rachel furata esset theraphim. 
\verse Ingressus itaque Laban tabernacula Iacob et Liae et utriusque famulae, non invenit. Egressus de tentorio Liae, intravit tentorium Rachelis. 
\verse Illa autem absconderat theraphim in stramento cameli et sedit desuper. Scrutantique omne tentorium et nihil invenienti 
\verse ait: “Ne irascatur dominus meus, quod coram te assurgere nequeo, quia iuxta consuetudinem feminarum nunc accidit mihi". Quaesivit ergo et non invenit theraphim. 
\verse Tumensque Iacob cum iurgio ait: “Quam ob culpam meam et ob quod peccatum meum sic persecutus es me, 
\verse quia scrutatus es omnem supellectilem meam? Quid invenisti de cuncta substantia domus tuae? Pone hic coram fratribus meis et fratribus tuis, et iudicent inter me et te. 
\verse Ecce, viginti annis fui tecum. Oves tuae et caprae non abortiverunt, arietes gregis tui non comedi;  
\verse nec dilaceratum a bestia ostendi tibi: ego damnum omne reddebam; quidquid die noctuque furto perierat, a me exigebas. 
\verse Die aestu consumebar et nocte gelu, fugiebatque somnus ab oculis meis. 
\verse Sic per viginti annos in domo tua servivi tibi: quattuordecim pro filiabus et sex pro gregibus tuis; immutasti quoque mercedem meam decem vicibus. 
\verse Nisi Deus patris mei, Deus Abraham et Timor Isaac, affuisset mihi, certemodo nudum me dimisisses; afflictionem meam et laborem manuum mearum respexit Deus et iudicavit heri". 
\verse Respondit ei Laban: “Filiae filiae meae et filii filii mei et greges greges mei et omnia, quae cernis, mea sunt; et filiabus meis quid possum facere illis hodie et filiis earum, quos genuerunt? 
\verse Veni ergo, et ineamus foedus ego et tu, ut sit in testimonium inter me et te". 
\verse Tulit itaque Iacob lapidem et erexit illum in titulum; 
\verse dixitque fratribus suis: “Afferte lapides". Qui congregantes fecerunt tumulum comederuntque ibi super eum. 
\verse Quem vocavit Laban Iegarsahadutha (id est Tumulus testimonii), et Iacob Galed (uterque iuxta proprietatem linguae suae).  
\verse Dixitque Laban: “Tumulus iste testis erit inter me et te hodie"; et idcirco appellatum est nomen eius Galed (id est Tumulus testis) 
\verse et etiam Maspha (id est Specula), quia dixit: “Speculetur Dominus inter me et te, quando absconditi erimus ab invicem. 
\verse Si afflixeris filias meas et si introduxeris uxores alias super eas, cum nemo nobiscum sit, vide, Deus est testis inter me et te". 
\verse Dixitque Laban ad Iacob: “En tumulus hic et lapis, quem erexi inter me et te. 
\verse Testis erit tumulus iste et lapis quod ego non transibo tumulum hunc pergens ad te, neque tu transibis tumulum hunc et lapidem hunc ad malum. 
\verse Deus Abraham et Deus Nachor iudicent inter nos". Iuravit Iacob per Timorem patris sui Isaac; 
\verse immolatisque victimis in monte, vocavit fratres suos, ut ederent panem. Qui cum comedissent, pernoctaverunt in monte. 
\end{biblechapter}

\begin{biblechapter}  
\verse Laban vero de nocte consurgens osculatus est filios et filias suas et benedixit illis reversusque est in locum suum. 
\verse Iacob quoque abiit itinere, quo coeperat, fueruntque ei obviam angeli Dei.  
\verse Quos cum vidisset, ait: “Castra Dei sunt haec"; et appellavit nomen loci illius Mahanaim (id est Castra). 
\verse Misit autem nuntios ante se ad Esau fratrem suum in terram Seir, in regionem Edom. 
\verse Praecepitque eis dicens: “Sic loquimini domino meo Esau: Haec dicit servus tuus Iacob: Apud Laban peregrinatus sum et fui usque in praesentem diem. 
\verse Habeo boves et asinos, oves et servos atque ancillas; mittoque nunc legationem ad dominum meum, ut inveniam gratiam in conspectu tuo". 
\verse Reversique sunt nuntii ad Iacob dicentes: “Venimus ad Esau fratrem tuum, et ecce properat in occursum tibi cum quadringentis viris". 
\verse Timuit Iacob valde et perterritus divisit populum, qui secum erat, greges quoque et oves et boves et camelos in duas turmas 
\verse dicens: “Si venerit Esau ad unam turmam et percusserit eam, alia turma, quae reliqua est, salvabitur". 
\verse Dixitque Iacob: “Deus patris mei Abraham et Deus patris mei Isaac, Domine, qui dixisti mihi: "Revertere in terram tuam et in locum nativitatis tuae, et benefaciam tibi", 
\verse minor sum cunctis miserationibus et cuncta veritate, quam explesti servo tuo. In baculo meo transivi Iordanem istum et nunc cum duabus turmis regredior. 
\verse Erue me de manu fratris mei, de manu Esau, quia valde eum timeo; ne forte veniens percutiat matrem cum filiis. 
\verse Tu locutus es quod bene mihi faceres et dilatares semen meum sicut arenam maris, quae prae multitudine numerari non potest". 
\verse Mansit ibi nocte illa et sumpsit de his, quae habebat, munera Esau fratri suo: 
\verse capras ducentas, hircos viginti, oves ducentas et arietes viginti,  
\verse camelos fetas cum pullis suis triginta, vaccas quadraginta et tauros decem, asinas viginti et pullos earum decem. 
\verse Et misit per manus servorum suorum singulos seorsum greges dixitque pueris suis: “Antecedite me, et sit spatium inter gregem et gregem". 
\verse Et praecepit priori dicens: “Si obvium habueris Esau fratrem meum, et interrogaverit te: "Cuius es?" et "Quo vadis?" et "Cuius sunt ista, quae sequeris?", 
\verse respondebis: Servi tui Iacob; munera misit domino meo Esau. Ipse quoque post nos venit". 
\verse Similiter mandata dedit secundo ac tertio et cunctis, qui sequebantur greges, dicens: “Iisdem verbis loquimini ad Esau, cum inveneritis eum, 
\verse et addetis: Ipse quoque servus tuus Iacob iter nostrum insequitur. Dixit enim: Placabo illum muneribus, quae praecedunt, et postea videbo faciem eius: forsitan propitiabitur mihi". 
\verse Praecesserunt itaque munera ante eum, ipse vero mansit nocte illa in castris. 
\verse Cumque nocte surrexisset, tulit duas uxores suas et totidem famulas cum undecim filiis et transivit vadum Iaboc; 
\verse sumptis ergo traductisque illis et omnibus, quae ad se pertinebant, per torrentem, 
\verse mansit solus. Et ecce vir luctabatur cum eo usque mane. 
\verse Qui cum videret quod eum superare non posset, tetigit acetabulum femoris eius, et statim luxatum est acetabulum femoris Iacob, cum luctaretur cum illo. 
\verse Dixitque: “Dimitte me, iam enim ascendit aurora". Respondit: “Non dimittam te, nisi benedixeris mihi". 
\verse Ait ad eum: “Quod nomen est tibi?". Respondit: “Iacob". 
\verse At ille: “Nequaquam, inquit, Iacob amplius appellabitur nomen tuum, sed Israel: quoniam certasti cum Deo et cum hominibus et praevaluisti!". 
\verse Interrogavit eum Iacob: “Dic mihi, quo appellaris nomine?". Respondit: “Cur quaeris nomen meum?". Et benedixit ei in eodem loco. 
\verse Vocavitque Iacob nomen loci illius Phanuel dicens: “Vidi Deum facie ad faciem, et salva facta est anima mea". 
\verse Ortusque est ei sol, cum transgrederetur Phanuel; ipse vero claudicabat propter femur. 
\verse Quam ob causam non comedunt filii Israel nervum, qui est in femore, usque in praesentem diem, eo quod tetigerit nervum femoris Iacob. 
\end{biblechapter}

\begin{biblechapter}  
\verse Elevans autem Iacob oculos suos vidit venientem Esau et cum eo quadringentos viros; divisitque filios Liae et Rachel ambarumque famularum.  
\verse Et posuit utramque ancillam et liberos earum in principio, Liam vero et filios eius in secundo loco, Rachel autem et Ioseph novissimos. 
\verse Et ipse praegrediens adoravit pronus in terram septies, donec appropinquaret ad fratrem suum. 
\verse Currens itaque Esau obviam fratri suo amplexatus est eum; stringensque collum eius osculatus est eum, et fleverunt. 
\verse Levatisque oculis, vidit mulieres et liberos earum et ait: “Qui sunt isti tibi?". Respondit: “Liberi sunt, quos donavit mihi Deus servo tuo". 
\verse Et appropinquantes ancillae et filii earum incurvati sunt. 
\verse Accessit quoque Lia cum liberis suis et, cum similiter adorassent, extremi Ioseph et Rachel adoraverunt. 
\verse “Quaenam sunt, inquit, istae turmae, quas obvias habui?". Respondit: “Ut invenirem gratiam coram domino meo". 
\verse At ille: “Habeo, ait, plurima, frater mi; sint tua tibi". 
\verse Dixit Iacob: “Noli ita, obsecro; sed, si inveni gratiam in oculis tuis, accipe munusculum de manibus meis; sic enim vidi faciem tuam quasi viderim vultum Dei, et mihi propitius fuisti. 
\verse Suscipe, quaeso, benedictionem, quae allata est tibi; quia Deus misertus est mihi, et habeo omnia". Et, cum compelleret illum, suscepit 
\verse et ait: “Gradiamur simul, eroque socius itineris tui". 
\verse Dixit Iacob: “Nosti, domine mi, quod parvulos habeam teneros et oves et boves fetas mecum; quas si plus in ambulando fecero laborare vel una die, morientur cuncti greges. 
\verse Praecedat dominus meus ante servum suum; et ego sequar paulatim secundum gressum pecorum ante me et secundum gressum parvulorum, donec veniam ad dominum meum in Seir". 
\verse Respondit Esau: “Oro te, ut de populo, qui mecum est, saltem socii remaneant viae tuae”. “Non est, inquit, necesse; hoc uno indigeo, ut inveniam gratiam in conspectu domini mei". 
\verse Reversus est itaque illo die Esau itinere suo in Seir. 
\verse Et Iacob venit in Succoth, ubi, aedificata sibi domo et fixis tentoriis pro gregibus suis, appellavit nomen loci illius Succoth (id est Tabernacula). 
\verse Transivitque Iacob incolumis ad urbem Sichem, quae est in terra Chanaan, cum veniret de Paddanaram; et habitavit iuxta oppidum. 
\verse Emitque partem agri, in qua fixerat tabernaculum suum, a filiis Hemmor patris Sichem centum argenteis. 
\verse Et erexit ibi altare et vocavit illud: “Deus est Deus Israel". 
\end{biblechapter}

\begin{biblechapter}  
\verse Egressa est autem Dina filia, quam Lia pepererat Iacob, ut videret filias regionis illius. 
\verse Quam cum vidisset Sichem filius Hemmor Hevaei principis terrae illius, adamavit eam et rapuit; et dormivit cum illa, vi opprimens illam.  
\verse Et conglutinata est anima eius cum ea, et amavit puellam et locutus est ad cor eius. 
\verse Dixitque ad Hemmor patrem suum: “Accipe mihi puellam hanc coniugem". 
\verse Cum audisset Iacob quod violasset Dinam filiam suam, absentibus filiis et in pastu pecorum occupatis, siluit, donec redirent. 
\verse Egresso autem Hemmor patre Sichem, ut loqueretur ad Iacob, 
\verse ecce filii Iacob veniebant de agro, auditoque, quod acciderat, contristati et irati sunt valde, eo quod foedam rem esset operatus in Israel et, violata filia Iacob, rem illicitam perpetrasset. 
\verse Locutus est itaque Hemmor ad eos: “Sichem filii mei adhaesit anima filiae vestrae; date eam illi uxorem, 
\verse et iungamus vicissim conubia: filias vestras tradite nobis et filias nostras accipite vobis. 
\verse Et habitate nobiscum; terra in potestate vestra est: manete, perambulate et possidete eam".  
\verse Sed et Sichem ad patrem et ad fratres eius ait: “Inveniam gratiam coram vobis et, quaecumque statueritis, dabo. 
\verse Augete mihi valde dotem et munera; libens tribuam, quod petieritis. Tantum date mihi puellam hanc uxorem". 
\verse Responderunt filii Iacob Sichem et Hemmor patri eius in dolo ob stuprum sororis: 
\verse “Non possumus facere, quod petitis, dare sororem nostram homini incircumciso, opprobrium enim esset nobis. 
\verse In hoc tantum valebimus acquiescere vobis: si esse volueritis similes nostri, circumcidatur in vobis omne masculini sexus; 
\verse tunc dabimus et accipiemus mutuo filias nostras ac vestras et habitabimus vobiscum erimusque unus populus. 
\verse Si autem circumcidi nolueritis, tollemus filiam nostram et recedemus". 
\verse Placuit oblatio eorum Hemmor et Sichem filio eius, 
\verse nec distulit adulescens quin statim, quod petebatur, expleret; amabat enim filiam Iacob valde, et ipse erat inclitus in omni domo patris sui. 
\verse Ingressique portam urbis, Hemmor et Sichem filius eius locuti sunt ad viros civitatis suae: 
\verse “Viri isti pacifici sunt erga nos; maneant in terra et perambulent eam, quae spatiosa et lata est eis; filias eorum accipiemus uxores et nostras illis dabimus. 
\verse Tantum in hoc valebunt viri acquiescere nobis, ut maneant nobiscum et efficiamur unus populus, si circumcidamus masculos nostros ritum gentis imitantes; 
\verse et pecora et substantia et armenta eorum nostra erunt. Tantum in hoc acquiescamus, et habitabunt nobiscum". 
\verse Assensique sunt omnes, circumcisis cunctis maribus, qui egrediebantur e porta civitatis suae. 
\verse Et ecce, die tertio, quando gravissimus vulnerum dolor est, arreptis duo filii Iacob Simeon et Levi fratres Dinae gladiis, ingressi sunt urbem securi; interfectisque omnibus masculis, 
\verse Hemmor et Sichem pariter necaverunt, tollentes Dinam de domo Sichem sororem suam. 
\verse Filii Iacob irruerunt super occisos, et depopulati sunt urbem in ultionem stupri. 
\verse Oves eorum et armenta et asinos cunctaque, quae in civitate et in agris erant, tulerunt.  
\verse Omnes opes eorum, parvulos quoque et uxores duxerunt captivas et diripuerunt omnia, quae in domibus erant. 
\verse Iacob autem dixit ad Simeon et Levi: “Turbastis me et odiosum fecistis me Chananaeis et Pherezaeis habitatoribus terrae huius. Nos pauci sumus; illi congregati percutient me, et delebor ego et domus mea". 
\verse Responderunt: “Numquid ut scorto abuti debuere sorore nostra?". 
\end{biblechapter}

\begin{biblechapter}  
\verse Locutus est Deus ad Iacob: “Surge et ascende Bethel et habita ibi; facque altare Deo, qui apparuit tibi, quando fugiebas Esau fratrem tuum". 
\verse Iacob vero, convocata omni domo sua, ait: “Abigite deos alienos, qui in medio vestri sunt, et mundamini ac mutate vestimenta vestra. 
\verse Surgamus et ascendamus in Bethel, ut faciamus ibi altare Deo, qui exaudivit me in die tribulationis meae et socius fuit itineris mei". 
\verse Dederunt ergo ei omnes deos alienos, quos habebant, et inaures, quae erant in auribus eorum; at ille infodit ea subter Quercum, quae est prope urbem Sichem. 
\verse Cumque profecti essent, terror Dei invasit omnes per circuitum civitates, et non sunt ausi persequi filios Iacob. 
\verse Venit igitur Iacob Luzam, quae est in terra Chanaan, id est Bethel, ipse et omnis populus cum eo. 
\verse Aedificavitque ibi altare et appellavit nomen loci illius Deus Bethel; ibi enim apparuit ei Deus, cum fugeret fratrem suum. 
\verse Eodem tempore mortua est Debora nutrix Rebeccae et sepulta est ad radices Bethel subter quercum; vocatumque est nomen loci illius Quercus fletus. 
\verse Apparuit iterum Deus Iacob, postquam reversus est de Paddanaram, benedixitque ei 
\verse dicens: “Non vocaberis ultra Iacob, sed Israel erit nomen tuum", et appellavit eum Israel. 
\verse Dixitque ei: “Ego Deus omnipotens. Cresce et multiplicare; gens et congregatio nationum erunt ex te, reges de lumbis tuis egredientur. 
\verse Terramque, quam dedi Abraham et Isaac, dabo tibi; et semini tuo post te dabo terram hanc". 
\verse Et ascendit ab eo Deus. 
\verse Ille vero erexit titulum lapideum in loco, quo locutus ei fuerat Deus, libans super eum libamina et effundens oleum 
\verse vocansque nomen loci illius Bethel. 
\verse Egressi sunt de Bethel. Et adhuc spatium quoddam erat usque ad Ephratham, cum parturiret Rachel; 
\verse ob difficultatem partus periclitari coepit, dixitque ei obstetrix: “Noli timere, quia et hac vice habes filium". 
\verse Egrediente autem anima et imminente iam morte, vocavit nomen filii sui Benoni (id est Filius doloris mei); pater vero appellavit eum Beniamin (id est Filius dextrae). 
\verse Mortua est ergo Rachel et sepulta est in via, quae ducit Ephratham; haec est Bethlehem. 
\verse Erexitque Iacob titulum super sepulcrum eius; hic est titulus monumenti Rachel usque in praesentem diem. 
\verse Egressus inde Israel, fixit tabernaculum trans Magdaleder (id est Turris gregis). 
\verse Cumque habitaret in illa regione, abiit Ruben et dormivit cum Bilha concubina patris sui; quod illum minime latuit. Erant autem filii Iacob duodecim. 
\verse Filii Liae: primogenitus Ruben et Simeon et Levi et Iudas et Issachar et Zabulon. 
\verse Filii Rachel: Ioseph et Beniamin. 
\verse Filii Bilhae ancillae Rachelis: Dan et Nephthali. 
\verse Filii Zelphae ancillae Liae: Gad et Aser. Hi sunt filii Iacob, qui nati sunt ei in Paddanaram. 
\verse Venit Iacob ad Isaac patrem suum in Mambre Cariatharbe, id est Hebron, ubi peregrinatus est Abraham et Isaac. 
\verse Et completi sunt dies Isaac centum octoginta annorum; 
\verse consumptusque aetate mortuus est et appositus est populo suo senex et plenus dierum. Et sepelierunt eum Esau et Iacob filii sui. 
\end{biblechapter}

\begin{biblechapter}  
\verse Hae sunt autem generationes Esau. Ipse est Edom. 
\verse Esau accepit uxores de filiabus Chanaan: Ada filiam Elon Hetthaei et Oolibama filiam Ana filii Sebeon Horraei; 
\verse Basemath quoque filiam Ismael sororem Nabaioth. 
\verse Peperit autem Ada Eliphaz, Basemath genuit Rahuel, 
\verse Oolibama genuit Iehus et Ialam et Core. Hi filii Esau, qui nati sunt ei in terra Chanaan. 
\verse Tulit autem Esau uxores suas et filios et filias et omnes animas domus suae et pecora armenta et cuncta, quae acquisierat in terra Chanaan, et abiit in terram Seir; recessitque a fratre suo Iacob. 
\verse Divites enim erant valde et simul habitare non poterant; nec sustinebat eos terra peregrinationis eorum prae multitudine gregum. 
\verse Habitavitque Esau in monte Seir. Ipse est Edom. 
\verse Hae autem sunt generationes Esau patris Edom in monte Seir, 
\verse et haec nomina filiorum eius: Eliphaz filius Ada uxoris Esau, Rahuel quoque filius Basemath uxoris eius. 
\verse Fueruntque Eliphaz filii: Theman, Omar, Sepho et Gatham et Cenez. 
\verse Erat autem Thamna concubina Eliphaz filii Esau, quae peperit ei Amalec. Hi sunt filii Ada uxoris Esau. 
\verse Filii autem Rahuel: Nahath et Zara, Samma et Meza; hi filii Basemath uxoris Esau. 
\verse Isti erant filii Oolibama filiae Ana filii Sebeon uxoris Esau, quos genuit ei: Iehus et Ialam et Core. 
\verse Hi duces filiorum Esau. Filii Eliphaz primogeniti Esau: dux Theman, dux Omar, dux Sepho, dux Cenez, 
\verse dux Core, dux Gatham, dux Amalec. Hi duces Eliphaz in terra Edom; hi filii Ada. 
\verse Hi filii Rahuel filii Esau: dux Nahath, dux Zara, dux Samma, dux Meza. Hi duces Rahuel in terra Edom; isti filii Basemath uxoris Esau. 
\verse Hi filii Oolibama uxoris Esau: dux Iehus, dux Ialam, dux Core. Hi duces Oolibama filiae Ana uxoris Esau. 
\verse Isti sunt filii Esau et hi duces eorum. Ipse est Edom. 
\verse Isti sunt filii Seir Horraei habitatores terrae: Lotan et Sobal et Sebeon et Ana 
\verse et Dison et Eser et Disan; hi duces Horraei filii Seir in terra Edom.  
\verse Facti sunt autem filii Lotan: Hori et Hemam; erat autem soror Lotan Thamna.  
\verse Et isti filii Sobal: Alvan et Manahath et Ebal, Sepho et Onam. 
\verse Et hi filii Sebeon: Aia et Ana. Iste est Ana, qui invenit aquas calidas in solitudine, cum pasceret asinos Sebeon patris sui. 
\verse Habuitque filium Dison et filiam Oolibama. 
\verse Et isti filii Dison: Hemdan et Eseban et Iethran et Charran. 
\verse Hi filii Eser: Bilhan et Zavan et Iacan. 
\verse Habuit autem filios Disan: Us et Aran. 
\verse Isti duces Horraeorum: dux Lotan, dux Sobal, dux Sebeon, dux Ana, 
\verse dux Dison, dux Eser, dux Disan; isti duces Horraeorum secundum tribus eorum in terra Seir. 
\verse Reges autem, qui regnaverunt in terra Edom, antequam haberent regem filii Israel, fuerunt hi. 
\verse Regnavit in Edom Bela filius Beor, nomenque urbis eius Denaba. 
\verse Mortuus est autem Bela, et regnavit pro eo Iobab filius Zarae de Bosra. 
\verse Cumque mortuus esset Iobab, regnavit pro eo Husam de terra Themanorum. 
\verse Hoc quoque mortuo, regnavit pro eo Adad filius Badad, qui percussit Madian in regione Moab; et nomen urbis eius Avith. 
\verse Cumque mortuus esset Adad, regnavit pro eo Semla de Masreca. 
\verse Hoc quoque mortuo, regnavit pro eo Saul de Rohoboth iuxta fluvium. 
\verse Cumque et hic obiisset, successit in regnum Baalhanan filius Achobor. 
\verse Isto quoque mortuo, regnavit pro eo Adad, nomenque urbis eius Phau; et appellabatur uxor eius Meetabel filia Matred filiae Mezaab. 
\verse Haec ergo nomina ducum Esau in cognationibus et locis et vocabulis suis: dux Thamna, dux Alva, dux Ietheth, 
\verse dux Oolibama, dux Ela, dux Phinon, 
\verse dux Cenez, dux Theman, dux Mabsar, 
\verse dux Magdiel, dux Iram. Hi duces Edom habitantes in terra imperii sui. Ipse est Esau pater Idumaeorum. 
\end{biblechapter}

\begin{biblechapter}  
\verse Habitavit autem Iacob in terra Chanaan, in qua peregrinatus est pater suus. 
\verse Hae sunt generationes Iacob. Ioseph, cum decem et scptem esset annorum, pascebat gregem cum fratribus suis adhuc puer; et erat cum filiis Bilhae et Zelphae uxorum patris sui; detulitque patri malam eorum famam. 
\verse Israel autem diligebat Ioseph super omnes filios suos, eo quod in senectute genuisset eum; fecitque ei tunicam talarem. 
\verse Videntes autem fratres eius quod a patre plus cunctis filiis amaretur, oderant eum nec poterant ei quidquam pacifice loqui. 
\verse Accidit quoque ut visum somnium referret fratribus suis; quae causa maioris odii seminarium fuit. 
\verse Dixitque ad eos: “Audite somnium meum, quod vidi.  
\verse Putabam ligare nos manipulos in agro, et quasi consurgere manipulum meum et stare, vestrosque manipulos circumstantes adorare manipulum meum". 
\verse Responderunt fratres eius: “Numquid rex noster eris? Aut subiciemur dicioni tuae?". Haec ergo causa somniorum atque sermonum, invidiae et odii fomitem ministravit. 
\verse Aliud quoque vidit somnium, quod narrans fratribus ait: “Vidi per somnium quasi solem et lunam et stellas undecim adorare me". 
\verse Quod cum patri suo et fratribus retulisset, increpavit eum pater suus et dixit: “Quid sibi vult hoc somnium, quod vidisti? Num ego et mater tua et fratres tui adorabimus te proni in terram?". 
\verse Invidebant igitur ei fratres sui; pater vero rem tacitus considerabat. 
\verse Cumque fratres illius in pascendis gregibus patris morarentur in Sichem,  
\verse dixit Israel ad Ioseph: “Fratres tui pascunt oves in Sichimis; veni, mittam te ad eos". Quo respondente: 
\verse “Praesto sum", ait ei: “Vade et vide, si cuncta prospera sint erga fratres tuos et pecora, et renuntia mihi quid agatur". Missus de valle Hebron venit in Sichem; 
\verse invenitque eum vir errantem in agro et interrogavit quid quaereret. 
\verse At ille respondit: “Fratres meos quaero; indica mihi, ubi pascant greges". 
\verse Dixitque ei vir: “Recesserunt de loco isto; audivi autem eos dicentes: "Eamus in Dothain"". Perrexit ergo Ioseph post fratres suos et invenit eos in Dothain. 
\verse Qui cum vidissent eum procul, antequam accederet ad eos, cogitaverunt illum occidere. 
\verse Et mutuo loquebantur: “Ecce somniator venit; 
\verse venite, occidamus eum et mittamus in unam cisternarum dicemusque: Fera pessima devoravit eum. Et tunc apparebit quid illi prosint somnia sua". 
\verse Audiens autem hoc Ruben nitebatur liberare eum de manibus eorum et dixit:  
\verse “Non interficiamus animam eius". Et dixit ad eos: “Non effundatis sanguinem; sed proicite eum in cisternam hanc, quae est in solitudine, manusque vestras servate innoxias". Hoc autem dicebat volens eripere eum de manibus eorum et reddere patri suo. 
\verse Confestim igitur, ut pervenit ad fratres suos, nudaverunt eum tunica talari 
\verse miseruntque eum in cisternam, quae non habebat aquam. 
\verse Et sederunt, ut comederent panem. Attollentes autem oculos viderunt Ismaelitas viatorcs venire de Galaad et camelos eorum portantes tragacanthum et masticem et ladanum in Aegyptum. 
\verse Dixit ergo Iudas fratribus suis: “Quid nobis prodest, si occiderimus fratrem nostrum et celaverimus sanguinem ipsius? 
\verse Melius est ut vendatur Ismaelitis, et manus nostrae non polluantur; frater enim et caro nostra est". Acquieverunt fratres sermonibus illius. 
\verse Et praetereuntibus Madianitis negotiatoribus, extrahentes Ioseph de cisterna, vendiderunt eum Ismaelitis viginti argenteis. Qui duxerunt eum in Aegyptum. 
\verse Reversusque Ruben ad cisternam non invenit puerum 
\verse et, scissis vestibus, pergens ad fratres suos ait: “Puer non comparet, et ego quo ibo?".  
\verse Tulerunt autem tunicam eius et in sanguinem haedi, quem occiderant, tinxerunt  
\verse mittentes, qui ferrent ad patrem et dicerent: “Hanc invenimus; vide, utrum tunica talaris filii tui sit an non?". 
\verse Quam cum agnovisset pater, ait: “Tunica filii mei est; fera pessima comedit eum, bestia devoravit Ioseph". 
\verse Scissisque vestibus, indutus est cilicio lugens filium suum multo tempore. 
\verse Congregatis autem cunctis liberis eius, ut lenirent dolorem patris, noluit consolationem accipere et ait: “Descendam ad filium meum lugens in infernum". Et flevit super eo pater eius. 
\verse Madianitae autem vendiderunt Ioseph in Aegypto Putiphari eunucho pharaonis, magistro satellitum. 
\end{biblechapter}

\begin{biblechapter}  
\verse Eo tempore descendens Iudas a fratribus suis divertit ad virum Odollamitem nomine Hiram. 
\verse Viditque ibi filiam hominis Chananaei vocabulo Sue et, accepta uxore, ingressus est ad eam. 
\verse Quae concepit et peperit filium vocavitque nomen eius Her. 
\verse Rursumque concepto fetu, natum filium nominavit Onan. 
\verse Tertium quoque peperit, quem appellavit Sela. Ipsa autem erat in Chasib, quando peperit illum. 
\verse Dedit autem Iudas uxorem primogenito suo Her nomine Thamar. 
\verse Fuit quoque Her primogenitus Iudae nequam in conspectu Domini, et ab eo occisus est. 
\verse Dixit ergo Iudas ad Onan: “Ingredere ad uxorem fratris tui et sociare illi, ut suscites semen fratri tuo". 
\verse Ille, sciens non sibi nasci hunc filium, introiens ad uxorem fratris sui semen fundebat in terram, ne proles fratris nomine nasceretur. 
\verse Et idcirco occidit et eum Dominus, quod rem detestabilem fecerat. 
\verse Quam ob rem dixit Iudas Thamar nurui suae: “Esto vidua in domo patris tui, donec crescat Sela filius meus". Timebat, enim, ne et ipse moreretur sicut fratres eius. Quae abiit et habitavit in domo patris sui. 
\verse Evolutis autem multis diebus, mortua est filia Sue uxor Iudae. Qui, post luctum consolatione suscepta, ascendebat ad tonsores ovium suarum ipse et Hiras amicus suus Odollamites in Thamnam. 
\verse Nuntiatumque est Thamar quod socer illius ascenderet in Thamnam ad tondendas oves. 
\verse Quae, depositis viduitatis vestibus, cooperuit se velo et, mutato habitu, sedit in porta Enaim in via, quae ducit Thamnam; eo quod crevisset Sela, et non eum accepisset maritum. 
\verse Quam cum vidisset Iudas, suspicatus est esse meretricem; operuerat enim vultum suum. 
\verse Declinansque ad eam in via ait: “Veni, coeam tecum"; nesciebat enim quod nurus sua esset. Qua respondente: “Quid mihi dabis, ut fruaris concubitu meo?", 
\verse dixit: “Mittam tibi haedum de gregibus". Rursum illa dicente: “Si dederis mihi arrabonem, donec mittas illum", 
\verse ait Iudas: “Quid vis tibi pro arrabone dari?". Respondit: “Sigillum tuum et funiculum et baculum, quem manu tenes". Et dedit ei. In coitu cum eo mulier concepit 
\verse et surgens abiit; depositoque velo, induta est viduitatis vestibus. 
\verse Misit autem Iudas haedum per amicum suum Odollamitem, ut reciperet pignus, quod dederat mulieri. Qui cum non invenisset eam, 
\verse interrogavit homines loci illius: “Ubi est meretrix, quae sedebat in Enaim in via?". Respondentibus cunctis: “Non fuit in loco isto meretrix", 
\verse reversus est ad Iudam et dixit ei: “Non inveni eam; sed et homines loci illius dixerunt mihi numquam ibi sedisse scortum". 
\verse Ait Iudas: “Habeat sibi; ne simus in ludibrium. Ego misi haedum, quem promiseram, et tu non invenisti eam". 
\verse Ecce autem post tres menses nuntiaverunt Iudae dicentes: “Fornicata est Thamar nurus tua et gravida est ex fornicatione". Dixitque Iudas: “Producite eam, ut comburatur". 
\verse Quae cum educeretur ad poenam, misit ad socerum suum dicens: “De viro, cuius haec sunt, concepi; cognosce cuius sit sigillum et funiculus et baculus". 
\verse Qui, agnitis pignoribus, ait: “Iustior me est, quia non tradidi eam Sela filio meo". Attamen ultra non cognovit illam. 
\verse Instante autem partu, apparuerunt gemini in utero; atque in ipsa effusione infantium unus protulit manum, in qua obstetrix ligavit coccinum dicens: 
\verse “Iste egressus est prior". 
\verse Illo vero retrahente manum, egressus est frater eius; dixitque mulier: “Qualem rupisti tibi rupturam?". Et ob hanc causam vocatum est nomen eius Phares (id est Ruptura). 
\verse Postea egressus est frater eius, in cuius manu erat coccinum; qui appellatus est Zara (id est Ortus solis). 
\end{biblechapter}

\begin{biblechapter}  
\verse Igitur Ioseph ductus est in Aegyptum; emitque eum Putiphar eunuchus pharaonis, princeps satellitum, vir Aegyptius, de manu Ismaelitarum, a quibus perductus erat. 
\verse Fuitque Dominus cum eo, et erat vir in cunctis prospere agens habitabatque in domo domini sui. 
\verse Qui optime noverat esse Dominum cum eo et omnia, quae gereret, ab eo dirigi in manu illius. 
\verse Invenitque loseph gratiam coram domino suo et ministrabat ei. Et factum est, postquam praeposuit eum domui suae et omnia, quae possidebat, tradidit in manum eius, 
\verse benedixit Dominus domui Aegyptii propter Ioseph, et benedictio Domini erat in omni possessione eius tam in aedibus quam in agris. 
\verse Et reliquit omnia, quae possidebat, in manu Ioseph nec cum eo quidquam aliud noverat nisi panem, quo vescebatur. Erat autem Ioseph pulchra facie et decorus aspectu. 
\verse Post haec ergo iniecit uxor domini eius oculos suos in Ioseph et ait: “Dormi mecum". 
\verse Qui nequaquam acquiescens dixit ad eam: “Ecce dominus meus, omnibus mihi traditis, non curat de ulla re in domo sua, 
\verse nec quisquam maior est in domo hac quam ego, et nihil mihi subtraxit praeter te, quae uxor eius es. Quomodo ergo possum malum hoc magnum facere et peccare in Deum?".  
\verse Huiuscemodi verbis per singulos dies et mulier molesta erat adulescenti, et ille recusabat stuprum. 
\verse Accidit autem quadam die, ut intraret Ioseph domum et opus suum absque arbitris faceret; 
\verse illa, apprehensa lacinia vestimenti eius, dixit: “Dormi mecum". Qui, relicto in manu illius pallio, fugit et egressus est foras.  
\verse Cumque vidisset illum mulier vestem reliquisse in manibus suis et fugisse foras, 
\verse vocavit homines domus suae et ait ad eos: “En introduxit virum Hebraeum, ut illuderet nobis; ingressus est ad me, ut coiret mecum. Cumque ego succlamassem, 
\verse et audisset vocem meam, reliquit pallium, quod tenebam, et fugit foras". 
\verse Retentum pallium ostendit marito revertenti domum 
\verse et secundum verba haec locuta est: “Ingressus est ad me servus Hebraeus, quem adduxisti, ut illuderet mihi; 
\verse cumque audisset me clamare, reliquit pallium, quod tenebam, et fugit foras". 
\verse Dominus, auditis his verbis coniugis, iratus est valde; 
\verse tradiditque Ioseph in carcerem, ubi vincti regis custodiebantur. Et erat ibi clausus. 
\verse Fuit autem Dominus cum Ioseph et misertus illius dedit ei gratiam in conspectu principis carceris. 
\verse Qui tradidit in manu Ioseph universos vinctos, qui in custodia tenebantur, et, quidquid ibi faciendum erat, ipse faciebat, 
\verse nec princeps carceris spectabat quidquid in manu eius erat: Dominus enim erat cum illo et omnia opera eius dirigebat. 
\end{biblechapter}

\begin{biblechapter}  
\verse His ita gestis, accidit ut peccarent pincerna regis Aegypti et pistor domino suo. 
\verse Iratusque pharao contra duos eunuchos, praepositum pincernarum et praepositum pistorum, 
\verse misit eos in carcerem principis satellitum, in quo erat vinctus et Ioseph. 
\verse Et princeps satellitum tradidit eos Ioseph, qui ministrabat eis. Aliquantulum temporis illi in custodia tenebantur. 
\verse Videruntque ambo somnium nocte una iuxta interpretationem congruam sibi. 
\verse Ad quos cum introisset Ioseph mane et vidisset eos tristes, 
\verse sciscitatus est eos dicens: “Cur tristior est hodie solito facies vestra?". 
\verse Qui responderunt: “Somnium vidimus, et non est qui interpretetur nobis". Dixitque ad eos Ioseph: “Numquid non Dei est interpretatio? Referte mihi quid videritis". 
\verse Narravit praepositus pincernarum somnium suum: “Videbam coram me vitem,  
\verse in qua erant tres propagines, crescere paulatim in gemmas et post flores uvas maturescere; 
\verse calicemque pharaonis in manu mea. Tuli ergo uvas et expressi in calicem, quem tenebam, et tradidi poculum pharaoni". 
\verse Respondit Ioseph: “Haec est interpretatio somnii: tres propagines, tres adhuc dies sunt,  
\verse post quos elevabit pharao caput tuum et restituet te in gradum pristinum; dabisque ei calicem iuxta officium tuum, sicut facere ante consueveras. 
\verse Tantum memento mei, cum tibi bene fuerit, et facias mecum misericordiam, ut suggeras pharaoni, ut educat me de isto carcere; 
\verse quia furto sublatus sum de terra Hebraeorum et hic innocens in lacum missus sum". 
\verse Videns pistorum magister quod somnium in bonum dissolvisset, ait: “Et ego vidi somnium, quod tria canistra farinae haberem super caput meum; 
\verse et in uno canistro, quod erat excelsius, portare me ex omnibus cibis pharaonis, qui fiunt arte pistoria, avesque comedere eos". 
\verse Respondit Ioseph: “Haec est interpretatio somnii: tria canistra, tres adhuc dies sunt, 
\verse post quos auferet pharao caput tuum ac suspendet te in patibulo, et comedent volucres carnes tuas". 
\verse Exinde dies tertius natalicius pharaonis erat; qui faciens grande convivium pueris suis elevavit caput magistri pincernarum et caput pistorum principis in medio puerorum suorum; 
\verse restituitque alterum in locum suum, ut porrigeret ei poculum, 
\verse alterum suspendit in patibulo, sicut interpretatus erat eis Ioseph. 
\verse Attamen praepositus pincernarum non est recordatus Ioseph, sed oblitus est interpretis sui. 
\end{biblechapter}

\begin{biblechapter}  
\verse Post duos annos vidit pharao somnium. Putabat se stare super fluvium,  
\verse de quo ascendebant septem boves pulchrae et crassae et pascebantur in locis palustribus. 
\verse Aliae quoque septem emergebant post illas de flumine foedae confectaeque macie et stabant in ipsa amnis ripa; 
\verse devoraveruntque septem boves pulchras et crassas. Expergefactus pharao 
\verse rursum dormivit et vidit alterum somnium. Septem spicae pullulabant in culmo uno plenae atque formosae.  
\verse Aliae quoque totidem spicae tenues et percussae vento urente oriebantur 
\verse devorantes omnem priorum pulchritudinem. Evigilavit pharao, et ecce erat somnium! 
\verse Et, facto mane, pavore perterritus misit ad omnes coniectores Aegypti cunctosque sapientes suos; et accersitis narravit somnium, nec erat qui interpretaretur. 
\verse Tunc demum reminiscens pincernarum magister ait: “Confiteor peccatum meum. 
\verse Iratus rex servis suis me et magistrum pistorum retrudi iussit in carcerem principis satellitum, 
\verse ubi una nocte uterque vidimus somnium praesagum futurorum. 
\verse Erat ibi puer Hebraeus eiusdem ducis satellitum famulus, cui narrantes somnia 
\verse audivimus quidquid postea rei probavit eventus. Ego enim redditus sum officio meo, et ille suspensus est in patibulo". 
\verse Protinus ad regis imperium eductum de carcere Ioseph totonderunt ac, veste mutata, obtulerunt ei. 
\verse Cui ille ait: “Vidi somnia, nec est qui edisserat; quae audivi te sapientissime conicere". 
\verse Respondit Ioseph: “Absque me Deus respondebit prospera pharaoni!". 
\verse Narravit ergo pharao, quod viderat: “Putabam me stare super ripam fluminis 
\verse et septem boves de amne conscendere pulchras nimis et obesis carnibus, quae in pastu paludis virecta carpebant. 
\verse Et ecce has sequebantur aliae septem boves in tantum deformes et macilentae, ut numquam tales in terra Aegypti viderim; 
\verse quae, devoratis et consumptis prioribus, 
\verse nullum saturitatis dedere vestigium; sed simili macie et squalore torpebant. Evigilans, rursus sopore depressus, 
\verse vidi somnium: Septem spicae pullulabant in culmo uno plenae atque pulcherrimae. 
\verse Aliae quoque septem tenues et percussae vento urente oriebantur e stipula; 
\verse quae priorum pulchritudinem devoraverunt. Narravi coniectoribus somnium, et nemo est qui edisserat". 
\verse Respondit Ioseph: “Somnium regis unum est: quae facturus est, Deus ostendit pharaoni. 
\verse Septem boves pulchrae et septem spicae plenae septem ubertatis anni sunt; eandemque vim somnii comprehendunt. 
\verse Septem quoque boves tenues atque macilentae, quae ascenderunt post eas, et septem spicae tenues et vento urente percussae septem anni sunt venturae famis, 
\verse qui hoc ordine complebuntur: 
\verse ecce septem anni venient fertilitatis magnae in universa terra Aegypti; 
\verse quos sequentur septem anni alii tantae sterilitatis, ut oblivioni tradatur cuncta retro abundantia. Consumptura est enim fames omnem terram, 
\verse et ubertatis magnitudinem perditura est inopiae magnitudo. 
\verse Quod autem vidisti secundo ad eandem rem pertinens somnium, firmitatis indicium est, eo quod fiat sermo Dei et velocius a Deo impleatur. 
\verse Nunc ergo provideat rex virum intellegentem et sapientem et praeficiat eum terrae Aegypti 
\verse constituatque praepositos per cunctas regiones et quintam partem fructuum per septem annos fertilitatis, 
\verse qui iam nunc futuri sunt, congreget in horrea; et omne frumentum sub pharaonis potestate condatur serveturque in urbibus; 
\verse et paretur futurae septem annorum fami, quae pressura est Aegyptum, et non consumetur terra inopia". 
\verse Placuit pharaoni consilium et cunctis ministris eius. 
\verse Locutusque est ad eos: “Num invenire poterimus talem virum, qui spiritu Dei plenus sit?".  
\verse Dixit ergo ad Ioseph: “Quia ostendit tibi Deus omnia, quae locutus es, numquid sapientiorem et consimilem tui invenire potero? 
\verse Tu eris super domum meam, et ad tui oris imperium cunctus populus meus oboediet; uno tantum regni solio te praecedam". 
\verse Dixitque rursus pharao ad Ioseph: “Ecce, constitui te super universam terram Aegypti". 
\verse Tulitque anulum de manu sua et dedit eum in manu eius; vestivitque eum stola byssina et collo torquem auream circumposuit. 
\verse Fecitque eum ascendere super currum suum secundum, clamante praecone: “Abrech!", ut omnes coram eo genuflecterent et praepositum esse scirent universae terrae Aegypti. 
\verse Dixit quoque rex ad Ioseph: “Ego sum pharao; absque tuo imperio non movebit quisquam manum aut pedem in omni terra Aegypti". 
\verse Vertitque nomen eius et vocavit eum lingua Aegyptiaca Saphaneth Phanec (quod interpretatur Salvator mundi) deditque illi uxorem Aseneth filiam Putiphare sacerdotis Heliopoleos. Egressus est itaque Ioseph ad terram Aegypti 
\verse ­ triginta autem annorum erat quando stetit in conspectu regis pharaonis ­ et circuivit omnes regiones Aegypti. 
\verse Venitque fertilitas septem annorum, et segetes congregavit in horrea Aegypti 
\verse condens in singulis urbibus frumentum camporum in circuitu. 
\verse Tantaque fuit abundantia tritici, ut arenae maris coaequaretur, et copia mensuram excederet. 
\verse Nati sunt autem Ioseph filii duo, antequam veniret fames, quos ei peperit Aseneth filia Putiphare sacerdotis Heliopoleos. 
\verse Vocavitque nomen primogeniti Manasses dicens: “Oblivisci me fecit Deus omnium laborum meorum et domus patris mei". 
\verse Nomen quoque secundi appellavit Ephraim dicens: “Crescere me fecit Deus in terra paupertatis meae". 
\verse Igitur, transactis septem annis ubertatis, qui fuerant in Aegypto, 
\verse coeperunt venire septem anni inopiae, quos praedixerat Ioseph, et in universo orbe fames praevaluit; in cuncta autem terra Aegypti erat panis. 
\verse Qua esuriente, clamavit populus ad pharaonem alimenta petens. Quibus ille respondit: “Ite ad Ioseph et, quidquid vobis dixerit, facite". 
\verse Et invaluit fames in omni terra Aegypti; aperuitque Ioseph universa horrea et vendebat Aegyptiis; nam et illos oppresserat fames. 
\verse Omnesque provinciae veniebant in Aegyptum, ut emerent escas apud Ioseph, quia inopia invaluerat in universa terra. 
\end{biblechapter}

\begin{biblechapter}  
\verse Audiens autem Iacob quod alimenta venderentur in Aegypto, dixit filiis suis: “Quare aspicitis vos invicem? 
\verse Audivi quod triticum venumdetur in Aegypto; descendite et emite nobis necessaria, ut possimus vivere et non consumamur inopia". 
\verse Descenderunt igitur fratres Ioseph decem, ut emerent frumenta in Aegypto,  
\verse Beniamin fratre Ioseph domi retento a Iacob, qui dixerat fratribus eius: “Ne forte in itinere quidquam patiatur mali". 
\verse Et ingressi sunt filii Israel terram Aegypti cum aliis, qui pergebant ad emendum. Erat autem fames in terra Chanaan. 
\verse Et Ioseph erat princeps in terra Aegypti, atque ad eius nutum frumenta populis vendebantur. Cumque venissent et adorassent eum fratres sui proni in terram,  
\verse et agnovisset eos, quasi ad alienos durius loquebatur interrogans eos: “Unde venistis?". Qui responderunt: “De terra Chanaan, ut emamus victui necessaria". 
\verse Et tamen fratres ipse cognoscens non est cognitus ab eis. 
\verse Recordatusque somniorum, quae aliquando viderat, ait ad eos: “Exploratores estis; ut videatis infirmiora terrae, venistis!". 
\verse Qui dixerunt: “Non est ita, domine; sed servi tui venerunt, ut emerent cibos. 
\verse Omnes filii unius viri sumus; sinceri sumus, nec quidquam famuli tui machinantur mali". 
\verse Quibus ille respondit: “Aliter est; immunita terrae huius considerare venistis!". 
\verse At illi: “Duodecim, inquiunt, servi tui fratres sumus filii viri unius in terra Chanaan; minimus cum patre nostro est, alius non est super". 
\verse “Hoc est, ait, quod locutus sum: exploratores estis! 
\verse Iam nunc experimentum vestri capiam: per salutem pharaonis, non egrediemini hinc, donec veniat frater vester minimus! 
\verse Mittite ex vobis unum, et adducat eum; vos autem eritis in vinculis, donec probentur, quae dixistis, utrum vera an falsa sint. Alioquin, per salutem pharaonis, exploratores estis!". 
\verse Tradidit ergo illos custodiae tribus diebus. 
\verse Die autem tertio eductis de carcere, ait: “Facite, quae dixi, et vivetis; Deum enim timeo. 
\verse Si sinceri estis, frater vester unus ligetur in carcere; vos autem abite et ferte frumenta, quae emistis, in domos vestras, 
\verse et fratrem vestrum minimum ad me adducite, ut possim vestros probare sermones, et non moriamini". Fecerunt, ut dixerat, 
\verse et locuti sunt ad invicem: “Merito haec patimur, quia peccavimus in fratrem nostrum videntes angustiam animae illius, cum deprecaretur nos, et non audivimus. Idcirco venit super nos ista tribulatio". 
\verse Et Ruben ait: “Numquid non dixi vobis: Nolite peccare in puerum? Et non audistis me. En sanguis eius exquiritur". 
\verse Nesciebant autem quod intellegeret Ioseph, eo quod per interpretem loquebatur ad eos. 
\verse Avertitque se parumper et flevit; et reversus locutus est ad eos.  
\verse Tollensque Simeon et ligans, illis praesentibus, iussit ministris, ut implerent eorum saccos tritico et reponerent pecunias singulorum in sacculis suis, datis supra cibariis in viam. Qui fecerunt ita. 
\verse At illi portantes frumenta in asinis suis profecti sunt. 
\verse Apertoque unus sacco, ut daret iumento pabulum in deversorio, contemplatus pecuniam in ore sacculi 
\verse dixit fratribus suis: “Reddita est mihi pecunia: en habetur in sacco!". Et obstupefacti turbatique mutuo dixerunt: “Quidnam est hoc, quod fecit nobis Deus?". 
\verse Veneruntque ad Iacob patrem suum in terram Chanaan; et narraverunt ei omnia, quae accidissent sibi, dicentes: 
\verse “Locutus est nobis dominus terrae dure et putavit nos exploratores esse provinciae". 
\verse Cui respondimus: “Sinceri sumus, nec ullas molimur insidias; 
\verse duodecim fratres uno patre geniti sumus, unus non est super, minimus cum patre nostro est in terra Chanaan. 
\verse Et dixit nobis vir, dominus terrae: "Sic probabo quod sinceri sitis: fratrem vestrum unum dimittite apud me et cibaria domibus vestris necessaria sumite et abite; 
\verse fratremque vestrum minimum adducite ad me, ut sciam quod non sitis exploratores et istum, qui tenetur in vinculis, recipere possitis ac deinceps peragrandi terram habeatis licentiam"". 
\verse His dictis, cum frumenta effunderent, singuli reppererunt in ore saccorum ligatas pecunias; exterritisque simul omnibus, 
\verse dixit pater Iacob: “Absque liberis me esse fecistis: Ioseph non est super, Simeon tenetur in vinculis, et Beniamin auferetis. In me haec omnia mala reciderunt". 
\verse Cui respondit Ruben: “Duos filios meos interfice, si non reduxero illum tibi; trade illum in manu mea, et ego eum tibi restituam". 
\verse At ille: “Non descendet, inquit, filius meus vobiscum. Frater mortuus est, et ipse solus remansit; si quid ei adversi acciderit in via, deducetis canos meos cum dolore ad inferos". 
\end{biblechapter}

\begin{biblechapter}  
\verse Interim fames omnem terram vehementer premebat; 
\verse consumptisque cibis, quos ex Aegypto detulerant, dixit Iacob ad filios suos: “Revertimini et emite nobis pauxillum escarum". 
\verse Respondit Iudas: “Denuntiavit nobis vir ille sub attestatione iurisiurandi dicens: "Non videbitis faciem meam, nisi fratrem vestrum minimum adduxeritis vobiscum". 
\verse Si ergo vis eum mittere nobiscum, pergemus pariter et ememus tibi necessaria; 
\verse sin autem non vis, non ibimus. Vir enim, ut saepe diximus, denuntiavit nobis dicens: "Non videbitis faciem meam absque fratre vestro minimo"". 
\verse Dixit eis Israel: “Cur in meam hoc fecistis miseriam, ut indicaretis ei et alium habere vos fratrem?". 
\verse At illi responderunt: “Interrogavit nos homo per ordinem nostram progeniem: si pater viveret, si haberemus fratrem; et nos respondimus ei consequenter iuxta id, quod fuerat sciscitatus. Numquid scire poteramus quod dicturus esset: "Adducite fratrem vestrum vobiscum?"". 
\verse Iudas quoque dixit patri suo Israel: “Mitte puerum mecum, ut proficiscamur et possimus vivere, ne moriamur nos et tu et parvuli nostri. 
\verse Ego spondeo pro puero; de manu mea require illum. Nisi reduxero et reddidero eum tibi, ero peccati reus in te omni tempore. 
\verse Si non intercessisset dilatio, iam vice altera venissemus". 
\verse Igitur Israel pater eorum dixit ad eos: “Si sic necesse est, facite, quod vultis; sumite de optimis terrae fructibus in vasis vestris et deferte viro munera: modicum resinae et mellis et tragacanthum et ladanum, pistacias terebinthi et amygdalas. 
\verse Pecuniam quoque duplicem ferte vobiscum et illam, quam invenistis in sacculis, reportate, ne forte errore factum sit;  
\verse sed et fratrem vestrum tollite et ite ad virum. 
\verse Deus autem meus omnipotens faciat vobis eum placabilem, et remittat vobiscum fratrem vestrum, quem tenet, et hunc Beniamin. Ego autem quasi orbatus absque liberis ero". 
\verse Tulerunt ergo viri munera et pecuniam duplicem et Beniamin descenderuntque in Aegyptum; et steterunt coram Ioseph. 
\verse Quos cum ille vidisset et Beniamin simul, praecepit dispensatori domus suae dicens: “Introduc viros domum et occide victimas et instrue convivium, quoniam mecum sunt comesturi meridie". 
\verse Fecit ille, quod sibi fuerat imperatum, et introduxit viros in domum Ioseph. 
\verse Ibique exterriti dixerunt mutuo: “Propter pecuniam, quam rettulimus prius in saccis nostris, introducti sumus, ut irruant in nos et violenter subiciant servituti et nos et asinos nostros". 
\verse Quam ob rem in ipsis foribus accedentes ad dispensatorem domus 
\verse locuti sunt: “Oramus, domine, ut audias nos. Iam ante descendimus, ut emeremus escas; 
\verse quibus emptis, cum venissemus ad deversorium, aperuimus saccos nostros et invenimus pecuniam in ore saccorum; quam nunc eodem pondere reportavimus. 
\verse Sed et aliud attulimus argentum, ut emamus, quae nobis necessaria sunt. Non est in nostra conscientia, quis posuerit argentum in marsupiis nostris". 
\verse At ille respondit: “Pax vobiscum, nolite timere. Deus vester et Deus patris vestri dedit vobis thesauros in saccis vestris; nam pecuniam, quam dedistis mihi, probatam ego habeo". Eduxitque ad eos Simeon. 
\verse Et introductis domum attulit aquam, et laverunt pedes suos; deditque pabulum asinis eorum. 
\verse Illi vero parabant munera, donec ingrederetur Ioseph meridie; audierant enim quod ibi comesturi essent panem. 
\verse Igitur ingressus est Ioseph domum suam, obtuleruntque ei munera tenentes in manibus suis; et adoraverunt proni in terram. 
\verse At ille, clementer resalutatis eis, interrogavit eos dicens: “Salvusne est pater vester senex, de quo dixeratis mihi? Adhuc vivit?". 
\verse Qui responderunt: “Sospes est servus tuus pater noster, adhuc vivit". Et incurvati adoraverunt eum. 
\verse Attollens autem Ioseph oculos vidit Beniamin fratrem suum uterinum et ait: “Iste est frater vester parvulus, de quo dixeratis mihi?". Et rursum: “Deus, inquit, misereatur tui, fili mi". 
\verse Festinavitque, quia commota fuerant viscera eius super fratre suo, et erumpebant lacrimae; et introiens cubiculum flevit. 
\verse Rursumque, lota facie, egressus continuit se et ait: “Ponite panes". 
\verse Quibus appositis, seorsum Ioseph et seorsum fratribus, Aegyptiis quoque, qui vescebantur simul, seorsum ­ illicitum est enim Aegyptiis comedere cum Hebraeis, et profanum putant huiuscemodi convivium ­ 
\verse sederunt coram eo, primogenitus iuxta primogenita sua et minimus iuxta aetatem suam. Et mirabantur nimis, 
\verse sumptis partibus, quas ab eo acceperant; maiorque pars venit Beniamin, ita ut quinque partibus excederet. Biberuntque et inebriati sunt cum eo. 
\end{biblechapter}

\begin{biblechapter}  
\verse Praecepit autem Ioseph dispensatori domus suae dicens: “Imple saccos eorum frumento, quantum possunt capere, et pone pecuniam singulorum in summitate sacci. 
\verse Scyphum autem meum argenteum et pretium, quod dedit tritici, pone in ore sacci iunioris". Factumque est ita. 
\verse Et, orto mane, dimissi sunt cum asinis suis. 
\verse Iamque urbem exierant et processerant paululum, tunc Ioseph, arcessito dispensatore domus: “Surge, inquit, et persequere viros; et apprehensis dicito: "Quare reddidistis malum pro bono? Cur furati estis mihi scyphum argenteum? 
\verse Nonne ipse est, in quo bibit dominus meus et in quo augurari solet? Pessimam rem fecistis!"". 
\verse Fecit ille, ut iusserat, et apprehensis per ordinem locutus est. 
\verse Qui responderunt: “Quare sic loquitur dominus noster? Absit a servis tuis, ut tantum flagitii commiserimus. 
\verse Pecuniam, quam invenimus in summitate saccorum, reportavimus ad te de terra Chanaan; et quomodo consequens est, ut furati simus de domo domini tui aurum vel argentum? 
\verse Apud quemcumque fuerit inventum servorum tuorum, quod quaeris, moriatur; et nos erimus servi domini nostri". 
\verse Qui dixit eis: “Fiat iuxta vestram sententiam: apud quemcumque fuerit inventum, ipse sit servus meus; vos autem eritis innoxii". 
\verse Itaque festinato deponentes in terram saccos aperuerunt singuli. 
\verse Quos scrutatus incipiens a maiore usque ad minimum invenit scyphum in sacco Beniamin. 
\verse At illi, scissis vestibus, oneratisque rursum asinis, reversi sunt in oppidum. 
\verse Et Iudas cum fratribus ingressus est ad Ioseph ­ necdum enim de loco abierat ­ omnesque ante eum pariter in terram corruerunt. 
\verse Quibus ille ait: “Cur sic agere voluistis? An ignoratis quod non sit similis mei in augurandi scientia?". 
\verse Cui Iudas: “Quid respondebimus, inquit, domino meo? Vel quid loquemur aut iuste poterimus obtendere? Deus invenit iniquitatem servorum tuorum; en omnes servi sumus domini mei, et nos et apud quem inventus est scyphus". 
\verse Respondit Ioseph: “Absit a me, ut sic agam! Qui furatus est scyphum, ipse sit servus meus; vos autem abite liberi ad patrem vestrum". 
\verse Accedens autem propius Iudas confidenter ait: “Oro, domine mi, loquatur servus tuus verbum in auribus tuis, et ne irascaris famulo tuo; tu es enim sicut pharao! 
\verse Dominus meus interrogavit prius servos suos: "Habetis patrem aut fratrem?". 
\verse Et nos respondimus domino meo: "Est nobis pater senex et puer parvulus, qui in senectute illius natus est, cuius uterinus frater mortuus est; et ipse solus superest a matre sua, pater vero tenere diligit eum"". 
\verse Dixistique servis tuis: "Adducite eum ad me, et ponam oculos meos super illum". 
\verse Suggessimus domino meo: "Non potest puer relinquere patrem suum; si enim illum dimiserit, morietur". 
\verse Et dixisti servis tuis: "Nisi venerit frater vester minimus vobiscum, non videbitis amplius faciem meam". 
\verse Cum ergo ascendissemus ad famulum tuum patrem nostrum, narravimus ei omnia, quae locutus est dominus meus, 
\verse et dixit pater noster: "Revertimini et emite nobis parum tritici". 
\verse Cui diximus: "Ire non possumus. Si frater noster minimus descenderit nobiscum, proficiscemur simul; alioquin, illo absente, non poterimus videre faciem viri". 
\verse Ad quae servus tuus pater meus respondit: "Vos scitis quod duos genuerit mihi uxor mea. 
\verse Egressus est unus a me, et dixi: Bestia devoravit eum! Et hucusque non comparet. 
\verse Si tuleritis et istum a facie mea, et aliquid ei in via contigerit, deducetis canos meos cum maerore ad inferos". 
\verse Igitur, si intravero ad servum tuum patrem meum, et puer defuerit ­ cum anima illius ex huius anima pendeat ­ 
\verse videritque eum non esse nobiscum, morietur; et deducent famuli tui canos eius cum dolore ad inferos. 
\verse Servus tuus pro puero patri meo spopondit: Nisi reduxero eum, peccati reus ero in patrem meum omni tempore. 
\verse Manebo itaque servus tuus pro puero in ministerio domini mei, et puer ascendat cum fratribus suis. 
\verse Non enim possum redire ad patrem meum, absente puero, ne calamitatis, quae oppressura est patrem meum, testis assistam". 
\end{biblechapter}

\begin{biblechapter}  
\verse Non se poterat ultra cohibere Ioseph omnibus coram astantibus, unde clamavit: “Egredimini, cuncti, foras!". Et nemo aderat cum eo, quando manifestavit se fratribus suis. 
\verse Elevavitque vocem cum fletu, quam audierunt Aegyptii omnisque domus pharaonis. 
\verse Et dixit Ioseph fratribus suis: “Ego sum Ioseph! Adhuc pater meus vivit?". Nec poterant respondere fratres nimio terrore perterriti. 
\verse Ad quos ille clementer: “Accedite, inquit, ad me". Et cum accessissent prope: “Ego sum, ait, Ioseph frater vester, quem vendidistis in Aegyptum. 
\verse Nolite contristari, neque vobis durum esse videatur quod vendidistis me in his regionibus. Pro salute enim vestra misit me Deus ante vos in Aegyptum. 
\verse Biennium est enim quod coepit fames esse in terra, et adhuc quinque anni restant, quibus nec arari poterit nec meti. 
\verse Praemisitque me Deus, ut reservemini super terram, et servetur vita vestra in salvationem magnam. 
\verse Non vestro consilio, sed Dei voluntate huc missus sum, qui fecit me quasi patrem pharaonis et dominum universae domus eius ac principem in omni terra Aegypti. 
\verse Festinate et ascendite ad patrem meum et dicetis ei: "Haec mandat filius tuus Ioseph: Deus fecit me dominum universae terrae Aegypti; descende ad me, ne moreris. 
\verse Et habitabis in terra Gessen; erisque iuxta me tu et filii tui et filii filiorum tuorum, oves tuae et armenta tua et universa, quae possides.  
\verse Ibique te pascam ­ adhuc enim quinque anni residui sunt famis ­ ne et tu pereas et domus tua et omnia, quae possides". 
\verse En oculi vestri et oculi fratris mei Beniamin vident quia os meum est, quod loquitur ad vos. 
\verse Nuntiate patri meo universam gloriam meam in Aegypto et cuncta, quae vidistis. Festinate et adducite eum ad me". 
\verse Cumque amplexatus recidisset in collum Beniamin fratris sui, flevit, illo quoque similiter flente, super collum eius. 
\verse Osculatusque est Ioseph omnes fratres suos et ploravit super singulos. Post quae ausi sunt loqui ad eum. 
\verse Auditumque est et celebri sermone vulgatum in aula regis: “Venerunt fratres Ioseph!". Et gavisus est pharao atque omnis familia eius. 
\verse Dixitque ad Ioseph, ut imperaret fratribus suis dicens: “Onerantes iumenta ite in terram Chanaan 
\verse et tollite inde patrem vestrum et cognationem et venite ad me; et ego dabo vobis omnia bona Aegypti, ut comedatis medullam terrae. 
\verse Praecipe etiam: tollite plaustra de terra Aegypti ad subvectionem parvulorum vestrorum ac coniugum et tollite patrem vestrum et properate quantocius venientes. 
\verse Nec doleatis super supellectilem vestram, quia omnes opes Aegypti vestrae erunt". 
\verse Feceruntque filii Israel, ut eis mandatum fuerat. Quibus dedit Ioseph plaustra secundum pharaonis imperium et cibaria in itinere. 
\verse Singulis quoque proferri iussit vestimentum mutatorium; Beniamin vero dedit trecentos argenteos cum quinque 
\verse vestimentis mutatoriis. Patri suo misit similiter asinos decem, qui subveherent ex omnibus divitiis Aegypti, et totidem asinas triticum et panem et cibum pro itinere portantes. 
\verse Dimisit ergo fratres suos et proficiscentibus ait: “Ne irascamini in via!". 
\verse Qui ascendentes ex Aegypto venerunt in terram Chanaan ad patrem suum Iacob  
\verse et nuntiaverunt ei dicentes: “Ioseph vivit et ipse dominatur in omni terra Aegypti!". At cor eius frigidum mansit; non enim credebat eis. 
\verse Tunc referebant omnia verba Ioseph, quae dixerat eis. Cumque vidisset plaustra et universa, quae miserat ad adducendum eum, revixit spiritus eius, 
\verse et ait: “Sufficit mihi, si adhuc Ioseph filius meus vivit. Vadam et videbo illum, antequam moriar". 
\end{biblechapter}

\begin{biblechapter}  
\verse Profectusque Israel cum omnibus, quae habebat, venit Bersabee et, mactatis ibi victimis Deo patris sui Isaac, 
\verse audivit eum per visionem noctis vocantem se: “Iacob, Iacob!". Cui respondit: “Ecce adsum!". 
\verse Ait illi: “Ego sum Deus, Deus patris tui. Noli timere descendere in Aegyptum, quia in gentem magnam faciam te ibi. 
\verse Ego descendam tecum illuc et ego inde adducam te revertentem; Ioseph quoque ponet manus suas super oculos tuos". 
\verse Surrexit igitur Iacob a Bersabee, tuleruntque eum filii cum parvulis et uxoribus suis in plaustris, quae miserat pharao ad portandum senem, 
\verse sumpserunt quoque omnia, quae possederant in terra Chanaan; veneruntque in Aegyptum Iacob et omne semen eius, 
\verse filii eius et nepotes, filiae et cuncta simul progenies. 
\verse Haec sunt autem nomina filiorum Israel, qui ingressi sunt in Aegyptum, ipse cum liberis suis. Primogenitus Ruben. 
\verse Filii Ruben: Henoch et Phallu et Hesron et Charmi. 
\verse Filii Simeon: Iamuel et Iamin et Ahod et Iachin et Sohar et Saul filius Chananitidis. 
\verse Filii Levi: Gerson et Caath et Merari. 
\verse Filii Iudae: Her et Onan et Sela et Phares et Zara. Mortui sunt autem Her et Onan in terra Chanaan. Natique sunt filii Phares: Esrom et Hamul. 
\verse Filii Issachar: Thola et Phua et Iasub et Semron. 
\verse Filii Zabulon: Sared et Elon et Iahelel. 
\verse Hi filii Liae, quos genuit in Paddanaram, cum Dina filia sua. Omnes animae filiorum eius et filiarum triginta tres. 
\verse Filii Gad: Sephon et Haggi, Suni et Esebon, Heri et Arodi et Areli. 
\verse Filii Aser: Iemna et lesua et Isui et Beria, Sara quoque soror eorum. Filii Beria: Heber et Melchiel. 
\verse Hi filii Zelphae, quam dedit Laban Liae filiae suae; et hos genuit Iacob: sedecim animas. 
\verse Filii Rachel uxoris Iacob: Ioseph et Beniamin. 
\verse Natique sunt Ioseph filii in terra Aegypti, quos genuit ei Aseneth filia Putiphare sacerdotis Heliopoleos: Manasses et Ephraim. 
\verse Filii Beniamin: Bela et Bochor et Asbel, Gera et Naaman et Echi et Ros, Mophim et Huphim et Ared. 
\verse Hi filii Rachel, quos genuit Iacob: omnes animae quattuordecim. 
\verse Filii Dan: Husim. 
\verse Filii Nephthali: Iasiel et Guni et Ieser et Sellem. 
\verse Hi filii Bilhae, quam dedit Laban Racheli filiae suae; et hos genuit Iacob: omnes animae septem. 
\verse Cunctae animae, quae ingressae sunt cum Iacob in Aegyptum et egressae de femore illius, absque uxoribus filiorum eius, sexaginta sex. 
\verse Filii autem Ioseph, qui nati sunt ei in terra Aegypti, animae duae. Omnes animae domus Iacob, quae ingressae sunt in Aegyptum, fuere septuaginta. 
\verse Misit autem Iudam ante se ad Ioseph, ut nuntiaret et occurreret in Gessen.  
\verse Et venerunt in terram Gessen. Iunctoque Ioseph curru suo, ascendit obviam patri suo in Gessen; vidensque eum irruit super collum eius et inter amplexus diu flevit. 
\verse Dixitque Israel ad Ioseph: “Iam laetus moriar, quia vidi faciem tuam et superstitem te relinquo". 
\verse Et ille locutus est ad fratres suos et ad omnem domum patris sui: “Ascendam et nuntiabo pharaoni dicamque ei: Fratres mei et domus patris mei, qui erant in terra Chanaan, venerunt ad me. 
\verse Et sunt viri pastores ovium curamque habent alendorum gregum; pecora sua et armenta et omnia, quae habere potuerunt, adduxerunt secum. 
\verse Cumque vocaverit vos et dixerit: "Quod est opus vestrum?". 
\verse Respondebitis: "Viri pastores sumus servi tui ab infantia nostra usque in praesens et nos et patres nostri". Haec autem dicetis, ut habitare possitis in terra Gessen, quia detestantur Aegyptii omnes pastores ovium". 
\end{biblechapter}

\begin{biblechapter}  
\verse Ingressus ergo Ioseph nuntiavit pharaoni dicens: “Pater meus et fratres, oves eorum et armenta et cuncta, quae possident, venerunt de terra Chanaan; et ecce consistunt in terra Gessen". 
\verse Ex omnibus fratribus suis quinque viros statuit coram rege, 
\verse quos ille interrogavit: “Quid habetis operis?". Responderunt: “Pastores ovium sumus servi tui et nos et patres nostri". 
\verse Dixeruntque ad pharaonem: “Ad peregrinandum in terra venimus, quoniam non est herba gregibus servorum tuorum, ingravescente fame, in terra Chanaan petimusque, ut esse nos iubeas servos tuos in terra Gessen". 
\verse Dixit itaque rex ad Ioseph: “Pater tuus et fratres tui venerunt ad te. 
\verse Terra Aegypti in conspectu tuo est; in optimo loco fac eos habitare et trade eis terram Gessen. Quod si nosti in eis esse viros industrios, constitue illos magistros pecorum meorum". 
\verse Post haec introduxit Ioseph patrem suum ad regem et statuit eum coram eo, qui benedicens illi 
\verse et interrogatus ab eo: “Quot sunt dies annorum vitae tuae?", 
\verse respondit: “Dies peregrinationis meae centum triginta annorum sunt, parvi et mali; et non pervenerunt usque ad dies patrum meorum, quibus peregrinati sunt". 
\verse Et benedicto rege, egressus est foras. 
\verse Ioseph vero patri et fratribus suis dedit possessionem in Aegypto in optimo terrae loco, in terra Ramesses, ut praeceperat pharao; 
\verse et alebat eos omnemque domum patris sui praebens cibaria singulis. 
\verse In tota terra panis deerat, et oppresserat fames terram valde, defecitque terra Aegypti et terra Chanaan prae fame. 
\verse E quibus omnem pecuniam congregavit pro venditione frumenti et intulit eam in aerarium regis. 
\verse Cumque defecisset emptoribus pretium, venit cuncta Aegyptus ad Ioseph dicens: “Da nobis panes! Quare morimur coram te, deficiente pecunia?". 
\verse Quibus ille respondit: “Adducite pecora vestra, et dabo vobis pro eis cibos, si pretium non habetis". 
\verse Quae cum adduxissent, dedit eis alimenta pro equis et ovibus et bobus et asinis; sustentavitque eos illo anno pro commutatione pecorum. 
\verse Venerunt quoque anno secundo et dixerunt ei: “Non celamus dominum nostrum quod, deficiente pecunia, pecora transierunt ad dominum nostrum; nec clam te est quod absque corporibus et terra nihil habeamus. 
\verse Cur ergo moriemur, te vidente, et nos et terra nostra? Eme nos et terram nostram in servitutem regiam et praebe semina, ne, pereunte cultore, redigatur terra in solitudinem". 
\verse Emit igitur Ioseph omnem terram Aegypti, vendentibus singulis possessiones suas prae magnitudine famis. Subiecitque eam pharaoni 
\verse et cunctos populos eius redegit ei in servitutem, a novissimis terminis Aegypti usque ad extremos fines eius. 
\verse Terram autem sacerdotum non emit, qui cibariis a rege statutis fruebantur, et idcirco non sunt compulsi vendere possessiones suas. 
\verse Dixit ergo Ioseph ad populos: “En, ut cernitis, et vos et terram vestram pharao possidet; accipite semina et serite agros, 
\verse ut fruges habere possitis. Quintam partem regi dabitis; quattuor reliquas permitto vobis in sementem et in cibum familiis et liberis vestris". 
\verse Qui responderunt: “Tu salvasti nos! Respiciat nos tantum dominus noster, et laeti serviemus regi". 
\verse Ex eo tempore usque in praesentem diem in universa terra Aegypti regibus quinta pars solvitur; et factum est a Ioseph in legem absque terra sacerdotali, quae libera ab hac condicione est. 
\verse Habitavit ergo Israel in Aegypto, id est in terra Gessen, et possedit eam; auctusque est et multiplicatus nimis. 
\verse Et vixit Iacob in terra Aegypti decem et septem annis; factique sunt omnes dies vitae illius centum quadraginta septem annorum. 
\verse Cumque appropinquare cerneret diem mortis suae, vocavit filium suum Ioseph et dixit ad eum: “Si inveni gratiam in conspectu tuo, pone manum tuam sub femore meo et facies mihi misericordiam et veritatem, ut non sepelias me in Aegypto,  
\verse sed dormiam cum patribus meis, et auferas me de terra hac condasque in sepulcro maiorum meorum". Cui respondit Ioseph: “Ego faciam, quod iussisti".  
\verse Et ille: “Iura ergo, inquit, mihi!". Quo iurante, adoravit Israel conversus ad lectuli caput. 
\end{biblechapter}

\begin{biblechapter}  
\verse His ita transactis, nuntiatum est Ioseph quod aegrotaret pater suus. Et assumpsit secum duos filios Manasse et Ephraim. 
\verse Dictumque est seni: “Ecce filius tuus Ioseph venit ad te". Qui confortatus sedit in lectulo 
\verse et ingresso ad se ait: “Deus omnipotens apparuit mihi in Luza, quae est in terra Chanaan, benedixitque mihi 
\verse et ait: "Ego te augebo et multiplicabo et faciam te in multitudinem populorum; daboque tibi terram hanc et semini tuo post te in possessionem sempiternam". 
\verse Duo ergo filii tui, qui nati sunt tibi in terra Aegypti, antequam huc venirem ad te, mei erunt: Ephraim et Manasses sicut Ruben et Simeon reputabuntur mihi. 
\verse Reliquos autem, quos genueris post eos, tui erunt et nomine fratrum suorum vocabuntur in possessionibus suis. 
\verse Mihi enim, quando veniebam de Paddanaram, mortua est Rachel mater tua in terra Chanaan in ipso itinere, cum adhuc esset spatium aliquod usque ad Ephratham, et sepelivi eam iuxta viam Ephrathae, quae alio nomine appellatur Bethlehem". 
\verse Videns autem filios eius dixit ad eum: “Qui sunt isti?". 
\verse Respondit: “Filii mei sunt, quos donavit mihi Deus in hoc loco". “Adduc, inquit, eos ad me, ut benedicam illis!". 
\verse Oculi enim Israel caligabant prae nimia senectute, et clare videre non poterat. Applicitosque ad se deosculatus et circumplexus eos 
\verse dixit ad filium suum: “Non sum fraudatus aspectu tuo; insuper ostendit mihi Deus semen tuum". 
\verse Cumque tulisset eos Ioseph de gremio patris, adoravit pronus in terram. 
\verse Et posuit Ephraim ad dexteram suam, id est ad sinistram Israel, Manassen vero in sinistra sua, ad dexteram scilicet patris; applicuitque ambos ad eum. 
\verse Qui extendens manum dexteram, posuit super caput Ephraim minoris fratris, sinistram autem super caput Manasse, qui maior natu erat, commutans manus. 
\verse Benedixitque Iacob Ioseph et ait: “Deus, in cuius conspectu ambulaverunt patres mei Abraham et Isaac, Deus, qui pascit me ab adulescentia mea usque in praesentem diem, 
\verse Angelus, qui eruit me de cunctis malis, benedicat pueris istis! Et invocetur super eos nomen meum, nomina quoque patrum meorum Abraham et Isaac, et crescant in multitudinem super terram!". 
\verse Videns autem Ioseph quod posuisset pater suus dexteram manum super caput Ephraim, graviter accepit et apprehensam manum patris levare conatus est de capite Ephraim et transferre super caput Manasse. 
\verse Dixitque ad patrem: “Non ita convenit, pater, quia hic est primogenitus; pone dexteram tuam super caput eius!". 
\verse Qui renuens ait: “Scio, fili mi, scio; et iste quidem erit in populos et multiplicabitur, sed frater eius minor maior erit illo, et semen illius crescet in plenitudinem gentium". 
\verse Benedixitque eis in die illo dicens: “In te benedicet Israel atque dicet: "Faciat te Deus sicut Ephraim et sicut Manasse!"". Constituitque Ephraim ante Manassen. 
\verse Et ait ad Ioseph filium suum: “En ego morior, et erit Deus vobiscum reducetque vos ad terram patrum vestrorum.  
\verse Do tibi partem unam extra fratres tuos, quam tuli de manu Amorraei in gladio et arcu meo". 
\end{biblechapter}

\begin{biblechapter}  
\verse Vocavit autem Iacob filios suos et ait eis: “Congregamini, ut annuntiem, quae ventura sunt vobis in diebus novissimis. 
\verse Congregamini et audite, filii Iacob, audite Israel patrem vestrum! 
\verse Ruben primogenitus meus, tu fortitudo mea et principium roboris mei; prior in dignitate, maior in robore! 
\verse Ebulliens sicut aqua non excellas, quia ascendisti cubile patris tui et maculasti stratum meum. 
\verse Simeon et Levi fratres, vasa violentiae arma eorum. 
\verse In consilium eorum ne veniat anima mea, et in coetu illorum non sit gloria mea; quia in furore suo occiderunt virum et in voluntate sua subnervaverunt tauros. 
\verse Maledictus furor eorum, quia pertinax, et indignatio eorum, quia dura! Dividam eos in Iacob et dispergam eos in Israel. 
\verse Iuda, te laudabunt fratres tui; manus tua in cervicibus inimicorum tuorum; adorabunt te filii patris tui. 
\verse Catulus leonis Iuda: a praeda, fili mi, ascendisti; requiescens accubuit ut leo et quasi leaena; quis suscitabit eum? 
\verse Non auferetur sceptrum de Iuda et baculus ducis de pedibus eius, donec veniat ille, cuius est, et cui erit oboedientia gentium; 
\verse ligans ad vineam pullum suum et ad vitem filium asinae suae, lavabit in vino stolam suam et in sanguine uvae pallium suum; 
\verse nigriores sunt oculi eius vino et dentes eius lacte candidiores. 
\verse Zabulon in litore maris habitabit et in statione navium, pertingens usque ad Sidonem. 
\verse Issachar asinus fortis, accubans inter caulas 
\verse vidit requiem quod esset bona, et terram quod optima; et supposuit umerum suum ad portandum factusque est tributis serviens. 
\verse Dan iudicabit populum suum sicut una tribuum Israel. 
\verse Fiat Dan coluber in via, cerastes in semita, mordens calcanea equi, ut cadat ascensor eius retro. 
\verse Salutare tuum exspectabo, Domine! 
\verse Gad, latrones aggredientur eum, ipse autem aggredietur calcaneum eorum. 
\verse Aser, pinguis panis eius, et praebebit delicias regales. 
\verse Nephthali cerva emissa, dans cornua pulchra. 
\verse Arbor fructifera Ioseph, arbor fructifera super fontem: rami transcendunt murum. 
\verse Sed exasperaverunt eum et iurgati sunt, et adversati sunt illi habentes iacula. 
\verse Et confractus est arcus eorum, et dissoluti sunt nervi brachiorum eorum per manus Potentis Iacob, per nomen Pastoris, Lapidis Israel. 
\verse Deus patris tui erit adiutor tuus, et Omnipotens benedicet tibi benedictionibus caeli desuper, benedictionibus abyssi iacentis deorsum, benedictionibus uberum et vulvae. 
\verse Benedictiones patris tui confortatae sunt super benedictiones montium aeternorum, desiderium collium antiquorum; fiant in capite Ioseph et in vertice nazaraei inter fratres suos. 
\verse Beniamin lupus rapax; mane comedet praedam et vespere dividet spolia". 
\verse Omnes hi in tribubus Israel duodecim. Haec locutus est eis pater suus benedixitque singulis benedictionibus propriis. 
\verse Et praecepit eis dicens: “Ego congregor ad populum meum; sepelite me cum patribus meis in spelunca Machpela, quae est in agro Ephron Hetthaei 
\verse contra Mambre in terra Chanaan, quam emit Abraham cum agro ab Ephron Hetthaeo in possessionem sepulcri; 
\verse ibi sepelierunt eum et Saram uxorem eius, ibi sepultus est Isaac cum Rebecca coniuge sua, ibi et Lia condita iacet". 
\verse Finitisque mandatis, quibus filios instruebat, collegit pedes suos super lectulum et obiit; appositusque est ad populum suum. 
\end{biblechapter}

\begin{biblechapter}  
\verse Ioseph ruit super faciem patris flens et deosculans eum. 
\verse Praecepitque servis suis medicis, ut aromatibus condirent patrem. 
\verse Quibus iussa explentibus, transierunt quadraginta dies; iste quippe mos erat cadaverum conditorum. Flevitque eum Aegyptus septuaginta diebus. 
\verse Et, expleto planctus tempore, locutus est Ioseph ad familiam pharaonis: “Si inveni gratiam in conspectu vestro, loquimini in auribus pharaonis, 
\verse eo quod pater meus adiuraverit me dicens: "En morior; in sepulcro meo, quod fodi mihi in terra Chanaan, sepelies me"; ascendam nunc et sepeliam patrem meum ac revertar". 
\verse Dixitque ei pharao: “Ascende et sepeli patrem tuum, sicut adiuratus es". 
\verse Quo ascendente, ierunt cum eo omnes servi pharaonis, senes domus eius cunctique maiores natu terrae Aegypti, 
\verse domus Ioseph cum fratribus suis, absque parvulis et gregibus atque armentis, quae dereliquerant in terra Gessen.  
\verse Habuit quoque in comitatu currus et equites; et facta est turba non modica.  
\verse Veneruntque ad Gorenatad (id est Aream rhamni), quae sita est trans Iordanem; ubi celebrantes exsequias planctu magno atque vehementi impleverunt septem dies. 
\verse Quod cum vidissent habitatores terrae Chanaan, dixerunt: “Planctus magnus est iste Aegyptiis"; et idcirco vocatum est nomen loci illius Abelmesraim (id est Planctus Aegypti). 
\verse Fecerunt ergo filii Iacob, sicut praeceperat eis; 
\verse et portantes eum in terram Chanaan sepelierunt eum in spelunca Machpela, quam emerat Abraham cum agro in possessionem sepulcri ab Ephron Hetthaeo contra faciem Mambre. 
\verse Reversusque est Ioseph in Aegyptum cum fratribus suis et omni comitatu, sepulto patre. 
\verse Quo mortuo, timentes fratres eius et mutuo colloquentes: “Ne forte memor sit iniuriae, quam passus est, et reddat nobis omne malum, quod fecimus", 
\verse mandaverunt ei dicentes: “Pater tuus praecepit nobis, antequam moreretur,  
\verse ut haec tibi verbis illius diceremus: "Obsecro, ut obliviscaris sceleris fratrum tuorum et peccati atque malitiae, quam exercuerunt in te". Nos quoque oramus, ut servis Dei patris tui dimittas iniquitatem hanc". Quibus auditis, flevit Ioseph. 
\verse Veneruntque ad eum fratres sui et proni coram eo dixerunt: “Servi tui sumus". 
\verse Quibus ille respondit: “Nolite timere. Num Dei possumus resistere voluntati? 
\verse Vos cogitastis de me malum; sed Deus vertit illud in bonum, ut exaltaret me, sicut in praesentiarum cernitis, et salvos faceret multos populos. 
\verse Nolite timere: ego pascam vos et parvulos vestros". Consolatusque est eos et blande ac leniter est locutus. 
\verse Et habitavit in Aegypto cum omni domo patris sui; vixitque centum decem annis 
\verse et vidit Ephraim filios usque ad tertiam generationem; filii quoque Machir filii Manasse nati sunt in genibus Ioseph. 
\verse Quibus transactis, locutus est fratribus suis: “Post mortem meam Deus visitabit vos et ascendere vos faciet de terra ista ad terram, quam iuravit Abraham, Isaac et Iacob". 
\verse Cumque adiurasset eos atque dixisset: “Deus visitabit vos; asportate ossa mea vobiscum de loco isto", 
\verse mortuus est, expletis centum decem vitae suae annis. Et conditus aromatibus repositus est in loculo in Aegypto.    
\end{biblechapter}
