\biblebook{ Liber II Maccabaeorum}
\begin{biblechapter}
 \verse Fratribus, qui sunt per Aegyptum, Iudaeis salutem dicunt fratres, qui sunt in Hierosolymis Iudaei et qui in regione Iudaeae, pacem bonam. 
 \verse Et benefaciat vobis Deus et meminerit testamenti sui, quod locutus est ad Abraham et Isaac et Iacob servorum suorum fidelium; 
\verse et det vobis cor omnibus, ut colatis eum et faciatis eius voluntatem corde magno et animo volenti; 
\verse et adaperiat cor vestrum in lege sua et in praeceptis suis et faciat pacem; 
\verse et exaudiat orationes vestras et reconcilietur vobis nec vos deserat in tempore malo. 
\verse Et nunc hic sumus orantes pro vobis. 
\verse Regnante Demetrio, anno centesimo sexagesimo nono, nos Iudaei scripsimus vobis in tribulatione et impetu, qui supervenit nobis in istis annis, ex quo recessit Iason et, qui cum eo erant, a sancta terra et a regno 
\verse et portam succenderunt et effuderunt sanguinem innocentem; et oravimus ad Dominum et exauditi sumus et obtulimus sacrificia et similaginem et accendimus lucernas et proposuimus panes. 
\verse Et nunc ut frequentetis dies Scenopegiae mensis Casleu, 
 \verse anno centesimo octogesimo octavo.
 Qui sunt Hierosolymis et in Iudaea, senatusque et Iudas Aristobulo magistro Ptolemaei regis, qui est de genere christorum sacerdotum, et his, qui in Aegypto sunt, Iudaeis salutem et sanitatem. 
\verse De magnis periculis a Deo liberati magnifice gratias agimus ipsi, utpote qui adversus regem dimicavimus; 
\verse ipse enim effervere fecit eos, qui pugnaverunt contra sanctam civitatem. 
\verse Nam cum in Perside esset dux ipse et qui cum ipso videbatur esse intolerabilis exercitus, concisi sunt in templo Naneae, fraude utentibus sacerdotibus Naneae. 
\verse Etenim quasi cum ea habitaturus venit ad locum Antiochus et, qui cum ipso erant, amici, ut acciperet pecunias multas dotis nomine. 
\verse Cumque proposuissent eas sacerdotes Naneae, et ipse cum paucis ingressus esset intra ambitum fani, clauserunt templum; cum intrasset Antiochus, 
\verse aperto occulto aditu laquearis, mittentes lapides percusserunt ducem et diviserunt membratim et, capitibus amputatis, foras proiecerunt. 
\verse Per omnia benedictus Deus, qui tradidi 
\verse Facturi igitur quinta et vicesima die mensis Casleu purificationem templi, necessarium duximus significare vobis, ut et vos quoque agatis diem Scenopegiae et ignis, qui datus est, quando Nehemias, aedificato templo et altari, obtulit sacrificia. 
\verse Nam cum in Persidem ducerentur patres nostri, sacerdotes, qui tunc cultores Dei erant, acceptum ignem de altari occulte absconderunt in cavo putei situm habentis siccum, in quo contutati sunt eum, ita ut omnibus ignotus esset locus. 
\verse Cum autem praeterissent anni multi, et placuit Deo, ut mitteretur Nehemias a rege Persidis, nepotes sacerdotum illorum, qui absconderant, misit ad ignem. 
\verse Sicut narraverunt nobis, non invenerunt ignem sed aquam crassam. Et iussit eos haurire et afferre. Utque imposita sunt sacrificia, iussit sacerdotes Nehemias aspergere aqua et ligna et, quae erant superposita. 
\verse Utque hoc factum est, et tempus transiit, et sol refulsit, qui prius erat in nubilo, accensus est ignis magnus, ita ut omnes mirarentur. 
\verse Orationem autem faciebant sacerdotes, dum consummaretur sacrificium: et sacerdotes et omnes, Ionatha inchoante, ceteris autem respondentibus ut Nehemias.
 \verse Erat autem oratio hunc habens modum: “ Domine, Domine Deus, omnium creator, terribilis et fortis, iustus et misericors, qui solus es rex et bonus, 
\verse solus praestans, solus iustus et omnipotens et aeternus; qui liberas Israel de omni malo, qui fecisti patres electos et sanctificasti eos, 
\verse accipe sacrificium pro universo populo tuo Israel et custodi partem tuam et sanctifica. 
 \verse Congrega dispersionem nostram, libera eos, qui serviunt gentibus, et contemptos et abominatos respice, ut sciant gentes quia tu es Deus noster. 
 \verse Afflige opprimentes nos et contumeliam facientes in superbia. 
\verse Constitue populum tuum in loco sancto tuo, sicut dixit Moyses ”. 
\verse Sacerdotes autem psallebant hymnos. 
\verse Cum autem consumptum esset sacrificium, ex residua aqua Nehemias iussit lapides maiores perfundi 
\verse quod ut factum est, flamma accensa est, sed a lumine, quod refulsit ex altari, consumpta est. 
\verse Ut vero manifestata est res, et renuntiatum est regi Persarum quod in loco, in quo ignem absconderant hi, qui translati fuerant, sacerdotes, aqua apparuit, de qua Nehemias et, qui cum eo erant, purificaverunt ea, quae essent sacrificii, 
\verse circumsaepiens autem rex et rem diligenter examinans, templum fecit. 
\verse Et quibus gratificabatur rex, multa dona accipiebat et tribuebat. 
\verse Appellaverunt autem, qui cum Nehemia erant, hunc locum Nephthar, quod interpretatur Purificatio; vocatur autem apud plures Nephthai.
 
\begin{biblechapter}
\verse Invenitur autem in descriptionibus quod Ieremias propheta iussit eos ignem accipere, qui transmigrabant, ut significatum est, 
\verse et ut mandavit propheta transmigratis dans illis legem, ne obliviscerentur praecepta Domini et ut non exerrarent mentibus videntes simulacra aurea et argentea et ornamenta eorum. 
\verse Et alia huiusmodi dicens hortabatur, ne legem amoverent a corde suo. 
\verse Erat autem in ipsa scriptura quomodo tabernaculum et arcam iussit propheta, divino responso ad se facto, comitari secum, usquequo exiit in montem, in quo Moyses ascendit et vidit Dei hereditatem. 
\verse Et veniens Ieremias invenit domum speluncae; et tabernaculum et arcam et altare incensi intulit illuc et ostium obstruxit. 
\verse Et accesserunt quidam ex his, qui simul sequebantur, ut notarent viam, et non potuerunt invenire. 
\verse Ut autem cognovit Ieremias, culpans illos dixit quod ignotus erit locus, donec congreget Deus congregationem populi et misericordia fiat; 
\verse et tunc Dominus ostendet haec, et apparebit maiestas Domini, et nubes erit, sicut et sub Moyse manifestabatur, sicut et Salomon petiit, ut locus sanctificaretur magnifice. 
 \verse Manifestabatur autem et ut sapientiam habens obtulit sacrificium dedicationis et consummationis templi. 
\verse Sicut et Moyses orabat ad Dominum, et descendit ignis de caelo et consumpsit sacrificia, sic et Salomon oravit, et descendit ignis de caelo et consumpsit holocausta. 
\verse Et dixit Moyses: “ Eo quod non sit comestum, quod erat pro peccato, consumptum est ”. 
\verse Similiter et Salomon octo dies celebravit.
 \verse Inferebantur autem in descriptionibus et commentariis secundum Nehemiam haec eadem, et ut construens bibliothecam congregavit libros de regibus et prophetis et libros David et epistulas regum de donariis. 
\verse Similiter autem et Iudas ea, quae deciderant per bellum, quod nobis acciderat, congregavit omnia, et sunt apud nos. 
\verse Si ergo desideratis haec, mittite, qui perferant vobis.
 \verse Acturi itaque purificationem, scripsimus vobis; bene ergo facietis, si egeritis hos dies. 
\verse Deus autem, qui liberavit universum populum suum et reddidit hereditatem omnibus et regnum et sacerdotium et sanctificationem, 
 \verse sicut promisit in lege. Speramus enim in Deo quod cito nostri miserebitur et congregabit de sub caelo in locum sanctum; eripuit enim nos de magnis periculis et locum purgavit.
 \verse De Iuda vero Maccabaeo et fratribus eius et de templi magni purificatione et de arae dedicatione, 
\verse sed et de proeliis, quae pertinent ad Antiochum Epiphanem et filium eius Eupatorem, 
\verse et de illuminationibus, quae de caelo factae sunt ad eos, qui generose pro Iudaismo fortiter fecerunt, ita ut universam regionem, cum pauci essent, vindicarent et barbaram multitudinem fugarent 
\verse et famosissimum in toto orbe templum recuperarent et civitatem liberarent et leges, quae futurum erat ut abolerentur, restituerentur, Domino cum omni clementia propitio facto illis, 
\verse quae omnia ab Iasone Cyrenaeo quinque libris declarata sunt, tentavimus nos uno volumine breviare. 
\verse Considerantes enim multitudinem numerorum et difficultatem, quae adest volentibus aggredi narrationes historiarum propter multitudinem rerum, 
\verse curavimus volentibus quidem legere, ut esset animi oblectatio, studiosis vero, ut facilius possint memoriae commendare, omnibus autem legentibus utilitas conferatur. 
\verse Et nobis quidem ipsis, qui hoc opus breviandi causa suscepimus, non facilem laborem, immo vero negotium plenum vigiliarum et sudoris assumpsimus. 
\verse Sicut praeparanti convivium et quaerenti aliorum utilitatem non facile est, tamen propter multorum gratiam libenter laborem sustinebimus, 
 \verse accurate quidem de singulis elaborare auctori concedentes, ipsi autem persequi datam formam brevitati studentes. 
\verse Sicut enim novae domus architecto de universa structura curandum est, ei vero, qui inurere et pingere curat, quae apta sunt ad ornatum exquirenda sunt, ita aestimo et in nobis. 
 \verse Inire quidem et deambulacrum facere verborum et curiosius partes singulas quasque disquirere historiae congruit auctori; 
\verse brevitatem vero dictionis sectari et exsecutionem rerum vitare brevianti concedendum est. 
\verse Hinc ergo narrationem incipiemus, praedictis tantulo subiuncto; stultum etenim est ante historiam efffluere, ipsam autem historiam concidere.
 
\begin{biblechapter}
\verse Cum sancta civitas habitaretur cum omni pace, et leges quam optime custodirentur propter Oniae pontificis pietatem et odium malitiae, 
\verse fiebat ut et ipsi reges locum honorarent et templum maximis muneribus illustrarent, 
 \verse ita ut Seleucus quoque Asiae rex de redditibus suis praestaret omnes sumptus ad ministeria sacrificiorum pertinentes. 
\verse Simon autem de tribu Belgae praepositus templi constitutus dissentiebat a principe sacerdotum de dispensatione in civitate. 
\verse Et cum vincere Oniam non posset, venit ad Apollonium Tharseae filium, qui eo tempore erat dux Coelesyriae et Phoenicis, 
 \verse et nuntiavit pecuniis inenarrabilibus plenum esse aerarium Hierosolymis, ita ut multitudo vectigalium innumerabilis esset et ea non pertinere ad rationem sacrificiorum; esse autem possibile sub potestate regis haec cadere.
 \verse Collocutus autem Apollonius cum rege, de indicatis sibi pecuniis aperuit; at ille vocans Heliodorum, qui erat super negotia, misit datis mandatis, ut praedictam pecuniam transportaret. 
\verse Statimque Heliodorus iter est aggressus, specie quidem quasi per Coelesyriam et Phoenicen civitates esset peragraturus, re vera autem regis propositum perfecturus. 
\verse Sed cum venisset Hierosolymam et benigne a summo sacerdote civitatis esset exceptus, narravit de dato indicio, et cuius rei gratia adesset aperuit; interrogabat autem, si vere haec ita essent. 
\verse Tunc summus sacerdos ostendit deposita esse viduarum et pupillorum; 
 \verse quaedam vero esse Hircani Thobiae, viri valde eminentis, non sicut detulerat obtrectans impius Simon; universa autem argenti talenta esse quadringenta et auri ducenta; 
\verse decipi vero eos, qui credidissent loci sanctitati et honorati per universum mundum templi venerationi inviolabili tutelae, omnino impossibile esse.
 \verse At ille, pro his, quae habebat, mandatis a rege, omnino dicebat in regium fiscum ea esse deferenda.
 \verse Constituta autem die, intrabat de his visitationem ordinaturus. Non modica vero per universam civitatem erat trepidatio. 
\verse Sacerdotes autem ante altare cum stolis sacerdotalibus iactaverunt se et invocabant in caelum eum, qui de deposito legem posuit, ut his, qui deposuerant, ea salva custodiret. 
\verse Erat autem, ut qui videret summi sacerdotis vultum, mente vulneraretur; facies enim et color immutatus declarabat internum animi dolorem. 
\verse Circumfusus enim erat metus quidam viro, et horror corporis, unde manifestus aspicientibus dolor instans cordi efficiebatur. 
\verse Alii autem de domibus gregatim prosiliebant ad publicam supplicationem, pro eo quod in contemptum locus esset venturus. 
\verse Accinctaeque mulieres ciliciis sub mammis per vias confluebant; sed et virgines, quae conclusae erant, aliae quidem procurrebant ad ianuas, aliae autem ad muros, quaedam vero per fenestras aspiciebant; 
\verse universae autem protendentes manus in caelum deprecabantur. 
\verse Erat enim misereri commixtae multitudinis prostrationem et summi sacerdotis in magna agonia constituti exspectationem. 
 \verse Et hi quidem invocabant omnipotentem Dominum, ut credita salva his, qui crediderant, conservaret cum omni tutela.
 \verse Heliodorus autem, quod fuerat decretum, perficiebat. 
\verse Eodem loco, ipso cum satellitibus circa aerarium praesente, spirituum et omnis potestatis Dominus magnam fecit ostensionem, ita ut omnes, qui ausi fuerant convenire, perterriti virtute Dei in dissolutionem et formidinem converterentur. 
\verse Apparuit enim illis quidam equus terribilem habens sessorem et optimo operimento adornatus; isque cum impetu invectus Heliodoro priores calces impegit; qui autem supersedebat, videbatur arma habere aurea. 
\verse Alii etiam apparuerunt duo iuvenes virtute decori, optimi gloria speciosique amictu, qui etiam circumsteterunt eum et ex utraque parte flagellabant sine intermissione multas inferentes ei plagas. 
\verse Subito autem concidit in terram; eumque multa caligine circumfusum rapuerunt atque in sellam gestatoriam imposuerunt; 
\verse et eum, qui cum multis cursoribus et satellitibus praedictum ingressus erat aerarium, portabant carentem auxilio ex armis constitutum, manifeste Dei virtutem cognoscentem. 
 \verse Et ille quidem per divinam virtutem iacebat mutus atque omni spe et salute privatus; 
\verse hi autem Dominum benedicebant, qui magnificabat locum suum; et templum, quod paulo ante timore ac tumultu erat plenum, apparente omnipotente Domino, gaudio et laetitia impletum est. 
\verse Confestim vero ex amicis Heliodori quidam rogabant Oniam, ut invocaret Altissimum, ut vitam donaret ei, qui prorsus in supremo spiritu erat constitutus. 
\verse Suspectus autem factus summus sacerdos, ne forte rex opinaretur malitiam aliquam ex Iudaeis circa Heliodorum consummatam, obtulit hostiam pro salute viri. 
\verse Cumque summus sacerdos litationem perficeret, iidem iuvenes rursus apparuerunt Heliodoro eisdem vestibus amicti et astantes dixerunt: “ Oniae summo sacerdoti multas gratias age, nam propter eum Dominus tibi vitam donavit; 
\verse tu autem a caelo flagellatus nuntia omnibus magnam Dei potestatem ”. Et his dictis, non comparuerunt.
 \verse Heliodorus autem, hostia Domino oblata et votis magnis promissis ei, qui vivere concessit, et Oniam acceptum habens cum exercitu repedavit ad regem; 
 \verse testabatur autem omnibus ea, quae sub oculis suis viderat, opera maximi Dei. 
 \verse Cum autem rex interrogasset Heliodorum, quis esset aptus adhuc semel Hierosolymam mitti, ait: 
\verse “ Si quem habes hostem aut rerum insidiatorem, mitte eum illuc et flagellatum eum recipies, si tamen evaserit, eo quod in loco sit vere Dei quaedam virtus; 
\verse nam ipse, qui habet in caelis habitationem, visitator et adiutor est loci illius et venientes ad malefaciendum percutit ac perdit ”. 
\verse Igitur de Heliodoro et aerarii custodia ita res processerunt.
 
\begin{biblechapter}
\verse Simon autem praedictus, qui pecuniarum et patriae delator exstitit, male loquebatur de Onia, tamquam ipse Heliodorum instigasset et malorum auctor fuisset; 
\verse benefactoremque civitatis et curatorem gentis suae et aemulatorem legum audebat insidiatorem rerum dicere. 
\verse Sed cum inimicitia in tantum procederet, ut etiam per quendam eorum, qui a Simone probati essent, homicidia fierent, 
\verse considerans Onias periculum contentionis et Apollonium Menesthei, ducem Coelesyriae et Phoenicis, augentem malitiam Simonis, 
\verse ad regem se contulit, non ut civium accusator, sed quod utile esset in commune et singulariter universae multitudinis prospiciens. 
\verse Videbat enim sine regali providentia impossibile esse pacem adhuc rebus obtingere, nec Simonem cessaturum a stultitia.
 \verse Sed post Seleuci vitae excessum, cum suscepisset regnum Antiochus, qui Epiphanes appellabatur, ambiebat Iason frater Oniae summum sacerdotium, 
\verse promittens regi per interpellationem argenti talenta trecenta sexaginta et ex reditu quodam alio talenta octoginta; 
\verse super haec autem promittebat et alia centum quinquaginta se perscripturum, si concederetur per potestatem eius gymnasium et ephebiam sibi constituere et eos, qui in Hierosolymis erant, Antiochenos scribere. 
\verse Quod cum rex annuisset, et obtinuisset principatum, statim ad Graecam consuetudinem contribules suos transferre coepit. 
\verse Et, amotis his, quae humanitatis causa Iudaeis a regibus fuerant constituta per Ioannem patrem Eupolemi, qui apud Romanos de amicitia et societate functus est legatione, et legitima civium iura destituens, pravos mores innovabat. 
\verse Prompte enim sub ipsa arce gymnasium constituit et optimos quosque epheborum subigens sub petasum ducebat. 
\verse Erat autem sic culmen quoddam Graecae conversationis et profectus alienigenarum moris, propter impii et non summi sacerdotis Iasonis inauditam contaminationem, 
\verse ita ut sacerdotes iam non circa altaris officia dediti essent, sed contempto templo et sacrificiis neglectis, festinarent participes fieri iniquae in palaestra praebitionis post disci provocationem 
\verse et patrios quidem honores nihil habentes, Graecas autem glorias optimas aestimantes. 
\verse Quarum gratia periculosa eos contentio habebat, et quorum instituta aemulabantur ac per omnia consimiles esse cupiebant, hos hostes et ultores habuerunt. 
\verse In leges enim divinas impie agere non est facile, sed haec tempus sequens declarabit.
 \verse Cum autem quinquennalis agon Tyri celebraretur, et rex praesens esset, 
\verse misit Iason facinorosus ab Hierosolymis spectatores Antiochenses portantes argenti drachmas trecentas in sacrificium Herculis; quas etiam postulaverunt hi, qui asportaverant, ne in sacrificium erogarentur, quia non oporteret, sed in alium sumptum eas deputari. 
\verse Sed haec ceciderunt: propter illum quidem, qui miserat, in sacrificium Herculis; propter eos autem, qui afferebant, in fabricam triremium. 
\verse Misso autem in Aegyptum Apollonio Menesthei filio propter ascensum ad solium Philometoris regis, cum cognovisset Antiochus alienum se ab illius negotiis effectum, propriae securitati consuluit; inde cum Ioppen venisset, se contulit Hierosolymam. 
\verse Et magnifice ab Iasone et civitate susceptus, cum facularum luminibus et acclamationibus introductus est; deinde sic in Phoenicen exercitum convertit.
 \verse Et post triennii tempus misit Iason Menelaum supradicti Simonis fratrem portantem pecunias regi et de negotiis necessariis commonitiones perlaturum. 
 \verse At ille commendatus regi, cum se magnificasset facie potestatis, in semetipsum contulit summum sacerdotium superponens Iasoni talenta argenti trecenta; 
\verse acceptisque regiis mandatis, venit nihil quidem gerens dignum sacerdotio, animos vero crudelis tyranni et ferae barbarae iram habens. 
\verse Et Iason quidem, qui proprium fratrem circumvenerat, ipse circumventus ab alio profugus in Ammanitem expulsus est regionem. 
\verse Menelaus autem principatum quidem obtinuit; de pecuniis vero regi promissis nihil debite agebat, 
\verse cum vero exactionem faceret Sostratus, qui arci erat praepositus, nam ad hunc exactio vectigalium pertinebat. Quam ob causam utrique a rege sunt advocati; 
 \verse et Menelaus quidem reliquit summi sacerdotii successorem Lysimachum fratrem suum, Sostratus autem Cratetem, qui praeerat Cypriis.
 \verse Talibus autem constitutis, contigit Tarsenses et Mallotas seditionem movere, eo quod Antiochidi, regis concubinae, dono essent dati. 
\verse Festinanter itaque rex venit sedare illos, relicto suffecto uno ex iis in dignitate constitutis Andronico. 
\verse Ratus autem Menelaus accepisse se tempus opportunum, aurea quaedam vasa e templo furatus donavit Andronico; et alia vendiderat Tyri et per vicinas civitates. 
\verse Quod cum certissime cognovisset Onias, arguebat eum, ipse in loco tuto se continens in Daphne secus Antiochiam. 
 \verse Unde Menelaus seorsum apprehendens Andronicum rogabat, ut Oniam interficeret. At vero ille, cum venisset ad Oniam et cum fidem dolo dedisset ac dexteram accepisset dedissetque cum iureiurando, quamvis esset ei suspectus, suasit de asylo procedere, quem statim peremit, non veritus iustitiam. 
\verse Ob quam causam non solum Iudaei, sed multi quoque ex aliis nationibus indignabantur et moleste ferebant de nece viri iniusta. 
\verse Sed regressum regem de Ciliciae locis interpellabant, qui erant per civitatem Iudaei, simul et Graecis scelus conquerentibus, de eo quod sine ratione Onias interfectus esset. 
\verse Contristatus itaque animo Antiochus et flexus ad misericordiam lacrimas fudit, propter defuncti sobrietatem et multam modestiam; 
\verse accensusque animis, confestim ablata Andronici purpura ac tunicis eius discissis, circumduxit per totam civitatem usque ad eundem locum, in quo in Oniam impietatem commiserat, atque illic sacrilegum interfectorem e mundo sustulit, Domino illi condignam retribuente poenam.
 \verse Multis autem sacrilegiis per civitatem a Lysimacho commissis Menelai consilio, et divulgata foris fama, congregata est multitudo adversum Lysimachum, vasis aureis iam multis dissipatis. 
\verse Turbis autem insurgentibus et ira repletis, Lysimachus, armatis fere tribus milibus, iniquis manibus coepit, duce quodam Aurano, aetate non minus ac dementia provecto. 
\verse Sed ut intellexerunt conatum Lymachi, alii lapides, alii fustes validos arripuere, quidam vero ex adiacente cinere manu apprehenderunt et mixtim iecerunt in eos, qui circa Lysimachum erant. 
\verse Quam ob causam multos quidem vulneraverunt, quosdam autem et prostraverunt, omnes vero in fugam compulerunt; ipsum vero sacrilegum secus aerarium interfecerunt.
 \verse De his ergo coepit iudicium adversus Menelaum agitari. 
\verse Et cum venisset rex Tyrum, apud ipsum causam egerunt missi tres viri a senatu. 
\verse Et cum iam superaretur Menelaus, promisit Ptolemaeo Dorymenis multas pecunias ad suadendum regi. 
\verse Unde Ptolemaeus, excipiens seorsum in quoddam atrium columnatum quasi refrigerandi gratia regem, deduxit a sententia. 
\verse Et Menelaum quidem universae malitiae reum criminibus absolvit; miseros autem, qui etiam si apud Scythas causam dixissent, innocentes iudicarentur, hos morte damnavit. 
\verse Cito ergo iniustam poenam dederunt, qui pro civitate et populo et sacris vasis causam prosecuti sunt. 
\verse Quam ob rem Tyrii quoque in malefactum indignati, quaeque ad sepulturam eorum necessaria essent, magno sumptu praestiterunt. 
 \verse Menelaus autem propter eorum, qui in potentia erant, avaritiam permanebat in potestate, crescens in malitia magnus civium insidiator constitutus.
 
\begin{biblechapter}
\verse Circa hoc autem tempus Antiochus secundam profectionem paravit in Aegyptum. 
\verse Contigit autem per universam civitatem fere per dies quadraginta videri per aera equites discurrentes, auratas stolas habentes et hastas, ad modum cohortium armatos, et gladiorum evaginationes 
\verse et turmas equorum per ordinem digestas et congressiones fieri et decursus utrorumque et scutorum motus et contorum multitudinem et telorum iactus et aureorum ornamentorum fulgores omnisque generis loricationes. 
\verse Quapropter omnes rogabant pro bono factam esse ostensionem.
 \verse Sed cum falsus rumor exisset, tamquam vita excessisset Antiochus, assumptis Iason non minus mille viris repente aggressus est civitatem; illis autem, qui erant in muro, compulsis in fugam et ad ultimum iam apprehensa civitate, Menelaus fugit in arcem. 
\verse Iason vero caedes civium suorum perpetrabat nulli parcens, non intellegens prosperitatem adversum cognatos calamitatem esse maximam, arbitrans autem hostium et non civium se trophaea constituere. 
\verse Et principatum quidem non obtinuit, finem vero insidiarum suarum confusionem adeptus, profugus iterum abiit in Ammanitidem. 
\verse Ad ultimum igitur malam reversionem sortitus est; conclusus apud Aretam Arabum tyrannum, fugiens de civitate in civitatem, expulsus ab omnibus, odiosus ut refuga legum et exsecrabilis ut patriae et civium carnifex in Aegyptum extrusus est. 
\verse Et, qui multos de patria expulerat, peregre periit ad Lacedaemonios pervectus, quasi pro cognatione habiturus protectionem; 
\verse et, qui insepultos multos abiecerat, ipse illamentatus permansit nec exsequiis ullis neque patrio sepulcro participavit.
 \verse Cum autem nuntia ad regem pervenissent de his, quae gesta erant, suspicatus est rex a societate defecturam Iudaeam; et ob hoc profectus ex Aegypto efferatus animo, civitatem quidem armis cepit 
\verse et iussit militibus interficere occursantes nemini parcendo et eos, qui in domos ascenderent, trucidare. 
\verse Fiebant ergo iuvenum ac seniorum caedes, mulierum et natorum exterminium virginumque et parvulorum neces. 
\verse Erant autem toto triduo octoginta milia perditi, quadraginta quidem milia in ipso manuum conflictu; non minus autem quam qui iugulati fuerant, venumdati sunt. 
\verse Non contentus autem his, ausus est intrare templum universae terrae sanctissimum, ducem habens Menelaum, qui legum et patriae fuit proditor, 
\verse et scelestis manibus sumens sancta vasa et, quae ab aliis regibus et civitatibus erant posita ad augmentum et gloriam loci et honorem, profanis manibus contrectans. 
\verse Ita extollebatur mente Antiochus non considerans quod propter peccata habitantium civitatem modicum Dominus fuerat iratus; propter quod accidit circa locum despectio. 
\verse Alioquin nisi contigisset eos multis peccatis esse involutos, sicut Heliodorus, qui missus est a Seleuco rege ad inspectionem aerarii, et ipse, mox ut accessisset, confestim flagellatus repulsus fuisset ab audacia. 
\verse Verum non propter locum gentem, sed propter gentem locum Dominus elegit. 
\verse Ideoque et ipse locus particeps factus populi malorum, postea factus est socius beneficiorum; et, qui derelictus in ira Omnipotentis est, iterum in magni Domini reconciliatione cum omni gloria restitutus est.
 \verse Igitur Antiochus mille et octingentis ablatis de templo talentis, velocius Antiochiam regressus est, existimans se prae superbia terram ad navigandum, pelagus vero ad ambulandum deducturum propter mentis elationem. 
\verse Reliquit autem et praepositos ad affligendam gentem: Hierosolymis quidem Philippum, genere Phrygem, moribus barbariorem eo ipso, a quo constitutus est; 
\verse in Garizim autem Andronicum; praeter autem hos Menelaum, qui gravius quam ceteri imminebat civibus. 
\verse Misit autem Apollonium Mysarcham cum exercitu — viginti vero et duo milia virorum — praecipiens omnes perfectae aetatis interficere, mulieres autem ac iuniores vendere. 
\verse Qui cum venisset Hierosolymam et pacificum se simulasset, quievit usque ad diem sanctum sabbati et, cum comprehenderet feriatos Iudaeos, arma capere suis praecepit; 
\verse omnesque, qui ad spectaculum processerant, trucidavit et civitatem cum armatis discurrens ingentem multitudinem peremit. 
\verse Iudas autem, qui et Maccabaeus, decimus factus secesserat in eremum et ferarum more in montibus vitam cum suis agebat; et feni cibo vescentes demorabantur, ne participes essent coinquinationis.
 
\begin{biblechapter}
\verse Sed non post multum temporis misit rex senem quendam Atheniensem, qui compelleret Iudaeos, ut se transferrent a patriis legibus et Dei legibus ne uterentur; 
\verse contaminare etiam, quod in Hierosolymis erat, templum et cognominare Iovis Olympii, et in Garizim, prout erant hi, qui locum inhabitabant, Iovis Hospitalis. 
\verse Pessima autem et universis gravis erat malorum incursio. 
\verse Nam templum luxuria et comissationibus gentium erat plenum, scortantium cum meretricibus et in sacratis porticibus mulieribus adhaerentium, insuper et intro inferentium ea, quae non licebat; 
\verse altare etiam plenum erat illicitis, quae legibus prohibebantur. 
\verse Neque autem sabbata custodiebantur, neque dies sollemnes patrii servabantur, nec simpliciter Iudaeum se esse quisquam confitebatur. 
\verse Ducebantur autem cum amara necessitate per singulos menses in die natalis regis ad sacrificium et, cum Liberi sacra celebrarentur, cogebantur hedera coronati pompam Libero celebrare. 
 \verse Decretum autem exiit in proximas Graecorum civitates, suggerente Ptolemaeo, ut pari modo et ipsi adversus Iudaeos agerent, ut sacrificarent; 
\verse eos autem, qui nollent transire ad instituta Graecorum, interficerent; erat ergo videre instantem miseriam. 
\verse Duae enim mulieres delatae sunt natos suos circumcidisse; quas infantibus ad ubera suspensis, cum publice per civitatem circumduxissent, per muros praecipitaverunt. 
\verse Alii vero ad proximas coeuntes speluncas, ut latenter septimam diem celebrarent, cum indicati essent Philippo, flammis succensi sunt, eo quod verebantur propter religionem sibimet auxilium ferre pro claritate sanctissimi diei.
 \verse Obsecro autem eos, qui hunc librum lecturi sunt, ne abhorrescant propter adversos casus, sed reputent illas poenas non ad interitum, sed ad correptionem esse generis nostri. 
\verse Etenim multo tempore non sinere eos, qui gerunt impie, sed statim ultiones adhibere, magni beneficii est indicium. 
\verse Non enim, sicut et in aliis nationibus, Dominus patienter ferens exspectat, ut eas, cum pervenerint in plenitudinem peccatorum, puniat, ita et in nobis statuit esse, 
\verse ne, peccatis nostris in finem devolutis, demum in nos vindicet; 
 \verse propter quod numquam quidem a nobis misericordiam suam amovet, corripiens vero per aerumnas populum suum non derelinquit. 
\verse Sed haec nobis ad commonitionem dicta sint; paucis autem veniendum est ad narrationem.
 \verse Eleazarus quidam, unus de primoribus scribarum, vir iam aetate provectus et aspectu faciei decorus, aperto ore compellebatur carnem porcinam manducare. 
 \verse At ille magis cum illustri fama mortem quam cum exsecratione vitam complectens, voluntarie praeibat ad supplicium, 
\verse exspuens autem, quemadmodum oportet accedere eos, qui sustinent non admittere illa, quae non est fas gustare, propter nimium vivendi amorem. 
\verse Hi autem, qui iniquo sacrificio praepositi erant, propter antiquam cum viro amicitiam tollentes eum secreto rogabant, ut afferret carnes, quibus uti ei liceret quaeque ab ipso paratae essent, et fingeret se eas manducare, quas rex imperaverat de sacrificii carnibus, 
\verse ut hoc facto a morte liberaretur et propter veterem cum illis amicitiam consequeretur humanitatem. 
\verse At ille, consilio decoro inito ac digno aetate et senectutis eminentia et acquisita nobilique canitie atque optima a puero vitae disciplina, magis autem sancta et a Deo condita legislatione, consequenter sententiam ostendit: cito, dicens, dimitterent ad inferos. 
\verse “ Non enim aetati nostrae dignum est fingere, ut multi adulescentium arbitrantes Eleazarum nonaginta annorum transisse ad morem alienigenarum 
\verse et ipsi propter meam simulationem et propter modicum et pusillum vitae tempus decipiantur propter me, et exsecrationem atque maculam meae senectuti conquiram. 
 \verse Nam etsi in praesenti tempore evasero eam, quae ex hominibus est, poenam, manus tamen Omnipotentis nec vivus nec defunctus effugiam. 
\verse Quam ob rem viriliter nunc vita excedendo, senectute quidem dignus apparebo; 
\verse adulescentibus autem exemplum forte reliquero, ut prompto animo ac fortiter pro sacris ac sanctis legibus honesta morte perfungantur ”. Et cum haec dixisset, confestim ad supplicium venit; 
\verse ipsis autem, qui eum ducebant, illam, quam paulo ante habuerant erga eum benevolentiam, in iram convertentibus, propterea quod sermones dicti, sicut ipsi arbitrabantur, essent amentia. 
\verse Cumque coepisset plagis mori, ingemiscens dixit: “ Domino, qui habet sanctam scientiam, manifestum est quia cum a morte possem liberari, duros secundum corpus sustineo dolores flagellatus, secundum animam vero propter ipsius timorem libenter haec patior ”. 
\verse Et iste quidem hoc modo vita decessit, non solum iuvenibus, sed et plurimis ex gente mortem suam ad exemplum fortitudinis et memoriam virtutis relinquens.
 
\begin{biblechapter}
\verse Contigit autem et septem fratres una cum matre apprehensos compelli a rege attingere contra fas carnes porcinas, flagris et nervis cruciatos. 
\verse Unus autem ex illis exstans prior locutor sic ait: “ Quid es quaesiturus, et quid vis discere a nobis? Parati sumus mori magis quam patrias leges praevaricari ”. 
\verse Iratus itaque rex iussit sartagines et ollas succendi. 
\verse Quibus statim succensis, iussit ei, qui prior illorum fuerat locutus, amputari linguam et, cute capitis abstracta, summas quoque manus et pedes ei praescindi, ceteris eius fratribus et matre inspicientibus. 
\verse Et cum iam per omnia inutilis factus esset, iussit eum igne admoveri adhuc spirantem et torreri in sartagine. Cum autem vapor sartaginis diu diffunderetur, ceteri una cum matre invicem se hortabantur mori fortiter ita dicentes: 
\verse “ Dominus Deus aspicit et veritate in nobis consolatur, quemadmodum per personam contestantis cantici declaravit Moyses: “Et in servis suis consolabitur” ”. 
\verse Mortuo itaque illo primo hoc modo, sequentem deducebant ad illudendum; et cute capitis eius cum capillis abstracta, interrogabant, si manducaret prius quam toto corpore per membra singula puniretur. 
\verse At ille respondens patria voce dixit: “ Non faciam ”. Propter quod et iste, sequenti loco, tormenta suscepit sicut primus. 
\verse Et in ultimo spiritu constitutus, sic ait: “ Tu quidem, scelestissime, de praesenti vita nos perdis; sed rex mundi defunctos nos pro suis legibus in aeternam vitae resurrectionem suscitabit ”. 
\verse Post hunc tertius illudebatur; et linguam postulatus cito protulit et manus constanter extendit 
\verse et fortiter ait: “ E caelo ista possideo et propter illius leges haec ipsa despicio et ab ipso rursus me ea recepturum spero ”, 
\verse ita ut rex et, qui cum ipso erant, mirarentur adulescentis animum, quomodo pro nihilo duceret cruciatus. 
\verse Et hoc ita defuncto, quartum vexabant similiter torquentes; 
\verse et, cum iam esset ad mortem, sic ait: “ Potius est ab hominibus morti datos spem exspectare a Deo, iterum ab ipso resuscitandos; tibi enim resurrectio ad vitam non erit ”. 
 \verse Et deinceps quintum, cum admovissent, vexabant; 
\verse at ille respiciens in eum dixit: “ Potestatem inter homines habens, cum sis corruptibilis, facis, quod vis; noli autem putare genus nostrum a Deo esse derelictum; 
\verse tu autem patienter sustine et videbis maiestatem virtutis ipsius, qualiter te et semen tuum torquebit ”. 
\verse Post hunc ducebant sextum, et is mori incipiens ait: “ Noli frustra errare; nos enim propter nosmetipsos haec patimur peccantes in Deum nostrum, et digna admiratione facta sunt in nobis: 
\verse tu autem ne existimes tibi impune futurum, quod contra Deum pugnare tentaveris ”.
 \verse Supra modum autem mater mirabilis et bona memoria digna, quae pereuntes septem filios sub unius diei tempore conspiciens bono animo ferebat propter spem, quam in Dominum habebat. 
\verse Singulos illorum hortabatur voce patria, forti repleta sensu et femineam cogitationem masculino excitans animo, dicens ad eos: 
\verse “ Nescio qualiter in utero meo apparuistis neque ego spiritum et vitam donavi vobis et singulorum vestrorum compagem non sum ego modulata; 
\verse sed enim mundi creator, qui formavit hominis nativitatem quique omnium invenit originem, et spiritum et vitam vobis iterum cum misericordia reddet, sicut nunc vosmetipsos despicitis propter leges eius ”. 
\verse Antiochus autem contemni se arbitratus, simul et exprobrantem dedignans vocem, cum adhuc adulescentior superesset, non solum verbis hortabatur, sed et cum iuramento affirmabat se divitem simul et beatum facturum, translatum a patriis legibus, et amicum habiturum et officia ei crediturum. 
\verse Sed ad haec cum adulescens nequaquam intenderet, vocavit rex matrem et suadebat ei, ut adulescenti fieret suasor in salutem. 
\verse Cum autem multis eam verbis esset hortatus, promisit suasuram se filio. 
\verse Itaque inclinata ad illum, irridens crudelem tyrannum sic ait patria voce: “ Fili, miserere mei, quae te in utero novem mensibus portavi et lac triennio dedi et alui et in aetatem istam perduxi et nutricem me tibi exhibui. 
\verse Peto, nate, ut aspicias ad caelum et terram et quae in ipsis sunt, universa videns intellegas quia non ex his, quae erant, fecit illa Deus; et hominum genus ita fit. 
\verse Ne timeas carnificem istum, sed dignus fratribus tuis effectus suscipe mortem, ut in illa miseratione cum fratribus tuis te recipiam ”. 
\verse Cum haec illa adhuc diceret, ait adulescens: “ Quem sustinetis? Non oboedio praecepto regis, sed obtempero praecepto legis, quae data est patribus nostris per Moysen. 
\verse Tu vero, qui inventor omnis malitiae factus es in Hebraeos, non effugies manus Dei. 
\verse Nos enim pro peccatis nostris haec patimur; 
\verse et si nobis propter increpationem et correptionem ille vivens Dominus noster modicum iratus est, sed iterum reconciliabitur servis suis. 
\verse Tu autem, o sceleste et omnium hominum flagitiosissime, noli frustra extolli elatus vanis spebus, in filios caeli levata manu; 
\verse nondum enim omnipotentis atque intuitoris Dei iudicium effugisti. 
\verse Nam fratres nostri, modico nunc dolore sustentato, sub Dei testamentum aeternae vitae reciderunt; tu vero iudicio Dei iustas superbiae tuae poenas exsolves. 
\verse Ego autem, sicut et fratres mei, et corpus et animam trado pro patriis legibus, invocans Deum maturius genti nostrae propitium fieri, teque cum tormentis et verberibus confiteri quod ipse est Deus solus; 
\verse in me vero et in fratribus meis restitit Omnipotentis ira, quae super omne genus nostrum iuste superducta est ”. 
\verse Tunc rex accensus ira in hunc super omnes crudelius desaevit, indigne ferens se derisum. 
\verse Et hic itaque mundus obiit per omnia in Domino confidens. 
\verse Novissima autem post filios et mater consumpta est.
 \verse Igitur de sacrificiis et de nimiis crudelitatibus satis sit dictum.
 
\begin{biblechapter}
\verse Iudas vero Maccabaeus et, qui cum illo erant, introeuntes latenter in castella convocabant cognatos; et eos, qui permanserunt in Iudaismo, assumentes, collegerunt circiter sex milia virorum. 
\verse Et invocabant Dominum, ut respiceret in populum, qui ab omnibus calcabatur; et misereretur templo, quod contaminabatur ab impiis; 
\verse et misereretur etiam pereunti civitati et incipienti solo complanari et vocem sanguinis ad se clamantis exaudiret; 
\verse memoraretur quoque iniquas mortes parvulorum innocentum et blasphemias nomini suo illatas et indignaretur super his. 
\verse At Maccabaeus, congregata multitudine, intolerabilis iam gentibus efficiebatur, ira Domini in misericordiam conversa. 
\verse Et civitates et castella superveniens improvisus succendebat et opportuna loca occupans non paucos hostium in fugam convertens, 
 \verse maxime noctes in huiusmodi excursus cooperantes captabat. Et fama virtutis eius ubique diffundebatur.
 \verse Videns autem Philippus paulatim virum ad profectum venire ac frequentius in prosperitatibus procedere, ad Ptolemaeum ducem Coelesyriae et Phoenicis scripsit, ut auxilium ferret regis negotiis. 
\verse At ille velociter sumpsit Nicanorem Patrocli de primoribus amicis et misit, datis ei de permixtis gentibus armatis non minus viginti milibus, ut universum Iudaeorum genus deleret; adiunxit autem ei et Gorgiam virum militarem et in bellicis rebus expertum. 
 \verse Constituit autem Nicanor, ut regi tributum, quod Romanis erat dandum, duo milia talentorum de captivitate Iudaeorum suppleret; 
\verse statimque ad maritimas civitates misit convocans ad coemptionem Iudaicorum mancipiorum, promittens se nonaginta mancipia talento distracturum, non exspectans vindictam, quae eum ab Omnipotente esset consecutura. 
\verse Iudas autem, ubi comperit de Nicanoris adventu, indicavit his, qui secum erant, exercitus praesentiam. 
\verse Ex quibus quidam formidantes et non credentes Dei iustitiae in fugam vertebantur et in alios locos seipsos transferebant; 
\verse alii vero omnia, quae eis supererant, vendebant simulque Dominum deprecabantur, ut eriperet eos, qui ab impio Nicanore, prius quam comminus venirent, venumdati essent: 
\verse et si non propter eos, sed tamen propter testamenta ad patres eorum et propter invocationem sancti et magnifici nominis eius super ipsos. 
\verse Convocatis autem Maccabaeus sex milibus, qui cum ipso erant, rogabat ne ab hostibus perterrerentur neque metuerent inique venientium adversum se gentium multitudinem, sed fortiter contenderent, 
\verse ante oculos habentes contumeliam, quae in locum sanctum ab his iniuste esset consummata, itemque et ludibrio habitae civitatis iniuriam, adhuc etiam veterum instituta convulsa. 
 \verse “ Nam illi quidem armis confidunt, ait, simul et audacia; nos autem in omnipotente Deo, qui potest et venientes adversum nos et universum mundum uno nutu delere, confidimus ”. 
\verse Cum autem admonuisset eos et de auxiliis, quae facta sunt erga parentes, et de illo sub Sennacherib, ut centum octoginta quinque milia perierunt, 
\verse et de illo in Babilonia, in proelio quod eis adversus Galatas fuit, ut omnes ad rem venerunt, octo milia cum quattuor milibus Macedonum — Macedonibus haesitantibus, ipsi octo milia peremerunt centum viginti milia propter auxilium illis datum de caelo et beneficia plurima consecuti sunt C; 
\verse quibus verbis cum eos constantes effecisset et paratos pro legibus et patria mori, in quattuor quasdam partes exercitum divisit. 
\verse Constitutis itaque fratribus suis ducibus uniuscuiusque ordinis, Simone et Iosepho et Ionatha, subiectis unicuique millenis et quingentenis, 
\verse insuper et Eleazaro, lecto sancto libro et dato signo adiutorii Dei, primae cohortis ipse ductor commisit cum Nicanore. 
\verse Et facto sibi adiutore Omnipotente, interfecerunt super novem milia hostium, saucios autem et membris debilitatos maiorem partem exercitus Nicanoris reddiderunt, omnes vero fugere compulerunt. 
 \verse Pecunias autem eorum, qui ad emptionem illorum advenerant, abstulerunt et, cum persecuti eos fuissent satis longe, reversi sunt hora conclusi; 
\verse nam erat ante sabbatum, quam ob causam non perseveraverunt insequentes eos. 
\verse Cum autem ipsorum arma collegissent spoliisque hostes exuissent, circa sabbatum versabantur impensius benedicentes et confitentes Domino, qui liberavit eos in isto die misericordiae initium constituens in eos. 
\verse Post sabbatum vero debilitatis et viduis et orphanis portione de spoliis data, residua ipsi cum pueris partiti sunt. 
\verse His itaque gestis et communi facta obsecratione, misericordem Dominum postulabant, ut in finem servis suis reconciliaretur.
 \verse Et contendentes cum his, qui cum Timotheo et Bacchide erant, super viginti milia eorum interfecerunt et munitiones excelsas facile obtinuerunt; et plures praedas diviserunt, aequaliter seipsos participes cum debilitatis et orphanis et viduis, sed et senioribus facientes. 
\verse Et cum arma eorum diligenter collegissent, omnia composuerunt in locis opportunis; residua vero spolia Hierosolymam detulerunt. 
\verse Et phylarchen eorum, qui cum Timotheo erant, interfecerunt, virum scelestissimum, qui in multis Iudaeos afflixerat; 
\verse et cum epinicia agerent in patria, eos, qui sacras ianuas incenderant, et Callisthenem succenderunt, qui in quoddam domicilium fugerat; et dignam pro impietate mercedem tulit. 
\verse Facinorosissimus autem Nicanor, qui mille negotiantes ad Iudaeorum venditionem adduxerat, 
\verse humiliatus ab his, qui secundum ipsum exsistimabantur exigui esse, auxilio Domini, deposita veste gloriae, per mediterranea fugitivi more solitarius effectus venit Antiochiam, super omnia prosperatus in interitu exercitus. 
\verse Et, qui Romanis promiserat se tributum de captivitate Hierosolymorum redigere, praedicabat propugnatorem habere Iudaeos, et hoc modo invulnerabiles esse, eo quod sequerentur leges ab ipso constitutas.
 
\begin{biblechapter}
\verse Eodem autem tempore Antiochus inhoneste revertebatur de regionibus circa Persidem. 
\verse Intraverat enim in eam, quae dicitur Persepolis, et tentavit exspoliare templum et civitatem opprimere; quapropter, multitudine ad armorum auxilium concurrente, in fugam versi sunt; et contigit ut Antiochus in fugam versus ab indigenis turpiter rediret. 
\verse Et cum esset circa Ecbatana, nuntiata sunt ea, quae erga Nicanorem et Timotheum gesta sunt. 
\verse Elatus autem ira arbitrabatur se etiam iniuriam illorum, qui se fugaverant, in Iudaeos retorquere; ideoque iussit, ut auriga sine intermissione iter perficeret, caelesti iam eum comitante iudicio. Ita enim superbe locutus erat: “ Congeriem sepulcri Iudaeorum Hierosolymam faciam, cum venero illo ”.
 \verse Sed qui universa conspicit, Dominus, Deus Israel, percussit eum insanabili et invisibili plaga; et continuo ut is finivit sermonem, apprehendit eum dolor dirus viscerum et amara internorum tormenta, 
\verse perquam iuste, quippe qui multis et novis cruciatibus aliorum torserat viscera. 
\verse Ille vero nullo modo ab arrogantia cessabat; super hoc autem superbia repletus erat, ignem spirans animo in Iudaeos et praecipiens iter accelerari. Contigit autem, ut et ille caderet de curru, qui ferebatur impetu, et gravi lapsu corruens in omnibus corporis membris vexaretur. 
\verse Isque, qui nuper videbatur fluctibus maris imperare propter super hominem iactantiam et in statera montium altitudines appendere, humiliatus ad terram in gestatorio portabatur manifestam Dei virtutem omnibus ostendens, 
\verse ita ut de oculis impii vermes scaturirent, ac viventis in doloribus et maeroribus carnes eius diffluerent, illiusque odore totus exercitus gravaretur propter putredinem. 
\verse Et qui paulo ante sidera caeli contingere se arbitrabatur, eum nemo poterat propter intolerabilem foetoris gravitatem portare.
 \verse Hinc igitur coepit multum superbiae deponere confractus et ad agnitionem venire divina plaga, per momenta doloribus extensus. 
\verse Et, cum nec ipse foetorem suum ferre posset, ita ait: “ Iustum est subditum esse Deo et mortalem non superbe sentire ”. 
\verse Orabat autem hic scelestus Dominum, ei non amplius miserturum, ita dicens: 
\verse sanctam quidem civitatem, ad quam festinans veniebat, ut eam solo aequalem faceret ac sepulcrum congestorum strueret, liberam ostendere; 
\verse Iudaeos autem, quos decreverat nec sepultura quidem se dignos habiturum, sed avibus devorandos cum parvulis se feris proiecturum, omnes hos aequales Atheniensibus facturum; 
\verse templum vero sanctum, quod prius exspoliaverat, pulcherrimis donis ornaturum et sacra vasa multiplicia cuncta se redditurum, et pertinentes ad sacrificia sumptus de redditibus suis praestaturum; 
\verse super haec autem et Iudaeum se futurum et omnem locum habitabilem perambulaturum praedicantem Dei potestatem.
 \verse Sed omnino non cessantibus doloribus — supervenerat enim in eum iustum Dei iudicium — semetipsum desperans scripsit ad Iudaeos hanc infra rescriptam epistulam modum deprecationis habentem, haec continentem: 
\verse “ Optimis civibus Iudaeis plurimam salutem et bene valere et esse felices, rex et dux Antiochus. 
\verse Si bene valetis et filii vestri, et res vestrae ex sententia sunt vobis, precans refero quidem Deo maximam gratiam, in caelum spem habens; 
\verse ego vero in infirmitate constitutus eram, vestri autem honoris et benevolentiae memineram cum affectione. Reversus de Persidis locis et in infirmitatem incidens molestiam habentem, necessarium duxi pro communi omnium securitate curam habere. 
 \verse Non desperans memetipsum, sed spem multam habens effugiendi infirmitatem, 
 \verse respiciens autem quod et pater meus, quibus temporibus in superiora loca duxit exercitum, ostendit, qui susciperet principatum; 
\verse ut, si quid contrarium accideret aut etiam quid difficile nuntiaretur, scientes hi, qui circa regionem erant, cui esset rerum summa derelicta, non turbarentur. 
\verse Ad haec autem considerans de proximo potentes et vicinos regno temporibus insidiantes et eventum exspectantes, designavi filium Antiochum regem, quem saepe recurrens in superiora regna plurimis vestrum committebam et commendabam; et scripsi ad eum, quae subiecta sunt. 
\verse Oro itaque vos et peto memores beneficiorum publice et privatim, ut unusquisque conservet hanc, quam habetis benevolentiam in me et in filium. 
\verse Confido enim eum modeste et humane, sequentem propositum meum, vobiscum acturum ”.
 \verse Igitur homicida et blasphemus pessima perpessus, ut ipse alios tractaverat, peregre in montibus miserabili obitu vita functus est. 
\verse Transferebat autem corpus Philippus collactaneus eius, qui etiam metuens filium Antiochi ad Ptolemaeum Philometorem in Aegyptum se contulit.
 
\begin{biblechapter}
\verse Maccabaeus autem et, qui cum eo erant, Domino eos praeeunte, templum quidem et civitatem receperunt; 
\verse aras autem, quas alienigenae per plateam exstruxerant, itemque delubra demoliti sunt 
\verse et, purgato templo, aliud altare fecerunt et, succensis lapidibus igneque de his concepto, sacrificia obtulerunt post biennium et incensum et lucernas et panum propositionem fecerunt. 
\verse Quibus autem gestis, rogaverunt Dominum prostrati in ventrem, ne amplius talibus malis inciderent, sed et, si quando peccassent, ut ab ipso cum clementia corriperentur et non blasphemis ac barbaris gentibus traderentur. 
 \verse Qua die autem templum ab alienigenis pollutum fuerat, contigit eadem die purificationem fieri templi vicesima quinta illius mensis, qui est Casleu. 
\verse Et cum laetitia diebus octo egerunt in modum Tabernaculorum, recordantes quod ante modicum temporis diem sollemnem Tabernaculorum in montibus et in speluncis more bestiarum egerant. 
\verse Propter quod thyrsos et ramos virides, adhuc et palmas habentes, hymnos tollebant ei, qui prosperavit mundari locum suum. 
\verse Et decreverunt communi praecepto et decreto universae genti Iudaeorum omnibus annis agere dies istos. 
\verse Res itaque de fine Antiochi, qui appellatus est Epiphanes, ita se habuerunt.
 \verse Nunc autem res de Antiocho Eupatore, qui vero filius erat impii, narrabimus, illa breviantes, quae continent bellorum mala. 
\verse Hic enim, suscepto regno, constituit super negotia regni Lysiam quendam Coelesyriae et Phoenicis ducem primarium. 
\verse Nam Ptolemaeus, qui dicebatur Macron, quod esset iustum conservare praeferens erga Iudaeos propter in eos factam iniquitatem, conabatur, quae ad illos spectabant, pacifice peragere. 
\verse Unde accusatus ab amicis apud Eupatorem et cum frequenter se proditorem esse audiret, eo quod Cyprum creditam sibi a Philometore deseruisset et ad Antiochum Epiphanem transiisset, cumque amplius nobilem potestatem digne ferre non posset, veneno hausto vitam finivit.
 \verse Gorgias autem, cum esset dux locorum, externos milites alebat et frequenter adversus Iudaeos bellum instruebat. 
\verse Atque una cum ipso etiam Idumaei, qui tenebant opportunas munitiones, exercebant Iudaeos et fugatos ab Hierosolymis suscipientes bellum alere tentabant. 
\verse Hi vero, qui erant cum Maccabaeo, supplicatione facta et rogato Deo, ut esset sibi adiutor, impetum fecerunt in munitiones Idumaeorum; 
\verse quas fortiter aggressi, loca obtinuerunt et omnes, qui pugnabant in muris, propulerunt et occurrentes interemerunt et non minus viginti milibus trucidaverunt. 
\verse Quidam autem, cum confugissent non minus quam novem milia in duas turres valde munitas et omnia ad repugnandum habentes, 
 \verse Maccabaeus, ad eorum expugnationem relicto Simone et Iosepho itemque Zacchaeo eisque, qui cum ipso erant satis multis, ipse ad ea, quae amplius perurgebant, loca discessit. 
\verse Hi vero, qui cum Simone erant, cupiditate ducti a quibusdam, qui in turribus erant, suasi sunt pecunia et, septuaginta milibus drachmis acceptis, dimiserunt quosdam effugere. 
\verse Cum autem Maccabaeo nuntiatum esset quod factum est, principibus populi congregatis accusavit quod pecunia fratres vendidissent, adversariis eorum dimissis. 
\verse Hos igitur proditores factos interfecit et confestim duas turres occupavit. 
\verse Armis autem in manibus omnia prospere agendo in duabus munitionibus plus quam viginti milia peremit.
 \verse At Timotheus, qui prius a Iudaeis fuerat superatus, convocatis peregrinis copiis valde multis et congregatis equis, qui erant ex Asia, non paucis, adfuit quasi armis victam Iudaeam capturus. 
\verse Qui autem cum Maccabaeo erant, appropinquante illo, ad supplicationem Dei terra capita aspergentes lumbosque ciliciis praecincti 
\verse super crepidinem contra altare provoluti rogabant, ut sibi propitius factus inimicis eorum esset inimicus et adversariis adversaretur, sicut lex declarat. 
\verse Digressi autem ab oratione, sumptis armis, longius de civitate processerunt et, proximi hostibus effecti, separatim steterunt. 
\verse Cum autem lux oriens coepisset diffundi, utrique commiserunt, isti quidem prosperitatis et victoriae tamquam sponsorem habentes cum virtute refugium in Dominum, illi autem ut ducem certaminum sibi ipsis statuentes animum. 
\verse Sed, cum vehemens pugna esset, apparuerunt adversariis de caelo viri quinque in equis, frenis aureis decori, et ducatum Iudaeis praestantes; 
\verse ex quibus duo Maccabaeum medium accipientes suisque armis protegentes incolumem conservabant, in adversarios autem tela et fulmina iaciebant, ex quo caecitate confusi evolaverunt repleti perturbatione. 
\verse Interfecti sunt autem viginti milia quingenti et equites sescenti.
 \verse Timotheus vero confugit in praesidium, quod Gazara dicitur, optimam munitionem, ducatum illic habente Chaerea. 
\verse Qui autem cum Maccabaeo erant laetantes obsederunt munitionem diebus quattuor. 
\verse At hi qui intus erant, loci munimento confisi, supra modum maledicebant et sermones nefandos iactabant; 
 \verse sed, cum dies quinta illucesceret, viginti iuvenes ex his, qui cum Maccabaeo erant, accensi animis propter blasphemias, murum viriliter aggressi feroci animo, occursantem quemque caedebant; 
\verse sed et alii similiter ascendentes in circumflexione contra eos, qui intus erant, turres incendebant atque ignes inferentes ipsos maledicos vivos concremabant, alii autem portas concidebant et, recepto residuo exercitu, occupaverunt civitatem; 
\verse et Timotheum occultantem se in quodam lacu peremerunt et fratrem illius Chaeream et Apollophanem. 
\verse Quibus gestis, in hymnis et confessionibus benedicebant Dominum, qui magnifice Israel benefaciebat et victoriam dabat illis.
 
\begin{biblechapter}
\verse Sed parvo prorsus post tempore, Lysias procurator regis et propinquus ac negotiorum praepositus graviter ferens de his, quae acciderant, 
\verse congregatis octoginta milibus et equitatu universo, veniebat adversus Iudaeos existimans se civitatem quidem Graecis habitaculum facturum; 
\verse templum vero in pecuniae quaestum sicut cetera delubra gentium habiturum, et per singulos annos venale sacerdotium facturum, 
\verse nequaquam recogitans Dei potestatem, sed elatus multitudine peditum et milibus equitum et octoginta elephantis. 
\verse Ingressus autem Iudaeam et appropians Bethsuris, munito quidem praesidio, distanti autem ab Hierosolymis intervallo quinque stadiorum, illud obsidione premebat. 
\verse Ut autem, qui cum Maccabaeo erant, cognoverunt eum expugnare praesidia, cum fletibus et lacrimis rogabant Dominum, et omnis turba simul, ut bonum angelum mitteret ad salutem Israel. 
\verse Et ipse primus Maccabaeus, sumptis armis, ceteros adhortatus est simul secum periculum subire et ferre auxilium fratribus suis; simul autem et prompto animo impetum fecerunt. 
\verse Ilico vero, cum prope Hierosolymam essent, apparuit praecedens eos eques in veste candida armaturam auream vibrans. 
\verse Tunc omnes simul benedixerunt misericordem Deum et convaluerunt animis non solum homines, sed et bestias ferocissimas et muros ferreos parati penetrare. 
\verse Praeibant in apparatu de caelo habentes adiutorem, miserante super eos Domino. 
\verse Leonum autem more impetu irruentes in hostes, prostraverunt ex eis undecim milia peditum et equitum mille sescentos, universos autem in fugam verterunt. 
\verse Plures autem ex eis vulnerati, nudi evaserunt; sed et ipse Lysias turpiter fugiens evasit.
 \verse Et, quia non insensatus erat, secum ipse reputans factam erga se deminutionem et intellegens invictos esse Hebraeos, potente Deo auxiliante, misit ad eos 
 \verse suasitque eis se consensurum omnibus, quae iusta sunt, et regem quoque persuasurum, ut necessarium crederet se amicum eis esse. 
\verse Annuit autem Maccabaeus in omnibus, quae Lysias rogabat, utilitati consulens; quaecumque enim Maccabaeus scriptis tradidit Lysiae de Iudaeis, rex concessit.
 \verse Nam erant scriptae Iudaeis epistulae a Lysia quidem hunc modum continentes: “ Lysias populo Iudaeorum salutem. 
\verse Ioannes et Abessalom, qui missi fuerant a vobis tradentes responsum rescriptum, postulabant circum ea, quae per illud significabantur. 
\verse Quaecumque igitur oportebat etiam regi perferri, exposui; et, quae res permittebat, concessit. 
\verse Si igitur in negotiis benevolentiam conservaveritis, et deinceps bonorum vobis causa esse tentabo. 
 \verse De ceteris autem per singula mandavi et istis et his, qui a me missi sunt, colloqui vobiscum. 
\verse Bene valete. Anno centesimo quadragesimo octavo, mensis Iovis Corinthii die vicesima et quarta ”.
 \verse Regis autem epistula ista continebat: “ Rex Antiochus Lysiae fratri salutem. 
 \verse Patre nostro inter deos translato, nos volentes eos, qui sunt in regno nostro, sine tumultu attendere ad rerum suarum curam, 
\verse audientes Iudaeos non consensisse patri, ut transferrentur ad Graecas institutiones, sed suo ipsorum instituto adhaerentes postulare sibi concedi legitima sua; 
\verse cupientes igitur hanc quoque gentem extra tumultum esse, iudicamus templum illis restitui remque agi secundum suorum maiorum consuetudinem. 
\verse Bene igitur feceris, si miseris ad eos et dexteram dederis ut, cognita nostra voluntate, bono animo sint et libenter propriarum rerum instaurationi deserviant ”.
 \verse Ad gentem vero regis epistula talis erat: “ Rex Antiochus senatui Iudaeorum et ceteris Iudaeis salutem. 
\verse Si valetis, sic est, ut volumus; sed et ipsi bene valemus. 
\verse Manifestavit nobis Menelaus velle vos redire et in negotiis propriis versari. 
\verse His igitur, qui commeant usque ad diem tricesimum mensis Xanthici, erit dextera cum securitate, 
\verse ut Iudaei utantur cibis et legibus suis sicut et prius, et nemo eorum ullo modo molestiam patietur de his, quae per ignorantiam gesta sunt. 
\verse Misimus autem et Menelaum, qui vos alloquatur. 
\verse Valete. Anno centesimo quadragesimo octavo, Xanthici mensis quinta decima die ”.
 \verse Miserunt autem etiam Romani epistulam ita se habentem: “ Quintus Memmius, Titus Manius, legati Romanorum populo Iudaeorum salutem. 
\verse De his, quae Lysias cognatus regis concessit vobis, et nos consentimus. 
\verse De quibus autem ad regem iudicavit referendum, confestim aliquem mittite inter vos conferentes de his, ut proponamus, sicut congruit vobis; nos enim Antiochiam accedimus. 
\verse Ideoque festinate et mittite aliquos, ut nos quoque sciamus cuius estis voluntatis. 
\verse Bene valete. Anno centesimo quadragesimo octavo, quinta decima die mensis Xanthici ”.
 
\begin{biblechapter}
\verse His factis pactionibus, Lysias pergebat ad regem, Iudaei autem agriculturae operam dabant. 
\verse Sed ex his, qui duces erant in singulis locis, Timotheus et Apollonius Gennaei filius, sed et Hieronymus et Demophon, super hos et Nicanor Cypriarches, non sinebant eos in silentio agere et quiete. 
\verse Ioppitae vero tale quoddam flagitium perpetrarunt: cum rogavissent Iudaeos, cum quibus habitabant, ascendere scaphas, quas ipsi paraverant, cum uxoribus et filiis, quasi nullis inimicitiis in eos subiacentibus, 
\verse secundum autem commune civitatis decretum, et ipsis acquiescentibus, utpote qui pacem obtinere cuperent et nihil suspectum haberent, eos provectos in altum submerserunt non minus ducentos. 
\verse Quam crudelitatem Iudas in suae gentis homines factam ut cognovit, praecepit viris, qui erant cum ipso, et, invocato iusto iudice Deo, 
 \verse venit adversus interfectores fratrum et portum quidem noctu succendit, scaphas exussit, eos autem, qui illuc refugerant, gladio peremit. 
\verse Et, cum conclusus esset locus, discessit quasi iterum reversurus et universam Ioppitarum civitatem eradicaturus. 
\verse Sed, cum cognovisset et eos, qui erant Iamniae, velle pari modo facere habitantibus secum Iudaeis, 
\verse Iamnitis quoque nocte supervenit et portum cum navibus succendit, ita ut lumen ignis appareret Hierosolymis a stadiis ducentis quadraginta.
 \verse Inde, cum iam abiissent novem stadiis et iter facerent ad Timotheum, commiserunt cum eo Arabes non minus quam quinque milia viri et equites quingenti. 
\verse Cumque pugna valida fieret et hi, qui circa Iudam erant, per auxilium Dei prospere gessissent, nomades victi petebant a Iuda dextram sibi dari, promittentes se pascua daturos et in ceteris profuturos eis. 
\verse Iudas autem arbitratus vere in multis eos utiles promisit se pacem acturum cum eis; dextrisque acceptis, discessere ad tabernacula sua. 
\verse Aggressus est autem et civitatem quandam firmam pontibus murisque circumsaeptam, quae a promiscuis gentibus habitabatur, cui nomen Caspin. 
\verse Hi vero, qui intus erant, confidentes in stabilitate murorum et apparatu alimoniarum contumeliosius agebant cum eis, qui circa Iudam erant, maledictis lacessentes et blasphemantes ac loquentes, quae fas non est. 
\verse Qui autem cum Iuda erant, invocato magno mundi Principe, qui sine arietibus et machinis organicis temporibus Iosue praecipitavit Iericho, irruerunt ferociter muris 
\verse et, capta civitate per Dei voluntatem, inenarrabiles caedes fecerunt, ita ut adiacens stagnum latitudinem habens stadiorum duorum defluere repletum sanguine videretur.
 \verse Inde autem discesserunt stadia septingenta quinquaginta et pervenerunt in Characa ad eos, qui dicuntur Tubiani, Iudaeos. 
\verse Et Timotheum quidem in illis locis non comprehenderunt, qui, nullo negotio perfecto, tunc de locis regressus erat, relicto tamen in quodam loco firmissimo praesidio. 
\verse Dositheus autem et Sosipater, ex ducibus, qui cum Maccabaeo erant, exeuntes peremerunt a Timotheo relictos in praesidio plures quam decem milia viros. 
 \verse At Maccabaeus, ordinato exercitu circum se per cohortes, constituit eos super cohortes et adversus Timotheum processit habentem secum centum viginti milia peditum equitumque duo milia quingentos. 
\verse Cognito autem Iudae adventu, Timotheus praemisit mulieres et filios et reliquum apparatum in locum, qui Carnion dicitur; erat enim inexpugnabile et accessu difficile praesidium propter locorum angustias. 
\verse Cumque cohors Iudae prima apparuisset, et pavor factus esset super hostes, ac timor ex praesentia illius, qui universa conspicit, super eos esset, in fugam exsiluerunt, alius alio se ferens, ita ut saepe a suis laederentur et gladiorum acuminibus configerentur. 
\verse Iudas autem vehementer instabat confodiens impios et prostravit ad triginta milia virorum. 
\verse Ipse vero Timotheus incidens in eos, qui erant cum Dositheo et Sosipatre, cum multa adulatione postulabat, ut vivus dimitteretur, eo quod multorum quidem parentes, aliorum autem fratres haberet, et contingeret horum curam non haberi. 
\verse Et cum pluribus modis fidem dedisset secundum hoc constitutum, restituturum se eos illaesos, dimiserunt eum propter fratrum salutem.
 \verse Egressus autem ad Carnion et Atergation interfecit viginti quinque milia corporum. 
\verse Post autem horum fugam et necem, movit exercitum etiam adversus Ephron civitatem munitam, in qua multitudo diversarum gentium inhabitabat, et robusti iuvenes pro muris consistentes fortiter repugnabant; in hac autem machinarum et telorum multi erant apparatus. 
\verse Sed, cum Potentem invocassent, qui potestate sua vires hostium confringit, ceperunt subiectam civitatem et ex eis, qui intus erant, ad viginti quinque milia prostraverunt. 
 \verse Inde profecti ad civitatem Scytharum perrexerunt, quae ab Hierosolymis sescentis stadiis aberat. 
\verse Contestantibus autem his, qui erant illic Iudaei, benevolentiam, quam Scythopolitae erga eos habebant, et mitem occursum temporibus infelicitatis, 
\verse gratias agentes et exhortati etiam de cetero erga genus ipsum benignos esse, venerunt Hierosolymam die sollemni Septimanarum instante.
 \verse Post eam vero, quae dicitur Pentecoste, abierunt contra Gorgiam praepositum Idumaeae. 
\verse Exivit autem cum peditibus tribus milibus et equitibus quadringentis. 
\verse Quibus autem congressis, contigit paucos ruere Iudaeorum. 
 \verse Dositheus vero quidam de iis, qui Bacenoris erant, eques vir et fortis, Gorgiam tenuit chlamydeque apprehensum ducebat eum fortiter; et, cum vellet illum capere vivum, eques quidam de Thracibus irruit in eum umerumque amputavit, et Gorgias effugit in Maresa. 
\verse At illis, qui cum Esdrin erant, diutius pugnantibus et fatigatis, cum invocasset Iudas Dominum, ut adiutorem se ostenderet et ducem belli, 
\verse incipiens patria voce clamorem cum hymnis, irruens improviso in eos, qui circa Gorgiam erant, fugam eis incussit.
 \verse Iudas autem, collecto exercitu, venit in civitatem Odollam et, cum septima dies superveniret, secundum consuetudinem purificati in eodem loco sabbatum egerunt. 
\verse Et sequenti die venerunt, qui cum Iuda erant, eo tempore, quo necessarium factum erat, ut corpora prostratorum tollerent et cum parentibus reponerent in sepulcris paternis. 
\verse Invenerunt autem sub tunicis uniuscuiusque interfectorum donaria idolorum, quae apud Iamniam fuerunt, a quibus lex prohibet Iudaeos. Omnibus ergo manifestum factum est ob hanc causam eos corruisse.
 \verse Omnes itaque, cum benedixissent, quae sunt iusti iudicis, Domini, qui occulta manifesta facit, 
\verse ad obsecrationem conversi sunt, rogantes, ut id, quod factum erat, delictum oblivioni ex integro traderetur. At vero fortissimus Iudas hortatus est populum conservare se sine peccato, cum sub oculis vidissent, quae facta sunt propter peccatum eorum, qui prostrati sunt. 
\verse Et, facta viritim collatione ad duo milia drachmas argenti, misit Hierosolymam offerri pro peccatis sacrificium, valde bene et honeste de resurrectione cogitans. 
\verse Nisi enim eos, qui ceciderant, resurrecturos speraret, superfluum et vanum esset orare pro mortuis. 
\verse Deinde considerans quod hi, qui cum pietate dormitionem acceperant, optimum haberent repositum gratiae donum: 
\verse sancta et pia cogitatio. Unde pro defunctis expiationem fecit, ut a peccato solverentur.
 
\begin{biblechapter}
\verse Anno centesimo quadragesimo nono his, qui erant circa Iudam, notum factum est Antiochum Eupatorem venire cum multitudine adversus Iudaeam 
\verse et cum eo Lysiam procuratorem et praepositum negotiorum, unumquemque habentem exercitum Graecum peditum centum decem milia et equitum quinque milia trecentos et elephantos viginti duos, currus autem cum falcibus trecentos. 
\verse Commiscuit autem se illis et Menelaus et cum multa fallacia hortabatur Antiochum non pro patriae salute, sed sperans se constitui in principatum. 
\verse Sed Rex regum suscitavit animos Antiochi in peccatorem; et, suggerente Lysia hunc esse causam omnium malorum, iussit, ut est consuetudo in loco, adductum in Beroeam necari. 
\verse Erat autem in loco turris quinquaginta cubitorum, cineris plena, et machinam habebat volubilem undique praecipitem in cinerem. 
\verse Illic reum sacrilegii vel quorundam etiam aliorum malorum summitatem factum, omnes propellunt ad interitum. 
\verse Et tali lege praevaricatorem legis contigit mori, nec terram adeptum Menelaum. 
\verse Valde iuste: nam, quia multa erga aram delicta commisit, cuius ignis et cinis erat sanctus, ipse in cinere mortem reportavit.
 \verse Sed rex mente efferatus veniebat, peiora quam quae sub patre suo facta erant, ostensurus Iudaeis. 
\verse Quibus Iudas cognitis, praecepit populo, ut die ac nocte Dominum invocarent, si quando et alias etiam nunc adiuvaret eos, 
\verse quippe qui lege et patria sanctoque templo in eo essent ut privarentur; ac populum, qui nuper paululum respirasset, ne sineret blasphemis nationibus subdi. 
 \verse Omnibus itaque simul idem facientibus et rogantibus misericordem Dominum cum fletu et ieiuniis et prostratione per triduum sine intermissione, hortatus eos Iudas praecepit adesse. 
\verse Ipse vero seorsum cum senioribus cogitavit, prius quam regis exercitus invaderet Iudaeam et obtinerent civitatem, egressos res adiudicare auxilio Dei. 
\verse Dans itaque procurationem Creatori mundi, exhortatus suos, ut fortiter dimicarent usque ad mortem pro legibus, templo, civitate, patria, institutionibus, circa Modin exercitum constituit. 
\verse Cumque suis dedisset signum: “ Victoriam Dei ”, cum iuvenibus fortissimis electis, nocte aggressus castra adversus aulam regiam, interfecit viros ad duo milia et primarium elephantorum una cum eo, qui intra habitaculum erat; 
\verse et postremo metu ac perturbatione castra repleverunt, rebusque prospere gestis, abierunt. 
\verse Die autem iam illucescente hoc factum erat, adiuvante eum Domini protectione.
 \verse Sed rex, accepto gustu audaciae Iudaeorum, artibus loca tentavit. 
\verse Et Bethsuris, quae erat Iudaeorum praesidium munitum, castra admovebat; sed fugabatur, impingebat, minorabatur. 
\verse His autem, qui intus erant, Iudas necessaria mittebat. 
\verse Enuntiavit autem mysteria hostibus Rhodocus quidam de Iudaico exercitu; qui requisitus, comprehensus est et conclusus.
 \verse Iterum rex sermonem habuit ad eos, qui erant in Bethsuris, dextram dedit, accepit, abiit; 
\verse commisit cum his, qui erant cum Iuda, superatus est; cognovit rebellasse Philippum Antiochiae, qui relictus erat super negotia, confusus est; Iudaeos deprecatus est, subditus est, iuravit de omnibus, quae iusta erant, reconciliatus est et obtulit sacrificium, honoravit templum et loco exhibuit humanitatem; 
\verse Maccabaeum excepit, reliquit ducem a Ptolemaide usque ad Gerrenos Hegemonidem, 
\verse venit Ptolemaidam: graviter ferebant Ptolemenses amicitiae conventiones — indignabantur enim supra modum — voluerunt irrita facere pacta. 
\verse Accessit Lysias ad tribunal, exposuit rationem congruenter, persuasit, sedavit, tranquillos fecit, regressus est Antiochiam. Hoc modo res gestae a rege, adventus et profectionis eius, processerunt.
 
\begin{biblechapter}
\verse Sed post triennii tempus cognoverunt, qui cum Iuda erant, Demetrium Seleuci per portum apud Tripolim adnavigantem cum multitudine valida et navibus, 
\verse tenuisse regionem, sublato Antiocho et procuratore eius Lysia. 
\verse Alcimus autem quidam, qui summus sacerdos fuerat, sed voluntarie coinquinatus temporibus seditionis, considerans nullo modo sibi esse salutem neque accessum ultra ad sanctum altare, 
\verse venit ad regem Demetrium, centesimo quinquagesimo primo anno, offerens ei coronam auream et palmam, super haec et thallos, qui templi esse videbantur; et ipsa quidem die siluit. 
\verse Tempus autem opportunum dementiae suae nactus, convocatus a Demetrio ad consilium et interrogatus quo proposito et consilio Iudaei niterentur, 
\verse ad haec respondit: “ Ipsi, qui dicuntur Asidaei, Iudaeorum, quibus praeest Iudas Maccabaeus, bella nutriunt et seditiones movent nec patiuntur regnum esse quietum. 
\verse Unde ego defraudatus parentum gloria, dico autem summo sacerdotio, huc nunc veni, 
\verse primo quidem de his, quae pertinent ad regem, mera fide sentiens, secundo autem etiam civibus meis consulens; nam illorum praedictorum inconsiderantia universum genus nostrum non modice laborat. 
\verse Sed his singulis, tu rex, cognitis, et regioni et obsesso generi nostro, secundum quam habes omnibus obviam humanitatem, prospice; 
\verse nam, quamdiu superest Iudas, impossibile est pacem esse negotiis ”.
 \verse Talibus autem ab hoc dictis, velocius ceteri amici hostiliter se habentes adversus Iudam inflammaverunt Demetrium. 
\verse Qui statim assumens Nicanorem, qui fuit praepositus elephantorum, et ducem ostendens Iudaeae misit, 
\verse datis mandatis, ut ipsum quidem Iudam occideret; eos vero, qui cum illo erant, dispergeret et constitueret Alcimum maximi templi summum sacerdotem. 
\verse Tunc gentes, quae de Iudaea fugerant Iudam, gregatim se Nicanori miscebant, miserias et clades Iudaeorum prosperitates rerum suarum existimantes fore.
 \verse Audito itaque Nicanoris adventu et conventu nationum, conspersi terra rogabant eum, qui populum suum constituit usque in aeternum quique suam portionem signis evidentibus protegit. 
\verse Imperante autem duce, statim inde profectus congreditur eis ad castellum Dessau. 
\verse Simon vero frater Iudae commiserat cum Nicanore, sed lente ob repentinum adversariorum silentium victus evaserat.
 \verse Nicanor tamen audiens quam virtutem haberent, qui cum Iuda erant, et animi magnitudinem pro patriae certaminibus, sanguine iudicium facere metuebat. 
\verse Quam ob rem misit Posidonium et Theodotum et Matthathiam, ut darent dextras atque acciperent. 
\verse Et, cum diu de his consilium ageretur, et ipse dux ad multitudinem rettulisset, et paribus suffragiis pareret sententia, sponsionibus pacis annuerunt. 
\verse Itaque diem constituerunt, qua secreto convenirent eodem, et processit utrimque currus, posuerunt sellas; 
\verse disposuit Iudas armatos paratos locis opportunis, ne forte ab hostibus repente mali aliquid fieret; congruum colloquium fecerunt. 
\verse Morabatur Nicanor Hierosolymis nihilque inique agebat gregesque turbarum, quae congregatae fuerant, dimisit. 
 \verse Habebat autem Iudam semper in conspectu, ex animo erat viro inclinatus. 
 \verse Rogavit eum ducere uxorem filiosque procreare. Nuptias fecit, quiete egit, communiter vivebat.
 \verse Alcimus autem, videns mutuam illorum benevolentiam et factas conventiones, accipiens venit ad Demetrium et dicebat Nicanorem aliena sentire a rebus; Iudam enim regni insidiatorem socium sibi designavit. 
\verse Itaque rex exasperatus et pessimi huius criminationibus irritatus, scripsit Nicanori dicens graviter quidem se ferre de conventionibus, iubere tamen Maccabaeum citius vinctum mittere Antiochiam. 
\verse Quibus cognitis, Nicanor confusus erat et aegre ferebat, si ea, quae convenerant, irrita faceret, nulla a viro facta iniuria; 
 \verse sed, quia regi resisti non poterat, opportunitatem observabat, ut artificio illud perficeret. 
\verse At Maccabaeus videns secum austerius agere Nicanorem et consuetum occursum ferocius exhibentem, intellegens non ex optimo esse austeritatem, non paucis suorum congregatis, occultavit se a Nicanore. 
\verse Quod cum ille cognovit fortiter se a viro astutia praeventum, venit ad maximum et sanctum templum et sacerdotibus solitas hostias offerentibus iussit sibi tradi virum. 
\verse Quibus cum iuramento dicentibus nescire se ubi esset, qui quaerebatur, extendens dexteram ad templum 
\verse iuravit haec: “ Nisi Iudam mihi vinctum tradideritis, istud Dei fanum in planitiem deducam et altare effodiam et templum hic Libero illustre erigam ”. 
\verse Et, his dictis, abiit. Sacerdotes autem protendentes manus in caelum invocabant eum, qui semper propugnator fuit gentis nostrae, haec dicentes: 
\verse “ Tu, Domine universorum, qui nullius indiges, voluisti templum habitationis tuae fieri in nobis; \verse et nunc, Sancte, omnis sanctificationis Domine, conserva in aeternum impollutam domum istam, quae nuper mundata est ”.
 \verse Razis autem quidam de senioribus ab Hierosolymis delatus est Nicanori, vir amator civitatis et valde bene audiens, qui pro affectu pater Iudaeorum appellabatur. 
\verse Hic enim pristinis temporibus seditionis iudicium pertulerat Iudaismi corpusque et animam pro Iudaismo tradiderat cum omni perseverantia. 
\verse Volens autem Nicanor manifestare odium, quod habebat in Iudaeos, misit milites supra quingentos, ut eum comprehenderent; 
\verse putabat enim, si illum cepisset, se cladem istis illaturum. 
\verse Turbis autem turrim iam occupaturis et atrii ianuae vim facientibus atque iubentibus ignem admovere et portas incendi, ipse undique comprehensus supposuit sibi gladium 
\verse volens nobiliter mori potius quam subditus fieri peccatoribus et nobilitate sua indignis iniuriis affici. 
\verse Sed, cum per contentionis festinationem non certo ictu plagam dedisset, et turbae intra ostia irrumperent, recurrens audacter ad murum praecipitavit semetipsum viriliter in turbas; 
\verse quibus velociter locum dantibus intervallo facto, venit per medium spatium vacuum. 
 \verse Et, cum adhuc spiraret, accensus animis surrexit et, cum sanguis ad modum fontis deflueret, et gravissima essent vulnera, cursu turbas pertransiens et stans supra quandam petram praeruptam, 
\verse prorsus exsanguis iam effectus, proferens intestina et sumens utrisque manibus proiecit super turbas et invocans Dominatorem vitae ac spiritus, ut haec ipsi iterum redderet, ita vita defunctus est.
 
\begin{biblechapter}
\verse Nicanor autem, ut comperit eos, qui cum Iuda erant, in locis esse iuxta Samariam, cogitavit requietionis die cum omni securitate eos aggredi. 
\verse Iudaeis vero, qui illum per necessitatem sequebantur, dicentibus: “ Ne ita ferociter et barbare disperdas, sed honorem tribue praehonoratae diei cum sanctificatione ab eo, qui universa conspicit ”, 
\verse ille infelix interrogavit, si est potens in caelo, qui imperavit agi diem sabbatorum. 
\verse Et respondentibus illis: “ Est Dominus vivus ipse in caelo potens, qui iussit colere septimam diem ”; 
\verse at ille ait: “ Et ego potens sum super terram, qui impero sumi arma et negotia regis impleri ”. Tamen non obtinuit, ut nefarium consilium perficeret.
 \verse Et Nicanor quidem cum summa superbia cervicem erigens cogitaverat commune trophaeum statuere de iis, qui cum Iuda erant. 
\verse Maccabaeus autem sine intermissione confidebat cum omni spe auxilium se consequi a Domino; 
\verse et hortabatur suos, ne formidarent adventum nationum, sed in mente habentes adiutoria sibi facta de caelo et nunc sperarent ab Omnipotente sibi affuturam victoriam. 
\verse Et allocutus eos de Lege et Prophetis, admonens eos etiam de certaminibus, quae perfecerant, promptiores constituit eos. 
\verse Et, animis eorum excitatis, denuntiavit simul ostendens gentium fallaciam et iuramentorum praevaricationem. 
\verse Cum autem singulos illorum armavisset, non tam clipeorum et hastarum munitione quam per bonos sermones exhortatione, cumque somnium fide dignum exposuisset, supra modum universos laetificavit.
 \verse Erat autem huiuscemodi visus eius: Oniam, qui fuerat summus sacerdos, virum honestum et bonum, verecundum occursu, modestum moribus et eloquium digne proferentem et qui a puero omnes virtutes domesticas exercuerat, manus protendentem orare pro omni populo Iudaeorum. 
\verse Post hoc sic apparuisse virum canitie et gloria praestantem et mirabilem quandam et magni decoris esse eminentiam circa illum. 
\verse Respondentem vero Oniam dixisse: “ Hic est fratrum amator, qui multum orat pro populo et sancta civitate, Ieremias propheta Dei ”. 
\verse Protendentem autem Ieremiam dextram dedisse Iudae gladium aureum et, cum daret, dixisse haec: 
\verse “ Accipe sanctum gladium munus a Deo, in quo confringes adversarios ”.
 \verse Exhortati itaque Iudae sermonibus bonis valde, et qui poterant ad virtutem incitare et animos iuvenum confortare, statuerunt castra non tendere, sed fortiter inferri et cum omni virtute confligentes de negotiis iudicare, eo quod civitas et sancta et templum periclitarentur. 
\verse Erat enim timor pro uxoribus et filiis itemque pro fratribus et cognatis in minore parte iacens, maximus vero et primus pro sanctificato templo. 
\verse Sed et eos, qui in civitate erant comprehensi, non minima sollicitudo habebat propter illum sub aperto concursum. 
\verse Et, cum iam omnes exspectarent iudicium futurum, hostesque iam committerent, atque exercitus esset ordinatus, et bestiae opportuno in loco constitutae, et equitatus dispositus, 
\verse considerans Maccabaeus adventum multitudinis et apparatum varium armorum et ferocitatem bestiarum, extendens manus in caelum prodigia facientem Dominum invocavit, sciens quoniam non est per arma, sed prout ab ipso iudicatum fuerit dignis tribuit victoriam. 
\verse Dixit autem invocans hoc modo: “ Tu, Domine, qui misisti angelum tuum sub Ezechia rege Iudaeae, et interfecit de castris Sennacherib ad centum octoginta quinque milia, 
 \verse et nunc, Dominator caelorum, mitte angelum bonum ante nos in timorem et tremorem; 
\verse magnitudine brachii tui exterreantur, qui cum blasphemia veniunt adversus sanctum populum tuum ”. Et hic quidem in his finem fecit.
 \verse Qui autem cum Nicanore erant, cum tubis et canticis admovebant; 
\verse hi vero qui erant cum Iuda, cum invocatione et orationibus congressi sunt cum hostibus. 
\verse Et manibus quidem pugnantes, sed Dominum cordibus orantes, prostraverunt non minus triginta quinque milia, praesentia Dei magnifice delectati. 
\verse Cumque cessassent ab opere et cum gaudio redirent, cognoverunt Nicanorem proruisse cum armis suis; 
\verse facto itaque clamore et tumultu, patria voce omnipotentem Dominum benedicebant. 
\verse Et praecepit ille, qui per omnia corpore et animo primus fuerat in certamine pro civibus, qui iuventutis benevolentiam in suam gentem conservaverat, caput Nicanoris abscindi et manum cum umero, ac Hierosolymam perferri. 
\verse Quo cum pervenisset, convocatis contribulibus et sacerdotibus, ante altare stans accersiit eos, qui in arce erant; 
\verse et, ostenso capite iniqui Nicanoris et manu nefarii, quam extendens contra domum sanctam omnipotentis Dei magnifice gloriatus est, 
\verse linguam etiam impii Nicanoris praecisam dixit particulatim avibus daturum, pretia autem dementiae contra templum suspendere.
 \verse Omnes igitur in caelum benedixerunt manifestum Dominum dicentes: “ Benedictus, qui locum suum incontaminatum servavit! ”. 
\verse Alligavit autem Nicanoris caput de summa arce evidens omnibus et manifestum signum auxilii Domini. 
\verse Itaque omnes communi consilio decreverunt nullo modo diem istum absque celebritate praeterire, habere autem celebrem tertiam decimam diem, mensis duodecimi — Adar dicitur voce Syriaca — pridie Mardochaei diei.
 \verse Igitur his erga Nicanorem sic gestis, et ex illis temporibus ab Hebraeis civitate possessa, ego quoque hic faciam finem sermonis. 
\verse Et, si quidem bene et apte compositioni, hoc et ipse volebam; sin autem exigue et modice, hoc est, quod assequi poteram. 
\verse Sicut enim vinum solummodo bibere, similiter autem rursus et aquam, contrarium est, quemadmodum autem vinum aquae contemperatum iam et delectabilem gratiam perficit, huiusmodi etiam structura sermonis delectat aures eorum, quibus contingat compositionem legere. Hic autem erit finis.
\end{biblechapter}
