\biblebook{Liber Exodus}

\begin{biblechapter}   
\verse Haec sunt nomina filiorum Israel, qui ingressi sunt Aegyptum cum Iacob; singuli cum domibus suis introierunt: 
\verse Ruben, Simeon, Levi, Iuda, 
\verse Issachar, Zabulon et Beniamin, 
\verse Dan et Nephthali, Gad et Aser. 
\verse Erant igitur omnes animae eorum, qui egressi sunt de femore Iacob, septuaginta; Ioseph autem in Aegypto erat. 
\verse Quo mortuo et universis fratribus eius omnique cognatione illa, 
\verse filii Israel creverunt et pullulantes multiplicati sunt ac roborati nimis impleverunt terram. 
\verse Surrexit interea rex novus super Aegyptum, qui ignorabat Ioseph; 
\verse et ait ad populum suum: “Ecce, populus filiorum Israel multus et fortior nobis est;  
\verse venite, prudenter agamus cum eo, ne forte multiplicetur et, si ingruerit contra nos bellum, addatur inimicis nostris, expugnatisque nobis, egrediatur de terra". 
\verse Praeposuit itaque eis magistros operum, ut affligerent eos oneribus; aedificaveruntque urbes promptuarias pharaoni, Phithom et Ramesses.  
\verse Quantoque opprimebant eos, tanto magis multiplicabantur et crescebant. 
\verse Formidaveruntque filios Israel Aegyptii et in servitutem redegerunt eos 
\verse atque ad amaritudinem perducebant vitam eorum operibus duris luti et lateris omnique famulatu, quo in terrae operibus premebantur. 
\verse Dixit autem rex Aegypti obstetricibus Hebraeorum, quarum una vocabatur Sephra, altera Phua, 
\verse praecipiens eis: “Quando obstetricabitis Hebraeas, et partus tempus advenerit, si masculus fuerit, interficite eum; si femina, reservate". 
\verse Timuerunt autem obstetrices Deum et non fecerunt iuxta praeceptum regis Aegypti, sed conservabant mares. 
\verse Quibus ad se accersitis rex ait: “Quidnam est hoc, quod facere voluistis, ut pueros servaretis?". 
\verse Quae responderunt: “Non sunt Hebraeae sicut Aegyptiae mulieres; ipsae enim robustae sunt et, priusquam veniamus ad eas, pariunt". 
\verse Bene ergo fecit Deus obstetricibus, et crevit populus confortatusque est nimis; 
\verse et, quia timuerunt obstetrices Deum, aedificavit illis domos. 
\verse Praecepit ergo pharao omni populo suo dicens: “Quidquid masculini sexus natum fuerit, in flumen proicite; quidquid feminei, reservate". 
\end{biblechapter}

\begin{biblechapter}  
\verse Egressus est vir de domo Levi et accepit uxorem stirpis suae; 
\verse quae concepit et peperit filium et videns eum elegantem abscondit tribus mensibus.  
\verse Cumque iam celare non posset, sumpsit fiscellam scirpeam et linivit eam bitumine ac pice; posuitque intus infantulum et exposuit eum in carecto ripae fluminis, 
\verse stante procul sorore eius et considerante eventum rei. 
\verse Ecce autem descendebat filia pharaonis, ut lavaretur in flumine, et puellae eius gradiebantur per crepidinem alvei. Quae cum vidisset fiscellam in papyrione, misit unam e famulabus suis; et allatam 
\verse aperiens cernensque in ea parvulum vagientem, miserta eius ait: “De infantibus Hebraeorum est hic". 
\verse Cui soror pueri: “Vis, inquit, ut vadam et vocem tibi mulierem Hebraeam, quae nutrire possit tibi infantulum?". 
\verse Respondit: “Vade". Perrexit puella et vocavit matrem infantis. 
\verse Ad quam locuta filia pharaonis: “Accipe, ait, puerum istum et nutri mihi; ego dabo tibi mercedem tuam". Suscepit mulier et nutrivit puerum adultumque tradidit filiae pharaonis. 
\verse Quem illa adoptavit in locum filii vocavitque nomen eius Moysen dicens: “Quia de aqua tuli eum". 
\verse In diebus illis, postquam creverat, Moyses egressus est ad fratres suos; viditque afflictionem eorum et virum Aegyptium percutientem quendam de Hebraeis fratribus suis. 
\verse Cumque circumspexisset huc atque illuc et nullum adesse vidisset, percussum Aegyptium abscondit sabulo. 
\verse Et egressus die altero conspexit duos Hebraeos rixantes dixitque ei, qui faciebat iniuriam: “Quare percutis proximum tuum?". 
\verse Qui respondit: “Quis te constituit principem et iudicem super nos? Num occidere me tu vis, sicut occidisti Aegyptium?". Timuit Moyses et ait: “Quomodo palam factum est verbum istud?". 
\verse Audivitque pharao sermonem hunc et quaerebat occidere Moysen. Qui fugiens de conspectu eius moratus est in terra Madian; venit ergo in terram Madian et sedit iuxta puteum. 
\verse Erant autem sacerdoti Madian septem filiae, quae venerunt ad hauriendam aquam; et impletis canalibus adaquare cupiebant greges patris sui.  
\verse Supervenere pastores et eiecerunt eas: surrexitque Moyses et, defensis puellis, adaquavit oves earum. 
\verse Quae cum revertissent ad Raguel patrem suum, dixit ad eas: “Cur velocius venistis solito?". 
\verse Responderunt: “Vir Aegyptius liberavit nos de manu pastorum; insuper et hausit aquam nobis potumque dedit ovibus". 
\verse At ille: “Ubi est?", inquit. “Quare dimisistis hominem? Vocate eum, ut comedat panem". 
\verse Consensit ergo Moyses habitare cum eo accepitque Sephoram filiam eius uxorem. 
\verse Quae peperit ei filium, quem vocavit Gersam dicens: “Advena sum in terra aliena". 
\verse Post multum vero temporis mortuus est rex Aegypti; et ingemiscentes filii Israel propter opera vociferati sunt, ascenditque clamor eorum ad Deum ab operibus. 
\verse Et audivit gemitum eorum ac recordatus est foederis, quod pepigit cum Abraham, Isaac et Iacob; 
\verse et respexit Dominus filios Israel et apparuit eis. 
\end{biblechapter}

\begin{biblechapter}  
\verse Moyses autem pascebat oves Iethro soceri sui sacerdotis Madian; cumque minasset gregem ultra desertum, venit ad montem Dei Horeb. 
\verse Apparuitque ei angelus Domini in flamma ignis de medio rubi; et videbat quod rubus arderet et non combureretur. 
\verse Dixit ergo Moyses: “Vadam et videbo visionem hanc magnam, quare non comburatur rubus". 
\verse Cernens autem Dominus quod pergeret ad videndum, vocavit eum Deus de medio rubi et ait: “Moyses, Moyses". Qui respondit: “Adsum". 
\verse At ille: “Ne appropies, inquit, huc; solve calceamentum de pedibus tuis; locus enim, in quo stas, terra sancta est". 
\verse Et ait: “Ego sum Deus patris tui, Deus Abraham, Deus Isaac et Deus Iacob". Abscondit Moyses faciem suam; non enim audebat aspicere contra Deum. 
\verse Cui ait Dominus: “Vidi afflictionem populi mei in Aegypto et clamorem eius audivi propter duritiam exactorum eorum. 
\verse Et sciens dolorem eius descendi, ut liberem eum de manibus Aegyptiorum et educam de terra illa in terram bonam et spatiosam, in terram, quae fluit lacte et melle, ad loca Chananaei et Hetthaei et Amorraei et Pherezaei et Hevaei et Iebusaei. 
\verse Clamor ergo filiorum Israel venit ad me, vidique afflictionem eorum, qua ab Aegyptiis opprimuntur;  
\verse sed veni, mittam te ad pharaonem, ut educas populum meum, filios Israel, de Aegypto". 
\verse Dixitque Moyses ad Deum: “Quis sum ego, ut vadam ad pharaonem et educam filios Israel de Aegypto?". 
\verse Qui dixit ei: “Ego ero tecum; et hoc habebis signum quod miserim te: cum eduxeris populum de Aegypto, servietis Deo super montem istum". 
\verse Ait Moyses ad Deum: “Ecce, ego vadam ad filios Israel et dicam eis: Deus patrum vestrorum misit me ad vos. Si dixerint mihi: "Quod est nomen eius?" quid dicam eis?". 
\verse Dixit Deus ad Moysen: “Ego sum qui sum". Ait: “Sic dices filiis Israel: Qui sum misit me ad vos". 
\verse Dixitque iterum Deus ad Moysen: “Haec dices filiis Israel: Dominus, Deus patrum vestrorum, Deus Abraham, Deus Isaac et Deus lacob, misit me ad vos; hoc nomen mihi est in aeternum, et hoc memoriale meum in generationem et generationem. 
\verse Vade et congrega seniores Israel et dices ad eos: Dominus, Deus patrum vestrorum, apparuit mihi, Deus Abraham, Deus Isaac et Deus Iacob, dicens: Visitans visitavi vos et vidi omnia, quae acciderunt vobis in Aegypto; 
\verse et dixi: Educam vos de afflictione Aegypti in terram Chananaei et Hetthaei et Amorraei et Pherezaei et Hevaei et Iebusaei, ad terram fluentem lacte et melle. 
\verse Et audient vocem tuam; ingredierisque tu et seniores Israel ad regem Aegypti, et dicetis ad eum: Dominus, Deus Hebraeorum, occurrit nobis; et nunc eamus viam trium dierum in solitudinem, ut immolemus Domino Deo nostro. 
\verse Sed ego scio quod non dimittet vos rex Aegypti, ut eatis, nisi per manum validam. 
\verse Extendam enim manum meam et percutiam Aegyptum in cunctis mirabilibus meis, quae facturus sum in medio eius; post haec dimittet vos.  
\verse Daboque gratiam populo huic coram Aegyptiis, et, cum egrediemini, non exibitis vacui. 
\verse Sed postulabit mulier a vicina sua et ab hospita sua vasa argentea et aurea ac vestes; ponetisque eas super filios et filias vestras et spoliabitis Aegyptum". 
\end{biblechapter}

\begin{biblechapter}  
\verse Respondens Moyses ait: “Quid autem, si non credent mihi neque audient vocem meam, sed dicent: "Non apparuit tibi Dominus?"". 
\verse Dixit ergo ad eum: “Quid est quod tenes in manu tua?". Respondit: “Virga". 
\verse Dixitque Dominus: “Proice eam in terram!". Proiecit, et versa est in serpentem, ita ut fugeret Moyses. 
\verse Dixitque Dominus: “Extende manum tuam et apprehende caudam eius!". Extendit et tenuit, versaque est in virgam. 
\verse “Ut credant, inquit, quod apparuerit tibi Dominus, Deus patrum suorum, Deus Abraham, Deus Isaac et Deus Iacob". 
\verse Dixitque Dominus rursum: “Mitte manum tuam in sinum tuum!". Quam cum misisset in sinum, protulit leprosam instar nivis. 
\verse “Retrahe, ait, manum tuam in sinum tuum!". Retraxit et protulit iterum, et erat similis carni reliquae. 
\verse “Si non crediderint, inquit, tibi, neque audierint sermonem signi prioris, credent verbo signi sequentis. 
\verse Quod si nec duobus quidem his signis crediderint neque audierint vocem tuam, sume aquam fluminis et effunde eam super aridam, et, quidquid hauseris de fluvio, vertetur in sanguinem". 
\verse Ait Moyses: “Obsecro, Domine, non sum eloquens ab heri et nudiustertius et ex quo locutus es ad servum tuum, nam impeditioris et tardioris linguae sum". 
\verse Dixit Dominus ad eum: “Quis fecit os hominis? Aut quis fabricatus est mutum vel surdum vel videntem vel caecum? Nonne ego? 
\verse Perge igitur, et ego ero in ore tuo; doceboque te quid loquaris". 
\verse At ille: “Obsecro, inquit, Domine, mitte quem missurus es". 
\verse Iratus Dominus in Moysen ait: “Aaron, frater tuus Levites, scio quod eloquens sit; ecce ipse egreditur in occursum tuum vidensque te laetabitur corde. 
\verse Loquere ad eum et pone verba mea in ore eius; et ego ero in ore tuo et in ore illius et ostendam vobis quid agere debeatis. 
\verse Ipse loquetur pro te ad populum et erit os tuum; tu autem eris ei ut Deus. 
\verse Virgam quoque hanc sume in manu tua, in qua facturus es signa". 
\verse Abiit Moyses et reversus est ad Iethro socerum suum dixitque ei: “Vadam, quaeso, et revertar ad fratres meos in Aegyptum, ut videam, si adhuc vivant". Cui ait Iethro: “Vade in pace". 
\verse Dixit ergo Dominus ad Moysen in Madian: “Vade, revertere in Aegyptum; mortui sunt enim omnes, qui quaerebant animam tuam". 
\verse Tulit Moyses uxorem suam et filios suos et imposuit eos super asinum; reversusque est in Aegyptum portans virgam Dei in manu sua. 
\verse Dixitque ei Dominus revertenti in Aegyptum: “Vide, ut omnia ostenta, quae posui in manu tua, facias coram pharaone; ego indurabo cor eius, et non dimittet populum. 
\verse Dicesque ad eum: Haec dicit Dominus: Filius meus primogenitus Israel.  
\verse Dico tibi: Dimitte filium meum, ut serviat mihi; si autem non vis dimittere eum, ecce ego interficiam filium tuum primogenitum". 
\verse Cumque esset in itinere, in deversorio, occurrit ei Dominus et volebat occidere eum. 
\verse Tulit ilico Sephora acutissimam petram et circumcidit praeputium filii sui; tetigitque pedes eius et ait: “Sponsus sanguinum tu mihi es". 
\verse Et dimisit eum, postquam dixerat: “Sponsus sanguinum", ob circumcisionem. 
\verse Dixit autem Dominus ad Aaron: “Vade in occursum Moysi in desertum". Qui perrexit obviam ei in montem Dei et osculatus est eum. 
\verse Narravitque Moyses Aaron omnia verba Domini, quibus miserat eum, et signa, quae mandaverat. 
\verse Veneruntque simul et congregaverunt cunctos seniores filiorum Israel. 
\verse Locutusque est Aaron omnia verba, quae dixerat Dominus ad Moysen, et fecit signa coram populo. 
\verse Et credidit populus, audieruntque quod visitasset Dominus filios Israel et quod respexisset afflictionem eorum; et proni adoraverunt. 
\end{biblechapter}

\begin{biblechapter}  
\verse Post haec ingressi sunt Moyses et Aaron et dixerunt pharaoni: “Haec dicit Dominus, Deus Israel: Dimitte populum meum, ut sacrificet mihi in deserto". 
\verse At ille responclit: “Quis est Dominus, ut audiam vocem eius et dimittam Israel? Nescio Dominum et Israel non dimittam". 
\verse Dixeruntque: “Deus Hebraeorum occurrit nobis; eamus, quaeso, viam trium dierum in solitudinem et sacrificemus Domino Deo nostro, ne forte accidat nobis pestis aut gladius". 
\verse Ait ad eos rex Aegypti: “Quare, Moyses et Aaron, sollicitatis populum ab operibus suis? Ite ad onera vestra". 
\verse Dixitque pharao: “Multus nimis iam est populus terrae; videtis quod turba succreverit; quanto magis si dederitis eis requiem ab operibus?". 
\verse Praecepit ergo in die illo exactoribus populi et praefectis eius dicens: 
\verse “Nequaquam ultra dabitis paleas populo ad conficiendos lateres sicut prius, sed ipsi vadant et colligant stipulas. 
\verse Et mensuram laterum, quam prius faciebant, imponetis super eos; nec minuetis quidquam. Vacant enim et idcirco vociferantur dicentes: "Eamus et sacrificemus Deo nostro". 
\verse Opprimantur operibus et expleant ea, ut non acquiescant verbis mendacibus". 
\verse Igitur egressi exactores populi et praefecti eius dixerunt ad populum: “Sic dicit pharao: "Non do vobis paleas. 
\verse Ite et colligite, sicubi invenire poteritis, nec minuetur quid quam de opere vestro"". 
\verse Dispersusque est populus per omnem terram Aegypti ad colligendas paleas. 
\verse Exactores quoque instabant dicentes: “Complete opus vestrum cotidie, ut prius facere solebatis, quando dabantur vobis paleae". 
\verse Flagellatique sunt praefecti filiorum Israel, quos constituerant super eos exactores pharaonis dicentes: “Quare non implestis mensuram laterum sicut prius, nec heri nec hodie?". 
\verse Veneruntque praefecti filiorum Israel et vociferati sunt ad pharaonem dicentes: “Cur ita agis contra servos tuos? 
\verse Paleae non dantur nobis, et lateres similiter imperantur; en famuli tui flagellis caedimur, et populus tuus est in culpa". 
\verse Qui ait: “Vacatis otio et idcirco dicitis: "Eamus et sacrificemus Domino". 
\verse Ite ergo et operamini; paleae non dabuntur vobis, et reddetis consuetum numerum laterum". 
\verse Videbantque se praefecti filiorum Israel in malo, eo quod diceretur eis: “Non minuetur quidquam de lateribus per singulos dies"; 
\verse occurreruntque Moysi et Aaron, qui stabant ex adverso egredientibus a pharaone, 
\verse et dixerunt ad eos: “Videat Dominus et iudicet, quoniam foetere fecistis odorem nostrum coram pharaone et servis eius; et praebuistis ei gladium, ut occideret nos". 
\verse Reversusque est Moyses ad Dominum et ait: “Domine, cur afflixisti populum istum? Quare misisti me? 
\verse Ex eo enim quo ingressus sum ad pharaonem, ut loquerer in nomine tuo, afflixit populum tuum; et non liberasti eos". 
\end{biblechapter}

\begin{biblechapter}  
\verse Dixitque Dominus ad Moysen: “Nunc videbis quae facturus sim pharaoni; per manum enim fortem dimittet eos et in manu robusta eiciet illos de terra sua". 
\verse Locutusque est Dominus ad Moysen dicens: “Ego Dominus, 
\verse qui apparui Abraham, Isaac et Iacob ut Deus omnipotens; et nomen meum Dominum non indicavi eis. 
\verse Pepigique cum eis foedus, ut darem illis terram Chanaan, terram peregrinationis eorum, in qua fuerunt advenae. 
\verse Ego audivi gemitum filiorum Israel, quia Aegyptii oppresserunt eos, et recordatus sum pacti mei. 
\verse Ideo dic filiis Israel: Ego Dominus, qui educam vos de ergastulo Aegyptiorum; et eruam de servitute ac redimam in brachio excelso et iudiciis magnis. 
\verse Et assumam vos mihi in populum et ero vester Deus; et scietis quod ego sum Dominus Deus vester, qui eduxerim vos de ergastulo Aegyptiorum 
\verse et induxerim in terram, super quam levavi manum meam, ut darem eam Abraham, Isaac et Iacob; daboque illam vobis possidendam, ego Dominus". 
\verse Narravit ergo Moyses omnia filiis Israel; qui non acquieverunt ei propter angustiam spiritus et opus durissimum. 
\verse Locutusque est Dominus ad Moysen dicens: 
\verse “Ingredere et loquere ad pharaonem regem Aegypti, ut dimittat filios Israel de terra sua". 
\verse Respondit Moyses coram Domino: “Ecce, filii Israel non audiunt me, et quomodo audiet me pharao, praesertim cum incircumcisus sim labiis?". 
\verse Locutusque est Dominus ad Moysen et Aaron et dedit mandatum ad filios Israel et ad pharaonem regem Aegypti, ut educerent filios Israel de terra Aegypti. 
\verse Isti sunt principes domorum per familias suas. Filii Ruben primogeniti Israelis: Henoch et Phallu, Hesron et Charmi; hae cognationes Ruben. 
\verse Filii Simeon: Iamuel et Iamin et Ahod et Iachin et Sohar er Saul filius Chananitidis; hae progenies Simeon. 
\verse Et haec nomina filiorum Levi per cognationes suas: Gerson et Caath et Merari; anni autem vitae Levi fuerunt centum triginta septem. 
\verse Filii Gerson: Lobni et Semei per cognationes suas. 
\verse Filii Caath: Amram et Isaar et Hebron et Oziel; anni quoque vitae Caath centum triginta tres. 
\verse Filii Merari: Moholi et Musi; hae cognationes Levi per familias suas. 
\verse Accepit autem Amram uxorem Iochabed amitam suam, quae peperit ei Aaron et Moysen; fueruntque anni vitae Amram centum triginta septem. 
\verse Filii quoque Isaar: Core et Napheg et Zechri. 
\verse Filii quoque Oziel: Misael et Elisaphan et Sethri. 
\verse Accepit autem Aaron uxorem Elisabeth filiam Aminadab sororem Naasson, quae peperit ei Nadab et Abiu et Eleazar et Ithamar. 
\verse Filii quoque Core: Asir et Elcana et Abiasaph; hae sunt cognationes Coritarum. 
\verse At vero Eleazar filius Aaron accepit uxorem de filiabus Phutiel, quae peperit ei Phinees; hi sunt principes familiarum Leviticarum per cognationes suas. 
\verse Iste est Aaron et Moyses, quibus praecepit Dominus, ut educerent filios Israel de terra Aegypti per turmas suas. 
\verse Hi sunt qui loquuntur ad pharaonem regem Aegypti, ut educant filios Israel de Aegypto; iste est Moyses et Aaron 
\verse in die, qua locutus est Dominus ad Moysen in terra Aegypti. 
\verse Et locutus est Dominus ad Moysen dicens: “Ego Dominus; loquere ad pharaonem regem Aegypti omnia, quae ego loquor tibi". 
\verse Et ait Moyses coram Domino: “En incircumcisus labiis sum. Quomodo audiet me pharao?". 
\end{biblechapter}

\begin{biblechapter}  
\verse Dixitque Dominus ad Moysen: “Ecce constitui te deum pharaonis, et Aaron frater tuus erit propheta tuus. 
\verse Tu loqueris omnia, quae mando tibi; et ille loquetur ad pharaonem, ut dimittat filios Israel de terra sua. 
\verse Sed ego indurabo cor eius et multiplicabo signa et ostenta mea in terra Aegypti.  
\verse Et non audiet vos; immittamque manum meam super Aegyptum et educam exercitum et populum meum, filios Israel, de terra Aegypti per iudicia maxima. 
\verse Et scient Aegyptii quia ego sum Dominus, qui extenderim manum meam super Aegyptum et eduxerim filios Israel de medio eorum". 
\verse Fecit itaque Moyses et Aaron, sicut praeceperat Dominus; ita egerunt. 
\verse Erat autem Moyses octoginta annorum, et Aaron octoginta trium, quando locuti sunt ad pharaonem. 
\verse Dixitque Dominus ad Moysen et Aaron: 
\verse “Cum dixerit vobis pharao: "Ostendite signum", dices ad Aaron: Tolle virgam tuam et proice eam coram pharaone, ac vertetur in colubrum". 
\verse Ingressi itaque Moyses et Aaron ad pharaonem fecerunt, sicut praeceperat Dominus; proiecitque Aaron virgam coram pharaone et servis eius, quae versa est in colubrum. 
\verse Vocavit autem pharao sapientes et maleficos, et fecerunt etiam ipsi magi Aegypti per incantationes suas similiter. 
\verse Proieceruntque singuli virgas suas, quae versae sunt in colubros; sed devoravit virga Aaron virgas eorum.  
\verse Induratumque est cor pharaonis, et non audivit eos, sicut dixerat Dominus. 
\verse Dixit autem Dominus ad Moysen: “Ingravatum est cor pharaonis: non vult dimittere populum. 
\verse Vade ad eum mane. Ecce egredietur ad aquas; et stabis in occursum eius super ripam fluminis. Et virgam, quae conversa est in serpentem, tolles in manu tua 
\verse dicesque ad eum: Dominus, Deus Hebraeorum, misit me ad te dicens: Dimitte populum meum, ut sacrificet mihi in deserto; et usque ad praesens audire noluisti. 
\verse Haec igitur dicit Dominus: In hoc scies quod sim Dominus: ecce percutiam virga, quae in manu mea est, aquam fluminis; et vertetur in sanguinem. 
\verse Pisces quoque, qui sunt in fluvio, morientur, et computrescent aquae, et taedebit Aegyptios bibere aquam fluminis". 
\verse Dixit quoque Dominus ad Moysen: “Dic ad Aaron: Tolle virgam tuam et extende manum tuam super aquas Aegypti, super fluvios eorum et rivos ac paludes et omnes lacus aquarum, ut vertantur in sanguinem; et sit cruor in omni terra Aegypti, tam in ligneis vasis quam in saxeis". 
\verse Feceruntque ita Moyses et Aaron, sicut praeceperat Dominus. Et elevans virgam percussit aquam fluminis coram pharaone et servis eius; quae versa est in sanguinem. 
\verse Et pisces, qui erant in flumine, mortui sunt, computruitque fluvius, et non poterant Aegyptii bibere aquam fluminis; et fuit sanguis in tota terra Aegypti. 
\verse Feceruntque similiter malefici Aegyptiorum incantationibus suis; et induratum est cor pharaonis, nec audivit eos, sicut dixerat Dominus. 
\verse Avertitque se et ingressus est domum suam nec ad hoc apposuit cor suum. 
\verse Foderunt autem omnes Aegyptii per circuitum fluminis aquam, ut biberent; non enim poterant bibere de aqua fluminis. 
\verse Impletique sunt septem dies, postquam percussit Dominus fluvium. 
\verse Dixit quoque Dominus ad Moysen: “Ingredere ad pharaonem et dices ad eum: Haec dicit Dominus: Dimitte populum meum, ut sacrificet mihi. 
\verse Sin autem nolueris dimittere, ecce ego percutiam omnes terminos tuos ranis. 
\verse Et ebulliet fluvius ranas, quae ascendent et ingredientur domum tuam et cubiculum lectuli tui et super stratum tuum et in domos servorum tuorum et in populum tuum et in furnos tuos et in pistrina tua; 
\verse et ad te et ad populum tuum et ad omnes servos tuos intrabunt ranae". 
\end{biblechapter}

\begin{biblechapter}  
\verse Dixitque Dominus ad Moysen: “Dic ad Aaron: Extende manum tuam cum baculo tuo super fluvios, super rivos ac paludes et educ ranas super terram Aegypti". 
\verse Et extendit Aaron manum super aquas Aegypti, et ascenderunt ranae operueruntque terram Aegypti. 
\verse Fecerunt autem et malefici per incantationes suas similiter eduxeruntque ranas super terram Aegypti. 
\verse Vocavit autem pharao Moysen et Aaron et dixit: “Orate Dominum, ut auferat ranas a me et a populo meo, et dimittam populum, ut sacrificet Domino". 
\verse Dixitque Moyses ad pharaonem: “Constitue mihi, quando deprecer pro te et pro servis et pro populo tuo, ut abigantur ranae a te et a domo tua et tantum in flumine remaneant". 
\verse Qui respondit: “Cras". At ille: “Iuxta verbum, inquit, tuum faciam, ut scias quoniam non est sicut Dominus Deus noster. 
\verse Et recedent ranae a te et a domo tua et a servis tuis et a populo tuo; tantum in flumine remanebunt". 
\verse Egressique sunt Moyses et Aaron a pharaone; et clamavit Moyses ad Dominum pro sponsione ranarum, quam condixerat pharaoni.  
\verse Fecitque Dominus iuxta verbum Moysi, et mortuae sunt ranae de domibus et de villis et de agris; 
\verse congregaveruntque eas in immensos aggeres, et computruit terra. 
\verse Videns autem pharao quod data esset requies, ingravavit cor suum et non audivit eos, sicut dixerat Dominus. 
\verse Dixitque Dominus ad Moysen: “Loquere ad Aaron: Extende virgam tuam et percute pulverem terrae, et sint scinifes in universa terra Aegypti". 
\verse Feceruntque ita; et extendit Aaron manum virgam tenens percussitque pulverem terrae. Et facti sunt scinifes in hominibus et in iumentis; omnis pulvis terrae versus est in scinifes per totam terram Aegypti. 
\verse Feceruntque similiter malefici incantationibus suis, ut educerent scinifes; et non potuerunt. Erantque scinifes tam in hominibus quam in iumentis; 
\verse et dixerunt malefici ad pharaonem: “Digitus Dei est hic". Induratumque est cor pharaonis et non audivit eos, sicut praeceperat Dominus. 
\verse Dixit quoque Dominus ad Moysen: “Consurge diluculo et sta coram pharaone. Egredietur enim ad aquas, et dices ad eum: Haec dicit Dominus: Dimitte populum meum, ut sacrificet mihi. 
\verse Quod si non dimiseris eum, ecce ego immittam in te et in servos tuos et in populum tuum et in domos tuas omne genus muscarum; et implebuntur domus Aegyptiorum muscis et etiam humus, in qua fuerint. 
\verse Et segregabo in die illa terram Gessen, in qua populus meus est, ut non sint ibi muscae, et scias quoniam ego Dominus in medio terrae; 
\verse ponamque divi sionem inter populum meum et populum tuum; cras erit signum istud". 
\verse Fecitque Dominus ita; et venit musca gravissima in domos pharaonis et servorum eius et in omnem terram Aegypti, corruptaque est terra ab huiuscemodi muscis. 
\verse Vocavitque pharao Moysen et Aaron et ait eis: “Ite, sacrificate Deo vestro in terra". 
\verse Et ait Moyses: “Non potest ita fieri: abominationes enim Aegyptiorum immolabimus Domino Deo nostro; quod si mactaverimus ea, quae colunt Aegyptii, coram eis, lapidibus nos obruent. 
\verse Viam trium dierum pergemus in solitudinem et sacrificabimus Domino Deo nostro, sicut praecepit nobis". 
\verse Dixitque pharao: “Ego dimittam vos, ut sacrificetis Domino Deo vestro in deserto, verumtamen longius ne abeatis; rogate pro me". 
\verse Et ait Moyses: “Egressus a te, orabo Dominum, et recedet musca a pharaone et a servis suis et a populo eius cras; verumtamen noli ultra fallere, ut non dimittas populum sacrificare Domino". 
\verse Egressusque Moyses a pharaone oravit Dominum;  
\verse qui fecit iuxta verbum illius et abstulit muscas a pharaone et a servis suis et a populo eius; non superfuit ne una quidem. 
\verse Et ingravatum est cor pharaonis, ita ut ne hac quidem vice dimitteret populum. 
\end{biblechapter}

\begin{biblechapter}  
\verse Dixit autem Dominus ad Moysen: “Ingredere ad pharaonem et loquere ad eum: Haec dicit Dominus, Deus Hebraeorum: Dimitte populum meum, ut sacrificet mihi. 
\verse Quod si adhuc renuis et retines eos, 
\verse ecce manus Domini erit super possessionem tuam in agris, super equos et asinos et camelos et boves et oves, pestis valde gravis; 
\verse et distinguet Dominus inter possessiones Israel et possessiones Aegyptiorum, ut nihil omnino pereat ex his, quae pertinent ad filios Israel. 
\verse Constituitque Dominus tempus dicens: Cras faciet Dominus verbum istud in terra". 
\verse Fecit ergo Dominus verbum hoc altera die, mortuaque sunt omnia animantia Aegyptiorum; de animalibus vero filiorum Israel nihil omnino periit. 
\verse Et misit pharao ad videndum; nec erat quidquam mortuum de his, quae possidebat Israel. Ingravatumque est cor pharaonis, et non dimisit populum. 
\verse Et dixit Dominus ad Moysen et Aaron: “Tollite plenas manus cineris de camino, et spargat illum Moyses in caelum coram pharaone; 
\verse sitque pulvis super omnem terram Aegypti; erunt enim in hominibus et iumentis ulcera et vesicae turgentes in universa terra Aegypti". 
\verse Tuleruntque cinerem de camino et steterunt coram pharaone, et sparsit illum Moyses in caelum; factaque sunt ulcera vesicarum turgentium in hominibus et iumentis. 
\verse Nec poterant malefici stare coram Moyse propter ulcera, quae in illis erant et in omni terra Aegypti. 
\verse Induravitque Dominus cor pharaonis, et non audivit eos, sicut locutus est Dominus ad Moysen. 
\verse Dixitque Dominus ad Moysen: “Mane consurge et sta coram pharaone et dices ad eum: Haec dicit Dominus, Deus Hebraeorum: Dimitte populum meum, ut sacrificet mihi; 
\verse quia in hac vice mittam omnes plagas meas super cor tuum et super servos tuos et super populum tuum, ut scias quod non sit similis mei in omni terra. 
\verse Nunc enim extendens manum si percussissem te et populum tuum peste, perisses de terra. 
\verse Idcirco autem servavi te, ut ostendam in te fortitudinem meam, et narretur nomen meum in omni terra. 
\verse Adhuc retines populum meum et non vis dimittere eum? 
\verse En pluam cras, hac ipsa hora, grandinem multam nimis, qualis non fuit in Aegypto a die, qua fundata est, usque in praesens tempus. 
\verse Mitte ergo iam nunc et congrega iumenta tua et omnia, quae habes in agro; homines enim et iumenta universa, quae inventa fuerint foris nec congregata de agris, cadet super ea grando, et morientur". 
\verse Qui timuit verbum Domini de servis pharaonis, fecit confugere servos suos et iumenta in domos; 
\verse qui autem neglexit sermonem Domini, dimisit servos suos et iumenta in agris. 
\verse Et dixit Dominus ad Moysen: “Extende manum tuam in caelum, ut fiat grando in universa terra Aegypti super homines et super iumenta et super omnem herbam agri in terra Aegypti". 
\verse Extenditque Moyses virgam in caelum, et Dominus dedit tonitrua et grandinem ac discurrentia fulgura super terram; pluitque Dominus grandinem super terram Aegypti. 
\verse Et grando et ignis immixta pariter ferebantur; tantaeque fuit magnitudinis, quanta ante numquam apparuit in universa terra Aegypti, ex quo gens illa condita est. 
\verse Et percussit grando in omni terra Aegypti cuncta, quae fuerunt in agris, ab homine usque ad iumentum; cunctamque herbam agri percussit grando et omne lignum regionis confregit. 
\verse Tantum in terra Gessen, ubi erant filii Israel, grando non cecidit. 
\verse Misitque pharao et vocavit Moysen et Aaron dicens ad eos: “Nunc peccavi; Dominus iustus, ego et populus meus rei.  
\verse Orate Dominum, ut desinant tonitrua Dei et grando, et dimittam vos, et nequaquam hic ultra manebitis".  
\verse Ait Moyses: “Cum egressus fuero de urbe, extendam palmas meas ad Dominum; et cessabunt tonitrua, et grando non erit, ut scias quia Domini est terra.  
\verse Novi autem quod et tu et servi tui necdum timeatis Dominum Deum".  
\verse Linum ergo et hordeum laesum est, eo quod hordeum iam spicas et linum iam folliculos germinaret;  
\verse triticum autem et far non sunt laesa, quia serotina erant. 
\verse Egressusque Moyses a pharaone ex urbe tetendit manus ad Dominum; et cessaverunt tonitrua et grando, nec ultra effundebatur pluvia super terram.  
\verse Videns autem pharao quod cessasset pluvia et grando et tonitrua, auxit peccatum;  
\verse et ingravatum est cor eius et servorum illius et induratum nimis; nec dimisit filios Israel, sicut dixerat Dominus per manum Moysi. 
\end{biblechapter}

\begin{biblechapter}    
\verse Et dixit Dominus ad Moysen: “Ingredere ad pharaonem: ego enim induravi cor eius et servorum illius, ut faciam signa mea haec in medio eorum,  
\verse et narres in auribus filii tui et nepotum tuorum, quotiens contriverim Aegyptios et signa mea fecerim in eis; et sciatis quia ego Dominus". 
\verse Introierunt ergo Moyses et Aaron ad pharaonem et dixerunt ei: “Haec dicit Dominus, Deus Hebraeorum: Usquequo non vis subici mihi? Dimitte populum meum, ut sacrificet mihi.  
\verse Sin autem resistis et non vis dimittere eum, ecce ego inducam cras locustam in fines tuos,  
\verse quae operiat superficiem terrae, ne quidquam eius appareat, sed comedatur, quod residuum fuerit grandini; corrodet enim omnia ligna, quae germinant in agris.  
\verse Et implebunt domos tuas et servorum tuorum et omnium Aegyptiorum, quantam non viderunt patres tui et avi, ex quo orti sunt super terram usque in praesentem diem". Avertitque se et egressus est a pharaone. 
\verse Dixerunt autem servi pharaonis ad eum: “Usquequo patiemur hoc scandalum? Dimitte homines, ut sacrificent Domino Deo suo; nonne vides quod perierit Aegyptus?".  
\verse Revocaveruntque Moysen et Aaron ad pharaonem, qui dixit eis: “Ite, sacrificate Domino Deo vestro. Quinam sunt qui ituri sunt?".  
\verse Ait Moyses: “Cum parvulis nostris et senioribus pergemus, cum filiis et filiabus, cum ovibus et armentis; est enim sollemnitas Domini nobis".  
\verse Et respondit eis: “Sic Dominus sit vobiscum, quomodo ego dimittam vos et parvulos vestros. Cui dubium est quod pessime cogitetis?  
\verse Non fiet ita, sed ite tantum viri et sacrificate Domino; hoc enim et ipsi petistis". Statimque eiecti sunt de conspectu pharaonis. 
\verse Dixit autem Dominus ad Moysen: “Extende manum tuam super terram Aegypti, ut veniat locusta et ascendat super eam et devoret omnem herbam, quidquid residuum fuerit grandini". 
\verse Et extendit Moyses virgam super terram Aegypti, et Dominus induxit ventum urentem tota die illa et nocte. Et mane facto, ventus urens levavit locustas; 
\verse quae ascenderunt super universam terram Aegypti et sederunt in cunctis finibus Aegyptiorum innumerabiles, quales ante illud tempus non fuerant nec postea futurae sunt. 
\verse Operueruntque universam superficiem terrae, et obscurata est terra. Devoraverunt igitur omnem herbam terrae et, quidquid pomorum in arboribus fuit, quae grando dimiserat; nihilque omnino virens relictum est in lignis et in herbis terrae in cuncta Aegypto. 
\verse Quam ob rem festinus pharao vocavit Moysen et Aaron et dixit eis: “Peccavi in Dominum Deum vestrum et in vos. 
\verse Sed nunc dimittite peccatum mihi tantum hac vice et rogate Dominum Deum vestrum, ut auferat a me saltem mortem istam". 
\verse Egressusque Moyses de conspectu pharaonis oravit Dominum, 
\verse qui flare fecit ventum ab occidente vehementissimum et arreptam locustam proiecit in mare Rubrum; non remansit ne una quidem in cunctis finibus Aegypti. 
\verse Et induravit Dominus cor pharaonis, nec dimisit filios Israel. 
\verse Dixit autem Dominus ad Moysen: “Extende manum tuam in caelum, et sint tenebrae super terram Aegypti tam densae ut palpari queant". 
\verse Extenditque Moyses manum in caelum, et factae sunt tenebrae horribiles in universa terra Aegypti tribus diebus. 
\verse Nemo vidit fratrem suum nec movit se de loco, in quo erat. Ubicumque autem habitabant filii Israel, lux erat. 
\verse Vocavitque pharao Moysen et Aaron et dixit eis: “Ite, sacrificate Domino; oves tantum vestrae et armenta remaneant, parvuli vestri eant vobiscum". 
\verse Ait Moyses: “Etiamsi tu hostias et holocausta dares nobis, quae offeramus Domino Deo nostro, 
\verse tamen et greges nostri pergent nobiscum; non remanebit ex eis ungula, quoniam ex ipsis sumemus, quae necessaria sunt in cultum Domini Dei nostri; praesertim cum ignoremus quid debeat immolari, donec ad ipsum locum perveniamus". 
\verse Induravit autem Dominus cor pharaonis, et noluit dimittere eos. 
\verse Dixitque pharao ad eum: “Recede a me. Cave, ne ultra videas faciem meam; quocumque die apparueris mihi, morieris". 
\verse Respondit Moyses: “Ita fiet, ut locutus es; non videbo ultra faciem tuam". 
\end{biblechapter}

\begin{biblechapter}  
\verse Et dixit Dominus ad Moysen: “Adhuc una plaga tangam pharaonem et Aegyptum, et post haec dimittet vos utique, immo et exire compellet. 
\verse Dices ergo omni plebi, ut postulet vir ab amico suo et mulier a vicina sua vasa argentea et aurea; 
\verse dabit autem Dominus gratiam populo coram Aegyptiis". Fuitque Moyses vir magnus valde in terra Aegypti coram servis pharaonis et omni populo. 
\verse Et ait Moyses: “Haec dicit Dominus: Media nocte egrediar in Aegyptum; 
\verse et morietur omne primogenitum in terra Aegyptiorum, a primogenito pharaonis, qui sedet in solio eius, usque ad primogenitum ancillae, quae est ad molam, et omnia primogenita iumentorum. 
\verse Eritque clamor magnus in universa terra Aegypti, qualis nec ante fuit nec postea futurus est. 
\verse Apud omnes autem filios Israel non mutiet canis contra hominem et pecus, ut sciatis quanto miraculo dividat Dominus Aegyptios et Israel. 
\verse Descendentque omnes servi tui isti ad me et adorabunt me dicentes: "Egredere tu et omnis populus, qui sequitur te". Post haec egrediar". Et exivit a pharaone iratus nimis. 
\verse Dixit autem Dominus ad Moysen: “Non audiet vos pharao, ut multa signa fiant in terra Aegypti". 
\verse Moyses autem et Aaron fecerunt omnia ostenta haec coram pharaone; et induravit Dominus cor pharaonis, nec dimisit filios Israel de terra sua. 
\end{biblechapter}

\begin{biblechapter}  
\verse Dixit Dominus ad Moysen et Aaron in terra Aegypti: 
\verse “Mensis iste vobis principium mensium, primus erit in mensibus anni. 
\verse Loquimini ad universum coetum filiorum Israel et dicite eis: Decima die mensis huius tollat unusquisque agnum per familias et domos suas. 
\verse Sin autem minor est numerus, ut sufficere possit ad vescendum agnum, assumet vicinum suum, qui iunctus est domui suae, iuxta numerum animarum, quae sufficere possunt ad esum agni. 
\verse Erit autem vobis agnus absque macula, masculus, anniculus; quem de agnis vel haedis tolletis 
\verse et servabitis eum usque ad quartam decimam diem mensis huius; immolabitque eum universa congregatio filiorum Israel ad vesperam. 
\verse Et sument de sanguine eius ac ponent super utrumque postem et in superliminaribus domorum, in quibus comedent illum; 
\verse et edent carnes nocte illa assas igni et azymos panes cum lactucis amaris. 
\verse Non comedetis ex eo crudum quid nec coctum aqua, sed tantum assum igni; caput cum pedibus eius et intestinis vorabitis. 
\verse Nec remanebit quidquam ex eo usque mane; si quid residuum fuerit, igne comburetis. 
\verse Sic autem comedetis illum: renes vestros accingetis, calceamenta habebitis in pedibus, tenentes baculos in manibus, et comedetis festinanter; est enim Pascha (id est Transitus) Domini! 
\verse Et transibo per terram Aegypti nocte illa percutiamque omne primogenitum in terra Aegypti ab homine usque ad pecus; et in cunctis diis Aegypti faciam iudicia, ego Dominus. 
\verse Erit autem sanguis vobis in signum in aedibus, in quibus eritis; et videbo sanguinem et transibo vos, nec erit in vobis plaga disperdens, quando percussero terram Aegypti. 
\verse Habebitis autem hanc diem in monumentum et celebrabitis eam sollemnem Domino in generationibus vestris cultu sempiterno. 
\verse Septem diebus azyma comedetis. Iam in die primo non erit fermentum in domibus vestris; quicumque comederit fermentatum, a primo die usque ad diem septimum, peribit anima illa de Israel. 
\verse Dies prima erit sancta atque sollemnis, et dies septima eadem festivitate venerabilis. Nihil operis facietis in eis, exceptis his, quae ad vescendum pertinent. 
\verse Et observabitis azyma, in eadem enim ipsa die eduxi exercitum vestrum de terra Aegypti; et custodietis diem istum in generationes vestras ritu perpetuo. 
\verse Primo mense, quarta decima die mensis ad vesperam comedetis azyma; usque ad diem vicesimam primam eiusdem mensis ad vesperam. 
\verse Septem diebus fermentum non invenietur in domibus vestris. Qui comederit fermentatum, peribit anima eius de coetu Israel, tam de advenis quam de indigenis terrae. 
\verse Omne fermentatum non comedetis; in cunctis habitaculis vestris edetis azyma". 
\verse Vocavit autem Moyses omnes seniores filiorum Israel et dixit ad eos: “Ite tollentes animal per familias vestras et immolate Pascha. 
\verse Fasciculumque hyssopi tingite in sanguine, qui est in pelvi, et aspergite ex eo superliminare et utrumque postem. Nullus vestrum egrediatur ostium domus suae usque mane.  
\verse Transibit enim Dominus percutiens Aegyptios; cumque viderit sanguinem in superliminari et in utroque poste, transcendet ostium et non sinet percussorem ingredi domos vestras et laedere. 
\verse Custodite verbum istud legitimum tibi et filiis tuis usque in aeternum. 
\verse Cumque introieritis terram, quam Dominus daturus est vobis, ut pollicitus est, observabitis caeremonias istas;  
\verse et, cum dixerint vobis filii vestri: "Quae est ista religio?", 
\verse dicetis eis: "Victima Paschae Domino est, quando transivit super domos filiorum Israel in Aegypto percutiens Aegyptios et domos nostras liberans"". Incurvatusque populus adoravit; 
\verse et egressi filii Israel fecerunt, sicut praeceperat Dominus Moysi et Aaron. 
\verse Factum est autem in noctis medio, percussit Dominus omne primogenitum in terra Aegypti, a primogenito pharaonis, qui in solio eius sedebat, usque ad primogenitum captivi, qui erat in carcere, et omne primogenitum iumentorum.  
\verse Surrexitque pharao nocte et omnes servi eius cunctaque Aegyptus, et ortus est clamor magnus in Aegypto, neque enim erat domus, in qua non iaceret mortuus.  
\verse Vocatisque pharao Moyse et Aaron nocte, ait: “Surgite, egredimini a populo meo, vos et filii Israel; ite, immolate Domino, sicut dicitis. 
\verse Oves vestras et armenta assumite, ut petieratis, et abeuntes benedicite mihi".  
\verse Urgebantque Aegyptii populum de terra exire velociter dicentes: “Omnes moriemur". 
\verse Tulit igitur populus conspersam farinam, antequam fermentaretur; et ligans pistrina in palliis suis posuit super umeros suos.  
\verse Feceruntque filii Israel, sicut praeceperat Moyses, et petierunt ab Aegyptiis vasa argentea et aurea vestemque plurimam. 
\verse Dominus autem dedit gratiam populo coram Aegyptiis, ut commodarent eis; et spoliaverunt Aegyptios. 
\verse Profectique sunt filii Israel de Ramesse in Succoth, sescenta fere milia peditum virorum absque parvulis. 
\verse Sed et vulgus promiscuum innumerabile ascendit cum eis, oves et armenta, animantia multa nimis. 
\verse Coxeruntque farinam, quam dudum de Aegypto conspersam tulerant, et fecerunt subcinericios panes azymos; neque enim poterant fermentari, cogentibus exire Aegyptiis et nullam facere sinentibus moram; nec pulmenti quidquam occurrerant praeparare. 
\verse Habitatio autem filiorum Israel, qua manserant in Aegypto, fuit quadringentorum triginta annorum. 
\verse Quibus expletis, eadem die egressus est omnis exercitus Domini de terra Aegypti. 
\verse Nox ista vigiliarum Domino, quando eduxit eos de terra Aegypti: hanc observare debent Domino omnes filii Israel in generationibus suis. 
\verse Dixitque Dominus ad Moysen et Aaron: “Haec est religio Paschae: Omnis alienigena non comedet ex eo; 
\verse omnis autem servus empticius circumcidetur et sic comedet; 
\verse advena et mercennarius non edent ex eo. 
\verse In una domo comedetur, nec efferetis de carnibus eius foras nec os illius confringetis.  
\verse Omnis coetus filiorum Israel faciet illud. 
\verse Quod si quis peregrinorum in vestram voluerit transire coloniam et facere Pascha Domini, circumcidetur prius omne masculinum eius, et tunc rite celebrabit eritque sicut indigena terrae; si quis autem circumcisus non fuerit, non vescetur ex eo. 
\verse Eadem lex erit indigenae et colono, qui peregrinatur apud vos". 
\verse Feceruntque omnes filii Israel, sicut praeceperat Dominus Moysi et Aaron; 
\verse et in eadem die eduxit Dominus filios Israel de terra Aegypti per turmas suas. 
\end{biblechapter}

\begin{biblechapter}  
\verse Locutusque est Dominus ad Moysen dicens: 
\verse “Sanctifica mihi omne primogenitum, quod aperit vulvam in filiis Israel, tam de hominibus quam de iumentis: mea sunt enim omnia". 
\verse Et ait Moyses ad populum: “Mementote diei huius, in qua egressi estis de Aegypto et de domo servitutis, quoniam in manu forti eduxit vos Dominus de loco isto, ut non comedatis fermentatum panem. 
\verse Hodie egredimini, mense Abib (id est novarum Frugum). 
\verse Cumque introduxerit te Dominus in terram Chananaei et Hetthaei et Amorraei et Hevaei et Iebusaei, quam iuravit patribus tuis, ut daret tibi, terram fluentem lacte et melle; celebrabis hunc morem sacrorum mense isto. 
\verse Septem diebus vesceris azymis, et in die septimo erit sollemnitas Domini.  
\verse Azyma comedetis septem diebus: non apparebit apud te aliquid fermentatum nec in cunctis finibus tuis. 
\verse Narrabisque filio tuo in die illo dicens: "Propter hoc, quod fecit mihi Dominus, quando egressus sum de Aegypto". 
\verse Et erit quasi signum in manu tua et quasi monumentum inter oculos tuos, ut lex Domini semper sit in ore tuo; in manu enim forti eduxit te Dominus de Aegypto.  
\verse Custodies huiuscemodi cultum statuto tempore a diebus in dies. 
\verse Cumque introduxerit te Dominus in terram Chananaei, sicut iuravit tibi et patribus tuis, et dederit tibi eam, 
\verse separabis omne, quod aperit vulvam, Domino et quod primitivum est in pecoribus tuis; quidquid habueris masculini sexus, consecrabis Domino. 
\verse Primogenitum asini mutabis ove; quod, si non redemeris, interficies. Omne autem primogenitum hominis de filiis tuis pretio redimes. 
\verse Cumque interrogaverit te filius tuus cras dicens: "Quid est hoc?", respondebis ei: "In manu forti eduxit nos Dominus de Aegypto, de domo servitutis. 
\verse Nam, cum induratus esset pharao et nollet nos dimittere, occidit Dominus omne primogenitum in terra Aegypti, a primogenito hominis usque ad primogenitum iumentorum; idcirco immolo Domino omne, quod aperit vulvam, masculini sexus, et omnia primogenita filiorum meorum redimo". 
\verse Erit igitur quasi signum in manu tua et quasi appensum quid ob recordationem inter oculos tuos, eo quod in manu forti eduxit nos Dominus de Aegypto". 
\verse Igitur cum emisisset pharao populum, non eos duxit Deus per viam terrae Philisthim, quae vicina est, reputans ne forte paeniteret populum, si vidisset adversum se bella consurgere, et reverteretur in Aegyptum, 
\verse sed circumduxit per viam deserti, quae est iuxta mare Rubrum. Et armati ascenderunt filii Israel de terra Aegypti. 
\verse Tulit quoque Moyses ossa Ioseph secum, eo quod adiurasset filios Israel dicens: “Visitabit vos Deus; efferte ossa mea hinc vobiscum". 
\verse Profectique de Succoth castrametati sunt in Etham, in extremis finibus solitudinis. 
\verse Dominus autem praecedebat eos ad ostendendam viam per diem in columna nubis et per noctem in columna ignis, ut dux esset itineris utroque tempore. 
\verse Nunquam defuit columna nubis per diem, nec columna ignis per noctem, coram populo. 
\end{biblechapter}

\begin{biblechapter}  
\verse Locutus est autem Dominus ad Moysen dicens: 
\verse “Loquere filiis Israel: Reversi castrametentur e regione Phihahiroth, quae est inter Magdolum et mare contra Beelsephon; in conspectu eius castra ponetis super mare. 
\verse Dicturusque est pharao super filiis Israel: "Errant in terra, conclusit eos desertum". 
\verse Et indurabo cor eius, ac persequetur eos, et glorificabor in pharaone et in omni exercitu eius; scientque Aegyptii quia ego sum Dominus". Feceruntque ita. 
\verse Et nuntiatum est regi Aegyptiorum quod fugisset populus; immutatumque est cor pharaonis et servorum eius super populo, et dixerunt: “Quid hoc fecimus, ut dimitteremus Israel, ne servirent nobis?". 
\verse Iunxit ergo currum et omnem populum suum assumpsit secum; 
\verse tulitque sescentos currus electos et quidquid in Aegypto curruum fuit et bellatores in singulis curribus. 
\verse Induravitque Dominus cor pharaonis regis Aegypti, et persecutus est filios Israel; at illi egressi erant in manu excelsa. 
\verse Cumque persequerentur Aegyptii vestigia praecedentium, reppererunt eos in castris super mare; omnes equi et currus pharaonis, equites et exercitus eius erant in Phihahiroth contra Beelsephon. 
\verse Cumque appropinquasset pharao, levantes filii Israel oculos viderunt Aegyptios post se et timuerunt valde clamaveruntque ad Dominum 
\verse et dixerunt ad Moysen: “Forsitan non erant sepulcra in Aegypto? Ideo tulisti nos, ut moreremur in solitudine. Quid hoc fecisti, ut educeres nos ex Aegypto? 
\verse Nonne iste est sermo, quem loquebamur ad te in Aegypto dicentes: Recede a nobis, ut serviamus Aegyptiis? Multo enim melius erat servire eis quam mori in solitudine". 
\verse Et ait Moyses ad populum: “Nolite timere; state et videte salutem Domini, quam facturus est vobis hodie; Aegyptios enim, quos nunc videtis, nequaquam ultra videbitis usque in sempiternum. 
\verse Dominus pugnabit pro vobis, et vos silebitis". 
\verse Dixitque Dominus ad Moysen: “Quid clamas ad me? Loquere filiis Israel, ut proficiscantur. 
\verse Tu autem eleva virgam tuam et extende manum tuam super mare et divide illud, ut gradiantur filii Israel in medio mari per siccum.  
\verse Ego autem indurabo cor Aegyptiorum, ut persequantur eos; et glorificabor in pharaone et in omni exercitu eius, in curribus et in equitibus illius. 
\verse Et scient Aegyptii quia ego sum Dominus, cum glorificatus fuero in pharaone, in curribus atque in equitibus eius". 
\verse Tollensque se angelus Dei, qui praecedebat castra Israel, abiit post eos; et cum eo pariter columna nubis, priora dimittens, post tergum. 
\verse Stetit inter castra Aegyptiorum et castra Israel; et erat nubes tenebrosa et illuminans noctem, ita ut ad se invicem toto noctis tempore accedere non valerent. 
\verse Cumque extendisset Moyses manum super mare, reppulit illud Dominus, flante vento vehementi et urente tota nocte, et vertit in siccum; divisaque est aqua. 
\verse Et ingressi sunt filii Israel per medium maris sicci; erat enim aqua quasi murus a dextra eorum et laeva. 
\verse Persequentesque Aegyptii ingressi sunt post eos, omnis equitatus pharaonis, currus eius et equites per medium maris. 
\verse Iamque advenerat vigilia matutina, et ecce respiciens Dominus super castra Aegyptiorum per columnam ignis et nubis perturbavit exercitum eorum; 
\verse et impedivit rotas curruum, ita ut difficile moverentur. Dixerunt ergo Aegyptii: “Fugiamus Israelem! Dominus enim pugnat pro eis contra nos". 
\verse Et ait Dominus ad Moysen: “Extende manum tuam super mare, ut revertantur aquae ad Aegyptios super currus et equites eorum". 
\verse Cumque extendisset Moyses manum contra mare, reversum est primo diluculo ad priorem locum; fugientibusque Aegyptiis occurrerunt aquae, et involvit eos Dominus in mediis fluctibus. 
\verse Reversaeque sunt aquae et operuerunt currus et equites cuncti exercitus pharaonis, qui sequentes ingressi fuerant mare; ne unus quidem superfuit ex eis. 
\verse Filii autem Israel perrexerunt per medium sicci maris, et aquae eis erant quasi pro muro a dextris et a sinistris. 
\verse Liberavitque Dominus in die illo Israel de manu Aegyptiorum. Et viderunt Aegyptios mortuos super litus maris 
\verse et manum magnam, quam exercuerat Dominus contra eos; timuitque populus Dominum et crediderunt Domino et Moysi servo eius. 
\end{biblechapter}

\begin{biblechapter}  
\verse Tunc cecinit Moyses et filii Israel carmen hoc Domino, et dixerunt: “Cantemus Domino, gloriose enim magnificatus est: equum et ascensorem eius deiecit in mare! 
\verse Fortitudo mea et robur meum Dominus, et factus est mihi in salutem. Iste Deus meus, et glorificabo eum; Deus patris mei, et exaltabo eum! 
\verse Dominus quasi vir pugnator; Dominus nomen eius! 
\verse Currus pharaonis et exercitum eius proiecit in mare; electi bellatores eius submersi sunt in mari Rubro. 
\verse Abyssi operuerunt eos, descenderunt in profundum quasi lapis. 
\verse Dextera tua, Domine, magnifice in fortitudine, dextera tua, Domine, percussit inimicum. 
\verse Et in multitudine gloriae tuae deposuisti adversarios tuos; misisti iram tuam, quae devoravit eos sicut stipulam. 
\verse Et in spiritu furoris tui congregatae sunt aquae; stetit ut agger unda fluens, coagulatae sunt abyssi in medio mari. 
\verse Dixit inimicus: "Persequar, comprehendam, dividam spolia, implebitur anima mea; evaginabo gladium meum, interficiet eos manus mea!". 
\verse Flavit spiritus tuus, et operuit eos mare; submersi sunt quasi plumbum in aquis vehementibus. 
\verse Quis similis tui in diis, Domine? Quis similis tui, magnificus in sanctitate, terribilis atque laudabilis, faciens mirabilia? 
\verse Extendisti manum tuam, devoravit eos terra. 
\verse Dux fuisti in misericordia tua populo, quem redemisti, et portasti eum in fortitudine tua ad habitaculum sanctum tuum. 
\verse Attenderunt populi et commoti sunt, dolores obtinuerunt habitatores Philisthaeae. 
\verse Tunc conturbati sunt principes Edom, potentes Moab obtinuit tremor, obriguerunt omnes habitatores Chanaan. 
\verse Irruit super eos formido et pavor; in magnitudine brachii tui fiunt immobiles quasi lapis, donec pertranseat populus tuus, Domine, donec pertranseat populus tuus iste, quem possedisti. 
\verse Introduces eos et plantabis in monte hereditatis tuae, firmissimo habitaculo tuo, quod operatus es, Domine, sanctuario, Domine, quod firmaverunt manus tuae. 
\verse Dominus regnabit in aeternum et ultra!". 
\verse Ingressi sunt enim equi pharaonis cum curribus et equitibus eius in mare, et reduxit super eos Dominus aquas maris; filii autem Israel ambu laverunt per siccum in medio eius. 
\verse Sumpsit ergo Maria prophetissa soror Aaron tympanum in manu sua; egressaeque sunt omnes mulieres post eam cum tympanis et choris,  
\verse quibus praecinebat dicens: “Cantemus Domino, gloriose enim magnificatus est: equum et ascensorem eius deiecit in mare!". 
\verse Tulit autem Moyses Israel de mari Rubro, et egressi sunt in desertum Sur; ambulaveruntque tribus diebus per solitudinem et non inveniebant aquam. 
\verse Et venerunt in Mara nec poterant bibere aquas de Mara, eo quod essent amarae; unde vocatum est nomen eius Mara (id est Amaritudo). 
\verse Et murmuravit populus contra Moysen dicens: “Quid bibemus?". 
\verse At ille clamavit ad Dominum, qui ostendit ei lignum; quod cum misisset in aquas, in dulcedinem versae sunt. Ibi constituit ei praecepta atque iudicia et ibi tentavit eum  
\verse dicens: “Si audieris vocem Domini Dei tui et, quod rectum est coram eo, feceris et oboedieris mandatis eius custodierisque omnia praecepta illius, cunctum languorem, quem posui in Aegypto, non inducam super te: Ego enim Dominus sanator tuus". 
\verse Venerunt autem in Elim, ubi erant duodecim fontes aquarum et septuaginta palmae; et castrametati sunt iuxta aquas. 
\end{biblechapter}

\begin{biblechapter}  
\verse Profectique sunt de Elim, et venit omnis congregatio filiorum Israel in desertum Sin, quod est inter Elim et Sinai, quinto decimo die mensis secundi postquam egressi sunt de terra Aegypti. 
\verse Et murmuravit omnis congregatio filiorum Israel contra Moysen et Aaron in solitudine, 
\verse dixeruntque filii Israel ad eos: “Utinam mortui essemus per manum Domini in terra Aegypti, quando sedebamus super ollas carnium et comedebamus panem in saturitate. Cur eduxistis nos in desertum istud, ut occideretis omnem coetum fame?". 
\verse Dixit autem Dominus ad Moysen: “Ecce ego pluam vobis panes de caelo; egrediatur populus et colligat, quae sufficiunt per singulos dies, ut tentem eum, utrum ambulet in lege mea an non. 
\verse Die autem sexta parabunt quod intulerint, et duplum erit quam colligere solebant per singulos dies". 
\verse Dixeruntque Moyses et Aaron ad omnes filios Israel: “Vespere scietis quod Dominus eduxerit vos de terra Aegypti; 
\verse et mane videbitis gloriam Domini. Audivit enim murmur vestrum contra Dominum. Nos vero quid sumus, quia mussitatis contra nos?". 
\verse Et ait Moyses: “Dabit Dominus vobis vespere carnes edere et mane panes in saturitate, eo quod audierit murmurationes vestras, quibus murmurati estis contra eum. Nos enim quid sumus? Nec contra nos est murmur vestrum, sed contra Dominum". 
\verse Dixitque Moyses ad Aaron: “Dic universae congregationi filiorum Israel: Accedite coram Domino; audivit enim murmur ve strum". 
\verse Cumque loqueretur Aaron ad omnem coetum filiorum Israel, respexerunt ad solitudinem, et ecce gloria Domini apparuit in nube. 
\verse Locutus est autem Dominus ad Moysen dicens: 
\verse “Audivi murmurationes filiorum Israel. Loquere ad eos: Vespere comedetis carnes et mane saturabimini panibus scietisque quod ego sum Dominus Deus vester". 
\verse Factum est ergo vespere, et ascendens coturnix operuit castra; mane quoque ros iacuit per circuitum castrorum. 
\verse Cumque operuisset superficiem deserti, apparuit minutum et squamatum in similitudinem pruinae super terram. 
\verse Quod cum vidissent filii Israel, dixerunt ad invicem: “Manhu?" (quod significat: “Quid est hoc?"). Ignorabant enim quid esset. Quibus ait Moyses: “Iste est panis, quem dedit Dominus vobis ad vescendum. 
\verse Hic est sermo, quem praecepit Dominus: "Colligat ex eo unusquisque quantum sufficiat ad vescendum; gomor per singula capita iuxta numerum animarum vestrarum, quae habitant in tabernaculo, sic tolletis"". 
\verse Feceruntque ita filii Israel; et collegerunt alius plus, alius minus.  
\verse Et mensi sunt ad mensuram gomor; nec qui plus collegerat, habuit amplius, nec qui minus paraverat, repperit minus, sed singuli, iuxta id quod edere poterant, congregaverunt. 
\verse Dixitque Moyses ad eos: “Nullus relinquat ex eo in mane". 
\verse Qui non audierunt eum, sed dimiserunt quidam ex eis usque mane, et scatere coepit vermibus atque computruit; et iratus est contra eos Moyses. 
\verse Colligebant autem mane singuli, quantum sufficere poterat ad vescendum; cumque incaluisset sol, liquefiebat. 
\verse In die autem sexta collegerunt cibos duplices, id est duo gomor per singulos homines. Venerunt autem omnes principes congregationis et narraverunt Moysi.  
\verse Qui ait eis: “Hoc est quod locutus est Dominus: Requies, sabbatum sanctum Domino cras; quodcumque torrendum est, torrete et, quae coquenda sunt, coquite; quidquid autem reliquum fuerit, reponite usque in mane". 
\verse Feceruntque ita, ut praeceperat Moyses, et non computruit, neque vermis inventus est in eo.  
\verse Dixitque Moyses: “Comedite illud hodie, quia sabbatum est Domino; non invenietur hodie in agro. 
\verse Sex diebus colligite; in die autem septimo sabbatum est Domino, idcirco non invenietur in eo". 
\verse Venitque septima dies; et egressi de populo, ut colligerent, non invenerunt. 
\verse Dixit autem Dominus ad Moysen: “Usquequo non vultis custodire mandata mea et legem meam? 
\verse Videte quod Dominus dederit vobis sabbatum et propter hoc die sexta tribuit vobis cibos duplices; maneat unusquisque apud semetipsum, nullus egrediatur de loco suo die septimo". 
\verse Et sabbatizavit populus die septimo. 
\verse Appellavitque domus Israel nomen eius Man: quod erat quasi semen coriandri album, gustusque eius quasi similae cum melle. 
\verse Dixit autem Moyses: “Iste est sermo, quem praecepit Dominus: "Imple gomor ex eo, et custodiatur in generationes vestras, ut noverint panem, quo alui vos in solitudine, quando educti estis de terra Aegypti"". 
\verse Dixitque Moyses ad Aaron: “Sume vas unum et mitte ibi man, quantum potest capere gomor; et repone coram Domino ad servandum in generationes vestras". 
\verse Sicut praecepit Dominus Moysi, posuit illud Aaron coram testimonio reservandum. 
\verse Filii autem Israel comederunt man quadraginta annis, donec venirent in terram habitabilem; hoc cibo aliti sunt, usquequo tangerent fines terrae Chanaan.  
\verse Gomor autem decima pars est ephi. 
\end{biblechapter}

\begin{biblechapter}  
\verse Igitur profecta omnis congregatio filiorum Israel de deserto Sin per mansiones suas iuxta sermonem Domini, castrametati sunt in Raphidim, ubi non erat aqua ad bibendum populo. 
\verse Qui iurgatus contra Moysen ait: “Da nobis aquam, ut bibamus". Quibus respondit Moyses: “Quid iurgamini contra me? Cur tentatis Dominum?". 
\verse Sitivit ergo ibi populus prae aquae penuria et murmuravit contra Moysen dicens: “Cur fecisti nos exire de Aegypto, ut occideres nos et liberos nostros ac iumenta siti?". 
\verse Clamavit autem Moyses ad Dominum dicens: “Quid faciam populo huic? Adhuc paululum et lapidabunt me".  
\verse Et ait Dominus ad Moysen: “Antecede populum et sume tecum de senioribus Israel, et virgam, qua percussisti fluvium, tolle in manu tua et vade. 
\verse En ego stabo coram te ibi super petram Horeb; percutiesque petram, et exibit ex ea aqua, ut bibat populus". Fecit Moyses ita coram senioribus Israel. 
\verse Et vocavit nomen loci illius Massa et Meriba, propter iurgium filiorum Israel et quia tentaverunt Dominum dicentes: “Estne Dominus in nobis an non?". 
\verse Venit autem Amalec et pugnabat contra Israel in Raphidim. 
\verse Dixitque Moyses ad Iosue: “Elige nobis viros et egressus pugna contra Amalec; cras ego stabo in vertice collis habens virgam Dei in manu mea". 
\verse Fecit Iosue, ut locutus erat ei Moyses, et pugnavit contra Amalec; Moyses autem et Aaron et Hur ascenderunt super verticem collis. 
\verse Cumque levaret Moyses manus, vincebat Israel; sin autem remisisset, superabat Amalec. 
\verse Manus autem Moysi erant graves; sumentes igitur lapidem posuerunt subter eum, in quo sedit; Aaron autem et Hur sustentabant manus eius ex utraque parte. Et factum est ut manus eius non lassarentur usque ad occasum solis. 
\verse Vicitque Iosue Amalec et populum eius in ore gladii. 
\verse Dixit autem Dominus ad Moysen: “Scribe hoc ob monumentum in libro et trade auribus Iosue; delebo enim memoriam Amalec sub caelo". 
\verse Aedificavitque Moyses altare et vocavit nomen eius Dominus Nissi (Dominus vexillum meum) 
\verse dicens: “Quia manus contra solium Domini: bellum Domino erit contra Amalec a generatione in generationem". 
\end{biblechapter}

\begin{biblechapter}  
\verse Cumque audisset Iethro sacerdos Madian socer Moysi omnia, quae fecerat Deus Moysi et Israel populo suo, eo quod eduxisset Dominus Israel de Aegypto,  
\verse tulit Sephoram uxorem Moysi, quam remiserat, 
\verse et duos filios eius, quorum unus vocabatur Gersam, dicente patre: “Advena fui in terra aliena", 
\verse alter vero Eliezer: “Deus enim, ait, patris mei adiutor meus, et eruit me de gladio pharaonis". 
\verse Venit ergo Iethro socer Moysi et filii eius et uxor eius ad Moysen in desertum, ubi erat castrametatus iuxta montem Dei; 
\verse et mandavit Moysi dicens: “Ego socer tuus Iethro venio ad te et uxor tua et duo filii tui cum ea". 
\verse Qui egressus in occursum soceri sui adoravit et osculatus est eum, salutaveruntque se mutuo verbis pacificis. Cumque intrasset tabernaculum, 
\verse narravit Moyses socero suo cuncta, quae fecerat Dominus pharaoni et Aegyptiis propter Israel, universumque laborem, qui accidisset eis in itinere, et quod liberaverat eos Dominus. 
\verse Laetatusque est Iethro super omnibus bonis, quae fecerat Dominus Israel, eo quod eruisset eum de manu Aegyptiorum, 
\verse et ait: “Benedictus Dominus, qui liberavit vos de manu Aegyptiorum et de manu pharaonis. 
\verse Nunc cognovi quia magnus Dominus super omnes deos, eo quod eruerit populum de manu Aegyptiorum, qui superbe egerunt contra illos". 
\verse Obtulit ergo Iethro socer Moysi holocausta et hostias Deo; veneruntque Aaron et omnes seniores Israel, ut comederent panem cum eo coram Deo. 
\verse Altero autem die sedit Moyses, ut iudicaret populum, qui assistebat Moysi de mane usque ad vesperam. 
\verse Quod cum vidisset socer eius, omnia scilicet, quae agebat in populo, ait: “Quid est hoc, quod facis in plebe? Cur solus sedes, et omnis populus praestolatur de mane usque ad vesperam?". 
\verse Cui respondit Moyses: “Venit ad me populus quaerens sententiam Dei. 
\verse Cumque acciderit eis aliqua disceptatio, veniunt ad me, ut iudicem inter eos et ostendam praecepta Dei et leges eius". 
\verse At ille: “Non bonam, inquit, rem facis. 
\verse Consumeris et tu et populus iste, qui tecum est. Ultra vires tuas est negotium; solus illud non poteris sustinere. 
\verse Sed audi verba mea atque consilia, et erit Deus tecum: Esto tu populo in his, quae ad Deum pertinent, ut referas causas ad Deum 
\verse ostendasque populo praecepta et leges viamque, per quam ingredi debeant, et opus, quod facere debeant. 
\verse Provide autem de omni plebe viros strenuos et timentes Deum, in quibus sit veritas, et qui oderint avaritiam, et constitue ex eis tribunos et centuriones et quinquagenarios et decanos, 
\verse qui iudicent populum omni tempore. Quidquid autem maius fuerit, referant ad te, et ipsi minora tantummodo iudicent; leviusque sit tibi, partito cum aliis onere. 
\verse Si hoc feceris, implebis imperium Dei et praecepta eius poteris sustentare, et omnis hic populus revertetur ad loca sua cum pace". 
\verse Quibus auditis, Moyses fecit omnia, quae ille suggesserat; 
\verse et, electis viris strenuis de cuncto Israel, constituit eos principes populi, tribunos et centuriones et quinquagenarios et decanos, 
\verse qui iudicabant plebem omni tempore. Quidquid autem gravius erat, referebant ad eum, faciliora tantummodo iudicantes. 
\verse Dimisitque socerum suum, qui reversus abiit in terram suam. 
\end{biblechapter}

\begin{biblechapter}  
\verse Mense tertio egressionis Israel de terra Aegypti, in die hac venerunt in solitudinem Sinai. 
\verse Nam profecti de Raphidim et pervenientes usque in desertum Sinai, castrametati sunt in eodem loco, ibique Israel fixit tentoria e regione montis. 
\verse Moyses autem ascendit ad Deum, vocavitque eum Dominus de monte et ait: “Haec dices domui Iacob et annuntiabis filiis Israel: 
\verse Vos ipsi vidistis, quae fecerim Aegyptiis, quomodo portaverim vos super alas aquilarum et adduxerim ad me. 
\verse Si ergo audieritis vocem meam et custodieritis pactum meum, eritis mihi in peculium de cunctis populis; mea est enim omnis terra. 
\verse Et vos eritis mihi regnum sacerdotum et gens sancta. Haec sunt verba, quae loqueris ad filios Israel". 
\verse Venit Moyses et, convocatis maioribus natu populi, exposuit omnes sermones, quos mandaverat Dominus. 
\verse Responditque universus populus simul: “Cuncta, quae locutus est Dominus, faciemus". Cumque rettulisset Moyses verba populi ad Dominum, 
\verse ait ei Dominus: “Ecce ego veniam ad te in caligine nubis, ut audiat me populus loquentem ad te et tibi quoque credat in perpetuum". Nuntiavit ergo Moyses verba populi ad Dominum, 
\verse qui dixit ei: “Vade ad populum et sanctifica illos hodie et cras; laventque vestimenta sua 
\verse et sint parati in diem tertium. In die enim tertio descendet Dominus coram omni plebe super montem Sinai. 
\verse Constituesque terminos populo per circuitum et dices: Cavete, ne ascendatis in montem nec tangatis fines illius; omnis, qui tetigerit montem, morte morietur. 
\verse Manus non tanget eum, sed lapidibus opprimetur aut confodietur iaculis; sive iumentum fuerit, sive homo, non vivet. Cum coeperit clangere bucina, tunc ascendant in montem". 
\verse Descenditque Moyses de monte ad populum et sanctificavit eum; cumque lavissent vestimenta sua, 
\verse ait ad eos: “Estote parati in diem tertium; ne appropinquetis uxoribus vestris". 
\verse Iamque advenerat tertius dies, et mane inclaruerat; et ecce coeperunt audiri tonitrua ac micare fulgura et nubes densissima operire montem, clangorque bucinae vehementius perstrepebat; et timuit populus, qui erat in castris. 
\verse Cumque eduxisset eos Moyses in occursum Dei de loco castrorum, steterunt ad radices montis. 
\verse Totus autem mons Sinai fumabat, eo quod descendisset Dominus super eum in igne, et ascenderet fumus ex eo quasi de fornace. Et tremuit omnis mons vehementer. 
\verse Et sonitus bucinae paulatim crescebat in maius; Moyses loquebatur, et Deus respondebat ei cum voce. 
\verse Descenditque Dominus super montem Sinai in ipso montis vertice et vocavit Moysen in cacumen eius. Quo cum ascendisset, 
\verse dixit ad eum: “Descende et contestare populum, ne velit transcendere terminos ad videndum Dominum, et pereat ex eis plurima multitudo. 
\verse Sacerdotes quoque, qui accedunt ad Dominum, sanctificentur, ne percutiat eos". 
\verse Dixitque Moyses ad Dominum: “Non poterit vulgus ascendere in montem Sinai, tu enim testificatus es et iussisti dicens: "Pone terminos circa montem et sanctifica illum"". 
\verse Cui ait Dominus: “Vade, descende; ascendesque tu et Aaron tecum, sacerdotes autem et populus ne transeant terminos nec ascendant ad Dominum, ne interficiat illos". 
\verse Descenditque Moyses ad populum et omnia narravit eis. 
\end{biblechapter}

\begin{biblechapter}  
\verse Locutusque est Deus cunctos sermones hos: 
\verse “Ego sum Dominus Deus tuus, qui eduxi te de terra Aegypti, de domo servitutis. 
\verse Non habebis deos alienos coram me. 
\verse Non facies tibi sculptile neque omnem similitudinem eorum, quae sunt in caelo desuper et quae in terra deorsum et quae in aquis sub terra. 
\verse Non adorabis ea neque coles, quia ego sum Dominus Deus tuus, Deus zelotes, visitans iniquitatem patrum in filiis in tertiam et quartam generationem eorum, qui oderunt me, 
\verse et faciens misericordiam in milia his, qui diligunt me et custodiunt praecepta mea. 
\verse Non assumes nomen Domini Dei tui in vanum, nec enim habebit insontem Dominus eum, qui assumpserit nomen Domini Dei sui frustra. 
\verse Memento, ut diem sabbati sanctifices. 
\verse Sex diebus operaberis et facies omnia opera tua; 
\verse septimus autem dies sabbatum Domino Deo tuo est; non facies omne opus tu et filius tuus et filia tua, servus tuus et ancilla tua, iumentum tuum et advena, qui est intra portas tuas. 
\verse Sex enim diebus fecit Dominus caelum et terram et mare et omnia, quae in eis sunt, et requievit in die septimo; idcirco benedixit Dominus diei sabbati et sanctificavit eum. 
\verse Honora patrem tuum et matrem tuam, ut sis longaevus super terram, quam Dominus Deus tuus dabit tibi. 
\verse Non occides. 
\verse Non moechaberis. 
\verse Non furtum facies. 
\verse Non loqueris contra proximum tuum falsum testimonium. 
\verse Non concupisces domum proximi tui: non desiderabis uxorem eius, non servum, non ancillam, non bovem, non asinum nec omnia, quae illius sunt". 
\verse Cunctus autem populus videbat voces et lampades et sonitum bucinae montemque fumantem; et perterriti ac pavore concussi steterunt procul 
\verse dicentes Moysi: “Loquere tu nobis, et audiemus; non loquatur nobis Deus, ne moriamur".  
\verse Et ait Moyses ad populum: “Nolite timere; ut enim probaret vos, venit Deus, et ut timor illius esset in vobis, ne peccaretis". 
\verse Stetitque populus de longe; Moyses autem accessit ad caliginem, in qua erat Deus. 
\verse Dixit praeterea Dominus ad Moysen: “Haec dices filiis Israel: Vos vidistis quod de caelo locutus sim vobis. 
\verse Non facietis praeter me deos argenteos nec deos aureos facietis vobis. 
\verse Altare de terra facietis mihi et offeretis super eo holocausta et pacifica vestra, oves vestras et boves; in omni loco, in quo memoriam fecero nominis mei, veniam ad te et benedicam tibi. 
\verse Quod si altare lapideum feceris mihi, non aedificabis illud de sectis lapidibus; si enim levaveris cultrum super eo, polluetur. 
\verse Non ascendes per gradus ad altare meum, ne reveletur turpitudo tua. 
\end{biblechapter}

\begin{biblechapter}  
\verse Haec sunt iudicia, quae propones eis: 
\verse Si emeris servum Hebraeum, sex annis serviet tibi; in septimo egredietur liber gratis. 
\verse Si solus intraverit, solus exeat; si habens uxorem, et uxor egredietur simul. 
\verse Sin autem dominus dederit illi uxorem, et pepererit filios et filias, mulier et liberi eius erunt domini sui; ipse vero exibit solus. 
\verse Quod si dixerit servus: "Diligo dominum meum et uxorem ac liberos, non egrediar liber", 
\verse afferet eum dominus ad Deum et applicabit eum ad ostium vel postes perforabitque aurem eius subula; et erit ei servus in saeculum. 
\verse Si quis vendiderit filiam suam in famulam, non egredietur sicut servi exire consueverunt. 
\verse Si displicuerit oculis domini sui, cui tradita fuerat, faciat eam redimi; populo autem alieno vendendi non habebit potestatem, quia fraudavit eam. 
\verse Sin autem filio suo desponderit eam, iuxta morem filiarum faciet illi. 
\verse Quod si alteram sibi acceperit, cibum et vestimentum et concubitum non negabit. 
\verse Si tria ista non fecerit ei, egredietur gratis absque pretio. 
\verse Qui percusserit hominem, et ille mortuus fuerit, morte moriatur. 
\verse Qui autem non est insidiatus, sed Deus illum tradidit in manus eius, constituam tibi locum, in quem fugere debeat. 
\verse Si quis de industria occiderit proximum suum et per insidias, ab altari meo evelles eum, ut moriatur. 
\verse Qui percusserit patrem suum aut matrem, morte moriatur. 
\verse Qui furatus fuerit hominem sive vendiderit eum sive inventus fuerit in manu eius, morte moriatur. 
\verse Qui maledixerit patri suo vel matri, morte moriatur. 
\verse Si rixati fuerint viri, et percusserit alter proximum suum lapide vel pugno, et ille mortuus non fuerit, sed iacuerit in lectulo, 
\verse si surrexerit et ambulaverit foris super baculum suum, impunitus erit, qui percusserit, ita tamen, ut operas eius deperditas et impensas pro medela restituat. 
\verse Qui percusserit servum suum vel ancillam virga, et mortui fuerint in manibus eius, ultioni subiacetur. 
\verse Sin autem uno die vel duobus supervixerit, non subiacebit poenae, quia pecunia illius est. 
\verse Si rixati fuerint viri, et percusserit quis mulierem praegnantem et abortivum quidem fecerit, sed aliud quid adversi non acciderit, subiacebit damno, quantum maritus mulieris expetierit, et arbitri iudicaverint. 
\verse Sin autem quid adversi acciderit, reddet animam pro anima, 
\verse oculum pro oculo, dentem pro dente, manum pro manu, pedem pro pede, 
\verse adustionem pro adustione, vulnus pro vulnere, livorem pro livore. 
\verse Si percusserit quispiam oculum servi sui aut ancillae et luscos eos fecerit, dimittet eos liberos pro oculo. 
\verse Dentem quoque si excusserit servo vel ancillae suae, dimittet eos liberos pro dente. 
\verse Si bos cornu percusserit virum aut mulierem, et mortui fuerint, lapidibus obruetur, et non comedentur carnes eius; dominus autem bovis innocens erit.  
\verse Quod si bos cornupeta fuerit ab heri et nudiustertius, et contestati sunt dominum eius, nec recluserit eum, occideritque virum aut mulierem: et bos lapidibus obruetur, et dominum illius occident. 
\verse Quod si pretium ei fuerit impositum, dabit pro anima sua, quidquid fuerit postulatus. 
\verse Filium quoque vel filiam si cornu percusserit, simili sententiae subiacebit. 
\verse Si servum vel ancillam invaserit, triginta siclos argenti dabit domino; bos vero lapidibus opprimetur. 
\verse Si quis aperuerit cisternam vel foderit et non operuerit eam, cedideritque bos vel asinus in eam, 
\verse dominus cisternae reddet pretium iumentorum; quod autem mortuum est, ipsius erit. 
\verse Si bos alienus bovem alterius vulneraverit, et ille mortuus fuerit, vendent bovem vivum et divident pretium; cadaver autem mortui inter se dispertient.  
\verse Sin autem notum erat quod bos cornupeta esset ab heri et nudiustertius, et non custodivit eum dominus suus, reddet bovem pro bove et cadaver integrum accipiet. 
\verse Si quis furatus fuerit bovem aut ovem et occiderit vel vendiderit, quinque boves pro uno bove restituet et quattuor oves pro una ove. 
\end{biblechapter}

\begin{biblechapter}  
\verse Si effringens fur domum sive suffodiens fuerit inventus et, accepto vulnere, mortuus fuerit, percussor non erit reus sanguinis. 
\verse Quod si orto sole hoc fecerit, erit reus sanguinis. Fur plene restituet. Si non habuerit, quod reddat, venumdabitur pro furto. 
\verse Si inventum fuerit apud eum, quod furatus est, vivens sive bos sive asinus sive ovis, duplum restituet. 
\verse Si quispiam depasci permiserit agrum vel vineam et dimiserit iumentum suum, ut depascatur agrum alienum, restituet plene ex agro suo secundum fruges eius; si autem totum agrum depastum fuerit, quidquid optimum habuerit in agro suo vel in vinea, restituet. 
\verse Si egressus ignis invenerit spinas et comprehenderit acervos frugum sive stantes segetes sive agrum, reddet damnum, qui ignem succenderit. 
\verse Si quis commendaverit amico pecuniam aut vasa in custodiam, et ab eo, qui susceperat, furto ablata fuerint, si invenitur fur, duplum reddet. 
\verse Si latet fur, dominus domus applicabitur ad Deum et iurabit quod non extenderit manum in rem proximi sui. 
\verse In omni causa fraudis tam de bove quam de asino et ove ac vestimento et, quidquid damnum inferre potest, si quis dixerit: “Hoc est!", ad Deum utriusque causa perveniet, et, quem Deus condemnaverit, duplum restituet proximo suo. 
\verse Si quis commendaverit proximo suo asinum, bovem, ovem vel omne iumentum ad custodiam, et mortuum fuerit aut fractum vel captum ab hostibus, nullusque hoc viderit, 
\verse iusiurandum per Dominum erit in medio quod non extenderit manum ad rem proximi sui; suscipietque dominus iuramentum, et ille reddere non cogetur. 
\verse Quod si furto ablatum fuerit, restituet damnum domino; 
\verse si dilaceratum a bestia, deferat, quod occisum est, in testimonium et non restituet. 
\verse Qui a proximo suo quidquam horum mutuo postulaverit, et fractum aut mortuum fuerit, domino non praesente, reddere compelletur. 
\verse Quod si impraesentiarum dominus fuerit, non restituet. Si mercennarius est, venit in mercedem operis sui. 
\verse Si seduxerit quis virginem necdum desponsatam dormieritque cum ea, pretio acquiret eam sibi uxorem. 
\verse Si pater virginis eam dare noluerit, appendet ei pecuniam iuxta pretium pro virginibus dandum. 
\verse Maleficam non patieris vivere. 
\verse Qui coierit cum iumento, morte moriatur. 
\verse Qui immolat diis, occidetur, praeter Domino soli. 
\verse Advenam non opprimes neque affliges eum; advenae enim et ipsi fuistis in terra Aegypti. 
\verse Viduae et pupillo non nocebitis. 
\verse Si laeseritis eos, vociferabuntur ad me, et ego audiam clamorem eorum; 
\verse et indignabitur furor meus, percutiamque vos gladio, et erunt uxores vestrae viduae et filii vestri pupilli. 
\verse Si pecuniam mutuam dederis in populo meo pauperi, qui habitat tecum, non eris ei quasi creditor; non imponetis ei usuram. 
\verse Si pignus a proximo tuo acceperis pallium, ante solis occasum reddes ei;  
\verse ipsum enim est solum, quo operitur, indumentum carnis eius, nec habet aliud, in quo dormiat; si clamaverit ad me, exaudiam eum, quia misericors sum. 
\verse Deo non detrahes et principi populi tui non maledices. 
\verse Abundantiam areae tuae et torcularis tui non tardabis reddere. Primogenitum filiorum tuorum dabis mihi. 
\verse De bobus quoque et ovibus similiter facies: septem diebus sit cum matre sua, die octavo reddes illum mihi. 
\verse Viri sancti eritis mihi; carnem animalis in agro dilacerati non comedetis, sed proicietis canibus. 
\end{biblechapter}

\begin{biblechapter}  
\verse Non suscipies famam falsam nec iunges manum tuam cum impio, ut dicas falsum testimonium. 
\verse Non sequeris turbam ad faciendum malum; nec in iudicio plurimorum acquiesces sententiae, ut a vero devies. 
\verse Pauperis quoque non misereberis in iudicio. 
\verse Si occurreris bovi inimici tui aut asino erranti, reduc ad eum. 
\verse Si videris asinum odientis te iacere sub onere suo, non pertransibis, sed sublevabis cum eo. 
\verse Non pervertes iudicium pauperis in lite eius. 
\verse Mendacium fugies. Insontem et iustum non occides, quia aversor impium. 
\verse Nec accipies munera, quae excaecant etiam prudentes et subvertunt verba iustorum. 
\verse Peregrinum non opprimes; scitis enim advenarum animas, quia et ipsi peregrini fuistis in terra Aegypti. 
\verse Sex annis seminabis terram tuam et congregabis fruges eius. 
\verse Anno autem septimo dimittes eam et requiescere facies, ut comedant pauperes populi tui; et quidquid reliquum fuerit, edant bestiae agri. Ita facies in vinea et in oliveto tuo. 
\verse Sex diebus operaberis; septima die cessabis, ut requiescat bos et asinus tuus, et refrigeretur filius ancillae tuae et advena. 
\verse Omnia, quae dixi vobis, custodite, et nomen externorum deorum non invocabitis, neque audietur ex ore tuo. 
\verse Tribus vicibus per singulos annos mihi festa celebrabitis. 
\verse Sollemnitatem Azymorum custodies: septem diebus comedes azyma, sicut praecepi tibi, tempore statuto mensis Abib, quando egressus es de Aegypto. Non apparebis in conspectu meo vacuus. 
\verse Et sollemnitatem Messis primitivorum operis tui, quaecumque seminaveris in agro; sollemnitatem quoque Collectae in exitu anni, quando congregaveris omnes fruges tuas de agro. 
\verse Ter in anno apparebit omne masculinum tuum coram Domino Deo. 
\verse Non immolabis super fermento sanguinem victimae meae, nec remanebit adeps sollemnitatis meae usque mane. 
\verse Primitias primarum frugum terrae tuae deferes in domum Domini Dei tui. Non coques haedum in lacte matris suae. 
\verse Ecce ego mittam angelum, qui praecedat te et custodiat in via et introducat ad locum, quem paravi. 
\verse Observa eum et audi vocem eius nec contemnendum putes; quia non dimittet, cum peccaveritis, quia est nomen meum in illo. 
\verse Quod si audieris vocem eius et feceris omnia, quae loquor, inimicus ero inimicis tuis et affligam affligentes te. 
\verse Praecedet enim te angelus meus et introducet te ad Amorraeum et Hetthaeum et Pherezaeum Chananaeumque et Hevaeum et Iebusaeum, quos ego conteram. 
\verse Non adorabis deos eorum nec coles eos; non facies secundum opera eorum, sed destrues eos et confringes lapides eorum. 
\verse Servietisque Domino Deo vestro, ut benedicam panibus tuis et aquis et auferam infirmitatem de medio tui. 
\verse Non erit abortiens nec sterilis in terra tua; numerum dierum tuorum implebo. 
\verse Terrorem meum mittam in praecursum tuum et perturbabo omnem populum, ad quem ingre dieris; cunctorumque inimicorum tuorum coram te terga vertam 
\verse emittens crabrones prius, qui fugabunt Hevaeum et Chananaeum et Hetthaeum, antequam introeas. 
\verse Non eiciam eos a facie tua anno uno, ne terra in solitudinem redigatur, et multiplicentur contra te bestiae agri. 
\verse Paulatim expellam eos de conspectu tuo, donec augearis et possideas terram. 
\verse Ponam autem terminos tuos a mari Rubro usque ad mare Palaestinorum et a deserto usque ad Fluvium. Tradam manibus vestris habitatores terrae et eiciam eos de conspectu vestro. 
\verse Non inibis cum eis foedus nec cum diis eorum. 
\verse Non habitent in terra tua, ne peccare te faciant in me, si servieris diis eorum; quod tibi certo erit in scandalum". 
\end{biblechapter}

\begin{biblechapter}  
\verse Moysi quoque dixit: “Ascende ad Dominum, tu et Aaron, Nadab et Abiu et septuaginta senes ex Israel, et adorabitis procul. 
\verse Solusque Moyses ascendet ad Dominum, et illi non appropinquabunt, nec populus ascendet cum eo". 
\verse Venit ergo Moyses et narravit plebi omnia verba Domini atque iudicia; responditque omnis populus una voce: “Omnia verba Domini, quae locutus est, faciemus". 
\verse Scripsit autem Moyses universos sermones Domini; et mane consurgens aedificavit altare ad radices montis et duodecim lapides per duodecim tribus Israel. 
\verse Misitque iuvenes de filiis Israel, et obtulerunt holocausta; immolaveruntque victimas pacificas Domino vitulos. 
\verse Tulit itaque Moyses dimidiam partem sanguinis et misit in crateras; partem autem residuam respersit super altare. 
\verse Assumensque volumen foederis legit, audiente populo, qui dixerunt: “Omnia, quae locutus est Dominus, faciemus et erimus oboedientes". 
\verse Ille vero sumptum sanguinem respersit in populum et ait: “Hic est sanguis foederis, quod pepigit Dominus vobiscum super cunctis sermonibus his". 
\verse Ascenderuntque Moyses et Aaron, Nadab et Abiu et septuaginta de senioribus Israel. 
\verse Et viderunt Deum Israel, et sub pedibus eius quasi opus lapidis sapphirini et quasi ipsum caelum, cum serenum est. 
\verse Nec in electos filiorum Israel misit manum suam; videruntque Deum et comederunt ac biberunt. 
\verse Dixit autem Dominus ad Moysen: “Ascende ad me in montem et esto ibi; daboque tibi tabulas lapideas et legem ac mandata, quae scripsi, ut doceas eos". 
\verse Surrexerunt Moyses et Iosue minister eius; ascendensque Moyses in montem Dei  
\verse senioribus ait: “Exspectate hic, donec revertamur ad vos. Habetis Aaron et Hur vobiscum; si quid natum fuerit quaestionis, referetis ad eos". 
\verse Cumque ascendisset Moyses in montem, operuit nubes montem; 
\verse et habitavit gloria Domini super Sinai tegens illum nube sex diebus; septimo autem die vocavit eum de medio caliginis. 
\verse Erat autem species gloriae Domini quasi ignis ardens super verticem montis in conspectu filiorum Israel. 
\verse Ingressusque Moyses medium nebulae ascendit in montem; et fuit ibi quadraginta diebus et quadraginta noctibus. 
\end{biblechapter}

\begin{biblechapter}  
\verse Locutusque est Dominus ad Moysen dicens: 
\verse “Loquere filiis Israel, ut tollant mihi donaria; ab omni homine, qui offert ultroneus, accipietis ea.  
\verse Haec sunt autem, quae accipere debetis: aurum et argentum et aes, 
\verse hyacinthum et purpuram coccumque et byssum, pilos caprarum 
\verse et pelles arietum rubricatas pellesque delphini et ligna acaciae, 
\verse oleum ad luminaria concinnanda, aromata in unguentum et in thymiama boni odoris, 
\verse lapides onychinos et gemmas ad ornandum ephod ac pectorale. 
\verse Facientque mihi sanctuarium, et habitabo in medio eorum. 
\verse Iuxta omnem similitudinem habitaculi, quam ostendam tibi, et omnium vasorum in cultum eius: sicque facietis illud. 
\verse Arcam de lignis acaciae compingent; cuius longitudo habeat duos semis cubitos, latitudo cubitum et dimidium, altitudo cubitum similiter ac semissem. 
\verse Et deaurabis eam auro mundissimo intus et foris; faciesque supra coronam auream per circuitum 
\verse et conflabis ei quattuor circulos aureos, quos pones in quattuor arcae pedibus: duo circuli sint in latere uno et duo in altero.  
\verse Facies quoque vectes de lignis acaciae et operies eos auro; 
\verse inducesque per circulos, qui sunt in arcae lateribus, ut portetur in eis; 
\verse qui semper erunt in circulis nec umquam extrahentur ab eis. 
\verse Ponesque in arcam testimonium, quod dabo tibi. 
\verse Facies et propitiatorium de auro mundissimo; duos cubitos et dimidium tenebit longitudo eius, et cubitum ac semissem latitudo. 
\verse Duos quoque cherubim aureos et productiles facies ex utraque parte propitiatorii, 
\verse cherub unus sit in latere uno et alter in altero; ex propitiatorio facies cherubim in utraque parte eius. 
\verse Expandent alas sursum et operient alis suis propitiatorium; respicientque se mutuo, versis vultibus in propitiatorium,  
\verse quo operienda est arca, in qua pones testimonium, quod dabo tibi. 
\verse Et conveniam te ibi et loquar ad te supra propitiatorium de medio duorum cherubim, qui erunt super arcam testimonii, cuncta, quae mandabo per te filiis Israel. 
\verse Facies et mensam de lignis acaciae habentem duos cubitos longitudinis et in latitudine cubitum et in altitudine cubitum ac semissem. 
\verse Et inaurabis eam auro purissimo; faciesque illi coronam auream per circuitum. 
\verse Facies quoque ei limbum altum quattuor digitis per circuitum et super illum coronam auream. 
\verse Quattuor quoque circulos aureos praeparabis et pones eos in quattuor angulis eiusdem mensae per singulos pedes. 
\verse Iuxta limbum erunt circuli aurei, ut mittantur vectes per eos, et possit mensa portari. 
\verse Ipsosque vectes facies de lignis acaciae et circumdabis auro, et per ipsos subvehitur mensa. 
\verse Parabis et acetabula ac phialas, vasa et cyathos, in quibus offerenda sunt libamina, ex auro purissimo. 
\verse Et pones super mensam panes propositionis in conspectu meo semper. 
\verse Facies et candelabrum ductile de auro mundissimo: basis et hastile eius, scyphi et sphaerulae ac flores in unum efformentur. 
\verse Sex calami egredientur de lateribus, tres ex uno latere et tres ex altero. 
\verse Tres scyphi quasi in nucis modum in calamo uno sphaerulaeque simul et flores; et tres similiter scyphi instar nucis in calamo altero sphaerulaeque simul et flores: hoc erit opus sex calamorum, qui producendi sunt de hastili. 
\verse In ipso autem hastili candelabri erunt quattuor scyphi in nucis modum sphaerulaeque et flores. 
\verse Singulae sphaerulae sub binis calamis per tria loca, qui simul sex fiunt, procedentes de hastili uno. 
\verse Sphaerulae igitur et calami unum cum ipso erunt, totum ductile de auro purissimo. 
\verse Facies et lucernas septem et pones eas super candelabrum, ut luceant in locum ex adverso. 
\verse Emunctoria quoque et vasa, in quibus emuncta condantur, fient de auro purissimo.  
\verse Omne pondus candelabri cum universis vasis suis habebit talentum auri purissimi. 
\verse Inspice et fac secundum exemplar, quod tibi in monte monstratum est. 
\end{biblechapter}

\begin{biblechapter}  
\verse Habitaculum vero ita facies: decem cortinas de bysso retorta et hyacintho ac purpura coccoque cum cherubim opere polymito facies. 
\verse Longitudo cortinae unius habebit viginti octo cubitos, latitudo quattuor cubitorum erit. Unius mensurae fient universae cortinae. 
\verse Quinque cortinae sibi iungentur mutuo, et aliae quinque nexu simili cohaerebunt. 
\verse Ansulas hyacinthinas in latere facies cortinae unius in extremitate iuncturae et similiter facies in latere cortinae extremae in iunctura altera. 
\verse Quinquaginta ansulas facies in cortina una et quinquaginta ansulas facies in summitate cortinae, quae est in iunctura altera, ita insertas, ut ansa contra ansam veniat. 
\verse Facies et quinquaginta fibulas aureas, quibus cortinarum vela iungenda sunt, ut unum habitaculum fiat. 
\verse Facies et saga cilicina undecim pro tabernaculo super habitaculum. 
\verse Longitudo sagi unius habebit triginta cubitos et latitudo quattuor; aequa erit mensura sagorum omnium. 
\verse E quibus quinque iunges seorsum et sex sibi mutuo copulabis, ita ut sextum sagum in fronte tecti duplices. 
\verse Facies et quinquaginta ansas in ora sagi ultimi iuncturae unius et quinquaginta ansas in ora sagi iuncturae alterius. 
\verse Facies et quinquaginta fibulas aeneas, quibus iungantur ansae, ut unum ex omnibus tabernaculum fiat. 
\verse Quod autem superfuerit in sagis, quae parantur tecto, id est unum sagum, quod amplius est, ex medietate eius operies posteriora habitaculi; 
\verse et cubitus ex una parte pendebit, et alter ex altera, qui plus est in longitudine sagorum tabernaculi utrumque latus habitaculi protegens. 
\verse Facies et operimentum aliud pro tabernaculo de pellibus arietum rubricatis et super hoc rursum aliud operimentum de pellibus delphini. 
\verse Facies et tabulas stantes habitaculi de lignis acaciae, 
\verse quae singulae denos cubitos in longitudine habeant et in latitudine singulos ac semissem.  
\verse In tabula una duo pedes fient, quibus tabula alteri tabulae conectatur; atque in hunc modum cunctae tabulae habitaculi parabuntur. 
\verse Quarum viginti erunt in latere meridiano, quod vergit ad austrum; 
\verse quibus quadraginta bases argenteas fundes, ut binae bases singulis pedibus singularum tabularum subiciantur. 
\verse In latere quoque secundo habitaculi, quod vergit ad aquilonem, viginti tabulae erunt, 
\verse quadraginta habentes bases argenteas; binae bases singulis tabulis supponentur. 
\verse Ad occidentalem vero plagam in tergo habitaculi facies sex tabulas; 
\verse et rursum alias duas, quae in angulis erigantur, post tergum habitaculi. 
\verse Eruntque geminae a deorsum usque sursum in compaginem unam; ita erit duabus istis, pro duabus angulis erunt. 
\verse Et erunt simul tabulae octo, bases earum argenteae sedecim, duabus basibus per unam tabulam supputatis. 
\verse Facies et vectes de lignis acaciae, quinque ad continendas tabulas in uno latere habitaculi 
\verse et quinque alios in altero et eiusdem numeri in tergo ad occidentalem plagam; 
\verse vectis autem medius transibit per medias tabulas a summo usque ad summum. 
\verse Ipsasque tabulas deaurabis et fundes eis anulos aureos, per quos vectes tabulata contineant, quos operies laminis aureis. 
\verse Et eriges habitaculum iuxta exemplar, quod tibi in monte monstratum est. 
\verse Facies et velum de hyacintho et purpura coccoque et bysso retorta, opere polymito, cum cherubim intextis. 
\verse Quod appendes in quattuor columnis de lignis acaciae, quae ipsae quidem deauratae erunt et habebunt uncos aureos, sed bases argenteas. 
\verse Inseres autem velum subter fibulas, intra quod pones arcam testimonii et quo sanctum et sanctum sanctorum dividentur. 
\verse Pones et propitiatorium super arcam testimonii in sancto sanctorum 
\verse mensamque extra velum et contra mensam candelabrum in latere habitaculi meridiano; mensa enim stabit in parte aquilonis. 
\verse Facies et velum in introitu tabernaculi de hyacintho et purpura coccoque et bysso retorta opere plumarii. 
\verse Et quinque columnas deaurabis lignorum acaciae, ante quas ducetur velum, quarum erunt unci aurei et bases aeneae. 
\end{biblechapter}

\begin{biblechapter}  
\verse Facies et altare de lignis acaciae, quod habebit quinque cubitos in longitudine et totidem in latitudine, id est quadrum, et tres cubitos in altitudine. 
\verse Cornua autem per quattuor angulos ex ipso erunt, et operies illud aere. 
\verse Faciesque in usus eius lebetes ad suscipiendos cineres et vatilla et pateras atque fuscinulas et ignium receptacula; omnia vasa ex aere fabricabis. 
\verse Craticulamque facies ei in modum retis aeneam, per cuius quattuor angulos erunt quattuor anuli aenei, 
\verse et pones eam subter marginem altaris; eritque craticula usque ad altaris medium. 
\verse Facies et vectes altaris de lignis acaciae duos, quos operies laminis aeneis, 
\verse et induces per anulos; eruntque ex utroque latere altaris ad portandum. 
\verse Cavum ex tabulis facies illud; sicut tibi in monte monstratum est, sic facient. 
\verse Facies et atrium habitaculi, in cuius plaga australi contra meridiem erunt tentoria de bysso retorta: centum cubitos unum latus tenebit in longitudine  
\verse et columnas viginti et bases totidem aeneas et uncos columnarum anulosque earum argenteos. 
\verse Similiter in latere aquilonis: per longum erunt tentoria centum cubitorum, columnae viginti et bases aeneae eiusdem numeri et unci columnarum anulique earum argentei. 
\verse In latitudine vero atrii, quae respicit ad occidentem, erunt tentoria per quinquaginta cubitos et columnae decem basesque totidem. 
\verse In ea quoque atrii latitudine, quae respicit ad orientem, quinquaginta cubiti erunt, 
\verse in quibus quindecim cubitorum tentoria lateri uno deputabuntur columnaeque tres et bases totidem; 
\verse et in latere altero erunt tentoria, cubitos obtinentia quindecim, columnae tres et bases totidem. 
\verse In introitu vero atrii fiet velum cubitorum viginti, ex hyacintho et purpura coccoque et bysso retorta opere plumarii; columnas habebit quattuor cum basibus totidem. 
\verse Omnes columnae atrii per circuitum cinctae erunt anulis argenteis, et unci earum erunt argentei et bases aeneae. 
\verse In longitudine occupabit atrium cubitos centum, in latitudine quinquaginta, altitudo quinque cubitorum erit; fietque de bysso retorta, et habebit bases aeneas. 
\verse Cuncta vasa habitaculi in omnes usus eius et omnes paxillos eius et omnes paxillos atrii ex aere facies. 
\verse Praecipe filiis Israel, ut afferant tibi oleum de arboribus olivarum purissimum piloque contusum, ut ardeat lucerna semper 
\verse in tabernaculo conventus, extra velum, quod oppansum est testimonio. Et parabunt eam Aaron et filii eius, ut a vespere usque mane luceat coram Domino. Perpetuus erit cultus per successiones eorum a filiis Israel. 
\end{biblechapter}

\begin{biblechapter}  
\verse Applica quoque ad te Aaron fratrem tuum cum filiis suis de medio filiorum Israel, ut sacerdotio fungantur mihi: Aaron, Nadab et Abiu, Eleazar et Ithamar. 
\verse Faciesque vestes sanctas Aaron fratri tuo in gloriam et decorem; 
\verse et loqueris cunctis sapientibus corde, quos replevi spiritu prudentiae, ut faciant vestes Aaron, in quibus sanctificatus ministret mihi. 
\verse Haec autem erunt vestimenta, quae facient: pectorale et ephod, tunicam et subuculam textam, tiaram et balteum. Facient vestimenta sancta Aaron fratri tuo et filiis eius, ut sacerdotio fungantur mihi; 
\verse accipientque aurum et hyacinthum et purpuram coccumque et byssum. 
\verse Facient autem ephod de auro et hyacintho ac purpura coccoque bysso retorta opere polymito. 
\verse Duas fascias umerales habebit et in utroque latere summitatum suarum copulabitur cum eis. 
\verse Et balteus super ephod ad constringendum, eiusdem operis et unum cum eo, erit ex auro et hyacintho et purpura coccoque et bysso retorta. 
\verse Sumesque duos lapides onychinos et sculpes in eis nomina filiorum Israel: 
\verse sex nomina in lapide uno et sex reliqua in altero, iuxta ordinem nativitatis eorum. 
\verse Opere sculptoris et caelatura gemmarii sculpes eos nominibus filiorum Israel, inclusos textura aurea; 
\verse et pones duos lapides super fascias umerales ephod, lapides memorialis filiorum Israel. Portabitque Aaron nomina eorum coram Domino super utrumque umerum ob recordationem. 
\verse Facies ergo margines textas ex auro  
\verse et duas catenulas ex auro purissimo quasi funiculos opus tortile et inseres catenulas tortas marginibus. 
\verse Pectorale quoque iudicii facies opere polymito, iuxta texturam ephod, ex auro, hyacintho et purpura coccoque et bysso retorta. 
\verse Quadrangulum erit et duplex; mensuram palmi habebit tam in longitudine quam in latitudine. 
\verse Ponesque in eo quattuor ordines lapidum: in primo versu erit lapis sardius et topazius et smaragdus; 
\verse in secundo carbunculus, sapphirus et iaspis;  
\verse in tertio hyacinthus, achates et amethystus; 
\verse in quarto chrysolithus, onychinus et beryllus. Inclusi auro erunt per ordines suos. 
\verse Habebuntque nomina filiorum Israel: duodecim nominibus caelabuntur, singuli lapides nominibus singulorum per duodecim tribus. 
\verse Facies in pectorali catenas quasi funiculos, opus tortile, ex auro purissimo; 
\verse et duos anulos aureos, quos pones in utraque pectoralis summitate; 
\verse catenasque aureas iunges anulis, qui sunt in marginalibus eius; 
\verse et ipsarum catenarum extrema duobus copulabis marginibus in fasciis umeralibus ephod in parte eius anteriore.  
\verse Facies et duos anulos aureos, quos pones in summitatibus pectoralis in ora interiore, quae respicit ephod. 
\verse Necnon et alios duos anulos aureos, qui ponendi sunt in utraque fascia umerali ephod deorsum, versus partem anteriorem eius iuxta iuncturam eius supra balteum ephod, 
\verse et stringatur pectorale anulis suis cum anulis ephod vitta hyacinthina, ut maneat supra balteum ephod, et a se invicem pectorale et ephod nequeant separari. 
\verse Portabitque Aaron nomina filiorum Israel in pectorali iudicii super cor suum, quando ingredietur sanctuarium: memoriale coram Domino in aeternum. 
\verse Pones autem in pectorali iudicii Urim et Tummim, quae erunt super cor Aaron, quando ingredietur coram Domino; et gestabit iudicium filiorum Israel super cor suum in conspectu Domini semper. 
\verse Facies et pallium ephod totum hyacinthinum, 
\verse in cuius medio supra erit capitium et ora per gyrum eius textilis, sicut in capitio loricae, ne rumpatur. 
\verse Deorsum vero, ad pedes eiusdem pallii per circuitum, quasi mala punica facies ex hyacintho et purpura et cocco, mixtis in medio tintinnabulis aureis;  
\verse ita ut sit tintinnabulum aureum inter singula mala punica. 
\verse Et vestietur eo Aaron in officio ministerii, ut audiatur sonitus, quando ingreditur et egreditur sanctuarium in conspectu Domini, et non moriatur. 
\verse Facies et laminam de auro purissimo, in qua sculpes opere caelatoris: "Sanctum Domino". 
\verse Ligabisque eam vitta hyacinthina, et erit super tiaram  
\verse super frontem Aaron. Portabitque Aaron iniquitatem contra sancta, quae sanctificabunt filii Israel in cunctis muneribus et donariis suis. Eritque lamina semper in fronte eius, ut placatus eis sit Dominus. 
\verse Texesque tunicam bysso et tiaram byssinam facies et balteum opere plumarii. 
\verse Porro filiis Aaron tunicas lineas parabis et balteos ac mitras in gloriam et decorem; 
\verse vestiesque his omnibus Aaron fratrem tuum et filios eius cum eo. Et unges eos et implebis manus eorum sanctificabisque illos, ut sacerdotio fungantur mihi. 
\verse Facies eis et feminalia linea, ut operiant carnem turpitudinis suae a renibus usque ad femora; 
\verse et utentur eis Aaron et filii eius, quando ingredientur tabernaculum conventus, vel quando appropinquant ad altare, ut ministrent in sanctuario, ne iniquitatis rei moriantur: legitimum sempiternum erit Aaron et semini eius post eum. 
\end{biblechapter}

\begin{biblechapter}  
\verse Sed et hoc facies eis, ut mihi in sacerdotio consecrentur: tolle vitulum unum de armento et arietes duos immaculatos 
\verse panesque azymos et crustulas absque fermento, quae conspersa sint oleo, lagana quoque azyma oleo lita; de simila triticea cuncta facies 
\verse et posita in canistro offeres, vitulum quoque et duos arietes. 
\verse Aaron ac filios eius applicabis ad ostium tabernaculi conventus. Cumque laveris patrem cum filiis suis aqua, 
\verse indues Aaron vestimentis suis, id est subucula et tunica ephod et ephod et pectorali, quod constringes ei cingulo ephod; 
\verse et pones tiaram in capite eius et diadema sanctum super tiaram 
\verse et oleum unctionis fundes super caput eius; atque hoc ritu consecrabitur. 
\verse Filios quoque illius applicabis et indues tunicis lineis cingesque balteo 
\verse et impones eis mitras; eruntque sacerdotes mihi iure perpetuo. Postquam impleveris manus Aaron et filiorum eius, 
\verse applicabis et vitulum coram tabernaculo conventus; imponentque Aaron et filii eius manus super caput illius, 
\verse et mactabis eum in conspectu Domini, iuxta ostium tabernaculi conventus.  
\verse Sumptumque de sanguine vituli, pones super cornua altaris digito tuo, reliquum autem sanguinem fundes iuxta basim eius. 
\verse Sumes et adipem totum, qui operit intestina, et reticulum iecoris ac duos renes et adipem, qui super eos est, et offeres comburens super altare; 
\verse carnes vero vituli et corium et fimum combures foris extra castra, eo quod pro peccato sit. 
\verse Unum quoque arietem sumes, super cuius caput ponent Aaron et filii eius manus; 
\verse quem cum mactaveris, tolles sanguinem eius et fundes super altare per circuitum. 
\verse Ipsum autem arietem secabis in frusta lotaque intestina eius ac pedes pones super concisas carnes et super caput illius. 
\verse Et adolebis totum arietem super altare: holocaustum est Domino, odor suavissimus, incensum est Domino. 
\verse Tolles quoque arietem alterum, super cuius caput Aaron et filii eius ponent manus; 
\verse quem cum immolaveris, sumes de sanguine ipsius et pones super extremum auriculae dextrae Aaron et filiorum eius et super pollices manus eorum ac pedis dextri; fundesque sanguinem super altare per circuitum. 
\verse Cumque tuleris de sanguine, qui est super altare, et de oleo unctionis, asperges Aaron et vestes eius, filios et vestimenta eorum cum ipso. Et sanctus erit ipse et vestimenta eius et filii eius et vestimenta eorum cum ipso. 
\verse Tollesque adipem de ariete et caudam et arvinam, quae operit intestina, ac reticulum iecoris et duos renes atque adipem, qui super eos est, armumque dextrum, eo quod sit aries consecrationis, 
\verse tortamque panis unam, crustulam unam conspersam oleo, laganum unum de canistro azymorum, quod positum est in conspectu Domini; 
\verse ponesque omnia super manus Aaron et filiorum eius, ut agitent ea coram Domino. 
\verse Suscipiesque universa de manibus eorum et incendes in altari super holocausto in odorem suavissimum in conspectu Domini; quia incensum est Domino. 
\verse Sumes quoque pectusculum de ariete, quo initiatus est Aaron, elevabisque illud coram Domino; et cedet in partem tuam.  
\verse Sanctificabisque pectusculum elevatum et armum oblatum, quem de ariete separasti, 
\verse quo initiatus est Aaron et filii eius; cedentque in partem Aaron et filiorum eius iure perpetuo a filiis Israel; quia oblatio est et oblatio erit a filiis Israel de victimis eorum pacificis, oblatio eorum Domino. 
\verse Vestem autem sanctam, qua utetur Aaron, habebunt filii eius post eum, ut ungantur in ea, et impleantur in ea manus eorum. 
\verse Septem diebus utetur illa, qui pontifex pro eo fuerit constitutus de filiis eius, qui ingredietur tabernaculum conventus, ut ministret in sanctuario. 
\verse Arietem autem consecrationis tolles et coques carnes eius in loco sancto.  
\verse Et vescetur Aaron et filii eius carnibus arietis et panibus, qui sunt in canistro, in vestibulo tabernaculi conventus. 
\verse Et comedent ea, quibus expiatio facta fuerit ad implendum manus eorum, ad sanctificandum eos. Alienigena non vescetur ex eis, quia sancta sunt. 
\verse Quod si remanserit de carnibus consecrationis sive de panibus usque mane, combures reliquias igni; non comedentur, quia sancta sunt. 
\verse Omnia, quae praecepi tibi, facies super Aaron et filiis eius. Septem diebus consecrabis manus eorum 
\verse et vitulum pro peccato offeres per singulos dies ad expiandum. Mundabisque altare expians illud et unges illud in sanctificationem. 
\verse Septem diebus expiabis altare et sanctificabis; et erit sanctum sanctorum: omnis, qui tetigerit illud, sanctificabitur. 
\verse Hoc est quod facies in altari: agnos anniculos duos per singulos dies iugiter, 
\verse unum agnum mane et alterum vespere; 
\verse decimam partem similae conspersae oleo tunso, quod habeat mensuram quartam partem hin, et vinum ad libandum eiusdem mensurae in agno uno. 
\verse Alterum vero agnum offeres ad vesperam iuxta ritum matutinae oblationis et libationis in odorem suavitatis, incensum Domino, 
\verse holocaustum perpetuum in generationes vestras, ad ostium tabernaculi conventus coram Domino, ubi conveniam vos, ut loquar ad te. 
\verse Ibi conveniam filios Israel, et sanctificabitur locus in gloria mea. 
\verse Sanctificabo et tabernaculum conventus cum altari et Aaron cum filiis eius, ut sacerdotio fungantur mihi. 
\verse Et habitabo in medio filiorum Israel eroque eis Deus; 
\verse et scient quia ego Dominus Deus eorum, qui eduxi eos de terra Aegypti, ut manerem inter illos: ego Dominus Deus ipsorum. 
\end{biblechapter}

\begin{biblechapter}  
\verse Facies quoque altare ad adolendum thymiama de lignis acaciae 
\verse habens cubitum longitudinis et alterum latitudinis, id est quadrangulum, et duos cubitos in altitudine; cornua ex ipso procedent. 
\verse Vestiesque illud auro purissimo, tam craticulam eius quam parietes per circuitum et cornua. Faciesque ei coronam aureolam per gyrum 
\verse et duos anulos aureos sub corona in duobus lateribus, ut mittantur in eos vectes, et altare portetur. 
\verse Ipsos quoque vectes facies de lignis acaciae et inaurabis. 
\verse Ponesque altare contra velum, quod ante arcam pendet testimonii, coram propitiatorio, quo tegitur testimonium, ubi conveniam ad te. 
\verse Et adolebit incensum super eo Aaron suave fragrans mane. Quando componet lucernas, incendet illud; 
\verse et quando collocabit eas ad vesperum, uret thymiama sempiternum coram Domino in generationes vestras. 
\verse Non offeretis super eo thymiama compositionis alterius nec holocaustum nec oblationem, nec libabitis libamina. 
\verse Et expiabit Aaron super cornua eius semel per annum in sanguine sacrificii pro peccato; et placabit super eo in generationibus vestris: sanctum sanctorum erit Domino". 
\verse Locutusque est Dominus ad Moysen dicens: 
\verse “Quando tuleris summam filiorum Israel iuxta numerum, dabunt singuli pretium expiationis pro animabus suis Domino; et non erit plaga in eis, cum fuerint recensiti. 
\verse Hoc autem dabit omnis, qui transit ad censum, dimidium sicli iuxta mensuram sanctuarii ­ siclus viginti obolos habet ­; media pars sicli offeretur Domino. 
\verse Qui habetur in numero a viginti annis et supra, dabit pretium; 
\verse dives non addet ad medium sicli, et pauper nihil minuet, quando dabitis oblationem Domino in expiationem animarum vestrarum. 
\verse Susceptamque expiationis pecuniam, quae collata est a filiis Israel, trades in usus tabernaculi conventus, ut sit monumentum eorum coram Domino et propitietur animabus illorum". 
\verse Locutusque est Dominus ad Moysen dicens: 
\verse “Facies et labrum aeneum cum basi aenea ad lavandum; ponesque illud inter tabernaculum conventus et altare. Et, missa aqua, 
\verse lavabunt in eo Aaron et filii eius manus suas ac pedes.  
\verse Quando ingressuri sunt tabernaculum conventus, lavabunt se aqua, ne moriantur; vel quando accessuri sunt ad altare, ut ministrent, ut adoleant victimam Domino. 
\verse Et lavabunt manus et pedes, ne moriantur: legitimum sempiternum erit, ipsi et semini eius per successiones". 
\verse Locutusque est Dominus ad Moysen 
\verse dicens: “Sume tibi aromata prima myrrhae electae quingentos siclos et cinnamomi boni odoris medium, id est ducentos quinquaginta siclos, calami suave olentis similiter ducentos quinquaginta, 
\verse casiae autem quingentos siclos, in pondere sanctuarii, olei de olivetis mensuram hin. 
\verse Faciesque unctionis oleum sanctum, unguentum compositum opere unguentarii; unctionis oleum sanctum erit. 
\verse Et unges ex eo tabernaculum conventus et arcam testamenti 
\verse mensamque cum vasis suis, candelabrum et utensilia eius, altaria thymiamatis 
\verse et holocausti et universam supellectilem, quae ad cultum eorum pertinet, et labrum cum basi sua.  
\verse Sanctificabisque omnia, et erunt sancta sanctorum: qui tetigerit ea, sanctificabitur. 
\verse Aaron et filios eius unges sanctificabisque eos, ut sacerdotio fungantur mihi. 
\verse Filiis quoque Israel dices: Hoc oleum unctionis sanctum erit mihi in generationes vestras. 
\verse Caro hominis non ungetur ex eo, et iuxta compositionem eius non facietis aliud, quia sanctum est et sanctum erit vobis.  
\verse Homo quicumque tale composuerit et dederit ex eo super alienum, exterminabitur de populo suo". 
\verse Dixitque Dominus ad Moysen: “Sume tibi aromata, stacten et onycha, galbanum boni odoris et tus lucidissimum; aequalis ponderis erunt omnia. 
\verse Faciesque thymiama compositum opere unguentarii, sale conditum et purum et sanctum. 
\verse Cumque in tenuissimum pulverem ex parte contuderis, pones ex eo coram testimonio in tabernaculo conventus, in quo conveniam ad te: sanctum sanctorum erit vobis thymiama. 
\verse Talem compositionem non facietis in usus vestros, quia tibi sanctum erit pro Domino; 
\verse homo quicumque fecerit simile, ut odore illius perfruatur, peribit de populis suis". 
\end{biblechapter}

\begin{biblechapter}  
\verse Locutusque est Dominus ad Moysen dicens: 
\verse “Ecce vocavi ex nomine Beseleel filium Uri filii Hur de tribu Iudae 
\verse et implevi eum spiritu Dei, sapientia et intellegentia et scientia in omni opere 
\verse ad excogitandum, quidquid fabrefieri potest ex auro et argento et aere, 
\verse ad scindendum et includendum gemmas et ad sculpendum ligna, ad faciendum omne opus; 
\verse dedique ei socium Ooliab filium Achisamech de tribu Dan et in corde omnis eruditi posui sapientiam, ut faciant cuncta, quae praecepi tibi: 
\verse tabernaculum conventus et arcam testimonii et propitiatorium, quod super eam est, et cuncta vasa tabernaculi 
\verse mensamque et vasa eius, candelabrum purissimum cum vasis suis et altaria thymiamatis 
\verse et holocausti et omnia vasa eorum, labrum cum basi sua 
\verse et vestes textas et vestes sanctas Aaron sacerdoti et vestes filiorum eius, ut fungantur officio suo in sacris, 
\verse oleum unctionis et thymiama aromatum in sanctuario: omnia, quae praecepi tibi, facient". 
\verse Et locutus est Dominus ad Moysen dicens: 
\verse “Loquere filiis Israel et dices ad eos: Videte ut sabbatum meum custodiatis, quia signum est inter me et vos in generationibus vestris, ut sciatis quia ego Dominus, qui sanctifico vos. 
\verse Custodite sabbatum, sanctum est enim vobis. Qui polluerit illud, morte morietur; qui fecerit in eo opus, peribit anima illius de medio populi sui.  
\verse Sex diebus facietis opus; in die septimo sabbatum est, requies sancta Domino: omnis, qui fecerit opus in hac die, morietur. 
\verse Custodiant filii Israel sabbatum et celebrent illud in generationibus suis: pactum est sempiternum  
\verse inter me et filios Israel signumque perpetuum; sex enim diebus fecit Dominus caelum et terram et in septimo ab opere cessavit et respiravit". 
\verse Deditque Dominus Moysi, completis huiuscemodi sermonibus in monte Sinai, duas tabulas testimonii lapideas scriptas digito Dei. 
\end{biblechapter}

\begin{biblechapter}  
\verse Videns autem populus quod moram faceret descendendi de monte Moyses, congregatus ad Aaron dixit: “Surge, fac nobis deos, qui nos praecedant; Moysi enim, huic viro, qui nos eduxit de terra Aegypti, ignoramus quid acciderit". 
\verse Dixitque ad eos Aaron: “Tollite inaures aureas de uxorum filiorumque et filiarum vestrarum auribus et afferte ad me". 
\verse Fecitque omnis populus, quae iusserat, deferens inaures ad Aaron. 
\verse Quas cum ille accepisset, formavit stilo imaginem et fecit ex eis vitulum conflatilem. Dixeruntque: “Hi sunt dii tui, Israel, qui te eduxerunt de terra Aegypti!". 
\verse Quod cum vidisset Aaron, aedificavit altare coram eo et praeconis voce clamavit dicens: “Cras sollemnitas Domini est". 
\verse Surgen tesque mane altero die obtulerunt holocausta et hostias pacificas; et sedit populus manducare et bibere et surrexerunt ludere. 
\verse Locutus est autem Dominus ad Moysen: “Vade, descende; peccavit populus tuus, quem eduxisti de terra Aegypti. 
\verse Recesserunt cito de via, quam praecepi eis, feceruntque sibi vitulum conflatilem et adoraverunt atque immolantes ei hostias dixerunt: "Isti sunt dii tui, Israel, qui te eduxerunt de terra Aegypti!"". 
\verse Rursumque ait Dominus ad Moysen: “Cerno quod populus iste durae cervicis sit; 
\verse dimitte me, ut irascatur furor meus contra eos et deleam eos faciamque te in gentem magnam". 
\verse Moyses autem orabat Dominum Deum suum dicens: “Cur, Domine, irascitur furor tuus contra populum tuum, quem eduxisti de terra Aegypti in fortitudine magna et in manu robusta? 
\verse Ne, quaeso, dicant Aegyptii: "Callide eduxit eos, ut interficeret in montibus et deleret e terra". Quiescat ira tua, et esto placabilis super nequitia populi tui. 
\verse Recordare Abraham, Isaac et Israel servorum tuorum, quibus iurasti per temetipsum dicens: "Multiplicabo semen vestrum sicut stellas caeli; et universam terram hanc, de qua locutus sum, dabo semini vestro, et possidebitis eam semper"". 
\verse Placatusque est Dominus, ne faceret malum, quod locutus fuerat adversus populum suum. 
\verse Et reversus est Moyses de monte portans duas tabulas testimonii in manu sua scriptas ex utraque parte 
\verse et factas opere Dei; scriptura quoque Dei erat sculpta in tabulis. 
\verse Audiens autem Iosue tumultum populi vociferantis dixit ad Moysen: “Ululatus pugnae auditur in castris". 
\verse Qui respondit: “Non est clamor vincentium neque clamor fugientium, sed clamorem cantantium ego audio". 
\verse Cumque appropinquasset ad castra, vidit vitulum et choros; iratusque valde proiecit de manu tabulas et confregit eas ad radices montis. 
\verse Arripiensque vitulum, quem fecerant, combussit et contrivit usque ad pulverem, quem sparsit in aquam et dedit ex eo potum filiis Israel. 
\verse Dixitque ad Aaron: “Quid tibi fecit hic populus, ut induceres super eum peccatum maximum?". 
\verse Cui ille respondit: “Ne indignetur dominus meus; tu enim nosti populum istum, quod pronus sit ad malum. 
\verse Dixerunt mihi: "Fac nobis deos, qui nos praecedant; huic enim Moysi, qui nos eduxit de terra Aegypti, nescimus quid acciderit". 
\verse Quibus ego dixi: Quis vestrum habet aurum? Abstulerunt et dederunt mihi, et proieci illud in ignem; egressusque est hic vitulus". 
\verse Vidit ergo Moyses populum quod esset effrenatus; relaxaverat enim ei Aaron frenum in ludibrium hostium eorum. 
\verse Et stans in porta castrorum ait: “Si quis est Domini, iungatur mihi!". Congregatique sunt ad eum omnes filii Levi. 
\verse Quibus ait: “Haec dicit Dominus, Deus Israel: Ponat unusquisque gladium super femur suum. Ite et redite de porta usque ad portam per medium castrorum, et occidat unusquisque fratrem et amicum et proximum suum". 
\verse Fecerunt filii Levi iuxta sermonem Moysi; cecideruntque de populo in die illa quasi tria milia hominum. 
\verse Et ait Moyses: “Implestis manus vestras hodie Domino unusquisque in filio et in fratre suo, ut detur vobis benedictio". 
\verse Facto autem altero die, locutus est Moyses ad populum: “Peccastis peccatum maximum; ascendam ad Dominum, si quo modo quivero eum deprecari pro scelere vestro". 
\verse Reversusque ad Dominum ait: “Obsecro, peccavit populus iste peccatum maximum, feceruntque sibi deos aureos; aut dimitte eis hanc noxam  
\verse aut, si non facis, dele me de libro tuo, quem scripsisti". 
\verse Cui respondit Dominus: “Qui peccaverit mihi, delebo eum de libro meo. 
\verse Tu autem vade et duc populum istum, quo locutus sum tibi: angelus meus praecedet te; ego autem in die ultionis visitabo et hoc peccatum eorum". 
\verse Percussit ergo Dominus populum pro reatu vituli, quem fecerat Aaron. 
\end{biblechapter}

\begin{biblechapter}  
\verse Locutusque est Dominus ad Moysen: “Vade, ascende de loco isto, tu et populus tuus, quem eduxisti de terra Aegypti, in terram, quam iuravi Abraham, Isaac et Iacob dicens: Semini tuo dabo eam. 
\verse Et mittam praecursorem tui angelum et eiciam Chananaeum et Amorraeum et Hetthaeum et Pherezaeum et Hevaeum et Iebusaeum, 
\verse et intres in terram fluentem lacte et melle. Non enim ascendam tecum, quia populus durae cervicis es, ne forte disperdam te in via".  
\verse Audiens populus sermonem hunc pessimum luxit, et nullus ex more indutus est cultu suo. 
\verse Dixitque Dominus ad Moysen: “Loquere filiis Israel: Populus durae cervicis es; uno momento, si ascendam in medio tui, delebo te. Nunc autem depone ornatum tuum, ut sciam quid faciam tibi". 
\verse Deposuerunt ergo filii Israel ornatum suum a monte Horeb. 
\verse Moyses autem tollens tabernaculum tetendit ei extra castra procul; vocavitque nomen eius Tabernaculum conventus. Et omnis, qui quaerebat Dominum, egrediebatur ad tabernaculum conventus extra castra. 
\verse Cumque egrederetur Moyses ad tabernaculum, surgebat universa plebs, et stabat unusquisque in ostio papilionis sui; aspiciebantque tergum Moysi, donec ingrederetur tabernaculum. 
\verse Ingresso autem illo tabernaculum, descendebat columna nubis et stabat ad ostium; loquebaturque cum Moyse, 
\verse cernentibus universis quod columna nubis staret ad ostium tabernaculi. Stabantque ipsi et adorabant per fores tabernaculorum suorum. 
\verse Loquebatur autem Dominus ad Moysen facie ad faciem, sicut solet loqui homo ad amicum suum. Cumque ille reverteretur in castra, minister eius Iosue filius Nun puer non recedebat de medio tabernaculi. 
\verse Dixit autem Moyses ad Dominum: “Praecipis, ut educam populum istum, et non indicas mihi, quem missurus es mecum; cum dixeris: "Novi te ex nomine, et invenisti gratiam coram me". 
\verse Si ergo inveni gratiam in conspectu tuo, ostende mihi viam tuam, ut sciam te et inveniam gratiam ante oculos tuos; respice quia populus tuus est natio haec". 
\verse Dixitque Dominus: “Facies mea ibit, et requiem dabo tibi". 
\verse Et ait Moyses: “Si non tu ipse eas, ne educas nos de loco isto; 
\verse in quo enim scietur me et populum tuum invenisse gratiam in conspectu tuo, nisi ambulaveris nobiscum, ut glorificemur ego et populus tuus prae omnibus populis, qui habitant super terram?". 
\verse Dixitque Dominus ad Moysen: “Et verbum istud, quod locutus es, faciam; invenisti enim gratiam coram me, et teipsum novi ex nomine". 
\verse Qui ait: “Ostende mihi gloriam tuam". 
\verse Respondit: “Ego ostendam omne bonum tibi et vocabo in nomine Domini coram te; et miserebor, cui voluero, et clemens ero, in quem mihi placuerit". 
\verse Rursumque ait: “Non poteris videre faciem meam; non enim videbit me homo et vivet". 
\verse Et iterum: “Ecce, inquit, est locus apud me, stabis super petram; 
\verse cumque transibit gloria mea, ponam te in foramine petrae et protegam dextera mea, donec transeam; 
\verse tollamque manum meam, et videbis posteriora mea; faciem autem meam videre non poteris". 
\end{biblechapter}

\begin{biblechapter}  
\verse Dixitque Dominus ad Moysen: “Praecide tibi duas tabulas lapideas instar priorum, et scribam super eas verba, quae habuerunt tabulae, quas fregisti. 
\verse Esto paratus mane, ut ascendas statim in montem Sinai; stabisque mihi super verticem montis. 
\verse Nullus ascendat tecum, nec videatur quispiam per totum montem; oves quoque et boves non pascantur e contra". 
\verse Excidit ergo duas tabulas lapideas, quales antea fuerant; et de nocte consurgens ascendit in montem Sinai, sicut praeceperat ei Dominus, portans secum tabulas. 
\verse Cumque descendisset Dominus per nubem, stetit cum eo vocans in nomine Domini. 
\verse Et transiens coram eo clamavit: “Dominus, Dominus Deus, misericors et clemens, patiens et multae miserationis ac verax, 
\verse qui custodit misericordiam in milia, qui aufert iniquitatem et scelera atque peccata, nihil autem impunitum sinit, qui reddit iniquitatem patrum in filiis ac nepotibus in tertiam et quartam progeniem". 
\verse Festinusque Moyses curvatus est pronus in terram et adorans 
\verse ait: “Si inveni gratiam in conspectu tuo, Domine, obsecro, ut gradiaris nobiscum; populus quidem durae cervicis est, sed tu auferes iniquitates nostras atque peccata nosque possidebis". 
\verse Respondit Dominus: “Ego inibo pactum coram universo populo tuo; mirabilia faciam, quae numquam visa sunt super totam terram nec in ullis gentibus, ut cernat cunctus populus, in cuius es medio, opus Domini terribile, quod facturus sum tecum. 
\verse Observa cuncta, quae hodie mando tibi: ego ipse eiciam ante faciem tuam Amorraeum et Chananaeum et Hetthaeum, Pherezaeum quoque et Hevaeum et Iebusaeum. 
\verse Cave, ne umquam cum habitatoribus terrae, quam intraveris, iungas amicitias, quae tibi sint in ruinam; 
\verse sed aras eorum destrue, confringe lapides palosque succide. 
\verse Noli adorare deum alienum: Dominus Zelotes nomen eius, Deus est aemulator.  
\verse Ne ineas pactum cum hominibus illarum regionum, ne, cum fornicati fuerint cum diis suis et sacrificaverint eis, vocet te quispiam, et comedas de immolatis.  
\verse Nec uxorem de filiabus eorum accipies filiis tuis, ne, postquam ipsae fuerint fornicatae cum diis suis, fornicari faciant et filios tuos in deos suos. 
\verse Deos conflatiles non facies tibi. 
\verse Sollemnitatem Azymorum custodies: septem diebus vesceris azymis, sicut praecepi tibi, in tempore constituto mensis Abib; mense enim verni temporis egressus es de Aegypto. 
\verse Omne, quod aperit vulvam generis masculini, meum erit; de cuncto grege tuo tam de bobus quam de ovibus meum erit. 
\verse Primogenitum asini redimes ove, sin autem nec pretium pro eo dederis, franges cervicem eius. Primogenitum filiorum tuorum redimes; nec apparebis in conspectu meo vacuus. 
\verse Sex diebus operaberis, die septimo cessabis etiam arare et metere. 
\verse Sollemnitatem Hebdomadarum facies tibi in primitiis frugum messis tuae triticeae et sollemnitatem Collectae, quando, redeunte anni tempore, cuncta conduntur. 
\verse Tribus temporibus anni apparebit omne masculinum tuum in conspectu omnipotentis Domini, Dei Israel. 
\verse Cum enim tulero gentes a facie tua et dilatavero terminos tuos, nullus insidiabitur terrae tuae, ascendente te et apparente in conspectu Domini Dei tui ter in anno. 
\verse Non immolabis super fermento sanguinem hostiae meae; neque residebit mane de victima sollemnitatis Paschae. 
\verse Primitias frugum terrae tuae afferes in domum Domini Dei tui. Non coques haedum in lacte matris suae". 
\verse Dixitque Dominus ad Moysen: “Scribe tibi verba haec, quibus et tecum et cum Israel pepigi foedus". 
\verse Fuit ergo ibi cum Domino quadraginta dies et quadraginta noctes; panem non comedit et aquam non bibit et scripsit in tabulis verba foederis, decem verba. 
\verse Cumque descenderet Moyses de monte Sinai, tenebat duas tabulas testimonii et ignorabat quod resplenderet cutis faciei suae ex consortio sermonis Domini.  
\verse Videntes autem Aaron et filii Israel resplendere cutem faciei Moysi, timuerunt prope accedere; 
\verse vocatique ab eo reversi sunt tam Aaron quam principes synagogae. Et postquam locutus est ad eos, 
\verse venerunt ad eum etiam omnes filii Israel; quibus praecepit cuncta, quae audierat a Domino in monte Sinai. 
\verse Impletisque sermonibus, posuit velamen super faciem suam, 
\verse quod ingressus ad Dominum et loquens cum eo auferebat, donec exiret; et tunc loquebatur ad filios Israel omnia, quae sibi fuerant imperata. 
\verse Qui videbant cutem faciei Moysi resplendere, sed operiebat ille rursus faciem suam, donec ingressus loqueretur cum eo. 
\end{biblechapter}

\begin{biblechapter}  
\verse Igitur, congregato omni coetu filiorum Israel, dixit ad eos: “Haec sunt, quae iussit Dominus fieri: 
\verse sex diebus facietis opus, septimus dies erit vobis sanctus, sabbatum et requies Domino; qui fecerit opus in eo, occidetur.  
\verse Non succendetis ignem in omnibus habitaculis vestris per diem sabbati". 
\verse Et ait Moyses ad omnem coetum filiorum Israel: “Iste est sermo, quem praecepit Dominus dicens: 
\verse "Separate apud vos donaria Domino". Omnis voluntarius et proni animi offerat ea Domino: aurum et argentum et aes, 
\verse hyacinthum et purpuram coccumque et byssum, pilos caprarum 
\verse et pelles arietum rubricatas et pelles delphini, ligna acaciae 
\verse et oleum ad luminaria concinnanda et aromata, ut conficiatur unguentum et thymiama suavissimum, 
\verse lapides onychinos et gemmas ad ornatum ephod et pectoralis. 
\verse Quisquis vestrum sapiens est, veniat et faciat, quod Dominus imperavit,  
\verse habitaculum scilicet et tentorium eius atque operimentum, fibulas et tabulata cum vectibus, columnas et bases; 
\verse arcam et vectes, propitiatorium et velum, quod ante illud oppanditur; 
\verse mensam cum vectibus et vasis et propositionis panibus; 
\verse candelabrum ad luminaria sustentanda, vasa illius et lucernas et oleum ad nutrimenta luminarium; 
\verse altare thymiamatis et vectes et oleum unctionis et thymiama ex aromatibus; velum ad ostium habitaculi;  
\verse altare holocausti et craticulam eius aeneam cum vectibus et vasis suis, labrum et basim eius; 
\verse cortinas atrii cum columnis et basibus, velum in foribus atrii; 
\verse paxillos habitaculi et atrii cum funiculis suis; 
\verse vestimenta texta, quorum usus est in ministerio sanctuarii, vestes sanctas Aaron pontificis ac vestes filiorum eius, ut sacerdotio fungantur mihi". 
\verse Egressus est omnis coetus filiorum Israel de conspectu Moysi, 
\verse et venit, quisquis erat mentis promptissimae, et attulit sponte sua donaria Domino ad faciendum opus tabernaculi conventus et quidquid ad cultum et ad vestes sanctas necessarium erat. 
\verse Viri cum mulieribus, omnes voluntarii praebuerunt fibulas et inaures, anulos et dextralia; omne vas aureum in donaria Domini separatum est. 
\verse Si quis habebat hyacinthum et purpuram coccumque, byssum et pilos caprarum, pelles arietum rubricatas et pelles delphini, 
\verse argenti aerisque metalla, obtulerunt Domino lignaque acaciae in varios usus. 
\verse Sed et mulieres eruditae dederunt, quae neverant, hyacinthum, purpuram et coccum ac byssum 
\verse et pilos caprarum, sponte propria cuncta tribuentes.  
\verse Principes vero obtulerunt lapides onychinos et gemmas ad ephod et pectorale  
\verse aromataque et oleum ad luminaria concinnanda et ad praeparandum unguentum ac thymiama odoris suavissimi componendum. 
\verse Omnes viri et mulieres mente prompta obtulerunt donaria, ut fierent opera, quae iusserat Dominus per manum Moysi. Cuncti filii Israel voluntaria Domino dedicaverunt. 
\verse Dixitque Moyses ad filios Israel: “Ecce vocavit Dominus ex nomine Beseleel filium Uri filii Hur de tribu Iudae; 
\verse implevitque eum spiritu Dei, sapientia et intellegentia et scientia ad omne opus, 
\verse ad excogitandum et faciendum opus in auro et argento et aere, 
\verse ad scindendum et includendum gemmas et ad sculpendum ligna, quidquid fabre adinveniri potest. 
\verse Dedit quoque in corde eius, ut alios doceret, ipsi et Ooliab filio Achisamech de tribu Dan. 
\verse Ambos implevit sapientia, ut faciant opera fabri polymitarii ac plumarii de hyacintho ac purpura coccoque et bysso et textoris, facientes omne opus ac nova quaeque reperientes". 
\end{biblechapter}

\begin{biblechapter}  
\verse Fecit ergo Beseleel et Ooliab et omnis vir sapiens, quibus dedit Dominus sapientiam et intellectum, ut scirent fabre operari, quae in usus sanctuarii necessaria sunt et quae praecepit Dominus. 
\verse Cumque vocasset Moyses Beseleel et Ooliab et omnem eruditum virum, cui dederat Dominus sapientiam, omnes, qui sponte sua obtulerant se ad faciendum opus, 
\verse acceperunt ab ipso universa donaria, quae attulerant filii Israel ad faciendum opus in cultum sanctuarii. Ipsi autem cotidie mane donaria ei offerebant. 
\verse Unde omnes sapientes artifices venerunt singuli de opere suo pro sanctuario  
\verse et dixerunt Moysi: “Plus offert populus quam necessarium est operi, quod Dominus iussit facere". 
\verse Iussit ergo Moyses praeconis voce per castra clamari: “Nec vir nec mulier quidquam offerat ultra pro omni opere sanctuario". Sicque cessatum est a muneribus offerendis, 
\verse eo quod oblata sufficerent et superabundarent. 
\verse Feceruntque omnes corde sapientes inter artifices habitaculi cortinas decem de bysso retorta et hyacintho et purpura coccoque, cum cherubim intextis arte polymita; 
\verse quarum una habebat in longitudine viginti octo cubitos et in latitudine quattuor: una mensura erat omnium cortinarum. 
\verse Coniunxitque cortinas quinque alteram alteri et alias quinque sibi invicem copulavit. 
\verse Fecit et ansas hyacinthinas in ora cortinae unius in extremitate iuncturae et in ora cortinae extremae in iunctura altera similiter. 
\verse Quinquagenas ansas fecit pro utraque cortina, ut contra se invicem venirent ansae et mutuo iungerentur. 
\verse Unde et quinquaginta fudit fibulas aureas, quae morderent cortinarum ansas, et fieret unum habitaculum. 
\verse Fecit et saga undecim de pilis caprarum pro tentorio super habitaculum; 
\verse unum sagum in longitudine habebat cubitos triginta et in latitudine cubitos quattuor: unius mensurae erant omnia saga. 
\verse Quorum quinque iunxit seorsum et sex alia separatim. 
\verse Fecitque ansas quinquaginta in ora sagi ultimi iuncturae unius et quinquaginta in ora sagi iuncturae alterius, ut sibi invicem iungerentur; 
\verse et fecit fibulas aeneas quinquaginta, quibus necteretur tentorium, ut esset unum. 
\verse Fecit et opertorium tentorio de pellibus arietum rubricatis aliudque desuper velamentum de pellibus delphini. 
\verse Fecit et tabulas habitaculi de lignis acaciae stantes. 
\verse Decem cubitorum erat longitudo tabulae unius, et unum ac semis cubitum latitudo retinebat.  
\verse Bini pedes erant per singulas tabulas, ut altera alteri iungeretur: sic fecit in omnibus tabulis habitaculi. 
\verse E quibus viginti ad plagam meridianam erant contra austrum 
\verse cum quadraginta basibus argenteis. Duae bases sub singulis tabulis ponebantur pro duabus pedibus. 
\verse Ad plagam quoque habitaculi, quae respicit ad aquilonem, fecit viginti tabulas 
\verse cum quadraginta basibus argenteis: duas bases per singulas tabulas. 
\verse Contra occidentem vero, id est ad eam partem habitaculi quae mare respicit, fecit sex tabulas 
\verse et duas alias per singulos angulos habitaculi retro; 
\verse quae gemellae erant a deorsum usque sursum in unam compaginem. Ita fecit duas tabulas in duobus angulis, 
\verse ut octo essent simul tabulae et haberent bases argenteas sedecim: binas scilicet bases sub singulis tabulis. 
\verse Fecit et vectes de lignis acaciae quinque ad continendas tabulas unius lateris habitaculi  
\verse et quinque alios ad alterius lateris coaptandas tabulas; et extra hos quinque alios vectes ad occidentalem plagam habitaculi contra mare. 
\verse Fecit autem vectem medium, qui per medias tabulas ab una extremitate usque ad alteram perveniret. 
\verse Ipsa autem tabulata deauravit. Et anulos eorum fecit aureos, per quos vectes induci possent; quos et ipsos laminis aureis operuit. 
\verse Fecit et velum de hyacintho et purpura coccoque ac bysso retorta, opere polymitario, cum cherubim intextis; 
\verse et quattuor columnas de lignis acaciae, quas cum uncis suis deauravit, fusis basibus earum argenteis. 
\verse Fecit et velum in introitu tabernaculi ex hyacintho, purpura, cocco byssoque retorta opere plumarii; 
\verse et columnas quinque cum uncis suis. Et operuit auro capita et anulos earum basesque earum fudit aeneas. 
\end{biblechapter}

\begin{biblechapter}  
\verse Fecit autem Beseleel et arcam de lignis acaciae habentem duos semis cubitos in longitudine et cubitum ac semissem in latitudine, altitudo quoque unius cubiti fuit et dimidii; vestivitque eam auro purissimo intus ac foris.  
\verse Et fecit illi coronam auream per gyrum, 
\verse conflans quattuor anulos aureos in quattuor pedibus eius; duos anulos in latere uno et duos in altero. 
\verse Vectes quoque fecit de lignis acaciae, quos vestivit auro 
\verse et quos misit in anulos, qui erant in lateribus arcae, ad portandum eam. 
\verse Fecit et propitiatorium de auro mundissimo: duorum cubitorum et dimidii in longitudine et cubiti ac semis in latitudine. 
\verse Duos etiam cherubim ex auro ductili fecit ex utraque parte propitiatorii: 
\verse cherub unum ex summitate unius partis et cherub alterum ex summitate partis alterius; duos cherubim ex singulis summitatibus propitiatorii 
\verse extendentes alas sursum et tegentes alis suis propitiatorium seque mutuo et illud respicientes. 
\verse Fecit et mensam de lignis acaciae in longitudine duorum cubitorum et in latitudine unius cubiti, quae habebat in altitudine cubitum ac semissem; 
\verse circumdeditque eam auro mundissimo et fecit illi coronam auream per gyrum.  
\verse Fecit ei quoque limbum aureum quattuor digitorum per circuitum et super illum coronam auream. 
\verse Fudit et quattuor circulos aureos, quos posuit in quattuor angulis per singulos pedes mensae 
\verse iuxta limbum; misitque in eos vectes, ut possit mensa portari. 
\verse Ipsos quoque vectes fecit de lignis acaciae et circumdedit eos auro; 
\verse et vasa ad diversos usus mensae, acetabula, phialas et cyathos et crateras ex auro puro, in quibus offerenda sunt libamina. 
\verse Fecit et candelabrum ductile de auro mundissimo, basim et hastile eius; scyphi sphaerulaeque ac flores unum cum ipso erant: 
\verse sex in utroque latere, tres calami ex parte una et tres ex altera; 
\verse tres scyphi in nucis modum in calamo uno sphaerulaeque simul et flores et tres scyphi instar nucis in calamo altero sphaerulaeque simul et flores. Aequum erat opus sex calamorum, qui procedebant de hastili candelabri. 
\verse In ipso autem hastili erant quattuor scyphi in nucis modum sphaerulaeque et flores; 
\verse singulae sphaerulae sub binis calamis per loca tria, qui simul sex fiunt calami procedentes de hastili uno. 
\verse Sphaerulae igitur et calami unum cum ipso erant, totum ductile ex auro purissimo. 
\verse Fecit et lucernas septem cum emunctoriis suis et vasa, ubi emuncta condantur, de auro mundissimo. 
\verse Talentum auri purissimi appendebat candelabrum cum omnibus vasis suis. 
\verse Fecit et altare thymiamatis de lignis acaciae habens per quadrum singulos cubitos et in altitudine duos; e cuius angulis procedebant cornua. 
\verse Vestivitque illud auro purissimo cum craticula ac parietibus et cornibus. 
\verse Fecitque ei coronam aureolam per gyrum et binos anulos aureos sub corona in duobus lateribus, ut mittantur in eos vectes, et possit altare portari. 
\verse Ipsos autem vectes fecit de lignis acaciae et operuit laminis aureis. 
\verse Composuit et oleum ad sanctificationis unguentum et thymiama de aromatibus mundissimis opere pigmentarii. 
\end{biblechapter}

\begin{biblechapter}  
\verse Fecit et altare holocausti de lignis acaciae quinque cubitorum per quadrum et trium in altitudine, 
\verse cuius cornua de angulis procedebant; operuitque illud laminis aeneis. 
\verse Et in usus eius paravit ex aere vasa diversa: lebetes, vatilla et pateras, fuscinulas et ignium receptacula. 
\verse Craticulamque eius in modum retis fecit aeneam subter marginem altaris ab imo usque ad medium eius, 
\verse fusis quattuor anulis per totidem craticulae summitates, ad immittendos vectes ad portandum. 
\verse Quos et ipsos fecit de lignis acaciae et operuit laminis aeneis; 
\verse induxitque in circulos, qui in lateribus altaris eminebant. Ipsum autem altare non erat solidum, sed cavum ex tabulis et intus vacuum. 
\verse Fecit et labrum aeneum cum basi sua de speculis mulierum, quae excubabant in ostio tabernaculi conventus. 
\verse Fecit et atrium, in cuius australi plaga erant tentoria de bysso retorta cubitorum centum; 
\verse columnae aeneae viginti cum basibus suis; unci columnarum et anuli earum argentei. 
\verse Aeque ad septentrionalem plagam tentoria, columnae basesque et unci anulique columnarum eiusdem mensurae et operis ac metalli erant. 
\verse In ea vero plaga, quae ad occidentem respicit, fuerunt tentoria cubitorum quinquaginta, columnae decem cum basibus suis; et unci columnarum anulique earum argentei. 
\verse Porro contra orientem quinquaginta cubitorum paravit tentoria, 
\verse e quibus quindecim cubitos columnarum trium cum basibus suis unum tenebat latus; 
\verse et in parte altera ­ quia inter utraque introitum tabernaculi fecit ­ quindecim aeque cubitorum erant tentoria columnaeque tres et bases totidem. 
\verse Cuncta atrii tentoria in circuitu ex bysso retorta texuerat. 
\verse Bases columnarum fuere aeneae, unci autem earum et anuli earum argentei et capita earum vestivit argento et omnes columnas atrii cinxit anulis argenteis. 
\verse Et in introitu eius opere plumario fecit velum ex hyacintho, purpura, cocco ac bysso retorta; quod habebat viginti cubitos in longitudine, altitudo vero quinque cubitorum erat iuxta mensuram, quam cuncta atrii tentoria habebant.  
\verse Columnae autem in ingressu fuere quattuor cum basibus aeneis, uncis argenteis; capitaque et anulos earum vestivit argento. 
\verse Paxillos quoque habitaculi et atrii per gyrum fecit aeneos. 
\verse Hic est census habitaculi, habitaculi testimonii, qui recensitus est iuxta praeceptum Moysi ministerio Levitarum per manum Ithamar filii Aaron sacerdotis. 
\verse Beseleel filius Uri filii Hur de tribu Iudae fecit cuncta, quae praeceperat Dominus Moysi, 
\verse iuncto sibi socio Ooliab filio Achisamech de tribu Dan fabro et polymitario atque plumario ex hyacintho, purpura, cocco et bysso. 
\verse Omne aurum, quod expensum est in opere sanctuarii et quod oblatum est in donariis, viginti novem talentorum fuit et septingentorum triginta siclorum ad mensuram sicli sanctuarii. 
\verse Argentum autem eorum, qui in congregatione recensiti sunt, centum talentorum fuit et mille septingentorum et septuaginta quinque siclorum ad mensuram sicli sanctuarii. 
\verse Beca, id est dimidium sicli iuxta mensuram sicli sanctuarii, dedit quisquis transit ad censum a viginti annis et supra, de sescentis tribus milibus et quingentis quinquaginta armatorum. 
\verse De talentis centum argenti conflatae sunt bases sanctuarii et veli, singulis talentis per bases singulas supputatis. 
\verse De mille autem septingentis et septuaginta quinque siclis fecit uncos columnarum et vestivit capita earum et cinxit eas argento. 
\verse Aeris quoque oblata sunt septuaginta talenta et duo milia et quadringenti sicli, 
\verse ex quibus fecit bases in introitu tabernaculi conventus et altare aeneum cum craticula sua omniaque vasa, quae ad usum eius pertinent, 
\verse et bases atrii tam in circuitu quam in ingressu eius et omnes paxillos habitaculi atque atrii per gyrum. 
\end{biblechapter}

\begin{biblechapter}  
\verse De hyacintho vero et purpura, cocco ac bysso fecerunt vestes textas pro ministerio sanctuarii. Et fecerunt vestes sacras Aaron, sicut praecepit Dominus Moysi. 
\verse Fecerunt igitur ephod de auro, hyacintho et purpura coccoque et bysso retorta 
\verse opere polymitario tundentes bratteas aureas et extenuantes in fila, ut possent torqueri cum priorum colorum subtegmine. 
\verse Fasciasque umerales fecerunt ei, cum quibus in utroque latere summitatum suarum copulabatur, 
\verse et balteum, quo constringebatur ephod, eiusdem operis et unum cum eo ex auro, et hyacintho et purpura coccoque et bysso retorta, sicut praeceperat Dominus Moysi.  
\verse Paraverunt et duos lapides onychinos, inclusos texturis aureis et sculptos arte gemmaria nominibus filiorum Israel; 
\verse posueruntque eos in fasciis umeralibus ephod, lapides memorialis filiorum Israel, sicut praeceperat Dominus Moysi. 
\verse Fecerunt et pectorale opere polymito iuxta opus ephod ex auro, hyacintho, purpura coccoque et bysso retorta, 
\verse quadrangulum duplex mensurae palmi.  
\verse Et posuerunt in eo gemmarum ordines quattuor: in primo versu erat sardius, topazius, smaragdus; 
\verse in secundo carbunculus, sapphirus et iaspis; 
\verse in tertio hyacinthus, achates et amethystus; 
\verse in quarto chrysolithus, onychinus et beryllus: inclusi textura aurea per ordines suos. 
\verse Ipsique lapides duodecim sculpti erant nominibus duodecim tribuum Israel, singuli per nomina singulorum. 
\verse Fecerunt in pectorali catenulas quasi funiculos opus tortile de auro purissimo 
\verse et duos margines aureos totidemque anulos aureos. Porro duos anulos posuerunt in utraque summitate pectoralis; 
\verse duos funiculos aureos inseruerunt anulis, qui in pectoralis angulis eminebant. 
\verse Duas summitates amborum funiculorum colligaverunt duobus marginibus in fasciis umeralibus ephod in parte eius anteriore. 
\verse Et fecerunt duos anulos aureos et posuerunt super duas summitates pectoralis in eius margine interiore contra ephod, sicut praecepit Dominus Moysi. 
\verse Feceruntque duos anulos aureos, quos posuerunt in duabus fasciis umeralibus ephod deorsum in latere eius anteriore secus iuncturam eius super balteum ephod. 
\verse Et strinxerunt pectorale anulis eius ad anulos ephod vitta hyacinthina, ut esset super balteum ephod, ne amoveretur ab ephod, sicut praecepit Dominus Moysi. 
\verse Fecerunt quoque pallium ephod opere textili totum hyacinthinum 
\verse et capitium in medio eius supra oramque per gyrum sicut in capitio loricae; 
\verse deorsum autem ad pedes mala punica ex hyacintho, purpura, cocco ac bysso retorta  
\verse et tintinnabula de auro purissimo, quae posuerunt inter malogranata in inferiore parte pallii per gyrum, 
\verse ut sit tintinnabulum inter singula mala punica, quibus ornatus incedebat pontifex, quando ministerio fungebatur, sicut praeceperat Dominus Moysi. 
\verse Fecerunt et tunicas byssinas opere textili Aaron et filiis eius 
\verse et tiaram et ornatum mitrarum ex bysso, feminalia quoque linea ex bysso retorta,  
\verse cingulum vero de bysso retorta, hyacintho, purpura ac cocco, arte plumaria, sicut praeceperat Dominus Moysi. 
\verse Fecerunt et laminam diadema sanctitatis de auro purissimo; scripseruntque in ea opere caelatoris: “Sanctum Domino"; 
\verse et strinxerunt eam desuper cum tiara vitta hyacinthina, sicut praeceperat Dominus Moysi. 
\verse Perfectum est igitur omne opus habitaculi et tabernaculi conventus; feceruntque filii Israel cuncta, quae praeceperat Dominus Moysi: sic fecerunt. 
\verse Et obtulerunt habitaculum et tabernaculum et universam supellectilem, fibulas, tabulas, vectes, columnas ac bases, 
\verse opertorium de pellibus arietum rubricatis et operimentum de pellibus delphini, velum, 
\verse arcam testimonii, vectes, propitiatorium, 
\verse mensam cum vasis suis et propositionis panibus, 
\verse candelabrum ex auro puro, lucernas in ordine earum et utensilia earum cum oleo candelabri, 
\verse altare aureum et unguentum et thymiama ex aromatibus et velum in introitu tabernaculi, 
\verse altare aeneum, craticulam aeneam, vectes et vasa eius omnia, labrum cum basi sua, 
\verse tentoria atrii et columnas cum basibus suis, velum in introitu atrii funiculosque illius et paxillos. Nihil ex vasis defuit, quae in ministerium habitaculi in tabernaculo conventus iussa sunt fieri. 
\verse Vestes quoque textas, quibus sacerdotes utuntur in sanctuario, et vestes sacras Aaron sacerdotis et vestes filiorum eius 
\verse obtulerunt filii Israel, sicut praeceperat Dominus Moysi. 
\verse Quae postquam Moyses cuncta vidit completa, benedixit eis. 
\end{biblechapter}

\begin{biblechapter}  
\verse Locutusque est Dominus ad Moysen dicens: 
\verse “Mense primo, die prima mensis eriges habitaculum, tabernaculum conventus, 
\verse et pones in eo arcam testimonii, abscondes illam velo; 
\verse et, illata mensa, pones super eam, quae rite praecepta sunt. Candelabrum stabit cum lucernis suis 
\verse et altare aureum, in quo adoletur incensum, coram arca testimonii. Velum in introitu habitaculi pones, 
\verse et ante tabernaculum conventus altare holocausti, 
\verse et labrum inter altare et tabernaculum conventus et implebis illud aqua. 
\verse Circumdabisque atrium tentoriis et pones velum in porta eius. 
\verse Et, assumpto unctionis oleo, unges habitaculum et omnia, quae in eo sunt, et consecrabis illud cum vasis suis, et erit sanctum. 
\verse Unges quoque altare holocausti et omnia vasa eius et consecrabis altare, et erit sanctum sanctorum. 
\verse Et unges labrum cum basi sua et consecrabis illud. 
\verse Applicabisque Aaron et filios eius ad fores tabernaculi conventus; et lotos aqua 
\verse indues Aaron sanctis vestibus, unges et consecrabis eum, ut mihi sacerdotio fungatur; 
\verse filios eius applicabis et vesties eos tunicis 
\verse et unges eos, sicut unxisti patrem eorum, ut mihi sacerdotio fungantur, et unctio eorum erit eis in sacerdotium sempiternum in generationibus eorum".  
\verse Fecitque Moyses omnia, quae praeceperat ei Dominus: sic fecit. 
\verse Igitur mense primo anni secundi, prima die mensis collocatum est habitaculum. 
\verse Erexitque Moyses illud et posuit bases ac tabulas et vectes statuitque columnas 
\verse et expandit tentorium super habitaculum, imposito desuper operimento, sicut Dominus imperaverat Moysi. 
\verse Sumpsit et posuit testimonium in arca et, subditis infra vectibus, posuit propitiatorium desuper.  
\verse Cumque intulisset arcam in habitaculum, appendit ante eam velum, sicut iusserat Dominus Moysi. 
\verse Posuit et mensam in tabernaculo conventus ad plagam septentrionalem extra velum, 
\verse ordinatis coram propositionis panibus, sicut praeceperat Dominus Moysi. 
\verse Posuit et candelabrum in tabernaculo conventus e regione mensae in parte australi, 
\verse locatis per ordinem lucernis, sicut praeceperat Dominus Moysi. 
\verse Posuit et altare aureum in tabernaculo conventus coram propitiatorio 
\verse et adolevit super eo incensum aromatum, sicut iusserat Dominus Moysi.  
\verse Posuit et velum in introitu habitaculi 
\verse et altare holocausti in vestibulo habitaculi, tabernaculi conventus, offerens in eo holocaustum et sacrificium, sicut Dominus imperaverat Moysi. 
\verse Labrum quoque statuit inter tabernaculum conventus et altare implens illud aqua; 
\verse laveruntque Moyses et Aaron ac filii eius manus suas et pedes, 
\verse cum ingrederentur tabernaculum conventus et accederent ad altare, sicut praeceperat Dominus Moysi.  
\verse Erexit et atrium per gyrum habitaculi et altaris, ducto in introitu eius velo. Sic complevit opus. 
\verse Et operuit nubes tabernaculum conventus, et gloria Domini implevit habitaculum. 
\verse Nec poterat Moyses ingredi tabernaculum conventus, quia habitavit nubes super illud, et gloria Domini replevit habitaculum. 
\verse Si quando nubes de tabernaculo ascendebat, proficiscebantur filii Israel in omnibus stationibus suis; 
\verse si autem non ascendebat nubes, non proficiscebantur usque in diem, quo levabatur. 
\verse Nubes quippe Domini incubabat per diem habitaculo, et ignis in nocte, ante oculos universae domus Israel per cunctas mansiones suas.
\end{biblechapter}
