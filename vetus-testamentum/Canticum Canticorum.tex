\biblebook{Canticum Canticorum}

\begin{biblechapter}   
\verse Canticum Canticorum Salomonis. 
\verse Osculetur me osculo oris sui! Nam meliores sunt amores tui vino: 
\verse in fragrantiam unguentorum tuorum optimorum. Oleum effusum nomen tuum; ideo adulescentulae dilexerunt te. 
\verse Trahe me post te. Curramus! Introducat me rex in cellaria sua; exsultemus et laetemur in te memores amorum tuorum super vinum; recte diligunt te. 
\verse Nigra sum sed formosa, filiae Ierusalem, sicut tabernacula Cedar, sicut pelles Salma. 
\verse Nolite me considerare quod fusca sim, quia decoloravit me sol. Filii matris meae irati sunt mihi; posuerunt me custodem in vineis, vineam meam non custodivi. 
\verse Indica mihi, tu, quem diligit anima mea, ubi pascas, ubi cubes in meridie, ne vagari incipiam post greges sodalium tuorum. 
\verse Si ignoras, o pulcherrima inter mulieres, egredere et abi post vestigia gregum et pasce haedos tuos iuxta tabernacula pastorum. 
\verse Equae in curribus pharaonis assimilavi te, amica mea. 
\verse Pulchrae sunt genae tuae inter inaures, collum tuum inter monilia. 
\verse Inaures aureas faciemus tibi vermiculatas argento. 
\verse Dum esset rex in accubitu suo, nardus mea dedit odorem suum. 
\verse Fasciculus myrrhae dilectus meus mihi, qui inter ubera mea commoratur. 
\verse Botrus cypri dilectus meus mihi in vineis Engaddi. 
\verse Ecce tu pulchra es, amica mea, ecce tu pulchra es: oculi tui columbarum. 
\verse Ecce tu pulcher es, dilecte mi, et decorus. Lectulus noster floridus, 
\verse tigna domorum nostrarum cedrina, laquearia nostra cupressina. 
\end{biblechapter}

\begin{biblechapter}  
\verse Ego flos campi et lilium convallium. 
\verse Sicut lilium inter spinas, sic amica mea inter filias. 
\verse Sicut malus inter ligna silvarum, sic dilectus meus inter filios. Sub umbra illius, quem desideraveram, sedi, et fructus eius dulcis gutturi meo. 
\verse Introduxit me in cellam vinariam, et vexillum eius super me est caritas. 
\verse Fulcite me uvarum placentis, stipate me malis, quia amore langueo. 
\verse Laeva eius sub capite meo, et dextera illius amplexatur me. 
\verse Adiuro vos, filiae Ierusalem, per capreas cervasque camporum, ne suscitetis neque evigilare faciatis dilectam, quoadusque ipsa velit. 
\verse Vox dilecti mei! Ecce iste venit saliens in montibus, transiliens colles. 
\verse Similis est dilectus meus capreae hinnuloque cervorum. En ipse stat post parietem nostrum respiciens per fenestras, prospiciens per cancellos. 
\verse En dilectus meus loquitur mihi: “Surge, amica mea, columba mea, formosa mea, et veni. 
\verse Iam enim hiems transiit, imber abiit et recessit. 
\verse Flores apparuerunt in terra, tempus putationis advenit; vox turturis audita est in terra nostra, 
\verse ficus protulit grossos suos, vineae florentes dederunt odorem suum; surge, amica mea, speciosa mea, et veni, 
\verse columba mea, in foraminibus petrae, in caverna abrupta. Ostende mihi faciem tuam, sonet vox tua in auribus meis; vox enim tua dulcis, et facies tua decora". 
\verse Capite nobis vulpes, vulpes parvulas, quae demoliuntur vineas, nam vineae nostrae florescunt. 
\verse Dilectus meus mihi, et ego illi, qui pascitur inter lilia, 
\verse antequam aspiret dies, et festinent umbrae. Revertere; similis esto, dilecte mi, capreae hinnuloque cervorum super montes Bether. 
\end{biblechapter}

\begin{biblechapter}  
\verse In lectulo meo per noctes quaesivi, quem diligit anima mea; quaesivi illum et non inveni. 
\verse “Surgam et circuibo civitatem; per vicos et plateas quaeram, quem diligit anima mea”. Quaesivi illum et non inveni. 
\verse Invenerunt me vigiles, qui circumeunt civitatem: “Num, quem diligit anima mea, vidistis?". 
\verse Paululum cum pertransissem eos, inveni, quem diligit anima mea; tenui eum nec dimittam, donec introducam illum in domum matris meae et in cubiculum genetricis meae. 
\verse Adiuro vos, filiae Ierusalem, per capreas cervasque camporum, ne suscitetis neque evigilare faciatis dilectam, donec ipsa velit. 
\verse Quid hoc, quod ascendit per desertum sicut virgula fumi, aromatizans tus et myrrham et universum pulverem pigmentarii? 
\verse En lectulum Salomonis. Sexaginta fortes ambiunt illum ex fortissimis Israel, 
\verse omnes tenentes gladios et ad bella doctissimi, uniuscuiusque ensis super femur suum propter timores nocturnos. 
\verse Ferculum fecit sibi rex Salomon de lignis Libani; 
\verse columnas eius fecit argenteas, reclinatorium aureum, sedile purpureum: medium eius stratum ebeneum. Filiae Ierusalem, 
\verse egredimini et videte, filiae Sion, regem Salomonem in diademate, quo coronavit illum mater sua in die desponsationis illius et in die laetitiae cordis eius. 
\end{biblechapter}

\begin{biblechapter}  
\verse Quam pulchra es, amica mea, quam pulchra es: oculi tui columbarum per velamen tuum. Capilli tui sicut grex caprarum, quae descenderunt de monte Galaad; 
\verse dentes tui sicut grex tonsarum, quae ascenderunt de lavacro: omnes gemellis fetibus, et sterilis non est inter eas. 
\verse Sicut vitta coccinea labia tua, et eloquium tuum dulce; sicut fragmen mali punici, ita genae tuae per velamen tuum. 
\verse Sicut turris David collum tuum, quae aedificata est cum propugnaculis: mille clipei pendent ex ea, omnis armatura fortium. 
\verse Duo ubera tua sicut duo hinnuli, capreae gemelli, qui pascuntur in liliis. 
\verse Antequam aspiret dies, et festinent umbrae, vadam ad montem myrrhae et ad collem turis. 
\verse Tota pulchra es, amica mea, et macula non est in te. 
\verse Veni de Libano, sponsa, veni de Libano, ingredere; respice de capite Amana, de vertice Sanir et Hermon, de cubilibus leonum, de montibus pardorum. 
\verse Vulnerasti cor meum, soror mea, sponsa, vulnerasti cor meum in uno oculorum tuorum et in uno monili torquis tui. 
\verse Quam pulchri sunt amores tui, soror, mea sponsa; meliores sunt amores tui vino, et odor unguentorum tuorum super omnia aromata. 
\verse Favus distillans labia tua, sponsa; mel et lac sub lingua tua, et odor vestimentorum tuorum sicut odor Libani. 
\verse Hortus conclusus, soror mea, sponsa, hortus conclusus, fons signatus; 
\verse propagines tuae paradisus malorum punicorum cum optimis fructibus, cypri cum nardo. 
\verse Nardus et crocus, fistula et cinnamomum cum universis lignis turiferis, myrrha et aloe cum omnibus primis unguentis. 
\verse Fons hortorum, puteus aquarum viventium, quae fluunt impetu de Libano. 
\verse Surge, aquilo, et veni, auster; perfla hortum meum, et fluant aromata illius. 
\end{biblechapter}

\begin{biblechapter}  
\verse Veniat dilectus meus in hortum suum et comedat fructus eius optimos. Veni in hortum meum, soror mea, sponsa; messui myrrham meam cum aromatibus meis, comedi favum cum melle, bibi vinum cum lacte meo. Comedite, amici, et bibite et inebriamini, carissimi. 
\verse Ego dormio, et cor meum vigilat. Vox dilecti mei pulsantis: “Aperi mihi, soror mea, amica mea, columba mea, immaculata mea, quia caput meum plenum est rore, et cincinni mei guttis noctium".  
\verse “Exspoliavi me tunica mea, quomodo induar illa? Lavi pedes meos, quomodo inquinabo illos?”. 
\verse Dilectus meus misit manum suam per foramen, et venter meus ilico intremuit. 
\verse Surrexi, ut aperirem dilecto meo; manus meae stillaverunt myrrham, et digiti mei pleni myrrha probatissima super ansam pessuli. 
\verse Aperui dilecto meo; at ille declinaverat atque transierat. Anima mea liquefacta est, quia discesserat. Quaesivi et non inveni illum; vocavi, et non respondit mihi. 
\verse Invenerunt me custodes, qui circumeunt civitatem; percusserunt me et vulneraverunt me, tulerunt pallium meum mihi custodes murorum. 
\verse Adiuro vos, filiae Ierusalem: si inveneritis dilectum meum, quid nuntietis ei? “Quia amore langueo". 
\verse Quid est dilecto tuo prae ceteris, o pulcherrima mulierum? Quid est dilecto tuo prae ceteris, quia sic adiurasti nos? 
\verse Dilectus meus candidus et rubicundus dignoscitur ex milibus. 
\verse Caput eius aurum optimum, cincinni eius sicut racemi palmarum, nigri quasi corvus. 
\verse Oculi eius sicut columbae super rivulos aquarum, quae lacte sunt lotae et resident iuxta fluenta plenissima. 
\verse Genae illius sicut areolae aromatum, turriculae unguentorum; labia eius lilia distillantia myrrham primam. 
\verse Manus illius tornatiles aureae, plenae hyacinthis; venter eius opus eburneum distinctum sapphiris. 
\verse Crura illius columnae marmoreae, quae fundatae sunt super bases aureas; species eius ut Libani, electus ut cedri. 
\verse Guttur illius suavissimum, et totus desiderabilis. Talis est dilectus meus, et ipse est amicus meus, filiae Ierusalem. 
\end{biblechapter}

\begin{biblechapter}  
\verse Quo abiit dilectus tuus, o pulcherrima mulierum? Quo declinavit dilectus tuus, et quaeremus eum tecum? 
\verse Dilectus meus descendit in hortum suum ad areolam aromatum, ut pascatur in hortis et lilia colligat. 
\verse Ego dilecto meo, et dilectus meus mihi, qui pascitur inter lilia. 
\verse Pulchra es, amica mea, sicut Thersa, decora sicut Ierusalem, terribilis ut castrorum acies ordinata. 
\verse Averte oculos tuos a me, quia ipsi me conturbant. Capilli tui sicut grex caprarum, quae descenderunt de Galaad. 
\verse Dentes tui sicut grex ovium, quae ascenderunt de lavacro: omnes gemellis fetibus, et sterilis non est in eis. 
\verse Sicut fragmen mali punici, sic genae tuae per velamen tuum. 
\verse Sexaginta sunt reginae, et octoginta concubinae, et adulescentularum non est numerus; 
\verse una est columba mea, perfecta mea, una est matri suae, electa genetrici suae. Viderunt eam filiae et beatissimam praedicaverunt; reginae et concubinae, et laudaverunt eam: 
\verse “Quae est ista, quae progreditur quasi aurora consurgens, pulchra ut luna, electa ut sol, terribilis ut castrorum acies ordinata?". 
\verse Descendi in hortum nucum, ut viderem poma convallium et inspicerem, si floruisset vinea, et germinassent mala punica. 
\verse Non advertit animus meus, cum posuit me in quadrigas principis populi mei. 
\end{biblechapter}

\begin{biblechapter}  
\verse Convertere, convertere, Sulamitis; convertere, convertere, ut intueamur te. Quid aspicitis in Sulamitem, cum saltat inter binos choros? 
\verse Quam pulchri sunt pedes tui in calceamentis, filia principis! Flexurae femorum tuorum sicut monilia, quae fabricata sunt manu artificis. 
\verse Gremium tuum crater tornatilis: numquam indigeat vino mixto; venter tuus sicut acervus tritici vallatus liliis. 
\verse Duo ubera tua sicut duo hinnuli, gemelli capreae, 
\verse collum tuum sicut turris eburnea. Oculi tui sicut piscinae in Hesebon, quae sunt ad portam Bathrabbim; nasus tuus sicut turris Libani, quae respicit contra Damascum. 
\verse Caput tuum ut Carmelus, et comae capitis tui sicut purpura; rex vincitur cincinnis. 
\verse Quam pulchra es et quam decora, carissima, in deliciis! 
\verse Statura tua assimilata est palmae, et ubera tua botris. 
\verse Dixi: “Ascendam in palmam et apprehendam fructus eius". Et erunt ubera tua sicut botri vineae, et odor oris tui sicut malorum. 
\verse Guttur tuum sicut vinum optimum, dignum dilecto meo ad potandum, labiisque et dentibus illius ad ruminandum. 
\verse Ego dilecto meo, et ad me appetitus eius. 
\verse Veni, dilecte mi, egrediamur in agrum, commoremur in villis; 
\verse mane properabimus ad vineas, videbimus; si floruit vinea, si flores aperiuntur, si floruerunt mala punica; ibi dabo tibi amores meos. 
\verse Mandragorae dederunt odorem; in portis nostris omnia poma optima, nova et vetera, dilecte mi, servavi tibi. 
\end{biblechapter}

\begin{biblechapter}  
\verse Quis mihi det te fratrem meum, sugentem ubera matris meae, ut inveniam te foris et deosculer te, et iam me nemo despiciat? 
\verse Apprehenderem te et ducerem in domum matris meae; ibi me doceres, et darem tibi poculum ex vino condito et mustum malorum granatorum meorum. 
\verse Laeva eius sub capite meo, et dextera illius amplexatur me. 
\verse Adiuro vos, filiae Ierusalem, ne suscitetis neque evigilare faciatis dilectam, donec ipsa velit. 
\verse Quae est ista, quae ascendit de deserto innixa super dilectum suum? Sub arbore malo suscitavi te; ibi parturivit te mater tua, ibi parturivit te genetrix tua. 
\verse Pone me ut signaculum super cor tuum, ut signaculum super brachium tuum, quia fortis est ut mors dilectio, dura sicut infernus aemulatio; lampades eius lampades ignis atque flammae divinae. 
\verse Aquae multae non potuerunt exstinguere caritatem, nec flumina obruent illam; si dederit homo omnem substantiam domus suae pro dilectione, quasi nihil despicient eum. 
\verse Soror nostra parva et ubera non habet; quid faciemus sorori nostrae in die, quando alloquenda est? 
\verse Si murus est, aedificemus super eum propugnacula argentea; si ostium est, compingamus illud tabulis cedrinis. 
\verse Ego murus, et ubera mea sicut turris; ex quo facta sum coram eo quasi pacem reperiens. 
\verse Vinea fuit Salomoni in Baalhamon. Tradidit eam custodibus; vir affert pro fructu eius mille argenteos. 
\verse Vinea mea coram me est; mille tibi, Salomon, et ducenti his, qui custodiunt fructus eius. 
\verse Quae habitas in hortis, amici auscultant, fac me audire vocem tuam. 
\verse Fuge, dilecte mi, et assimilare capreae hinnuloque cervorum super montes aromatum.
\end{biblechapter}
