\biblebook{Evangelium secundum Marcum}

\begin{biblechapter}   
\verse Initium evangelii Iesu Christi Filii Dei. 
\verse Sicut scriptum est in Isaia propheta: “Ecce mitto angelum meum ante faciem tuam, qui praeparabit viam tuam; 
\verse vox clamantis in deserto: “Parate viam Domini, rectas facite semitas eius”", 
\verse fuit Ioannes Baptista in deserto praedicans baptismum paenitentiae in remissionem peccatorum. 
\verse Et egrediebatur ad illum omnis Iudaeae regio et Hierosolymitae universi et baptizabantur ab illo in Iordane flumine confitentes peccata sua. 
\verse Et erat Ioannes vestitus pilis cameli, et zona pellicea circa lumbos eius, et locustas et mel silvestre edebat. 
\verse Et praedicabat dicens: “Venit fortior me post me, cuius non sum dignus procumbens solvere corrigiam calceamentorum eius. 
\verse Ego baptizavi vos aqua; ille vero baptizabit vos in Spiritu Sancto". 
\verse Et factum est in diebus illis, venit Iesus a Nazareth Galilaeae et baptizatus est in Iordane ab Ioanne. 
\verse Et statim ascendens de aqua vidit apertos caelos et Spiritum tamquam columbam descendentem in ipsum; 
\verse et vox facta est de caelis: “Tu es Filius meus dilectus; in te complacui". 
\verse Et statim Spiritus expellit eum in desertum. 
\verse Et erat in deserto quadraginta diebus et tentabatur a Satana; eratque cum bestiis, et angeli ministrabant illi. 
\verse Postquam autem traditus est Ioannes, venit Iesus in Galilaeam praedicans evangelium Dei 
\verse et dicens: “Impletum est tempus, et appropinquavit regnum Dei; paenitemini et credite evangelio". 
\verse Et praeteriens secus mare Galilaeae vidit Simonem et Andream fratrem Simonis mittentes in mare; erant enim piscatores. 
\verse Et dixit eis Iesus: “Venite post me, et faciam vos fieri piscatores hominum". 
\verse Et protinus, relictis retibus, secuti sunt eum. 
\verse Et progressus pusillum vidit Iacobum Zebedaei et Ioannem fratrem eius, et ipsos in navi componentes retia, 
\verse et statim vocavit illos. Et, relicto patre suo Zebedaeo in navi cum mercennariis, abierunt post eum. 
\verse Et ingrediuntur Capharnaum. Et statim sabbatis ingressus synagogam docebat. 
\verse Et stupebant super doctrina eius: erat enim docens eos quasi potestatem habens et non sicut scribae. 
\verse Et statim erat in synagoga eorum homo in spiritu immundo; et exclamavit 
\verse dicens: “Quid nobis et tibi, Iesu Nazarene? Venisti perdere nos? Scio qui sis: Sanctus Dei". 
\verse Et comminatus est ei Iesus dicens: “Obmutesce et exi de homine!". 
\verse Et discerpens eum spiritus immundus et exclamans voce magna exivit ab eo. 
\verse Et mirati sunt omnes, ita ut conquirerent inter se dicentes: “Quidnam est hoc? Doctrina nova cum potestate; et spiritibus immundis imperat, et oboediunt ei". 
\verse Et processit rumor eius statim ubique in omnem regionem Galilaeae. 
\verse Et protinus egredientes de synagoga venerunt in domum Simonis et Andreae cum Iacobo et Ioanne. 
\verse Socrus autem Simonis decumbebat febricitans; et statim dicunt ei de illa. 
\verse Et accedens elevavit eam apprehensa manu; et dimisit eam febris, et ministrabat eis. 
\verse Vespere autem facto, cum occidisset sol, afferebant ad eum omnes male habentes et daemonia habentes; 
\verse et erat omnis civitas congregata ad ianuam. 
\verse Et curavit multos, qui vexabantur variis languoribus, et daemonia multa eiecit et non sinebat loqui daemonia, quoniam sciebant eum. 
\verse Et diluculo valde mane surgens egressus est et abiit in desertum locum ibique orabat. 
\verse Et persecutus est eum Simon et qui cum illo erant; 
\verse et cum invenissent eum, dixerunt ei: “Omnes quaerunt te!". 
\verse Et ait illis: “Eamus alibi in proximos vicos, ut et ibi praedicem: ad hoc enim veni". 
\verse Et venit praedicans in synagogis eorum per omnem Galilaeam et daemonia eiciens. 
\verse Et venit ad eum leprosus deprecans eum et genu flectens et dicens ei: “Si vis, potes me mundare". 
\verse Et misertus extendens manum suam tetigit eum et ait illi: “Volo, mundare!"; 
\verse et statim discessit ab eo lepra, et mundatus est. 
\verse Et infremuit in eum statimque eiecit illum 
\verse et dicit ei: “Vide, nemini quidquam dixeris; sed vade, ostende te sacerdoti et offer pro emundatione tua, quae praecepit Moyses, in testimonium illis". 
\verse At ille egressus coepit praedicare multum et diffamare sermonem, ita ut iam non posset manifesto in civitatem introire, sed foris in desertis locis erat; et conveniebant ad eum undique. 
\end{biblechapter}

\begin{biblechapter}  
\verse Et iterum intravit Capharnaum post dies, et auditum est quod in domo esset. 
\verse Et convenerunt multi, ita ut non amplius caperentur neque ad ianuam, et loquebatur eis verbum. 
\verse Et veniunt ferentes ad eum paralyticum, qui a quattuor portabatur. 
\verse Et cum non possent offerre eum illi prae turba, nudaverunt tectum, ubi erat, et perfodientes summittunt grabatum, in quo paralyticus iacebat. 
\verse Cum vidisset autem Iesus fidem illorum, ait paralytico: “Fili, dimittuntur peccata tua". 
\verse Erant autem illic quidam de scribis sedentes et cogitantes in cordibus suis:  
\verse “Quid hic sic loquitur? Blasphemat! Quis potest dimittere peccata nisi solus Deus?". 
\verse Quo statim cognito Iesus spiritu suo quia sic cogitarent intra se, dicit illis: “Quid ista cogitatis in cordibus vestris? 
\verse Quid est facilius, dicere paralytico: “Dimittuntur peccata tua”, an dicere: “Surge et tolle grabatum tuum et ambula”? 
\verse Ut autem sciatis quia potestatem habet Filius hominis interra dimittendi peccata — ait paralytico - : 
\verse Tibi dico: Surge, tolle grabatum tuum et vade in domum tuam". 
\verse Et surrexit et protinus sublato grabato abiit coram omnibus, ita ut admirarentur omnes et glorificarent Deum dicentes: “Numquam sic vidimus!". 
\verse Et egressus est rursus ad mare; omnisque turba veniebat ad eum, et docebat eos. 
\verse Et cum praeteriret, vidit Levin Alphaei sedentem ad teloneum et ait illi: “Sequere me". Et surgens secutus est eum. 
\verse Et factum est, cum accumberet in domo illius, et multi publicani et peccatores simul discumbebant cum Iesu et discipulis eius; erant enim multi et sequebantur eum. 
\verse Et scribae pharisaeorum, videntes quia manducaret cum peccatoribus et publicanis, dicebant discipulis eius: “Quare cum publicanis et peccatoribus manducat?".  
\verse Et Iesus hoc audito ait illis: “Non necesse habent sani medicum, sed qui male habent; non veni vocare iustos sed peccatores". 
\verse Et erant discipuli Ioannis et pharisaei ieiunantes. Et veniunt et dicunt illi: “Cur discipuli Ioannis et discipuli pharisaeorum ieiunant, tui autem discipuli non ieiunant?". 
\verse Et ait illis Iesus: “Numquid possunt convivae nuptiarum, quamdiu sponsus cum illis est, ieiunare? Quanto tempore habent secum sponsum, non possunt ieiunare; 
\verse venient autem dies, cum auferetur ab eis sponsus, et tunc ieiunabunt in illa die. 
\verse Nemo assumentum panni rudis assuit vestimento veteri; alioquin supplementum aufert aliquid ab eo, novum a veteri, et peior scissura fit. 
\verse Et nemo mittit vinum novellum in utres veteres, alioquin dirumpet vinum utres et vinum perit et utres; sed vinum novum in utres novos". 
\verse Et factum est, cum ipse sabbatis ambularet per sata, discipuli eius coeperunt praegredi vellentes spicas. 
\verse Pharisaei autem dicebant ei: “Ecce, quid faciunt sabbatis, quod non licet?". 
\verse Et ait illis: “Numquam legistis quid fecerit David, quando necessitatem habuit et esuriit ipse et qui cum eo erant? 
\verse Quomodo introivit in domum Dei sub Abiathar principe sacerdotum et panes propositionis manducavit, quos non licet manducare nisi sacerdotibus, et dedit etiam eis, qui cum eo erant?". 
\verse Et dicebat eis: “Sabbatum propter hominem factum est, et non homo propter sabbatum; 
\verse itaque dominus est Filius hominis etiam sabbati". 
\end{biblechapter}

\begin{biblechapter}  
\verse Et introivit iterum in synagogam. Et erat ibi homo habens manum aridam; 
\verse et observabant eum, si sabbatis curaret illum, ut accusarent eum. 
\verse Et ait homini habenti manum aridam: “Surge in medium". 
\verse Et dicit eis: “Licet sabbatis bene facere an male? Animam salvam facere an perdere?". At illi tacebant. 
\verse Et circumspiciens eos cum ira, contristatus super caecitate cordis eorum, dicit homini: “Extende manum". Et extendit, et restituta est manus eius. 
\verse Et exeuntes pharisaei statim cum herodianis consilium faciebant adversus eum quomodo eum perderent. 
\verse Et Iesus cum discipulis suis secessit ad mare. Et multa turba a Galilaea secuta est et a Iudaea 
\verse et ab Hierosolymis et ab Idumaea; et, qui trans Iordanem et circa Tyrum et Sidonem, multitudo magna, audientes, quae faciebat, venerunt ad eum. 
\verse Et dixit discipulis suis, ut navicula sibi praesto esset propter turbam, ne comprimerent eum. 
\verse Multos enim sanavit, ita ut irruerent in eum, ut illum tangerent, quotquot habebant plagas. 
\verse Et spiritus immundi, cum illum videbant, procidebant ei et clamabant dicentes: “Tu es Filius Dei!". 
\verse Et vehementer comminabatur eis, ne manifestarent illum. 
\verse Et ascendit in montem et vocat ad se, quos voluit ipse, et venerunt ad eum. 
\verse Et fecit Duodecim, ut essent cum illo, et ut mitteret eos praedicare  
\verse habentes potestatem eiciendi daemonia: 
\verse et imposuit Simoni nomen Petrum; 
\verse et Iacobum Zebedaei et Ioannem fratrem Iacobi, et imposuit eis nomina Boanerges, quod est Filii tonitrui; 
\verse et Andream et Philippum et Bartholomaeum et Matthaeum et Thomam et Iacobum Alphaei et Thaddaeum et Simonem Chananaeum 
\verse et Iudam Iscarioth, qui et tradidit illum. 
\verse Et venit ad domum; et convenit iterum turba, ita ut non possent neque panem manducare. 
\verse Et cum audissent sui, exierunt tenere eum; dicebant enim: “In furorem versus est". 
\verse Et scribae, qui ab Hierosolymis descenderant, dicebant: “Beelzebul habet" et: “In principe daemonum eicit daemonia". 
\verse Et convocatis eis, in parabolis dicebat illis: “Quomodo potest Satanas Satanam eicere? 
\verse Et si regnum in se dividatur, non potest stare regnum illud; 
\verse et si domus in semetipsam dispertiatur, non poterit domus illa stare. 
\verse Et si Satanas consurrexit in semetipsum et dispertitus est, non potest stare, sed finem habet. 
\verse Nemo autem potest in domum fortis ingressus vasa eius diripere, nisi prius fortem alliget; et tunc domum eius diripiet. 
\verse Amen dico vobis: Omnia dimittentur filiis hominum peccata et blasphemiae, quibus blasphemaverint;  
\verse qui autem blasphemaverit in Spiritum Sanctum, non habet remissionem in aeternum, sed reus est aeterni delicti". 
\verse Quoniam dicebant: “Spiritum immundum habet". 
\verse Et venit mater eius et fratres eius, et foris stantes miserunt ad eum vocantes eum. 
\verse Et sedebat circa eum turba, et dicunt ei: “Ecce mater tua et fratres tui et sorores tuae foris quaerunt te". 
\verse Et respondens eis ait: “Quae est mater mea et fratres mei?". 
\verse Et circumspiciens eos, qui in circuitu eius sedebant, ait: “Ecce mater mea et fratres mei. 
\verse Qui enim fecerit voluntatem Dei, hic frater meus et soror mea et mater est". 
\end{biblechapter}

\begin{biblechapter}  
\verse Et iterum coepit docere ad mare. Et congregatur ad eum turba plurima, ita ut in navem ascendens sederet in mari, et omnis turba circa mare super terram erant. 
\verse Et docebat eos in parabolis multa et dicebat illis in doctrina sua: 
\verse “Audite. Ecce exiit seminans ad seminandum. 
\verse Et factum est, dum seminat, aliud cecidit circa viam, et venerunt volucres et comederunt illud. 
\verse Aliud cecidit super petrosa, ubi non habebat terram multam, et statim exortum est, quoniam non habebat altitudinem terrae; 
\verse et quando exortus est sol, exaestuavit et, eo quod non haberet radicem, exaruit. 
\verse Et aliud cecidit in spinas, et ascenderunt spinae et suffocaverunt illud, et fructum non dedit. 
\verse Et alia ceciderunt in terram bonam et dabant fructum: ascendebant et crescebant et afferebant unum triginta et unum sexaginta et unum centum". 
\verse Et dicebat: “Qui habet aures audiendi, audiat". 
\verse Et cum esset singularis, interrogaverunt eum hi, qui circa eum erant cum Duodecim, parabolas. 
\verse Et dicebat eis: “Vobis datum est mysterium regni Dei; illis autem, qui foris sunt, in parabolis omnia fiunt, 
\verse ut videntes videant et non videant, et audientes audiant et non intellegant, ne quando convertantur, et dimittatur eis". 
\verse Et ait illis: “Nescitis parabolam hanc, et quomodo omnes parabolas cognoscetis? 
\verse Qui seminat, verbum seminat. 
\verse Hi autem sunt, qui circa viam, ubi seminatur verbum: et cum audierint, confestim venit Satanas et aufert verbum, quod seminatum est in eos. 
\verse Et hi sunt, qui super petrosa seminantur: qui cum audierint verbum, statim cum gaudio accipiunt illud 
\verse et non habent radicem in se, sed temporales sunt; deinde orta tribulatione vel persecutione propter verbum, confestim scandalizantur. 
\verse Et alii sunt, qui in spinis seminantur: hi sunt, qui verbum audierunt, 
\verse et aerumnae saeculi et deceptio divitiarum et circa reliqua concupiscentiae introeuntes suffocant verbum, et sine fructu efficitur. 
\verse Et hi sunt, qui super terram bonam seminati sunt: qui audiunt verbum et suscipiunt et fructificant unum triginta et unum sexaginta et unum centum". 
\verse Et dicebat illis: “Numquid venit lucerna, ut sub modio ponatur aut sub lecto? Nonne ut super candelabrum ponatur? 
\verse Non enim est aliquid absconditum, nisi ut manifestetur, nec factum est occultum, nisi ut in palam veniat. 
\verse Si quis habet aures audiendi, audiat". 
\verse Et dicebat illis: “Videte quid audiatis. In qua mensura mensi fueritis, remetietur vobis et adicietur vobis. 
\verse Qui enim habet, dabitur illi; et, qui non habet, etiam quod habet, auferetur ab illo". 
\verse Et dicebat: “Sic est regnum Dei, quemadmodum si homo iaciat sementem in terram 
\verse et dormiat et exsurgat nocte ac die, et semen germinet et increscat, dum nescit ille. 
\verse Ultro terra fructificat primum herbam, deinde spicam, deinde plenum frumentum in spica. 
\verse Et cum se produxerit fructus, statim mittit falcem, quoniam adest messis". 
\verse Et dicebat: “Quomodo assimilabimus regnum Dei aut in qua parabola ponemus illud? 
\verse Sicut granum sinapis, quod cum seminatum fuerit in terra, minus est omnibus seminibus, quae sunt in terra; 
\verse et cum seminatum fuerit, ascendit et fit maius omnibus holeribus et facit ramos magnos, ita ut possint sub umbra eius aves caeli habitare". 
\verse Et talibus multis parabolis loquebatur eis verbum, prout poterant audire;  
\verse sine parabola autem non loquebatur eis. Seorsum autem discipulis suis disserebat omnia. 
\verse Et ait illis illa die, cum sero esset factum: “Transeamus contra". 
\verse Et dimittentes turbam, assumunt eum, ut erat in navi; et aliae naves erant cum illo. 
\verse Et exoritur procella magna venti, et fluctus se mittebant in navem, ita ut iam impleretur navis. 
\verse Et erat ipse in puppi supra cervical dormiens; et excitant eum et dicunt ei: “Magister, non ad te pertinet quia perimus?". 
\verse Et exsurgens comminatus est vento et dixit mari: “Tace, obmutesce!". Et cessavit ventus, et facta est tranquillitas magna. 
\verse Et ait illis: “Quid timidi estis? Necdum habetis fidem?". 
\verse Et timuerunt magno timore et dicebant ad alterutrum: “Quis putas est iste, quia et ventus et mare oboediunt ei?". 
\end{biblechapter}

\begin{biblechapter}  
\verse Et venerunt trans fretum maris in regionem Gerasenorum. 
\verse Et exeunte eo de navi, statim occurrit ei de monumentis homo in spiritu immundo, 
\verse qui domicilium habebat in monumentis; et neque catenis iam quisquam eum poterat ligare, 
\verse quoniam saepe compedibus et catenis vinctus dirupisset catenas et compedes comminuisset, et nemo poterat eum domare; 
\verse et semper nocte ac die in monumentis et in montibus erat clamans et concidens se lapidibus. 
\verse Et videns Iesum a longe cucurrit et adoravit eum 
\verse et clamans voce magna dicit: “Quid mihi et tibi, Iesu, fili Dei Altissimi? Adiuro te per Deum, ne me torqueas". 
\verse Dicebat enim illi: “Exi, spiritus immunde, ab homine". 
\verse Et interrogabat eum: “Quod tibi nomen est?". Et dicit ei: “Legio nomen mihi est, quia multi sumus". 
\verse Et deprecabatur eum multum, ne se expelleret extra regionem. 
\verse Erat autem ibi circa montem grex porcorum magnus pascens; 
\verse et deprecati sunt eum dicentes: “Mitte nos in porcos, ut in eos introeamus". 
\verse Et concessit eis. Et exeuntes spiritus immundi introierunt in porcos. Et magno impetu grex ruit per praecipitium in mare, ad duo milia, et suffocabantur in mari. 
\verse Qui autem pascebant eos, fugerunt et nuntiaverunt in civitatem et in agros; et egressi sunt videre quid esset facti. 
\verse Et veniunt ad Iesum; et vident illum, qui a daemonio vexabatur, sedentem, vestitum et sanae mentis, eum qui legionem habuerat, et timuerunt. 
\verse Et qui viderant, narraverunt illis qualiter factum esset ei, qui daemonium habuerat, et de porcis. 
\verse Et rogare eum coeperunt, ut discederet a finibus eorum. 
\verse Cumque ascenderet navem, qui daemonio vexatus fuerat, deprecabatur eum, ut esset cum illo. 
\verse Et non admisit eum, sed ait illi: “Vade in domum tuam ad tuos et annuntia illis quanta tibi Dominus fecerit et misertus sit tui". 
\verse Et abiit et coepit praedicare in Decapoli quanta sibi fecisset Iesus, et omnes mirabantur. 
\verse Et cum transcendisset Iesus in navi rursus trans fretum, convenit turba multa ad illum, et erat circa mare. 
\verse Et venit quidam de archisynagogis nomine Iairus et videns eum procidit ad pedes eius 
\verse et deprecatur eum multum dicens: “Filiola mea in extremis est; veni, impone manus super eam, ut salva sit et vivat". 
\verse Et abiit cum illo. Et sequebatur eum turba multa et comprimebant illum. 
\verse Et mulier, quae erat in profluvio sanguinis annis duodecim 
\verse et fuerat multa perpessa a compluribus medicis et erogaverat omnia sua nec quidquam profecerat, sed magis deterius habebat, 
\verse cum audisset de Iesu, venit in turba retro et tetigit vestimentum eius; 
\verse dicebat enim: “Si vel vestimenta eius tetigero, salva ero". 
\verse Et confestim siccatus est fons sanguinis eius, et sensit corpore quod sanata esset a plaga. 
\verse Et statim Iesus cognoscens in semetipso virtutem, quae exierat de eo, conversus ad turbam aiebat: “Quis tetigit vestimenta mea?". 
\verse Et dicebant ei discipuli sui: “Vides turbam comprimentem te et dicis: “Quis me tetigit?”". 
\verse Et circumspiciebat videre eam, quae hoc fecerat. 
\verse Mulier autem timens et tremens, sciens quod factum esset in se, venit et procidit ante eum et dixit ei omnem veritatem. 
\verse Ille autem dixit ei: “Filia, fides tua te salvam fecit. Vade in pace et esto sana a plaga tua". 
\verse Adhuc eo loquente, veniunt ab archisynagogo dicentes: “Filia tua mortua est; quid ultra vexas magistrum?". 
\verse Iesus autem, verbo, quod dicebatur, audito, ait archisynagogo: “Noli timere; tantummodo crede!". 
\verse Et non admisit quemquam sequi se nisi Petrum et Iacobum et Ioannem fratrem Iacobi.  
\verse Et veniunt ad domum archisynagogi; et videt tumultum et flentes et eiulantes multum, 
\verse et ingressus ait eis: “Quid turbamini et ploratis? Puella non est mortua, sed dormit". 
\verse Et irridebant eum. Ipse vero, eiectis omnibus, assumit patrem puellae et matrem et, qui secum erant, et ingreditur, ubi erat puella; 
\verse et tenens manum puellae ait illi: “Talitha, qum!" — quod est interpretatum: “Puella, tibi dico: Surge!" - . 
\verse Et confestim surrexit puella et ambulabat; erat enim annorum duodecim. Et obstupuerunt continuo stupore magno. 
\verse Et praecepit illis vehementer, ut nemo id sciret, et dixit dari illi manducare. 
\end{biblechapter}

\begin{biblechapter}  
\verse Et egressus est inde et venit in patriam suam, et sequuntur illum discipuli sui. 
\verse Et facto sabbato, coepit in synagoga docere; et multi audientes admirabantur dicentes: “Unde huic haec, et quae est sapientia, quae data est illi, et virtutes tales, quae per manus eius efficiuntur? 
\verse Nonne iste est faber, filius Mariae et frater Iacobi et Iosetis et Iudae et Simonis? Et nonne sorores eius hic nobiscum sunt?". Et scandalizabantur in illo. 
\verse Et dicebat eis Iesus: “Non est propheta sine honore nisi in patria sua et in cognatione sua et in domo sua". 
\verse Et non poterat ibi virtutem ullam facere, nisi paucos infirmos impositis manibus curavit; 
\verse et mirabatur propter incredulitatem eorum. Et circumibat castella in circuitu docens. 
\verse Et convocat Duodecim et coepit eos mittere binos et dabat illis potestatem in spiritus immundos; 
\verse et praecepit eis, ne quid tollerent in via nisi virgam tantum: non panem, non peram neque in zona aes, 
\verse sed ut calcearentur sandaliis et ne induerentur duabus tunicis. 
\verse Et dicebat eis: “Quocumque introieritis in domum, illic manete, donec exeatis inde. 
\verse Et quicumque locus non receperit vos nec audierint vos, exeuntes inde excutite pulverem de pedibus vestris in testimonium illis". 
\verse Et exeuntes praedicaverunt, ut paenitentiam agerent; 
\verse et daemonia multa eiciebant et ungebant oleo multos aegrotos et sanabant. 
\verse Et audivit Herodes rex; manifestum enim factum est nomen eius. Et dicebant: “Ioannes Baptista resurrexit a mortuis, et propterea inoperantur virtutes in illo". 
\verse Alii autem dicebant: “Elias est". Alii vero dicebant: “Propheta est, quasi unus ex prophetis". 
\verse Quo audito, Herodes aiebat: “Quem ego decollavi Ioannem, hic resurrexit!". 
\verse Ipse enim Herodes misit ac tenuit Ioannem et vinxit eum in carcere propter Herodiadem uxorem Philippi fratris sui, quia duxerat eam. 
\verse Dicebat enim Ioannes Herodi: “Non licet tibi habere uxorem fratris tui". 
\verse Herodias autem insidiabatur illi et volebat occidere eum nec poterat: 
\verse Herodes enim metuebat Ioannem, sciens eum virum iustum et sanctum, et custodiebat eum, et, audito eo, multum haesitabat et libenter eum audiebat. 
\verse Et cum dies opportunus accidisset, quo Herodes natali suo cenam fecit principibus suis et tribunis et primis Galilaeae, 
\verse cumque introisset filia ipsius Herodiadis et saltasset, placuit Herodi simulque recumbentibus. Rex ait puellae: “Pete a me, quod vis, et dabo tibi". 
\verse Et iuravit illi multum: “Quidquid petieris a me, dabo tibi, usque ad dimidium regni mei". 
\verse Quae cum exisset, dixit matri suae: “Quid petam?". At illa dixit: “Caput Ioannis Baptistae". 
\verse Cumque introisset statim cum festinatione ad regem, petivit dicens: “Volo ut protinus des mihi in disco caput Ioannis Baptistae". 
\verse Et contristatus rex, propter iusiurandum et propter recumbentes noluit eam decipere; 
\verse et statim misso spiculatore rex praecepit afferri caput eius. Et abiens decollavit eum in carcere 
\verse et attulit caput eius in disco; et dedit illud puellae, et puella dedit illud matri suae. 
\verse Quo audito, discipuli eius venerunt et tulerunt corpus eius et posuerunt illud in monumento. 
\verse Et convenientes apostoli ad Iesum renuntiaverunt illi omnia, quae egerant et docuerant. 
\verse Et ait illis: “Venite vos ipsi seorsum in desertum locum et requiescite pusillum". Erant enim, qui veniebant et redibant, multi, et nec manducandi spatium habebant. 
\verse Et abierunt in navi in desertum locum seorsum. 
\verse Et viderunt eos abeuntes et cognoverunt multi; et pedestre de omnibus civitatibus concurrerunt illuc et praevenerunt eos. 
\verse Et exiens vidit multam turbam et misertus est super eos, quia erant sicut oves non habentes pastorem, et coepit docere illos multa. 
\verse Et cum iam hora multa facta esset, accesserunt discipuli eius dicentes: “Desertus est locus hic, et hora iam est multa; 
\verse dimitte illos, ut euntes in villas et vicos in circuitu emant sibi, quod manducent". 
\verse Respondens autem ait illis: “Date illis vos manducare". Et dicunt ei: “Euntes emamus denariis ducentis panes et dabimus eis manducare?". 
\verse Et dicit eis: “Quot panes habetis? Ite, videte". Et cum cognovissent, dicunt: “Quinque et duos pisces". 
\verse Et praecepit illis, ut accumbere facerent omnes secundum contubernia super viride fenum.  
\verse Et discubuerunt secundum areas per centenos et per quinquagenos. 
\verse Et acceptis quinque panibus et duobus piscibus, intuens in caelum benedixit et fregit panes et dabat discipulis suis, ut ponerent ante eos; et duos pisces divisit omnibus. 
\verse Et manducaverunt omnes et saturati sunt; 
\verse et sustulerunt fragmenta duodecim cophinos plenos, et de piscibus. 
\verse Et erant, qui manducaverunt panes, quinque milia virorum. 
\verse Et statim coegit discipulos suos ascendere navem, ut praecederent trans fretum ad Bethsaidam, dum ipse dimitteret populum. 
\verse Et cum dimisisset eos, abiit in montem orare. 
\verse Et cum sero factum esset, erat navis in medio mari, et ipse solus in terra. 
\verse Et videns eos laborantes in remigando, erat enim ventus contrarius eis, circa quartam vigiliam noctis venit ad eos ambulans super mare et volebat praeterire eos. 
\verse At illi, ut viderunt eum ambulantem super mare, putaverunt phantasma esse et exclamaverunt; 
\verse omnes enim eum viderunt et conturbati sunt. Statim autem locutus est cum eis et dicit illis: “Confidite, ego sum; nolite timere!". 
\verse Et ascendit ad illos in navem, et cessavit ventus. Et valde nimis intra se stupebant; 
\verse non enim intellexerant de panibus, sed erat cor illorum obcaecatum. 
\verse Et cum transfretassent in terram, pervenerunt Gennesaret et applicuerunt.  
\verse Cumque egressi essent de navi, continuo cognoverunt eum 
\verse et percurrentes universam regionem illam coeperunt in grabatis eos, qui se male habebant, circumferre, ubi audiebant eum esse. 
\verse Et quocumque introibat in vicos aut in civitates vel in villas, in plateis ponebant infirmos; et deprecabantur eum, ut vel fimbriam vestimenti eius tangerent; et, quotquot tangebant eum, salvi fiebant. 
\end{biblechapter}

\begin{biblechapter}  
\verse Et conveniunt ad eum pharisaei et quidam de scribis venientes ab Hierosolymis; 
\verse et cum vidissent quosdam ex discipulis eius communibus manibus, id est non lotis, manducare panes 
\verse — pharisaei enim et omnes Iudaei, nisi pugillo lavent manus, non manducant, tenentes traditionem seniorum;  
\verse et a foro nisi baptizentur, non comedunt; et alia multa sunt, quae acceperunt servanda: baptismata calicum et urceorum et aeramentorum et lectorum — 
\verse et interrogant eum pharisaei et scribae: “Quare discipuli tui non ambulant iuxta traditionem seniorum, sed communibus manibus manducant panem?". 
\verse At ille dixit eis: “Bene prophetavit Isaias de vobis hypocritis, sicut scriptum est: “Populus hic labiis me honorat, cor autem eorum longe est a me; 
\verse in vanum autem me colunt docentes doctrinas praecepta hominum”. 
\verse Relinquentes mandatum Dei tenetis traditionem hominum". 
\verse Et dicebat illis: “Bene irritum facitis praeceptum Dei, ut traditionem vestram servetis.  
\verse Moyses enim dixit: “Honora patrem tuum et matrem tuam” et: “Qui maledixerit patri aut matri, morte moriatur”; 
\verse vos autem dicitis: “Si dixerit homo patri aut matri: Corban, quod est donum, quodcumque ex me tibi profuerit”,  
\verse ultra non permittitis ei facere quidquam patri aut matri 
\verse rescindentes verbum Dei per traditionem vestram, quam tradidistis; et similia huiusmodi multa facitis". 
\verse Et advocata iterum turba, dicebat illis: “Audite me, omnes, et intellegite:  
\verse Nihil est extra hominem introiens in eum, quod possit eum coinquinare; sed quae de homine procedunt, illa sunt, quae coinquinant hominem!". (16) 
\verse Et cum introisset in domum a turba, interrogabant eum discipuli eius parabolam. 
\verse Et ait illis: “Sic et vos imprudentes estis? Non intellegitis quia omne extrinsecus introiens in hominem non potest eum coinquinare, 
\verse quia non introit in cor eius sed in ventrem et in secessum exit?", purgans omnes escas. 
\verse Dicebat autem: “Quod de homine exit, illud coinquinat hominem; 
\verse ab intus enim de corde hominum cogitationes malae procedunt, fornicationes, furta, homicidia, 
\verse adulteria, avaritiae, nequitiae, dolus, impudicitia, oculus malus, blasphemia, superbia, stultitia:  
\verse omnia haec mala ab intus procedunt et coinquinant hominem". 
\verse Inde autem surgens abiit in fines Tyri et Sidonis. Et ingressus domum neminem voluit scire et non potuit latere. 
\verse Sed statim ut audivit de eo mulier, cuius habebat filia spiritum immundum, veniens procidit ad pedes eius. 
\verse Erat autem mulier Graeca, Syrophoenissa genere. Et rogabat eum, ut daemonium eiceret de filia eius. 
\verse Et dicebat illi: “Sine prius saturari filios; non est enim bonum sumere panem filiorum et mittere catellis". 
\verse At illa respondit et dicit ei: “Domine, etiam catelli sub mensa comedunt de micis puerorum". 
\verse Et ait illi: “Propter hunc sermonem vade; exiit daemonium de filia tua". 
\verse Et cum abisset domum suam, invenit puellam iacentem supra lectum et daemonium exisse. 
\verse Et iterum exiens de finibus Tyri venit per Sidonem ad mare Galilaeae inter medios fines Decapoleos. 
\verse Et adducunt ei surdum et mutum et deprecantur eum, ut imponat illi manum. 
\verse Et apprehendens eum de turba seorsum misit digitos suos in auriculas eius et exspuens tetigit linguam eius 
\verse et suspiciens in caelum ingemuit et ait illi: “Effetha", quod est: “Adaperire".  
\verse Et statim apertae sunt aures eius, et solutum est vinculum linguae eius, et loquebatur recte. 
\verse Et praecepit illis, ne cui dicerent; quanto autem eis praecipiebat, tanto magis plus praedicabant. 
\verse Et eo amplius admirabantur dicentes: “Bene omnia fecit, et surdos facit audire et mutos loqui!". 
\end{biblechapter}

\begin{biblechapter}  
\verse In illis diebus iterum cum turba multa esset nec haberent, quod manducarent, convocatis discipulis, ait illis: 
\verse “Misereor super turbam, quia iam triduo sustinent me nec habent, quod manducent; 
\verse et si dimisero eos ieiunos in domum suam, deficient in via; et quidam ex eis de longe venerunt". 
\verse Et responderunt ei discipuli sui: “Unde istos poterit quis hic saturare panibus in solitudine?". 
\verse Et interrogabat eos: “Quot panes habetis?". Qui dixerunt: “Septem". 
\verse Et praecipit turbae discumbere supra terram; et accipiens septem panes, gratias agens fregit et dabat discipulis suis, ut apponerent; et apposuerunt turbae. 
\verse Et habebant pisciculos paucos; et benedicens eos, iussit hos quoque apponi. 
\verse Et manducaverunt et saturati sunt; et sustulerunt, quod superaverat de fragmentis, septem sportas. 
\verse Erant autem quasi quattuor milia. Et dimisit eos. 
\verse Et statim ascendens navem cum discipulis suis venit in partes Dalmanutha.  
\verse Et exierunt pharisaei et coeperunt conquirere cum eo quaerentes ab illo signum de caelo, tentantes eum. 
\verse Et ingemiscens spiritu suo ait: “Quid generatio ista quaerit signum? Amen dico vobis: Non dabitur generationi isti signum". 
\verse Et dimittens eos, iterum ascendens abiit trans fretum. 
\verse Et obliti sunt sumere panes et nisi unum panem non habebant secum in navi.  
\verse Et praecipiebat eis dicens: “Videte, cavete a fermento pharisaeorum et fermento Herodis!". 
\verse Et disputabant ad invicem, quia panes non haberent.  
\verse Quo cognito, ait illis: “Quid disputatis, quia panes non habetis? Nondum cognoscitis nec intellegitis? Caecatum habetis cor vestrum? 
\verse Oculos habentes non videtis, et aures habentes non auditis? Nec recordamini, 
\verse quando quinque panes fregi in quinque milia, quot cophinos fragmentorum plenos sustulistis?". Dicunt ei: “Duodecim". 
\verse “Quando illos septem in quattuor milia, quot sportas plenas fragmentorum tulistis?". Et dicunt ei: “Septem". 
\verse Et dicebat eis: “Nondum intellegitis?". 
\verse Et veniunt Bethsaida. Et adducunt ei caecum et rogant eum, ut illum tangat. 
\verse Et apprehendens manum caeci eduxit eum extra vicum; et exspuens in oculos eius, impositis manibus ei, interrogabat eum: “Vides aliquid?". 
\verse Et aspiciens dicebat: “Video homines, quia velut arbores video ambulantes". 
\verse Deinde iterum imposuit manus super oculos eius; et coepit videre et restitutus est et videbat clare omnia. 
\verse Et misit illum in domum suam dicens: “Nec in vicum introieris". 
\verse Et egressus est Iesus et discipuli eius in castella Caesareae Philippi; et in via interrogabat discipulos suos dicens eis: “Quem me dicunt esse homines?".  
\verse Qui responderunt illi dicentcs: “Ioannem Baptistam, alii Eliam, alii vero unum de prophetis". 
\verse Et ipse interrogabat eos: “Vos vero quem me dicitis esse?". Respondens Petrus ait ei: “Tu es Christus". 
\verse Et comminatus est eis, ne cui dicerent de illo. 
\verse Et coepit docere illos: “Oportet Filium hominis multa pati et reprobari a senioribus et a summis sacerdotibus et scribis et occidi et post tres dies resurgere"; 
\verse et palam verbum loquebatur. Et apprehendens eum Petrus coepit increpare eum. 
\verse Qui conversus et videns discipulos suos comminatus est Petro et dicit: “Vade retro me, Satana, quoniam non sapis, quae Dei sunt, sed quae sunt hominum". 
\verse Et convocata turba cum discipulis suis, dixit eis: “Si quis vult post me sequi, deneget semetipsum et tollat crucem suam et sequatur me. 
\verse Qui enim voluerit animam suam salvam facere, perdet eam; qui autem perdiderit animam suam propter me et evangelium, salvam eam faciet. 
\verse Quid enim prodest homini, si lucretur mundum totum et detrimentum faciat animae suae? 
\verse Quid enim dabit homo commutationem pro anima sua? 
\verse Qui enim me confusus fuerit et mea verba in generatione ista adultera et peccatrice, et Filius hominis confundetur eum, cum venerit in gloria Patris sui cum angelis sanctis". 
\end{biblechapter}

\begin{biblechapter}  
\verse Et dicebat illis: “Amen dico vobis: Sunt quidam de hic stantibus, qui non gustabunt mortem, donec videant regnum Dei venisse in virtute". 
\verse Et post dies sex assumit Iesus Petrum et Iacobum et Ioannem, et ducit illos in montem excelsum seorsum solos. Et transfiguratus est coram ipsis; 
\verse et vestimenta eius facta sunt splendentia, candida nimis, qualia fullo super terram non potest tam candida facere. 
\verse Et apparuit illis Elias cum Moyse, et erant loquentes cum Iesu. 
\verse Et respondens Petrus ait Iesu: “Rabbi, bonum est nos hic esse; et faciamus tria tabernacula: tibi unum et Moysi unum et Eliae unum".  
\verse Non enim sciebat quid responderet; erant enim exterriti. 
\verse Et facta est nubes obumbrans eos, et venit vox de nube: “Hic est Filius meus dilectus; audite illum". 
\verse Et statim circumspicientes neminem amplius viderunt nisi Iesum tantum secum. 
\verse Et descendentibus illis de monte, praecepit illis, ne cui, quae vidissent, narrarent, nisi cum Filius hominis a mortuis resurrexerit. 
\verse Et verbum continuerunt apud se, conquirentes quid esset illud: “a mortuis resurgere". 
\verse Et interrogabant eum dicentes: “Quid ergo dicunt scribae quia Eliam oporteat venire primum?". 
\verse Qui ait illis: “Elias veniens primo, restituit omnia; et quomodo scriptum est super Filio hominis, ut multa patiatur et contemnatur? 
\verse Sed dico vobis: Et Elias venit; et fecerunt illi, quaecumque volebant, sicut scriptum est de eo". 
\verse Et venientes ad discipulos viderunt turbam magnam circa eos et scribas conquirentes cum illis. 
\verse Et confestim omnis populus videns eum stupefactus est, et accurrentes salutabant eum. 
\verse Et interrogavit eos: “Quid inter vos conquiritis?". 
\verse Et respondit ei unus de turba: “Magister, attuli filium meum ad te habentem spiritum mutum; 
\verse et ubicumque eum apprehenderit, allidit eum, et spumat et stridet dentibus et arescit. Et dixi discipulis tuis, ut eicerent illum, et non potuerunt". 
\verse Qui respondens eis dicit: “O generatio incredula, quamdiu apud vos ero? Quamdiu vos patiar? Afferte illum ad me". 
\verse Et attulerunt illum ad eum. Et cum vidisset illum, spiritus statim conturbavit eum; et corruens in terram volutabatur spumans. 
\verse Et interrogavit patrem eius: “Quantum temporis est, ex quo hoc ei accidit?". At ille ait: “Ab infantia; 
\verse et frequenter eum etiam in ignem et in aquas misit, ut eum perderet; sed si quid potes, adiuva nos, misertus nostri". 
\verse Iesus autem ait illi: ““Si potes!”. Omnia possibilia credenti". 
\verse Et continuo exclamans pater pueri aiebat: “Credo; adiuva incredulitatem meam".  
\verse Et cum videret Iesus concurrentem turbam, comminatus est spiritui immundo dicens illi: “Mute et surde spiritus, ego tibi praecipio: Exi ab eo et amplius ne introeas in eum". 
\verse Et clamans et multum discerpens eum exiit; et factus est sicut mortuus, ita ut multi dicerent: “Mortuus est!". 
\verse Iesus autem tenens manum eius elevavit illum, et surrexit. 
\verse Et cum introisset in domum, discipuli eius secreto interrogabant eum: “Quare nos non potuimus eicere eum?". 
\verse Et dixit illis: “Hoc genus in nullo potest exire nisi in oratione". 
\verse Et inde profecti peragrabant Galilaeam; nec volebat quemquam scire. 
\verse Docebat enim discipulos suos et dicebat illis: “Filius hominis traditur in manus hominum, et occident eum, et occisus post tres dies resurget". 
\verse At illi ignorabant verbum et timebant eum interrogare. 
\verse Et venerunt Capharnaum. Qui cum domi esset, interrogabat eos: “Quid in via tractabatis?". 
\verse At illi tacebant. Siquidem inter se in via disputaverant, quis esset maior. 
\verse Et residens vocavit Duodecim et ait illis: “Si quis vult primus esse, erit omnium novissimus et omnium minister". 
\verse Et accipiens puerum, statuit eum in medio eorum; quem ut complexus esset, ait illis: 
\verse “Quisquis unum ex huiusmodi pueris receperit in nomine meo, me recipit; et, quicumque me susceperit, non me suscipit, sed eum qui me misit". 
\verse Dixit illi Ioannes: “Magister, vidimus quendam in nomine tuo eicientem daemonia, et prohibebamus eum, quia non sequebatur nos". 
\verse Iesus autem ait: “Nolite prohibere eum. Nemo est enim, qui faciat virtutem in nomine meo et possit cito male loqui de me; 
\verse qui enim non est adversum nos, pro nobis est. 
\verse Quisquis enim potum dederit vobis calicem aquae in nomine, quia Christi estis, amen dico vobis: Non perdet mercedem suam. 
\verse Et quisquis scandalizaverit unum ex his pusillis credentibus in me, bonum est ei magis, ut circumdetur mola asinaria collo eius, et in mare mittatur. 
\verse Et si scandalizaverit te manus tua, abscide illam: bonum est tibi debilem introire in vitam, quam duas manus habentem ire in gehennam, in ignem inexstinguibilem. (44) 
\verse Et si pes tuus te scandalizat, amputa illum: bonum est tibi claudum introire in vitam, quam duos pedes habentem mitti in gehennam. (46) 
\verse Et si oculus tuus scandalizat te, eice eum: bonum est tibi luscum introire in regnum Dei, quam duos oculos habentem mitti in gehennam, 
\verse ubi vermis eorum non moritur, et ignis non exstinguitur; 
\verse omnis enim igne salietur. 
\verse Bonum est sal; quod si sal insulsum fuerit, in quo illud condietis? Habete in vobis sal et pacem habete inter vos". 
\end{biblechapter}

\begin{biblechapter}  
\verse Et inde exsurgens venit in fines Iudaeae ultra Iordanem; et conveniunt iterum turbae ad eum, et, sicut consueverat, iterum docebat illos. 
\verse Et accedentes pharisaei interrogabant eum, si licet viro uxorem dimittere, tentantes eum. 
\verse At ille respondens dixit eis: “Quid vobis praecepit Moyses?". 
\verse Qui dixerunt: “Moyses permisit libellum repudii scribere et dimittere". 
\verse Iesus autem ait eis: “Ad duritiam cordis vestri scripsit vobis praeceptum istud. 
\verse Ab initio autem creaturae masculum et feminam fecit eos. 
\verse Propter hoc relinquet homo patrem suum et matrem et adhaerebit ad uxorern suam, 
\verse et erunt duo in carne una; itaque iam non sunt duo sed una caro. 
\verse Quod ergo Deus coniunxit, homo non separet". 
\verse Et domo iterum discipuli de hoc interrogabant eum. 
\verse Et dicit illis: “Quicumque dimiserit uxorem suam et aliam duxerit, adulterium committit in eam; 
\verse et si ipsa dimiserit virum suum et alii nupserit, moechatur". 
\verse Et offerebant illi parvulos, ut tangeret illos; discipuli autem comminabantur eis. 
\verse At videns Iesus, indigne tulit et ait illis: “Sinite parvulos venire ad me. Ne prohibueritis eos; talium est enim regnum Dei. 
\verse Amen dico vobis: Quisquis non receperit regnum Dei velut parvulus, non intrabit in illud". 
\verse Et complexans eos benedicebat imponens manus super illos. 
\verse Et cum egrederetur in viam, accurrens quidam et, genu flexo ante eum, rogabat eum: “Magister bone, quid faciam ut vitam aeternam percipiam?". 
\verse Iesus autem dixit ei: “Quid me dicis bonum? Nemo bonus, nisi unus Deus. 
\verse Praecepta nosti: ne occidas, ne adulteres, ne fureris, ne falsum testimonium dixeris, ne fraudem feceris, honora patrem tuum et matrem". 
\verse Ille autem dixit ei: “Magister, haec omnia conservavi a iuventute mea". 
\verse Iesus autem intuitus eum dilexit eum et dixit illi: “Unum tibi deest: vade, quaecumque habes, vende et da pauperibus et habebis thesaurum in caelo; et veni, sequere me". 
\verse Qui contristatus in hoc verbo, abiit maerens: erat enim habens possessiones multas. 
\verse Et circumspiciens Iesus ait discipulis suis: “Quam difficile, qui pecunias habent, in regnum Dei introibunt". 
\verse Discipuli autem obstupescebant in verbis eius. At Iesus rursus respondens ait illis: “Filii, quam diffficile est in regnum Dei introire. 
\verse Facilius est camelum per foramen acus transire quam divitem intrare in regnum Dei". 
\verse Qui magis admirabantur dicentes ad semetipsos: “Et quis potest salvus fieri?". 
\verse Intuens illos Iesus ait: “Apud homines impossibile est sed non apud Deum: omnia enim possibilia sunt apud Deum". 
\verse Coepit Petrus ei dicere: “Ecce nos dimisimus omnia et secuti sumus te". 
\verse Ait Iesus: “Amen dico vobis: Nemo est, qui reliquerit domum aut fratres aut sorores aut matrem aut patrem aut filios aut agros propter me et propter evangelium, 
\verse qui non accipiat centies tantum nunc in tempore hoc, domos et fratres et sorores et matres et filios et agros cum persecutionibus, et in saeculo futuro vitam aeternam. 
\verse Multi autem erunt primi novissimi, et novissimi primi". 
\verse Erant autem in via ascendentes in Hierosolymam, et praecedebat illos Iesus, et stupebant; illi autem sequentes timebant. Et assumens iterum Duodecim coepit illis dicere, quae essent ei eventura: 
\verse “Ecce ascendimus in Hierosolymam; et Filius hominis tradetur principibus sacerdotum et scribis, et damnabunt eum morte et tradent eum gentibus 
\verse et illudent ei et conspuent eum et flagellabunt eum et interficient eum, et post tres dies resurget". 
\verse Et accedunt ad eum Iacobus et Ioannes filii Zebedaei dicentes ei: “Magister, volumus, ut quodcumque petierimus a te, facias nobis". 
\verse At ille dixit eis: “Quid vultis, ut faciam vobis?". 
\verse Illi autem dixerunt ei: “Da nobis, ut unus ad dexteram tuam et alius ad sinistram sedeamus in gloria tua".  
\verse Iesus autem ait eis: “Nescitis quid petatis. Potestis bibere calicem, quem ego bibo, aut baptismum, quo ego baptizor, baptizari?". 
\verse At illi dixerunt ei: “Possumus". Iesus autem ait eis: “Calicem quidem, quem ego bibo, bibetis et baptismum, quo ego baptizor, baptizabimini; 
\verse sedere autem ad dexteram meam vel ad sinistram non est meum dare, sed quibus paratum est". 
\verse Et audientes decem coeperunt indignari de Iacobo et Ioanne. 
\verse Et vocans eos Iesus ait illis: “Scitis quia hi, qui videntur principari gentibus, dominantur eis, et principes eorum potestatem habent ipsorum. 
\verse Non ita est autem in vobis, sed quicumque voluerit fieri maior inter vos, erit vester minister; 
\verse et, quicumque voluerit in vobis primus esse, erit omnium servus; 
\verse nam et Filius hominis non venit, ut ministraretur ei, sed ut ministraret et daret animam suam redemptionem pro multis". 
\verse Et veniunt Ierichum. Et proficiscente eo de Iericho et discipulis eius et plurima multitudine, filius Timaei Bartimaeus caecus sedebat iuxta viam mendicans. 
\verse Qui cum audisset quia Iesus Nazarenus est, coepit clamare et dicere: “Fili David Iesu, miserere mei!". 
\verse Et comminabantur ei multi, ut taceret; at ille multo magis clamabat: “Fili David, miserere mei!". 
\verse Et stans Iesus dixit: “Vocate illum". Et vocant caecum dicentes ei: “Animaequior esto. Surge, vocat te". 
\verse Qui, proiecto vestimento suo, exsiliens venit ad Iesum. 
\verse Et respondens ei Iesus dixit: “Quid vis tibi faciam?". Caecus autem dixit ei: “Rabboni, ut videam". 
\verse Et Iesus ait illi: “Vade; fides tua te salvum fecit". Et confestim vidit et sequebatur eum in via. 
\end{biblechapter}

\begin{biblechapter}  
\verse Et cum appropinquarent Hierosolymae, Bethphage et Bethaniae ad montem Olivarum, mittit duos ex discipulis suis 
\verse et ait illis: “Ite in castellum, quod est contra vos, et statim introeuntes illud invenietis pullum ligatum, super quem nemo adhuc hominum sedit; solvite illum et adducite. 
\verse Et si quis vobis dixerit: “Quid facitis hoc?”, dicite: “Domino necessarius est, et continuo illum remittit iterum huc”". 
\verse Et abeuntes invenerunt pullum ligatum ante ianuam foris in bivio et solvunt eum. 
\verse Et quidam de illic stantibus dicebant illis: “Quid facitis solventes pullum?". 
\verse Qui dixerunt eis, sicut dixerat Iesus; et dimiserunt eis. 
\verse Et ducunt pullum ad Iesum et imponunt illi vestimenta sua; et sedit super eum. 
\verse Et multi vestimenta sua straverunt in via, alii autem frondes, quas exciderant in agris. 
\verse Et qui praeibant et qui sequebantur, clamabant: “Hosanna! Benedictus, qui venit in nomine Domini! 
\verse Benedictum, quod venit regnum patris nostri David! Hosanna in excelsis!". 
\verse Et introivit Hierosolymam in templum; et circumspectis omnibus, cum iam vespera esset hora, exivit in Bethaniam cum Duodecim. 
\verse Et altera die cum exirent a Bethania, esuriit. 
\verse Cumque vidisset a longe ficum habentem folia, venit si quid forte inveniret in ea; et cum venisset ad eam, nihil invenit praeter folia: non enim erat tempus ficorum. 
\verse Et respondens dixit ei: “Iam non amplius in aeternum quisquam fructum ex te manducet". Et audiebant discipuli eius. 
\verse Et veniunt Hierosolymam. Et cum introisset in templum, coepit eicere vendentes et ementes in templo et mensas nummulariorum et cathedras vendentium columbas evertit; 
\verse et non sinebat, ut quisquam vas transferret per templum. 
\verse Et docebat dicens eis: “Non scriptum est: “Domus mea domus orationis vocabitur omnibus gentibus”? Vos autem fecistis eam speluncam latronum". 
\verse Quo audito, principes sacerdotum et scribae quaerebant quomodo eum perderent; timebant enim eum, quoniam universa turba admirabatur super doctrina eius.  
\verse Et cum vespera facta esset, egrediebantur de civitate. 
\verse Et cum mane transirent, viderunt ficum aridam factam a radicibus. 
\verse Et recordatus Petrus dicit ei: “Rabbi, ecce ficus, cui maledixisti, aruit". 
\verse Et respondens Iesus ait illis: “Habete fidem Dei! 
\verse Amen dico vobis: Quicumque dixerit huic monti: “Tollere et mittere in mare”, et non haesitaverit in corde suo, sed crediderit quia, quod dixerit, fiat, fiet ei.  
\verse Propterea dico vobis: Omnia, quaecumque orantes petitis, credite quia iam accepistis, et erunt vobis. 
\verse Et cum statis in oratione, dimittite, si quid habetis adversus aliquem, ut et Pater vester, qui in caelis est, dimittat vobis peccata vestra". (26) 
\verse Et veniunt rursus Hierosolymam. Et cum ambularet in templo, accedunt ad eum summi sacerdotes et scribae et seniores 
\verse et dicebant illi: “In qua potestate haec facis? Vel quis tibi dedit hanc potestatem, ut ista facias?". 
\verse Iesus autem ait illis: “Interrogabo vos unum verbum, et respondete mihi; et dicam vobis, in qua potestate haec faciam: 
\verse Baptismum Ioannis de caelo erat an ex hominibus? Respondete mihi". 
\verse At illi cogitabant secum dicentes: “Si dixerimus: “De caelo”, dicet: “Quare ergo non credidistis ei?”; 
\verse si autem dixerimus: “Ex hominibus?”". Timebant populum: omnes enim habebant Ioannem quia vere propheta esset. 
\verse Et respondentes dicunt Iesu: “Nescimus". Et Iesus ait illis: “Neque ego dico vobis in qua potestate haec faciam". 
\end{biblechapter}

\begin{biblechapter}  
\verse Et coepit illis in parabolis loqui: “Vineam pastinavit homo et circumdedit saepem et fodit lacum et aedificavit turrim et locavit eam agricolis et peregre profectus est. 
\verse Et misit ad agricolas in tempore servum, ut ab agricolis acciperet de fructu vineae; 
\verse qui apprehensum eum caeciderunt et dimiserunt vacuum. 
\verse Et iterum misit ad illos alium servum; et illum in capite vulneraverunt et contumeliis affecerunt. 
\verse Et alium misit, et illum occiderunt, et plures alios, quosdam caedentes, alios vero occidentes. 
\verse Adhuc unum habebat, filium dilectum. Misit illum ad eos novissimum dicens: “Reverebuntur filium meum”. 
\verse Coloni autem illi dixerunt ad invicem: “Hic est heres. Venite, occidamus eum, et nostra erit hereditas”. 
\verse Et apprehendentes eum occiderunt et eiecerunt extra vineam. 
\verse Quid ergo faciet dominus vineae? Veniet et perdet colonos et dabit vineam aliis. 
\verse Nec Scripturam hanc legistis: “Lapidem quem reprobaverunt aedificantes, hic factus est in caput anguli; 
\verse a Domino factum est istud et est mirabile in oculis nostris”?". 
\verse Et quaerebant eum tenere et timuerunt turbam; cognoverunt enim quoniam ad eos parabolam hanc dixerit. Et relicto eo abierunt. 
\verse Et mittunt ad eum quosdam ex pharisaeis et herodianis, ut eum caperent in verbo. 
\verse Qui venientes dicunt ei: “Magister, scimus quia verax es et non curas quemquam; nec enim vides in faciem hominum, sed in veritate viam Dei doces. Licet dare tributum Caesari an non? Dabimus an non dabimus?". 
\verse Qui sciens versutiam eorum ait illis: “Quid me tentatis? Afferte mihi denarium, ut videam". 
\verse At illi attulerunt. Et ait illis: “Cuius est imago haec et inscriptio?". Illi autem dixerunt ei: “Caesaris". 
\verse Iesus autem dixit illis: “Quae sunt Caesaris, reddite Caesari et, quae sunt Dei, Deo". Et mirabantur super eo. 
\verse Et veniunt ad eum sadducaei, qui dicunt resurrectionem non esse, et interrogabant eum dicentes: 
\verse “Magister, Moyses nobis scripsit, ut si cuius frater mortuus fuerit et reliquerit uxorem et filium non reliquerit, accipiat frater eius uxorem et resuscitet semen fratri suo. 
\verse Septem fratres erant: et primus accepit uxorem et moriens non reliquit semen; 
\verse et secundus accepit eam et mortuus est, non relicto semine; et tertius similiter;  
\verse et septem non reliquerunt semen. Novissima omnium defuncta est et mulier.  
\verse In resurrectione, cum resurrexerint, cuius de his erit uxor? Septem enim habuerunt eam uxorem". 
\verse Ait illis Iesus: “Non ideo erratis, quia non scitis Scripturas neque virtutem Dei? 
\verse Cum enim a mortuis resurrexerint, neque nubent neque nubentur, sed sunt sicut angeli in caelis. 
\verse De mortuis autem quod resurgant, non legistis in libro Moysis super rubum, quomodo dixerit illi Deus inquiens: “Ego sum Deus Abraham et Deus Isaac et Deus Iacob”? 
\verse Non est Deus mortuorum sed vivorum! Multum erratis". 
\verse Et accessit unus de scribis, qui audierat illos conquirentes, videns quoniam bene illis responderit, interrogavit eum: “Quod est primum omnium mandatum?".  
\verse Iesus respondit: “Primum est: “Audi, Israel: Dominus Deus noster Dominus unus est, 
\verse et diliges Dominum Deum tuum ex toto corde tuo et ex tota anima tua et ex tota mente tua et ex tota virtute tua”. 
\verse Secundum est illud: “Diliges proximum tuum tamquam teipsum”. Maius horum aliud mandatum non est".  
\verse Et ait illi scriba: “Bene, Magister, in veritate dixisti: “Unus est, et non est alius praeter eum; 
\verse et diligere eum ex toto corde et ex toto intellectu et ex tota fortitudine” et: “Diligere proximum tamquam seipsum” maius est omnibus holocautomatibus et sacrificiis". 
\verse Et Iesus videns quod sapienter respondisset, dixit illi: “Non es longe a regno Dei". Et nemo iam audebat eum interrogare. 
\verse Et respondens Iesus dicebat docens in templo: “Quomodo dicunt scribae Christum filium esse David? 
\verse Ipse David dixit in Spiritu Sancto: “Dixit Dominus Domino meo: Sede a dextris meis, donec ponam inimicos tuos sub pedibus tuis”. 
\verse Ipse David dicit eum Dominum, et unde est filius eius?". Et multa turba eum libenter audiebat. 
\verse Et dicebat in doctrina sua: “Cavete a scribis, qui volunt in stolis ambulare et salutari in foro 
\verse et in primis cathedris sedere in synagogis et primos discubitus in cenis; 
\verse qui devorant domos viduarum et ostentant prolixas orationes. Hi accipient amplius iudicium". 
\verse Et sedens contra gazophylacium aspiciebat quomodo turba iactaret aes in gazophylacium; et multi divites iactabant multa. 
\verse Et cum venisset una vidua pauper, misit duo minuta, quod est quadrans. 
\verse Et convocans discipulos suos ait illis: “Amen dico vobis: Vidua haec pauper plus omnibus misit, qui miserunt in gazophylacium: 
\verse Omnes enim ex eo, quod abundabat illis, miserunt; haec vero de penuria sua omnia, quae habuit, misit, totum victum suum". 
\end{biblechapter}

\begin{biblechapter}  
\verse Et cum egrederetur de templo, ait illi unus ex discipulis suis: “Magister, aspice quales lapides et quales structurae". 
\verse Et Iesus ait illi: “Vides has magnas aedificationes? Hic non relinquetur lapis super lapidem, qui non destruatur". 
\verse Et cum sederet in montem Olivarum contra templum, interrogabat eum separatim Petrus et Iacobus et Ioannes et Andreas: 
\verse “Dic nobis: Quando ista erunt, et quod signum erit, quando haec omnia incipient consummari?". 
\verse Iesus autem coepit dicere illis: “Videte, ne quis vos seducat. 
\verse Multi venient in nomine meo dicentes: “Ego sum”, et multos seducent. 
\verse Cum audieritis autem bella et opiniones bellorum, ne timueritis; oportet fieri sed nondum finis. 
\verse Exsurget enim gens super gentem, et regnum super regnum, erunt terrae motus per loca, erunt fames; initium dolorum haec. 
\verse Videte autem vosmetipsos. Tradent vos conciliis, et in synagogis vapulabitis et ante praesides et reges stabitis propter me in testimonium illis. 
\verse Et in omnes gentes primum oportet praedicari evangelium. 
\verse Et cum duxerint vos tradentes, nolite praecogitare quid loquamini, sed, quod datum vobis fuerit in illa hora, id loquimini: non enim estis vos loquentes sed Spiritus Sanctus.  
\verse Et tradet frater fratrem in mortem, et pater filium; et consurgent filii in parentes et morte afficient eos; 
\verse et eritis odio omnibus propter nomen meum. Qui autem sustinuerit in finem, hic salvus erit. 
\verse Cum autem videritis abominationem desolationis stantem, ubi non debet, qui legit, intellegat: tunc, qui in Iudaea sunt, fugiant in montes; 
\verse qui autem super tectum, ne descendat nec introeat, ut tollat quid de domo sua; 
\verse et, qui in agro erit, non revertatur retro tollere vestimentum suum. 
\verse Vae autem praegnantibus et nutrientibus in illis diebus! 
\verse Orate vero, ut hieme non fiat: 
\verse erunt enim dies illi tribulatio talis, qualis non fuit ab initio creaturae, quam condidit Deus, usque nunc, neque fiet. 
\verse Et nisi breviasset Dominus dies, non fuisset salva omnis caro. Sed propter electos, quos elegit, breviavit dies. 
\verse Et tunc, si quis vobis dixerit: “Ecce hic est Christus, ecce illic”, ne credideritis. 
\verse Exsurgent enim pseudochristi et pseudoprophetae et dabunt signa et portenta ad seducendos, si potest fieri, electos. 
\verse Vos autem videte; praedixi vobis omnia. 
\verse Sed in illis diebus post tribulationem illam sol contenebrabitur, et luna non dabit splendorem suum, 
\verse et erunt stellae de caelo decidentes, et virtutes, quae sunt in caelis, movebuntur. 
\verse Et tunc videbunt Filium hominis venientem in nubibus cum virtute multa et gloria. 
\verse Et tunc mittet angelos et congregabit electos suos a quattuor ventis, a summo terrae usque ad summum caeli. 
\verse A ficu autem discite parabolam: cum iam ramus eius tener fuerit et germinaverit folia, cognoscitis quia in proximo sit aestas. 
\verse Sic et vos, cum videritis haec fieri, scitote quod in proximo sit in ostiis. 
\verse Amen dico vobis: Non transiet generatio haec, donec omnia ista fiant. 
\verse Caelum et terra transibunt, verba autem mea non transibunt. 
\verse De die autem illo vel hora nemo scit, neque angeli in caelo neque Filius, nisi Pater. 
\verse Videte, vigilate; nescitis enim, quando tempus sit. 
\verse Sicut homo, qui peregre profectus reliquit domum suam et dedit servis suis potestatem, unicuique opus suum, ianitori quoque praecepit, ut vigilaret. 
\verse Vigilate ergo; nescitis enim quando dominus domus veniat, sero an media nocte an galli cantu an mane; 
\verse ne, cum venerit repente, inveniat vos dormientes. 
\verse Quod autem vobis dico, omnibus dico: Vigilate!". 
\end{biblechapter}

\begin{biblechapter}  
\verse Erat autem Pascha et Azyma post biduum. Et quaerebant summi sacerdotes et scribae, quomodo eum dolo tenerent et occiderent; 
\verse dicebant enim: “Non in die festo, ne forte tumultus fieret populi". 
\verse Et cum esset Bethaniae in domo Simonis leprosi et recumberet, venit mulier habens alabastrum unguenti nardi puri pretiosi; fracto alabastro, effudit super caput eius. 
\verse Erant autem quidam indigne ferentes intra semetipsos: “Ut quid perditio ista unguenti facta est? 
\verse Poterat enim unguentum istud veniri plus quam trecentis denariis et dari pauperibus". Et fremebant in eam. 
\verse Iesus autem dixit: “Sinite eam; quid illi molesti estis? Bonum opus operata est in me. 
\verse Semper enim pauperes habetis vobiscum et, cum volueritis, potestis illis bene facere; me autem non semper habetis. 
\verse Quod habuit, operata est: praevenit ungere corpus meum in sepulturam. 
\verse Amen autem dico vobis: Ubicumque praedicatum fuerit evangelium in universum mundum, et, quod fecit haec, narrabitur in memoriam eius". 
\verse Et Iudas Iscarioth, unus de Duodecim, abiit ad summos sacerdotes, ut proderet eum illis. 
\verse Qui audientes gavisi sunt et promiserunt ei pecuniam se daturos. Et quaerebat quomodo illum opportune traderet. 
\verse Et primo die Azymorum, quando Pascha immolabant, dicunt ei discipuli eius: “Quo vis eamus et paremus, ut manduces Pascha?". 
\verse Et mittit duos ex discipulis suis et dicit eis: “Ite in civitatem, et occurret vobis homo lagoenam aquae baiulans; sequimini eum 
\verse et, quocumque introierit, dicite domino domus: “Magister dicit: Ubi est refectio mea, ubi Pascha cum discipulis meis manducem?”. 
\verse Et ipse vobis demonstrabit cenaculum grande stratum paratum; et illic parate nobis". 
\verse Et abierunt discipuli et venerunt in civitatem et invenerunt, sicut dixerat illis, et paraverunt Pascha. 
\verse Et vespere facto, venit cum Duodecim. 
\verse Et discumbentibus eis et manducantibus, ait Iesus: “Amen dico vobis: Unus ex vobis me tradet, qui manducat mecum". 
\verse Coeperunt contristari et dicere ei singillatim: “Numquid ego?". 
\verse Qui ait illis: “Unus ex Duodecim, qui intingit mecum in catino. 
\verse Nam Filius quidem hominis vadit, sicut scriptum est de eo. Vae autem homini illi, per quem Filius hominis traditur! Bonum est ei, si non esset natus homo ille". 
\verse Et manducantibus illis, accepit panem et benedicens fregit et dedit eis et ait: “Sumite: hoc est corpus meum". 
\verse Et accepto calice, gratias agens dedit eis; et biberunt ex illo omnes. 
\verse Et ait illis: “Hic est sanguis meus novi testamenti, qui pro multis effunditur. 
\verse Amen dico vobis: Iam non bibam de genimine vitis usque in diem illum, cum illud bibam novum in regno Dei". 
\verse Et hymno dicto, exierunt in montem Olivarum. 
\verse Et ait eis Iesus: “Omnes scandalizabimini, quia scriptum est: “Percutiam pastorem, et dispergentur oves”. 
\verse Sed posteaquam resurrexero, praecedam vos in Galilaeam". 
\verse Petrus autem ait ei: “Et si omnes scandalizati fuerint, sed non ego". 
\verse Et ait illi Iesus: “Amen dico tibi: Tu hodie, in nocte hac, priusquam bis gallus vocem dederit, ter me es negaturus". 
\verse At ille amplius loquebatur: “Et si oportuerit me commori tibi, non te negabo". Similiter autem et omnes dicebant. 
\verse Et veniunt in praedium, cui nomen Gethsemani; et ait discipulis suis: “Sedete hic, donec orem". 
\verse Et assumit Petrum et Iacobum et Ioannem secum et coepit pavere et taedere; 
\verse et ait illis: “Tristis est anima mea usque ad mortem; sustinete hic et vigilate". 
\verse Et cum processisset paululum, procidebat super terram et orabat, ut, si fieri posset, transiret ab eo hora;  
\verse et dicebat: “Abba, Pater! Omnia tibi possibilia sunt. Transfer calicem hunc a me; sed non quod ego volo, sed quod tu". 
\verse Et venit et invenit eos dormientes; et ait Petro: “Simon, dormis? Non potuisti una hora vigilare?  
\verse Vigilate et orate, ut non intretis in tentationem; spiritus quidem promptus, caro vero infirma". 
\verse Et iterum abiens oravit, eundem sermonem dicens.  
\verse Et veniens denuo invenit eos dormientes; erant enim oculi illorum ingravati, et ignorabant quid responderent ei. 
\verse Et venit tertio et ait illis: “Dormite iam et requiescite? Sufficit, venit hora: ecce traditur Filius hominis in manus peccatorum. 
\verse Surgite, eamus; ecce, qui me tradit, prope est". 
\verse Et confestim, adhuc eo loquente, venit Iudas unus ex Duodecim, et cum illo turba cum gladiis et lignis a summis sacerdotibus et scribis et senioribus.  
\verse Dederat autem traditor eius signum eis dicens: “Quemcumque osculatus fuero, ipse est; tenete eum et ducite caute". 
\verse Et cum venisset, statim accedens ad eum ait: “Rabbi"; et osculatus est eum. 
\verse At illi manus iniecerunt in eum et tenuerunt eum. 
\verse Unus autem quidam de circumstantibus educens gladium percussit servum summi sacerdotis et amputavit illi auriculam. 
\verse Et respondens Iesus ait illis: “Tamquam ad latronem existis cum gladiis et lignis comprehendere me? 
\verse Cotidie eram apud vos in templo docens, et non me tenuistis; sed adimpleantur Scripturae". 
\verse Et relinquentes eum omnes fugerunt. 
\verse Et adulescens quidam sequebatur eum amictus sindone super nudo, et tenent eum; 
\verse at ille, reiecta sindone, nudus profugit. 
\verse Et adduxerunt Iesum ad summum sacerdotem; et conveniunt omnes summi sacerdotes et seniores et scribae. 
\verse Et Petrus a longe secutus est eum usque intro in atrium summi sacerdotis et sedebat cum ministris et calefaciebat se ad ignem. 
\verse Summi vero sacerdotes et omne concilium quaerebant adversus Iesum testimonium, ut eum morte afficerent, nec inveniebant. 
\verse Multi enim testimonium falsum dicebant adversus eum, et convenientia testimonia non erant.  
\verse Et quidam surgentes falsum testimonium ferebant adversus eum dicentes: 
\verse “Nos audivimus eum dicentem: “Ego dissolvam templum hoc manu factum et intra triduum aliud non manu factum aedificabo”". 
\verse Et ne ita quidem conveniens erat testimonium illorum. 
\verse Et exsurgens summus sacerdos in medium interrogavit Iesum dicens: “Non respondes quidquam ad ea, quae isti testantur adversum te?". 
\verse Ille autem tacebat et nihil respondit. Rursum summus sacerdos interrogabat eum et dicit ei: “Tu es Christus filius Benedicti?".  
\verse Iesus autem dixit: “Ego sum, et videbitis Filium hominis a dextris sedentem Virtutis et venientem cum nubibus caeli". 
\verse Summus autem sacerdos scindens vestimenta sua ait: “Quid adhuc necessarii sunt nobis testes? 
\verse Audistis blasphemiam. Quid vobis videtur?". Qui omnes condemnaverunt eum esse reum mortis. 
\verse Et coeperunt quidam conspuere eum et velare faciem eius et colaphis eum caedere et dicere ei: “Prophetiza"; et ministri alapis eum caedebant. 
\verse Et cum esset Petrus in atrio deorsum, venit una ex ancillis summi sacerdotis 
\verse et, cum vidisset Petrum calefacientem se, aspiciens illum ait: “Et tu cum hoc Nazareno, Iesu, eras!". 
\verse At ille negavit dicens: “Neque scio neque novi quid tu dicas!". Et exiit foras ante atrium, et gallus cantavit.  
\verse Et ancilla, cum vidisset illum, rursus coepit dicere circumstantibus: “Hic ex illis est!". 
\verse At ille iterum negabat. Et post pusillum rursus, qui astabant, dicebant Petro: “Vere ex illis es, nam et Galilaeus es". 
\verse Ille autem coepit anathematizare et iurare: “Nescio hominem istum, quem dicitis!".  
\verse Et statim iterum gallus cantavit; et recordatus est Petrus verbi, sicut dixerat ei Iesus: “Priusquam gallus cantet bis, ter me negabis". Et coepit flere. 
\end{biblechapter}

\begin{biblechapter}  
\verse Et confestim mane consilium facientes summi sacerdotes cum senioribus et scribis, id est universum concilium, vincientes Iesum duxerunt et tradiderunt Pilato. 
\verse Et interrogavit eum Pilatus: “Tu es rex Iudaeorum?". At ille respondens ait illi: “Tu dicis". 
\verse Et accusabant eum summi sacerdotes in multis. 
\verse Pilatus autem rursum interrogabat eum dicens: “Non respondes quidquam? Vide in quantis te accusant". 
\verse Iesus autem amplius nihil respondit, ita ut miraretur Pilatus. 
\verse Per diem autem festum dimittere solebat illis unum ex vinctis, quem peterent. 
\verse Erat autem qui dicebatur Barabbas, vinctus cum seditiosis, qui in seditione fecerant homicidium. 
\verse Et cum ascendisset turba, coepit rogare, sicut faciebat illis. 
\verse Pilatus autem respondit eis et dixit: “Vultis dimittam vobis regem Iudaeorum?". 
\verse Sciebat enim quod per invidiam tradidissent eum summi sacerdotes. 
\verse Pontifices autem concitaverunt turbam, ut magis Barabbam dimitteret eis. 
\verse Pilatus autem iterum respondens aiebat illis: “Quid ergo vultis faciam regi Iudaeorum?". 
\verse At illi iterum clamaverunt: “Crucifige eum!". 
\verse Pilatus vero dicebat eis: “Quid enim mali fecit?". At illi magis clamaverunt: “Crucifige eum!". 
\verse Pilatus autem, volens populo satisfacere, dimisit illis Barabbam et tradidit Iesum flagellis caesum, ut crucifigeretur. 
\verse Milites autem duxerunt eum intro in atrium, quod est praetorium, et convocant totam cohortem. 
\verse Et induunt eum purpuram et imponunt ei plectentes spineam coronam; 
\verse et coeperunt salutare eum: “Ave, rex Iudaeorum!", 
\verse et percutiebant caput eius arundine et conspuebant eum et ponentes genua adorabant eum. 
\verse Et postquam illuserunt ei, exuerunt illum purpuram et induerunt eum vestimentis suis. Et educunt illum, ut crucifigerent eum. 
\verse Et angariant praetereuntem quempiam Simonem Cyrenaeum venientem de villa, patrem Alexandri et Rufi, ut tolleret crucem eius. 
\verse Et perducunt illum in Golgotha locum, quod est interpretatum Calvariae locus. 
\verse Et dabant ei myrrhatum vinum; ille autem non accepit. 
\verse Et crucifigunt eum et dividunt vestimenta eius, mittentes sortem super eis, quis quid tolleret. 
\verse Erat autem hora tertia, et crucifixerunt eum. 
\verse Et erat titulus causae eius inscriptus: “Rex Iudaeorum". 
\verse Et cum eo crucifigunt duos latrones, unum a dextris et alium a sinistris eius. (28) 
\verse Et praetereuntes blasphemabant eum moventes capita sua et dicentes: “Vah, qui destruit templum et in tribus diebus aedificat;  
\verse salvum fac temetipsum descendens de cruce!". 
\verse Similiter et summi sacerdotes ludentes ad alterutrum cum scribis dicebant: “Alios salvos fecit, seipsum non potest salvum facere. 
\verse Christus rex Israel descendat nunc de cruce, ut videamus et credamus". Etiam qui cum eo crucifixi erant, conviciabantur ei. 
\verse Et, facta hora sexta, tenebrae factae sunt per totam terram usque in horam nonam. 
\verse Et hora nona exclamavit Iesus voce magna: “Heloi, Heloi, lema sabacthani?", quod est interpretatum: “Deus meus, Deus meus, ut quid dereliquisti me?". 
\verse Et quidam de circumstantibus audientes dicebant: “Ecce, Eliam vocat". 
\verse Currens autem unus et implens spongiam aceto circumponensque calamo potum dabat ei dicens: “Sinite, videamus, si veniat Elias ad deponendum eum". 
\verse Iesus autem, emissa voce magna, exspiravit. 
\verse Et velum templi scissum est in duo a sursum usque deorsum. 
\verse Videns autem centurio, qui ex adverso stabat, quia sic clamans exspirasset, ait: “Vere homo hic Filius Dei erat". 
\verse Erant autem et mulieres de longe aspicientes, inter quas et Maria Magdalene et Maria Iacobi minoris et Iosetis mater et Salome, 
\verse quae, cum esset in Galilaea, sequebantur eum et ministrabant ei, et aliae multae, quae simul cum eo ascenderant Hierosolymam. 
\verse Et cum iam sero esset factum, quia erat Parasceve, quod est ante sabbatum,  
\verse venit Ioseph ab Arimathaea nobilis decurio, qui et ipse erat exspectans regnum Dei, et audacter introivit ad Pilatum et petiit corpus Iesu. 
\verse Pilatus autem miratus est si iam obisset, et, accersito centurione, interrogavit eum si iam mortuus esset, 
\verse et, cum cognovisset a centurione, donavit corpus Ioseph. 
\verse Is autem mercatus sindonem et deponens eum involvit sindone et posuit eum in monumento, quod erat excisum de petra, et advolvit lapidem ad ostium monumenti. 
\verse Maria autem Magdalene et Maria Iosetis aspiciebant, ubi positus esset. 
\end{biblechapter}

\begin{biblechapter}  
\verse Et cum transisset sabbatum, Maria Magdalene et Maria Iacobi et Salome emerunt aromata, ut venientes ungerent eum. 
\verse Et valde mane, prima sabbatorum, veniunt ad monumentum, orto iam sole. 
\verse Et dicebant ad invicem: “Quis revolvet nobis lapidem ab ostio monumenti?". 
\verse Et respicientes vident revolutum lapidem; erat quippe magnus valde. 
\verse Et introeuntes in monumentum viderunt iuvenem sedentem in dextris, coopertum stola candida, et obstupuerunt. 
\verse Qui dicit illis: “Nolite expavescere! Iesum quaeritis Nazarenum crucifixum. Surrexit, non est hic; ecce locus, ubi posuerunt eum. 
\verse Sed ite, dicite discipulis eius et Petro: “Praecedit vos in Galilaeam. Ibi eum videbitis, sicut dixit vobis”". 
\verse Et exeuntes fugerunt de monumento; invaserat enim eas tremor et pavor, et nemini quidquam dixerunt, timebant enim. 
\verse Surgens autem mane, prima sabbati, apparuit primo Mariae Magdalenae, de qua eiecerat septem daemonia. 
\verse Illa vadens nuntiavit his, qui cum eo fuerant, lugentibus et flentibus; 
\verse et illi audientes quia viveret et visus esset ab ea, non crediderunt. 
\verse Post haec autem duobus ex eis ambulantibus ostensus est in alia effigie euntibus in villam; 
\verse et illi euntes nuntiaverunt ceteris, nec illis crediderunt. 
\verse Novissime recumbentibus illis Undecim apparuit, et exprobravit incredulitatem illorum et duritiam cordis, quia his, qui viderant eum resuscitatum, non crediderant. 
\verse Et dixit eis: “Euntes in mundum universum praedicate evangelium omni creaturae. 
\verse Qui crediderit et baptizatus fuerit, salvus erit; qui vero non crediderit, condemnabitur. 
\verse Signa autem eos, qui crediderint, haec sequentur: in nomine meo daemonia eicient, linguis loquentur novis, 
\verse serpentes tollent, et, si mortiferum quid biberint, non eos nocebit, super aegrotos manus imponent, et bene habebunt". 
\verse Et Dominus quidem Iesus, postquam locutus est eis, assumptus est in caelum et sedit a dextris Dei. 
\verse Illi autem profecti praedicaverunt ubique, Domino cooperante et sermonem confirmante, sequentibus signis.
\end{biblechapter}
