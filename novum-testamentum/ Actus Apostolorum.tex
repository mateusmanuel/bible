\biblebook{ Actus Apostolorum}
\begin{biblechapter}
 \verse Primum quidem sermonem feci de omnibus, o Theophile, quae coepit Iesus facere et docere, 
 \verse usque in diem, qua, cum praecepisset apostolis per Spiritum Sanctum, quos elegit, assumptus est;
\verse quibus et praebuit seipsum vivum post passionem suam in multis argumentis, per dies quadraginta apparens eis et loquens ea, quae sunt de regno Dei. 
 \verse Et convescens praecepit eis ab Hierosolymis ne discederent, sed exspectarent promissionem Patris: “ Quam audistis a me, 
 \verse quia Ioannes quidem baptizavit aqua, vos autem baptizabimini in Spiritu Sancto non post multos hos dies ”.
 \verse Igitur qui convenerant, interrogabant eum dicentes: “ Domine, si in tempore hoc restitues regnum Israeli? ”. 
 \verse Dixit autem eis: “ Non est vestrum nosse tempora vel momenta, quae Pater posuit in sua potestate, 
 \verse sed accipietis virtutem, superveniente Sancto Spiritu in vos, et eritis mihi testes et in Ierusalem et in omni Iudaea et Samaria et usque ad ultimum terrae ”.
 \verse Et cum haec dixisset, videntibus illis, elevatus est, et nubes suscepit eum ab oculis eorum. 
 \verse Cumque intuerentur in caelum, eunte illo, ecce duo viri astiterunt iuxta illos in vestibus albis, 
 \verse qui et dixerunt: “ Viri Galilaei, quid statis aspicientes in caelum? Hic Iesus, qui assumptus est a vobis in caelum, sic veniet quemadmodum vidistis eum euntem in caelum ”.
 \verse Tunc reversi sunt in Ierusalem a monte, qui vocatur Oliveti, qui est iuxta Ierusalem sabbati habens iter. 
 \verse Et cum introissent, in cenaculum ascenderunt, ubi manebant et Petrus et Ioannes et Iacobus et Andreas, Philippus et Thomas, Bartholomaeus et Matthaeus, Iacobus Alphaei et Simon Zelotes et Iudas Iacobi. 
 \verse Hi omnes erant perseverantes unanimiter in oratione cum mulieribus et Maria matre Iesu et fratribus eius.
 \verse Et in diebus illis exsurgens Petrus in medio fratrum dixit — erat autem turba hominum simul fere centum viginti C: 
 \verse “ Viri fratres, oportebat impleri Scripturam, quam praedixit Spiritus Sanctus per os David de Iuda, qui fuit dux eorum, qui comprehenderunt Iesum, 
 \verse quia connumeratus erat in nobis et sortitus est sortem ministerii huius. 
 \verse Hic quidem possedit agrum de mercede iniquitatis; et pronus factus crepuit medius, et diffusa sunt omnia viscera eius. 
 \verse Et notum factum est omnibus habitantibus Ierusalem, ita ut appellaretur ager ille lingua eorum Aceldamach, hoc est ager Sanguinis. 
 \verse Scriptum est enim in libro Psalmorum:
 “Fiat commoratío eius deserta,
 et non sit qui inhabitet in ea”
 et: “Episcopatum eius accipiat alius”.
 \verse Oportet ergo ex his viris, qui nobiscum congregati erant in omni tempore, quo intravit et exivit inter nos Dominus Iesus, 
 \verse incipiens a baptismate Ioannis usque in diem, qua assumptus est a nobis, testem resurrectionis eius nobiscum fieri unum ex istis ”.
 \verse Et statuerunt duos, Ioseph, qui vocabatur Barsabbas, qui cognominatus est Iustus, et Matthiam.
\verse Et orantes dixerunt: “ Tu, Domine, qui corda nosti omnium, ostende quem elegeris ex his duobus unum 
 \verse accipere locum ministerii huius et apostolatus, de quo praevaricatus est Iudas, ut abiret in locum suum ”. 
 \verse Et dederunt sortes eis, et cecidit sors super Matthiam, et annumeratus est cum undecim apostolis.
 
\begin{biblechapter}
 \verse Et cum compleretur dies Pen tecostes, erant omnes pariter in eodem loco. 
 \verse Et factus est repente de caelo sonus tamquam advenientis spiritus vehementis et replevit totam domum, ubi erant sedentes. 
 \verse Et apparuerunt illis dispertitae linguae tamquam ignis, seditque supra singulos eorum; 
 \verse et repleti sunt omnes Spiritu Sancto et coeperunt loqui aliis linguis, prout Spiritus dabat eloqui illis.
 \verse Erant autem in Ierusalem habitantes Iudaei, viri religiosi ex omni natione, quae sub caelo est; 
 \verse facta autem hac voce, convenit multitudo et confusa est, quoniam audiebat unusquisque lingua sua illos loquentes. 
 \verse Stupebant autem et mirabantur dicentes: “ Nonne ecce omnes isti, qui loquuntur, Galilaei sunt? 
 \verse Et quomodo nos audimus unusquisque propria lingua nostra, in qua nati sumus? 
 \verse Parthi et Medi et Elamitae et qui habitant Mesopotamiam, Iudaeam quoque et Cappadociam, Pontum et Asiam, 
 \verse Phrygiam quoque et Pamphyliam, Aegyptum et partes Libyae, quae est circa Cyrenem, et advenae Romani, 
 \verse Iudaei quoque et proselyti, Cretes et Arabes, audimus loquentes eos nostris linguis magnalia Dei ”. 
 \verse Stupebant autem omnes et haesitabant ad invicem dicentes: “ Quidnam hoc vult esse? ”; 
 \verse alii autem irridentes dicebant: “ Musto pleni sunt isti ”.
 \verse Stans autem Petrus cum Undecim levavit vocem suam et locutus est eis: “ Viri Iudaei et qui habitatis Ierusalem universi, hoc vobis notum sit, et auribus percipite verba mea. 
 \verse Non enim, sicut vos aestimatis, hi ebrii sunt, est enim hora diei tertia; 
 \verse sed hoc est, quod dictum est per prophetam Ioel:
 \verse “Et erit: in novissimis diebus, dicit Deus,
 effundam de Spiritu meo super omnem carnem,
 et prophetabunt filii vestri et filiae vestrae,
 et iuvenes vestri visiones videbunt,
 et seniores vestri somnia somniabunt;
 \verse et quidem super servos meos et super ancillas meas
 in diebus illis effundam de Spiritu meo,
 et prophetabunt.
 \verse Et dabo prodigia in caelo sursum et signa in terra deorsum,
 sanguinem et ignem et vaporem fumi;
 \verse sol convertetur in tenebras,
 et luna in sanguinem,
 antequam veniat dies Domini
 magnus et manifestus.
 \verse Et erit:
 omnis quicumque invocaverit nomen Domini, salvus erit”.
 \verse Viri Israelitae, audite verba haec: Iesum Nazarenum, virum approbatum a Deo apud vos virtutibus et prodigiis et signis, quae fecit per illum Deus in medio vestri, sicut ipsi scitis, 
 \verse hunc definito consilio et praescientia Dei traditum per manum iniquorum affigentes interemistis,
\verse quem Deus suscitavit, solutis doloribus mortis, iuxta quod impossibile erat teneri illum ab ea. 
 \verse David enim dicit circa eum:
 “Providebam Dominum coram me semper, quoniam a dextris meis est, ne commovear.
 \verse Propter hoc laetatum est cor meum,
 et exsultavit lingua mea;
 insuper et caro mea requiescet in spe.
 \verse Quoniam non derelinques animam meam in inferno
 neque dabis Sanctum tuum videre corruptionem.
 \verse Notas fecisti mihi vias vitae,
 replebis me iucunditate cum facie tua”.
 \verse Viri fratres, liceat audenter dicere ad vos de patriarcha David, quoniam et defunctus est et sepultus est, et sepulcrum eius est apud nos usque in hodiernum diem; 
 \verse propheta igitur cum esset et sciret quia iure iurando iurasset illi Deus de fructu lumbi eius sedere super sedem eius, 
 \verse providens locutus est de resurrectione Christi, quia neque derelictus est in inferno, neque caro eius vidit corruptionem.
 \verse Hunc Iesum resuscitavit Deus, cuius omnes nos testes sumus. 
 \verse Dextera igitur Dei exaltatus, et promissione Spiritus Sancti accepta a Patre, effudit hunc, quem vos videtis et auditis. 
 \verse Non enim David ascendit in caelos; dicit autem ipse:
 “Dixit Dominus Domino meo: Sede a dextris meis,
 \verse donec ponam inimicos tuos scabellum pedum tuorum”.
 \verse Certissime ergo sciat omnis domus Israel quia et Dominum eum et Christum Deus fecit, hunc Iesum, quem vos crucifixistis ”.
 \verse His auditis, compuncti sunt corde et dixerunt ad Petrum et reliquos apostolos: “ Quid faciemus, viri fratres? ”. 
 \verse Petrus vero ad illos: “ Paenitentiam, inquit, agite, et baptizetur unusquisque vestrum in nomine Iesu Christi in remissionem peccatorum vestrorum, et accipietis donum Sancti Spiritus; 
 \verse vobis enim est repromissio et filiis vestris et omnibus, qui longe sunt, quoscumque advocaverit Dominus Deus noster ”. 
 \verse Aliis etiam verbis pluribus testificatus est et exhortabatur eos dicens: “ Salvamini a generatione ista prava ”. 
 \verse Qui ergo, recepto sermone eius, baptizati sunt; et appositae sunt in il la die animae circiter tria milia. 
 \verse Erant autem perseverantes in doctrina apostolorum et communicatione, in fractione panis et orationibus. 
 \verse Fiebat autem omni animae timor; multa quoque prodigia et signa per apostolos fiebant. 
 \verse Omnes autem, qui crediderant, erant pariter et habebant omnia communia; 
 \verse et possessiones et substantias vendebant et dividebant illas omnibus, prout cuique opus erat; 
 \verse cotidie quoque perdurantes unanimiter in templo et frangentes circa domos panem, sumebant cibum cum exsultatione et simplicitate cordis, 
 \verse collaudantes Deum et habentes gratiam ad omnem plebem. Dominus autem augebat, qui salvi fierent cotidie in idipsum.
 
\begin{biblechapter}
 \verse Petrus autem et Ioannes ascen debant in templum ad horam orationis nonam. 
 \verse Et quidam vir, qui erat claudus ex utero matris suae, baiulabatur; quem ponebant cotidie ad portam templi, quae dicitur Speciosa, ut peteret eleemosynam ab introeuntibus in templum; 
 \verse is cum vidisset Petrum et Ioannem incipientes introire in templum, rogabat, ut eleemosynam acciperet. 
 \verse Intuens autem in eum Petrus cum Ioanne dixit: “ Respice in nos ”. 
 \verse At ille intendebat in eos, sperans se aliquid accepturum ab eis. 
 \verse Petrus autem dixit: “ Argentum et aurum non est mihi; quod autem habeo, hoc tibi do: In nomine Iesu Christi Nazareni surge et ambula! ”. 
 \verse Et apprehensa ei manu dextera, allevavit eum; et protinus consolidatae sunt bases eius et tali, 
 \verse et exsiliens stetit et ambulabat; et intravit cum illis in templum, ambulans et exsiliens et laudans Deum. 
 \verse Et vidit omnis populus eum ambulantem et laudantem Deum; 
 \verse cognoscebant autem illum quoniam ipse erat, qui ad eleemosynam sedebat ad Speciosam portam templi, et impleti sunt stupore et exstasi in eo, quod contigerat illi.
 \verse Cum teneret autem Petrum et Ioannem, concurrit omnis populus ad eos ad porticum, qui appellatur Salomonis, stupentes. 
 \verse Videns autem Petrus respondit ad populum: “ Viri Israelitae, quid miramini in hoc aut nos quid intuemini, quasi nostra virtute aut pietate fecerimus hunc ambulare? 
 \verse Deus Abraham et Deus Isaac et Deus Iacob, Deus patrum nostrorum, glorificavit puerum suum Iesum, quem vos quidem tradidistis et negastis ante faciem Pilati, iudicante illo dimitti; 
 \verse vos autem Sanctum et Iustum negastis et petistis virum homicidam donari vobis, 
 \verse ducem vero vitae interfecistis, quem Deus suscitavit a mortuis, cuius nos testes sumus. 
 \verse Et in fide nominis eius hunc, quem videtis et nostis, confirmavit nomen eius; et fides, quae per eum est, dedit huic integritatem istam in conspectu omnium vestrum.
 \verse Et nunc, fratres, scio quia per ignorantiam fecistis, sicut et principes vestri; \verse Deus autem, quae praenuntiavit per os omnium Prophetarum pati Christum suum, implevit sic. 
\verse Paenitemini igitur et convertimini, ut deleantur vestra peccata, 
 \verse ut veniant tempora refrigerii a conspectu Domini, et mittat eum, qui praedestinatus est vobis Christus, Iesum, 
 \verse quem oportet caelum quidem suscipere usque in tempora restitutionis omnium, quae locutus est Deus per os sanctorum a saeculo suorum prophetarum. 
 \verse Moyses quidem dixit: “Prophetam vobis suscitabit Dominus Deus vester de fratribus vestris tamquam me; ipsum audietis iuxta omnia, quaecumque locutus fuerit vobis. 
 \verse Erit autem: omnis anima, quae non audierit prophetam illum, exterminabitur de plebe”. 
 \verse Et omnes prophetae a Samuel et deinceps quotquot locuti sunt, etiam annuntiaverunt dies istos.
\verse Vos estis filii prophetarum et testamenti, quod disposuit Deus ad patres vestros dicens ad Abraham: “Et in semine tuo benedicentur omnes familiae terrae”. 
 \verse Vobis primum Deus suscitans Puerum suum, misit eum benedicentem vobis in avertendo unumquemque a nequitiis vestris ”.
 
\begin{biblechapter}
 \verse Loquentibus autem illis ad populum, supervenerunt eis sa cerdotes et magistratus templi et sadducaei, 
 \verse dolentes quod docerent populum et annuntiarent in Iesu resurrectionem ex mortuis; 
 \verse et iniecerunt in eos manus et posuerunt in custodiam in crastinum; erat enim iam vespera. 
 \verse Multi autem eorum, qui audierant verbum, crediderunt; et factus est numerus virorum quinque milia.
 \verse Factum est autem in crastinum, ut congregarentur principes eorum et seniores et scribae in Ierusalem, 
 \verse et Annas princeps sacerdotum et Caiphas et Ioannes et Alexander et quotquot erant de genere sacerdotali, 
 \verse et statuentes eos in medio interrogabant: “ In qua virtute aut in quo nomine fecistis hoc vos? ”. 
 \verse Tunc Petrus repletus Spiritu Sancto dixit ad eos: “ Principes populi et seniores, 
 \verse si nos hodie diiudicamur in benefacto hominis infirmi, in quo iste salvus factus est, 
 \verse notum sit omnibus vobis et omni plebi Israel quia in nomine Iesu Christi Nazareni, quem vos crucifixistis, quem Deus suscitavit a mortuis, in hoc iste astat coram vobis sanus. 
 \verse Hic est
 lapis, qui reprobatus est a vobis aedificatoribus,
 qui factus est in caput anguli.
 \verse Et non est in alio aliquo salus, nec enim nomen aliud est sub caelo datum in hominibus, in quo oportet nos salvos fieri ”.
 \verse Videntes autem Petri fiduciam et Ioannis, et comperto quod homines essent sine litteris et idiotae, admirabantur et cognoscebant eos quoniam cum Iesu fuerant; 
 \verse hominem quoque videntes stantem cum eis, qui curatus fuerat, nihil poterant contradicere. 
 \verse Iubentes autem eos foras extra concilium secedere, conferebant ad invicem 
 \verse dicentes: “ Quid faciemus hominibus istis? Quoniam quidem notum signum factum est per eos omnibus habitantibus in Ierusalem manifestum, et non possumus negare; 
 \verse sed ne amplius divulgetur in populum, comminemur eis, ne ultra loquantur in nomine hoc ulli hominum ”. 
 \verse Et vocantes eos denuntiaverunt, ne omnino loquerentur neque docerent in nomine Iesu. 
 \verse Petrus vero et Ioannes respondentes dixerunt ad eos: “ Si iustum est in conspectu Dei vos potius audire quam Deum, iudicate; 
 \verse non enim possumus nos, quae vidimus et audivimus, non loqui ”. 
 \verse At illi ultra comminantes dimiserunt eos, nequaquam invenientes, quomodo punirent eos, propter populum, quia omnes glorificabant Deum in eo, quod acciderat; 
 \verse annorum enim erat amplius quadraginta homo, in quo factum erat signum istud sanitatis.
 \verse Dimissi autem venerunt ad suos et annuntiaverunt quanta ad eos principes sacerdotum et seniores dixissent. 
 \verse Qui cum audissent, unanimiter levaverunt vocem ad Deum et dixerunt: “ Domine, tu, qui fecisti caelum et terram et mare et omnia, quae in eis sunt, 
 \verse qui Spiritu Sancto per os patris nostri David pueri tui dixisti:
 “Quare fremuerunt gentes,
 et populi meditati sunt inania?
 \verse Astiterunt reges terrae,
 et principes convenerunt in unum
 adversus Dominum et adversus Christum eius”.
 \verse Convenerunt enim vere in civitate ista adversus sanctum puerum tuum Iesum, quem unxisti, Herodes et Pontius Pilatus cum gentibus et populis Israel 
 \verse facere, quaecumque manus tua et consilium praedestinavit fieri. 
 \verse Et nunc, Domine, respice in minas eorum et da servis tuis cum omni fiducia loqui verbum tuum, \verse in eo quod manum tuam extendas ad sanitatem et signa et prodigia facienda per nomen sancti pueri tui Iesu ”.
 \verse Et cum orassent, motus est locus, in quo erant congregati, et repleti sunt omnes Sancto Spiritu et loquebantur verbum Dei cum fiducia.
 \verse Multitudinis autem credentium erat cor et anima una, nec quisquam eorum, quae possidebant, aliquid suum esse dicebat, sed erant illis omnia communia. 
 \verse Et virtute magna reddebant apostoli testimonium resurrectionis Domini Iesu, et gratia magna erat super omnibus illis. 
 \verse Neque enim quisquam egens erat inter illos; quotquot enim possessores agrorum aut domorum erant, vendentes afferebant pretia eorum, quae vendebant, 
 \verse et ponebant ante pedes apostolorum; dividebatur autem singulis, prout cuique opus erat. 
 \verse Ioseph autem, qui cognominatus est Barnabas ab apostolis — quod est interpretatum filius Consolationis — Levites, Cyprius genere, 
\verse cum haberet agrum, vendidit et attulit pecuniam et posuit ante pedes apostolorum.
 
\begin{biblechapter}
 \verse Vir autem quidam nomine Ananias cum Sapphira uxore sua vendidit agrum 
 \verse et subtraxit de pretio, conscia quoque uxore, et afferens partem quandam ad pedes apostolorum posuit. 
 \verse Dixit autem Petrus: “ Anania, cur implevit Satanas cor tuum mentiri te Spiritui Sancto et subtrahere de pretio agri? 
 \verse Nonne manens tibi manebat et venumdatum in tua erat potestate? Quare posuisti in corde tuo hanc rem? Non es mentitus hominibus sed Deo! ”. 
 \verse Audiens autem Ananias haec verba cecidit et exspiravit; et factus est timor magnus in omnes audientes. 
 \verse Surgentes autem iuvenes involverunt eum et efferentes sepelierunt.
 \verse Factum est autem quasi horarum trium spatium, et uxor ipsius nesciens, quod factum fuerat, introivit. 
 \verse Respondit autem ei Petrus: “ Dic mihi, si tanti agrum vendidistis? ”. At illa dixit: “ Etiam, tanti ”. 
 \verse Petrus autem ad eam: “ Quid est quod convenit vobis tentare Spiritum Domini? Ecce pedes eorum, qui sepelierunt virum tuum, ad ostium, et efferent te ”. 
 \verse Confestim cecidit ante pedes eius et exspiravit; intrantes autem iuvenes invenerunt illam mortuam et efferentes sepelierunt ad virum suum. 
 \verse Et factus est timor magnus super universam ecclesiam et in omnes, qui audierunt haec.
 \verse Per manus autem apostolorum fiebant signa et prodigia multa in plebe; et erant unanimiter omnes in porticu Salomonis. 
 \verse Ceterorum autem nemo audebat coniungere se illis, sed magnificabat eos populus; 
 \verse magis autem addebantur credentes Domino multitudines virorum ac mulierum, 
 \verse ita ut et in plateas efferrent infirmos et ponerent in lectulis et grabatis, ut, veniente Petro, saltem umbra illius obumbraret quemquam eorum. 
 \verse Concurrebat autem et multitudo vicinarum civitatum Ierusalem, afferentes aegros et vexatos ab spiritibus immundis, qui curabantur omnes.
 \verse Exsurgens autem princeps sacerdotum et omnes, qui cum illo erant, quae est haeresis sadducaeorum, repleti sunt zelo 
 \verse et iniecerunt manus in apostolos et posuerunt illos in custodia publica. 
 \verse Angelus autem Domini per noctem aperuit ianuas carceris et educens eos dixit: 
 \verse “ Ite et stantes loquimini in templo plebi omnia verba vitae huius ”. 
 \verse Qui cum audissent, intraverunt diluculo in templum et docebant.
 Adveniens autem princeps sacerdotum et, qui cum eo erant, convocaverunt concilium et omnes seniores filiorum Israel et miserunt in carcerem, ut adducerentur illi. 
 \verse Cum venissent autem ministri, non invenerunt illos in carcere; reversi autem nuntiaverunt 
 \verse dicentes: “ Carcerem invenimus clausum cum omni diligentia et custodes stantes ad ianuas; aperientes autem intus neminem invenimus! ”. 
 \verse Ut audierunt autem hos sermones, magistratus templi et principes sacerdotum ambigebant de illis quidnam fieret illud. 
 \verse Adveniens autem quidam nuntiavit eis: “ Ecce viri, quos posuistis in carcere, sunt in templo stantes et docentes populum ”.
 \verse Tunc abiens magistratus cum ministris adducebat illos, non per vim; timebant enim populum, ne lapidarentur. 
 \verse Et cum adduxissent illos, statuerunt in concilio. Et interrogavit eos princeps sacerdotum 
 \verse dicens: “ Nonne praecipiendo praecepimus vobis, ne doceretis in nomine isto? Et ecce replevistis Ierusalem doctrina vestra et vultis inducere super nos sanguinem hominis istius ”. 
 \verse Respondens autem Petrus et apostoli dixerunt: “ Oboedire oportet Deo magis quam hominibus.
\verse Deus patrum nostrorum suscitavit Iesum, quem vos interemistis suspendentes in ligno; 
 \verse hunc Deus Ducem et Salvatorem exaltavit dextera sua ad dandam paenitentiam Israel et remissionem peccatorum. 
 \verse Et nos sumus testes horum verborum, et Spiritus Sanctus, quem dedit Deus oboedientibus sibi ”.
\verse Haec cum audissent, dissecabantur et volebant interficere illos.
\verse Surgens autem quidam in concilio pharisaeus nomine Gamaliel, legis doctor honorabilis universae plebi, iussit foras ad breve homines fieri 
 \verse dixitque ad illos: “ Viri Israelitae, attendite vobis super hominibus istis quid acturi sitis. 
 \verse Ante hos enim dies exstitit Theudas dicens esse se aliquem, cui consensit virorum numerus circiter quadringentorum; qui occisus est, et omnes, quicumque credebant ei, dissipati sunt et redacti sunt ad nihilum. 
 \verse Post hunc exstitit Iudas Galilaeus in diebus census et avertit populum post se; et ipse periit, et omnes, quotquot consentiebant ei, dispersi sunt. 
 \verse Et nunc dico vobis: Discedite ab hominibus istis et sinite illos. Quoniam si est ex hominibus consilium hoc aut opus hoc, dissolvetur; 
 \verse si vero ex Deo est, non poteritis dissolvere eos, ne forte et adversus Deum pugnantes inveniamini! ”.
 Consenserunt autem illi 
 \verse et convocantes apostolos, caesis denuntiaverunt, ne loquerentur in nomine Iesu, et dimiserunt eos. 
 \verse Et illi quidem ibant gaudentes a conspectu concilii, quoniam digni habiti sunt pro nomine contumeliam pati; 
 \verse et omni die in templo et circa domos non cessabant docentes et evangelizantes Christum, Iesum.
 
\begin{biblechapter}
 \verse In diebus autem illis, crescente numero discipulorum, factus est murmur Graecorum adversus Hebraeos, eo quod neglegerentur in ministerio cotidiano viduae eorum. 
 \verse Convocantes autem Duodecim multitudinem discipulorum, dixerunt: “ Non est aequum nos derelinquentes verbum Dei ministrare mensis; 
 \verse considerate vero, fratres, viros ex vobis boni testimonii septem plenos Spiritu et sapientia, quos constituemus super hoc opus; 
 \verse nos vero orationi et ministerio verbi instantes erimus ”. 
 \verse Et placuit sermo coram omni multitudine; et elegerunt Stephanum, virum plenum fide et Spiritu Sancto, et Philippum et Prochorum et Nicanorem et Timonem et Parmenam et Nicolaum proselytum Antiochenum, 
 \verse quos statuerunt ante conspectum apostolorum, et orantes imposuerunt eis manus.
 \verse Et verbum Dei crescebat, et multiplicabatur numerus discipulorum in Ierusalem valde; multa etiam turba sacerdotum oboediebat fidei.
 \verse Stephanus autem plenus gratia et virtute faciebat prodigia et signa magna in populo. 
 \verse Surrexerunt autem quidam de synagoga, quae appellatur Libertinorum et Cyrenensium et Alexandrinorum et eorum, qui erant a Cilicia et Asia, disputantes cum Stephano; 
 \verse et non poterant resistere sapientiae et Spiritui, quo loquebatur. 
 \verse Tunc submiserunt viros, qui dicerent: “ Audivimus eum dicentem verba blasphema in Moysen et Deum ”; 
 \verse et commoverunt plebem et seniores et scribas, et concurrentes rapuerunt eum et adduxerunt in concilium 
 \verse et statuerunt testes falsos dicentes: “ Homo iste non cessat loqui verba adversus locum sanctum et Legem; 
 \verse audivimus enim eum dicentem quoniam Iesus Nazarenus hic destruet locum istum et mutabit consuetudines, quas tradidit nobis Moyses ”.
 \verse Et intuentes eum omnes, qui sedebant in concilio, viderunt faciem eius tamquam faciem angeli.
 
\begin{biblechapter}
 \verse Dixit autem princeps sacerdo tum: “ Si haec ita se habent? ”. 
 \verse Qui ait: “ Viri fratres et patres, audite. Deus gloriae apparuit patri nostro Abraham, cum esset in Mesopotamia, priusquam moraretur in Charran, 
 \verse et dixit ad illum: “Exi de terra tua et de cognatione tua et veni in terram, quam tibi monstravero”. 
 \verse Tunc egressus de terra Chaldaeorum habitavit in Charran. Et inde, postquam mortuus est pater eius, transtulit illum in terram istam, in qua nunc vos habitatis; 
 \verse et non dedit illi hereditatem in ea nec passum pedis et repromisit dare illi eam in possessionem et semini eius post ipsum, cum non haberet filium. 
 \verse Locutus est autem sic Deus: “Erit semen eius accola in terra aliena, et servituti eos subicient et male tractabunt annis quadringentis; 
 \verse et gentem, cui servierint, iudicabo ego, dixit Deus; et post haec exibunt et deservient mihi in loco isto”. 
 \verse Et dedit illi testamentum circumcisionis; et sic genuit Isaac et circumcidit eum die octava, et Isaac Iacob, et Iacob duodecim patriarchas. 
 \verse Et patriarchae aemulantes Ioseph vendiderunt in Aegyptum; et erat Deus cum eo 
 \verse et eripuit eum ex omnibus tribulationibus eius et dedit ei gratiam et sapientiam in conspectu pharaonis regis Aegypti; et constituit eum praepositum super Aegyptum et super omnem domum suam. 
 \verse Venit autem fames in universam Aegyptum et Chanaan et tribulatio magna, et non inveniebant cibos patres nostri. 
 \verse Cum audisset autem Iacob esse frumentum in Aegypto, misit patres nostros primum; 
 \verse et in secundo cognitus est Ioseph a fratribus suis, et manifestatum est pharaoni genus Ioseph. 
 \verse Mittens autem Ioseph accersivit Iacob patrem suum et omnem cognationem in animabus septuaginta quinque; 
 \verse et descendit Iacob in Aegyptum. Et defunctus est ipse et patres nostri; 
 \verse et translati sunt in Sichem et positi sunt in sepulcro, quod emit Abraham pretio argenti a filiis Hemmor in Sichem.
 \verse Cum appropinquaret autem tempus repromissionis, quam confessus erat Deus Abrahae, crevit populus et multiplicatus est in Aegypto, 
 \verse quoadusque surrexit rex alius super Aegypto, qui non sciebat Ioseph. 
 \verse Hic circumveniens genus nostrum, afflixit patres, ut exponerent infantes suos, ne vivi servarentur.
\verse Eodem tempore natus est Moyses et erat formosus coram Deo; qui nutritus est tribus mensibus in domo patris. 
 \verse Exposito autem illo, sustulit eum filia pharaonis et enutrivit eum sibi in filium; 
 \verse et eruditus est Moyses in omni sapientia Aegyptiorum; et erat potens in verbis et in operibus suis. 
 \verse Cum autem impleretur ei quadraginta annorum tempus, ascendit in cor eius, ut visitaret fratres suos filios Israel. 
 \verse Et cum vidisset quendam iniuriam patientem, vindicavit et fecit ultionem ei, qui opprimebatur, percusso Aegyptio. 
 \verse Existimabat autem intellegere fratres, quoniam Deus per manum ipsius daret salutem illis; at illi non intellexerunt. 
 \verse Atque sequenti die apparuit illis litigantibus et reconciliabat eos in pacem dicens: “Viri, fratres estis; ut quid nocetis alterutrum?”. 
 \verse Qui autem iniuriam faciebat proximo, reppulit eum dicens: “Quis te constituit principem et iudicem super nos? 
 \verse Numquid interficere me tu vis, quemadmodum interfecisti heri Aegyptium?”. 
 \verse Fugit autem Moyses propter verbum istud; et factus est advena in terra Madian, ubi generavit filios duos.
 \verse Et expletis annis quadraginta, apparuit illi in deserto montis Sinai angelus in ignis flamma rubi. 
 \verse Moyses autem videns admirabatur visum; accedente autem illo, ut consideraret, facta est vox Domini: 
 \verse “Ego Deus patrum tuorum, Deus Abraham et Isaac et Iacob”. Tremefactus autem Moyses non audebat considerare. 
 \verse Dixit autem illi Dominus: “Solve calceamentum pedum tuorum; locus enim, in quo stas, terra sancta est. 
 \verse Videns vidi afflictionem populi mei, qui est in Aegypto, et gemitum eorum audivi et descendi liberare eos; et nunc veni, mittam te in Aegyptum”.
 \verse Hunc Moysen, quem negaverunt dicentes: “Quis te constituit principem et iudicem?”, hunc Deus et principem et redemptorem misit cum manu angeli, qui apparuit illi in rubo. 
 \verse Hic eduxit illos faciens prodigia et signa in terra Aegypti et in Rubro mari et in deserto annis quadraginta. 
 \verse Hic est Moyses, qui dixit filiis Israel: “Prophetam vobis suscitabit Deus de fratribus vestris tamquam me”. 
 \verse Hic est qui fuit in ecclesia in solitudine cum angelo, qui loquebatur ei in monte Sinai, et cum patribus nostris; qui accepit verba viva dare nobis; 
 \verse cui noluerunt oboedire patres nostri, sed reppulerunt et aversi sunt in cordibus suis in Aegyptum
\verse dicentes ad Aaron: “Fac nobis deos, qui praecedant nos; Moyses enim hic, qui eduxit nos de terra Aegypti, nescimus quid factum sit ei”. 
 \verse Et vitulum fecerunt in illis diebus et obtulerunt hostiam simulacro et laetabantur in operibus manuum suarum. 
 \verse Convertit autem Deus et tradidit eos servire militiae caeli, sicut scriptum est in libro Prophetarum:
 “Numquid victimas et hostias obtulistis mihi
 annis quadraginta in deserto, domus Israel?
 \verse Et suscepistis tabernaculum Moloch
 et sidus dei vestri Rhaephan,
 figuras, quas fecistis ad adorandum eas.
 Et transferam vos trans Babylonem”.
 \verse Tabernaculum testimonii erat patribus nostris in deserto, sicut disposuit, qui loquebatur ad Moysen, ut faceret illud secundum formam, quam viderat; 
 \verse quod et induxerunt suscipientes patres nostri cum Iesu in possessionem gentium, quas expulit Deus a facie patrum nostrorum, usque in diebus David, 
 \verse qui invenit gratiam ante Deum et petiit, ut inveniret tabernaculum domui Iacob. 
 \verse Salomon autem aedificavit illi domum. 
 \verse Sed non Altissimus in manufactis habitat, sicut propheta dicit:
 \verse “Caelum mihi thronus est,
 terra autem scabellum pedum meorum.
 Quam domum aedificabitis mihi, dicit Dominus,
 aut quis locus requietionis meae?
 \verse Nonne manus mea fecit haec omnia?”.
 \verse Duri cervice et incircumcisi cordibus et auribus, vos semper Spiritui Sancto resistitis; sicut patres vestri, et vos. 
 \verse Quem prophetarum non sunt persecuti patres vestri? Et occiderunt eos, qui praenuntiabant de adventu Iusti, cuius vos nunc proditores et homicidae fuistis, 
 \verse qui accepistis legem in dispositionibus angelorum et non custodistis ”.
 \verse Audientes autem haec, dissecabantur cordibus suis et stridebant dentibus in eum. 
 \verse Cum autem esset plenus Spiritu Sancto, intendens in caelum vidit gloriam Dei et Iesum stantem a dextris Dei 
 \verse et ait: “ Ecce video caelos apertos et Filium hominis a dextris stantem Dei ”. 
 \verse Exclamantes autem voce magna continuerunt aures suas et impetum fecerunt unanimiter in eum 
 \verse et eicientes extra civitatem lapidabant. Et testes deposuerunt vestimenta sua secus pedes adulescentis, qui vocabatur Saulus. 
 \verse Et lapidabant Stephanum invocantem et dicentem: “ Domine Iesu, suscipe spiritum meum ”. 
 \verse Positis autem genibus clamavit voce magna: “ Domine, ne statuas illis hoc peccatum ”; et cum hoc dixisset, obdormivit.
 
\begin{biblechapter}
 \verse Saulus autem erat consentiens neci eius. Facta est autem in illa die persecutio magna in ecclesiam, quae erat Hierosolymis; et omnes dispersi sunt per regiones Iudaeae et Samariae praeter apostolos.
\verse Sepelierunt autem Stephanum viri timorati et fecerunt planctum magnum super illum. 
 \verse Saulus vero devastabat ecclesiam, per domos intrans et trahens viros ac mulieres tradebat in custodiam.
 \verse Igitur qui dispersi erant, pertransierunt evangelizantes verbum.
 \verse Philippus autem descendens in civitatem Samariae praedicabat illis Christum. 
 \verse Intendebant autem turbae his, quae a Philippo dicebantur, unanimiter, audientes et videntes signa, quae faciebat: 
 \verse ex multis enim eorum, qui habebant spiritus immundos, clamantes voce magna exibant; multi autem paralytici et claudi curati sunt. 
 \verse Factum est autem magnum gaudium in illa civitate. 
 \verse Vir autem quidam nomine Simon iampridem erat in civitate magias faciens et dementans gentem Samariae, dicens esse se aliquem magnum; 
\verse cui attendebant omnes a minimo usque ad maximum dicentes: “ Hic est virtus Dei, quae vocatur Magna ”. 
\verse Attendebant autem eum, propter quod multo tempore magiis dementasset eos. 
\verse Cum vero credidissent Philippo evangelizanti de regno Dei et nomine Iesu Christi, baptizabantur viri ac mulieres. 
\verse Tunc Simon et ipse credidit et, cum baptizatus esset, adhaerebat Philippo; videns etiam signa et virtutes magnas fieri stupens admirabatur.
 \verse Cum autem audissent apostoli, qui erant Hierosolymis, quia recepit Samaria verbum Dei, miserunt ad illos Petrum et Ioannem; 
\verse qui cum descendissent, oraverunt pro ipsis, ut acciperent Spiritum Sanctum: 
\verse nondum enim super quemquam illorum venerat, sed baptizati tantum erant in nomine Domini Iesu. 
 \verse Tunc imposuerunt manus super illos, et accipiebant Spiritum Sanctum.
 \verse Cum vidisset autem Simon quia per impositionem manuum apostolorum daretur Spiritus, obtulit eis pecuniam 
\verse dicens: “ Date et mihi hanc potestatem, ut cuicumque imposuero manus, accipiat Spiritum Sanctum ”. 
\verse Petrus autem dixit ad eum: “ Argentum tuum tecum sit in perditionem, quoniam donum Dei existimasti pecunia possideri! 
\verse Non est tibi pars neque sors in verbo isto; cor enim tuum non est rectum coram Deo. 
\verse Paenitentiam itaque age ab hac nequitia tua et roga Dominum, si forte remittatur tibi haec cogitatio cordis tui; 
\verse in felle enim amaritudinis et obligatione iniquitatis video te esse ”. 
\verse Respondens autem Simon dixit: “ Precamini vos pro me ad Dominum, ut nihil veniat super me horum, quae dixistis ”. 
\verse Et illi quidem testificati et locuti verbum Domini, redibant Hierosolymam et multis vicis Samaritanorum evangelizabant.
 \verse Angelus autem Domini locutus est ad Philippum dicens: “ Surge et vade contra meridianum ad viam, quae descendit ab Ierusalem in Gazam; haec est deserta ”. 
 \verse Et surgens abiit; et ecce vir Aethiops eunuchus potens Candacis reginae Aethiopum, qui erat super omnem gazam eius, qui venerat adorare in Ierusalem 
 \verse et revertebatur sedens super currum suum et legebat prophetam Isaiam. 
\verse Dixit autem Spiritus Philippo: “ Accede et adiunge te ad currum istum ”. 
\verse Accurrens autem Philippus audivit illum legentem Isaiam prophetam et dixit: “ Putasne intellegis, quae legis? ”. 
\verse Qui ait: “ Et quomodo possum, si non aliquis ostenderit mihi? ”. Rogavitque Philippum, ut ascenderet et sederet secum. 
\verse Locus autem Scripturae, quem legebat, erat hic:
 “ Tamquam ovis ad occisionem ductus est
 et sicut agnus coram tondente se sine voce,
 sic non aperit os suum.
 \verse In humilitate eius iudicium eius sublatum est.
 Generationem illius quis enarrabit?
 Quoniam tollitur de terra vita eius ”.
 \verse Respondens autem eunuchus Philippo dixit: “ Obsecro te, de quo propheta dicit hoc? De se an de alio aliquo? ”. 
\verse Aperiens autem Philippus os suum et incipiens a Scriptura ista, evangelizavit illi Iesum. 
\verse Et dum irent per viam, venerunt ad quandam aquam; et ait eunuchus: “ Ecce aqua; quid prohibet me baptizari? ”.
(\verse) \verse Et iussit stare currum; et descenderunt uterque in aquam Philippus et eunuchus, et baptizavit eum. 
\verse Cum autem ascendissent de aqua, Spiritus Domini rapuit Philippum, et amplius non vidit eum eunuchus; ibat autem per viam suam gaudens. 
\verse Philippus autem inventus est in Azoto et pertransiens evangelizabat civitatibus cunctis, donec veniret Caesaream.
 
\begin{biblechapter}
\verse Saulus autem, adhuc spirans minarum et caedis in discipulos Domini, accessit ad principem sacerdotum 
\verse et petiit ab eo epistulas in Damascum ad synagogas, ut, si quos invenisset huius viae viros ac mulieres, vinctos perduceret in Ierusalem. 
\verse Et cum iter faceret, contigit ut appropinquaret Damasco; et subito circumfulsit eum lux de caelo, 
\verse et cadens in terram audivit vocem dicentem sibi: “ Saul, Saul, quid me persequeris? ”. 
\verse Qui dixit: “ Quis es, Domine? ”. Et ille: “ Ego sum Iesus, quem tu persequeris! 
 \verse Sed surge et ingredere civitatem, et dicetur tibi quid te oporteat facere ”. 
 \verse Viri autem illi, qui comitabantur cum eo, stabant stupefacti, audientes quidem vocem, neminem autem videntes. 
\verse Surrexit autem Saulus de terra; apertisque oculis, nihil videbat; ad manus autem illum trahentes introduxerunt Damascum. 
 \verse Et erat tribus diebus non videns et non manducavit neque bibit.
 \verse Erat autem quidam discipulus Damasci nomine Ananias; et dixit ad illum in visu Dominus: “ Anania ”. At ille ait: “ Ecce ego, Domine ”. 
\verse Et Dominus ad illum: “ Surgens vade in vicum, qui vocatur Rectus, et quaere in domo Iudae Saulum nomine Tarsensem; ecce enim orat 
\verse et vidit virum Ananiam nomine introeuntem et imponentem sibi manus, ut visum recipiat ”. 
\verse Respondit autem Ananias: “ Domine, audivi a multis de viro hoc, quanta mala sanctis tuis fecerit in Ierusalem; 
\verse et hic habet potestatem a principibus sacerdotum alligandi omnes, qui invocant nomen tuum ”. 
\verse Dixit autem ad eum Dominus: “ Vade, quoniam vas electionis est mihi iste, ut portet nomen meum coram gentibus et regibus et filiis Israel; 
\verse ego enim ostendam illi quanta oporteat eum pro nomine meo pati ”.
 \verse Et abiit Ananias; et introivit in domum et imponens ei manus dixit: “ Saul frater, Dominus misit me, Iesus qui apparuit tibi in via, qua veniebas, ut videas et implearis Spiritu Sancto ”. 
\verse Et confestim ceciderunt ab oculis eius tamquam squamae, et visum recepit. Et surgens baptizatus est 
\verse et, cum accepisset cibum, confortatus est.
 Fuit autem cum discipulis, qui erant Damasci, per dies aliquot; 
\verse et continuo in synagogis praedicabat Iesum, quoniam hic est Filius Dei. 
\verse Stupebant autem omnes, qui audiebant, et dicebant: “ Nonne hic est, qui expugnabat in Ierusalem eos, qui invocabant nomen istud, et huc ad hoc venerat, ut vinctos illos duceret ad principes sacerdotum? ”. 
\verse Saulus autem magis convalescebat et confundebat Iudaeos, qui habitabant Damasci, affirmans quoniam hic est Christus. 
\verse Cum implerentur autem dies multi, consilium fecerunt Iudaei, ut eum interficerent; 
\verse notae autem factae sunt Saulo insidiae eorum. Custodiebant autem et portas die ac nocte, ut eum interficerent; 
\verse accipientes autem discipuli eius nocte per murum dimiserunt eum submittentes in sporta.
 \verse Cum autem venisset in Ierusalem, tentabat iungere se discipulis; et omnes timebant eum, non credentes quia esset discipulus. 
\verse Barnabas autem apprehensum illum duxit ad apostolos et narravit illis quomodo in via vidisset Dominum, et quia locutus est ei, et quomodo in Damasco fiducialiter egerit in nomine Iesu. 
\verse Et erat cum illis intrans et exiens in Ierusalem, fiducialiter agens in nomine Domini. 
\verse Loquebatur quoque et disputabat cum Graecis; illi autem quaerebant occidere eum. 
\verse Quod cum cognovissent, fratres deduxerunt eum Caesaream et dimiserunt Tarsum.
 \verse Ecclesia quidem per totam Iudaeam et Galilaeam et Samariam habebat pacem; aedificabatur et ambulabat in timore Domini et consolatione Sancti Spiritus crescebat.
 \verse Factum est autem Petrum, dum pertransiret universos, devenire et ad sanctos, qui habitabant Lyddae. 
\verse Invenit autem ibi hominem quendam nomine Aeneam ab annis octo iacentem in grabato, qui erat paralyticus. 
\verse Et ait illi Petrus: “ Aenea, sanat te Iesus Christus; surge et sterne tibi ”. Et continuo surrexit. 
\verse Et viderunt illum omnes, qui inhabitabant Lyddam et Saron, qui conversi sunt ad Dominum.
 \verse In Ioppe autem erat quaedam discipula nomine Tabitha, quae interpretata dicitur Dorcas; haec erat plena operibus bonis et eleemosynis, quas faciebat. 
 \verse Factum est autem in diebus illis ut infirmata moreretur; quam cum lavissent, posuerunt in cenaculo. 
\verse Cum autem prope esset Lydda ab Ioppe, discipuli audientes quia Petrus esset in ea, miserunt duos viros ad eum rogantes: “ Ne pigriteris venire usque ad nos! ”. 
\verse Exsurgens autem Petrus venit cum illis; et cum advenisset, duxerunt illum in cenaculum; et circumsteterunt illum omnes viduae flentes et ostendentes tunicas et vestes, quas faciebat Dorcas, cum esset cum illis. 
\verse Eiectis autem omnibus foras Petrus, et ponens genua oravit et conversus ad corpus dixit: “ Tabitha, surge! ”. At illa aperuit oculos suos et, viso Petro, resedit. 
\verse Dans autem illi manum erexit eam et, cum vocasset sanctos et viduas, exhibuit eam vivam.
 \verse Notum autem factum est per universam Ioppen, et crediderunt multi in Domino. 
\verse Factum est autem, ut dies multos moraretur in Ioppe apud quendam Simonem coriarium.
 
\begin{biblechapter}
\verse Vir autem quidam in Cae sarea nomine Cornelius, cen turio cohortis, quae dicitur Italica, 
\verse religiosus et timens Deum cum omni domo sua, faciens eleemosynas multas plebi et deprecans Deum semper, 
\verse vidit in visu manifeste quasi hora nona diei angelum Dei introeuntem ad se et dicentem sibi: “ Corneli ”. 
\verse At ille intuens eum et timore correptus dixit: “ Quid est, domine? ”. Dixit autem illi: “ Orationes tuae et eleemosynae tuae ascenderunt in memoriam in conspectu Dei. 
\verse Et nunc mitte viros in Ioppen et accersi Simonem quendam, qui cognominatur Petrus; 
\verse hic hospitatur apud Simonem quendam coriarium, cui est domus iuxta mare ”. 
\verse Ut autem discessit angelus, qui loquebatur illi, cum vocasset duos domesticos suos et militem religiosum ex his, qui illi parebant, 
\verse et narrasset illis omnia, misit illos in Ioppen.
 \verse Postera autem die, iter illis facientibus et appropinquantibus civitati, ascendit Petrus super tectum, ut oraret circa horam sextam. 
\verse Et cum esuriret, voluit gustare; parantibus autem eis, cecidit super eum mentis excessus, 
\verse et videt caelum apertum et descendens vas quoddam velut linteum magnum quattuor initiis submitti in terram, 
\verse in quo erant omnia quadrupedia et serpentia terrae et volatilia caeli. 
\verse Et facta est vox ad eum: “ Surge, Petre, occide et manduca! ”. 
\verse Ait autem Petrus: “ Nequaquam, Domine, quia numquam manducavi omne commune et immundum ”. 
\verse Et vox iterum secundo ad eum: “ Quae Deus purificavit, ne tu commune dixeris ”. 
\verse Hoc autem factum est per ter, et statim receptum est vas in caelum. 
\verse Et dum intra se haesitaret Petrus quidnam esset visio, quam vidisset, ecce viri, qui missi erant a Cornelio, inquirentes domum Simonis astiterunt ad ianuam 
\verse et, cum vocassent, interrogabant si Simon, qui cognominatur Petrus, illic haberet hospitium. 
\verse Petro autem cogitante de visione, dixit Spiritus ei: “ Ecce viri tres quaerunt te; 
\verse surge itaque et descende et vade cum eis nihil dubitans, quia ego misi illos ”. 
\verse Descendens autem Petrus ad viros dixit: “ Ecce ego sum, quem quaeritis; quae causa est, propter quam venistis? ”. 
 \verse Qui dixerunt: “ Cornelius centurio, vir iustus et timens Deum et testimonium habens ab universa gente Iudaeorum, responsum accepit ab angelo sancto accersire te in domum suam et audire verba abs te ”. 
\verse Invitans igitur eos recepit hospitio.
 Sequenti autem die, surgens profectus est cum eis, et quidam ex fratribus ab Ioppe comitati sunt eum. 
\verse Altera autem die introivit Caesaream; Cornelius vero exspectabat illos, convocatis cognatis suis et necessariis amicis. 
\verse Et factum est, cum introisset Petrus, obvius ei Cornelius procidens ad pedes adoravit. 
\verse Petrus vero levavit eum dicens: “ Surge, et ego ipse homo sum ”. 
\verse Et loquens cum illo intravit et invenit multos, qui convenerant; 
 \verse dixitque ad illos: “ Vos scitis quomodo illicitum sit viro Iudaeo coniungi aut accedere ad alienigenam. Et mihi ostendit Deus neminem communem aut immundum dicere hominem; 
\verse propter quod sine dubitatione veni accersitus. Interrogo ergo quam ob causam accersistis me ”. 
\verse Et Cornelius ait: “ A nudius quarta die usque in hanc horam orans eram hora nona in domo mea, et ecce vir stetit ante me in veste candida 
\verse et ait: “Corneli, exaudita est oratio tua, et eleemosynae tuae commemoratae sunt in conspectu Dei. 
\verse Mitte ergo in Ioppen et accersi Simonem, qui cognominatur Petrus; hic hospitatur in domo Simonis coriarii iuxta mare”. 
\verse Confestim igitur misi ad te, et tu bene fecisti veniendo. Nunc ergo omnes nos in conspectu Dei adsumus audire omnia, quaecumque tibi praecepta sunt a Domino ”.
 \verse Aperiens autem Petrus os dixit: “ In veritate comperio quoniam non est personarum acceptor Deus, 
\verse sed in omni gente, qui timet eum et operatur iustitiam, acceptus est illi. 
\verse Verbum misit filiis Israel evangelizans pacem per Iesum Christum; hic est omnium Dominus. 
\verse Vos scitis quod factum est verbum per universam Iudaeam incipiens a Galilaea post baptismum, quod praedicavit Ioannes: 
\verse Iesum a Nazareth, quomodo unxit eum Deus Spiritu Sancto et virtute, qui pertransivit benefaciendo et sanando omnes oppressos a Diabolo, quoniam Deus erat cum illo. 
\verse Et nos testes sumus omnium, quae fecit in regione Iudaeorum et Ierusalem; quem et occiderunt suspendentes in ligno. 
\verse Hunc Deus suscitavit tertia die et dedit eum manifestum fieri 
 \verse non omni populo, sed testibus praeordinatis a Deo, nobis, qui manducavimus et bibimus cum illo postquam resurrexit a mortuis; 
\verse et praecepit nobis praedicare populo et testificari quia ipse est, qui constitutus est a Deo iudex vivorum et mortuorum. 
\verse Huic omnes Prophetae testimonium perhibent remissionem peccatorum accipere per nomen eius omnes, qui credunt in eum ”.
 \verse Adhuc loquente Petro verba haec, cecidit Spiritus Sanctus super omnes, qui audiebant verbum. 
\verse Et obstupuerunt, qui ex circumcisione fideles, qui venerant cum Petro, quia et in nationes gratia Spiritus Sancti effusa est; 
 \verse audiebant enim illos loquentes linguis et magnificantes Deum. Tunc respondit Petrus: 
\verse “ Numquid aquam quis prohibere potest, ut non baptizentur hi, qui Spiritum Sanctum acceperunt sicut et nos? ”. 
\verse Et iussit eos in nomine Iesu Christi baptizari. Tunc rogaverunt eum, ut maneret aliquot diebus.
 
\begin{biblechapter}
\verse Audierunt autem apostoli et fratres, qui erant in Iudaea, quoniam et gentes receperunt verbum Dei. 
\verse Cum ascendisset autem Petrus in Ierusalem, disceptabant adversus illum, qui erant ex circumcisione, 
\verse dicentes: “ Introisti ad viros praeputium habentes et manducasti cum illis! ”.
 \verse Incipiens autem Petrus exponebat illis ex ordine dicens: 
\verse “ Ego eram in civitate Ioppe orans et vidi in excessu mentis visionem, descendens vas quoddam velut linteum magnum quattuor initiis submitti de caelo et venit usque ad me; 
 \verse in quod intuens considerabam et vidi quadrupedia terrae et bestias et reptilia et volatilia caeli. 
\verse Audivi autem et vocem dicentem mihi: “Surgens, Petre, occide et manduca!”. 
\verse Dixi autem: Nequaquam, Domine, quia commune aut immundum numquam introivit in os meum. 
\verse Respondit autem vox secundo de caelo: “Quae Deus mundavit, tu ne commune dixeris”. 
\verse Hoc autem factum est per ter, et retracta sunt rursum omnia in caelum. 
\verse Et ecce confestim tres viri astiterunt in domo, in qua eramus, missi a Caesarea ad me. 
\verse Dixit autem Spiritus mihi, ut irem cum illis nihil haesitans. Venerunt autem mecum et sex fratres isti, et ingressi sumus in domum viri. 
\verse Narravit autem nobis quomodo vidisset angelum ad domum suam stantem et dicentem: “Mitte in Ioppen et accersi Simonem, qui cognominatur Petrus, 
\verse qui loquetur tibi verba, in quibus salvus eris tu et universa domus tua”. 
\verse Cum autem coepissem loqui, decidit Spiritus Sanctus super eos, sicut et super nos in initio. 
\verse Recordatus sum autem verbi Domini, sicut dicebat: “Ioannes quidem baptizavit aqua, vos autem baptizabimini in Spiritu Sancto”. 
\verse Si ergo aequale donum dedit illis Deus sicut et nobis, qui credidimus in Dominum Iesum Christum, ego quis eram qui possem prohibere Deum? ”.
 \verse His autem auditis, acquieverunt et glorificaverunt Deum dicentes: “ Ergo et gentibus Deus paenitentiam ad vitam dedit ”.
 \verse Et illi quidem, qui dispersi fuerant a tribulatione, quae facta fuerat sub Stephano, perambulaverunt usque Phoenicen et Cyprum et Antiochiam, nemini loquentes verbum; nisi solis Iudaeis. 
\verse Erant autem quidam ex eis viri Cyprii et Cyrenaei, qui, cum introissent Antiochiam, loquebantur et ad Graecos evangelizantes Dominum Iesum. 
\verse Et erat manus Domini cum eis; multusque numerus credentium conversus est ad Dominum.
 \verse Auditus est autem sermo in auribus ecclesiae, quae erat in Ierusalem, super istis, et miserunt Barnabam usque Antiochiam; 
\verse qui cum pervenisset et vidisset gratiam Dei, gavisus est et hortabatur omnes proposito cordis permanere in Domino, 
\verse quia erat vir bonus et plenus Spiritu Sancto et fide. Et apposita est turba multa Domino. 
\verse Profectus est autem Tarsum, ut quaereret Saulum; 
\verse quem cum invenisset, perduxit Antiochiam. Factum est autem eis, ut annum totum conversarentur in ecclesia et docerent turbam multam, et cognominarentur primum Antiochiae discipuli Christiani.
 \verse In his autem diebus supervenerunt ab Hierosolymis prophetae Antiochiam; 
 \verse et surgens unus ex eis nomine Agabus significavit per Spiritum famem magnam futuram in universo orbe terrarum; quae facta est sub Claudio. 
\verse Discipuli autem, prout quis habebat, proposuerunt singuli eorum in ministerium mittere habitantibus in Iudaea fratribus; 
\verse quod et fecerunt, mittentes ad presbyteros per manum Barnabae et Sauli.
 
\begin{biblechapter}
\verse Illo autem tempore, misit Herodes rex manus, ut affli geret quosdam de ecclesia. 
\verse Occidit autem Iacobum fratrem Ioannis gladio. 
\verse Videns autem quia placeret Iudaeis, apposuit apprehendere et Petrum — erant autem dies Azymorum — 
\verse quem cum apprehendisset, misit in carcerem tradens quattuor quaternionibus militum custodire eum, volens post Pascha producere eum populo. 
 \verse Et Petrus quidem servabatur in carcere; oratio autem fiebat sine intermissione ab ecclesia ad Deum pro eo. 
\verse Cum autem producturus eum esset Herodes, in ipsa nocte erat Petrus dormiens inter duos milites vinctus catenis duabus, et custodes ante ostium custodiebant carcerem. 
\verse Et ecce angelus Domini astitit, et lumen refulsit in habitaculo; percusso autem latere Petri, suscitavit eum dicens: “ Surge velociter! ”. Et ceciderunt catenae de manibus eius. 
\verse Dixit autem angelus ad eum: “ Praecingere et calcea te sandalia tua! ”. Et fecit sic. Et dicit illi: “ Circumda tibi vestimentum tuum et sequere me! ”. 
\verse Et exiens sequebatur et nesciebat quia verum est, quod fiebat per angelum; aestimabat autem se visum videre.
 \verse Transeuntes autem primam custodiam et secundam venerunt ad portam ferream, quae ducit ad civitatem, quae ultro aperta est eis, et exeuntes processerunt vicum unum, et continuo discessit angelus ab eo. 
\verse Et Petrus ad se reversus dixit: “ Nunc scio vere quia misit Dominus angelum suum et eripuit me de manu Herodis et de omni exspectatione plebis Iudaeorum ”. 
\verse Consideransque venit ad domum Mariae matris Ioannis, qui cognominatur Marcus, ubi erant multi congregati et orantes. 
\verse Pulsante autem eo ostium ianuae, processit puella ad audiendum, nomine Rhode; 
\verse et ut cognovit vocem Petri, prae gaudio non aperuit ianuam, sed intro currens nuntiavit stare Petrum ante ianuam. 
\verse At illi dixerunt ad eam: “ Insanis! ”. Illa autem affirmabat sic se habere. Illi autem dicebant: “ Angelus eius est ”. 
\verse Petrus autem perseverabat pulsans; cum autem aperuissent, viderunt eum et obstupuerunt. 
\verse Annuens autem eis manu, ut tacerent, enarravit quomodo Dominus eduxisset eum de carcere dixitque: “ Nuntiate Iacobo ct fratribus haec ”. Et egressus abiit in alium locum.
 \verse Facta autem die, erat non parva turbatio inter milites, quidnam de Petro factum esset. 
\verse Herodes autem, cum requisisset eum et non invenisset, interrogatis custodibus, iussit eos abduci; descendensque a Iudaea in Caesaream ibi commorabatur.
 \verse Erat autem iratus Tyriis et Sidoniis; at illi unanimes venerunt ad eum et, persuaso Blasto, qui erat super cubiculum regis, postulabant pacem, eo quod aleretur regio eorum ab annona regis. 
\verse Statuto autem die, Herodes, vestitus veste regia, sedens pro tribunalicontionabatur ad eos; 
\verse populus autem acclamabat: “ Dei vox et non hominis! ”. 
\verse Confestim autem percussit eum angelus Domini, eo quod non dedisset gloriam Deo; et consumptus a vermibus exspiravit.
 \verse Verbum autem Dei crescebat et multiplicabatur. 
\verse Barnabas autem et Saulus reversi sunt in Ierusalem expleto ministerio, assumpto Ioanne, qui cognominatus est Marcus.
 
\begin{biblechapter}
\verse Erant autem in ecclesia, quae erat Antiochiae, pro phetae et doctores: Barnabas et Simeon, qui vocabatur Niger, et Lucius Cyrenensis et Manaen, qui erat Herodis tetrarchae collactaneus, et Saulus. 
\verse Ministrantibus autem illis Domino et ieiunantibus, dixit Spiritus Sanctus: “ Separate mihi Barnabam et Saulum in opus, ad quod vocavi eos ”. 
\verse Tunc ieiunantes et orantes imponentesque eis manus dimiserunt illos.
 \verse Et ipsi quidem missi ab Spiritu Sancto devenerunt Seleuciam et inde navigaverunt Cyprum 
\verse et, cum venissent Salamina, praedicabant verbum Dei in synagogis Iudaeorum; habebant autem et Ioannem ministrum. 
\verse Et cum perambulassent universam insulam usque Paphum, invenerunt quendam virum magum pseudoprophetam Iudaeum, cui nomen Bariesu, 
\verse qui erat cum proconsule Sergio Paulo, viro prudente. Hic, accitis Barnaba et Saulo, quaesivit audire verbum Dei; 
\verse resistebat autem illis Elymas, magus, sic enim interpretatur nomen eius, quaerens avertere proconsulem a fide. 
\verse Saulus autem, qui et Paulus, repletus Spiritu Sancto, intuens in eum 
\verse dixit: “ O plene omni dolo et omni fallacia, fili Diaboli, inimice omnis iustitiae, non desines subvertere vias Domini rectas? 
\verse Et nunc, ecce manus Domini super te; et eris caecus, non videns solem usque ad tempus ”. Et confestim cecidit in eum caligo et tenebrae, et circumiens quaerebat, qui eum manum darent. 
\verse Tunc proconsul, cum vidisset factum, credidit admirans super doctrinam Domini.
 \verse Et cum a Papho navigassent, qui erant cum Paulo, venerunt Pergen Pamphyliae; Ioannes autem discedens ab eis reversus est Hierosolymam. 
\verse Illi vero pertranseuntes, a Perge venerunt Antiochiam Pisidiae, et ingressi synagogam die sabbatorum sederunt. 
\verse Post lectionem autem Legis et Prophetarum, miserunt principes synagogae ad eos dicentes: “ Viri fratres, si quis est in vobis sermo exhortationis ad plebem, dicite! ”.
 \verse Surgens autem Paulus et manu silentium indicens ait: “ Viri Israelitae et qui timetis Deum, audite. 
\verse Deus plebis huius Israel elegit patres nostros et plebem exaltavit, cum essent incolae in terra Aegypti, et in brachio excelso eduxit eos ex ea; 
\verse et per quadraginta fere annorum tempus mores eorum sustinuit in deserto; 
\verse et destruens gentes septem in terra Chanaan sorte distribuit terram eorum, 
\verse quasi quadringentos et quinquaginta annos. Et post haec dedit iudices usque ad Samuel prophetam. 
\verse Et exinde postulaverunt regem, et dedit illis Deus Saul filium Cis, virum de tribu Beniamin, annis quadraginta. 
\verse Et amoto illo, suscitavit illis David in regem, cui et testimonium perhibens dixit: “Inveni David filium Iesse, virum secundum cor meum, qui faciet omnes voluntates meas”.
 \verse Huius Deus ex semine secundum promissionem eduxit Israel salvatorem Iesum, 
 \verse praedicante Ioanne ante adventum eius baptismum paenitentiae omni populo Israel. 
\verse Cum impleret autem Ioannes cursum suum, dicebat: “Quid me arbitramini esse? Non sum ego; sed ecce venit post me, cuius non sum dignus calceamenta pedum solvere”.
 \verse Viri fratres, filii generis Abraham et qui in vobis timent Deum, nobis verbum salutis huius missum est. 
\verse Qui enim habitabant Ierusalem et principes eorum, hunc ignorantes et voces Prophetarum, quae per omne sabbatum leguntur, iudicantes impleverunt; 
\verse et nullam causam mortis invenientes petierunt a Pilato, ut interficeretur; 
\verse cumque consummassent omnia, quae de eo scripta erant, deponentes eum de ligno posuerunt in monumento. 
\verse Deus vero suscitavit eum a mortuis; 
\verse qui visus est per dies multos his, qui simul ascenderant cum eo de Galilaea in Ierusalem, qui nunc sunt testes eius ad plebem.
 \verse Et nos vobis evangelizamus eam, quae ad patres promissio facta est, 
\verse quoniam hanc Deus adimplevit filiis eorum, nobis resuscitans Iesum, sicut et in Psalmo secundo scriptum est:
 “Filius meus es tu; ego hodie genui te”.
 \verse Quod autem suscitaverit eum a mortuis, amplius iam non reversurum in corruptionem, ita dixit: “Dabo vobis sancta David fidelia”.
 \verse Ideoque et in alio dicit:
 “Non dabis Sanctum tuum videre corruptionem”.
 \verse David enim sua generatione cum administrasset voluntati Dei, dormivit et appositus est ad patres suos et vidit corruptionem; 
\verse quem vero Deus suscitavit, non vidit corruptionem. 
\verse Notum igitur sit vobis, viri fratres, quia per hunc vobis remissio peccatorum annuntiatur; ab omnibus, quibus non potuistis in lege Moysi iustificari, 
\verse in hoc omnis, qui credit, iustificatur. 
\verse Videte ergo, ne superveniat, quod dictum est in Prophetis:
 \verse “Videte, contemptores,
 et admiramini et disperdimini,
 quia opus operor ego in diebus vestris,
 opus, quod non credetis, si quis enarraverit vobis!” ”.
 \verse Exeuntibus autem illis, rogabant, ut sequenti sabbato loquerentur sibi verba haec. 
\verse Cumque dimissa esset synagoga, secuti sunt multi Iudaeorum et colentium proselytorum Paulum et Barnabam, qui loquentes suadebant eis, ut permanerent in gratia Dei.
 \verse Sequenti vero sabbato paene universa civitas convenit audire verbum Domini. 
\verse Videntes autem turbas Iudaei, repleti sunt zelo; et contradicebant his, quae a Paulo dicebantur, blasphemantes.
 \verse Tunc audenter Paulus et Barnabas dixerunt: “ Vobis oportebat primum loqui verbum Dei; sed quoniam repellitis illud et indignos vos iudicatis aeternae vitae, ecce convertimur ad gentes. 
\verse Sic enim praecepit nobis Dominus:
 “Posui te in lumen gentium,
 ut sis in salutem usque ad extremum terrae” ”.
 \verse Audientes autem gentes gaudebant et glorificabant verbum Domini, et crediderunt, quotquot erant praeordinati ad vitam aeternam; 
\verse ferebatur autem verbum Domini per universam regionem. 
\verse Iudaei autem concitaverunt honestas inter colentes mulieres et primos civitatis et excitaverunt persecutionem in Paulum et Barnabam et eiecerunt eos de finibus suis. 
\verse At illi, excusso pulvere pedum in eos, venerunt Iconium; 
\verse discipuli quoque replebantur gaudio et Spiritu Sancto.
 
\begin{biblechapter}
\verse Factum est autem Iconii, ut eodem modo introirent syna gogam Iudaeorum et ita loquerentur, ut crederet Iudaeorum et Graecorum copiosa multitudo. 
\verse Qui vero increduli fuerunt Iudaei, suscitaverunt et exacerbaverunt animas gentium adversus fratres. 
\verse Multo igitur tempore demorati sunt, fiducialiter agentes in Domino, testimonium perhibente verbo gratiae suae, dante signa et prodigia fieri per manus eorum. 
\verse Divisa est autem multitudo civitatis: et quidam quidem erant cum Iudaeis, quidam vero cum apostolis. 
\verse Cum autem factus esset impetus gentilium et Iudaeorum cum principibus suis, ut contumeliis afficerent et lapidarent eos, 
\verse intellegentes confugerunt ad civitates Lycaoniae, Lystram et Derben et ad regionem in circuitu 
\verse et ibi evangelizantes erant.
 \verse Et quidam vir in Lystris infirmus pedibus sedebat, claudus ex utero matris suae, qui numquam ambulaverat. 
\verse Hic audivit Paulum loquentem; qui intuitus eum et videns quia haberet fidem, ut salvus fieret, 
\verse dixit magna voce: “ Surge super pedes tuos rectus! ”. Et exsilivit et ambulabat. 
\verse Turbae autem cum vidissent, quod fecerat Paulus, levaverunt vocem suam Lycaonice dicentes: “ Dii similes facti hominibus descenderunt ad nos! ”; 
\verse et vocabant Barnabam Iovem, Paulum vero Mercurium, quoniam ipse erat dux verbi.
 \verse Sacerdos quoque templi Iovis, quod erat ante civitatem, tauros et coronas ad ianuas afferens cum populis, volebat sacrificare. 
\verse Quod ubi audierunt apostoli Barnabas et Paulus, conscissis tunicis suis, exsilierunt in turbam clamantes 
\verse et dicentes: “ Viri, quid haec facitis? Et nos mortales sumus similes vobis homines, evangelizantes vobis ab his vanis converti ad Deum vivum, qui fecit caelum et terram et mare et omnia, quae in eis sunt. 
\verse Qui in praeteritis generationibus permisit omnes gentes ambulare in viis suis; 
\verse et quidem non sine testimonio semetipsum reliquit benefaciens, de caelo dans vobis pluvias et tempora fructifera, implens cibo et laetitia corda vestra ”. 
 \verse Et haec dicentes vix sedaverunt turbas, ne sibi immolarent. 
\verse Supervenerunt autem ab Antiochia et Iconio Iudaei et persuasis turbis lapidantesque Paulum trahebant extra civitatem aestimantes eum mortuum esse. 
 \verse Circumdantibus autem eum discipulis, surgens intravit civitatem. Et postera die profectus est cum Barnaba in Derben.
 \verse Cumque evangelizassent civitati illi et docuissent multos, reversi sunt Lystram et Iconium et Antiochiam 
\verse confirmantes animas discipulorum, exhortantes, ut permanerent in fide, et quoniam per multas tribulationes oportet nos intrare in regnum Dei. 
\verse Et cum ordinassent illis per singulas ecclesias presbyteros et orassent cum ieiunationibus, commendaverunt eos Domino, in quem crediderant. 
\verse Transeuntesque Pisidiam venerunt Pamphyliam; 
\verse et loquentes in Perge verbum descenderunt in Attaliam. 
\verse Et inde navigaverunt Antiochiam, unde erant traditi gratiae Dei in opus, quod compleverunt.
 \verse Cum autem venissent et congregassent ecclesiam, rettulerunt quanta fecisset Deus cum illis et quia aperuisset gentibus ostium fidei. 
\verse Morati sunt autem tempus non modicum cum discipulis.
 
\begin{biblechapter}
\verse Et quidam descendentes de Iudaea docebant fratres: “ Nisi circumcidamini secundum morem Moysis, non potestis salvi fieri ”.
 \verse Facta autem seditione et conquisitione non minima Paulo et Barnabae adversum illos, statuerunt, ut ascenderent Paulus et Barnabas et quidam alii ex illis ad apostolos et presbyteros in Ierusalem super hac quaestione. 
\verse Illi igitur deducti ab ecclesia pertransiebant Phoenicen et Samariam narrantes conversionem gentium et faciebant gaudium magnum omnibus fratribus. 
\verse Cum autem venissent Hierosolymam, suscepti sunt ab ecclesia et apostolis et presbyteris et annuntiaverunt quanta Deus fecisset cum illis. 
\verse Surrexerunt autem quidam de haeresi pharisaeorum, qui crediderant, dicentes: “ Oportet circumcidere eos, praecipere quoque servare legem Moysis! ”.
 \verse Conveneruntque apostoli et presbyteri videre de verbo hoc. 
\verse Cum autem magna conquisitio fieret, surgens Petrus dixit ad eos: “ Viri fratres, vos scitis quoniam ab antiquis diebus in vobis elegit Deus per os meum audire gentes verbum evangelii et credere; 
\verse et qui novit corda, Deus, testimonium perhibuit illis dans Spiritum Sanctum sicut et nobis; 
\verse et nihil discrevit inter nos et illos fide purificans corda eorum. 
\verse Nunc ergo quid tentatis Deum imponere iugum super cervicem discipulorum, quod neque patres nostri neque nos portare potuimus? 
\verse Sed per gratiam Domini Iesu credimus salvari quemadmodum et illi ”.
 \verse Tacuit autem omnis multitudo, et audiebant Barnabam et Paulum narrantes quanta fecisset Deus signa et prodigia in gentibus per eos. 
\verse Et postquam tacuerunt, respondit Iacobus dicens: “ Viri fratres, audite me. 
\verse Simeon narravit quemadmodum primum Deus visitavit sumere ex gentibus populum nomini suo; 
\verse et huic concordant verba Prophetarum, sicut scriptum est:
 \verse “Post haec revertar
 et reaedificabo tabernaculum David, quod decidit,
 et diruta eius reaedificabo et erigam illud.
 \verse ut requirant reliqui hominum Dominum
 et omnes gentes, super quas invocatum est nomen meum,
 dicit Dominus faciens haec \verse nota a saeculo”.
 \verse Propter quod ego iudico non inquietari eos, qui ex gentibus convertuntur ad Deum, 
\verse sed scribere ad eos, ut abstineant se a contaminationibus simulacrorum et fornicatione et suffocato et sanguine. 
\verse Moyses enim a generationibus antiquis habet in singulis civitatibus, qui eum praedicent in synagogis, ubi per omne sabbatum legitur ”.
 \verse Tunc placuit apostolis et presbyteris cum omni ecclesia electos viros ex eis mittere Antiochiam cum Paulo et Barnaba: Iudam, qui cognominatur Barsabbas, et Silam, viros primos in fratribus, 
\verse scribentes per manum eorum: “ Apostoli et presbyteri fratres his, qui sunt Antiochiae et Syriae et Ciliciae, fratribus ex gentibus, salutem! 
\verse Quoniam audivimus quia quidam ex nobis, quibus non mandavimus, exeuntes turbaverunt vos verbis evertentes animas vestras, 
\verse placuit nobis collectis in unum eligere viros et mittere ad vos cum carissimis nobis Barnaba et Paulo, 
\verse hominibus, qui tradiderunt animas suas pro nomine Domini nostri Iesu Christi. 
\verse Misimus ergo Iudam et Silam, qui et ipsi verbis referent eadem. 
\verse Visum est enim Spiritui Sancto et nobis nihil ultra imponere vobis oneris quam haec necessario: 
\verse abstinere ab idolothytis et sanguine et suffocatis et fornicatione; a quibus custodientes vos bene agetis. Valete ”.
 \verse Illi igitur dimissi descenderunt Antiochiam et, congregata multitudine, tradiderunt epistulam; 
\verse quam cum legissent, gavisi sunt super consolatione. 
\verse Iudas quoque et Silas, cum et ipsi essent prophetae, verbo plurimo consolati sunt fratres et confirmaverunt. 
\verse Facto autem tempore, dimissi sunt cum pace a fratribus ad eos, qui miserant illos. 
(\verse) \verse Paulus autem et Barnabas demorabantur Antiochiae docentes et evangelizantes cum aliis pluribus verbum Domini.
 \verse Post aliquot autem dies dixit ad Barnabam Paulus: “ Revertentes visitemus fratres per universas civitates, in quibus praedicavimus verbum Domini, quomodo se habeant ”. 
\verse Barnabas autem volebat secum assumere et Ioannem, qui cognominatur Marcus; 
\verse Paulus autem iudicabat eum, qui discessisset ab eis a Pamphylia et non isset cum eis in opus, non debere recipi eum. 
\verse Facta est autem exacerbatio, ita ut discederent ab invicem, et Barnabas, assumpto Marco, navigaret Cyprum. 
\verse Paulus vero, electo Sila, profectus est, traditus gratiae Domini a fratribus; 
\verse perambulabat autem Syriam et Ciliciam confirmans ecclesias.
 
\begin{biblechapter}
\verse Pervenit autem in Derben et Lystram. Et ecce discipulus quidam erat ibi nomine Timotheus, filius mulieris Iudaeae fidelis, patre autem Graeco; 
\verse huic testimonium reddebant, qui in Lystris erant et Iconii fratres. 
\verse Hunc voluit Paulus secum proficisci et assumens circumcidit eum propter Iudaeos, qui erant in illis locis; sciebant enim omnes quod pater eius Graecus esset. 
\verse Cum autem pertransirent civitates, tradebant eis custodire dogmata, quae erant decreta ab apostolis et presbyteris, qui essent Hierosolymis. 
\verse Ecclesiae quidem confirmabantur fide et abundabant numero cotidie.
 \verse Transierunt autem Phrygiam et Galatiae regionem, vetati a Sancto Spiritu loqui verbum in Asia; 
\verse cum venissent autem circa Mysiam, tentabant ire Bithyniam, et non permisit eos Spiritus Iesu; 
\verse cum autem praeterissent Mysiam, descenderunt Troadem. 
\verse Et visio per noctem Paulo ostensa est: vir Macedo quidam erat stans et deprecans eum et dicens: “ Transiens in Macedoniam, adiuva nos! ”. 
\verse Ut autem visum vidit, statim quaesivimus proficisci in Macedoniam, certi facti quia vocasset nos Deus evangelizare eis.
 \verse Navigantes autem a Troade recto cursu venimus Samothraciam et sequenti die Neapolim 
\verse et inde Philippos, quae est prima partis Macedoniae civitas, colonia. Eramus autem in hac urbe diebus aliquot commorantes. 
\verse Die autem sabbatorum egressi sumus foras portam iuxta flumen, ubi putabamus orationem esse, et sedentes loquebamur mulieribus, quae convenerant. 
\verse Et quaedam mulier nomine Lydia, purpuraria civitatis Thyatirenorum colens Deum, audiebat, cuius Dominus aperuit cor intendere his, quae dicebantur a Paulo. 
\verse Cum autem baptizata esset et domus eius, deprecata est dicens: “ Si iudicastis me fidelem Domino esse, introite in domum meam et manete ”; et coegit nos.
 \verse Factum est autem, euntibus nobis ad orationem, puellam quandam habentem spiritum pythonem obviare nobis, quae quaestum magnum praestabat dominis suis divinando. 
\verse Haec subsecuta Paulum et nos clamabat dicens: “ Isti homines servi Dei Altissimi sunt, qui annuntiant vobis viam salutis ”. 
\verse Hoc autem faciebat multis diebus. Dolens autem Paulus et conversus spiritui dixit: “ Praecipio tibi in nomine Iesu Christi exire ab ea ”; et exiit eadem hora. 
\verse Videntes autem domini eius quia exivit spes quaestus eorum, apprehendentes Paulum et Silam traxerunt in forum ad principes; 
\verse et producentes eos magistratibus dixerunt: “ Hi homines conturbant civitatem nostram, cum sint Iudaei, 
\verse et annuntiant mores, quos non licet nobis suscipere neque facere, cum simus Romani ”. 
\verse Et concurrit plebs adversus eos; et magistratus, scissis tunicis eorum, iusserunt virgis caedi 
\verse et, cum multas plagas eis imposuissent, miserunt eos in carcerem, praecipientes custodi, ut caute custodiret eos; 
\verse qui cum tale praeceptum accepisset, misit eos in interiorem carcerem et pedes eorum strinxit in ligno.
 \verse Media autem nocte, Paulus et Silas orantes laudabant Deum, et audiebant eos, qui in custodia erant; 
\verse subito vero terraemotus factus est magnus, ita ut moverentur fundamenta carceris, et aperta sunt statim ostia omnia, et universorum vincula soluta sunt. 
\verse Expergefactus autem custos carceris et videns apertas ianuas carceris, evaginato gladio volebat se interficere, aestimans fugisse vinctos. 
\verse Clamavit autem Paulus magna voce dicens: “ Nihil feceris tibi mali; universi enim hic sumus ”. 
\verse Petitoque lumine, intro cucurrit et tremefactus procidit Paulo et Silae; 
\verse et producens eos foras ait: “ Domini, quid me oportet facere, ut salvus fiam? ”. 
\verse At illi dixerunt: “ Crede in Domino Iesu et salvus eris tu et domus tua ”. 
\verse Et locuti sunt ei verbum Domini cum omnibus, qui erant in domo eius. 
\verse Et tollens eos in illa hora noctis lavit eos a plagis, et baptizatus est ipse et omnes eius continuo; 
\verse cumque perduxisset eos in domum, apposuit mensam et laetatus est cum omni domo sua credens Deo.
 \verse Et cum dies factus esset, miserunt magistratus lictores dicentes: “ Dimitte homines illos! ”. 
\verse Nuntiavit autem custos carceris verba haec Paulo: “ Miserunt magistratus, ut dimittamini; nunc igitur exeuntes ite in pace ”. 
\verse Paulus autem dixit eis: “ Caesos nos publice, indemnatos, cum homines Romani essemus, miserunt in carcerem; et nunc occulte nos eiciunt? Non ita, sed veniant et ipsi nos educant ”. 
\verse Nuntiaverunt autem magistratibus lictores verba haec. Timueruntque audito quod Romani essent; 
\verse et venientes deprecati sunt eos et educentes rogabant, ut egrederentur urbem. 
\verse Exeuntes autem de carcere introierunt ad Lydiam et, visis fratribus, consolati sunt eos et profecti sunt.
 
\begin{biblechapter}
 \verseCum autem perambulassent Amphipolim et Apolloniam, venerunt Thessalonicam, ubi erat synagoga Iudaeorum. 
\verse Secundum consuetudinem autem suam Paulus introivit ad eos et per sabbata tria disserebat eis de Scripturis 
\verse adaperiens et comprobans quia Christum oportebat pati et resurgere a mortuis, et: “ Hic est Christus, Iesus, quem ego annuntio vobis ”. 
\verse Et quidam ex eis crediderunt et adiuncti sunt Paulo et Silae et de colentibus Graecis multitudo magna et mulieres nobiles non paucae. 
\verse Zelantes autem Iudaei assumentesque de foro viros quosdam malos et turba facta concitaverunt civitatem; et assistentes domui Iasonis quaerebant eos producere in populum. 
\verse Et cum non invenissent eos, trahebant Iasonem et quosdam fratres ad politarchas clamantes: “ Qui orbem concitaverunt, isti et huc venerunt, 
\verse quos suscepit Iason; et hi omnes contra decreta Caesaris faciunt, regem alium dicentes esse, Iesum ”. 
 \verse Concitaverunt autem plebem et politarchas audientes haec; 
\verse et accepto satis ab Iasone et a ceteris, dimiserunt eos.
 \verse Fratres vero confestim per noctem dimiserunt Paulum et Silam in Beroeam; qui cum advenissent, in synagogam Iudaeorum introierunt. 
\verse Hi autem erant nobiliores eorum, qui sunt Thessalonicae, qui susceperunt verbum cum omni aviditate, cotidie scrutantes Scripturas, si haec ita se haberent. 
\verse Et multi quidem crediderunt ex eis et Graecarum mulierum honestarum et virorum non pauci. 
\verse Cum autem cognovissent in Thessalonica Iudaei quia et Beroeae annuntiatum est a Paulo verbum Dei, venerunt et illuc commoventes et turbantes multitudinem. 
 \verse Statimque tunc Paulum dimiserunt fratres, ut iret usque ad mare; Silas autem et Timotheus remanserunt ibi. 
\verse Qui autem deducebant Paulum, perduxerunt usque Athenas; et accepto mandato ad Silam et Timotheum, ut quam celerrime venirent ad illum, profecti sunt.
 \verse Paulus autem cum Athenis eos exspectaret, irritabatur spiritus eius in ipso videns idololatriae deditam civitatem. 
\verse Disputabat igitur in synagoga cum Iudaeis et colentibus et in foro per omnes dies ad eos, qui aderant. 
\verse Quidam autem ex Epicureis et Stoicis philosophi disserebant cum eo. Et quidam dicebant: “ Quid vult seminiverbius hic dicere? ”; alii vero: “ Novorum daemoniorum videtur annuntiator esse ”, quia Iesum et resurrectionem evangelizabat. 
\verse Et apprehensum eum ad Areopagum duxerunt dicentes: “ Possumus scire quae est haec nova, quae a te dicitur, doctrina? 
\verse Mira enim quaedam infers auribus nostris; volumus ergo scire quidnam velint haec esse ”. 
 \verse Athenienses autem omnes et advenae hospites ad nihil aliud vacabant nisi aut dicere aut audire aliquid novi.
 \verse Stans autem Paulus in medio Areopagi ait: “ Viri Athenienses, per omnia quasi superstitiosiores vos video; 
\verse praeteriens enim et videns simulacra vestra inveni et aram, in qua scriptum erat: “Ignoto deo”. Quod ergo ignorantes colitis, hoc ego annuntio vobis. 
\verse Deus, qui fecit mundum et omnia, quae in eo sunt, hic, caeli et terrae cum sit Dominus, non in manufactis templis inhabitat 
\verse nec manibus humanis colitur indigens aliquo, cum ipse det omnibus vitam et inspirationem et omnia; 
\verse fecitque ex uno omne genus hominum inhabitare super universam faciem terrae, definiens statuta tempora et terminos habitationis eorum, 
\verse quaerere Deum, si forte attrectent eum et inveniant, quamvis non longe sit ab unoquoque nostrum. 
\verse In ipso enim vivimus et movemur et sumus, sicut et quidam vestrum poetarum dixerunt:
 “Ipsius enim et genus sumus”.
 \verse Genus ergo cum simus Dei, non debemus aestimare auro aut argento aut lapidi, sculpturae artis et cogitationis hominis, divinum esse simile. 
\verse Et tempora quidem ignorantiae despiciens Deus, nunc annuntiat hominibus, ut omnes ubique paenitentiam agant, 
\verse eo quod statuit diem, in qua iudicaturus est orbem in iustitia in viro, quem constituit, fidem praebens omnibus suscitans eum a mortuis ”.
 \verse Cum audissent autem resurrectionem mortuorum, quidam quidem irridebant, quidam vero dixerunt: “ Audiemus te de hoc iterum ”. 
\verse Sic Paulus exivit de medio eorum. 
\verse Quidam vero viri adhaerentes ei crediderunt; in quibus et Dionysius Areopagita et mulier nomine Damaris et alii cum eis.
 
\begin{biblechapter}
\verse Post haec discedens ab Athe nis venit Corinthum. 
\verse Et in veniens quendam Iudaeum nomine Aquilam, Ponticum genere, qui nuper venerat ab Italia, et Priscillam uxorem eius, eo quod praecepisset Claudius discedere omnes Iudaeos a Roma, accessit ad eos 
\verse et, quia eiusdem erat artis, manebat apud eos et operabatur; erant autem scenofactoriae artis. 
\verse Disputabat autem in synagoga per omne sabbatum suadebatque Iudaeis et Graecis.
 \verse Cum venissent autem de Macedonia Silas et Timotheus, instabat verbo Paulus testificans Iudaeis esse Christum Iesum. 
\verse Contradicentibus autem eis et blasphemantibus, excutiens vestimenta dixit ad eos: “ Sanguis vester super caput vestrum! Mundus ego. Ex hoc nunc ad gentes vadam ”. 
\verse Et migrans inde intravit in domum cuiusdam nomine Titi Iusti, colentis Deum, cuius domus erat coniuncta synagogae. 
\verse Crispus autem archisynagogus credidit Domino cum omni domo sua, et multi Corinthiorum audientes credebant et baptizabantur.
 \verse Dixit autem Dominus nocte per visionem Paulo: “ Noli timere, sed loquere et ne taceas, 
\verse quia ego sum tecum, et nemo apponetur tibi, ut noceat te, quoniam populus est mihi multus in hac civitate ”.
 \verse Sedit autem annum et sex menses docens apud eos verbum Dei.
 \verse Gallione autem proconsule Achaiae, insurrexerunt uno animo Iudaei in Paulum et adduxerunt eum ad tribunal 
\verse dicentes: “ Contra legem hic persuadet hominibus colere Deum ”. 
\verse Incipiente autem Paulo aperire os, dixit Gallio ad Iudaeos: “ Si quidem esset iniquum aliquid aut facinus pessimum, o Iudaei, merito vos sustinerem; 
\verse si vero quaestiones sunt de verbo et nominibus et lege vestra, vos ipsi videritis; iudex ego horum nolo esse ”. 
\verse Et minavit eos a tribunali. 
\verse Apprehendentes autem omnes Sosthenen, principem synagogae, percutiebant ante tribunal; et nihil horum Gallioni curae erat.
 \verse Paulus vero, cum adhuc sustinuisset dies multos, fratribus valefaciens navigabat Syriam, et cum eo Priscilla et Aquila, qui sibi totonderat in Cenchreis caput; habebat enim votum. 
\verse Deveneruntque Ephesum, et illos ibi reliquit; ipse vero ingressus synagogam disputabat cum Iudaeis. 
\verse Rogantibus autem eis, ut ampliore tempore maneret, non consensit, 
\verse sed valefaciens et dicens: “ Iterum revertar ad vos Deo volente ”, navigavit ab Epheso; 
\verse et descendens Caesaream ascendit et salutavit ecclesiam et descendit Antiochiam.
 \verse Et facto ibi aliquanto tempore, profectus est perambulans ex ordine Galaticam regionem et Phrygiam, confirmans omnes discipulos.
 \verse Iudaeus autem quidam Apollo nomine, Alexandrinus natione, vir eloquens, devenit Ephesum, potens in Scripturis. 
\verse Hic erat catechizatus viam Domini; et fervens spiritu loquebatur et docebat diligenter ea, quae sunt de Iesu, sciens tantum baptisma Ioannis. 
\verse Hic ergo coepit fiducialiter agere in synagoga; quem cum audissent Priscilla et Aquila, assumpserunt eum et diligentius exposuerunt ei viam Dei. 
\verse Cum autem vellet transire in Achaiam, exhortati fratres scripserunt discipulis, ut susciperent eum; qui cum venisset, contulit multum his, qui crediderant per gratiam; 
\verse vehementer enim Iudaeos revincebat publice, ostendens per Scripturas esse Christum Iesum.
 
\begin{biblechapter}
\verse Factum est autem, cum Apollo esset Corinthi, ut Paulus, peragratis superioribus partibus, veniret Ephesum et inveniret quosdam discipulos; 
\verse dixitque ad eos: “ Si Spiritum Sanctum accepistis credentes? ”. At illi ad eum: “ Sed neque, si Spiritus Sanctus est, audivimus ”. 
\verse Ille vero ait: “ In quo ergo baptizati estis? ”. Qui dixerunt: “ In Ioannis baptismate ”. 
\verse Dixit autem Paulus: “ Ioannes baptizavit baptisma paenitentiae, populo dicens in eum, qui venturus esset post ipsum ut crederent, hoc est in Iesum ”. 
\verse His auditis, baptizati sunt in nomine Domini Iesu; 
\verse et cum imposuisset illis manus Paulus, venit Spiritus Sanctus super eos, et loquebantur linguis et prophetabant. 
\verse Erant autem omnes viri fere duodecim.
 \verse Introgressus autem synagogam cum fiducia loquebatur per tres menses disputans et suadens de regno Dei. 
\verse Cum autem quidam indurarentur et non crederent maledicentes viam coram multitudine, discedens ab eis segregavit discipulos, cotidie disputans in schola Tyranni. 
\verse Hoc autem factum est per biennium, ita ut omnes, qui habitabant in Asia, audirent verbum Domini, Iudaei atque Graeci.
 \verse Virtutesque non quaslibet Deus faciebat per manus Pauli, 
\verse ita ut etiam super languidos deferrentur a corpore eius sudaria vel semicinctia, et recederent ab eis languores, et spiritus nequam egrederentur.
 \verse Tentaverunt autem quidam et de circumeuntibus Iudaeis exorcistis invocare super eos, qui habebant spiritus malos, nomen Domini Iesu dicentes: “ Adiuro vos per Iesum, quem Paulus praedicat ”. 
\verse Erant autem cuiusdam Scevae Iudaei principis sacerdotum septem filii, qui hoc faciebant. 
\verse Respondens autem spiritus nequam dixit eis: “ Iesum novi et Paulum scio; vos autem qui estis? ”. 
 \verse Et insiliens homo in eos, in quo erat spiritus malus, dominatus amborum invaluit contra eos, ita ut nudi et vulnerati effugerent de domo illa. 
\verse Hoc autem notum factum est omnibus Iudaeis atque Graecis, qui habitabant Ephesi, et cecidit timor super omnes illos, et magnificabatur nomen Domini Iesu. 
\verse Multique credentium veniebant confitentes et annuntiantes actus suos. 
\verse Multi autem ex his, qui fuerant curiosa sectati, conferentes libros combusserunt coram omnibus; et computaverunt pretia illorum et invenerunt argenti quinquaginta milia. 
\verse Ita fortiter verbum Domini crescebat et convalescebat.
 \verse His autem expletis, proposuit Paulus in Spiritu, transita Macedonia et Achaia, ire Hierosolymam, dicens: “ Postquam fuero ibi, oportet me et Romam videre ”. 
\verse Mittens autem in Macedoniam duos ex ministrantibus sibi, Timotheum et Erastum, ipse remansit ad tempus in Asia.
 \verse Facta est autem in illo tempore turbatio non minima de via. 
\verse Demetrius enim quidam nomine, argentarius, faciens aedes argenteas Dianae praestabat artificibus non modicum quaestum; 
\verse quos congregans et eos, qui eiusmodi erant opifices, dixit: “ Viri, scitis quia de hoc artificio acquisitio est nobis; 
\verse et videtis et auditis quia non solum Ephesi, sed paene totius Asiae Paulus hic suadens avertit multam turbam dicens quoniam non sunt dii, qui manibus fiunt. 
\verse Non solum autem haec periclitatur nobis pars in redargutionem venire, sed et magnae deae Dianae templum in nihilum reputari, et destrui incipiet maiestas eius, quam tota Asia et orbis colit ”.
 \verse His auditis, repleti sunt ira et clamabant dicentes: “ Magna Diana Ephesiorum! ”; 
\verse et impleta est civitas confusione, et impetum fecerunt uno animo in theatrum, rapto Gaio et Aristarcho Macedonibus, comitibus Pauli. 
\verse Paulo autem volente intrare in populum, non permiserunt discipuli; 
\verse quidam autem de Asiarchis, qui erant amici eius, miserunt ad eum rogantes, ne se daret in theatrum. 
\verse Alii autem aliud clamabant; erat enim ecclesia confusa, et plures nesciebant qua ex causa convenissent.
 \verse De turba autem instruxerunt Alexandrum, propellentibus eum Iudaeis; Alexander ergo, manu silentio postulato, volebat rationem reddere populo. 
\verse Quem ut cognoverunt Iudaeum esse, vox facta est una omnium quasi per horas duas clamantium: “ Magna Diana Ephesiorum ”. 
\verse Et cum sedasset scriba turbam, dixit: “ Viri Ephesii, quis enim est hominum, qui nesciat Ephesiorum civitatem cultricem esse magnae Dianae et simulacri a Iove delapsi? 
\verse Cum ergo his contradici non possit, oportet vos sedatos esse et nihil temere agere. 
\verse Adduxistis enim homines istos neque sacrilegos neque blasphemantes deam nostram. 
\verse Quod si Demetrius et, qui cum eo sunt, artifices habent adversus aliquem causam, conventus forenses aguntur, et proconsules sunt: accusent invicem. 
 \verse Si quid autem ulterius quaeritis, in legitima ecclesia poterit absolvi. 
 \verse Nam et periclitamur argui seditionis hodiernae, cum nullus obnoxius sit, de quo non possimus reddere rationem concursus istius ”. Et cum haec dixisset, dimisit ecclesiam.
 
\begin{biblechapter}
\verse Postquam autem cessavit tumultus, accersitis Paulus discipulis et exhortatus eos, valedixit et profectus est, ut iret in Macedoniam. 
\verse Cum autem perambulasset partes illas et exhortatus eos fuisset multo sermone, venit ad Graeciam; 
\verse cumque fecisset menses tres, factae sunt illi insidiae a Iudaeis navigaturo in Syriam, habuitque consilium, ut reverteretur per Macedoniam. 
\verse Comitabatur autem eum Sopater Pyrrhi Beroeensis, Thessalonicensium vero Aristarchus et Secundus et Gaius Derbeus et Timotheus, Asiani vero Tychicus et Trophimus. 
\verse Hi cum praecessissent, sustinebant nos Troade; 
\verse nos vero navigavimus post dies Azymorum a Philippis et venimus ad eos Troadem in diebus quinque, ubi demorati sumus diebus septem.
 \verse In una autem sabbatorum, cum convenissemus ad frangendum panem, Paulus disputabat eis, profecturus in crastinum, protraxitque sermonem usque in mediam noctem. 
\verse Erant autem lampades copiosae in cenaculo, ubi eramus congregati; 
 \verse sedens autem quidam adulescens nomine Eutychus super fenestram, cum mergeretur somno gravi, disputante diutius Paulo, eductus somno cecidit de tertio cenaculo deorsum et sublatus est mortuus. 
\verse Cum descendisset autem Paulus, incubuit super eum et complexus dixit: “ Nolite turbari, anima enim ipsius in eo est! ”. 
 \verse Ascendens autem frangensque panem et gustans satisque allocutus usque in lucem, sic profectus est. 
\verse Adduxerunt autem puerum viventem et consolati sunt non minime.
 \verse Nos autem praecedentes navi enavigavimus in Asson, inde suscepturi Paulum; sic enim disposuerat volens ipse per terram iter facere. 
\verse Cum autem convenisset nos in Asson, assumpto eo, venimus Mitylenen 
\verse et inde navigantes sequenti die pervenimus contra Chium et alia applicuimus Samum et sequenti venimus Miletum. 
\verse Proposuerat enim Paulus transnavigare Ephesum, ne qua mora illi fieret in Asia; festinabat enim, si possibile sibi esset, ut diem Pentecosten faceret Hierosolymis.
 \verse A Mileto autem mittens Ephesum convocavit presbyteros ecclesiae. 
\verse Qui cum venissent ad eum, dixit eis: “ Vos scitis a prima die, qua ingressus sum in Asiam, qualiter vobiscum per omne tempus fuerim, 
\verse serviens Domino cum omni humilitate et lacrimis et tentationibus, quae mihi acciderunt in insidiis Iudaeorum; 
\verse quomodo nihil subtraxerim utilium, quominus annuntiarem vobis et docerem vos publice et per domos, 
\verse testificans Iudaeis atque Graecis in Deum paenitentiam et fidem in Dominum nostrum Iesum. 
\verse Et nunc ecce alligatus ego Spiritu vado in Ierusalem, quae in ea eventura sint mihi ignorans, 
 \verse nisi quod Spiritus Sanctus per omnes civitates protestatur mihi dicens quoniam vincula et tribulationes me manent. 
\verse Sed nihili facio animam meam pretiosam mihi, dummodo consummem cursum meum et ministerium, quod accepi a Domino Iesu, testificari evangelium gratiae Dei.
 \verse Et nunc ecce ego scio quia amplius non videbitis faciem meam vos omnes, per quos transivi praedicans regnum; 
\verse quapropter contestor vos hodierna die, quia mundus sum a sanguine omnium; 
\verse non enim subterfugi, quominus annuntiarem omne consilium Dei vobis. 
\verse Attendite vobis et universo gregi, in quo vos Spiritus Sanctus posuit episcopos, pascere ecclesiam Dei, quam acquisivit sanguine suo. 
\verse Ego scio quoniam intrabunt post discessionem meam lupi graves in vos non parcentes gregi; 
\verse et ex vobis ipsis exsurgent viri loquentes perversa, ut abstrahant discipulos post se. 
\verse Propter quod vigilate, memoria retinentes quoniam per triennium nocte et die non cessavi cum lacrimis monens unumquemque vestrum.
 \verse Et nunc commendo vos Deo et verbo gratiae ipsius, qui potens est aedificare et dare hereditatem in sanctificatis omnibus. 
\verse Argentum aut aurum aut vestem nullius concupivi; 
\verse ipsi scitis quoniam ad ea, quae mihi opus erant et his, qui mecum sunt, ministraverunt manus istae. 
\verse Omnia ostendi vobis quoniam sic laborantes oportet suscipere infirmos, ac meminisse verborum Domini Iesu, quoniam ipse dixit: “Beatius est magis dare quam accipere!” ”.
 \verse Et cum haec dixisset, positis genibus suis, cum omnibus illis oravit. 
\verse Magnus autem fletus factus est omnium; et procumbentes super collum Pauli osculabantur eum 
\verse dolentes maxime in verbo, quod dixerat, quoniam amplius faciem eius non essent visuri. Et deducebant eum ad navem.
 
\begin{biblechapter}
\verse Cum autem factum esset, ut navigaremus abstracti ab eis, recto cursu venimus Cho et sequenti die Rhodum et inde Patara; 
\verse et cum invenissemus navem transfretantem in Phoenicen, ascendentes navigavimus. 
\verse Cum paruissemus autem Cypro, et relinquentes eam ad sinistram navigabamus in Syriam et venimus Tyrum, ibi enim navis erat expositura onus. 
\verse Inventis autem discipulis, mansimus ibi diebus septem; qui Paulo dicebant per Spiritum, ne iret Hierosolymam. 
\verse Et explicitis diebus, profecti ibamus, deducentibus nos omnibus cum uxoribus et filiis usque foras civitatem; et positis genibus in litore orantes, 
\verse valefecimus invicem et ascendimus in navem; illi autem redierunt in sua. 
\verse Nos vero, navigatione explicita, a Tyro devenimus Ptolemaida et, salutatis fratribus, mansimus die una apud illos.
 \verse Alia autem die profecti venimus Caesaream et intrantes in domum Philippi evangelistae, qui erat de septem, mansimus apud eum. 
\verse Huic autem erant filiae quattuor virgines prophetantes. 
\verse Et cum moraremur plures dies, supervenit quidam a Iudaea propheta nomine Agabus; 
\verse is cum venisset ad nos et tulisset zonam Pauli, alligans sibi pedes et manus dixit: “ Haec dicit Spiritus Sanctus: Virum, cuius est zona haec, sic alligabunt in Ierusalem Iudaei et tradent in manus gentium ”. 
\verse Quod cum audissemus, rogabamus nos et, qui loci illius erant, ne ipse ascenderet Ierusalem. 
\verse Tunc respondit Paulus: “ Quid facitis flentes et affligentes cor meum? Ego enim non solum alligari sed et mori in Ierusalem paratus sum propter nomen Domini Iesu ”. 
\verse Et cum ei suadere non possemus, quievimus dicentes: “ Domini voluntas fiat! ”.
 \verse Post dies autem istos praeparati ascendebamus Hierosolymam; 
\verse venerunt autem et ex discipulis a Caesarea nobiscum adducentes apud quem hospitaremur, Mnasonem quendam Cyprium, antiquum discipulum.
 \verse Et cum venissemus Hierosolymam, libenter exceperunt nos fratres. 
\verse Sequenti autem die introibat Paulus nobiscum ad Iacobum, omnesque collecti sunt presbyteri. 
\verse Quos cum salutasset, narrabat per singula, quae fecisset Deus in gentibus per ministerium ipsius. 
\verse At illi cum audissent, glorificabant Deum dixeruntque ei: “ Vides, frater, quot milia sint in Iudaeis, qui crediderunt, et omnes aemulatores sunt legis; 
\verse audierunt autem de te quia discessionem doceas a Moyse omnes, qui per gentes sunt, Iudaeos, dicens non debere circumcidere eos filios suos neque secundum consuetudines ambulare. 
 \verse Quid ergo est? Utique audient te supervenisse. 
\verse Hoc ergo fac, quod tibi dicimus. Sunt nobis viri quattuor votum habentes super se; 
\verse his assumptis, sanctifica te cum illis et impende pro illis, ut radant capita, et scient omnes quia, quae de te audierunt, nihil sunt, sed ambulas et ipse custodiens legem. 
 \verse De his autem, qui crediderunt, gentibus nos scripsimus iudicantes, ut abstineant ab idolothyto et sanguine et suffocato et fornicatione ”.
 \verse Tunc Paulus, assumptis viris, postera die purificatus cum illis intravit in templum annuntians expletionem dierum purificationis, donec offerretur pro unoquoque eorum oblatio.
 \verse Dum autem septem dies consummarentur, hi, qui de Asia erant, Iudaei cum vidissent eum in templo, concitaverunt omnem turbam et iniecerunt ei manus 
 \verse clamantes: “ Viri Israelitae, adiuvate! Hic est homo, qui adversus populum et legem et locum hunc omnes ubique docens, insuper et Graecos induxit in templum et polluit sanctum locum istum ”. 
\verse Viderant enim Trophimum Ephesium in civitate cum ipso, quem aestimabant quoniam in templum induxisset Paulus. 
\verse Commotaque est civitas tota, et facta est concursio populi, et apprehendentes Paulum trahebant eum extra templum, et statim clausae sunt ianuae. 
\verse Quaerentibus autem eum occidere, nuntiatum est tribuno cohortis quia tota confunditur Ierusalem, 
\verse qui statim, assumptis militibus et centurionibus, decucurrit ad illos; qui cum vidissent tribunum et milites, cessaverunt percutere Paulum. 
\verse Tunc accedens tribunus apprehendit eum et iussit alligari catenis duabus et interrogabat quis esset et quid fecisset. 
\verse Alii autem aliud clamabant in turba; et cum non posset certum cognoscere prae tumultu, iussit duci eum in castra. 
\verse Et cum venisset ad gradus, contigit ut portaretur a militibus propter vim turbae; 
\verse sequebatur enim multitudo populi clamantes: “ Tolle eum! ”.
 \verse Et cum coepisset induci in castra, Paulus dicit tribuno: “ Si licet mihi loqui aliquid ad te? ”. Qui dixit: “ Graece nosti? 
\verse Nonne tu es Aegyptius, qui ante hos dies tumultum concitasti et eduxisti in desertum quattuor milia virorum sicariorum? ”. 
\verse Et dixit Paulus: “ Ego homo sum quidem Iudaeus a Tarso Ciliciae, non ignotae civitatis municeps; rogo autem te, permitte mihi loqui ad populum ”. 
\verse Et cum ille permisisset, Paulus stans in gradibus annuit manu ad plebem et, magno silentio facto, allocutus est Hebraea lingua dicens:
 
\begin{biblechapter}
\verse “ Viri fratres et patres, audi te a me, quam ad vos nunc reddo, rationem ”. \verse Cum audissent autem quia Hebraea lingua loquebatur ad illos, magis praestiterunt silentium. Et dixit: 
\verse “ Ego sum vir Iudaeus, natus Tarso Ciliciae, enutritus autem in ista civitate, secus pedes Gamaliel eruditus iuxta veritatem paternae legis, aemulator Dei, sicut et vos omnes estis hodie. 
\verse Qui hanc viam persecutus sum usque ad mortem, alligans et tradens in custodias viros ac mulieres, 
\verse sicut et princeps sacerdotum testimonium mihi reddit et omne concilium; a quibus et epistulas accipiens ad fratres, Damascum pergebam, ut adducerem et eos, qui ibi essent, vinctos in Ierusalem, uti punirentur.
 \verse Factum est autem, eunte me et appropinquante Damasco, circa mediam diem subito de caelo circumfulsit me lux copiosa; 
\verse et decidi in terram et audivi vocem dicentem mihi: “Saul, Saul, quid me persequeris?”. 
\verse Ego autem respondi: “Quis es, Domine?”. Dixitque ad me: “Ego sum Iesus Nazarenus, quem tu persequeris”. 
\verse Et, qui mecum erant, lumen quidem viderunt, vocem autem non audierunt eius, qui loquebatur mecum. 
\verse Et dixi: “Quid faciam, Domine?”. Dominus autem dixit ad me: “Surgens vade Damascum, et ibi tibi dicetur de omnibus, quae statutum est tibi, ut faceres”. 
\verse Et cum non viderem prae claritate luminis illius, ad manum deductus a comitibus veni Damascum.
 \verse Ananias autem quidam vir religiosus secundum legem, testimonium habens ab omnibus habitantibus Iudaeis, 
\verse veniens ad me et astans dixit mihi: “Saul frater, respice!”. Et ego eadem hora respexi in eum. 
\verse At ille dixit: “Deus patrum nostrorum praeordinavit te, ut cognosceres voluntatem eius et videres Iustum et audires vocem ex ore eius, 
\verse quia eris testis illi ad omnes homines eorum, quae vidisti et audisti. 
\verse Et nunc quid moraris? Exsurgens baptizare et ablue peccata tua, invocato nomine ipsius”.
 \verse Factum est autem, revertenti mihi in Ierusalem et oranti in templo fieri me in stupore mentis 
\verse et videre illum dicentem mihi: “Festina et exi velociter ex Ierusalem, quoniam non recipient testimonium tuum de me”. 
\verse Et ego dixi: “Domine, ipsi sciunt quia ego eram concludens in carcerem et caedens per synagogas eos, qui credebant in te; 
\verse et cum funderetur sanguis Stephani testis tui, et ipse astabam et consentiebam et custodiebam vestimenta interficientium illum”. 
\verse Et dixit ad me: “Vade, quoniam ego in nationes longe mittam te” ”.
 \verse Audiebant autem eum usque ad hoc verbum et levaverunt vocem suam dicentes: “ Tolle de terra eiusmodi, non enim fas est eum vivere! ”. 
\verse Vociferantibus autem eis et proicientibus vestimenta sua et pulverem iactantibus in aerem, 
 \verse iussit tribunus induci eum in castra dicens flagellis eum interrogari, ut sciret propter quam causam sic acclamarent ei.
 \verse Et cum astrinxissent eum loris, dixit astanti centurioni Paulus: “ Si hominem Romanum et indemnatum licet vobis flagellare? ”. 
\verse Quo audito, centurio accedens ad tribunum nuntiavit dicens: “ Quid acturus es? Hic enim homo Romanus est ”. 
\verse Accedens autem tribunus dixit illi: “ Dic mihi, tu Romanus es? ”. At ille dixit: “ Etiam ”. 
\verse Et respondit tribunus: “ Ego multa summa civitatem hanc consecutus sum ”. Et Paulus ait: “ Ego autem et natus sum ”. 
 \verse Protinus ergo discesserunt ab illo, qui eum interrogaturi erant; tribunus quoque timuit, postquam rescivit quia Romanus esset, et quia alligasset eum.
 \verse Postera autem die, volens scire diligenter qua ex causa accusaretur a Iudaeis, solvit eum et iussit principes sacerdotum convenire et omne concilium et producens Paulum statuit coram illis.
 
\begin{biblechapter}
\verse Intendens autem concilium Paulus ait: “ Viri fratres, ego omni conscientia bona conversatus sum ante Deum usque in hodiernum diem ”. 
\verse Princeps autem sacerdotum Ananias praecepit astantibus sibi percutere os eius. 
 \verse Tunc Paulus ad eum dixit: “ Percutiet te Deus, paries dealbate! Et tu sedes iudicans me secundum legem et contra legem iubes me percuti? ”. 
\verse Et, qui astabant, dixerunt: “ Summum sacerdotem Dei maledicis?”. 
\verse Dixit autem Paulus: “ Nesciebam, fratres, quia princeps est sacerdotum; scriptum est enim: “Principem populi tui non maledices” ”.
 \verse Sciens autem Paulus quia una pars esset sadducaeorum, et altera pharisaeorum, exclamabat in concilio: “ Viri fratres, ego pharisaeus sum, filius pharisaeorum; de spe et resurrectione mortuorum ego iudicor ”. 
\verse Et cum haec diceret, facta est dissensio inter pharisaeos et sadducaeos; et divisa est multitudo. 
 \verse Sadducaei enim dicunt non esse resurrectionem neque angelum neque spiritum; pharisaei autem utrumque confitentur. 
\verse Factus est autem clamor magnus; et surgentes scribae quidam partis pharisaeorum pugnabant dicentes: “ Nihil mali invenimus in homine isto: quod si spiritus locutus est ei aut angelus ”; 
\verse et cum magna dissensio facta esset, timens tribunus ne discerperetur Paulus ab ipsis, iussit milites descendere, ut raperent eum de medio eorum ac deducerent in castra. 
\verse Sequenti autem nocte, assistens ei Dominus ait: “ Constans esto! Sicut enim testificatus es, quae sunt de me, in Ierusalem, sic te oportet et Romae testificari ”.
 \verse Facta autem die, faciebant concursum Iudaei et devoverunt se dicentes neque manducaturos neque bibituros, donec occiderent Paulum. 
\verse Erant autem plus quam quadraginta, qui hanc coniurationem fecerant; 
\verse qui accedentes ad principes sacerdotum et seniores dixerunt: “ Devotione devovimus nos nihil gustaturos, donec occidamus Paulum. 
\verse Nunc ergo vos notum facite tribuno cum concilio, ut producat illum ad vos, tamquam aliquid certius cognituri de eo; nos vero, priusquam appropiet, parati sumus interficere illum ”.
 \verse Quod cum audisset filius sororis Pauli insidias, venit et intravit in castra nuntiavitque Paulo. 
\verse Vocans autem Paulus ad se unum ex centurionibus ait: “ Adulescentem hunc perduc ad tribunum, habet enim aliquid indicare illi ”. 
 \verse Et ille quidem assumens eum duxit ad tribunum et ait: “ Vinctus Paulus vocans rogavit me hunc adulescentem perducere ad te, habentem aliquid loqui tibi ”. 
\verse Apprehendens autem tribunus manum illius, secessit cum eo seorsum et interrogabat: “ Quid est, quod habes indicare mihi? ”. 
\verse Ille autem dixit: “ Iudaei constituerunt rogare te, ut crastina die Paulum producas in concilium, quasi aliquid certius inquisiturum sit de illo. 
\verse Tu ergo ne credideris illis; insidiantur enim ei ex eis viri amplius quadraginta, qui se devoverunt non manducare neque bibere, donec interficiant eum; et nunc parati sunt exspectantes promissum tuum ”.
 \verse Tribunus igitur dimisit adulescentem praecipiens, ne cui eloqueretur quoniam “ haec nota mihi fecisti ”. 
\verse Et vocatis duobus centurionibus, dixit: “ Parate milites ducentos, ut eant usque Caesaream, et equites septuaginta et lancearios ducentos, a tertia hora noctis, 
\verse et iumenta praeparate ”, ut imponentes Paulum salvum perducerent ad Felicem praesidem, 
\verse scribens epistulam habentem formam hanc: 
\verse “ Claudius Lysias optimo praesidi Felici salutem. 
\verse Virum hunc comprehensum a Iudaeis et incipientem interfici ab eis, superveniens cum exercitu eripui, cognito quia Romanus est. 
\verse Volensque scire causam, propter quam accusabant illum, deduxi in concilium eorum; 
\verse quem inveni accusari de quaestionibus legis ipsorum, nihil vero dignum morte aut vinculis habentem crimen. 
\verse Et cum mihi perlatum esset de insidiis, quae in virum pararentur, confestim misi ad te denuntians et accusatoribus, ut dicant adversum eum apud te ”.
 \verse Milites ergo, secundum praeceptum sibi assumentes Paulum, duxerunt per noctem in Antipatridem; 
\verse et postera die, dimissis equitibus, ut abirent cum eo, reversi sunt ad castra. 
\verse Qui cum venissent Caesaream et tradidissent epistulam praesidi, statuerunt ante illum et Paulum. 
\verse Cum legisset autem et interrogasset de qua provincia esset, et cognoscens quia de Cilicia: 
\verse “ Audiam te, inquit, cum et accusatores tui venerint ”; iussitque in praetorio Herodis custodiri eum.
 
\begin{biblechapter}
\verse Post quinque autem dies, descendit princeps sacerdo tum Ananias cum senioribus quibusdam et Tertullo quodam oratore, qui adierunt praesidem adversus Paulum. 
\verse Et citato eo, coepit accusare Tertullus dicens: “ Cum in multa pace agamus per te, et multa corrigantur genti huic per tuam providentiam, 
\verse semper et ubique suscipimus, optime Felix, cum omni gratiarum actione. 
\verse Ne diutius autem te protraham, oro, breviter audias nos pro tua clementia. 
\verse Invenimus enim hunc hominem pestiferum et concitantem seditiones omnibus Iudaeis, qui sunt in universo orbe, et auctorem seditionis sectae Nazarenorum, 
 \verse qui etiam templum violare conatus est, quem et apprehendimus, 
(\verse) \verse a quo poteris ipse diiudicans de omnibus istis cognoscere, de quibus nos accusamus eum ”. 
\verse Adiecerunt autem et Iudaei dicentes haec ita se habere.
 \verse Respondit autem Paulus, annuente sibi praeside dicere: “ Ex multis annis esse te iudicem genti huic sciens bono animo de causa mea rationem reddam, 
\verse cum possis cognoscere quia non plus sunt dies mihi quam duodecim, ex quo ascendi adorare in Ierusalem, 
\verse et neque in templo invenerunt me cum aliquo disputantem aut concursum facientem turbae neque in synagogis neque in civitate, 
 \verse neque probare possunt tibi, de quibus nunc accusant me. 
\verse Confiteor autem hoc tibi, quod secundum viam, quam dicunt haeresim, sic deservio patrio Deo credens omnibus, quae secundum Legem sunt et in Prophetis scripta, 
\verse spem habens in Deum, quam et hi ipsi exspectant, resurrectionem futuram iustorum et iniquorum. 
\verse In hoc et ipse studeo sine offendiculo conscientiam habere ad Deum et ad homines semper. 
\verse Post annos autem plures, eleemosynas facturus in gentem meam veni et oblationes; 
\verse in quibus invenerunt me purificatum in templo, non cum turba neque cum tumultu; 
\verse quidam autem ex Asia Iudaei, quos oportebat apud te praesto esse et accusare, si quid haberent adversum me; 
\verse aut hi ipsi dicant quid invenerint iniquitatis, cum starem in concilio, 
\verse nisi de una hac voce, qua clamavi inter eos stans: De resurrectione mortuorum ego iudicor hodie apud vos! ”.
 \verse Distulit autem illos Felix certissime sciens ea, quae de hac via sunt, dicens: “ Cum tribunus Lysias descenderit, cognoscam causam vestram ”, 
\verse iubens centurioni custodiri eum et habere mitigationem, nec quemquam prohibere de suis ministrare ei.
 \verse Post aliquot autem dies, adveniens Felix cum Drusilla uxore sua, quae erat Iudaea, vocavit Paulum et audivit ab eo de fide, quae est in Christum Iesum. 
 \verse Disputante autem illo de iustitia et continentia et de iudicio futuro, timefactus Felix respondit: “ Quod nunc attinet, vade; tempore autem opportuno accersiam te ”, 
\verse simul et sperans quia pecunia daretur sibi a Paulo; propter quod et frequenter accersiens eum loquebatur cum eo.
 \verse Biennio autem expleto, accepit successorem Felix Porcium Festum; volensque gratiam praestare Iudaeis, Felix reliquit Paulum vinctum.
 
\begin{biblechapter}
\verse Festus ergo cum venisset in provinciam, post triduum ascendit Hierosolymam a Caesarea; 
\verse adieruntque eum principes sacerdotum et primi Iudaeorum adversus Paulum, et rogabant eum 
\verse postulantes gratiam adversum eum, ut iuberet perduci eum in Ierusalem, insidias tendentes, ut eum interficerent in via. 
\verse Festus igitur respondit servari Paulum in Caesarea, se autem maturius profecturum: 
\verse “ Qui ergo in vobis, ait, potentes sunt, descendentes simul, si quod est in viro crimen, accusent eum ”. 
\verse Demoratus autem inter eos dies non amplius quam octo aut decem, descendit Caesaream; et altera die sedit pro tribunali et iussit Paulum adduci. 
\verse Qui cum perductus esset, circumsteterunt eum, qui ab Hierosolyma descenderant, Iudaei, multas et graves causas obicientes, quas non poterant probare, 
\verse Paulo rationem reddente: “ Neque in legem Iudaeorum neque in templum neque in Caesarem quidquam peccavi ”. 
\verse Festus autem volens Iudaeis gratiam praestare, respondens Paulo dixit: “ Vis Hierosolymam ascendere et ibi de his iudicari apud me? ”. 
\verse Dixit autem Paulus: “ Ad tribunal Caesaris sto, ubi me oportet iudicari. Iudaeis nihil nocui, sicut et tu melius nosti. 
\verse Si ergo iniuste egi et dignum morte aliquid feci, non recuso mori; si vero nihil est eorum, quae hi accusant me, nemo potest me illis donare. Caesarem appello! ”. 
\verse Tunc Festus cum consilio locutus respondit: “ Caesarem appellasti; ad Caesarem ibis ”.
 \verse Et cum dies aliquot transacti essent, Agrippa rex et Berenice descenderunt Caesaream et salutaverunt Festum. 
\verse Et cum dies plures ibi demorarentur, Festus regi indicavit de Paulo dicens: “ Vir quidam est derelictus a Felice vinctus, 
\verse de quo, cum essem Hierosolymis, adierunt me principes sacerdotum et seniores Iudaeorum postulantes adversus illum damnationem; 
\verse ad quos respondi, quia non est consuetudo Romanis donare aliquem hominem, priusquam is, qui accusatur, praesentes habeat accusatores locumque defendendi se ab accusatione accipiat. 
\verse Cum ergo huc convenissent, sine ulla dilatione sequenti die sedens pro tribunali iussi adduci virum; 
\verse de quo, cum stetissent accusatores, nullam causam deferebant, de quibus ego suspicabar malis; 
\verse quaestiones vero quasdam de sua superstitione habebant adversus eum et de quodam Iesu defuncto, quem affirmabat Paulus vivere. 
\verse Haesitans autem ego de huiusmodi quaestione, dicebam si vellet ire Hierosolymam et ibi iudicari de istis. 
\verse Paulo autem appellante, ut servaretur ad Augusti cognitionem, iussi servari eum, donec mittam eum ad Caesarem ”. 
\verse Agrippa autem ad Festum: “ Volebam et ipse hominem audire! ”. “ Cras, inquit, audies eum ”.
 \verse Altera autem die, cum venisset Agrippa et Berenice cum multa ambitione, et introissent in auditorium cum tribunis et viris principalibus civitatis, et iubente Festo, adductus est Paulus. 
\verse Et dicit Festus: “ Agrippa rex et omnes, qui simul adestis nobiscum viri, videtis hunc, de quo omnis multitudo Iudaeorum interpellavit me Hierosolymis et hic, clamantes non oportere eum vivere amplius. 
\verse Ego vero comperi nihil dignum eum morte fecisse, ipso autem hoc appellante Augustum, iudicavi mittere. 
\verse De quo quid certum scribam domino, non habeo; propter quod produxi eum ad vos et maxime ad te, rex Agrippa, ut, interrogatione facta, habeam quid scribam; 
\verse sine ratione enim mihi videtur mittere vinctum et causas eius non significare ”.
 
\begin{biblechapter}
\verse Agrippa vero ad Paulum ait: “ Permittitur tibi loqui pro temetipso ”. Tunc Paulus, extenta manu, coepit rationem reddere: 
\verse “ De omnibus, quibus accusor a Iudaeis, rex Agrippa, aestimo me beatum, apud te cum sim defensurus me hodie, 
\verse maxime te sciente omnia, quae apud Iudaeos sunt consuetudines et quaestiones; propter quod, obsecro, patienter me audias. 
\verse Et quidem vitam meam a iuventute, quae ab initio fuit in gente mea et in Hierosolymis, noverunt omnes Iudaei; 
\verse praescientes me ab initio, si velint testimonium perhibere, quoniam secundum diligentissimam sectam nostrae religionis vixi pharisaeus. 
 \verse Et nunc propter spem eius, quae ad patres nostros repromissionis facta est a Deo, sto iudicio subiectus, 
\verse in quam duodecim tribus nostrae cum perseverantia nocte ac die deservientes sperant devenire; de qua spe accusor a Iudaeis, rex! 
\verse Quid incredibile iudicatur apud vos, si Deus mortuos suscitat?
 \verse Et ego quidem existimaveram me adversus nomen Iesu Nazareni debere multa contraria agere; 
\verse quod et feci Hierosolymis, et multos sanctorum ego in carceribus inclusi, a principibus sacerdotum potestate accepta, et cum occiderentur, detuli sententiam; 
\verse et per omnes synagogas frequenter puniens eos compellebam blasphemare, et abundantius insaniens in eos persequebar usque in exteras civitates.
 \verse In quibus, dum irem Damascum cum potestate et permissu principum sacerdotum, 
\verse die media in via vidi, rex, de caelo supra splendorem solis circumfulgens me lumen et eos, qui mecum simul ibant; 
\verse omnesque nos cum decidissemus in terram, audivi vocem loquentem mihi Hebraica lingua: “Saul, Saul, quid me persequeris? Durum est tibi contra stimulum calcitrare”. 
\verse Ego autem dixi: “Quis es, Domine?”. Dominus autem dixit: “Ego sum Iesus, quem tu persequeris. 
 \verse Sed exsurge et sta super pedes tuos; ad hoc enim apparui tibi, ut constituam te ministrum et testem eorum, quae vidisti, et eorum, quibus apparebo tibi, 
 \verse eripiens te de populo et de gentibus, in quas ego mitto te 
\verse aperire oculos eorum, ut convertantur a tenebris ad lucem et de potestate Satanae ad Deum, ut accipiant remissionem peccatorum et sortem inter sanctificatos per fidem, quae est in me”.
 \verse Unde, rex Agrippa, non fui incredulus caelestis visionis, 
\verse sed his, qui sunt Damasci primum et Hierosolymis, et in omnem regionem Iudaeae et gentibus annuntiabam, ut paenitentiam agerent et converterentur ad Deum digna paenitentiae opera facientes. 
\verse Hac ex causa me Iudaei, cum essem in templo comprehensum, tentabant interficere. 
\verse Auxilium igitur assecutus a Deo usque in hodiernum diem sto testificans minori atque maiori, nihil extra dicens quam ea, quae Prophetae sunt locuti futura esse et Moyses: 
\verse si passibilis Christus, si primus ex resurrectione mortuorum lumen annuntiaturus est populo et gentibus ”.
 \verse Sic autem eo rationem reddente, Festus magna voce dixit: “ Insanis, Paule; multae te litterae ad insaniam convertunt! ”. 
\verse At Paulus: “ Non insanio, inquit, optime Feste, sed veritatis et sobrietatis verba eloquor. 
\verse Scit enim de his rex, ad quem et audenter loquor; latere enim eum nihil horum arbitror, neque enim in angulo hoc gestum est. 
\verse Credis, rex Agrippa, Prophetis? Scio quia credis ”. 
\verse Agrippa autem ad Paulum: “ In modico suades me Christianum fieri! ”. 
\verse Et Paulus: “ Optarem apud Deum et in modico et in magno non tantum te sed et omnes hos, qui audiunt me hodie, fieri tales, qualis et ego sum, exceptis vinculis his! ”.
 \verse Et exsurrexit rex et praeses et Berenice et qui assidebant eis; 
\verse et cum secessissent, loquebantur ad invicem dicentes: “ Nihil morte aut vinculis dignum quid facit homo iste ”. 
\verse Agrippa autem Festo dixit: “ Dimitti poterat homo hic, si non appellasset Caesarem ”.
 
\begin{biblechapter}
\verse Ut autem iudicatum est na vigare nos in Italiam, tradiderunt et Paulum et quosdam alios vinctos centurioni nomine Iulio, cohortis Augustae. 
\verse Ascendentes autem navem Hadramyttenam, incipientem navigare circa Asiae loca, sustulimus, perseverante nobiscum Aristarcho Macedone Thessalonicensi; 
\verse sequenti autem die, devenimus Sidonem, et humane tractans Iulius Paulum permisit ad amicos ire et curam sui agere. 
\verse Et inde cum sustulissemus, subnavigavimus Cypro, propterea quod essent venti contrarii; 
\verse et pelagus Ciliciae et Pamphyliae navigantes venimus Myram, quae est Lyciae. 
\verse Et ibi inveniens centurio navem Alexandrinam navigantem in Italiam transposuit nos in eam. 
\verse Et cum multis diebus tarde navigaremus et vix devenissemus contra Cnidum, prohibente nos vento, subnavigavimus Cretae secundum Salmonem; 
\verse et vix iuxta eam navigantes venimus in locum quendam, qui vocatur Boni Portus, cui iuxta erat civitas Lasaea. 
\verse Multo autem tempore peracto, et cum iam non esset tuta navigatio, eo quod et ieiunium iam praeterisset, monebat Paulus 
 \verse dicens eis: “ Viri, video quoniam cum iniuria et multo damno non solum oneris et navis sed etiam animarum nostrarum incipit esse navigatio ”. 
\verse Centurio autem gubernatori et nauclero magis credebat quam his, quae a Paulo dicebantur. 
 \verse Et cum aptus portus non esset ad hiemandum, plurimi statuerunt consilium enavigare inde, si quo modo possent devenientes Phoenicen hiemare, portum Cretae respicientem ad africum et ad caurum.
 \verse Aspirante autem austro, aestimantes propositum se tenere, cum sustulissent, propius legebant Cretam. 
\verse Non post multum autem misit se contra ipsam ventus typhonicus, qui vocatur euroaquilo; 
\verse cumque arrepta esset navis et non posset conari in ventum, data nave flatibus, ferebamur. 
\verse Insulam autem quandam decurrentes, quae vocatur Cauda, potuimus vix obtinere scapham, 
\verse qua sublata, adiutoriis utebantur accingentes navem; et timentes, ne in Syrtim inciderent, submisso vase, sic ferebantur. 
\verse Valide autem nobis tempestate iactatis, sequenti die iactum fecerunt 
\verse et tertia die suis manibus armamenta navis proiecerunt. 
\verse Neque sole autem neque sideribus apparentibus per plures dies, et tempestate non exigua imminente, iam auferebatur spes omnis salutis nostrae.
 \verse Et cum multa ieiunatio fuisset, tunc stans Paulus in medio eorum dixit: “ Oportebat quidem, o viri, audito me, non tollere a Creta lucrique facere iniuriam hanc et iacturam. 
\verse Et nunc suadeo vobis bono animo esse, nulla enim amissio animae erit ex vobis praeterquam navis; 
\verse astitit enim mihi hac nocte angelus Dei, cuius sum ego, cui et deservio, 
\verse dicens: “Ne timeas, Paule; Caesari te oportet assistere, et ecce donavit tibi Deus omnes, qui navigant tecum”. 
\verse Propter quod bono animo estote, viri; credo enim Deo, quia sic erit, quemadmodum dictum est mihi. 
\verse In insulam autem quandam oportet nos incidere ”. 
\verse Sed posteaquam quarta decima nox supervenit, cum ferremur in Hadria, circa mediam noctem suspicabantur nautae apparere sibi aliquam regionem. 
\verse Qui submittentes bolidem invenerunt passus viginti; et pusillum inde separati et rursum submittentes invenerunt passus quindecim; 
\verse timentes autem, ne in aspera loca incideremus, de puppi mittentes ancoras quattuor optabant diem fieri. 
\verse Nautis vero quaerentibus fugere de navi, cum demisissent scapham in mare sub obtentu, quasi a prora inciperent ancoras extendere, 
\verse dixit Paulus centurioni et militibus: “ Nisi hi in navi manserint, vos salvi fieri non potestis ”. 
\verse Tunc absciderunt milites funes scaphae et passi sunt eam excidere.
 \verse Donec autem lux inciperet fieri, rogabat Paulus omnes sumere cibum dicens: “ Quarta decima hodie die exspectantes ieiuni permanetis nihil accipientes; 
\verse propter quod rogo vos accipere cibum, hoc enim pro salute vestra est, quia nullius vestrum capillus de capite peribit ”. 
\verse Et cum haec dixisset et sumpsisset panem, gratias egit Deo in conspectu omnium et, cum fregisset, coepit manducare. 
\verse Animaequiores autem facti omnes et ipsi assumpserunt cibum. 
 \verse Eramus vero universae animae in navi ducentae septuaginta sex. 
\verse Et satiati cibo alleviabant navem iactantes triticum in mare.
 \verse Cum autem dies factus esset, terram non agnoscebant; sinum vero quendam considerabant habentem litus, in quem cogitabant, si possent, eicere navem. 
 \verse Et cum ancoras abstulissent, committebant mari simul laxantes iuncturas gubernaculorum et, levato artemone, secundum flatum aurae tendebant ad litus. 
 \verse Et cum incidissent in locum dithalassum, impegerunt navem; et prora quidem fixa manebat immobilis, puppis vero solvebatur a vi fluctuum. 
\verse Militum autem consilium fuit, ut custodias occiderent, ne quis, cum enatasset, effugeret; 
\verse centurio autem volens servare Paulum prohibuit eos a consilio iussitque eos, qui possent natare, mittere se primos et ad terram exire 
\verse et ceteros, quosdam in tabulis, quosdam vero super ea, quae de navi essent; et sic factum est ut omnes evaderent ad terram.
 
\begin{biblechapter}
\verse Et cum evasissemus, tunc cognovimus quia Melita in sula vocatur. 
\verse Barbari vero praestabant non modicam humanitatem nobis; accensa enim pyra, suscipiebant nos omnes propter imbrem, qui imminebat, et frigus. 
\verse Cum congregasset autem Paulus sarmentorum aliquantam multitudinem et imposuisset super ignem, vipera, a calore cum processisset, invasit manum eius. 
\verse Ut vero viderunt barbari pendentem bestiam de manu eius, ad invicem dicebant: “ Utique homicida est homo hic, qui cum evaserit de mari, Ultio non permisit vivere ”.
 \verse Et ille quidem excutiens bestiam in ignem, nihil mali passus est; 
\verse at illi exspectabant eum in tumorem convertendum aut subito casurum et mori. Diu autem illis exspectantibus et videntibus nihil mali in eo fieri, convertentes se dicebant eum esse deum.
 \verse In locis autem illis erant praedia principis insulae nomine Publii, qui nos suscipiens triduo benigne hospitio recepit. 
\verse Contigit autem patrem Publii febribus et dysenteria vexatum iacere, ad quem Paulus intravit et, cum orasset et imposuisset ei manus, sanavit eum. 
\verse Quo facto, et ceteri, qui in insula habebant infirmitates, accedebant et curabantur; 
\verse qui etiam multis honoribus nos honoraverunt et navigantibus imposuerunt, quae necessaria erant.
 \verse Post menses autem tres, navigavimus in navi Alexandrina, quae in insula hiemaverat, cui erat insigne Castorum. 
\verse Et cum venissemus Syracusam, mansimus ibi triduo; 
\verse inde solventes devenimus Rhegium. Et post unum diem, superveniente austro, secunda die venimus Puteolos, 
\verse ubi, inventis fratribus, rogati sumus manere apud eos dies septem; et sic venimus Romam. 
 \verse Et inde cum audissent de nobis fratres, occurrerunt nobis usque ad Appii Forum et Tres Tabernas; quos cum vidisset Paulus, gratias agens Deo, accepit fiduciam.
 \verse Cum introissemus autem Romam, permissum est Paulo manere sibimet cum custodiente se milite. 
\verse Factum est autem, ut post tertium diem convocaret primos Iudaeorum; cumque convenissent dicebat eis: “ Ego, viri fratres, nihil adversus plebem faciens aut mores paternos, vinctus ab Hierosolymis traditus sum in manus Romanorum, 
\verse qui cum interrogationem de me habuissent, volebant dimittere, eo quod nulla causa esset mortis in me; 
\verse contradicentibus autem Iudaeis, coactus sum appellare Caesarem, non quasi gentem meam habens aliquid accusare. 
\verse Propter hanc igitur causam rogavi vos videre et alloqui; propter spem enim Israel catena hac circumdatus sum ”. 
\verse At illi dixerunt ad eum: “ Nos neque litteras accepimus de te a Iudaea, neque adveniens aliquis fratrum nuntiavit aut locutus est quid de te malum. 
\verse Rogamus autem a te audire quae sentis; nam de secta hac notum est nobis quia ubique ei contradicitur ”.
 \verse Cum constituissent autem illi diem, venerunt ad eum in hospitium plures, quibus exponebat testificans regnum Dei suadensque eos de Iesu ex Lege Moysis et Prophetis a mane usque ad vesperam. 
\verse Et quidam credebant his, quae dicebantur, quidam vero non credebant; 
\verse cumque invicem non essent consentientes, discedebant, dicente Paulo unum verbum: “ Bene Spiritus Sanctus locutus est per Isaiam prophetam ad patres vestros 
\verse dicens:
 “Vade ad populum istum et dic:
 Auditu audietis et non intellegetis,
 et videntes videbitis et non perspicietis.
 \verse Incrassatum est enim cor populi huius,
 et auribus graviter audierunt
 et oculos suos compresserunt,
 ne forte videant oculis
 et auribus audiant
 et corde intellegant et convertantur,
 et sanabo illos”.
 \verse Notum ergo sit vobis quoniam gentibus missum est hoc salutare Dei; ipsi et audient! ”. 
(\verse) \verse Mansit autem biennio toto in suo conducto; et suscipiebat omnes, qui ingrediebantur ad eum, 
\verse praedicans regnum Dei et docens quae sunt de Domino Iesu Christo cum omni fiducia sine prohibitione.
    
\end{biblechapter}
