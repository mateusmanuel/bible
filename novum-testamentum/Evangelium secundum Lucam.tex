\biblebook{Evangelium secundum Lucam}

\begin{biblechapter}   
\verse Quoniam quidem multi conati sunt ordinare narrationem, quae in nobis completae sunt, rerum, 
\verse sicut tradiderunt nobis, qui ab initio ipsi viderunt et ministri fuerunt verbi, 
\verse visum est et mihi, adsecuto a principio omnia, diligenter ex ordine tibi scribere, optime Theophile, 
\verse ut cognoscas eorum verborum, de quibus eruditus es, firmitatem. 
\verse Fuit in diebus Herodis regis Iudaeae sacerdos quidam nomine Zacharias de vice Abiae, et uxor illi de filiabus Aaron, et nomen eius Elisabeth. 
\verse Erant autem iusti ambo ante Deum, incedentes in omnibus mandatis et iustificationibus Domini, irreprehensibiles. 
\verse Et non erat illis filius, eo quod esset Elisabeth sterilis, et ambo processissent in diebus suis. 
\verse Factum est autem, cum sacerdotio fungeretur in ordine vicis suae ante Deum,  
\verse secundum consuetudinem sacerdotii sorte exiit, ut incensum poneret ingressus in templum Domini; 
\verse et omnis multitudo erat populi orans foris hora incensi. 
\verse Apparuit autem illi angelus Domini stans a dextris altaris incensi; 
\verse et Zacharias turbatus est videns, et timor irruit super eum.  
\verse Ait autem ad illum angelus: “Ne timeas, Zacharia, quoniam exaudita est deprecatio tua, et uxor tua Elisabeth pariet tibi filium, et vocabis nomen eius Ioannem. 
\verse Et erit gaudium tibi et exsultatio, et multi in nativitate eius gaudebunt: 
\verse erit enim magnus coram Domino et vinum et siceram non bibet et Spiritu Sancto replebitur adhuc ex utero matris suae 
\verse et multos filiorum Israel convertet ad Dominum Deum ipsorum. 
\verse Et ipse praecedet ante illum in spiritu et virtute Eliae, ut convertat corda patrum in filios et incredibiles ad prudentiam iustorum, parare Domino plebem perfectam". 
\verse Et dixit Zacharias ad angelum: “Unde hoc sciam? Ego enim sum senex, et uxor mea processit in diebus suis". 
\verse Et respondens angelus dixit ei: “Ego sum Gabriel, qui adsto ante Deum, et missus sum loqui ad te et haec tibi evangelizare. 
\verse Et ecce: eris tacens et non poteris loqui usque in diem, quo haec fiant, pro eo quod non credidisti verbis meis, quae implebuntur in tempore suo". 
\verse Et erat plebs exspectans Zachariam, et mirabantur quod tardaret ipse in templo. 
\verse Egressus autem non poterat loqui ad illos, et cognoverunt quod visionem vidisset in templo; et ipse erat innuens illis et permansit mutus. 
\verse Et factum est, ut impleti sunt dies officii eius, abiit in domum suam. 
\verse Post hos autem dies concepit Elisabeth uxor eius et occultabat se mensibus quinque dicens: 
\verse “Sic mihi fecit Dominus in diebus, quibus respexit auferre opprobrium meum inter homines". 
\verse In mense autem sexto missus est angelus Gabriel a Deo in civitatem Galilaeae, cui nomen Nazareth, 
\verse ad virginem desponsatam viro, cui nomen erat Ioseph de domo David, et nomen virginis Maria. 
\verse Et ingressus ad eam dixit: “Ave, gratia plena, Dominus tecum". 
\verse Ipsa autem turbata est in sermone eius et cogitabat qualis esset ista salutatio. 
\verse Et ait angelus ei: “Ne timeas, Maria; invenisti enim gratiam apud Deum. 
\verse Et ecce concipies in utero et paries filium et vocabis nomen eius Iesum. 
\verse Hic erit magnus et Filius Altissimi vocabitur, et dabit illi Dominus Deus sedem David patris eius, 
\verse et regnabit super domum Iacob in aeternum, et regni eius non erit finis". 
\verse Dixit autem Maria ad angelum: “Quomodo fiet istud, quoniam virum non cognosco?". 
\verse Et respondens angelus dixit ei: “Spiritus Sanctus superveniet in te, et virtus Altissimi obumbrabit tibi: ideoque et quod nascetur sanctum, vocabitur Filius Dei. 
\verse Et ecce Elisabeth cognata tua et ipsa concepit filium in senecta sua, et hic mensis est sextus illi, quae vocatur sterilis, 
\verse quia non erit impossibile apud Deum omne verbum". 
\verse Dixit autem Maria: “Ecce ancilla Domini; fiat mihi secundum verbum tuum". Et discessit ab illa angelus. 
\verse Exsurgens autem Maria in diebus illis abiit in montana cum festinatione in civitatem Iudae 
\verse et intravit in domum Zachariae et salutavit Elisabeth.  
\verse Et factum est, ut audivit salutationem Mariae Elisabeth, exsultavit infans in utero eius, et repleta est Spiritu Sancto Elisabeth 
\verse et exclamavit voce magna et dixit: “Benedicta tu inter mulieres, et benedictus fructus ventris tui. 
\verse Et unde hoc mihi, ut veniat mater Domini mei ad me? 
\verse Ecce enim ut facta est vox salutationis tuae in auribus meis, exsultavit in gaudio infans in utero meo. 
\verse Et beata, quae credidit, quoniam perficientur ea, quae dicta sunt ei a Domino". 
\verse Et ait Maria: “Magnificat anima mea Dominum, 
\verse et exsultavit spiritus meus in Deo salvatore meo, 
\verse quia respexit humilitatem ancillae suae. Ecce enim ex hoc beatam me dicent omnes generationes, 
\verse quia fecit mihi magna, qui potens est, et sanctum nomen eius, 
\verse et misericordia eius in progenies et progenies timentibus eum. 
\verse Fecit potentiam in brachio suo, dispersit superbos mente cordis sui; 
\verse deposuit potentes de sede et exaltavit humiles; 
\verse esurientes implevit bonis et divites dimisit inanes. 
\verse Suscepit Israel puerum suum, recordatus misericordiae, 
\verse sicut locutus est ad patres nostros, Abraham et semini eius in saecula". 
\verse Mansit autem Maria cum illa quasi mensibus tribus et reversa est in domum suam. 
\verse Elisabeth autem impletum est tempus pariendi, et peperit filium.  
\verse Et audierunt vicini et cognati eius quia magnificavit Dominus misericordiam suam cum illa, et congratulabantur ei. 
\verse Et factum est, in die octavo venerunt circumcidere puerum et vocabant eum nomine patris eius, Zachariam.  
\verse Et respondens mater eius dixit: “Nequaquam, sed vocabitur Ioannes". 
\verse Et dixerunt ad illam: “Nemo est in cognatione tua, qui vocetur hoc nomine". 
\verse Innuebant autem patri eius quem vellet vocari eum. 
\verse Et postulans pugillarem scripsit dicens: “Ioannes est nomen eius". Et mirati sunt universi.  
\verse Apertum est autem ilico os eius et lingua eius, et loquebatur benedicens Deum. 
\verse Et factus est timor super omnes vicinos eorum, et super omnia montana Iudaeae divulgabantur omnia verba haec. 
\verse Et posuerunt omnes, qui audierant, in corde suo dicentes: “Quid putas puer iste erit?". Etenim manus Domini erat cum illo. 
\verse Et Zacharias pater eius impletus est Spiritu Sancto et prophetavit dicens: 
\verse “Benedictus Dominus, Deus Israel, quia visitavit et fecit redemptionem plebi suae 
\verse et erexit cornu salutis nobis in domo David pueri sui, 
\verse sicut locutus est per os sanctorum, qui a saeculo sunt, prophetarum eius, 
\verse salutem ex inimicis nostris et de manu omnium, qui oderunt nos; 
\verse ad faciendam misericordiam cum patribus nostris et memorari testamenti sui sancti, 
\verse iusiurandum, quod iuravit ad Abraham patrem nostrum, daturum se nobis, 
\verse ut sine timore, de manu inimicorum liberati, serviamus illi 
\verse in sanctitate et iustitia coram ipso omnibus diebus nostris. 
\verse Et tu, puer, propheta Altissimi vocaberis: praeibis enim ante faciem Domini parare vias eius, 
\verse ad dandam scientiam salutis plebi eius in remissionem peccatorum eorum, 
\verse per viscera misericordiae Dei nostri, in quibus visitabit nos oriens ex alto, 
\verse illuminare his, qui in tenebris et in umbra mortis sedent, ad dirigendos pedes nostros in viam pacis". 
\verse Puer autem crescebat et confortabatur spiritu et erat in deserto usque in diem ostensionis suae ad Israel. 
\end{biblechapter}

\begin{biblechapter}  
\verse Factum est autem, in diebus illis exiit edictum a Caesare Augusto, ut describeretur universus orbis. 
\verse Haec descriptio prima facta est praeside Syriae Quirino. 
\verse Et ibant omnes, ut profiterentur, singuli in suam civitatem. 
\verse Ascendit autem et Ioseph a Galilaea de civitate Nazareth in Iudaeam in civitatem David, quae vocatur Bethlehem, eo quod esset de domo et familia David, 
\verse ut profiteretur cum Maria desponsata sibi, uxore praegnante. 
\verse Factum est autem, cum essent ibi, impleti sunt dies, ut pareret, 
\verse et peperit filium suum primogenitum; et pannis eum involvit et reclinavit eum in praesepio, quia non erat eis locus in deversorio. 
\verse Et pastores erant in regione eadem vigilantes et custodientes vigilias noctis supra gregem suum. 
\verse Et angelus Domini stetit iuxta illos, et claritas Domini circumfulsit illos, et timuerunt timore magno. 
\verse Et dixit illis angelus: “Nolite timere; ecce enim evangelizo vobis gaudium magnum, quod erit omni populo, 
\verse quia natus est vobis hodie Salvator, qui est Christus Dominus, in civitate David. 
\verse Et hoc vobis signum: invenietis infantem pannis involutum et positum in praesepio". 
\verse Et subito facta est cum angelo multitudo militiae caelestis laudantium Deum et dicentium: 
\verse “Gloria in altissimis Deo, et super terram pax in hominibus bonae voluntatis". 
\verse Et factum est, ut discesserunt ab eis angeli in caelum, pastores loquebantur ad invicem: “Transeamus usque Bethlehem et videamus hoc verbum, quod factum est, quod Dominus ostendit nobis". 
\verse Et venerunt festinantes et invenerunt Mariam et Ioseph et infantem positum in praesepio. 
\verse Videntes autem notum fecerunt verbum, quod dictum erat illis de puero hoc. 
\verse Et omnes, qui audierunt, mirati sunt de his, quae dicta erant a pastoribus ad ipsos. 
\verse Maria autem conservabat omnia verba haec conferens in corde suo. 
\verse Et reversi sunt pastores glorificantes et laudantes Deum in omnibus, quae audierant et viderant, sicut dictum est ad illos. 
\verse Et postquam consummati sunt dies octo, ut circumcideretur, vocatum est nomen eius Iesus, quod vocatum est ab angelo, priusquam in utero conciperetur. 
\verse Et postquam impleti sunt dies purgationis eorum secundum legem Moysis, tulerunt illum in Hierosolymam, ut sisterent Domino, 
\verse sicut scriptum est in lege Domini: “Omne masculinum adaperiens vulvam sanctum Domino vocabitur", 
\verse et ut darent hostiam secundum quod dictum est in lege Domini: par turturum aut duos pullos columbarum. 
\verse Et ecce homo erat in Ierusalem, cui nomen Simeon, et homo iste iustus et timoratus, exspectans consolationem Israel, et Spiritus Sanctus erat super eum; 
\verse et responsum acceperat ab Spiritu Sancto non visurum se mortem nisi prius videret Christum Domini. 
\verse Et venit in Spiritu in templum. Et cum inducerent puerum Iesum parentes eius, ut facerent secundum consuetudinem legis pro eo, 
\verse et ipse accepit eum in ulnas suas et benedixit Deum et dixit: 
\verse “Nunc dimittis servum tuum, Domine, secundum verbum tuum in pace, 
\verse quia viderunt oculi mei salutare tuum, 
\verse quod parasti ante faciem omnium populorum, 
\verse lumen ad revelationem gentium et gloriam plebis tuae Israel". 
\verse Et erat pater eius et mater mirantes super his, quae dicebantur de illo.  
\verse Et benedixit illis Simeon et dixit ad Mariam matrem eius: “Ecce positus est hic in ruinam et resurrectionem multorum in Israel et in signum, cui contradicetur 
\verse — et tuam ipsius animam pertransiet gladius — ut revelentur ex multis cordibus cogitationes". 
\verse Et erat Anna prophetissa, filia Phanuel, de tribu Aser. Haec processerat in diebus multis et vixerat cum viro annis septem a virginitate sua; 
\verse et haec vidua usque ad annos octoginta quattuor, quae non discedebat de templo, ieiuniis et obsecrationibus serviens nocte ac die. 
\verse Et haec ipsa hora superveniens confitebatur Deo et loquebatur de illo omnibus, qui exspectabant redemptionem Ierusalem. 
\verse Et ut perfecerunt omnia secundum legem Domini, reversi sunt in Galilaeam in civitatem suam Nazareth. 
\verse Puer autem crescebat et confortabatur plenus sapientia; et gratia Dei erat super illum. 
\verse Et ibant parentes eius per omnes annos in Ierusalem in die festo Paschae.  
\verse Et cum factus esset annorum duodecim, ascendentibus illis secundum consuetudinem diei festi, 
\verse consummatisque diebus, cum redirent, remansit puer Iesus in Ierusalem, et non cognoverunt parentes eius. 
\verse Existimantes autem illum esse in comitatu, venerunt iter diei et requirebant eum inter cognatos et notos; 
\verse et non invenientes regressi sunt in Ierusalem requirentes eum. 
\verse Et factum est, post triduum invenerunt illum in templo sedentem in medio doctorum, audientem illos et interrogantem eos; 
\verse stupebant autem omnes, qui eum audiebant, super prudentia et responsis eius.  
\verse Et videntes eum admirati sunt, et dixit Mater eius ad illum: “Fili, quid fecisti nobis sic? Ecce pater tuus et ego dolentes quaerebamus te". 
\verse Et ait ad illos: “Quid est quod me quaerebatis? Nesciebatis quia in his, quae Patris mei sunt, oportet me esse?". 
\verse Et ipsi non intellexerunt verbum, quod locutus est ad illos. 
\verse Et descendit cum eis et venit Nazareth et erat subditus illis. Et mater eius conservabat omnia verba in corde suo. 
\verse Et Iesus proficiebat sapientia et aetate et gratia apud Deum et homines. 
\end{biblechapter}

\begin{biblechapter}  
\verse Anno autem quinto decimo imperii Tiberii Caesaris, procurante Pontio Pilato Iudaeam, tetrarcha autem Galilaeae Herode, Philippo autem fratre eius tetrarcha Ituraeae et Trachonitidis regionis, et Lysania Abilinae tetrarcha,  
\verse sub principe sacerdotum Anna et Caipha, factum est verbum Dei super Ioannem Zachariae filium in deserto. 
\verse Et venit in omnem regionem circa Iordanem praedicans baptismum paenitentiae in remissionem peccatorum, 
\verse sicut scriptum est in libro sermonum Isaiae prophetae: “Vox clamantis in deserto: “Parate viam Domini, rectas facite semitas eius. 
\verse Omnis vallis implebitur, et omnis mons et collis humiliabitur; et erunt prava in directa, et aspera in vias planas: 
\verse et videbit omnis caro salutare Dei”". 
\verse Dicebat ergo ad turbas, quae exibant, ut baptizarentur ab ipso: “Genimina viperarum, quis ostendit vobis fugere a ventura ira? 
\verse Facite ergo fructus dignos paenitentiae et ne coeperitis dicere in vobis ipsis: “Patrem habemus Abraham”; dico enim vobis quia potest Deus de lapidibus istis suscitare Abrahae filios. 
\verse Iam enim et securis ad radicem arborum posita est; omnis ergo arbor non faciens fructum bonum exciditur et in ignem mittitur". 
\verse Et interrogabant eum turbae dicentes: “Quid ergo faciemus?". 
\verse Respondens autem dicebat illis: “Qui habet duas tunicas, det non habenti; et, qui habet escas, similiter faciat". 
\verse Venerunt autem et publicani, ut baptizarentur, et dixerunt ad illum: “Magister, quid faciemus?". 
\verse At ille dixit ad eos: “Nihil amplius quam constitutum est vobis, faciatis". 
\verse Interrogabant autem eum et milites dicentes: “Quid faciemus et nos?". Et ait illis: “Neminem concutiatis neque calumniam faciatis et contenti estote stipendiis vestris". 
\verse Existimante autem populo et cogitantibus omnibus in cordibus suis de Ioanne, ne forte ipse esset Christus, 
\verse respondit Ioannes dicens omnibus: “Ego quidem aqua baptizo vos. Venit autem fortior me, cuius non sum dignus solvere corrigiam calceamentorum eius: ipse vos baptizabit in Spiritu Sancto et igni;  
\verse cuius ventilabrum in manu eius ad purgandam aream suam et ad congregandum triticum in horreum suum, paleas autem comburet igni inexstinguibili". 
\verse Multa quidem et alia exhortans evangelizabat populum. 
\verse Herodes autem tetrarcha, cum corriperetur ab illo de Herodiade uxore fratris sui et de omnibus malis, quae fecit Herodes, 
\verse adiecit et hoc supra omnia et inclusit Ioannem in carcere. 
\verse Factum est autem, cum baptizaretur omnis populus, et Iesu baptizato et orante, apertum est caelum, 
\verse et descendit Spiritus Sanctus corporali specie sicut columba super ipsum; et vox de caelo facta est: “Tu es Filius meus dilectus; in te complacui mihi". 
\verse Et ipse Iesus erat incipiens quasi annorum triginta, ut putabatur, filius Ioseph, qui fuit Heli, 
\verse qui fuit Matthat, qui fuit Levi, qui fuit Melchi, qui fuit Iannae, qui fuit Ioseph, 
\verse qui fuit Matthathiae, qui fuit Amos, qui fuit Nahum, qui fuit Esli, qui fuit Naggae, 
\verse qui fuit Maath, qui fuit Matthathiae, qui fuit Semei, qui fuit Iosech, qui fuit Ioda, 
\verse qui fuit Ioanna, qui fuit Resa, qui fuit Zorobabel, qui fuit Salathiel, qui fuit Neri,  
\verse qui fuit Melchi, qui fuit Addi, qui fuit Cosam, qui fuit Elmadam, qui fuit Her, 
\verse qui fuit Iesu, qui fuit Eliezer, qui fuit Iorim, qui fuit Matthat, qui fuit Levi, 
\verse qui fuit Simeon, qui fuit Iudae, qui fuit Ioseph, qui fuit Iona, qui fuit Eliachim, 
\verse qui fuit Melea, qui fuit Menna, qui fuit Matthatha, qui fuit Nathan, qui fuit David, 
\verse qui fuit Iesse, qui fuit Obed, qui fuit Booz, qui fuit Salmon, qui fuit Naasson, 
\verse qui fuit Aminadab, qui fuit Admin, qui fuit Arni, qui fuit Esrom, qui fuit Phares, qui fuit Iudae, 
\verse qui fuit Iacob, qui fuit Isaac, qui fuit Abrahae, qui fuit Thare, qui fuit Nachor, 
\verse qui fuit Seruch, qui fuit Ragau, qui fuit Phaleg, qui fuit Heber, qui fuit Sala, 
\verse qui fuit Cainan, qui fuit Arphaxad, qui fuit Sem, qui fuit Noe, qui fuit Lamech, 
\verse qui fuit Mathusala, qui fuit Henoch, qui fuit Iared, qui fuit Malaleel, qui fuit Cainan, 
\verse qui fuit Enos, qui fuit Seth, qui fuit Adam, qui fuit Dei. 
\end{biblechapter}

\begin{biblechapter}  
\verse Iesus autem plenus Spiritu Sancto regressus est ab Iordane et agebatur in Spiritu in deserto 
\verse diebus quadraginta et tentabatur a Diabolo. Et nihil manducavit in diebus illis et, consummatis illis, esuriit. 
\verse Dixit autem illi Diabolus: “Si Filius Dei es, dic lapidi huic, ut panis fiat". 
\verse Et respondit ad illum Iesus: “Scriptum est: “Non in pane solo vivet homo”". 
\verse Et sustulit illum et ostendit illi omnia regna orbis terrae in momento temporis;  
\verse et ait ei Diabolus: “Tibi dabo potestatem hanc universam et gloriam illorum, quia mihi tradita est, et, cui volo, do illam: 
\verse tu ergo, si adoraveris coram me, erit tua omnis". 
\verse Et respondens Iesus dixit illi: “Scriptum est: “Dominum Deum tuum adorabis et illi soli servies”". 
\verse Duxit autem illum in Ierusalem et statuit eum supra pinnam templi et dixit illi: “Si Filius Dei es, mitte te hinc deorsum. 
\verse Scriptum est enim: “Angelis suis mandabit de te, ut conservent te” 
\verse et: “In manibus tollent te, ne forte offendas ad lapidem pedem tuum”". 
\verse Et respondens Iesus ait illi: “Dictum est: “Non tentabis Dominum Deum tuum”". 
\verse Et consummata omni tentatione, Diabolus recessit ab illo usque ad tempus. 
\verse Et regressus est Iesus in virtute Spiritus in Galilaeam. Et fama exiit per universam regionem de illo. 
\verse Et ipse docebat in synagogis eorum et magnificabatur ab omnibus. 
\verse Et venit Nazareth, ubi erat nutritus, et intravit secundum consuetudinem suam die sabbati in synagogam et surrexit legere. 
\verse Et tradi tus est illi liber prophetae Isaiae; et ut revolvit librum, invenit locum, ubi scriptum erat: 
\verse “Spiritus Domini super me; propter quod unxit me evangelizare pauperibus, misit me praedicare captivis remissionem et caecis visum, dimittere confractos in remissione, 
\verse praedicare annum Domini acceptum". 
\verse Et cum plicuisset librum, reddidit ministro et sedit; et omnium in synagoga oculi erant intendentes in eum. 
\verse Coepit autem dicere ad illos: “Hodie impleta est haec Scriptura in auribus vestris". 
\verse Et omnes testimonium illi dabant et mirabantur in verbis gratiae, quae procedebant de ore ipsius, et dicebant: “Nonne hic filius est Ioseph?". 
\verse Et ait illis: “Utique dicetis mihi hanc similitudinem: “Medice, cura teipsum; quanta audivimus facta in Capharnaum, fac et hic in patria tua”". 
\verse Ait autem: “Amen dico vobis: Nemo propheta acceptus est in patria sua. 
\verse In veritate autem dico vobis: Multae viduae erant in diebus Eliae in Israel, quando clausum est caelum annis tribus et mensibus sex, cum facta est fames magna in omni terra; 
\verse et ad nullam illarum missus est Elias nisi in Sarepta Sidoniae ad mulierem viduam.  
\verse Et multi leprosi erant in Israel sub Eliseo propheta; et nemo eorum mundatus est nisi Naaman Syrus". 
\verse Et repleti sunt omnes in synagoga ira haec audientes; 
\verse et surrexerunt et eiecerunt illum extra civitatem et duxerunt illum usque ad supercilium montis, supra quem civitas illorum erat aedificata, ut praecipitarent eum.  
\verse Ipse autem transiens per medium illorum ibat. 
\verse Et descendit in Capharnaum civitatem Galilaeae. Et docebat illos sabbatis;  
\verse et stupebant in doctrina eius, quia in potestate erat sermo ipsius. 
\verse Et in synagoga erat homo habens spiritum daemonii immundi; et exclamavit voce magna: 
\verse “Sine; quid nobis et tibi, Iesu Nazarene? Venisti perdere nos? Scio te qui sis: Sanctus Dei". 
\verse Et increpavit illi Iesus dicens: “Obmutesce et exi ab illo!". Et cum proiecisset illum daemonium in medium, exiit ab illo nihilque illum nocuit. 
\verse Et factus est pavor in omnibus; et colloquebantur ad invicem dicentes: “Quod est hoc verbum, quia in potestate et virtute imperat immundis spiritibus, et exeunt?". 
\verse Et divulgabatur fama de illo in omnem locum regionis. 
\verse Surgens autem de synagoga introivit in domum Simonis. Socrus autem Simonis tenebatur magna febri; et rogaverunt illum pro ea. 
\verse Et stans super illam imperavit febri, et dimisit illam; et continuo surgens ministrabat illis. 
\verse Cum sol autem occidisset, omnes, qui habebant infirmos variis languoribus, ducebant illos ad eum; at ille singulis manus imponens curabat eos. 
\verse Exibant autem daemonia a multis clamantia et dicentia: “Tu es Filius Dei". Et increpans non sinebat ea loqui, quia sciebant ipsum esse Christum. 
\verse Facta autem die, egressus ibat in desertum locum; et turbae requirebant eum et venerunt usque ad ipsum et detinebant illum, ne discederet ab eis. 
\verse Quibus ille ait: “Et aliis civitatibus oportet me evangelizare regnum Dei, quia ideo missus sum". 
\verse Et erat praedicans in synagogis Iudaeae. 
\end{biblechapter}

\begin{biblechapter}  
\verse Factum est autem, cum turba urgeret illum et audiret verbum Dei, et ipse stabat secus stagnum Genesareth 
\verse et vidit duas naves stantes secus stagnum; piscatores autem descenderant de illis et lavabant retia. 
\verse Ascendens autem in unam navem, quae erat Simonis, rogavit eum a terra reducere pusillum; et sedens docebat de navicula turbas. 
\verse Ut cessavit autem loqui, dixit ad Simonem: “Duc in altum et laxate retia vestra in capturam". 
\verse Et respondens Simon dixit: “Praeceptor, per totam noctem laborantes nihil cepimus; in verbo autem tuo laxabo retia". 
\verse Et cum hoc fecissent, concluserunt piscium multitudinem copiosam; rumpebantur autem retia eorum. 
\verse Et annuerunt sociis, qui erant in alia navi, ut venirent et adiuvarent eos; et venerunt et impleverunt ambas naviculas, ita ut mergerentur. 
\verse Quod cum videret Simon Petrus, procidit ad genua Iesu dicens: “Exi a me, quia homo peccator sum, Domine". 
\verse Stupor enim circumdederat eum et omnes, qui cum illo erant, in captura piscium, quos ceperant; 
\verse similiter autem et Iacobum et Ioannem, filios Zebedaei, qui erant socii Simonis. Et ait ad Simonem Iesus: “Noli timere; ex hoc iam homines eris capiens". 
\verse Et subductis ad terram navibus, relictis omnibus, secuti sunt illum. 
\verse Et factum est, cum esset in una civitatum, et ecce vir plenus lepra; et videns Iesum et procidens in faciem rogavit eum dicens: “Domine, si vis, potes me mundare". 
\verse Et extendens manum tetigit illum dicens: “Volo, mundare!"; et confestim lepra discessit ab illo. 
\verse Et ipse praecepit illi, ut nemini diceret, sed: “Vade, ostende te sacerdoti et offer pro emundatione tua, sicut praecepit Moyses, in testimonium illis". 
\verse Perambulabat autem magis sermo de illo, et conveniebant turbae multae, ut audirent et curarentur ab infirmitatibus suis; 
\verse ipse autem secedebat in desertis et orabat. 
\verse Et factum est, in una dierum, et ipse erat docens, et erant pharisaei sedentes et legis doctores, qui venerant ex omni castello Galilaeae et Iudaeae et Ierusalem; et virtus Domini erat ei ad sanandum. 
\verse Et ecce viri portantes in lecto hominem, qui erat paralyticus, et quaerebant eum inferre et ponere ante eum. 
\verse Et non invenientes qua parte illum inferrent prae turba, ascenderunt supra tectum et per tegulas summiserunt illum cum lectulo in medium ante Iesum. 
\verse Quorum fidem ut vidit, dixit: “Homo, remittuntur tibi peccata tua".  
\verse Et coeperunt cogitare scribae et pharisaei dicentes: “Quis est hic, qui loquitur blasphemias? Quis potest dimittere peccata nisi solus Deus?". 
\verse Ut cognovit autem Iesus cogitationes eorum, respondens dixit ad illos: “Quid cogitatis in cordibus vestris? 
\verse Quid est facilius, dicere: “Dimittuntur tibi peccata tua”, an dicere: “Surge et ambula”? 
\verse Ut autem sciatis quia Filius hominis potestatem habet in terra dimittere peccata — ait paralytico - : Tibi dico: Surge, tolle lectulum tuum et vade in domum tuam". 
\verse Et confestim surgens coram illis tulit, in quo iacebat, et abiit in domum suam magnificans Deum. 
\verse Et stupor apprehendit omnes, et magnificabant Deum; et repleti sunt timore dicentes: “Vidimus mirabilia hodie". 
\verse Et post haec exiit et vidit publicanum nomine Levi sedentem ad teloneum et ait illi: “Sequere me". 
\verse Et relictis omnibus, surgens secutus est eum. 
\verse Et fecit ei convivium magnum Levi in domo sua; et erat turba multa publicanorum et aliorum, qui cum illis erant discumbentes. 
\verse Et murmurabant pharisaei et scribae eorum adversus discipulos eius dicentes: “Quare cum publicanis et peccatoribus manducatis et bibitis?". 
\verse Et respondens Iesus dixit ad illos: “Non egent, qui sani sunt, medico, sed qui male habent. 
\verse Non veni vocare iustos sed peccatores in paenitentiam". 
\verse At illi dixerunt ad eum: “Discipuli Ioannis ieiunant frequenter et obsecrationes faciunt, similiter et pharisaeorum; tui autem edunt et bibunt". 
\verse Quibus Iesus ait: “Numquid potestis convivas nuptiarum, dum cum illis est sponsus, facere ieiunare? 
\verse Venient autem dies; et cum ablatus fuerit ab illis sponsus, tunc ieiunabunt in illis diebus". 
\verse Dicebat autem et similitudinem ad illos: “Nemo abscindit commissuram a vestimento novo et immittit in vestimentum vetus; alioquin et novum rumpet, et veteri non conveniet commissura a novo. 
\verse Et nemo mittit vinum novum in utres veteres; alioquin rumpet vinum novum utres et ipsum effundetur, et utres peribunt; 
\verse sed vinum novum in utres novos mittendum est. 
\verse Et nemo bibens vetus vult novum; dicit enim: “Vetus melius est!”". 
\end{biblechapter}

\begin{biblechapter}  
\verse Factum est autem in sabbato, cum transiret per sata, et velle bant discipuli eius spicas et manducabant confricantes manibus. 
\verse Quidam autem pharisaeorum dixerunt: “Quid facitis, quod non licet in sabbatis?". 
\verse Et respondens Iesus ad eos dixit: “Nec hoc legistis, quod fecit David, cum esurisset ipse et qui cum eo erant? 
\verse Quomodo intravit in domum Dei et panes propositionis sumpsit et manducavit et dedit his, qui cum ipso erant, quos non licet manducare nisi tantum sacerdotibus?". 
\verse Et dicebat illis: “Dominus est sabbati Filius hominis". 
\verse Factum est autem in alio sabbato, ut intraret in synagogam et doceret; et erat ibi homo, et manus eius dextra erat arida. 
\verse Observabant autem illum scribae et pharisaei, si sabbato curaret, ut invenirent accusare illum. 
\verse Ipse vero sciebat cogitationes eorum et ait homini, qui habebat manum aridam: “Surge et sta in medium". Et surgens stetit. 
\verse Ait autem ad illos Iesus: “Interrogo vos, si licet sabbato bene facere an male; animam salvam facere an perdere?".  
\verse Et circumspectis omnibus illis, dixit illi: “Extende manum tuam". Et fecit; et restituta est manus eius. 
\verse Ipsi autem repleti sunt insipientia et colloquebantur ad invicem quidnam facerent Iesu. 
\verse Factum est autem in illis diebus, exiit in montem orare et erat pernoctans in oratione Dei. 
\verse Et cum dies factus esset, vocavit discipulos suos et elegit Duodecim ex ipsis, quos et apostolos nominavit: 
\verse Simonem, quem et cognominavit Petrum, et Andream fratrem eius et Iacobum et Ioannem et Philippum et Bartholomaeum 
\verse et Matthaeum et Thomam et Iacobum Alphaei et Simonem, qui vocatur Zelotes, 
\verse et Iudam Iacobi et Iudam Iscarioth, qui fuit proditor. 
\verse Et descendens cum illis stetit in loco campestri, et turba multa discipulorum eius, et multitudo copiosa plebis ab omni Iudaea et Ierusalem et maritima Tyri et Sidonis, 
\verse qui venerunt, ut audirent eum et sanarentur a languoribus suis; et, qui vexabantur a spiritibus immundis, curabantur. 
\verse Et omnis turba quaerebant eum tangere, quia virtus de illo exibat et sanabat omnes. 
\verse Et ipse, elevatis oculis suis in discipulos suos, dicebat: “Beati pauperes, quia vestrum est regnum Dei. 
\verse Beati, qui nunc esuritis, quia saturabimini. Beati, qui nunc fletis, quia ridebitis. 
\verse Beati eritis, cum vos oderint homines et cum separaverint vos et exprobraverint et eiecerint nomen vestrum tamquam malum propter Filium hominis.  
\verse Gaudete in illa die et exsultate, ecce enim merces vestra multa in caelo; secundum haec enim faciebant prophetis patres eorum. 
\verse Verumtamen vae vobis divitibus, quia habetis consolationem vestram! 
\verse Vae vobis, qui saturati estis nunc, quia esurietis! Vae vobis, qui ridetis nunc, quia lugebitis et flebitis! 
\verse Vae, cum bene vobis dixerint omnes homines! Secundum haec enim faciebant pseudoprophetis patres eorum. 
\verse Sed vobis dico, qui auditis: Diligite inimicos vestros, bene facite his, qui vos oderunt; 
\verse benedicite male dicentibus vobis, orate pro calumniantibus vos. 
\verse Ei, qui te percutit in maxillam, praebe et alteram; et ab eo, qui aufert tibi vestimentum, etiam tunicam noli prohibere. 
\verse Omni petenti te tribue; et ab eo, qui aufert, quae tua sunt, ne repetas. 
\verse Et prout vultis, ut faciant vobis homines, facite illis similiter. 
\verse Et si diligitis eos, qui vos diligunt, quae vobis est gratia? Nam et peccatores diligentes se diligunt.  
\verse Et si bene feceritis his, qui vobis bene faciunt, quae vobis est gratia? Si quidem et peccatores idem faciunt. 
\verse Et si mutuum dederitis his, a quibus speratis recipere, quae vobis gratia est? Nam et peccatores peccatoribus fenerantur, ut recipiant aequalia. 
\verse Verumtamen diligite inimicos vestros et bene facite et mutuum date nihil desperantes; et erit merces vestra multa, et eritis filii Altissimi, quia ipse benignus est super ingratos et malos. 
\verse Estote misericordes, sicut et Pater vester misericors est. 
\verse Et nolite iudicare et non iudicabimini; et nolite condemnare et non condemnabimini. Dimittite et dimittemini; 
\verse date, et dabitur vobis: mensuram bonam, confertam, coagitatam, supereffluentem dabunt in sinum vestrum; eadem quippe mensura, qua mensi fueritis, remetietur vobis". 
\verse Dixit autem illis et similitudinem: “Numquid potest caecus caecum ducere? Nonne ambo in foveam cadent? 
\verse Non est discipulus super magistrum; perfectus autem omnis erit sicut magister eius. 
\verse Quid autem vides festucam in oculo fratris tui, trabem autem, quae in oculo tuo est, non consideras? 
\verse Quomodo potes dicere fratri tuo: “Frater, sine eiciam festucam, quae est in oculo tuo”, ipse in oculo tuo trabem non videns? Hypocrita, eice primum trabem de oculo tuo et tunc perspicies, ut educas festucam, quae est in oculo fratris tui. 
\verse Non est enim arbor bona faciens fructum malum, neque iterum arbor mala faciens fructum bonum. 
\verse Unaquaeque enim arbor de fructu suo cognoscitur; neque enim de spinis colligunt ficus, neque de rubo vindemiant uvam. 
\verse Bonus homo de bono thesauro cordis profert bonum, et malus homo de malo profert malum: ex abundantia enim cordis os eius loquitur. 
\verse Quid autem vocatis me: “Domine, Domine”, et non facitis, quae dico? 
\verse Omnis, qui venit ad me et audit sermones meos et facit eos, ostendam vobis cui similis sit: 
\verse similis est homini aedificanti domum, qui fodit in altum et posuit fundamentum supra petram; inundatione autem facta, illisum est flumen domui illi et non potuit eam movere; bene enim aedificata erat. 
\verse Qui autem audivit et non fecit, similis est homini aedificanti domum suam supra terram sine fundamento; in quam illisus est fluvius, et continuo cecidit, et facta est ruina domus illius magna". 
\end{biblechapter}

\begin{biblechapter}  
\verse Cum autem implesset omnia verba sua in aures plebis, intravit Capharnaum. 
\verse Centurionis autem cuiusdam servus male habens erat moriturus, qui illi erat pretiosus. 
\verse Et cum audisset de Iesu, misit ad eum seniores Iudaeorum rogans eum, ut veniret et salvaret servum eius. 
\verse At illi cum venissent ad Iesum, rogabant eum sollicite dicentes: “Dignus est, ut hoc illi praestes: 
\verse diligit enim gentem nostram et synagogam ipse aedificavit nobis". 
\verse Iesus autem ibat cum illis. At cum iam non longe esset a domo, misit centurio amicos dicens ei: “Domine, noli vexari; non enim dignus sum, ut sub tectum meum intres, 
\verse propter quod et meipsum non sum dignum arbitratus, ut venirem ad te; sed dic verbo, et sanetur puer meus. 
\verse Nam et ego homo sum sub potestate constitutus, habens sub me milites, et dico huic: “Vade”, et vadit; et alii: “Veni”, et venit; et servo meo: “Fac hoc”, et facit". 
\verse Quo audito, Iesus miratus est eum et conversus sequentibus se turbis dixit: “Dico vobis, nec in Israel tantam fidem inveni!". 
\verse Et reversi, qui missi fuerant, domum, invenerunt servum sanum. 
\verse Et factum est, deinceps ivit in civitatem, quae vocatur Naim, et ibant cum illo discipuli eius et turba copiosa. 
\verse Cum autem appropinquaret portae civitatis, et ecce defunctus efferebatur filius unicus matri suae; et haec vidua erat, et turba civitatis multa cum illa. 
\verse Quam cum vidisset Dominus, misericordia motus super ea dixit illi: “Noli flere!". 
\verse Et accessit et tetigit loculum; hi autem, qui portabant, steterunt. Et ait: “Adulescens, tibi dico: Surge!". 
\verse Et resedit, qui erat mortuus, et coepit loqui; et dedit illum matri suae. 
\verse Accepit autem omnes timor, et magnificabant Deum dicentes: “Propheta magnus surrexit in nobis" et: “Deus visitavit plebem suam". 
\verse Et exiit hic sermo in universam Iudaeam de eo et omnem circa regionem. 
\verse Et nuntiaverunt Ioanni discipuli eius de omnibus his. 
\verse Et convocavit duos de discipulis suis Ioannes et misit ad Dominum dicens: “Tu es qui venturus es, an alium exspectamus?". 
\verse Cum autem venissent ad eum viri, dixerunt: “Ioannes Baptista misit nos ad te dicens: “Tu es qui venturus es, an alium exspectamus?”". 
\verse In ipsa hora curavit multos a languoribus et plagis et spiritibus malis et caecis multis donavit visum. 
\verse Et respondens dixit illis: “Euntes nuntiate Ioanni, quae vidistis et audistis: caeci vident, claudi ambulant, leprosi mundantur et surdi audiunt, mortui resurgunt, pauperes evangelizantur; 
\verse et beatus est, quicumque non fuerit scandalizatus in me". 
\verse Et cum discessissent nuntii Ioannis, coepit dicere de Ioanne ad turbas: “Quid existis in desertum videre? Arundinem vento moveri? 
\verse Sed quid existis videre? Hominem mollibus vestimentis indutum? Ecce, qui in veste pretiosa sunt et deliciis, in domibus regum sunt. 
\verse Sed quid existis videre? Prophetam? Utique, dico vobis, et plus quam prophetam. 
\verse Hic est, de quo scriptum est: “Ecce mitto angelum meum ante faciem tuam, qui praeparabit viam tuam ante te”. 
\verse Dico vobis: Maior inter natos mulierum Ioanne nemo est; qui autem minor est in regno Dei, maior est illo. 
\verse Et omnis populus audiens et publicani iustificaverunt Deum, baptizati baptismo Ioannis; 
\verse pharisaei autem et legis periti consilium Dei spreverunt in semetipsos, non baptizati ab eo. 
\verse Cui ergo similes dicam homines generationis huius, et cui similes sunt?  
\verse Similes sunt pueris sedentibus in foro et loquentibus ad invicem, quod dicit: “Cantavimus vobis tibiis, et non saltastis; lamentavimus, et non plorastis!”. 
\verse Venit enim Ioannes Baptista neque manducans panem neque bibens vinum, et dicitis: “Daemonium habet!”; 
\verse venit Filius hominis manducans et bibens, et dicitis: “Ecce homo devorator et bibens vinum, amicus publicanorum et peccatorum!”. 
\verse Et iustificata est sapientia ab omnibus filiis suis". 
\verse Rogabat autem illum quidam de pharisaeis, ut manducaret cum illo; et ingressus domum pharisaei discubuit. 
\verse Et ecce mulier, quae erat in civitate peccatrix, ut cognovit quod accubuit in domo pharisaei, attulit alabastrum unguenti; 
\verse et stans retro secus pedes eius flens lacrimis coepit rigare pedes eius et capillis capitis sui tergebat, et osculabatur pedes eius et unguento ungebat. 
\verse Videns autem pharisaeus, qui vocaverat eum, ait intra se dicens: “Hic si esset propheta, sciret utique quae et qualis mulier, quae tangit eum, quia peccatrix est". 
\verse Et respondens Iesus dixit ad illum: “Simon, habeo tibi aliquid dicere". At ille ait: “Magister, dic". 
\verse “Duo debitores erant cuidam feneratori: unus debebat denarios quingentos, alius quinquaginta. 
\verse Non habentibus illis, unde redderent, donavit utrisque. Quis ergo eorum plus diliget eum?". 
\verse Respondens Simon dixit: “Aestimo quia is, cui plus donavit". At ille dixit ei: “Recte iudicasti". 
\verse Et conversus ad mulierem, dixit Simoni: “Vides hanc mulierem? Intravi in domum tuam: aquam pedibus meis non dedisti; haec autem lacrimis rigavit pedes meos et capillis suis tersit. 
\verse Osculum mihi non dedisti; haec autem, ex quo intravi, non cessavit osculari pedes meos. 
\verse Oleo caput meum non unxisti; haec autem unguento unxit pedes meos. 
\verse Propter quod dico tibi: Remissa sunt peccata eius multa, quoniam dilexit multum; cui autem minus dimittitur, minus diligit".  
\verse Dixit autem ad illam: “Remissa sunt peccata tua". 
\verse Et coeperunt, qui simul accumbebant, dicere intra se: “Quis est hic, qui etiam peccata dimittit?”. 
\verse Dixit autem ad mulierem: “Fides tua te salvam fecit; vade in pace!". 
\end{biblechapter}

\begin{biblechapter}  
\verse Et factum est deinceps, et ipse iter faciebat per civitatem et castellum praedicans et evangelizans regnum Dei; et Duodecim cum illo 
\verse et mulieres aliquae, quae erant curatae ab spiritibus malignis et infirmitatibus: Maria, quae vocatur Magdalene, de qua daemonia septem exierant, 
\verse et Ioanna uxor Chuza, procuratoris Herodis, et Susanna et aliae multae, quae ministrabant eis de facultatibus suis. 
\verse Cum autem turba plurima conveniret, et de singulis civitatibus properarent ad eum, dixit per similitudinem: 
\verse “Exiit, qui seminat, seminare semen suum. Et dum seminat ipse, aliud cecidit secus viam et conculcatum est, et volucres caeli comederunt illud. 
\verse Et aliud cecidit super petram et natum aruit, quia non habebat umorem. 
\verse Et aliud cecidit inter spinas, et simul exortae spinae suffocaverunt illud. 
\verse Et aliud cecidit in terram bonam et ortum fecit fructum centuplum". Haec dicens clamabat: “Qui habet aures audiendi, audiat". 
\verse Interrogabant autem eum discipuli eius, quae esset haec parabola. 
\verse Quibus ipse dixit: “Vobis datum est nosse mysteria regni Dei, ceteris autem in parabolis, ut videntes non videant et audientes non intellegant. 
\verse Est autem haec parabola: Semen est verbum Dei. 
\verse Qui autem secus viam, sunt qui audiunt; deinde venit Diabolus et tollit verbum de corde eorum, ne credentes salvi fiant. 
\verse Qui autem supra petram: qui cum audierint, cum gaudio suscipiunt verbum; et hi radices non habent, qui ad tempus credunt, et in tempore tentationis recedunt. 
\verse Quod autem in spinis cecidit: hi sunt, qui audierunt et a sollicitudinibus et divitiis et voluptatibus vitae euntes suffocantur et non referunt fructum. 
\verse Quod autem in bonam terram: hi sunt, qui in corde bono et optimo audientes verbum retinent et fructum afferunt in patientia. 
\verse Nemo autem lucernam accendens operit eam vaso aut subtus lectum ponit, sed supra candelabrum ponit, ut intrantes videant lumen. 
\verse Non enim est occultum, quod non manifestetur, nec absconditum, quod non cognoscatur et in palam veniat. 
\verse Videte ergo quomodo audiatis: qui enim habet, dabitur illi; et, quicumque non habet, etiam quod putat se habere, auferetur ab illo". 
\verse Venerunt autem ad illum mater et fratres eius, et non poterant adire ad eum prae turba. 
\verse Et nuntiatum est illi: “Mater tua et fratres tui stant foris volentes te videre". 
\verse Qui respondens dixit ad eos: “Mater mea et fratres mei hi sunt, qui verbum Dei audiunt et faciunt". 
\verse Factum est autem in una dierum, et ipse ascendit in navem et discipuli eius, et ait ad illos: “Transfretemus trans stagnum". Et ascenderunt. 
\verse Navigantibus autem illis, obdormivit. Et descendit procella venti in stagnum, et complebantur et periclitabantur. 
\verse Accedentes autem suscitaverunt eum dicentes: “Praeceptor, praeceptor, perimus!". At ille surgens increpavit ventum et tempestatem aquae, et cessaverunt, et facta est tranquillitas. 
\verse Dixit autem illis: “Ubi est fides vestra?". Qui timentes mirati sunt dicentes ad invicem: “Quis putas hic est, quia et ventis imperat et aquae, et oboediunt ei?". 
\verse Enavigaverunt autem ad regionem Gergesenorum, quae est contra Galilaeam.  
\verse Et cum egressus esset ad terram, occurrit illi vir quidam de civitate, qui habebat daemonia et iam tempore multo vestimento non induebatur neque in domo manebat sed in monumentis. 
\verse Is ut vidit Iesum, exclamans procidit ante illum et voce magna dixit: “Quid mihi et tibi est, Iesu, Fili Dei Altissimi? Obsecro te, ne me torqueas". 
\verse Praecipiebat enim spiritui immundo, ut exiret ab homine. Multis enim temporibus arripiebat illum, vinciebatur catenis et compedibus custoditus; et ruptis vinculis, agebatur a daemonio in deserta.  
\verse Interrogavit autem illum Iesus dicens: “Quod tibi nomen est?". At ille dixit: “Legio", quia intraverunt daemonia multa in eum. 
\verse Et rogabant eum, ne imperaret illis, ut in abyssum irent. 
\verse Erat autem ibi grex porcorum multorum pascentium in monte; et rogaverunt eum, ut permitteret eis in illos ingredi. Et permisit illis. 
\verse Exierunt ergo daemonia ab homine et intraverunt in porcos, et impetu abiit grex per praeceps in stagnum et suffocatus est. 
\verse Quod ut viderunt factum, qui pascebant, fugerunt et nuntiaverunt in civitatem et in villas. 
\verse Exierunt autem videre, quod factum est, et venerunt ad Iesum et invenerunt hominem sedentem, a quo daemonia exierant, vestitum ac sana mente ad pedes Iesu et timuerunt. 
\verse Nuntiaverunt autem illis hi, qui viderant, quomodo sanus factus esset, qui a daemonio vexabatur. 
\verse Et rogaverunt illum omnis multitudo regionis Gergesenorum, ut discederet ab ipsis, quia timore magno tenebantur. Ipse autem ascendens navem reversus est. 
\verse Et rogabat illum vir, a quo daemonia exierant, ut cum eo esset. Dimisit autem eum dicens: 
\verse “Redi domum tuam et narra quanta tibi fecit Deus". Et abiit per universam civitatem praedicans quanta illi fecisset Iesus. 
\verse Cum autem rediret Iesus, excepit illum turba; erant enim omnes exspectantes eum. 
\verse Et ecce venit vir, cui nomen Iairus, et ipse princeps synagogae erat, et cecidit ad pedes Iesu rogans eum, ut intraret in domum eius, 
\verse quia filia unica erat illi fere annorum duodecim, et haec moriebatur. Et dum iret, a turbis comprimebatur. 
\verse Et mulier quaedam erat in fluxu sanguinis ab annis duodecim, quae in medicos erogaverat omnem substantiam suam nec ab ullo potuit curari; 
\verse accessit retro et tetigit fimbriam vestimenti eius, et confestim stetit fluxus sanguinis eius. 
\verse Et ait Iesus: “Quis est, qui me tetigit?". Negantibus autem omnibus, dixit Petrus: “Praeceptor, turbae te comprimunt et affligunt". 
\verse At dixit Iesus: “Tetigit me aliquis; nam et ego novi virtutem de me exisse". 
\verse Videns autem mulier quia non latuit, tremens venit et procidit ante eum et ob quam causam tetigerit eum indicavit coram omni populo et quemadmodum confestim sanata sit. 
\verse At ipse dixit illi: “Filia, fides tua te salvam fecit. Vade in pace". 
\verse Adhuc illo loquente, venit quidam e domo principis synagogae dicens: “Mortua est filia tua; noli amplius vexare magistrum". 
\verse Iesus autem, audito hoc verbo, respondit ei: “Noli timere; crede tantum, et salva erit". 
\verse Et cum venisset domum, non permisit intrare secum quemquam nisi Petrum et Ioannem et Iacobum et patrem puellae et matrem. 
\verse Flebant autem omnes et plangebant illam. At ille dixit: “Nolite flere; non est enim mortua, sed dormit". 
\verse Et deridebant eum scientes quia mortua esset. 
\verse Ipse autem tenens manum eius clamavit dicens: “Puella, surge!". 
\verse Et reversus est spiritus eius, et surrexit continuo; et iussit illi dari manducare. 
\verse Et stupuerunt parentes eius, quibus praecepit, ne alicui dicerent, quod factum erat. 
\end{biblechapter}

\begin{biblechapter}  
\verse Convocatis autem Duodecim, dedit illis virtutem et potestatem super omnia daemonia, et ut languores curarent, 
\verse et misit illos praedicare regnum Dei et sanare infirmos; 
\verse et ait ad illos: “Nihil tuleritis in via, neque virgam neque peram neque panem neque pecuniam, neque duas tunicas habeatis.  
\verse Et in quamcumque domum intraveritis, ibi manete et inde exite. 
\verse Et quicumque non receperint vos, exeuntes de civitate illa pulverem pedum vestrorum excutite in testimonium supra illos". 
\verse Egressi autem circumibant per castella evangelizantes et curantes ubique. 
\verse Audivit autem Herodes tetrarcha omnia, quae fiebant, et haesitabat, eo quod diceretur a quibusdam: “Ioannes surrexit a mortuis"; 
\verse a quibusdam vero: “Elias apparuit"; ab aliis autem: “Propheta unus de antiquis surrexit". 
\verse Et ait Herodes: “Ioannem ego decollavi; quis autem est iste, de quo audio ego talia?". Et quaerebat videre eum. 
\verse Et reversi apostoli narraverunt illi, quaecumque fecerunt. Et assumptis illis, secessit seorsum ad civitatem, quae vocatur Bethsaida. 
\verse Quod cum cognovissent turbae, secutae sunt illum. Et excepit illos et loquebatur illis de regno Dei et eos, qui cura indigebant, sanabat. 
\verse Dies autem coeperat declinare; et accedentes Duodecim dixerunt illi: “Dimitte turbam, ut euntes in castella villasque, quae circa sunt, divertant et inveniant escas, quia hic in loco deserto sumus". 
\verse Ait autem ad illos: “Vos date illis manducare". At illi dixerunt: “Non sunt nobis plus quam quinque panes et duo pisces, nisi forte nos eamus et emamus in omnem hanc turbam escas". 
\verse Erant enim fere viri quinque milia. Ait autem ad discipulos suos: “Facite illos discumbere per convivia ad quinquagenos". 
\verse Et ita fecerunt et discumbere fecerunt omnes. 
\verse Acceptis autem quinque panibus et duobus piscibus, respexit in caelum et benedixit illis et fregit et dabat discipulis suis, ut ponerent ante turbam. 
\verse Et manducaverunt et saturati sunt omnes; et sublatum est, quod superfuit illis, fragmentorum cophini duodecim. 
\verse Et factum est, cum solus esset orans, erant cum illo discipuli, et interrogavit illos dicens: “Quem me dicunt esse turbae?". 
\verse At illi responderunt et dixerunt: “Ioannem Baptistam, alii autem Eliam, alii vero: Propheta unus de prioribus surrexit". 
\verse Dixit autem illis: “Vos autem quem me esse dicitis?". Respondens Petrus dixit: “Christum Dei". 
\verse At ille increpans illos praecepit, ne cui dicerent hoc, 
\verse dicens: “Oportet Filium hominis multa pati et reprobari a senioribus et principibus sacerdotum et scribis et occidi et tertia die resurgere". 
\verse Dicebat autem ad omnes: “Si quis vult post me venire, abneget semetipsum et tollat crucem suam cotidie et sequatur me. 
\verse Qui enim voluerit animam suam salvam facere, perdet illam; qui autem perdiderit animam suam propter me, hic salvam faciet illam. 
\verse Quid enim proficit homo, si lucretur universum mundum, se autem ipsum perdat vel detrimentum sui faciat? 
\verse Nam qui me erubuerit et meos sermones, hunc Filius hominis erubescet, cum venerit in gloria sua et Patris et sanctorum angelorum. 
\verse Dico autem vobis vere: Sunt aliqui hic stantes, qui non gustabunt mortem, donec videant regnum Dei". 
\verse Factum est autem post haec verba fere dies octo, et assumpsit Petrum et Ioannem et Iacobum et ascendit in montem, ut oraret. 
\verse Et facta est, dum oraret, species vultus eius altera, et vestitus eius albus, refulgens. 
\verse Et ecce duo viri loquebantur cum illo, et erant Moyses et Elias, 
\verse qui visi in gloria dicebant exodum eius, quam completurus erat in Ierusalem. 
\verse Petrus vero et qui cum illo gravati erant somno; et evigilantes viderunt gloriam eius et duos viros, qui stabant cum illo. 
\verse Et factum est, cum discederent ab illo, ait Petrus ad Iesum: “Praeceptor, bonum est nos hic esse; et faciamus tria tabernacula: unum tibi et unum Moysi et unum Eliae", nesciens quid diceret. 
\verse Haec autem illo loquente, facta est nubes et obumbravit eos; et timuerunt intrantibus illis in nubem. 
\verse Et vox facta est de nube dicens: “Hic est Filius meus electus; ipsum audite". 
\verse Et dum fieret vox, inventus est Iesus solus. Et ipsi tacuerunt et nemini dixerunt in illis diebus quidquam ex his, quae viderant. 
\verse Factum est autem in sequenti die, descendentibus illis de monte, occurrit illi turba multa. 
\verse Et ecce vir de turba exclamavit dicens: “Magister, obsecro te, respice in filium meum, quia unicus est mihi; 
\verse et ecce spiritus apprehendit illum, et subito clamat, et dissipat eum cum spuma et vix discedit ab eo dilanians eum; 
\verse et rogavi discipulos tuos, ut eicerent illum, et non potuerunt". 
\verse Respondens autem Iesus dixit: “O generatio infidelis et perversa, usquequo ero apud vos et patiar vos? Adduc huc filium tuum". 
\verse Et cum accederet, elisit illum daemonium et dissipavit. Et increpavit Iesus spiritum immundum et sanavit puerum et reddidit illum patri eius. 
\verse Stupebant autem omnes in magnitudine Dei. Omnibusque mirantibus in omnibus, quae faciebat, dixit ad discipulos suos:  
\verse “Ponite vos in auribus vestris sermones istos: Filius enim hominis futurum est ut tradatur in manus hominum". 
\verse At illi ignorabant verbum istud, et erat velatum ante eos, ut non sentirent illud, et time bant interrogare eum de hoc verbo. 
\verse Intravit autem cogitatio in eos, quis eorum maior esset.  
\verse At Iesus sciens cogitationem cordis illorum, apprehendens puerum statuit eum secus se 
\verse et ait illis: “Quicumque susceperit puerum istum in nomine meo, me recipit; et, quicumque me receperit, recipit eum, qui me misit; nam qui minor est inter omnes vos, hic maior est". 
\verse Respondens autem Ioannes dixit: “Praeceptor, vidimus quendam in nomine tuo eicientem daemonia et prohibuimus eum, quia non sequitur nobiscum". 
\verse Et ait ad illum Iesus: “Nolite prohibere; qui enim non est adversus vos, pro vobis est". 
\verse Factum est autem, dum complerentur dies assumptionis eius, et ipse faciem suam firmavit, ut iret Ierusalem, 
\verse et misit nuntios ante conspectum suum. Et euntes intraverunt in castellum Samaritanorum, ut pararent illi. 
\verse Et non receperunt eum, quia facies eius erat euntis Ierusalem. 
\verse Cum vidissent autem discipuli Iacobus et Ioannes, dixerunt: “Domine, vis dicamus, ut ignis descendat de caelo et consumat illos?". 
\verse Et conversus increpavit illos.  
\verse Et ierunt in aliud castellum. 
\verse Et euntibus illis in via, dixit quidam ad illum: “Sequar te, quocumque ieris". 
\verse Et ait illi Iesus: “Vulpes foveas habent, et volucres caeli nidos, Filius autem hominis non habet, ubi caput reclinet". 
\verse Ait autem ad alterum: “Sequere me". Ille autem dixit: “Domine, permitte mihi primum ire et sepelire patrem meum". 
\verse Dixitque ei Iesus: “Sine, ut mortui sepeliant mortuos suos; tu autem vade, annuntia regnum Dei". 
\verse Et ait alter: “Sequar te, Domine, sed primum permitte mihi renuntiare his, qui domi sunt".  
\verse Ait ad illum Iesus: “Nemo mittens manum suam in aratrum et aspiciens retro, aptus est regno Dei". 
\end{biblechapter}

\begin{biblechapter}  
\verse Post haec autem designavit Dominus alios septuaginta duos et misit illos binos ante faciem suam in omnem civitatem et locum, quo erat ipse venturus.  
\verse Et dicebat illis: “Messis quidem multa, operarii autem pauci; rogate ergo Dominum messis, ut mittat operarios in messem suam. 
\verse Ite; ecce ego mitto vos sicut agnos inter lupos. 
\verse Nolite portare sacculum neque peram neque calceamenta et neminem per viam salutaveritis. 
\verse In quamcumque domum intraveritis, primum dicite: “Pax huic domui”. 
\verse Et si ibi fuerit filius pacis, requiescet super illam pax vestra; sin autem, ad vos revertetur. 
\verse In eadem autem domo manete edentes et bibentes, quae apud illos sunt: dignus enim est operarius mercede sua. Nolite transire de domo in domum. 
\verse Et in quamcumque civitatem intraveritis, et susceperint vos, manducate, quae apponuntur vobis, 
\verse et curate infirmos, qui in illa sunt, et dicite illis: “Appropinquavit in vos regnum Dei”. 
\verse In quamcumque civitatem intraveritis, et non receperint vos, exeuntes in plateas eius dicite: 
\verse “Etiam pulverem, qui adhaesit nobis ad pedes de civitate vestra, extergimus in vos; tamen hoc scitote, quia appropinquavit regnum Dei”. 
\verse Dico vobis quia Sodomis in die illa remissius erit quam illi civitati. 
\verse Vae tibi, Chorazin! Vae tibi, Bethsaida! Quia si in Tyro et Sidone factae fuissent virtutes, quae in vobis factae sunt, olim in cilicio et cinere sedentes paeniterent. 
\verse Verumtamen Tyro et Sidoni remissius erit in iudicio quam vobis. 
\verse Et tu, Capharnaum, numquid usque in caelum exaltaberis? Usque ad infernum demergeris! 
\verse Qui vos audit, me audit; et, qui vos spernit, me spernit; qui autem me spernit, spernit eum, qui me misit". 
\verse Reversi sunt autem septuaginta duo cum gaudio dicentes: “Domine, etiam daemonia subiciuntur nobis in nomine tuo!". 
\verse Et ait illis: “Videbam Satanam sicut fulgur de caelo cadentem. 
\verse Ecce dedi vobis potestatem calcandi supra serpentes et scorpiones et supra omnem virtutem inimici; et nihil vobis nocebit. 
\verse Verumtamen in hoc nolite gaudere, quia spiritus vobis subiciuntur; gaudete autem quod nomina vestra scripta sunt in caelis". 
\verse In ipsa hora exsultavit Spiritu Sancto et dixit: “Confiteor tibi, Pater, Domine caeli et terrae, quod abscondisti haec a sapientibus et prudentibus et revelasti ea parvulis; etiam, Pater, quia sic placuit ante te. 
\verse Omnia mihi tradita sunt a Patre meo; et nemo scit qui sit Filius, nisi Pater, et qui sit Pater, nisi Filius et cui voluerit Filius revelare". 
\verse Et conversus ad discipulos seorsum dixit: “Beati oculi, qui vident, quae videtis. 
\verse Dico enim vobis: Multi prophetae et reges voluerunt videre, quae vos videtis, et non viderunt, et audire, quae auditis, et non audierunt". 
\verse Et ecce quidam legis peritus surrexit tentans illum dicens: “Magister, quid faciendo vitam aeternam possidebo?". 
\verse At ille dixit ad eum: “In Lege quid scriptum est? Quomodo legis?". 
\verse Ille autem respondens dixit: “Diliges Dominum Deum tuum ex toto corde tuo et ex tota anima tua et ex omnibus viribus tuis et ex omni mente tua et proximum tuum sicut teipsum". 
\verse Dixitque illi: “Recte respondisti; hoc fac et vives". 
\verse Ille autem, volens iustificare seipsum, dixit ad Iesum: “Et quis est meus proximus?". 
\verse Suscipiens autem Iesus dixit: “Homo quidam descendebat ab Ierusalem in Iericho et incidit in latrones, qui etiam despoliaverunt eum et, plagis impositis, abierunt, semivivo relicto. 
\verse Accidit autem, ut sacerdos quidam descenderet eadem via et, viso illo, praeterivit; 
\verse similiter et Levita, cum esset secus locum et videret eum, pertransiit. 
\verse Samaritanus autem quidam iter faciens, venit secus eum et videns eum misericordia motus est,  
\verse et appropians alligavit vulnera eius infundens oleum et vinum; et imponens illum in iumentum suum duxit in stabulum et curam eius egit. 
\verse Et altera die protulit duos denarios et dedit stabulario et ait: “Curam illius habe, et, quodcumque supererogaveris, ego, cum rediero, reddam tibi”. 
\verse Quis horum trium videtur tibi proximus fuisse illi, qui incidit in latrones?". 
\verse At ille dixit: “Qui fecit misericordiam in illum". Et ait illi Iesus: “Vade et tu fac similiter". 
\verse Cum autem irent, ipse intravit in quoddam castellum, et mulier quaedam Martha nomine excepit illum. 
\verse Et huic erat soror nomine Maria, quae etiam sedens secus pedes Domini audiebat verbum illius. 
\verse Martha autem satagebat circa frequens ministerium; quae stetit et ait: “Domine, non est tibi curae quod soror mea reliquit me solam ministrare? Dic ergo illi, ut me adiuvet". 
\verse Et respondens dixit illi Dominus: “Martha, Martha, sollicita es et turbaris erga plurima, 
\verse porro unum est necessarium; Maria enim optimam partem elegit, quae non auferetur ab ea". 
\end{biblechapter}

\begin{biblechapter}  
\verse Et factum est cum esset in loco quodam orans, ut cessa vit, dixit unus ex discipulis eius ad eum: “Domine, doce nos orare, sicut et Ioannes docuit discipulos suos". 
\verse Et ait illis: “Cum oratis, dicite: Pater, sanctificetur nomen tuum, adveniat regnum tuum; 
\verse panem nostrum cotidianum da nobis cotidie, 
\verse et dimitte nobis peccata nostra, si quidem et ipsi dimittimus omni debenti nobis, et ne nos inducas in tentationem". 
\verse Et ait ad illos: “Quis vestrum habebit amicum et ibit ad illum media nocte et dicet illi: “Amice, commoda mihi tres panes, 
\verse quoniam amicus meus venit de via ad me, et non habeo, quod ponam ante illum”; 
\verse et ille de intus respondens dicat: “Noli mihi molestus esse; iam ostium clausum est, et pueri mei mecum sunt in cubili; non possum surgere et dare tibi”. 
\verse Dico vobis: Et si non dabit illi surgens, eo quod amicus eius sit, propter improbitatem tamen eius surget et dabit illi, quotquot habet necessarios. 
\verse Et ego vobis dico: Petite, et dabitur vobis; quaerite, et invenietis; pulsate, et aperietur vobis. 
\verse Omnis enim qui petit, accipit; et, qui quaerit, invenit; et pulsanti aperietur. 
\verse Quem autem ex vobis patrem filius petierit piscem, numquid pro pisce serpentem dabit illi? 
\verse Aut si petierit ovum, numquid porriget illi scorpionem? 
\verse Si ergo vos, cum sitis mali, nostis dona bona dare filiis vestris, quanto magis Pater de caelo dabit Spiritum Sanctum petentibus se". 
\verse Et erat eiciens daemonium, et illud erat mutum; et factum est, cum daemonium exisset, locutus est mutus. Et admiratae sunt turbae; 
\verse quidam autem ex eis dixerunt: “In Beelzebul principe daemoniorum eicit daemonia". 
\verse Et alii tentantes signum de caelo quaerebant ab eo. 
\verse Ipse autem sciens cogitationes eorum dixit eis: “Omne regnum in seipsum divisum desolatur, et domus supra domum cadit. 
\verse Si autem et Satanas in seipsum divisus est, quomodo stabit regnum ipsius? Quia dicitis in Beelzebul eicere me daemonia.  
\verse Si autem ego in Beelzebul eicio daemonia, filii vestri in quo eiciunt? Ideo ipsi iudices vestri erunt. 
\verse Porro si in digito Dei eicio daemonia, profecto pervenit in vos regnum Dei. 
\verse Cum fortis armatus custodit atrium suum, in pace sunt ea, quae possidet;  
\verse si autem fortior illo superveniens vicerit eum, universa arma eius auferet, in quibus confidebat, et spolia eius distribuet. 
\verse Qui non est mecum, adversum me est; et, qui non colligit mecum, dispergit. 
\verse Cum immundus spiritus exierit de homine, perambulat per loca inaquosa quaerens requiem; et non inveniens dicit: “Revertar in domum meam unde exivi”. 
\verse Et cum venerit, invenit scopis mundatam et exornatam. 
\verse Et tunc vadit et assumit septem alios spiritus nequiores se, et ingressi habitant ibi; et sunt novissima hominis illius peiora prioribus". 
\verse Factum est autem, cum haec diceret, extollens vocem quaedam mulier de turba dixit illi: “Beatus venter, qui te portavit, et ubera, quae suxisti!". 
\verse At ille dixit: “Quinimmo beati, qui audiunt verbum Dei et custodiunt!". 
\verse Turbis autem concurrentibus, coepit dicere: “Generatio haec generatio nequam est; signum quaerit, et signum non dabitur illi, nisi signum Ionae. 
\verse Nam sicut Ionas fuit signum Ninevitis, ita erit et Filius hominis generationi isti.  
\verse Regina austri surget in iudicio cum viris generationis huius et condemnabit illos, quia venit a finibus terrae audire sapientiam Salomonis, et ecce plus Salomone hic. 
\verse Viri Ninevitae surgent in iudicio cum generatione hac et condemnabunt illam, quia paenitentiam egerunt ad praedicationem Ionae, et ecce plus Iona hic. 
\verse Nemo lucernam accendit et in abscondito ponit neque sub modio sed supra candelabrum, ut, qui ingrediuntur, lumen videant. 
\verse Lucerna corporis est oculus tuus. Si oculus tuus fuerit simplex, totum corpus tuum lucidum erit; si autem nequam fuerit, etiam corpus tuum tenebrosum erit. 
\verse Vide ergo, ne lumen, quod in te est, tenebrae sint. 
\verse Si ergo corpus tuum totum lucidum fuerit non habens aliquam partem tenebrarum, erit lucidum totum, sicut quando lucerna in fulgore suo illuminat te". 
\verse Et cum loqueretur, rogavit illum quidam pharisaeus, ut pranderet apud se; et ingressus recubuit. 
\verse Pharisaeus autem videns miratus est quod non baptizatus esset ante prandium. 
\verse Et ait Dominus ad illum: “Nunc vos pharisaei, quod de foris est calicis et catini, mundatis; quod autem intus est vestrum, plenum est rapina et iniquitate. 
\verse Stulti! Nonne, qui fecit, quod de foris est, etiam id, quod de intus est, fecit? 
\verse Verumtamen, quae insunt, date eleemosynam; et ecce omnia munda sunt vobis. 
\verse Sed vae vobis pharisaeis, quia decimatis mentam et rutam et omne holus et praeteritis iudicium et caritatem Dei! Haec autem oportuit facere et illa non omittere. 
\verse Vae vobis pharisaeis, quia diligitis primam cathedram in synagogis et salutationes in foro! 
\verse Vae vobis, quia estis ut monumenta, quae non parent, et homines ambulantes supra nesciunt!". 
\verse Respondens autem quidam ex legis peritis ait illi: “Magister, haec dicens etiam nobis contumeliam facis". 
\verse At ille ait: “Et vobis legis peritis: Vae, quia oneratis homines oneribus, quae portari non possunt, et ipsi uno digito vestro non tangitis sarcinas! 
\verse Vae vobis, quia aedificatis monumenta prophetarum, patres autem vestri occiderunt illos! 
\verse Profecto testificamini et consentitis operibus patrum vestrorum, quoniam ipsi quidem eos occiderunt, vos autem aedificatis. 
\verse Propterea et sapientia Dei dixit: Mittam ad illos prophetas et apostolos, et ex illis occident et persequentur,  
\verse ut requiratur sanguis omnium prophetarum, qui effusus est a constitutione mundi, a generatione ista, 
\verse a sanguine Abel usque ad sanguinem Zachariae, qui periit inter altare et aedem. Ita dico vobis: Requiretur ab hac generatione.  
\verse Vae vobis legis peritis, quia tulistis clavem scientiae! Ipsi non introistis et eos, qui introibant, prohibuistis". 
\verse Cum autem inde exisset, coeperunt scribae et pharisaei graviter insistere et eum allicere in sermone de multis 
\verse insidiantes ei, ut caperent aliquid ex ore eius. 
\end{biblechapter}

\begin{biblechapter}  
\verse Interea multis turbis circumstantibus, ita ut se invicem conculcarent, coepit dicere ad discipulos suos primum: “Attendite a fermento pharisaeorum, quod est hypocrisis. 
\verse Nihil autem opertum est, quod non reveletur, neque absconditum, quod non sciatur. 
\verse Quoniam, quae in tenebris dixistis, in lumine audientur; et, quod in aurem locuti estis in cubiculis, praedicabitur in tectis. 
\verse Dico autem vobis amicis meis: Ne terreamini ab his, qui occidunt corpus et post haec non habent amplius, quod faciant. 
\verse Ostendam autem vobis quem timeatis: Timete eum, qui postquam occiderit, habet potestatem mittere in gehennam. Ita dico vobis: Hunc timete. 
\verse Nonne quinque passeres veneunt dipundio? Et unus ex illis non est in oblivione coram Deo. 
\verse Sed et capilli capitis vestri omnes numerati sunt. Nolite timere; multis passeribus pluris estis. 
\verse Dico autem vobis: Omnis, quicumque confessus fuerit in me coram hominibus, et Filius hominis confitebitur in illo coram angelis Dei; 
\verse qui autem negaverit me coram hominibus, denegabitur coram angelis Dei. 
\verse Et omnis, qui dicet verbum in Filium hominis, remittetur illi; ei autem, qui in Spiritum Sanctum blasphemaverit, non remittetur. 
\verse Cum autem inducent vos in synagogas et ad magistratus et potestates, nolite solliciti esse qualiter aut quid respondeatis aut quid dicatis: 
\verse Spiritus enim Sanctus docebit vos in ipsa hora, quae oporteat dicere". 
\verse Ait autem quidam ei de turba: “Magister, dic fratri meo, ut dividat mecum hereditatem". 
\verse At ille dixit ei: “Homo, quis me constituit iudicem aut divisorem super vos?". 
\verse Dixitque ad illos: “Videte et cavete ab omni avaritia, quia si cui res abundant, vita eius non est ex his, quae possidet". 
\verse Dixit autem similitudinem ad illos dicens: “Hominis cuiusdam divitis uberes fructus ager attulit. 
\verse Et cogitabat intra se dicens: “Quid faciam, quod non habeo, quo congregem fructus meos?”. 
\verse Et dixit: “Hoc faciam: destruam horrea mea et maiora aedificabo et illuc congregabo omne triticum et bona mea; 
\verse et dicam animae meae: Anima, habes multa bona posita in annos plurimos; requiesce, comede, bibe, epulare”. 
\verse Dixit autem illi Deus: “Stulte! Hac nocte animam tuam repetunt a te; quae autem parasti, cuius erunt?”.  
\verse Sic est qui sibi thesaurizat et non fit in Deum dives". 
\verse Dixitque ad discipulos suos: “Ideo dico vobis: nolite solliciti esse animae quid manducetis, neque corpori quid vestiamini. 
\verse Anima enim plus est quam esca, et corpus quam vestimentum. 
\verse Considerate corvos, quia non seminant neque metunt, quibus non est cellarium neque horreum, et Deus pascit illos; quanto magis vos pluris estis volucribus. 
\verse Quis autem vestrum cogitando potest adicere ad aetatem suam cubitum? 
\verse Si ergo neque, quod minimum est, potestis, quid de ceteris solliciti estis? 
\verse Considerate lilia quomodo crescunt: non laborant neque nent; dico autem vobis: Nec Salomon in omni gloria sua vestiebatur sicut unum ex istis. 
\verse Si autem fenum, quod hodie in agro est et cras in clibanum mittitur, Deus sic vestit, quanto magis vos, pusillae fidei. 
\verse Et vos nolite quaerere quid manducetis aut quid bibatis et nolite solliciti esse. 
\verse Haec enim omnia gentes mundi quaerunt; Pater autem vester scit quoniam his indigetis. 
\verse Verumtamen quaerite regnum eius; et haec adicientur vobis. 
\verse Noli timere, pusillus grex, quia complacuit Patri vestro dare vobis regnum. 
\verse Vendite, quae possidetis, et date eleemosynam. Facite vobis sacculos, qui non veterescunt, thesaurum non deficientem in caelis, quo fur non appropiat, neque tinea corrumpit; 
\verse ubi enim thesaurus vester est, ibi et cor vestrum erit. 
\verse Sint lumbi vestri praecincti et lucernae ardentes, 
\verse et vos similes hominibus exspectantibus dominum suum, quando revertatur a nuptiis, ut, cum venerit et pulsaverit, confestim aperiant ei. 
\verse Beati, servi illi, quos, cum venerit dominus, invenerit vigilantes. Amen dico vobis, quod praecinget se et faciet illos discumbere et transiens ministrabit illis. 
\verse Et si venerit in secunda vigilia, et si in tertia vigilia venerit, et ita invenerit, beati sunt illi. 
\verse Hoc autem scitote, quia, si sciret pater familias, qua hora fur veniret, non sineret perfodi domum suam. 
\verse Et vos estote parati, quia, qua hora non putatis, Filius hominis venit". 
\verse Ait autem Petrus: “Domine, ad nos dicis hanc parabolam an et ad omnes?". 
\verse Et dixit Dominus: “Quis putas est fidelis dispensator et prudens, quem constituet dominus super familiam suam, ut det illis in tempore tritici mensuram? 
\verse Beatus ille servus, quem, cum venerit dominus eius, invenerit ita facientem. 
\verse Vere dico vobis: Supra omnia, quae possidet, constituet illum. 
\verse Quod si dixerit servus ille in corde suo: “Moram facit dominus meus venire”, et coeperit percutere pueros et ancillas et edere et bibere et inebriari, 
\verse veniet dominus servi illius in die, qua non sperat, et hora, qua nescit, et dividet eum partemque eius cum infidelibus ponet. 
\verse Ille autem servus, qui cognovit voluntatem domini sui et non praeparavit vel non fecit secundum voluntatem eius, vapulabit multis; 
\verse qui autem non cognovit et fecit digna plagis, vapulabit paucis. Omni autem, cui multum datum est, multum quaeretur ab eo; et cui commendaverunt multum, plus petent ab eo. 
\verse Ignem veni mittere in terram et quid volo? Si iam accensus esset! 
\verse Baptisma autem habeo baptizari et quomodo coartor, usque dum perficiatur! 
\verse Putatis quia pacem veni dare in terram? Non, dico vobis, sed separationem.  
\verse Erunt enim ex hoc quinque in domo una divisi: tres in duo, et duo in tres;  
\verse dividentur pater in filium et filius in patrem, mater in filiam et filia in matrem, socrus in nurum suam et nurus in socrum". 
\verse Dicebat autem et ad turbas: “Cum videritis nubem orientem ab occasu, statim dicitis: “Nimbus venit”, et ita fit; 
\verse et cum austrum flantem, dicitis: “Aestus erit”, et fit. 
\verse Hypocritae, faciem terrae et caeli nostis probare, hoc autem tempus quomodo nescitis probare? 
\verse Quid autem et a vobis ipsis non iudicatis, quod iustum est? 
\verse Cum autem vadis cum adversario tuo ad principem, in via da operam liberari ab illo, ne forte trahat te apud iudicem, et iudex tradat te exactori, et exactor mittat te in carcerem. 
\verse Dico tibi: Non exies inde, donec etiam novissimum minutum reddas". 
\end{biblechapter}

\begin{biblechapter}  
\verse Aderant autem quidam ipso in tempore nuntiantes illi de Galilaeis, quorum sanguinem Pilatus miscuit cum sacrificiis eorum. 
\verse Et respondens dixit illis: “Putatis quod hi Galilaei prae omnibus Galilaeis peccatores fuerunt, quia talia passi sunt? 
\verse Non, dico vobis, sed, nisi paenitentiam egeritis, omnes similiter peribitis. 
\verse Vel illi decem et octo, supra quos cecidit turris in Siloam et occidit eos, putatis quia et ipsi debitores fuerunt praeter omnes homines habitantes in Ierusalem? 
\verse Non, dico vobis, sed, si non paenitentiam egeritis, omnes similiter peribitis". 
\verse Dicebat autem hanc similitudinem: “Arborem fici habebat quidam plantatam in vinea sua et venit quaerens fructum in illa et non invenit. 
\verse Dixit autem ad cultorem vineae: “Ecce anni tres sunt, ex quo venio quaerens fructum in ficulnea hac et non invenio. Succide ergo illam. Ut quid etiam terram evacuat?”. 
\verse At ille respondens dicit illi: “Domine, dimitte illam et hoc anno, usque dum fodiam circa illam et mittam stercora, 
\verse et si quidem fecerit fructum in futurum; sin autem succides eam”". 
\verse Erat autem docens in una synagogarum sabbatis. 
\verse Et ecce mulier, quae habebat spiritum infirmitatis annis decem et octo et erat inclinata nec omnino poterat sursum respicere. 
\verse Quam cum vidisset Iesus, vocavit et ait illi: “Mulier, dimissa es ab infirmitate tua", 
\verse et imposuit illi manus; et confestim erecta est et glorificabat Deum. 
\verse Respondens autem archisynagogus, indignans quia sabbato curasset Iesus, dicebat turbae: “Sex dies sunt, in quibus oportet operari; in his ergo venite et curamini et non in die sabbati". 
\verse Respondit autem ad illum Dominus et dixit: “Hypocritae, unusquisque vestrum sabbato non solvit bovem suum aut asinum a praesepio et ducit adaquare? 
\verse Hanc autem filiam Abrahae, quam alligavit Satanas ecce decem et octo annis, non oportuit solvi a vinculo isto die sabbati?". 
\verse Et cum haec diceret, erubescebant omnes adversarii eius, et omnis populus gaudebat in universis, quae gloriose fiebant ab eo. 
\verse Dicebat ergo: “Cui simile est regnum Dei, et cui simile existimabo illud? 
\verse Simile est grano sinapis, quod acceptum homo misit in hortum suum, et crevit et factum est in arborem, et volucres caeli requieverunt in ramis eius". 
\verse Et iterum dixit: “Cui simile aestimabo regnum Dei? 
\verse Simile est fermento, quod acceptum mulier abscondit in farinae sata tria, donec fermentaretur totum". 
\verse Et ibat per civitates et castella docens et iter faciens in Hierosolymam.  
\verse Ait autem illi quidam: “Domine, pauci sunt, qui salvantur?". Ipse autem dixit ad illos: 
\verse “Contendite intrare per angustam portam, quia multi, dico vobis, quaerent intrare et non poterunt. 
\verse Cum autem surrexerit pater familias et clauserit ostium, et incipietis foris stare et pulsare ostium dicentes: “Domine, aperi nobis”; et respondens dicet vobis: “Nescio vos unde sitis”. 
\verse Tunc incipietis dicere: “Manducavimus coram te et bibimus, et in plateis nostris docuisti”; 
\verse et dicet loquens vobis: “Nescio vos unde sitis; discedite a me, omnes operarii iniquitatis”. 
\verse Ibi erit fletus et stridor dentium, cum videritis Abraham et Isaac et Iacob et omnes prophetas in regno Dei, vos autem expelli foras. 
\verse Et venient ab oriente et occidente et aquilone et austro et accumbent in regno Dei. 
\verse Et ecce sunt novissimi, qui erunt primi, et sunt primi, qui erunt novissimi". 
\verse In ipsa hora accesserunt quidam pharisaeorum dicentes illi: “Exi et vade hinc, quia Herodes vult te occidere". 
\verse Et ait illis: “Ite, dicite vulpi illi: “Ecce eicio daemonia et sanitates perficio hodie et cras et tertia consummor. 
\verse Verumtamen oportet me hodie et cras et sequenti ambulare, quia non capit prophetam perire extra Ierusalem”. 
\verse Ierusalem, Ierusalem, quae occidis prophetas et lapidas eos, qui missi sunt ad te, quotiens volui congregare filios tuos, quemadmodum avis nidum suum sub pinnis, et noluistis. 
\verse Ecce relinquitur vobis domus vestra. Dico autem vobis: Non videbitis me, donec veniat cum dicetis: “Benedictus, qui venit in nomine Domini”". 
\end{biblechapter}

\begin{biblechapter}  
\verse Et factum est, cum intraret in domum cuiusdam principis pharisaeorum sabbato manducare panem, et ipsi observabant eum. 
\verse Et ecce homo quidam hydropicus erat ante illum. 
\verse Et respondens Iesus dixit ad legis peritos et pharisaeos dicens: “Licet sabbato curare an non?". 
\verse At illi tacuerunt. Ipse vero apprehensum sanavit eum ac dimisit. 
\verse Et ad illos dixit: “Cuius vestrum filius aut bos in puteum cadet, et non continuo extrahet illum die sabbati?". 
\verse Et non poterant ad haec respondere illi. 
\verse Dicebat autem ad invitatos parabolam, intendens quomodo primos accubitus eligerent, dicens ad illos: 
\verse “Cum invitatus fueris ab aliquo ad nuptias, non discumbas in primo loco, ne forte honoratior te sit invitatus ab eo, 
\verse et veniens is qui te et illum vocavit, dicat tibi: “Da huic locum”; et tunc incipias cum rubore novissimum locum tenere. 
\verse Sed cum vocatus fueris, vade, recumbe in novissimo loco, ut, cum venerit qui te invitavit, dicat tibi: “Amice, ascende superius”; tunc erit tibi gloria coram omnibus simul discumbentibus. 
\verse Quia omnis, qui se exaltat, humiliabitur; et, qui se humiliat, exaltabitur". 
\verse Dicebat autem et ei, qui se invitaverat: “Cum facis prandium aut cenam, noli vocare amicos tuos neque fratres tuos neque cognatos neque vicinos divites, ne forte et ipsi te reinvitent, et fiat tibi retributio. 
\verse Sed cum facis convivium, voca pauperes, debiles, claudos, caecos; 
\verse et beatus eris, quia non habent retribuere tibi. Retribuetur enim tibi in resurrectione iustorum". 
\verse Haec cum audisset quidam de simul discumbentibus, dixit illi: “Beatus, qui manducabit panem in regno Dei". 
\verse At ipse dixit ei: “Homo quidam fecit cenam magnam et vocavit multos; 
\verse et misit servum suum hora cenae dicere invitatis: “Venite, quia iam paratum est”. 
\verse Et coeperunt simul omnes excusare. Primus dixit ei: “Villam emi et necesse habeo exire et videre illam; rogo te, habe me excusatum”. 
\verse Et alter dixit: “Iuga boum emi quinque et eo probare illa; rogo te, habe me excusatum”. 
\verse Et alius dixit: “Uxorem duxi et ideo non possum venire”. 
\verse Et reversus servus nuntiavit haec domino suo. Tunc iratus pater familias dixit servo suo: “Exi cito in plateas et vicos civitatis et pauperes ac debiles et caecos et claudos introduc huc”. 
\verse Et ait servus: “Domine, factum est, ut imperasti, et adhuc locus est”. 
\verse Et ait dominus servo: “Exi in vias et saepes, et compelle intrare, ut impleatur domus mea. 
\verse Dico autem vobis, quod nemo virorum illorum, qui vocati sunt, gustabit cenam meam”". 
\verse Ibant autem turbae multae cum eo; et conversus dixit ad illos: 
\verse “Si quis venit ad me et non odit patrem suum et matrem et uxorem et filios et fratres et sorores, adhuc et animam suam, non potest esse meus discipulus.  
\verse Et, qui non baiulat crucem suam et venit post me, non potest esse meus discipulus. 
\verse Quis enim ex vobis volens turrem aedificare, non prius sedens computat sumptus, si habet ad perficiendum? 
\verse Ne, posteaquam posuerit fundamentum et non potuerit perficere, omnes, qui vident, incipiant illudere ei 
\verse dicentes: “Hic homo coepit aedificare et non potuit consummare”. 
\verse Aut quis rex, iturus committere bellum adversus alium regem, non sedens prius cogitat, si possit cum decem milibus occurrere ei, qui cum viginti milibus venit ad se?  
\verse Alioquin, adhuc illo longe agente, legationem mittens rogat ea, quae pacis sunt. 
\verse Sic ergo omnis ex vobis, qui non renuntiat omnibus, quae possidet, non potest meus esse discipulus. 
\verse Bonum est sal; si autem sal quoque evanuerit, in quo condietur? 
\verse Neque in terram neque in sterquilinium utile est, sed foras proiciunt illud. Qui habet aures audiendi, audiat". 
\end{biblechapter}

\begin{biblechapter}  
\verse Erant autem appropinquantes ei omnes publicani et peccatores, ut audirent illum. 
\verse Et murmurabant pharisaei et scribae dicentes: “Hic peccatores recipit et manducat cum illis". 
\verse Et ait ad illos parabolam istam dicens: 
\verse “Quis ex vobis homo, qui habet centum oves et si perdiderit unam ex illis, nonne dimittit nonaginta novem in deserto et vadit ad illam, quae perierat, donec inveniat illam? 
\verse Et cum invenerit eam, imponit in umeros suos gaudens 
\verse et veniens domum convocat amicos et vicinos dicens illis: “Congratulamini mihi, quia inveni ovem meam, quae perierat”. 
\verse Dico vobis: Ita gaudium erit in caelo super uno peccatore paenitentiam agente quam super nonaginta novem iustis, qui non indigent paenitentia. 
\verse Aut quae mulier habens drachmas decem, si perdiderit drachmam unam, nonne accendit lucernam et everrit domum et quaerit diligenter, donec inveniat? 
\verse Et cum invenerit, convocat amicas et vicinas dicens: “Congratulamini mihi, quia inveni drachmam, quam perdideram”. 
\verse Ita dico vobis: Gaudium fit coram angelis Dei super uno peccatore paenitentiam agente". 
\verse Ait autem: “Homo quidam habebat duos filios. 
\verse Et dixit adulescentior ex illis patri: “Pater, da mihi portionem substantiae, quae me contingit”. Et divisit illis substantiam. 
\verse Et non post multos dies, congregatis omnibus, adulescentior filius peregre profectus est in regionem longinquam et ibi dissipavit substantiam suam vivendo luxuriose. 
\verse Et postquam omnia consummasset, facta est fames valida in regione illa, et ipse coepit egere.  
\verse Et abiit et adhaesit uni civium regionis illius, et misit illum in villam suam, ut pasceret porcos; 
\verse et cupiebat saturari de siliquis, quas porci manducabant, et nemo illi dabat. 
\verse In se autem reversus dixit: “Quanti mercennarii patris mei abundant panibus, ego autem hic fame pereo. 
\verse Surgam et ibo ad patrem meum et dicam illi: Pater, peccavi in caelum et coram te 
\verse et iam non sum dignus vocari filius tuus; fac me sicut unum de mercennariis tuis”. 
\verse Et surgens venit ad patrem suum. Cum autem adhuc longe esset, vidit illum pater ipsius et misericordia motus est et accurrens cecidit supra collum eius et osculatus est illum. 
\verse Dixitque ei filius: “Pater, peccavi in caelum et coram te; iam non sum dignus vocari filius tuus”. 
\verse Dixit autem pater ad servos suos: “Cito proferte stolam primam et induite illum et date anulum in manum eius et calceamenta in pedes  
\verse et adducite vitulum saginatum, occidite et manducemus et epulemur, 
\verse quia hic filius meus mortuus erat et revixit, perierat et inventus est”. Et coeperunt epulari. 
\verse Erat autem filius eius senior in agro et, cum veniret et appropinquaret domui, audivit symphoniam et choros 
\verse et vocavit unum de servis et interrogavit quae haec essent. 
\verse Isque dixit illi: “Frater tuus venit, et occidit pater tuus vitulum saginatum, quia salvum illum recepit”. 
\verse Indignatus est autem et nolebat introire. Pater ergo illius egressus coepit rogare illum. 
\verse At ille respondens dixit patri suo: “Ecce tot annis servio tibi et numquam mandatum tuum praeterii, et numquam dedisti mihi haedum, ut cum amicis meis epularer; 
\verse sed postquam filius tuus hic, qui devoravit substantiam tuam cum meretricibus, venit, occidisti illi vitulum saginatum”. 
\verse At ipse dixit illi: “Fili, tu semper mecum es, et omnia mea tua sunt; 
\verse epulari autem et gaudere oportebat, quia frater tuus hic mortuus erat et revixit, perierat et inventus est”". 
\end{biblechapter}

\begin{biblechapter}  
\verse Dicebat autem et ad discipulos: “Homo quidam erat dives, qui habebat vilicum, et hic diffamatus est apud illum quasi dissipasset bona ipsius. 
\verse Et vocavit illum et ait illi: “Quid hoc audio de te? Redde rationem vilicationis tuae; iam enim non poteris vilicare”. 
\verse Ait autem vilicus intra se: “Quid faciam, quia dominus meus aufert a me vilicationem? Fodere non valeo, mendicare erubesco. 
\verse Scio quid faciam, ut, cum amotus fuero a vilicatione, recipiant me in domos suas”. 
\verse Convocatis itaque singulis debitoribus domini sui, dicebat primo: “Quantum debes domino meo?”. 
\verse At ille dixit: “Centum cados olei”. Dixitque illi: “Accipe cautionem tuam et sede cito, scribe quinquaginta”.  
\verse Deinde alii dixit: “Tu vero quantum debes?”. Qui ait: “Centum coros tritici”. Ait illi: “Accipe litteras tuas et scribe octoginta”. 
\verse Et laudavit dominus vilicum iniquitatis, quia prudenter fecisset, quia filii huius saeculi prudentiores filiis lucis in generatione sua sunt. 
\verse Et ego vobis dico: Facite vobis amicos de mammona iniquitatis, ut, cum defecerit, recipiant vos in aeterna tabernacula. 
\verse Qui fidelis est in minimo, et in maiori fidelis est; et, qui in modico iniquus est, et in maiori iniquus est. 
\verse Si ergo in iniquo mammona fideles non fuistis, quod verum est, quis credet vobis? 
\verse Et si in alieno fideles non fuistis, quod vestrum est, quis dabit vobis? 
\verse Nemo servus potest duobus dominis servire: aut enim unum odiet et alterum diliget, aut uni adhaerebit et alterum contemnet. Non potestis Deo servire et mammonae". 
\verse Audiebant autem omnia haec pharisaei, qui erant avari, et deridebant illum. 
\verse Et ait illis: “Vos estis, qui iustificatis vos coram hominibus; Deus autem novit corda vestra, quia, quod hominibus altum est, abominatio est ante Deum. 
\verse Lex et Prophetae usque ad Ioannem; ex tunc regnum Dei evangelizatur, et omnis in illud vim facit. 
\verse Facilius est autem caelum et terram praeterire, quam de Lege unum apicem cadere. 
\verse Omnis, qui dimittit uxorem suam et ducit alteram, moechatur; et, qui dimissam a viro ducit, moechatur. 
\verse Homo quidam erat dives et induebatur purpura et bysso et epulabatur cotidie splendide. 
\verse Quidam autem pauper nomine Lazarus iacebat ad ianuam eius ulceribus plenus 
\verse et cupiens saturari de his, quae cadebant de mensa divitis; sed et canes veniebant et lingebant ulcera eius. 
\verse Factum est autem ut moreretur pauper et portaretur ab angelis in sinum Abrahae; mortuus est autem et dives et sepultus est. 
\verse Et in inferno elevans oculos suos, cum esset in tormentis, videbat Abraham a longe et Lazarum in sinu eius. 
\verse Et ipse clamans dixit: “Pater Abraham, miserere mei et mitte Lazarum, ut intingat extremum digiti sui in aquam, ut refrigeret linguam meam, quia crucior in hac flamma”. 
\verse At dixit Abraham: “Fili, recordare quia recepisti bona tua in vita tua, et Lazarus similiter mala; nunc autem hic consolatur, tu vero cruciaris. 
\verse Et in his omnibus inter nos et vos chaos magnum firmatum est, ut hi, qui volunt hinc transire ad vos, non possint, neque inde ad nos transmeare”. 
\verse Et ait: “Rogo ergo te, Pater, ut mittas eum in domum patris mei 
\verse — habeo enim quinque fratres — ut testetur illis, ne et ipsi veniant in locum hunc tormentorum”. 
\verse Ait autem Abraham: “Habent Moysen et Prophetas; audiant illos”. 
\verse At ille dixit: “Non, pater Abraham, sed si quis ex mortuis ierit ad eos, paenitentiam agent”. 
\verse Ait autem illi: “Si Moysen et Prophetas non audiunt, neque si quis ex mortuis resurrexerit, credent”". 
\end{biblechapter}

\begin{biblechapter}  
\verse Et ad discipulos suos ait: “Impossibile est ut non veniant scandala; vae autem illi, per quem veniunt! 
\verse Utilius est illi, si lapis molaris imponatur circa collum eius et proiciatur in mare, quam ut scandalizet unum de pusillis istis. 
\verse Attendite vobis! Si peccaverit frater tuus, increpa illum et, si paenitentiam egerit, dimitte illi; 
\verse et si septies in die peccaverit in te et septies conversus fuerit ad te dicens: “Paenitet me”, dimittes illi". 
\verse Et dixerunt apostoli Domino: “Adauge nobis fidem!". 
\verse Dixit autem Dominus: “Si haberetis fidem sicut granum sinapis, diceretis huic arbori moro: “Eradicare et transplantare in mare”, et oboediret vobis. 
\verse Quis autem vestrum habens servum arantem aut pascentem, qui regresso de agro dicet illi: “Statim transi, recumbe”, 
\verse et non dicet ei: “Para, quod cenem, et praecinge te et ministra mihi, donec manducem et bibam, et post haec tu manducabis et bibes”? 
\verse Numquid gratiam habet servo illi, quia fecit, quae praecepta sunt?  
\verse Sic et vos, cum feceritis omnia, quae praecepta sunt vobis, dicite: “Servi inutiles sumus; quod debuimus facere, fecimus”". 
\verse Et factum est, dum iret in Ierusalem, et ipse transibat per mediam Samariam et Galilaeam. 
\verse Et cum ingrederetur quoddam castellum, occurrerunt ei decem viri leprosi, qui steterunt a longe 
\verse et levaverunt vocem dicentes: “Iesu praeceptor, miserere nostri!". 
\verse Quos ut vidit, dixit: “Ite, ostendite vos sacerdotibus". Et factum est, dum irent, mundati sunt. 
\verse Unus autem ex illis, ut vidit quia sanatus est, regressus est cum magna voce magnificans Deum 
\verse et cecidit in faciem ante pedes eius gratias agens ei; et hic erat Samaritanus. 
\verse Respondens autem Iesus dixit: “Nonne decem mundati sunt? Et novem ubi sunt? 
\verse Non sunt inventi qui redirent, ut darent gloriam Deo, nisi hic alienigena?". 
\verse Et ait illi: “Surge, vade; fides tua te salvum fecit". 
\verse Interrogatus autem a pharisaeis: “Quando venit regnum Dei?", respondit eis et dixit: “Non venit regnum Dei cum observatione, 
\verse neque dicent: “Ecce hic” aut: “Illic”; ecce enim regnum Dei intra vos est". 
\verse Et ait ad discipulos: “Venient dies, quando desideretis videre unum diem Filii hominis et non videbitis. 
\verse Et dicent vobis: “Ecce hic”, “Ecce illic”; nolite ire neque sectemini. 
\verse Nam sicut fulgur coruscans de sub caelo in ea, quae sub caelo sunt, fulget, ita erit Filius hominis in die sua. 
\verse Primum autem oportet illum multa pati et reprobari a generatione hac. 
\verse Et sicut factum est in diebus Noe, ita erit et in diebus Filii hominis: 
\verse edebant, bibebant, uxores ducebant, dabantur ad nuptias, usque in diem, qua intravit Noe in arcam, et venit diluvium et perdidit omnes. 
\verse Similiter sicut factum est in diebus Lot: edebant, bibebant, emebant, vendebant, plantabant, aedificabant;  
\verse qua die autem exiit Lot a Sodomis, pluit ignem et sulphur de caelo et omnes perdidit. 
\verse Secundum haec erit, qua die Filius hominis revelabitur. 
\verse In illa die, qui fuerit in tecto, et vasa eius in domo, ne descendat tollere illa; et, qui in agro, similiter non redeat retro. 
\verse Memores estote uxoris Lot. 
\verse Quicumque quaesierit animam suam salvam facere, perdet illam; et, quicumque perdiderit illam, vivificabit eam. 
\verse Dico vobis: Illa nocte erunt duo in lecto uno: unus assumetur, et alter relinquetur; 
\verse duae erunt molentes in unum: una assumetur, et altera relinquetur". (36) 
\verse Respondentes dicunt illi: “Ubi, Domine?". Qui dixit eis: “Ubicumque fuerit corpus, illuc congregabuntur et aquilae". 
\end{biblechapter}

\begin{biblechapter}  
\verse Dicebat autem parabolam ad illos, quoniam oportet semper orare et non deficere, 
\verse dicens: “Iudex quidam erat in quadam civitate, qui Deum non timebat et hominem non reverebatur. 
\verse Vidua autem erat in civitate illa et veniebat ad eum dicens: “Vindica me de adversario meo”. 
\verse Et nolebat per multum tempus; post haec autem dixit intra se: “Etsi Deum non timeo nec hominem revereor, 
\verse tamen quia molesta est mihi haec vidua, vindicabo illam, ne in novissimo veniens suggillet me”". 
\verse Ait autem Dominus: “Audite quid iudex iniquitatis dicit; 
\verse Deus autem non faciet vindictam electorum suorum clamantium ad se die ac nocte, et patientiam habebit in illis? 
\verse Dico vobis: Cito faciet vindictam illorum. Verumtamen Filius hominis veniens, putas, inveniet fidem in terra?". 
\verse Dixit autem et ad quosdam, qui in se confidebant tamquam iusti et aspernabantur ceteros, parabolam istam: 
\verse “Duo homines ascenderunt in templum, ut orarent: unus pharisaeus et alter publicanus. 
\verse Pharisaeus stans haec apud se orabat: “Deus, gratias ago tibi, quia non sum sicut ceteri hominum, raptores, iniusti, adulteri, velut etiam hic publicanus; 
\verse ieiuno bis in sabbato, decimas do omnium, quae possideo”. 
\verse Et publicanus a longe stans nolebat nec oculos ad caelum levare, sed percutiebat pectus suum dicens: “Deus, propitius esto mihi peccatori”. 
\verse Dico vobis: Descendit hic iustificatus in domum suam ab illo. Quia omnis, qui se exaltat, humiliabitur; et, qui se humiliat, exaltabitur". 
\verse Afferebant autem ad illum et infantes, ut eos tangeret; quod cum viderent, discipuli increpabant illos. 
\verse Iesus autem convocans illos dixit: “Sinite pueros venire ad me et nolite eos vetare; talium est enim regnum Dei. 
\verse Amen dico vobis: Quicumque non acceperit regnum Dei sicut puer, non intrabit in illud". 
\verse Et interrogavit eum quidam princeps dicens: “Magister bone, quid faciens vitam aeternam possidebo?". 
\verse Dixit autem ei Iesus: “Quid me dicis bonum? Nemo bonus nisi solus Deus. 
\verse Mandata nosti: non moechaberis, non occides, non furtum facies, non falsum testimonium dices, honora patrem tuum et matrem". 
\verse Qui ait: “Haec omnia custodivi a iuventute". 
\verse Quo audito, Iesus ait ei: “Adhuc unum tibi deest: omnia, quaecumque habes, vende et da pauperibus et habebis thesaurum in caelo: et veni, sequere me". 
\verse His ille auditis, contristatus est, quia dives erat valde. 
\verse Videns autem illum Iesus tristem factum dixit: “Quam difficile, qui pecunias habent, in regnum Dei intrant.  
\verse Facilius est enim camelum per foramen acus transire, quam divitem intrare in regnum Dei". 
\verse Et dixerunt, qui audiebant: “Et quis potest salvus fieri?". 
\verse Ait autem illis: “Quae impossibilia sunt apud homi nes, possibilia sunt apud Deum". 
\verse Ait autem Petrus: “Ecce nos dimisimus nostra et secuti sumus te". 
\verse Qui dixit eis: “Amen dico vobis: Nemo est, qui reliquit domum aut uxorem aut fratres aut parentes aut filios propter regnum Dei, 
\verse et non recipiat multo plura in hoc tempore et in saeculo venturo vitam aeternam". 
\verse Assumpsit autem Duodecim et ait illis: “Ecce ascendimus Ierusalem, et consummabuntur omnia, quae scripta sunt per Prophetas de Filio hominis: 
\verse tradetur enim gentibus et illudetur et contumeliis afficietur et conspuetur;  
\verse et, postquam flagellaverint, occident eum, et die tertia resurget". 
\verse Et ipsi nihil horum intellexerunt; et erat verbum istud absconditum ab eis, et non intellegebant, quae dicebantur. 
\verse Factum est autem, cum appropinquaret Iericho, caecus quidam sedebat secus viam mendicans. 
\verse Et cum audiret turbam praetereuntem, interrogabat quid hoc esset. 
\verse Dixerunt autem ei: “Iesus Nazarenus transit". 
\verse Et clamavit dicens: “Iesu, fili David, miserere mei!". 
\verse Et qui praeibant, increpabant eum, ut taceret; ipse vero multo magis clamabat: “Fili David, miserere mei!". 
\verse Stans autem Iesus iussit illum adduci ad se. Et cum appropinquasset, interrogavit illum: 
\verse “Quid tibi vis faciam?". At ille dixit: “Domine, ut videam". 
\verse Et Iesus dixit illi: “Respice! Fides tua te salvum fecit". 
\verse Et confestim vidit et sequebatur illum magnificans Deum. Et omnis plebs, ut vidit, dedit laudem Deo. 
\end{biblechapter}

\begin{biblechapter}  
\verse Et ingressus perambulabat Iericho. 
\verse Et ecce vir nomine Zacchaeus, et hic erat princeps publicanorum et ipse dives. 
\verse Et quaerebat videre Iesum, quis esset, et non poterat prae turba, quia statura pusillus erat. 
\verse Et praecurrens ascendit in arborem sycomorum, ut videret illum, quia inde erat transiturus. 
\verse Et cum venisset ad locum, suspiciens Iesus dixit ad eum: “Zacchaee, festinans descende, nam hodie in domo tua oportet me manere". 
\verse Et festinans descendit et excepit illum gaudens. 
\verse Et cum viderent, omnes murmurabant dicentes: “Ad hominem peccatorem divertit!". 
\verse Stans autem Zacchaeus dixit ad Dominum: “Ecce dimidium bonorum meorum, Domine, do pauperibus et, si quid aliquem defraudavi, reddo quadruplum". 
\verse Ait autem Iesus ad eum: “Hodie salus domui huic facta est, eo quod et ipse filius sit Abrahae; 
\verse venit enim Filius hominis quaerere et salvum facere, quod perierat". 
\verse Haec autem illis audientibus, adiciens dixit parabolam, eo quod esset prope Ierusalem, et illi existimarent quod confestim regnum Dei manifestaretur. 
\verse Dixit ergo: “Homo quidam nobilis abiit in regionem longinquam accipere sibi regnum et reverti. 
\verse Vocatis autem decem servis suis, dedit illis decem minas et ait ad illos: “Negotiamini, dum venio”. 
\verse Cives autem eius oderant illum et miserunt legationem post illum dicentes: “Nolumus hunc regnare super nos!”. 
\verse Et factum est ut rediret, accepto regno, et iussit ad se vocari servos illos, quibus dedit pecuniam, ut sciret quantum negotiati essent. 
\verse Venit autem primus dicens: “Domine, mina tua decem minas acquisivit”. 
\verse Et ait illi: “Euge, bone serve; quia in modico fidelis fuisti, esto potestatem habens supra decem civitates”. 
\verse Et alter venit dicens: “Mina tua, domine, fecit quinque minas”. 
\verse Et huic ait: “Et tu esto supra quinque civitates”. 
\verse Et alter venit dicens: “Domine, ecce mina tua, quam habui repositam in sudario; 
\verse timui enim te, quia homo austerus es: tollis, quod non posuisti, et metis, quod non seminasti”. 
\verse Dicit ei: “De ore tuo te iudico, serve nequam! Sciebas quod ego austerus homo sum, tollens quod non posui et metens quod non seminavi?  
\verse Et quare non dedisti pecuniam meam ad mensam? Et ego veniens cum usuris utique exegissem illud”. 
\verse Et adstantibus dixit: “Auferte ab illo minam et date illi, qui decem minas habet”. 
\verse Et dixerunt ei: “Domine, habet decem minas!”. 
\verse Dico vobis: “Omni habenti dabitur; ab eo autem, qui non habet, et, quod habet, auferetur. 
\verse Verumtamen inimicos meos illos, qui noluerunt me regnare super se, adducite huc et interficite ante me!". 
\verse Et his dictis, praecedebat ascendens Hierosolymam. 
\verse Et factum est, cum appropinquasset ad Bethfage et Bethaniam, ad montem, qui vocatur Oliveti, misit duos discipulos 
\verse dicens: “Ite in castellum, quod contra est, in quod introeuntes invenietis pullum asinae alligatum, cui nemo umquam hominum sedit; solvite illum et adducite. 
\verse Et si quis vos interrogaverit: “Quare solvitis?”, sic dicetis: “Dominus eum necessarium habet”". 
\verse Abierunt autem, qui missi erant, et invenerunt, sicut dixit illis.  
\verse Solventibus autem illis pullum, dixerunt domini eius ad illos: “Quid solvitis pullum?". 
\verse At illi dixerunt: “Dominus eum necessarium habet".  
\verse Et duxerunt illum ad Iesum; et iactantes vestimenta sua supra pullum, imposuerunt Iesum. 
\verse Eunte autem illo, substernebant vestimenta sua in via.  
\verse Et cum appropinquaret iam ad descensum montis Oliveti, coeperunt omnis multitudo discipulorum gaudentes laudare Deum voce magna super omnibus, quas viderant, virtutibus 
\verse dicentes: “Benedictus, qui venit rex in nomine Domini! Pax in caelo, et gloria in excelsis!". 
\verse Et quidam pharisaeorum de turbis dixerunt ad illum: “Magister, increpa discipulos tuos!". 
\verse Et respondens dixit: “Dico vobis: Si hi tacuerint, lapides clamabunt!". 
\verse Et ut appropinquavit, videns civitatem flevit super illam 
\verse dicens: “Si cognovisses et tu in hac die, quae ad pacem tibi! Nunc autem abscondita sunt ab oculis tuis. 
\verse Quia venient dies in te, et circumdabunt te inimici tui vallo et obsidebunt te et coangustabunt te undique 
\verse et ad terram prosternent te et filios tuos, qui in te sunt, et non relinquent in te lapidem super lapidem, eo quod non cognoveris tempus visitationis tuae". 
\verse Et ingressus in templum, coepit eicere vendentes 
\verse dicens illis: “Scriptum est: “Et erit domus mea domus orationis”. Vos autem fecistis illam speluncam latronum". 
\verse Et erat docens cotidie in templo. Principes autem sacerdotum et scribae et principes plebis quaerebant illum perdere 
\verse et non inveniebant quid facerent; omnis enim populus suspensus erat audiens illum. 
\end{biblechapter}

\begin{biblechapter}  
\verse Et factum est in una dierum, docente illo populum in templo et evangelizante, supervenerunt principes sacerdotum et scribae cum senioribus  
\verse et aiunt dicentes ad illum: “Dic nobis: In qua potestate haec facis, aut quis est qui dedit tibi hanc potestatem?". 
\verse Respondens autem dixit ad illos: “Interrogabo vos et ego verbum; et dicite mihi: 
\verse Baptismum Ioannis de caelo erat an ex hominibus?". 
\verse At illi cogitabant inter se dicentes: “Si dixerimus: “De caelo”, dicet: “Quare non credidistis illi?; 
\verse si autem dixerimus: “Ex hominibus”, plebs universa lapidabit nos; certi sunt enim Ioannem prophetam esse". 
\verse Et responderunt se nescire unde esset. 
\verse Et Iesus ait illis: “Neque ego dico vobis in qua potestate haec facio". 
\verse Coepit autem dicere ad plebem parabolam hanc: “Homo plantavit vineam et locavit eam colonis et ipse peregre fuit multis temporibus. 
\verse Et in tempore misit ad cultores servum, ut de fructu vineae darent illi; cultores autem caesum dimiserunt eum inanem. 
\verse Et addidit alterum servum mittere; illi autem hunc quoque caedentes et afficientes contumelia dimiserunt inanem. 
\verse Et addidit tertium mittere; qui et illum vulnerantes eiecerunt. 
\verse Dixit autem dominus vineae: “Quid faciam? Mittam filium meum dilectum; forsitan hunc verebuntur”. 
\verse Quem cum vidissent coloni, cogitaverunt inter se dicentes: “Hic est heres. Occidamus illum, ut nostra fiat hereditas”. 
\verse Et eiectum illum extra vineam occiderunt. Quid ergo faciet illis dominus vineae? 
\verse Veniet et perdet colonos istos et dabit vineam aliis". Quo audito, dixerunt: “Absit!". 
\verse Ille autem aspiciens eos ait: “Quid est ergo hoc, quod scriptum est: “Lapidem quem reprobaverunt aedificantes, hic factus est in caput anguli”? 
\verse Omnis, qui ceciderit supra illum lapidem, conquassabitur; supra quem autem ceciderit, comminuet illum". 
\verse Et quaerebant scribae et principes sacerdotum mittere in illum manus in illa hora et timuerunt populum; cognoverunt enim quod ad ipsos dixerit similitudinem istam. 
\verse Et observantes miserunt insidiatores, qui se iustos simularent, ut caperent eum in sermone, et sic traderent illum principatui et potestati praesidis.  
\verse Et interrogaverunt illum dicentes: “Magister, scimus quia recte dicis et doces et non accipis personam, sed in veritate viam Dei doces. 
\verse Licet nobis dare tributum Caesari an non?". 
\verse Considerans autem dolum illorum dixit ad eos: 
\verse “Ostendite mihi denarium. Cuius habet imaginem et inscriptionem?".  
\verse At illi dixerunt: “Caesaris". Et ait illis: “Reddite ergo, quae Caesaris sunt, Caesari et, quae Dei sunt, Deo". 
\verse Et non potuerunt verbum eius reprehendere coram plebe et mirati in responso eius tacuerunt. 
\verse Accesserunt autem quidam sadducaeorum, qui negant esse resurrectionem, et interrogaverunt eum 
\verse dicentes: “Magister, Moyses scripsit nobis, si frater alicuius mortuus fuerit habens uxorem et hic sine filiis fuerit, ut accipiat eam frater eius uxorem et suscitet semen fratri suo. 
\verse Septem ergo fratres erant: et primus accepit uxorem et mortuus est sine filiis; 
\verse et sequens 
\verse et tertius accepit illam, similiter autem et septem non reliquerunt filios et mortui sunt. 
\verse Novissima mortua est et mulier. 
\verse Mulier ergo in resurrectione cuius eorum erit uxor? Si quidem septem habuerunt eam uxorem". 
\verse Et ait illis Iesus: “Filii saeculi huius nubunt et traduntur ad nuptias; 
\verse illi autem, qui digni habentur saeculo illo et resurrectione ex mortuis, neque nubunt neque ducunt uxores. 
\verse Neque enim ultra mori possunt: aequales enim angelis sunt et filii sunt Dei, cum sint filii resurrectionis. 
\verse Quia vero resurgant mortui, et Moyses ostendit secus rubum, sicut dicit: “Dominum Deum Abraham et Deum Isaac et Deum Iacob”. 
\verse Deus autem non est mortuorum sed vivorum: omnes enim vivunt ei". 
\verse Respondentes autem quidam scribarum dixerunt: “Magister, bene dixisti". 
\verse Et amplius non audebant eum quidquam interrogare. 
\verse Dixit autem ad illos: “Quomodo dicunt Christum filium David esse? 
\verse Ipse enim David dicit in libro Psalmorum: “Dixit Dominus Domino meo: Sede a dextris meis, 
\verse donec ponam inimicos tuos scabellum pedum tuorum”. 
\verse David ergo Dominum illum vocat; et quomodo filius eius est?". 
\verse Audiente autem omni populo, dixit discipulis suis: 
\verse “Attendite a scribis, qui volunt ambulare in stolis et amant salutationes in foro et primas cathedras in synagogis et primos discubitus in conviviis, 
\verse qui devorant domos viduarum et simulant longam orationem. Hi accipient damnationem maiorem". 
\end{biblechapter}

\begin{biblechapter}  
\verse Respiciens autem vidit eos, qui mittebant munera sua in gazophylacium, divites. 
\verse Vidit autem quandam viduam pauperculam mittentem illuc minuta duo  
\verse et dixit: “Vere dico vobis: Vidua haec pauper plus quam omnes misit. 
\verse Nam omnes hi ex abundantia sua miserunt in munera; haec autem ex inopia sua omnem victum suum, quem habebat, misit". 
\verse Et quibusdam dicentibus de templo, quod lapidibus bonis et donis ornatum, esset dixit: 
\verse “Haec quae videtis, venient dies, in quibus non relinquetur lapis super lapidem, qui non destruatur". 
\verse Interrogaverunt autem illum dicentes: “Praeceptor, quando ergo haec erunt, et quod signum, cum fieri incipient?". 
\verse Qui dixit: “Videte, ne seducamini. Multi enim venient in nomine meo dicentes: “Ego sum” et: “Tempus appropinquavit”. Nolite ergo ire post illos. 
\verse Cum autem audieritis proelia et seditiones, nolite terreri; oportet enim primum haec fieri, sed non statim finis". 
\verse Tunc dicebat illis: “Surget gens contra gentem, et regnum adversus regnum; 
\verse et terrae motus magni et per loca fames et pestilentiae erunt, terroresque et de caelo signa magna erunt. 
\verse Sed ante haec omnia inicient vobis manus suas et persequentur tradentes in synagogas et custodias, et trahemini ad reges et praesides propter nomen meum; 
\verse continget autem vobis in testimonium. 
\verse Ponite ergo in cordibus vestris non praemeditari quemadmodum respondeatis; 
\verse ego enim dabo vobis os et sapientiam, cui non poterunt resistere vel contradicere omnes adversarii vestri. 
\verse Trademini autem et a parentibus et fratribus et cognatis et amicis, et morte afficient ex vobis, 
\verse et eritis odio omnibus propter nomen meum. 
\verse Et capillus de capite vestro non peribit. 
\verse In patientia vestra possidebitis animas vestras. 
\verse Cum autem videritis circumdari ab exercitu Ierusalem, tunc scitote quia appropinquavit desolatio eius. 
\verse Tunc, qui in Iudaea sunt, fugiant in montes; et, qui in medio eius, discedant; et, qui in regionibus, non intrent in eam. 
\verse Quia dies ultionis hi sunt, ut impleantur omnia, quae scripta sunt.  
\verse Vae autem praegnantibus et nutrientibus in illis diebus! Erit enim pressura magna super terram et ira populo huic, 
\verse et cadent in ore gladii et captivi ducentur in omnes gentes, et Ierusalem calcabitur a gentibus, donec impleantur tempora nationum. 
\verse Et erunt signa in sole et luna et stellis, et super terram pressura gentium prae confusione sonitus maris et fluctuum, 
\verse arescentibus hominibus prae timore et exspectatione eorum, quae supervenient orbi, nam virtutes caelorum movebuntur. 
\verse Et tunc videbunt Filium hominis venientem in nube cum potestate et gloria magna. 
\verse His autem fieri incipientibus, respicite et levate capita vestra, quoniam appropinquat redemptio vestra". 
\verse Et dixit illis similitudinem: “Videte ficulneam et omnes arbores: 
\verse cum iam germinaverint, videntes vosmetipsi scitis quia iam prope est aestas. 
\verse Ita et vos, cum videritis haec fieri, scitote quoniam prope est regnum Dei.  
\verse Amen dico vobis: Non praeteribit generatio haec, donec omnia fiant. 
\verse Caelum et terra transibunt, verba autem mea non transibunt. 
\verse Attendite autem vobis, ne forte graventur corda vestra in crapula et ebrietate et curis huius vitae, et superveniat in vos repentina dies illa;  
\verse tamquam laqueus enim superveniet in omnes, qui sedent super faciem omnis terrae. 
\verse Vigilate itaque omni tempore orantes, ut possitis fugere ista omnia, quae futura sunt, et stare ante Filium hominis". 
\verse Erat autem diebus docens in templo, noctibus vero exiens morabatur in monte, qui vocatur Oliveti. 
\verse Et omnis populus manicabat ad eum in templo audire eum. 
\end{biblechapter}

\begin{biblechapter}  
\verse Appropinquabat autem dies festus Azymorum, qui dicitur Pascha. 
\verse Et quaerebant principes sacerdotum et scribae quomodo eum interficerent; timebant vero plebem. 
\verse Intravit autem Satanas in Iudam, qui cognominabatur Iscarioth, unum de Duodecim; 
\verse et abiit et locutus est cum principibus sacerdotum et magistratibus, quemadmodum illum traderet eis. 
\verse Et gavisi sunt et pacti sunt pecuniam illi dare. 
\verse Et spopondit et quaerebat opportunitatem, ut eis traderet illum sine turba. 
\verse Venit autem dies Azymorum, in qua necesse erat occidi Pascha. 
\verse Et misit Petrum et Ioannem dicens: “Euntes parate nobis Pascha, ut manducemus". 
\verse At illi dixerunt ei: “Ubi vis paremus?". 
\verse Et dixit ad eos: “Ecce, introeuntibus vobis in civitatem, occurret vobis homo amphoram aquae portans; sequimini eum in domum, in quam intrat. 
\verse Et dicetis patri familias domus: “Dicit tibi Magister: Ubi est deversorium, ubi Pascha cum discipulis meis manducem?”. 
\verse Ipse vobis ostendet cenaculum magnum stratum; ibi parate". 
\verse Euntes autem invenerunt, sicut dixit illis, et paraverunt Pascha. 
\verse Et cum facta esset hora, discubuit, et apostoli cum eo. 
\verse Et ait illis: “Desiderio desideravi hoc Pascha manducare vobiscum, antequam patiar. 
\verse Dico enim vobis: Non manducabo illud, donec impleatur in regno Dei". 
\verse Et accepto calice, gratias egit et dixit: “Accipite hoc et dividite inter vos.  
\verse Dico enim vobis: Non bibam amodo de generatione vitis, donec regnum Dei veniat". 
\verse Et accepto pane, gratias egit et fregit et dedit eis dicens: “Hoc est corpus meum, quod pro vobis datur. Hoc facite in meam commemorationem". 
\verse Similiter et calicem, postquam cenavit, dicens: “Hic calix novum testamentum est in sanguine meo, qui pro vobis funditur. 
\verse Verumtamen ecce manus tradentis me mecum est in mensa; 
\verse et quidem Filius hominis, secundum quod definitum est, vadit; verumtamen vae illi homini, per quem traditur!". 
\verse Et ipsi coeperunt quaerere inter se, quis esset ex eis, qui hoc facturus esset. 
\verse Facta est autem et contentio inter eos, quis eorum videretur esse maior.  
\verse Dixit autem eis: “Reges gentium dominantur eorum; et, qui potestatem habent super eos, benefici vocantur. 
\verse Vos autem non sic, sed qui maior est in vobis, fiat sicut iunior; et, qui praecessor est, sicut ministrator. 
\verse Nam quis maior est: qui recumbit, an qui ministrat? Nonne qui recumbit? Ego autem in medio vestrum sum, sicut qui ministrat. 
\verse Vos autem estis, qui permansistis mecum in tentationibus meis; 
\verse et ego dispono vobis, sicut disposuit mihi Pater meus regnum, 
\verse ut edatis et bibatis super mensam meam in regno meo et sedeatis super thronos iudicantes duodecim tribus Israel. 
\verse Simon, Simon, ecce Satanas expetivit vos, ut cribraret sicut triticum; 
\verse ego autem rogavi pro te, ut non deficiat fides tua. Et tu, aliquando conversus, confirma fratres tuos". 
\verse Qui dixit ei: “Domine, tecum paratus sum et in carcerem et in mortem ire". 
\verse Et ille dixit: “Dico tibi, Petre, non cantabit hodie gallus, donec ter abneges nosse me". 
\verse Et dixit eis: “Quando misi vos sine sacculo et pera et calceamentis, numquid aliquid defuit vobis?". At illi dixerunt: “Nihil". 
\verse Dixit ergo eis: “Sed nunc, qui habet sacculum, tollat, similiter et peram; et, qui non habet, vendat tunicam suam et emat gladium. 
\verse Dico enim vobis: Hoc, quod scriptum est, oportet impleri in me, illud: “Cum iniustis deputatus est”. Etenim ea, quae sunt de me, adimpletionem habent". 
\verse At illi dixerunt: “Domine, ecce gladii duo hic". At ille dixit eis: “Satis est". 
\verse Et egressus ibat secundum consuetudinem in montem Olivarum; secuti sunt autem illum et discipuli. 
\verse Et cum pervenisset ad locum, dixit illis: “Orate, ne intretis in tentationem". 
\verse Et ipse avulsus est ab eis, quantum iactus est lapidis, et, positis genibus, orabat 
\verse dicens: “Pater, si vis, transfer calicem istum a me; verumtamen non mea voluntas sed tua fiat". 
\verse Apparuit autem illi angelus de caelo confortans eum. Et factus in agonia prolixius orabat. 
\verse Et factus est sudor eius sicut guttae sanguinis decurrentis in terram. 
\verse Et cum surrexisset ab oratione et venisset ad discipulos, invenit eos dormientes prae tristitia 
\verse et ait illis: “Quid dormitis? Surgite; orate, ne intretis in tentationem". 
\verse Adhuc eo loquente, ecce turba; et, qui vocabatur Iudas, unus de Duodecim, antecedebat eos et appropinquavit Iesu, ut oscularetur eum. 
\verse Iesus autem dixit ei: “Iuda, osculo Filium hominis tradis?". 
\verse Videntes autem hi, qui circa ipsum erant, quod futurum erat, dixerunt: “Domine, si percutimus in gladio?". 
\verse Et percussit unus ex illis servum principis sacerdotum et amputavit auriculam eius dextram. 
\verse Respondens autem Iesus ait: “Sinite usque huc!". Et cum tetigisset auriculam eius, sanavit eum. 
\verse Dixit autem Iesus ad eos, qui venerant ad se principes sacerdotum et magistratus templi et seniores: “Quasi ad latronem existis cum gladiis et fustibus? 
\verse Cum cotidie vobiscum fuerim in templo, non extendistis manus in me; sed haec est hora vestra et potestas tenebrarum". 
\verse Comprehendentes autem eum, duxerunt et introduxerunt in domum principis sacerdotum. Petrus vero sequebatur a longe. 
\verse Accenso autem igni in medio atrio et circumsedentibus illis, sedebat Petrus in medio eorum. 
\verse Quem cum vidisset ancilla quaedam sedentem ad lumen et eum fuisset intuita, dixit: 
\verse “Et hic cum illo erat!". At ille negavit eum dicens: 
\verse “Mulier, non novi illum!". Et post pusillum alius videns eum dixit: “Et tu de illis es!". Petrus vero ait: “O homo, non sum!". 
\verse Et intervallo facto quasi horae unius, alius quidam affirmabat dicens: “Vere et hic cum illo erat, nam et Galilaeus est!". 
\verse Et ait Petrus: “Homo, nescio quid dicis!". Et continuo adhuc illo loquente cantavit gallus. 
\verse Et conversus Dominus respexit Petrum; et recordatus est Petrus verbi Domini, sicut dixit ei: “Priusquam gallus cantet hodie, ter me negabis". 
\verse Et egressus foras flevit amare. 
\verse Et viri, qui tenebant illum, illudebant ei caedentes; 
\verse et velaverunt eum et interrogabant eum dicentes: “Prophetiza: Quis est, qui te percussit?".  
\verse Et alia multa blasphemantes dicebant in eum. 
\verse Et ut factus est dies, convenerunt seniores plebis et principes sacerdotum et scribae et duxerunt illum in concilium suum 
\verse dicentes: “Si tu es Christus, dic nobis". Et ait illis: “Si vobis dixero, non credetis; 
\verse si autem interrogavero, non respondebitis mihi. 
\verse Ex hoc autem erit Filius hominis sedens a dextris virtutis Dei". 
\verse Dixerunt autem omnes: “Tu ergo es Filius Dei?". Qui ait ad illos: “Vos dicitis quia ego sum". 
\verse At illi dixerunt: “Quid adhuc desideramus testimonium? Ipsi enim audivimus de ore eius!". 
\end{biblechapter}

\begin{biblechapter}  
\verse Et surgens omnis multitudo eorum duxerunt illum ad Pilatum. 
\verse Coeperunt autem accusare illum dicentes: “Hunc invenimus subvertentem gentem nostram et prohibentem tributa dare Caesari et dicentem se Christum regem esse". 
\verse Pilatus autem interrogavit eum dicens: “Tu es rex Iudaeorum?". At ille respondens ait: “Tu dicis". 
\verse Ait autem Pilatus ad principes sacerdotum et turbas: “Nihil invenio causae in hoc homine". 
\verse At illi invalescebant dicentes: “Commovet populum docens per universam Iudaeam et in cipiens a Galilaea usque huc!". 
\verse Pilatus autem audiens interrogavit si homo Galilaeus esset; 
\verse et ut cognovit quod de Herodis potestate esset, remisit eum ad Herodem, qui et ipse Hierosolymis erat illis diebus. 
\verse Herodes autem, viso Iesu, gavisus est valde; erat enim cupiens ex multo tempore videre eum, eo quod audiret de illo et sperabat signum aliquod videre ab eo fieri. 
\verse Interrogabat autem illum multis sermonibus; at ipse nihil illi respondebat. 
\verse Stabant etiam principes sacerdotum et scribae constanter accusantes eum. 
\verse Sprevit autem illum Herodes cum exercitu suo et illusit indutum veste alba et remisit ad Pilatum. 
\verse Facti sunt autem amici inter se Herodes et Pilatus in ipsa die; nam antea inimici erant ad invicem. 
\verse Pilatus autem, convocatis principibus sacerdotum et magistratibus et plebe, 
\verse dixit ad illos: “Obtulistis mihi hunc hominem quasi avertentem populum, et ecce ego coram vobis interrogans nullam causam inveni in homine isto ex his, in quibus eum accusatis, 
\verse sed neque Herodes; remisit enim illum ad nos. Et ecce nihil dignum morte actum est ei. 
\verse Emendatum ergo illum dimittam". (17) 
\verse Exclamavit autem universa turba dicens: “Tolle hunc et dimitte nobis Barabbam!", 
\verse qui erat propter seditionem quandam factam in civitate et homicidium missus in carcerem. 
\verse Iterum autem Pilatus locutus est ad illos volens dimittere Iesum, 
\verse at illi succlamabant dicentes: “Crucifige, crucifige illum!". 
\verse Ille autem tertio dixit ad illos: “Quid enim mali fecit iste? Nullam causam mortis invenio in eo; corripiam ergo illum et dimittam". 
\verse At illi instabant vocibus magnis postulantes, ut crucifigeretur, et invalescebant voces eorum. 
\verse Et Pilatus adiudicavit fieri petitionem eorum: 
\verse dimisit autem eum, qui propter seditionem et homicidium missus fuerat in carcerem, quem petebant; Iesum vero tradidit voluntati eorum. 
\verse Et cum abducerent eum, apprehenderunt Simonem quendam Cyrenensem venientem de villa et imposuerunt illi crucem portare post Iesum. 
\verse Sequebatur autem illum multa turba populi et mulierum, quae plangebant et lamentabant eum. 
\verse Conversus autem ad illas Iesus dixit: “Filiae Ierusalem, nolite flere super me, sed super vos ipsas flete et super filios vestros, 
\verse quoniam ecce venient dies, in quibus dicent: “Beatae steriles et ventres, qui non genuerunt, et ubera, quae non lactaverunt!”. 
\verse Tunc incipient dicere montibus: “Cadite super nos!”, et collibus: “Operite nos!”, 
\verse quia si in viridi ligno haec faciunt, in arido quid fiet?". 
\verse Ducebantur autem et alii duo nequam cum eo, ut interficerentur. 
\verse Et postquam venerunt in locum, qui vocatur Calvariae, ibi crucifixerunt eum et latrones, unum a dextris et alterum a sinistris. 
\verse Iesus autem dicebat: “Pater, dimitte illis, non enim sciunt quid faciunt". Dividentes vero vestimenta eius miserunt sortes. 
\verse Et stabat populus exspectans. Et deridebant illum et principes dicentes: “Alios salvos fecit; se salvum faciat, si hic est Christus Dei electus!". 
\verse Illudebant autem ei et milites accedentes, acetum offerentes illi 
\verse et dicentes: “Si tu es rex Iudaeorum, salvum te fac!". 
\verse Erat autem et superscriptio super illum: “Hic est rex Iudaeorum". 
\verse Unus autem de his, qui pendebant, latronibus blasphemabat eum dicens: “Nonne tu es Christus? Salvum fac temetipsum et nos!". 
\verse Respondens autem alter increpabat illum dicens: “Neque tu times Deum, quod in eadem damnatione es?  
\verse Et nos quidem iuste, nam digna factis recipimus! Hic vero nihil mali gessit". 
\verse Et dicebat: “Iesu, memento mei, cum veneris in regnum tuum". 
\verse Et dixit illi: “Amen dico tibi: Hodie mecum eris in paradiso". 
\verse Et erat iam fere hora sexta, et tenebrae factae sunt in universa terra usque in horam nonam, 
\verse et obscuratus est sol, et velum templi scissum est medium. 
\verse Et clamans voce magna Iesus ait: “Pater, in manus tuas commendo spiritum meum"; et haec dicens exspiravit. 
\verse Videns autem centurio, quod factum fuerat, glorificavit Deum dicens: “Vere hic homo iustus erat!". 
\verse Et omnis turba eorum, qui simul aderant ad spectaculum istud et videbant, quae fiebant, percutientes pectora sua revertebantur. 
\verse Stabant autem omnes noti eius a longe et mulieres, quae secutae erant eum a Galilaea, haec videntes. 
\verse Et ecce vir nomine Ioseph, qui erat decurio, vir bonus et iustus 
\verse Hic non consenserat consilio et actibus eorum — ab Arimathaea civitate Iudaeorum, qui exspectabat regnum Dei, 
\verse hic accessit ad Pilatum et petiit corpus Iesu  
\verse et depositum involvit sindone et posuit eum in monumento exciso, in quo nondum quisquam positus fuerat. 
\verse Et dies erat Parasceves, et sabbatum illucescebat. 
\verse Subsecutae autem mulieres, quae cum ipso venerant de Galilaea, viderunt monumentum et quemadmodum positum erat corpus eius; 
\verse et revertentes paraverunt aromata et unguenta et sabbato quidem siluerunt secundum mandatum. 
\end{biblechapter}

\begin{biblechapter}  
\verse Prima autem sabbatorum, valde diluculo venerunt ad monumentum portantes, quae paraverant, aromata. 
\verse Et invenerunt lapidem revolutum a monumento;  
\verse et ingressae non invenerunt corpus Domini Iesu. 
\verse Et factum est, dum mente haesitarent de isto, ecce duo viri steterunt secus illas in veste fulgenti.  
\verse Cum timerent autem et declinarent vultum in terram, dixerunt ad illas: “Quid quaeritis viventem cum mortuis? 
\verse Non est hic, sed surrexit. Recordamini qualiter locutus est vobis, cum adhuc in Galilaea esset, 
\verse dicens: “Oportet Filium hominis tradi in manus hominum peccatorum et crucifigi et die tertia resurgere”". 
\verse Et recordatae sunt verborum eius 
\verse et regressae a monumento nuntiaverunt haec omnia illis Undecim et ceteris omnibus. 
\verse Erat autem Maria Magdalene et Ioanna et Maria Iacobi; et ceterae cum eis dicebant ad apostolos haec. 
\verse Et visa sunt ante illos sicut deliramentum verba ista, et non credebant illis. 
\verse Petrus autem surgens cucurrit ad monumentum et procumbens videt linteamina sola; et rediit ad sua mirans, quod factum fuerat. 
\verse Et ecce duo ex illis ibant ipsa die in castellum, quod erat in spatio stadiorum sexaginta ab Ierusalem nomine Emmaus; 
\verse et ipsi loquebantur ad invicem de his omnibus, quae acciderant. 
\verse Et factum est, dum fabularentur et secum quaererent, et ipse Iesus appropinquans ibat cum illis; 
\verse oculi autem illorum tenebantur, ne eum agnoscerent. 
\verse Et ait ad illos: “Qui sunt hi sermones, quos confertis ad invicem ambulantes?". Et steterunt tristes.  
\verse Et respondens unus, cui nomen Cleopas, dixit ei: “Tu solus peregrinus es in Ierusalem et non cognovisti, quae facta sunt in illa his diebus?". 
\verse Quibus ille dixit: “Quae?". Et illi dixerunt ei: “De Iesu Nazareno, qui fuit vir propheta, potens in opere et sermone coram Deo et omni populo; 
\verse et quomodo eum tradiderunt summi sacerdotes et principes nostri in damnationem mortis et crucifixerunt eum. 
\verse Nos autem sperabamus, quia ipse esset redempturus Israel; at nunc super haec omnia tertia dies hodie quod haec facta sunt. 
\verse Sed et mulieres quaedam ex nostris terruerunt nos, quae ante lucem fuerunt ad monumentum 
\verse et, non invento corpore eius, venerunt dicentes se etiam visionem angelorum vidisse, qui dicunt eum vivere. 
\verse Et abierunt quidam ex nostris ad monumentum et ita invenerunt, sicut mulieres dixerunt, ipsum vero non viderunt". 
\verse Et ipse dixit ad eos: “O stulti et tardi corde ad credendum in omnibus, quae locuti sunt Prophetae! 
\verse Nonne haec oportuit pati Christum et intrare in gloriam suam?". 
\verse Et incipiens a Moyse et omnibus Prophetis interpretabatur illis in omnibus Scripturis, quae de ipso erant. 
\verse Et appropinquaverunt castello, quo ibant, et ipse se finxit longius ire. 
\verse Et coegerunt illum dicentes: “Mane nobiscum, quoniam advesperascit, et inclinata est iam dies". Et intravit, ut maneret cum illis. 
\verse Et factum est, dum recumberet cum illis, accepit panem et benedixit ac fregit et porrigebat illis.  
\verse Et aperti sunt oculi eorum, et cognoverunt eum; et ipse evanuit ab eis.  
\verse Et dixerunt ad invicem: “Nonne cor nostrum ardens erat in nobis, dum loqueretur nobis in via et aperiret nobis Scripturas?". 
\verse Et surgentes eadem hora regressi sunt in Ierusalem et invenerunt congregatos Undecim et eos, qui cum ipsis erant, 
\verse dicentes: “Surrexit Dominus vere et apparuit Simoni". 
\verse Et ipsi narrabant, quae gesta erant in via, et quomodo cognoverunt eum in fractione panis. 
\verse Dum haec autem loquuntur, ipse stetit in medio eorum et dicit eis: “Pax vobis!". 
\verse Conturbati vero et conterriti existimabant se spiritum videre.  
\verse Et dixit eis: “Quid turbati estis, et quare cogitationes ascendunt in corda vestra? 
\verse Videte manus meas et pedes meos, quia ipse ego sum! Palpate me et videte, quia spiritus carnem et ossa non habet, sicut me videtis habere".  
\verse Et cum hoc dixisset, ostendit eis manus et pedes. 
\verse Adhuc autem illis non credentibus prae gaudio et mirantibus, dixit eis: “Habetis hic aliquid, quod manducetur?". 
\verse At illi obtulerunt ei partem piscis assi. 
\verse Et sumens, coram eis manducavit. 
\verse Et dixit ad eos: “Haec sunt verba, quae locutus sum ad vos, cum adhuc essem vobiscum, quoniam necesse est impleri omnia, quae scripta sunt in Lege Moysis et Prophetis et Psalmis de me". 
\verse Tunc aperuit illis sensum, ut intellegerent Scripturas. 
\verse Et dixit eis: “Sic scriptum est, Christum pati et resurgere a mortuis die tertia, 
\verse et praedicari in nomine eius paenitentiam in remissionem peccatorum in omnes gentes, incipientibus ab Ierusalem. 
\verse Vos estis testes horum. 
\verse Et ecce ego mitto promissum Patris mei in vos; vos autem sedete in civitate, quoadusque induamini virtutem ex alto". 
\verse Eduxit autem eos foras usque in Bethaniam et, elevatis manibus suis, benedixit eis. 
\verse Et factum est, dum benediceret illis, recessit ab eis et ferebatur in caelum. 
\verse Et ipsi adoraverunt eum et regressi sunt in Ierusalem cum gaudio magno 
\verse et erant semper in templo benedicentes Deum.
\end{biblechapter}
