\biblebook{Evangelium secundum Matthaeum}

\begin{biblechapter}   
\verse Liber generationis Iesu Christi filii David filii Abraham. 
\verse Abraham genuit Isaac, Isaac autem genuit Iacob, Iacob autem genuit Iudam et fratres eius, 
\verse Iudas autem genuit Phares et Zara de Thamar, Phares autem genuit Esrom, Esrom autem genuit Aram, 
\verse Aram autem genuit Aminadab, Aminadab autem genuit Naasson, Naasson autem genuit Salmon, 
\verse Salmon autem genuit Booz de Rahab, Booz autem genuit Obed ex Ruth, Obed autem genuit Iesse,  
\verse Iesse autem genuit David regem. David autem genuit Salomonem ex ea, quae fuit Uriae, 
\verse Salomon autem genuit Roboam, Roboam autem genuit Abiam, Abia autem genuit Asa, 
\verse Asa autem genuit Iosaphat, Iosaphat autem genuit Ioram, Ioram autem genuit Oziam, 
\verse Ozias autem genuit Ioatham, Ioatham autem genuit Achaz, Achaz autem genuit Ezechiam,  
\verse Ezechias autem genuit Manassen, Manasses autem genuit Amon, Amon autem genuit Iosiam, 
\verse Iosias autem genuit Iechoniam et fratres eius in transmigratione Babylonis. 
\verse Et post transmigrationem Babylonis Iechonias genuit Salathiel, Salathiel autem genuit Zorobabel, 
\verse Zorobabel autem genuit Abiud, Abiud autem genuit Eliachim, Eliachim autem genuit Azor, 
\verse Azor autem genuit Sadoc, Sadoc autem genuit Achim, Achim autem genuit Eliud, 
\verse Eliud autem genuit Eleazar, Eleazar autem genuit Matthan, Matthan autem genuit Iacob, 
\verse Iacob autem genuit Ioseph virum Mariae, de qua natus est Iesus, qui vocatur Christus. 
\verse Omnes ergo generationes ab Abraham usque ad David generationes quattuordecim; et a David usque ad transmigrationem Babylonis generationes quattuordecim; et a transmigratione Babylonis usque ad Christum generationes quattuordecim. 
\verse Iesu Christi autem generatio sic erat. Cum esset desponsata mater eius Maria Ioseph, antequam convenirent inventa est in utero habens de Spiritu Sancto. 
\verse Ioseph autem vir eius, cum esset iustus et nollet eam traducere, voluit occulte dimittere eam. 
\verse Haec autem eo cogitante, ecce angelus Domini in somnis apparuit ei dicens: “Ioseph fili David, noli timere accipere Mariam coniugem tuam. Quod enim in ea natum est, de Spiritu Sancto est; 
\verse pariet autem filium, et vocabis nomen eius Iesum: ipse enim salvum faciet populum suum a peccatis eorum". 
\verse Hoc autem totum factum est, ut adimpleretur id, quod dictum est a Domino per prophetam dicentem: 
\verse “Ecce, virgo in utero habebit et pariet filium, et vocabunt nomen eius Emmanuel", quod est interpretatum Nobiscum Deus. 
\verse Exsurgens autem Ioseph a somno fecit, sicut praecepit ei angelus Domini, et accepit coniugem suam; 
\verse et non cognoscebat eam, donec peperit filium, et vocavit nomen eius Iesum. 
\end{biblechapter}

\begin{biblechapter}  
\verse Cum autem natus esset Iesus in Bethlehem Iudaeae in diebus Herodis regis, ecce Magi ab oriente venerunt Hierosolymam 
\verse dicentes: “Ubi est, qui natus est, rex Iudaeorum? Vidimus enim stellam eius in oriente et venimus adorare eum". 
\verse Audiens autem Herodes rex turbatus est et omnis Hierosolyma cum illo;  
\verse et congregans omnes principes sacerdotum et scribas populi, sciscitabatur ab eis ubi Christus nasceretur. 
\verse At illi dixerunt ei: “In Bethlehem Iudaeae. Sic enim scriptum est per prophetam: 
\verse "Et tu, Bethlehem terra Iudae, nequaquam minima es in principibus Iudae; ex te enim exiet dux, qui reget populum meum Israel"". 
\verse Tunc Herodes, clam vocatis Magis, diligenter didicit ab eis tempus stellae, quae apparuit eis; 
\verse et mittens illos in Bethlehem dixit: “Ite et interrogate diligenter de puero; et cum inveneritis, renuntiate mihi, ut et ego veniens adorem eum". 
\verse Qui cum audissent regem, abierunt. Et ecce stella, quam viderant in oriente, antecedebat eos, usque dum veniens staret supra, ubi erat puer. 
\verse Videntes autem stellam gavisi sunt gaudio magno valde. 
\verse Et intrantes domum viderunt puerum cum Maria matre eius, et procidentes adoraverunt eum; et apertis thesauris suis, obtulerunt ei munera, aurum et tus et myrrham. 
\verse Et responso accepto in somnis, ne redirent ad Herodem, per aliam viam reversi sunt in regionem suam. 
\verse Qui cum recessissent, ecce angelus Domini apparet in somnis Ioseph dicens: “Surge et accipe puerum et matrem eius et fuge in Aegyptum et esto ibi, usque dum dicam tibi; futurum est enim ut Herodes quaerat puerum ad perdendum eum". 
\verse Qui consurgens accepit puerum et matrem eius nocte et recessit in Aegyptum  
\verse et erat ibi usque ad obitum Herodis, ut adimpleretur, quod dictum est a Domino per prophetam dicentem: “Ex Aegypto vocavi filium meum". 
\verse Tunc Herodes videns quoniam illusus esset a Magis, iratus est valde et mittens occidit omnes pueros, qui erant in Bethlehem et in omnibus finibus eius, a bimatu et infra, secundum tempus, quod exquisierat a Magis. 
\verse Tunc adimpletum est, quod dictum est per Ieremiam prophetam dicentem: 
\verse “Vox in Rama audita est, ploratus et ululatus multus: Rachel plorans filios suos, et noluit consolari, quia non sunt". 
\verse Defuncto autem Herode, ecce apparet angelus Domini in somnis Ioseph in Aegypto 
\verse dicens: “Surge et accipe puerum et matrem eius et vade in terram Israel; defuncti sunt enim, qui quaerebant animam pueri". 
\verse Qui surgens accepit puerum et matrem eius et venit in terram Israel. 
\verse Audiens autem quia Archelaus regnaret in Iudaea pro Herode patre suo, timuit illuc ire; et admonitus in somnis, secessit in partes Galilaeae 
\verse et veniens habitavit in civitate, quae vocatur Nazareth, ut adimpleretur, quod dictum est per Prophetas: “Nazaraeus vocabitur". 
\end{biblechapter}

\begin{biblechapter}  
\verse In diebus autem illis venit Ioannes Baptista praedicans in deserto Iudaeae 
\verse et dicens: “Paenitentiam agite; appropinquavit enim regnum caelorum". 
\verse Hic est enim, qui dictus est per Isaiam prophetam dicentem: “Vox clamantis in deserto: "Parate viam Domini, rectas facite semitas eius!"". 
\verse Ipse autem Ioannes habebat vestimentum de pilis cameli et zonam pelliceam circa lumbos suos; esca autem eius erat locustae et mel silvestre. 
\verse Tunc exibat ad eum Hierosolyma et omnis Iudaea et omnis regio circa Iordanem,  
\verse et baptizabantur in Iordane flumine ab eo, confitentes peccata sua. 
\verse Videns autem multos pharisaeorum et sadducaeorum venientes ad baptismum suum, dixit eis: “Progenies viperarum, quis demonstravit vobis fugere a futura ira? 
\verse Facite ergo fructum dignum paenitentiae 
\verse et ne velitis dicere intra vos: “Patrem habemus Abraham”; dico enim vobis quoniam potest Deus de lapidibus istis suscitare Abrahae filios. 
\verse Iam enim securis ad radicem arborum posita est; omnis ergo arbor, quae non facit fructum bonum, exciditur et in ignem mittitur. 
\verse Ego quidem vos baptizo in aqua in paenitentiam; qui autem post me venturus est, fortior me est, cuius non sum dignus calceamenta portare; ipse vos baptizabit in Spiritu Sancto et igni, 
\verse cuius ventilabrum in manu sua, et permundabit aream suam et congregabit triticum suum in horreum, paleas autem comburet igni inexstinguibili". 
\verse Tunc venit Iesus a Galilaea in Iordanem ad Ioannem, ut baptizaretur ab eo.  
\verse Ioannes autem prohibebat eum dicens: “Ego a te debeo baptizari, et tu venis ad me?". 
\verse Respondens autem Iesus dixit ei: “Sine modo, sic enim decet nos implere omnem iustitiam". Tunc dimittit eum. 
\verse Baptizatus autem Iesus, confestim ascendit de aqua; et ecce aperti sunt ei caeli, et vidit Spiritum Dei descendentem sicut columbam et venientem super se. 
\verse Et ecce vox de caelis dicens: “Hic est Filius meus dilectus, in quo mihi complacui". 
\end{biblechapter}

\begin{biblechapter}   
\verse Tunc Iesus ductus est in desertum a Spiritu, ut tentaretur a Diabolo.  
\verse Et cum ieiunasset quadraginta diebus et quadraginta noctibus, postea esuriit.  
\verse Et accedens tentator dixit ei: “Si Filius Dei es, dic, ut lapides isti panes fiant". 
\verse Qui respondens dixit: “Scriptum est: "Non in pane solo vivet homo, sed in omni verbo, quod procedit de ore Dei"". 
\verse Tunc assumit eum Diabolus in sanctam civitatem et statuit eum supra pinnaculum templi 
\verse et dicit ei: “Si Filius Dei es, mitte te deorsum. Scriptum est enim: "Angelis suis mandabit de te, et in manibus tollent te, ne forte offendas ad lapidem pedem tuum"". 
\verse Ait illi Iesus: “Rursum scriptum est: "Non tentabis Dominum Deum tuum"". 
\verse Iterum assumit eum Diabolus in montem excelsum valde et ostendit ei omnia regna mundi et gloriam eorum 
\verse et dicit illi: “Haec tibi omnia dabo, si cadens adoraveris me”. 
\verse Tunc dicit ei Iesus: “Vade, Satanas! Scriptum est enim: "Dominum Deum tuum adorabis et illi soli servies"". 
\verse Tunc reliquit eum Diabolus, et ecce angeli accesserunt et ministrabant ei. 
\verse Cum autem audisset quod Ioannes traditus esset, secessit in Galilaeam. 
\verse Et relicta Nazareth, venit et habitavit in Capharnaum maritimam 
\verse in finibus Zabulon et Nephthali, ut impleretur, quod dictum est per Isaiam prophetam dicentem: 
\verse “Terra Zabulon et terra Nephthali, ad viam maris, trans Iordanem, Galilaea gentium; 
\verse populus, qui sedebat in tenebris, lucem vidit magnam, et sedentibus in regione et umbra mortis lux orta est eis". 
\verse Exinde coepit Iesus praedicare et dicere: “Paenitentiam agite; appropinquavit enim regnum caelorum". 
\verse Ambulans autem iuxta mare Galilaeae, vidit duos fratres, Simonem, qui vocatur Petrus, et Andream fratrem eius, mittentes rete in mare; erant enim piscatores. 
\verse Et ait illis: “Venite post me, et faciam vos piscatores hominum".  
\verse At illi continuo, relictis retibus, secuti sunt eum. 
\verse Et procedens inde vidit alios duos fratres, Iacobum Zebedaei et Ioannem fratrem eius, in navi cum Zebedaeo patre eorum reficientes retia sua; et vocavit eos. 
\verse Illi autem statim, relicta navi et patre suo, secuti sunt eum. 
\verse Et circumibat Iesus totam Galilaeam, docens in synagogis eorum et praedicans evangelium regni et sanans omnem languorem et omnem infirmitatem in populo.  
\verse Et abiit opinio eius in totam Syriam; et obtulerunt ei omnes male habentes, variis languoribus et tormentis comprehensos, et qui daemonia habebant, et lunaticos et paralyticos, et curavit eos. 
\verse Et secutae sunt eum turbae multae de Galilaea et Decapoli et Hierosolymis et Iudaea et de trans Iordanem. 
\end{biblechapter}

\begin{biblechapter}  
\verse Videns autem turbas, ascendit in montem; et cum sedisset, accesserunt ad eum discipuli eius; 
\verse et aperiens os suum docebat eos dicens: 
\verse “Beati pauperes spiritu, quoniam ipsorum est regnum caelorum. 
\verse Beati, qui lugent, quoniam ipsi consolabuntur. 
\verse Beati mites, quoniam ipsi possidebunt terram. 
\verse Beati, qui esuriunt et sitiunt iustitiam, quoniam ipsi saturabuntur. 
\verse Beati misericordes, quia ipsi misericordiam consequentur. 
\verse Beati mundo corde, quoniam ipsi Deum videbunt. 
\verse Beati pacifici, quoniam filii Dei vocabuntur. 
\verse Beati, qui persecutionem patiuntur propter iustitiam, quoniam ipsorum est regnum caelorum. 
\verse Beati estis cum maledixerint vobis et persecuti vos fuerint et dixerint omne malum adversum vos, mentientes, propter me. 
\verse Gaudete et exsultate, quoniam merces vestra copiosa est in caelis; sic enim persecuti sunt prophetas, qui fuerunt ante vos. 
\verse Vos estis sal terrae; quod si sal evanuerit, in quo salietur? Ad nihilum valet ultra, nisi ut mittatur foras et conculcetur ab hominibus. 
\verse Vos estis lux mundi. Non potest civitas abscondi supra montem posita; 
\verse neque accendunt lucernam et ponunt eam sub modio, sed super candelabrum, ut luceat omnibus, qui in domo sunt. 
\verse Sic luceat lux vestra coram hominibus, ut videant vestra bona opera et glorificent Patrem vestrum, qui in caelis est. 
\verse Nolite putare quoniam veni solvere Legem aut Prophetas; non veni solvere, sed adimplere. 
\verse Amen quippe dico vobis: Donec transeat caelum et terra, iota unum aut unus apex non praeteribit a Lege, donec omnia fiant. 
\verse Qui ergo solverit unum de mandatis istis minimis et docuerit sic homines, minimus vocabitur in regno caelorum; qui autem fecerit et docuerit, hic magnus vocabitur in regno caelorum. 
\verse Dico enim vobis: Nisi abundaverit iustitia vestra plus quam scribarum et pharisaeorum, non intrabitis in regnum caelorum. 
\verse Audistis quia dictum est antiquis: “Non occides; qui autem occiderit, reus erit iudicio”. 
\verse Ego autem dico vobis: Omnis, qui irascitur fratri suo, reus erit iudicio; qui autem dixerit fratri suo: “Racha”, reus erit concilio; qui autem dixerit: “Fatue”, reus erit gehennae ignis. 
\verse Si ergo offeres munus tuum ad altare, et ibi recordatus fueris quia frater tuus habet aliquid adversum te, 
\verse relinque ibi munus tuum ante altare et vade, prius, reconciliare fratri tuo et tunc veniens offer munus tuum. 
\verse Esto consentiens adversario tuo cito, dum es in via cum eo, ne forte tradat te adversarius iudici, et iudex tradat te ministro, et in carcerem mittaris. 
\verse Amen dico tibi: Non exies inde, donec reddas novissimum quadrantem. 
\verse Audistis quia dictum est: "Non moechaberis". 
\verse Ego autem dico vobis: Omnis, qui viderit mulierem ad concupiscendum eam, iam moechatus est eam in corde suo. 
\verse Quod si oculus tuus dexter scandalizat te, erue eum et proice abs te; expedit enim tibi, ut pereat unum membrorum tuorum, quam totum corpus tuum mittatur in gehennam. 
\verse Et si dextera manus tua scandalizat te, abscide eam et proice abs te; expedit enim tibi, ut pereat unum membrorum tuorum, quam totum corpus tuum abeat in gehennam. 
\verse Dictum est autem: “Quicumque dimiserit uxorem suam, det illi libellum repudii”. 
\verse Ego autem dico vobis: Omnis, qui dimiserit uxorem suam, excepta fornicationis causa, facit eam moechari; et, qui dimissam duxerit, adulterat. 
\verse Iterum audistis quia dictum est antiquis: "Non periurabis; reddes autem Domino iuramenta tua". 
\verse Ego autem dico vobis: Non iurare omnino, neque per caelum, quia thronus Dei est, 
\verse neque per terram, quia scabellum est pedum eius, neque per Hierosolymam, quia civitas est magni Regis; 
\verse neque per caput tuum iuraveris, quia non potes unum capillum album facere aut nigrum.  
\verse Sit autem sermo vester: “Est, est”, “Non, non”; quod autem his abundantius est, a Malo est. 
\verse Audistis quia dictum est: "Oculum pro oculo et dentem pro dente". 
\verse Ego autem dico vobis: Non resistere malo; sed si quis te percusserit in dextera maxilla tua, praebe illi et alteram; 
\verse et ei, qui vult tecum iudicio contendere et tunicam tuam tollere, remitte ei et pallium; 
\verse et quicumque te angariaverit mille passus, vade cum illo duo. 
\verse Qui petit a te, da ei; et volenti mutuari a te, ne avertaris. 
\verse Audistis quia dictum est: "Diliges proximum tuum et odio habebis inimicum tuum". 
\verse Ego autem dico vobis: Diligite inimicos vestros et orate pro persequentibus vos, 
\verse ut sitis filii Patris vestri, qui in caelis est, quia solem suum oriri facit super malos et bonos et pluit super iustos et iniustos. 
\verse Si enim dilexeritis eos, qui vos diligunt, quam mercedem habetis? Nonne et publicani hoc faciunt? 
\verse Et si salutaveritis fratres vestros tantum, quid amplius facitis? Nonne et ethnici hoc faciunt? 
\verse Estote ergo vos perfecti, sicut Pater vester caelestis perfectus est. 
\end{biblechapter}

\begin{biblechapter}  
\verse Attendite, ne iustitiam vestram faciatis coram hominibus, ut videamini ab eis; alioquin mercedem non habetis apud Patrem vestrum, qui in caelis est.  
\verse Cum ergo facies eleemosynam, noli tuba canere ante te, sicut hypocritae faciunt in synagogis et in vicis, ut honorificentur ab hominibus. Amen dico vobis: Receperunt mercedem suam. 
\verse Te autem faciente eleemosynam, nesciat sinistra tua quid faciat dextera tua,  
\verse ut sit eleemosyna tua in abscondito, et Pater tuus, qui videt in abscondito, reddet tibi. 
\verse Et cum oratis, non eritis sicut hypocritae, qui amant in synagogis et in angulis platearum stantes orare, ut videantur ab hominibus. Amen dico vobis: Receperunt mercedem suam. 
\verse Tu autem cum orabis, intra in cubiculum tuum et, clauso ostio tuo, ora Patrem tuum, qui est in abscondito; et Pater tuus, qui videt in abscondito, reddet tibi. 
\verse Orantes autem nolite multum loqui sicut ethnici; putant enim quia in multiloquio suo exaudiantur. 
\verse Nolite ergo assimilari eis; scit enim Pater vester, quibus opus sit vobis, antequam petatis eum. 
\verse Sic ergo vos orabitis: Pater noster, qui es in caelis, sanctificetur nomen tuum, 
\verse adveniat regnum tuum, fiat voluntas tua, sicut in caelo, et in terra. 
\verse Panem nostrum supersubstantialem da nobis hodie; 
\verse et dimitte nobis debita nostra, sicut et nos dimittimus debitoribus nostris; 
\verse et ne inducas nos in tentationem, sed libera nos a Malo. 
\verse Si enim dimiseritis hominibus peccata eorum, dimittet et vobis Pater vester caelestis; 
\verse si autem non dimiseritis hominibus, nec Pater vester dimittet peccata vestra. 
\verse Cum autem ieiunatis, nolite fieri sicut hypocritae tristes; demoliuntur enim facies suas, ut pareant hominibus ieiunantes. Amen dico vobis: Receperunt mercedem suam. 
\verse Tu autem cum ieiunas, unge caput tuum et faciem tuam lava,  
\verse ne videaris hominibus ieiunans sed Patri tuo, qui est in abscondito; et Pater tuus, qui videt in abscondito, reddet tibi. 
\verse Nolite thesaurizare vobis thesauros in terra, ubi aerugo et tinea demolitur, et ubi fures effodiunt et furantur; 
\verse thesaurizate autem vobis thesauros in caelo, ubi neque aerugo neque tinea demolitur, et ubi fures non effodiunt nec furantur; 
\verse ubi enim est thesaurus tuus, ibi erit et cor tuum. 
\verse Lucerna corporis est oculus. Si ergo fuerit oculus tuus simplex, totum corpus tuum lucidum erit; 
\verse si autem oculus tuus nequam fuerit, totum corpus tuum tenebrosum erit. Si ergo lumen, quod in te est, tene brae sunt, tenebrae quantae erunt! 
\verse Nemo potest duobus dominis servire: aut enim unum odio habebit et alterum diliget, aut unum sustinebit et alterum contemnet; non potestis Deo servire et mammonae. 
\verse Ideo dico vobis: Ne solliciti sitis animae vestrae quid manducetis, neque corpori vestro quid induamini. Nonne anima plus est quam esca, et corpus quam vestimentum? 
\verse Respicite volatilia caeli, quoniam non serunt neque metunt neque congregant in horrea, et Pater vester caelestis pascit illa. Nonne vos magis pluris estis illis? 
\verse Quis autem vestrum cogitans potest adicere ad aetatem suam cubitum unum? 
\verse Et de vestimento quid solliciti estis? Considerate lilia agri quomodo crescunt: non laborant neque nent. 
\verse Dico autem vobis quoniam nec Salomon in omni gloria sua coopertus est sicut unum ex istis. 
\verse Si autem fenum agri, quod hodie est et cras in clibanum mittitur, Deus sic vestit, quanto magis vos, modicae fidei? 
\verse Nolite ergo solliciti esse dicentes: “Quid manducabimus?”, aut: “Quid bibemus?”, aut: “Quo operiemur?”. 
\verse Haec enim omnia gentes inquirunt; scit enim Pater vester caelestis quia his omnibus indigetis. 
\verse Quaerite autem primum regnum Dei et iustitiam eius, et haec omnia adicientur vobis. 
\verse Nolite ergo esse solliciti in crastinum; crastinus enim dies sollicitus erit sibi ipse. Sufficit diei malitia sua. 
\end{biblechapter}

\begin{biblechapter}  
\verse Nolite iudicare, ut non iudice mini; 
\verse in quo enim iudicio iudicaveritis, iudicabimini, et in qua mensura mensi fueritis, metietur vobis. 
\verse Quid autem vides festucam in oculo fratris tui, et trabem in oculo tuo non vides? 
\verse Aut quomodo dices fratri tuo: “Sine, eiciam festucam de oculo tuo”, et ecce trabes est in oculo tuo? 
\verse Hypocrita, eice primum trabem de oculo tuo, et tunc videbis eicere festucam de oculo fratris tui. 
\verse Nolite dare sanctum canibus neque mittatis margaritas vestras ante porcos, ne forte conculcent eas pedibus suis et conversi dirumpant vos. 
\verse Petite, et dabitur vobis; quaerite et invenietis; pulsate, et aperietur vobis. 
\verse Omnis enim qui petit, accipit; et, qui quaerit, invenit; et pulsanti aperietur. 
\verse Aut quis est ex vobis homo, quem si petierit filius suus panem, numquid lapidem porriget ei? 
\verse Aut si piscem petierit, numquid serpentem porriget ei? 
\verse Si ergo vos, cum sitis mali, nostis dona bona dare filiis vestris, quanto magis Pater vester, qui in caelis est, dabit bona petentibus se. 
\verse Omnia ergo, quaecumque vultis ut faciant vobis homines, ita et vos facite eis; haec est enim Lex et Prophetae. 
\verse Intrate per angustam portam, quia lata porta et spatiosa via, quae ducit ad perditionem, et multi sunt, qui intrant per eam; 
\verse quam angusta porta et arta via, quae ducit ad vitam, et pauci sunt, qui inveniunt eam! 
\verse Attendite a falsis prophetis, qui veniunt ad vos in vestimentis ovium, intrinsecus autem sunt lupi rapaces. 
\verse A fructibus eorum cognoscetis eos; numquid colligunt de spinis uvas aut de tribulis ficus? 
\verse Sic omnis arbor bona fructus bonos facit, mala autem arbor fructus malos facit; 
\verse non potest arbor bona fructus malos facere, neque arbor mala fructus bonos facere.  
\verse Omnis arbor, quae non facit fructum bonum, exciditur et in ignem mittitur.  
\verse Igitur ex fructibus eorum cognoscetis eos. 
\verse Non omnis, qui dicit mihi: “Domine, Domine”, intrabit in regnum caelorum, sed qui facit voluntatem Patris mei, qui in caelis est. 
\verse Multi dicent mihi in illa die: “Domine, Domine, nonne in tuo nomine prophetavimus, et in tuo nomine daemonia eiecimus, et in tuo nomine virtutes multas fecimus?”. 
\verse Et tunc confitebor illis: Numquam novi vos; discedite a me, qui operamini iniquitatem. 
\verse Omnis ergo, qui audit verba mea haec et facit ea, assimilabitur viro sapienti, qui aedificavit domum suam supra petram. 
\verse Et descendit pluvia, et venerunt flumina, et flaverunt venti et irruerunt in domum illam, et non cecidit; fundata enim erat supra petram. 
\verse Et omnis, qui audit verba mea haec et non facit ea, similis erit viro stulto, qui aedificavit domum suam supra arenam. 
\verse Et descendit pluvia, et venerunt flumina, et flaverunt venti et irruerunt in domum illam, et cecidit, et fuit ruina eius magna". 
\verse Et factum est, cum consummasset Iesus verba haec, admirabantur turbae super doctrinam eius; 
\verse erat enim docens eos sicut potestatem habens, et non sicut scribae eorum. 
\end{biblechapter}

\begin{biblechapter}  
\verse Cum autem descendisset de monte, secutae sunt eum turbae multae. 
\verse Et ecce leprosus veniens adorabat eum dicens: “Domine, si vis, potes me mundare". 
\verse Et extendens manum, tetigit eum dicens: “Volo, mundare!"; et confestim mundata est lepra eius. 
\verse Et ait illi Iesus: “Vide, nemini dixeris; sed vade, ostende te sacerdoti et offer munus, quod praecepit Moyses, in testimonium illis". 
\verse Cum autem introisset Capharnaum, accessit ad eum centurio rogans eum 
\verse et dicens: “Domine, puer meus iacet in domo paralyticus et male torquetur". 
\verse Et ait illi: “Ego veniam et curabo eum". 
\verse Et respondens centurio ait: “Domine, non sum dignus, ut intres sub tectum meum, sed tantum dic verbo, et sanabitur puer meus. 
\verse Nam et ego homo sum sub potestate, habens sub me milites, et dico huic: “Vade”, et vadit; et alii: “Veni”, et venit; et servo meo: “Fac hoc”, et facit”. 
\verse Audiens autem Iesus, miratus est et sequentibus se dixit: “Amen dico vobis: Apud nullum inveni tantam fidem in Israel! 
\verse Dico autem vobis quod multi ab oriente et occidente venient et recumbent cum Abraham et Isaac et Iacob in regno caelorum; 
\verse filii autem regni eicientur in tenebras exteriores: ibi erit fletus et stridor dentium". 
\verse Et dixit Iesus centurioni: “Vade; sicut credidisti, fiat tibi". Et sanatus est puer in hora illa. 
\verse Et cum venisset Iesus in domum Petri, vidit socrum eius iacentem et febricitantem; 
\verse et tetigit manum eius, et dimisit eam febris; et surrexit et ministrabat ei. 
\verse Vespere autem facto, obtulerunt ei multos daemonia habentes; et eiciebat spiritus verbo et omnes male habentes curavit, 
\verse ut adimpleretur, quod dictum est per Isaiam prophetam dicentem: “Ipse infirmitates nostras accepit et aegrotationes portavit". 
\verse Videns autem Iesus turbas multas circum se, iussit ire trans fretum. 
\verse Et accedens unus scriba ait illi: “Magister, sequar te, quocumque ieris".  
\verse Et dicit ei Iesus: “Vulpes foveas habent, et volucres caeli tabernacula, Filius autem hominis non habet, ubi caput reclinet". 
\verse Alius autem de discipulis eius ait illi: “Domine, permitte me primum ire et sepelire patrem meum". 
\verse Iesus autem ait illi: “Sequere me et dimitte mortuos sepelire mortuos suos". 
\verse Et ascendente eo in naviculam, secuti sunt eum discipuli eius. 
\verse Et ecce motus magnus factus est in mari, ita ut navicula operiretur fluctibus; ipse vero dormiebat. 
\verse Et accesserunt et suscitaverunt eum dicentes: “Domine, salva nos, perimus!". 
\verse Et dicit eis: “Quid timidi estis, modicae fidei?". Tunc surgens increpavit ventis et mari, et facta est tranquillitas magna. 
\verse Porro homines mirati sunt dicentes: “Qualis est hic, quia et venti et mare oboediunt ei?". 
\verse Et cum venisset trans fretum in regionem Gadarenorum, occurrerunt ei duo habentes daemonia, de monumentis exeuntes, saevi nimis, ita ut nemo posset transire per viam illam. 
\verse Et ecce clamaverunt dicentes: “Quid nobis et tibi, Fili Dei? Venisti huc ante tempus torquere nos?". 
\verse Erat autem longe ab illis grex porcorum multorum pascens. 
\verse Daemones autem rogabant eum dicentes: “Si eicis nos, mitte nos in gregem porcorum". 
\verse Et ait illis: “Ite". Et illi exeuntes abierunt in porcos; et ecce impetu abiit totus grex per praeceps in mare, et mortui sunt in aquis. 
\verse Pastores autem fugerunt et venientes in civitatem nuntiaverunt omnia et de his, qui daemonia habuerant. 
\verse Et ecce tota civitas exiit obviam Iesu, et viso eo rogabant, ut transiret a finibus eorum. 
\end{biblechapter}

\begin{biblechapter}  
\verse Et ascendens in naviculam transfretavit et venit in civitatem suam.  
\verse Et ecce offerebant ei paralyticum iacentem in lecto. Et videns Iesus fidem illorum, dixit paralytico: “Confide, fili; remittuntur peccata tua". 
\verse Et ecce quidam de scribis dixerunt intra se: “Hic blasphemat". 
\verse Et cum vidisset Iesus cogitationes eorum, dixit: “Ut quid cogitatis mala in cordibus vestris? 
\verse Quid enim est facilius, dicere: “Dimittuntur peccata tua”, aut dicere: “Surge et ambula”? 
\verse Ut sciatis autem quoniam Filius hominis habet potestatem in terra dimittendi peccata — tunc ait paralytico - : Surge, tolle lectum tuum et vade in domum tuam". 
\verse Et surrexit et abiit in domum suam.  
\verse Videntes autem turbae timuerunt et glorificaverunt Deum, qui dedit potestatem talem hominibus. 
\verse Et cum transiret inde Iesus, vidit hominem sedentem in teloneo, Matthaeum nomine, et ait illi: “Sequere me”. Et surgens secutus est eum. 
\verse Et factum est, discumbente eo in domo, ecce multi publicani et peccatores venientes simul discumbebant cum Iesu et discipulis eius. 
\verse Et videntes pharisaei dicebant discipulis eius: “Quare cum publicanis et peccatoribus manducat magister vester?". 
\verse At ille audiens ait: “Non est opus valentibus medico sed male habentibus. 
\verse Euntes autem discite quid est: “Misericordiam volo et non sacrificium”. Non enim veni vocare iustos sed peccatores". 
\verse Tunc accedunt ad eum discipuli Ioannis dicentes: “Quare nos et pharisaei ieiunamus frequenter, discipuli autem tui non ieiunant?". 
\verse Et ait illis Iesus: “Numquid possunt convivae nuptiarum lugere, quamdiu cum illis est sponsus? Venient autem dies, cum auferetur ab eis sponsus, et tunc ieiunabunt. 
\verse Nemo autem immittit commissuram panni rudis in vestimentum vetus; tollit enim supplementum eius a vestimento, et peior scissura fit. 
\verse Neque mittunt vinum novum in utres veteres, alioquin rumpuntur utres, et vinum effunditur, et utres pereunt; sed vinum novum in utres novos mittunt, et ambo conservantur". 
\verse Haec illo loquente ad eos, ecce princeps unus accessit et adorabat eum dicens: “Filia mea modo defuncta est; sed veni, impone manum tuam super eam, et vivet". 
\verse Et surgens Iesus sequebatur eum et discipuli eius. 
\verse Et ecce mulier, quae sanguinis fluxum patiebatur duodecim annis, accessit retro et tetigit fimbriam vestimenti eius. 
\verse Dicebat enim intra se: “Si tetigero tantum vestimentum eius, salva ero". 
\verse At Iesus conversus et videns eam dixit: “Confide, filia; fides tua te salvam fecit". Et salva facta est mulier ex illa hora. 
\verse Et cum venisset Iesus in domum principis et vidisset tibicines et turbam tumultuantem, 
\verse dicebat: “Recedite; non est enim mortua puella, sed dormit". Et deridebant eum. 
\verse At cum eiecta esset turba, intravit et tenuit manum eius, et surrexit puella. 
\verse Et exiit fama haec in universam terram illam. 
\verse Et transeunte inde Iesu, secuti sunt eum duo caeci clamantes et dicentes: “Miserere nostri, fili David!". 
\verse Cum autem venisset domum, accesserunt ad eum caeci, et dicit eis Iesus: “Creditis quia possum hoc facere?". Dicunt ei: “Utique, Domine”. 
\verse Tunc tetigit oculos eorum dicens: “Secundum fidem vestram fiat vobis”. 
\verse Et aperti sunt oculi illorum. Et comminatus est illis Iesus dicens: “Videte, ne quis sciat". 
\verse Illi autem exeuntes diffamaverunt eum in universa terra illa. 
\verse Egressis autem illis, ecce obtulerunt ei hominem mutum, daemonium habentem. 
\verse Et eiecto daemone, locutus est mutus. Et miratae sunt turbae dicentes: “Numquam apparuit sic in Israel!". 
\verse Pharisaei autem dicebant: “In principe daemoniorum eicit daemones". 
\verse Et circumibat Iesus civitates omnes et castella, docens in synagogis eorum et praedicans evangelium regni et curans omnem languorem et omnem infirmitatem.  
\verse Videns autem turbas, misertus est eis, quia erant vexati et iacentes sicut oves non habentes pastorem. 
\verse Tunc dicit discipulis suis: “Messis quidem multa, operarii autem pauci; 
\verse rogate ergo Dominum messis, ut mittat operarios in messem suam". 
\end{biblechapter}

\begin{biblechapter}  
\verse Et convocatis Duodecim discipulis suis, dedit illis potestatem spirituum immundorum, ut eicerent eos et curarent omnem languorem et omnem infirmitatem. 
\verse Duodecim autem apostolorum nomina sunt haec: primus Simon, qui dicitur Petrus, et Andreas frater eius, et Iacobus Zebedaei et Ioannes frater eius, 
\verse Philippus et Bartholomaeus, Thomas et Matthaeus publicanus, Iacobus Alphaei et Thaddaeus, 
\verse Simon Chananaeus et Iudas Iscariotes, qui et tradidit eum. 
\verse Hos Duodecim misit Iesus praecipiens eis et dicens: “In viam gentium ne abieritis et in civitates Samaritanorum ne intraveritis; 
\verse sed potius ite ad oves, quae perierunt domus Israel. 
\verse Euntes autem praedicate dicentes: “Appropinquavit regnum caelorum”. 
\verse Infirmos curate, mortuos suscitate, leprosos mundate, daemones eicite; gratis accepistis, gratis date. 
\verse Nolite possidere aurum neque argentum neque pecuniam in zonis vestris, 
\verse non peram in via neque duas tunicas neque calceamenta neque virgam; dignus enim est operarius cibo suo. 
\verse In quamcumque civitatem aut castellum intraveritis, interrogate quis in ea dignus sit; et ibi manete donec exeatis. 
\verse Intrantes autem in domum, salutate eam; 
\verse et si quidem fuerit domus digna, veniat pax vestra super eam; si autem non fuerit digna, pax vestra ad vos revertatur. 
\verse Et quicumque non receperit vos neque audierit sermones vestros, exeuntes foras de domo vel de civitate illa, excutite pulverem de pedibus vestris. 
\verse Amen dico vobis: Tolerabilius erit terrae Sodomorum et Gomorraeorum in die iudicii quam illi civitati. 
\verse Ecce ego mitto vos sicut oves in medio luporum; estote ergo prudentes sicut serpentes et simplices sicut columbae. 
\verse Cavete autem ab hominibus; tradent enim vos in conciliis, et in synagogis suis flagellabunt vos; 
\verse et ad praesides et ad reges ducemini propter me in testimonium illis et gentibus.  
\verse Cum autem tradent vos, nolite cogitare quomodo aut quid loquamini; dabitur enim vobis in illa hora quid loquamini. 
\verse Non enim vos estis, qui loquimini, sed Spiritus Patris vestri, qui loquitur in vobis. 
\verse Tradet autem frater fratrem in mortem, et pater filium; et insurgent filii in parentes et morte eos afficient. 
\verse Et eritis odio omnibus propter nomen meum; qui autem perseveraverit in finem, hic salvus erit. 
\verse Cum autem persequentur vos in civitate ista, fugite in aliam; amen enim dico vobis: Non consummabitis civitates Israel, donec veniat Filius hominis. 
\verse Non est discipulus super magistrum nec servus super dominum suum. 
\verse Sufficit discipulo, ut sit sicut magister eius, et servus sicut dominus eius. Si patrem familias Beelzebul vocaverunt, quanto magis domesticos eius! 
\verse Ne ergo timueritis eos. Nihil enim est opertum, quod non revelabitur, et occultum, quod non scietur. 
\verse Quod dico vobis in tenebris, dicite in lumine; et, quod in aure auditis, praedicate super tecta. 
\verse Et nolite timere eos, qui occidunt corpus, animam autem non possunt occidere; sed potius eum timete, qui potest et animam et corpus perdere in gehenna. 
\verse Nonne duo passeres asse veneunt? Et unus ex illis non cadet super terram sine Patre vestro. 
\verse Vestri autem et capilli capitis omnes numerati sunt. 
\verse Nolite ergo timere; multis passeribus meliores estis vos. 
\verse Omnis ergo qui confitebitur me coram hominibus, confitebor et ego eum coram Patre meo, qui est in caelis; 
\verse qui autem negaverit me coram hominibus, negabo et ego eum coram Patre meo, qui est in caelis. 
\verse Nolite arbitrari quia venerim mittere pacem in terram; non veni pacem mittere sed gladium. 
\verse Veni enim separare hominem adversus patrem suum et filiam adversus matrem suam et nurum adversus socrum suam: 
\verse et inimici hominis domestici eius. 
\verse Qui amat patrem aut matrem plus quam me, non est me dignus; et, qui amat filium aut filiam super me, non est me dignus; 
\verse et, qui non accipit crucem suam et sequitur me, non est me dignus. 
\verse Qui invenerit animam suam, perdet illam; et, qui perdiderit animam suam propter me, inveniet eam. 
\verse Qui recipit vos, me recipit; et, qui me recipit, recipit eum, qui me misit. 
\verse Qui recipit prophetam in nomine prophetae, mercedem prophetae accipiet; et, qui recipit iustum in nomine iusti, mercedem iusti accipiet. 
\verse Et, quicumque potum dederit uni ex minimis istis calicem aquae frigidae tantum in nomine discipuli, amen dico vobis: Non perdet mercedem suam". 
\end{biblechapter}

\begin{biblechapter}  
\verse Et factum est, cum consummasset Iesus praecipiens Duodecim discipulis suis, transiit inde, ut doceret et praedicaret in civitatibus eorum. 
\verse Ioannes autem, cum audisset in vinculis opera Christi, mittens per discipulos suos 
\verse ait illi: “Tu es qui venturus es, an alium exspectamus?". 
\verse Et respondens Iesus ait illis: “Euntes renuntiate Ioanni, quae auditis et videtis:  
\verse caeci vident et claudi ambulant, leprosi mundantur et surdi audiunt et mortui resurgunt et pauperes evangelizantur; 
\verse et beatus est, qui non fuerit scandalizatus in me". 
\verse Illis autem abeuntibus, coepit Iesus dicere ad turbas de Ioanne: “Quid existis in desertum videre? Arundinem vento agitatam?  
\verse Sed quid existis videre? Hominem mollibus vestitum? Ecce, qui mollibus vestiuntur, in domibus regum sunt. 
\verse Sed quid existis videre? Prophetam? Etiam, dico vobis, et plus quam prophetam. 
\verse Hic est, de quo scriptum est: "Ecce ego mitto angelum meum ante faciem tuam, qui praeparabit viam tuam ante te". 
\verse Amen dico vobis: Non surrexit inter natos mulierum maior Ioanne Baptista; qui autem minor est in regno caelorum, maior est illo. 
\verse A diebus autem Ioannis Baptistae usque nunc regnum caelorum vim patitur, et violenti rapiunt illud.  
\verse Omnes enim Prophetae et Lex usque ad Ioannem prophetaverunt; 
\verse et si vultis recipere, ipse est Elias, qui venturus est. 
\verse Qui habet aures, audiat. 
\verse Cui autem similem aestimabo generationem istam? Similis est pueris sedentibus in foro, qui clamantes coaequalibus 
\verse dicunt: “Cecinimus vobis, et non saltastis; lamentavimus, et non planxistis”. 
\verse Venit enim Ioannes neque manducans neque bibens, et dicunt: “Daemonium habet!”; 
\verse venit Filius hominis manducans et bibens, et dicunt: “Ecce homo vorax et potator vini, publicanorum amicus et peccatorum!”. Et iustificata est sapientia ab operibus suis". 
\verse Tunc coepit exprobrare civitatibus, in quibus factae sunt plurimae virtutes eius, quia non egissent paenitentiam: 
\verse “Vae tibi, Chorazin! Vae tibi, Bethsaida! Quia si in Tyro et Sidone factae essent virtutes, quae factae sunt in vobis, olim in cilicio et cinere paenitentiam egissent. 
\verse Verumtamen dico vobis: Tyro et Sidoni remissius erit in die iudicii quam vobis. 
\verse Et tu, Capharnaum, numquid usque in caelum exaltaberis? Usque in infernum descendes! Quia si in Sodomis factae fuissent virtutes, quae factae sunt in te, mansissent usque in hunc diem. 
\verse Verumtamen dico vobis: Terrae Sodomorum remissius erit in die iudicii quam tibi". 
\verse In illo tempore respondens Iesus dixit: “Confiteor tibi, Pater, Domine caeli et terrae, quia abscondisti haec a sapientibus et prudentibus et revelasti ea parvulis. 
\verse Ita, Pater, quoniam sic fuit placitum ante te. 
\verse Omnia mihi tradita sunt a Patre meo; et nemo novit Filium nisi Pater, neque Patrem quis novit nisi Filius et cui voluerit Filius revelare. 
\verse Venite ad me, omnes, qui laboratis et onerati estis, et ego reficiam vos.  
\verse Tollite iugum meum super vos et discite a me, quia mitis sum et humilis corde, et invenietis requiem animabus vestris. 
\verse Iugum enim meum suave, et onus meum leve est". 
\end{biblechapter}

\begin{biblechapter}  
\verse In illo tempore abiit Iesus sabbatis per sata; discipuli autem eius esurierunt et coeperunt vellere spicas et manducare. 
\verse Pharisaei autem videntes dixerunt ei: “Ecce discipuli tui faciunt, quod non licet facere sabbato". 
\verse At ille dixit eis: “Non legistis quid fecerit David, quando esuriit, et qui cum eo erant? 
\verse Quomodo intravit in domum Dei et panes propositionis comedit, quod non licebat ei edere neque his, qui cum eo erant, nisi solis sacerdotibus? 
\verse Aut non legistis in Lege quia sabbatis sacerdotes in templo sabbatum violant et sine crimine sunt? 
\verse Dico autem vobis quia templo maior est hic.  
\verse Si autem sciretis quid est: “Misericordiam volo et non sacrificium”, numquam condemnassetis innocentes. 
\verse Dominus est enim Filius hominis sabbati". 
\verse Et cum inde transisset, venit in synagogam eorum; 
\verse et ecce homo manum habens aridam. Et interrogabant eum dicentes: “Licet sabbatis curare?", ut accusarent eum. 
\verse Ipse autem dixit illis: “Quis erit ex vobis homo, qui habeat ovem unam et, si ceciderit haec sabbatis in foveam, nonne tenebit et levabit eam? 
\verse Quanto igitur melior est homo ove! Itaque licet sabbatis bene facere". 
\verse Tunc ait homini: “Extende manum tuam". Et extendit, et restituta est sana sicut altera.  
\verse Exeuntes autem pharisaei consilium faciebant adversus eum, quomodo eum perderent. 
\verse Iesus autem sciens secessit inde. Et secuti sunt eum multi, et curavit eos omnes 
\verse et comminatus est eis, ne manifestum eum facerent, 
\verse ut adimpleretur, quod dictum est per Isaiam prophetam dicentem: 
\verse “Ecce puer meus, quem elegi, dilectus meus, in quo bene placuit animae meae; ponam Spiritum meum super eum, et iudicium gentibus nuntiabit. 
\verse Non contendet neque clamabit, neque audiet aliquis in plateis vocem eius. 
\verse Arundinem quassatam non confringet et linum fumigans non exstinguet, donec eiciat ad victoriam iudicium; 
\verse et in nomine eius gentes sperabunt". 
\verse Tunc oblatus est ei daemonium habens, caecus et mutus, et curavit eum, ita ut mutus loqueretur et videret. 
\verse Et stupebant omnes turbae et dicebant: “Numquid hic est filius David?". 
\verse Pharisaei autem audientes dixerunt: “Hic non eicit daemones nisi in Beelzebul, principe daemonum". 
\verse Sciens autem cogitationes eorum dixit eis: “Omne regnum divisum contra se desolatur, et omnis civitas vel domus divisa contra se non stabit. 
\verse Et si Satanas Satanam eicit, adversus se divisus est; quomodo ergo stabit regnum eius? 
\verse Et si ego in Beelzebul eicio daemones, filii vestri in quo eiciunt? Ideo ipsi iudices erunt vestri. 
\verse Si autem in Spiritu Dei ego eicio daemones, igitur pervenit in vos regnum Dei. 
\verse Aut quomodo potest quisquam intrare in domum fortis et vasa eius diripere, nisi prius alligaverit fortem? Et tunc domum illius diripiet. 
\verse Qui non est mecum, contra me est; et, qui non congregat mecum, spargit. 
\verse Ideo dico vobis: Omne peccatum et blasphemia remittetur hominibus, Spiritus autem blasphemia non remittetur. 
\verse Et quicumque dixerit verbum contra Filium hominis, remittetur ei; qui autem dixerit contra Spiritum Sanctum, non remittetur ei neque in hoc saeculo neque in futuro. 
\verse Aut facite arborem bonam et fructum eius bonum, aut facite arborem malam et fructum eius malum: si quidem ex fructu arbor agnoscitur. 
\verse Progenies viperarum, quomodo potestis bona loqui, cum sitis mali? Ex abundantia enim cordis os loquitur. 
\verse Bonus homo de bono thesauro profert bona, et malus homo de malo thesauro profert mala. 
\verse Dico autem vobis: Omne verbum otiosum, quod locuti fuerint homines, reddent rationem de eo in die iudicii:  
\verse ex verbis enim tuis iustificaberis, et ex verbis tuis condemnaberis". 
\verse Tunc responderunt ei quidam de scribis et pharisaeis dicentes: “Magister, volumus a te signum videre". 
\verse Qui respondens ait illis: “Generatio mala et adultera signum requirit; et signum non dabitur ei, nisi signum Ionae prophetae. 
\verse Sicut enim fuit Ionas in ventre ceti tribus diebus et tribus noctibus, sic erit Filius hominis in corde terrae tribus diebus et tribus noctibus. 
\verse Viri Ninevitae surgent in iudicio cum generatione ista et condemnabunt eam, quia paenitentiam egerunt in praedicatione Ionae; et ecce plus quam Iona hic! 
\verse Regina austri surget in iudicio cum generatione ista et condemnabit eam, quia venit a finibus terrae audire sapientiam Salomonis; et ecce plus quam Salomon hic! 
\verse Cum autem immundus spiritus exierit ab homine, ambulat per loca arida quaerens requiem et non invenit. 
\verse Tunc dicit: “Revertar in domum meam unde exivi”; et veniens invenit vacantem, scopis mundatam et ornatam. 
\verse Tunc vadit et assumit secum septem alios spiritus nequiores se, et intrantes habitant ibi; et fiunt novissima hominis illius peiora prioribus. Sic erit et generationi huic pessimae". 
\verse Adhuc eo loquente ad turbas, ecce mater et fratres eius stabant foris quaerentes loqui ei. 
\verse Dixit autem ei quidam: “Ecce mater tua et fratres tui foris stant quaerentes loqui tecum". 
\verse At ille respondens dicenti sibi ait: “Quae est mater mea, et qui sunt fratres mei?". 
\verse Et extendens manum suam in discipulos suos dixit: “Ecce mater mea et fratres mei. 
\verse Quicumque enim fecerit voluntatem Patris mei, qui in caelis est, ipse meus frater et soror et mater est". 
\end{biblechapter}

\begin{biblechapter}  
\verse In illo die exiens Iesus de domo sedebat secus mare; 
\verse et congregatae sunt ad eum turbae multae, ita ut in naviculam ascendens sederet, et omnis turba stabat in litore. 
\verse Et locutus est eis multa in parabolis dicens: “Ecce exiit, qui seminat, seminare. 
\verse Et dum seminat, quaedam ceciderunt secus viam, et venerunt volucres et comederunt ea. 
\verse Alia autem ceciderunt in petrosa, ubi non habebant terram multam, et continuo exorta sunt, quia non habebant altitudinem terrae; 
\verse sole autem orto, aestuaverunt et, quia non habebant radicem, aruerunt. 
\verse Alia autem ceciderunt in spinas, et creverunt spinae et suffocaverunt ea. 
\verse Alia vero ceciderunt in terram bonam et dabant fructum: aliud centesimum, aliud sexagesimum, aliud tricesimum. 
\verse Qui habet aures, audiat". 
\verse Et accedentes discipuli dixerunt ei: “Quare in parabolis loqueris eis?". 
\verse Qui respondens ait illis: “Quia vobis datum est nosse mysteria regni caelorum, illis autem non est datum. 
\verse Qui enim habet, dabitur ei, et abundabit; qui autem non habet, et quod habet, auferetur ab eo.  
\verse Ideo in parabolis loquor eis, quia videntes non vident et audientes non audiunt neque intellegunt; 
\verse et adimpletur eis prophetia Isaiae dicens: "Auditu audietis et non intellegetis et videntes videbitis et non videbitis. 
\verse Incrassatum est enim cor populi huius, et auribus graviter audierunt et oculos suos clauserunt, ne quando oculis videant et auribus audiant et corde intellegant et convertantur, et sanem eos". 
\verse Vestri autem beati oculi, quia vident, et aures vestrae, quia audiunt. 
\verse Amen quippe dico vobis: Multi prophetae et iusti cupierunt videre, quae videtis, et non viderunt, et audire, quae auditis, et non audierunt! 
\verse Vos ergo audite parabolam seminantis. 
\verse Omnis, qui audit verbum regni et non intellegit, venit Malus et rapit, quod seminatum est in corde eius; hic est, qui secus viam seminatus est. 
\verse Qui autem supra petrosa seminatus est, hic est, qui verbum audit et continuo cum gaudio accipit illud, 
\verse non habet autem in se radicem, sed est temporalis; facta autem tribulatione vel persecutione propter verbum, continuo scandalizatur. 
\verse Qui autem est seminatus in spinis, hic est, qui verbum audit, et sollicitudo saeculi et fallacia divitiarum suffocat verbum, et sine fructu efficitur. 
\verse Qui vero in terra bona seminatus est, hic est, qui audit verbum et intellegit et fructum affert et facit aliud quidem centum, aliud autem sexaginta, porro aliud triginta". 
\verse Aliam parabolam proposuit illis dicens: “Simile factum est regnum caelorum homini, qui seminavit bonum semen in agro suo. 
\verse Cum autem dormirent homines, venit inimicus eius et superseminavit zizania in medio tritici et abiit. 
\verse Cum autem crevisset herba et fructum fecisset, tunc apparuerunt et zizania. 
\verse Accedentes autem servi patris familias dixerunt ei: “Domine, nonne bonum semen seminasti in agro tuo? Unde ergo habet zizania?”. 
\verse Et ait illis: “Inimicus homo hoc fecit”. Servi autem dicunt ei: “Vis, imus et colligimus ea?”. 
\verse Et ait: “Non; ne forte colligentes zizania eradicetis simul cum eis triticum, 
\verse sinite utraque crescere usque ad messem. Et in tempore messis dicam messoribus: Colligite primum zizania et alligate ea in fasciculos ad comburendum ea, triticum autem congregate in horreum meum”". 
\verse Aliam parabolam proposuit eis dicens: “Simile est regnum caelorum grano sinapis, quod accipiens homo seminavit in agro suo. 
\verse Quod minimum quidem est omnibus seminibus; cum autem creverit, maius est holeribus et fit arbor, ita ut volucres caeli veniant et habitent in ramis eius". 
\verse Aliam parabolam locutus est eis: “Simile est regnum caelorum fermento, quod acceptum mulier abscondit in farinae satis tribus, donec fermentatum est totum". 
\verse Haec omnia locutus est Iesus in parabolis ad turbas; et sine parabola nihil loquebatur eis, 
\verse ut adimpleretur, quod dictum erat per prophetam dicentem: “Aperiam in parabolis os meum, eructabo abscondita a constitutione mundi". 
\verse Tunc, dimissis turbis, venit in domum, et accesserunt ad eum discipuli eius dicentes: “Dissere nobis parabolam zizaniorum agri". 
\verse Qui respondens ait: “Qui seminat bonum semen, est Filius hominis; 
\verse ager autem est mundus; bonum vero semen, hi sunt filii regni; zizania autem filii sunt Mali;  
\verse inimicus autem, qui seminavit ea, est Diabolus; messis vero consummatio saeculi est; messores autem angeli sunt. 
\verse Sicut ergo colliguntur zizania et igni comburuntur, sic erit in consummatione saeculi: 
\verse mittet Filius hominis angelos suos, et colligent de regno eius omnia scandala et eos, qui faciunt iniquitatem, 
\verse et mittent eos in caminum ignis; ibi erit fletus et stridor dentium.  
\verse Tunc iusti fulgebunt sicut sol in regno Pa tris eorum. Qui habet aures, audiat. 
\verse Simile est regnum caelorum thesauro abscondito in agro; quem qui invenit homo abscondit et prae gaudio illius vadit et vendit universa, quae habet, et emit agrum illum. 
\verse Iterum simile est regnum caelorum homini negotiatori quaerenti bonas margaritas. 
\verse Inventa autem una pretiosa margarita, abiit et vendidit omnia, quae habuit, et emit eam. 
\verse Iterum simile est regnum caelorum sagenae missae in mare et ex omni genere congreganti; 
\verse quam, cum impleta esset, educentes secus litus et sedentes collegerunt bonos in vasa, malos autem foras miserunt. 
\verse Sic erit in consummatione saeculi: exibunt angeli et separabunt malos de medio iustorum  
\verse et mittent eos in caminum ignis; ibi erit fletus et stridor dentium. 
\verse Intellexistis haec omnia?". Dicunt ei: “Etiam". 
\verse Ait autem illis: “Ideo omnis scriba doctus in regno caelorum similis est homini patri familias, qui profert de thesauro suo nova et vetera". 
\verse Et factum est, cum consummasset Iesus parabolas istas, transiit inde. 
\verse Et veniens in patriam suam, docebat eos in synagoga eorum, ita ut mirarentur et dicerent: “Unde huic sapientia haec et virtutes? 
\verse Nonne hic est fabri filius? Nonne mater eius dicitur Maria, et fratres eius Iacobus et Ioseph et Simon et Iudas? 
\verse Et sorores eius nonne omnes apud nos sunt? Unde ergo huic omnia ista?". 
\verse Et scandalizabantur in eo. Iesus autem dixit eis: “Non est propheta sine honore nisi in patria et in domo sua". 
\verse Et non fecit ibi virtutes multas propter incredulitatem illorum. 
\end{biblechapter}

\begin{biblechapter}  
\verse In illo tempore audivit Herodes tetrarcha famam Iesu 
\verse et ait pueris suis: “Hic est Ioannes Baptista; ipse surrexit a mortuis, et ideo virtutes operantur in eo". 
\verse Herodes enim tenuit Ioannem et alligavit eum et posuit in carcere propter Herodiadem uxorem Philippi fratris sui. 
\verse Dicebat enim illi Ioannes: “Non licet tibi habere eam". 
\verse Et volens illum occidere, timuit populum, quia sicut prophetam eum habebant. 
\verse Die autem natalis Herodis saltavit filia Herodiadis in medio et placuit Herodi, 
\verse unde cum iuramento pollicitus est ei dare, quodcumque postulasset.  
\verse At illa, praemonita a matre sua: “Da mihi, inquit, hic in disco caput Ioannis Baptistae". 
\verse Et contristatus rex propter iuramentum et eos, qui pariter recumbebant, iussit dari 
\verse misitque et decollavit Ioannem in carcere;  
\verse et allatum est caput eius in disco et datum est puellae, et tulit matri suae.  
\verse Et accedentes discipuli eius tulerunt corpus et sepelierunt illud et venientes nuntiaverunt Iesu. 
\verse Quod cum audisset Iesus, secessit inde in navicula in locum desertum seorsum; et cum audissent, turbae secutae sunt eum pedestres de civitatibus. 
\verse Et exiens vidit turbam multam et misertus est eorum et curavit languidos eorum.  
\verse Vespere autem facto, accesserunt ad eum discipuli dicentes: “Desertus est locus, et hora iam praeteriit; dimitte turbas, ut euntes in castella emant sibi escas". 
\verse Iesus autem dixit eis: “Non habent necesse ire; date illis vos manducare". 
\verse Illi autem dicunt ei: “Non habemus hic nisi quinque panes et duos pisces". 
\verse Qui ait: “Afferte illos mihi huc". 
\verse Et cum iussisset turbas discumbere supra fenum, acceptis quinque panibus et duobus piscibus, aspiciens in caelum benedixit et fregit et dedit discipulis panes, discipuli autem turbis. 
\verse Et manducaverunt omnes et saturati sunt; et tulerunt reliquias fragmentorum duodecim cophinos plenos. 
\verse Manducantium autem fuit numerus fere quinque milia virorum, exceptis mulieribus et parvulis. 
\verse Et statim iussit discipulos ascendere in naviculam et praecedere eum trans fretum, donec dimitteret turbas. 
\verse Et dimissis turbis, ascendit in montem solus orare. Vespere autem facto, solus erat ibi. 
\verse Navicula autem iam multis stadiis a terra distabat, fluctibus iactata; erat enim contrarius ventus. 
\verse Quarta autem vigilia noctis venit ad eos ambulans supra mare. 
\verse Discipuli autem, videntes eum supra mare ambulantem, turbati sunt dicentes: “Phantasma est", et prae timore clamaverunt. 
\verse Statimque Iesus locutus est eis dicens: “Habete fiduciam, ego sum; nolite timere!".  
\verse Respondens autem ei Petrus dixit: “Domine, si tu es, iube me venire ad te super aquas". 
\verse At ipse ait: “Veni!". Et descendens Petrus de navicula ambulavit super aquas et venit ad Iesum. 
\verse Videns vero ventum validum timuit et, cum coepisset mergi, clamavit dicens: “Domine, salvum me fac!".  
\verse Continuo autem Iesus extendens manum apprehendit eum et ait illi: “Modicae fidei, quare dubitasti?". 
\verse Et cum ascendissent in naviculam, cessavit ventus. 
\verse Qui autem in navicula erant, adoraverunt eum dicentes: “Vere Filius Dei es!". 
\verse Et cum transfretassent, venerunt in terram Gennesaret. 
\verse Et cum cognovissent eum viri loci illius, miserunt in universam regionem illam et obtulerunt ei omnes male habentes, 
\verse et rogabant eum, ut vel fimbriam vestimenti eius tangerent; et, quicumque tetigerunt, salvi facti sunt. 
\end{biblechapter}

\begin{biblechapter}  
\verse Tunc accedunt ad Iesum ab Hierosolymis pharisaei et scribae dicentes:  
\verse “Quare discipuli tui transgrediuntur traditionem seniorum? Non enim lavant manus suas, cum panem manducant". 
\verse Ipse autem respondens ait illis: “Quare et vos transgredimini mandatum Dei propter traditionem vestram? 
\verse Nam Deus dixit: "Honora patrem tuum et matrem" et: "Qui maledixerit patri vel matri, morte moriatur". 
\verse Vos autem dicitis: “Quicumque dixerit patri vel matri: Munus est, quodcumque ex me profuerit, 
\verse non honorificabit patrem suum”; et irritum fecistis verbum Dei propter traditionem vestram. 
\verse Hypocritae! Bene prophetavit de vobis Isaias dicens: 
\verse "Populus hic labiis me honorat, cor autem eorum longe est a me; 
\verse sine causa autem colunt me docentes doctrinas mandata homi num"". 
\verse Et convocata ad se turba, dixit eis: “Audite et intellegite: 
\verse Non quod intrat in os, coinquinat hominem; sed quod procedit ex ore, hoc coinquinat hominem!". 
\verse Tunc accedentes discipuli dicunt ei: “Scis quia pharisaei, audito verbo, scandalizati sunt?". 
\verse At ille respondens ait: “Omnis plantatio, quam non plantavit Pater meus caelestis, eradicabitur. 
\verse Sinite illos: caeci sunt, duces caecorum. Caecus autem si caeco ducatum praestet, ambo in foveam cadent".  
\verse Respondens autem Petrus dixit ei: “Edissere nobis parabolam istam". 
\verse At ille dixit: “Adhuc et vos sine intellectu estis? 
\verse Non intellegitis quia omne quod in os intrat, in ventrem vadit et in secessum emittitur? 
\verse Quae autem procedunt de ore, de corde exeunt, et ea coinquinant hominem. 
\verse De corde enim exeunt cogitationes malae, homicidia, adulteria, fornicationes, furta, falsa testimonia, blasphemiae. 
\verse Haec sunt, quae coinquinant hominem; non lotis autem manibus manducare non coinquinat hominem". 
\verse Et egressus inde Iesus, secessit in partes Tyri et Sidonis. 
\verse Et ecce mulier Chananaea a finibus illis egressa clamavit dicens: “Miserere mei, Domine, fili David! Filia mea male a daemonio vexatur". 
\verse Qui non respondit ei verbum. Et accedentes discipuli eius rogabant eum dicentes: “Dimitte eam, quia clamat post nos". 
\verse Ipse autem respondens ait: “Non sum missus nisi ad oves, quae perierunt domus Israel". 
\verse At illa venit et adoravit eum dicens: “Domine, adiuva me!". 
\verse Qui respondens ait: “Non est bonum sumere panem filiorum et mittere catellis". 
\verse At illa dixit: “Etiam, Domine, nam et catelli edunt de micis, quae cadunt de mensa dominorum suorum". 
\verse Tunc respondens Iesus ait illi: “O mulier, magna est fides tua! Fiat tibi, sicut vis". Et sanata est filia illius ex illa hora. 
\verse Et cum transisset inde, Iesus venit secus mare Galilaeae et ascendens in montem sedebat ibi. 
\verse Et accesserunt ad eum turbae multae habentes secum claudos, caecos, debiles, mutos et alios multos et proiecerunt eos ad pedes eius, et curavit eos, 
\verse ita ut turba miraretur videntes mutos loquentes, debiles sanos et claudos ambulantes et caecos videntes. Et magnificabant Deum Israel. 
\verse Iesus autem convocatis discipulis suis dixit: “Misereor turbae, quia triduo iam perseverant mecum et non habent, quod manducent; et dimittere eos ieiunos nolo, ne forte deficiant in via". 
\verse Et dicunt ei discipuli: “Unde nobis in deserto panes tantos, ut saturemus turbam tantam?". 
\verse Et ait illis Iesus: “Quot panes habetis?". At illi dixerunt: “Septem et paucos pisciculos". 
\verse Et praecepit turbae, ut discumberet super terram; 
\verse et accipiens septem panes et pisces et gratias agens fregit et dedit discipulis, discipuli autem turbis. 
\verse Et comederunt omnes et saturati sunt; et, quod superfuit de fragmentis, tulerunt septem sportas plenas. 
\verse Erant autem, qui manducaverant, quattuor milia hominum extra mulieres et parvulos. 
\verse Et dimissis turbis, ascendit in naviculam et venit in fines Magadan. 
\end{biblechapter}

\begin{biblechapter}  
\verse Et accesserunt ad eum pharisaei et sadducaei tentantes et rogaverunt eum, ut signum de caelo ostenderet eis. 
\verse At ille respondens ait eis: “Facto vespere dicitis: “Serenum erit, rubicundum est enim caelum”; 
\verse et mane: “Hodie tempestas, rutilat enim triste caelum”. Faciem quidem caeli diiudicare nostis, signa autem temporum non potestis. 
\verse Generatio mala et adultera signum quaerit, et signum non dabitur ei, nisi signum Ionae". Et, relictis illis, abiit. 
\verse Et cum venissent discipuli trans fretum, obliti sunt panes accipere. 
\verse Iesus autem dixit illis: “Intuemini et cavete a fermento pharisaeorum et sadducaeorum". 
\verse At illi cogitabant inter se dicentes: “Panes non accepimus!”. 
\verse Sciens autem Iesus dixit: “Quid cogitatis inter vos, modicae fidei, quia panes non habetis? 
\verse Nondum intellegitis neque recordamini quinque panum quinque milium hominum, et quot cophinos sumpsistis? 
\verse Neque septem panum quattuor milium hominum, et quot sportas sumpsistis? 
\verse Quomodo non intellegitis quia non de panibus dixi vobis? Sed cavete a fermento pharisaeorum et sadducaeorum". 
\verse Tunc intellexerunt quia non dixerit cavendum a fermento panum sed a doctrina pharisaeorum et sadducaeorum. 
\verse Venit autem Iesus in partes Caesareae Philippi et interrogabat discipulos suos dicens: “Quem dicunt homines esse Filium hominis?”. 
\verse At illi dixerunt: “Alii Ioannem Baptistam, alii autem Eliam, alii vero Ieremiam, aut unum ex prophetis". 
\verse Dicit illis: “Vos autem quem me esse dicitis?". 
\verse Respondens Simon Petrus dixit: “Tu es Christus, Filius Dei vivi".  
\verse Respondens autem Iesus dixit ei: “Beatus es, Simon Bariona, quia caro et sanguis non revelavit tibi sed Pater meus, qui in caelis est. 
\verse Et ego dico tibi: Tu es Petrus, et super hanc petram aedificabo Ecclesiam meam; et portae inferi non praevalebunt adversum eam. 
\verse Tibi dabo claves regni caelorum; et quodcumque ligaveris super terram, erit ligatum in caelis, et quodcumque solveris super terram, erit solutum in caelis". 
\verse Tunc praecepit discipulis, ut nemini dicerent quia ipse esset Christus. 
\verse Exinde coepit Iesus ostendere discipulis suis quia oporteret eum ire Hierosolymam et multa pati a senioribus et principibus sacerdotum et scribis et occidi et tertia die resurgere. 
\verse Et assumens eum Petrus coepit increpare illum dicens: “Absit a te, Domine; non erit tibi hoc". 
\verse Qui conversus dixit Petro: “Vade post me, Satana! Scandalum es mihi, quia non sapis ea, quae Dei sunt, sed ea, quae hominum!". 
\verse Tunc Iesus dixit discipulis suis: “Si quis vult post me venire, abneget semetipsum et tollat crucem suam et sequatur me. 
\verse Qui enim voluerit animam suam salvam facere, perdet eam; qui autem perdiderit animam suam propter me, inveniet eam. 
\verse Quid enim prodest homini, si mundum universum lucretur, animae vero suae detrimentum patiatur? Aut quam dabit homo commutationem pro anima sua? 
\verse Filius enim hominis venturus est in gloria Patris sui cum angelis suis, et tunc reddet unicuique secundum opus eius. 
\verse Amen dico vobis: Sunt quidam de hic stantibus, qui non gustabunt mortem, donec videant Filium hominis venientem in regno suo". 
\end{biblechapter}

\begin{biblechapter}  
\verse Et post dies sex assumit Iesus Petrum et Iacobum et Ioannem fratrem eius et ducit illos in montem excelsum seorsum. 
\verse Et transfiguratus est ante eos; et resplenduit facies eius sicut sol, vestimenta autem eius facta sunt alba sicut lux. 
\verse Et ecce apparuit illis Moyses et Elias cum eo loquentes. 
\verse Respondens autem Petrus dixit ad Iesum: “Domine, bonum est nos hic esse. Si vis, faciam hic tria tabernacula: tibi unum et Moysi unum et Eliae unum". 
\verse Adhuc eo loquente, ecce nubes lucida obumbravit eos; et ecce vox de nube dicens: “Hic est Filius meus dilectus, in quo mihi bene complacui; ipsum audite". 
\verse Et audientes discipuli ceciderunt in faciem suam et timuerunt valde. 
\verse Et accessit Iesus et tetigit eos dixitque eis: “Surgite et nolite timere".  
\verse Levantes autem oculos suos, neminem viderunt nisi solum Iesum. 
\verse Et descendentibus illis de monte, praecepit eis Iesus dicens: “Nemini dixeritis visionem, donec Filius hominis a mortuis resurgat". 
\verse Et interrogaverunt eum discipuli dicentes: “Quid ergo scribae dicunt quod Eliam oporteat primum venire?". 
\verse At ille respondens ait: “Elias quidem venturus est et restituet omnia. 
\verse Dico autem vobis quia Elias iam venit, et non cognoverunt eum, sed fecerunt in eo, quaecumque voluerunt; sic et Filius hominis passurus est ab eis". 
\verse Tunc intellexerunt discipuli quia de Ioanne Baptista dixisset eis. 
\verse Et cum venissent ad turbam, accessit ad eum homo genibus provolutus ante eum 
\verse et dicens: “Domine, miserere filii mei, quia lunaticus est et male patitur; nam saepe cadit in ignem et crebro in aquam. 
\verse Et obtuli eum discipulis tuis, et non potuerunt curare eum". 
\verse Respondens autem Iesus ait: “O generatio incredula et perversa, quousque ero vobiscum? Usquequo patiar vos? Afferte huc illum ad me". 
\verse Et increpavit eum Iesus, et exiit ab eo daemonium, et curatus est puer ex illa hora. 
\verse Tunc accesserunt discipuli ad Iesum secreto et dixerunt: “Quare nos non potuimus eicere illum?". 
\verse Ille autem dicit illis: “Propter modicam fidem vestram. Amen quippe dico vobis: Si habueritis fidem sicut granum sinapis, dicetis monti huic: “Transi hinc illuc!”, et transibit, et nihil impossibile erit vobis".(21) 
\verse Conversantibus autem eis in Galilaea, dixit illis Iesus: “Filius hominis tradendus est in manus hominum, 
\verse et occident eum, et tertio die resurget". Et contristati sunt vehementer. 
\verse Et cum venissent Capharnaum, accesserunt, qui didrachma accipiebant, ad Petrum et dixerunt: “Magister vester non solvit didrachma?". 
\verse Ait: “Etiam”. Et cum intrasset domum, praevenit eum Iesus dicens: “Quid tibi videtur, Simon? Reges terrae a quibus accipiunt tributum vel censum? A filiis suis an ab alienis?". 
\verse Cum autem ille dixisset: “Ab alienis", dixit illi Iesus: “Ergo liberi sunt filii. 
\verse Ut autem non scandalizemus eos, vade ad mare et mitte hamum; et eum piscem, qui primus ascenderit, tolle; et, aperto ore, eius invenies staterem. Illum sumens, da eis pro me et te". 
\end{biblechapter}

\begin{biblechapter}  
\verse In illa hora accesserunt discipuli ad Iesum dicentes: “Quis putas maior est in regno caelorum?". 
\verse Et advocans parvulum, statuit eum in medio eorum  
\verse et dixit: “Amen dico vobis: Nisi conversi fueritis et efiiciamini sicut parvuli, non intrabitis in regnum caelorum. 
\verse Quicumque ergo humiliaverit se sicut parvulus iste, hic est maior in regno caelorum. 
\verse Et, qui susceperit unum parvulum talem in nomine meo, me suscipit. 
\verse Qui autem scandalizaverit unum de pusillis istis, qui in me credunt, expedit ei, ut suspendatur mola asinaria in collo eius et demergatur in profundum maris. 
\verse Vae mundo ab scandalis! Necesse est enim ut veniant scandala; verumtamen vae homini, per quem scandalum venit! 
\verse Si autem manus tua vel pes tuus scandalizat te, abscide eum et proice abs te: bonum tibi est ad vitam ingredi debilem vel claudum, quam duas manus vel duos pedes habentem mitti in ignem aeternum. 
\verse Et si oculus tuus scandalizat te, erue eum et proice abs te: bonum tibi est unoculum in vitam intrare, quam duos oculos habentem mitti in gehennam ignis. 
\verse Videte, ne contemnatis unum ex his pusillis; dico enim vobis quia angeli eorum in caelis semper vident faciem Patris mei, qui in caelis est. (11) 
\verse Quid vobis videtur? Si fuerint alicui centum oves, et erraverit una ex eis, nonne relinquet nonaginta novem in montibus et vadit quaerere eam, quae erravit? 
\verse Et si contigerit ut inveniat eam, amen dico vobis quia gaudebit super eam magis quam super nonaginta novem, quae non erraverunt. 
\verse Sic non est voluntas ante Patrem vestrum, qui in caelis est, ut pereat unus de pusillis istis. 
\verse Si autem peccaverit in te frater tuus, vade, corripe eum inter te et ipsum solum. Si te audierit, lucratus es fratrem tuum; 
\verse si autem non audierit, adhibe tecum adhuc unum vel duos, ut in ore duorum testium vel trium stet omne verbum; 
\verse quod si noluerit audire eos, dic ecclesiae; si autem et ecclesiam noluerit audire, sit tibi sicut ethnicus et publicanus. 
\verse Amen dico vobis: Quaecumque alligaveritis super terram, erunt ligata in caelo; et, quaecumque solveritis super terram, erunt soluta in caelo. 
\verse Iterum dico vobis: Si duo ex vobis consenserint super terram de omni re, quamcumque petierint, fiet illis a Patre meo, qui in caelis est. 
\verse Ubi enim sunt duo vel tres congregati in nomine meo, ibi sum in medio eorum". 
\verse Tunc accedens Petrus dixit ei: “Domine, quotiens peccabit in me frater meus, et dimittam ei? Usque septies?". 
\verse Dicit illi Iesus: “Non dico tibi usque septies sed usque septuagies septies. 
\verse Ideo assimilatum est regnum caelorum homini regi, qui voluit rationem ponere cum servis suis. 
\verse Et cum coepisset rationem ponere, oblatus est ei unus, qui debebat decem milia talenta. 
\verse Cum autem non haberet, unde redderet, iussit eum dominus venumdari et uxorem et filios et omnia, quae habebat, et reddi. 
\verse Procidens igitur servus ille adorabat eum dicens: “Patientiam habe in me, et omnia reddam tibi”. 
\verse Misertus autem dominus servi illius dimisit eum et debitum dimisit ei. 
\verse Egressus autem servus ille invenit unum de conservis suis, qui debebat ei centum denarios, et tenens suffocabat eum dicens: “Redde, quod debes!”. 
\verse Procidens igitur conservus eius rogabat eum dicens: “Patientiam habe in me, et reddam tibi”. 
\verse Ille autem noluit, sed abiit et misit eum in carcerem, donec redderet debitum. 
\verse Videntes autem conservi eius, quae fiebant, contristati sunt valde et venerunt et narraverunt domino suo omnia, quae facta erant. 
\verse Tunc vocavit illum dominus suus et ait illi: “Serve nequam, omne debitum illud dimisi tibi, quoniam rogasti me; 
\verse non oportuit et te misereri conservi tui, sicut et ego tui misertus sum?”. 
\verse Et iratus dominus eius tradidit eum tortoribus, quoadusque redderet universum debitum. 
\verse Sic et Pater meus caelestis faciet vobis, si non remiseritis unusquisque fratri suo de cordibus vestris". 
\end{biblechapter}

\begin{biblechapter}  
\verse Et factum est, cum consummasset Iesus sermones istos, migravit a Galilaea et venit in fines Iudaeae trans Iordanem. 
\verse Et secutae sunt eum turbae multae, et curavit eos ibi. 
\verse Et accesserunt ad eum pharisaei tentantes eum et dicentes: “Licet homini dimittere uxorem suam quacumque ex causa?". 
\verse Qui respondens ait: “Non legistis quia, qui creavit ab initio, masculum et feminam fecit eos 
\verse et dixit: "Propter hoc dimittet homo patrem et matrem et adhaerebit uxori suae, et erunt duo in carne una?". 
\verse Itaque iam non sunt duo sed una caro. Quod ergo Deus coniunxit, homo non separet". 
\verse Dicunt illi: “Quid ergo Moyses mandavit dari libellum repudii et dimittere?". 
\verse Ait illis: “Moyses ad duritiam cordis vestri permisit vobis dimittere uxores vestras; ab initio autem non sic fuit. 
\verse Dico autem vobis quia quicumque dimiserit uxorem suam, nisi ob fornicationem, et aliam duxerit, moechatur". 
\verse Dicunt ei discipuli eius: “Si ita est causa hominis cum uxore, non expedit nubere". 
\verse Qui dixit eis: “Non omnes capiunt verbum istud, sed quibus datum est. 
\verse Sunt enim eunuchi, qui de matris utero sic nati sunt; et sunt eunuchi, qui facti sunt ab hominibus; et sunt eunuchi, qui seipsos castraverunt propter regnum caelorum. Qui potest capere, capiat". 
\verse Tunc oblati sunt ei parvuli, ut manus eis imponeret et oraret; discipuli autem increpabant eis. 
\verse Iesus vero ait: “Sinite parvulos et nolite eos prohibere ad me venire; talium est enim regnum caelorum". 
\verse Et cum imposuisset eis manus, abiit inde. 
\verse Et ecce unus accedens ait illi: “Magister, quid boni faciam, ut habeam vitam aeternam?". Qui dixit ei: 
\verse “Quid me interrogas de bono? Unus est bonus. Si autem vis ad vitam ingredi, serva mandata". 
\verse Dicit illi: “Quae?". Iesus autem dixit: “Non homicidium facies, non adulterabis, non facies furtum, non falsum testimonium dices, 
\verse honora patrem et matrem et diliges proximum tuum sicut teipsum". 
\verse Dicit illi adulescens: “Omnia haec custodivi. Quid adhuc mihi deest?". 
\verse Ait illi Iesus: “Si vis perfectus esse, vade, vende, quae habes, et da pauperibus, et habebis thesaurum in caelo; et veni, sequere me". 
\verse Cum audisset autem adulescens verbum, abiit tristis; erat enim habens multas possessiones. 
\verse Iesus autem dixit discipulis suis: “Amen dico vobis: Dives difficile intrabit in regnum caelorum. 
\verse Et iterum dico vobis: Facilius est camelum per foramen acus transire, quam divitem intrare in regnum Dei". 
\verse Auditis autem his, discipuli mirabantur valde dicentes: “Quis ergo poterit salvus esse?". 
\verse Aspiciens autem Iesus dixit illis: “Apud homines hoc impossibile est, apud Deum autem omnia possibilia sunt". 
\verse Tunc respondens Petrus dixit ei: “Ecce nos reliquimus omnia et secuti sumus te. Quid ergo erit nobis?". 
\verse Iesus autem dixit illis: “Amen dico vobis quod vos, qui secuti estis me, in regeneratione, cum sederit Filius hominis in throno gloriae suae, sedebitis et vos super thronos duodecim, iudicantes duodecim tribus Israel. 
\verse Et omnis, qui reliquit domos vel fratres aut sorores aut patrem aut matrem aut filios aut agros propter nomen meum, centuplum accipiet et vitam aeternam possidebit. 
\verse Multi autem erunt primi novissimi, et novissimi primi. 
\end{biblechapter}

\begin{biblechapter}  
\verse Simile est enim regnum caelorum homini patri familias, qui exiit primo mane conducere operarios in vineam suam; 
\verse conventione autem facta cum operariis ex denario diurno, misit eos in vineam suam. 
\verse Et egressus circa horam tertiam vidit alios stantes in foro otiosos 
\verse et illis dixit: “Ite et vos in vineam; et, quod iustum fuerit, dabo vobis”. 
\verse Illi autem abierunt. Iterum autem exiit circa sextam et nonam horam et fecit similiter. 
\verse Circa undecimam vero exiit et invenit alios stantes et dicit illis: “Quid hic statis tota die otiosi?”. 
\verse Dicunt ei: “Quia nemo nos conduxit”. Dicit illis: “Ite et vos in vineam”. 
\verse Cum sero autem factum esset, dicit dominus vineae procuratori suo: “Voca operarios et redde illis mercedem incipiens a novissimis usque ad primos".  
\verse Et cum venissent, qui circa undecimam horam venerant, acceperunt singuli denarium. 
\verse Venientes autem primi arbitrati sunt quod plus essent accepturi; acceperunt autem et ipsi singuli denarium. 
\verse Accipientes autem murmurabant adversus patrem familias 
\verse dicentes: “Hi novissimi una hora fecerunt, et pares illos nobis fecisti, qui portavimus pondus diei et aestum!”.  
\verse At ille respondens uni eorum dixit: “Amice, non facio tibi iniuriam; nonne ex denario convenisti mecum? 
\verse Tolle, quod tuum est, et vade; volo autem et huic novissimo dare sicut et tibi. 
\verse Aut non licet mihi, quod volo, facere de meis? An oculus tuus nequam est, quia ego bonus sum?”. 
\verse Sic erunt novissimi primi, et primi novissimi". 
\verse Et ascendens Iesus Hierosolymam assumpsit Duodecim discipulos secreto et ait illis in via: 
\verse “Ecce ascendimus Hierosolymam, et Filius hominis tradetur principibus sacerdotum et scribis, et condemnabunt eum morte 
\verse et tradent eum gentibus ad illudendum et flagellandum et crucifigendum, et tertia die resurget". 
\verse Tunc accessit ad eum mater filiorum Zebedaei cum filiis suis, adorans et petens aliquid ab eo. 
\verse Qui dixit ei: “Quid vis?". Ait illi: “Dic ut sedeant hi duo filii mei unus ad dexteram tuam et unus ad sinistram in regno tuo". 
\verse Respondens autem Iesus dixit: “Nescitis quid petatis. Potestis bibere calicem, quem ego bibiturus sum?". Dicunt ei: “Possumus". 
\verse Ait illis: “Calicem quidem meum bibetis, sedere autem ad dexteram meam et sinistram non est meum dare illud, sed quibus paratum est a Patre meo". 
\verse Et audientes decem indignati sunt de duobus fratribus. 
\verse Iesus autem vocavit eos ad se et ait: “Scitis quia principes gentium dominantur eorum et, qui magni sunt, potestatem exercent in eos. 
\verse Non ita erit inter vos, sed quicumque voluerit inter vos magnus fieri, erit vester minister; 
\verse et, quicumque voluerit inter vos primus esse, erit vester servus; 
\verse sicut Filius hominis non venit ministrari sed ministrare et dare animam suam redemptionem pro multis". 
\verse Et egredientibus illis ab Iericho, secuta est eum turba multa. 
\verse Et ecce duo caeci sedentes secus viam audierunt quia Iesus transiret et clamaverunt dicentes: “Domine, miserere nostri, fili David!". 
\verse Turba autem increpabat eos, ut tacerent; at illi magis clamabant dicentes: “Domine, miserere nostri, fili David!". 
\verse Et stetit Iesus et vocavit eos et ait: “Quid vultis, ut faciam vobis?”. 
\verse Dicunt illi: “Domine, ut aperiantur oculi nostri". 
\verse Misertus autem Iesus, tetigit oculos eorum; et confestim viderunt et secuti sunt eum. 
\end{biblechapter}

\begin{biblechapter}  
\verse Et cum appropinquassent Hierosolymis et venissent Bethfage, ad montem Oliveti, tunc Iesus misit duos discipulos 
\verse dicens eis: “Ite in castellum, quod contra vos est, et statim invenietis asinam alligatam et pullum cum ea; solvite et adducite mihi. 
\verse Et si quis vobis aliquid dixerit, dicite: “Dominus eos necessarios habet”, et confestim dimittet eos". 
\verse Hoc autem factum est, ut impleretur, quod dictum est per prophetam dicentem: 
\verse “Dicite filiae Sion: Ecce Rex tuus venit tibi, mansuetus et sedens super asinam et super pullum filium subiugalis". 
\verse Euntes autem discipuli fecerunt, sicut praecepit illis Iesus, 
\verse et adduxerunt asinam et pullum; et imposuerunt super eis vestimenta sua, et sedit super ea. 
\verse Plurima autem turba straverunt vestimenta sua in via; alii autem caedebant ramos de arboribus et sternebant in via. 
\verse Turbae autem, quae praecedebant eum et quae sequebantur, clamabant dicentes: “Hosanna filio David! Benedictus, qui venit in nomine Domini! Hosanna in altissimis!". 
\verse Et cum intrasset Hierosolymam, commota est universa civitas dicens: “Quis est hic?”. 
\verse Turbae autem dicebant: “Hic est Iesus propheta a Nazareth Galilaeae". 
\verse Et intravit Iesus in templum et eiciebat omnes vendentes et ementes in templo, et mensas nummulariorum evertit et cathedras vendentium columbas, 
\verse et dicit eis: “Scriptum est: "Domus mea domus orationis vocabitur". Vos autem facitis eam speluncam latronum". 
\verse Et accesserunt ad eum caeci et claudi in templo, et sanavit eos. 
\verse Videntes autem principes sacerdotum et scribae mirabilia, quae fecit, et pueros clamantes in templo et dicentes: “Hosanna filio David", indignati sunt 
\verse et dixerunt ei: “Audis quid isti dicant?". Iesus autem dicit eis: “Utique; numquam legistis: "Ex ore infantium et lactantium perfecisti laudem"?". 
\verse Et relictis illis, abiit foras extra civitatem in Bethaniam ibique mansit. 
\verse Mane autem revertens in civitatem, esuriit. 
\verse Et videns fici arborem unam secus viam, venit ad eam; et nihil invenit in ea nisi folia tantum et ait illi: “Numquam ex te fructus nascatur in sempiternum". Et arefacta est continuo ficulnea. 
\verse Et videntes discipuli mirati sunt dicentes: “Quomodo continuo aruit ficulnea?". 
\verse Respondens autem Iesus ait eis: “Amen dico vobis: Si habueritis fidem et non haesitaveritis, non solum de ficulnea facietis, sed et si monti huic dixeritis: “Tolle et iacta te in mare”, fiet.  
\verse Et omnia, quaecumque petieritis in oratione credentes, accipietis". 
\verse Et cum venisset in templum, accesserunt ad eum docentem principes sacerdotum et seniores populi dicentes: “In qua potestate haec facis? Et quis tibi dedit hanc potestatem?". 
\verse Respondens autem Iesus dixit illis: “Interrogabo vos et ego unum sermonem, quem si dixeritis mihi, et ego vobis dicam, in qua potestate haec facio: 
\verse Baptismum Ioannis unde erat? A caelo an ex hominibus?". At illi cogitabant inter se dicentes: “Si dixerimus: “E caelo”, dicet nobis: “Quare ergo non credidistis illi?”; 
\verse si autem dixerimus: “Ex hominibus”, timemus turbam; omnes enim habent Ioannem sicut prophetam". 
\verse Et respondentes Iesu dixerunt: “Nescimus". Ait illis et ipse: “Nec ego dico vobis in qua potestate haec facio". 
\verse “Quid autem vobis videtur? Homo quidam habebat duos filios. Et accedens ad primum dixit: “Fili, vade hodie, operare in vinea”. 
\verse Ille autem respondens ait: “Nolo”; postea autem paenitentia motus abiit. 
\verse Accedens autem ad alterum dixit similiter. At ille respondens ait: “Eo, domine”; et non ivit. 
\verse Quis ex duobus fecit voluntatem patris?". Dicunt: “Primus". Dicit illis Iesus: “Amen dico vobis: Publicani et meretrices praecedunt vos in regnum Dei. 
\verse Venit enim ad vos Ioannes in via iustitiae, et non credidistis ei; publicani autem et meretrices crediderunt ei. Vos autem videntes nec paenitentiam habuistis postea, ut crederetis ei. 
\verse Aliam parabolam audite. Homo erat pater familias, qui plantavit vineam et saepem circumdedit ei et fodit in ea torcular et aedificavit turrim et locavit eam agricolis et peregre profectus est. 
\verse Cum autem tempus fructuum appropinquasset, misit servos suos ad agricolas, ut acciperent fructus eius.  
\verse Et agricolae, apprehensis servis eius, alium ceciderunt, alium occiderunt, alium vero lapidaverunt. 
\verse Iterum misit alios servos plures prioribus, et fecerunt illis similiter. 
\verse Novissime autem misit ad eos filium suum dicens: “Verebuntur filium meum”. 
\verse Agricolae autem videntes filium dixerunt intra se: “Hic est heres. Venite, occidamus eum et habebimus hereditatem eius”. 
\verse Et apprehensum eum eiecerunt extra vineam et occiderunt. 
\verse Cum ergo venerit dominus vineae, quid faciet agricolis illis?". 
\verse Aiunt illi: “Malos male perdet et vineam locabit aliis agricolis, qui reddant ei fructum temporibus suis". 
\verse Dicit illis Iesus: “Numquam legistis in Scripturis: "Lapidem quem reprobaverunt aedificantes, hic factus est in caput anguli; a Domino factum est istud et est mirabile in oculis nostris" ? 
\verse Ideo dico vobis quia auferetur a vobis regnum Dei et dabitur genti facienti fructus eius. 
\verse Et, qui ceciderit super lapidem istum confringetur; super quem vero ceciderit, conteret eum". 
\verse Et cum audissent principes sacerdotum et pharisaei parabolas eius, cognoverunt quod de ipsis diceret; 
\verse et quaerentes eum tenere, timuerunt turbas, quoniam sicut prophetam eum habebant. 
\end{biblechapter}

\begin{biblechapter}  
\verse Et respondens Iesus dixit iterum in parabolis eis dicens: 
\verse “Simile factum est regnum caelorum homini regi, qui fecit nuptias filio suo. 
\verse Et misit servos suos vocare invitatos ad nuptias, et nolebant venire. 
\verse Iterum misit alios servos dicens: “Dicite invitatis: Ecce prandium meum paravi, tauri mei et altilia occisa, et omnia parata; venite ad nuptias”. 
\verse Illi autem neglexerunt et abierunt, alius in villam suam, alius vero ad negotiationem suam; 
\verse reliqui vero tenuerunt servos eius et contumelia affectos occiderunt. 
\verse Rex autem iratus est et, missis exercitibus suis, perdidit homicidas illos et civitatem illorum succendit. 
\verse Tunc ait servis suis: “Nuptiae quidem paratae sunt, sed qui invitati erant, non fuerunt digni; 
\verse ite ergo ad exitus viarum, et quoscumque inveneritis, vocate ad nuptias”. 
\verse Et egressi servi illi in vias, congregaverunt omnes, quos invenerunt, malos et bonos; et impletae sunt nuptiae discumbentium. 
\verse Intravit autem rex, ut videret discumbentes, et vidit ibi hominem non vestitum veste nuptiali 
\verse et ait illi: “Amice, quomodo huc intrasti, non habens vestem nuptialem?”. At ille obmutuit. 
\verse Tunc dixit rex ministris: “Ligate pedes eius et manus et mittite eum in tenebras exteriores: ibi erit fletus et stridor dentium”. 
\verse Multi enim sunt vocati, pauci vero electi". 
\verse Tunc abeuntes pharisaei consilium inierunt, ut caperent eum in sermone.  
\verse Et mittunt ei discipulos suos cum herodianis dicentes: “Magister, scimus quia verax es et viam Dei in veritate doces, et non est tibi cura de aliquo; non enim respicis personam hominum. 
\verse Dic ergo nobis quid tibi videatur: Licet censum dare Caesari an non?". 
\verse Cognita autem Iesus nequitia eorum, ait: “Quid me tentatis, hypocritae? 
\verse Ostendite mihi nomisma census". At illi obtulerunt ei denarium. 
\verse Et ait illis: “Cuius est imago haec et suprascriptio?". 
\verse Dicunt ei: “Caesaris". Tunc ait illis: “Reddite ergo, quae sunt Caesaris, Caesari et, quae sunt Dei, Deo". 
\verse Et audientes mirati sunt et, relicto eo, abierunt. 
\verse In illo die accesserunt ad eum sadducaei, qui dicunt non esse resurrectionem, et interrogaverunt eum 
\verse dicentes: “Magister, Moyses dixit, si quis mortuus fuerit non habens filios, ut ducat frater eius uxorem illius et suscitet semen fratri suo. 
\verse Erant autem apud nos septem fratres: et primus, uxore ducta, defunctus est et non habens semen reliquit uxorem suam fratri suo; 
\verse similiter secundus et tertius usque ad septimum. 
\verse Novissime autem omnium mulier defuncta est. 
\verse In resurrectione ergo cuius erit de septem uxor? Omnes enim habuerunt eam". 
\verse Respondens autem Iesus ait illis: “Erratis nescientes Scripturas neque virtutem Dei; 
\verse in resurrectione enim neque nubent neque nubentur, sed sunt sicut angeli in caelo. 
\verse De resurrectione autem mortuorum non legistis, quod dictum est vobis a Deo dicente: 
\verse “Ego sum Deus Abraham et Deus Isaac et Deus Iacob”? Non est Deus mortuorum sed viventium". 
\verse Et audientes turbae mirabantur in doctrina eius. 
\verse Pharisaei autem audientes quod silentium imposuisset sadducaeis, convenerunt in unum. 
\verse Et interrogavit unus ex eis legis doctor tentans eum: 
\verse “Magister, quod est mandatum magnum in Lege?". 
\verse Ait autem illi: “Diliges Dominum Deum tuum in toto corde tuo et in tota anima tua et in tota mente tua: 
\verse hoc est magnum et primum mandatum. 
\verse Secundum autem simile est huic: Diliges proximum tuum sicut teipsum. 
\verse In his duobus mandatis universa Lex pendet et Prophetae". 
\verse Congregatis autem pharisaeis, interrogavit eos Iesus 
\verse dicens: “Quid vobis videtur de Christo? Cuius filius est?". Dicunt ei: “David". 
\verse Ait illis: “Quomodo ergo David in Spiritu vocat eum Dominum dicens: 
\verse “Dixit Dominus Domino meo: Sede a dextris meis, donec ponam inimicos tuos sub pedibus tuis”? 
\verse Si ergo David vocat eum Dominum, quomodo filius eius est?". 
\verse Et nemo poterat respondere ei verbum, neque ausus fuit quisquam ex illa die eum amplius interrogare. 
\end{biblechapter}

\begin{biblechapter}  
\verse Tunc Iesus locutus est ad turbas et ad discipulos suos 
\verse dicens: “Super cathedram Moysis sederunt scribae et pharisaei. 
\verse Omnia ergo, quaecumque dixerint vobis, facite et servate; secundum opera vero eorum nolite facere: dicunt enim et non faciunt. 
\verse Alligant autem onera gravia et importabilia et imponunt in umeros hominum, ipsi autem digito suo nolunt ea movere. 
\verse Omnia vero opera sua faciunt, ut videantur ab hominibus: dilatant enim phylacteria sua et magnificant fimbrias, 
\verse amant autem primum recubitum in cenis et primas cathedras in synagogis 
\verse et salutationes in foro et vocari ab hominibus Rabbi. 
\verse Vos autem nolite vocari Rabbi; unus enim est Magister vester, omnes autem vos fratres estis. 
\verse Et Patrem nolite vocare vobis super terram, unus enim est Pater vester, caelestis. 
\verse Nec vocemini Magistri, quia Magister vester unus est, Christus. 
\verse Qui maior est vestrum, erit minister vester. 
\verse Qui autem se exaltaverit, humiliabitur; et, qui se humiliaverit, exaltabitur. 
\verse Vae autem vobis, scribae et pharisaei hypocritae, quia clauditis regnum caelorum ante homines! Vos enim non intratis nec introeuntes sinitis intrare. (14) 
\verse Vae vobis, scribae et pharisaei hypocritae, quia circuitis mare et aridam, ut faciatis unum proselytum, et cum fuerit factus, facitis eum filium gehennae duplo quam vos! 
\verse Vae vobis, duces caeci, qui dicitis: “Quicumque iuraverit per templum, nihil est; quicumque autem iuraverit in auro templi, debet”. 
\verse Stulti et caeci! Quid enim maius est: aurum an templum, quod sanctificat aurum? 
\verse Et: “Quicumque iuraverit in altari, nihil est; quicumque autem iuraverit in dono, quod est super illud, debet”. 
\verse Caeci! Quid enim maius est: donum an altare, quod sanctificat donum? 
\verse Qui ergo iuraverit in altari, iurat in eo et in omnibus, quae super illud sunt; 
\verse et, qui iuraverit in templo, iurat in illo et in eo, qui inhabitat in ipso; 
\verse et, qui iuraverit in caelo, iurat in throno Dei et in eo, qui sedet super eum. 
\verse Vae vobis, scribae et pharisaei hypocritae, quia decimatis mentam et anethum et cyminum et reliquistis, quae graviora sunt legis: iudicium et misericordiam et fidem! Haec oportuit facere et illa non omittere. 
\verse Duces caeci, excolantes culicem, camelum autem glutientes. 
\verse Vae vobis, scribae et pharisaei hypocritae, quia mundatis, quod de foris est calicis et paropsidis, intus autem pleni sunt rapina et immunditia! 
\verse Pharisaee caece, munda prius, quod intus est calicis, ut fiat et id, quod de foris eius est, mundum. 
\verse Vae vobis, scribae et pharisaei hypocritae, quia similes estis sepulcris dealbatis, quae a foris quidem parent speciosa, intus vero plena sunt ossibus mortuorum et omni spurcitia! 
\verse Sic et vos a foris quidem paretis hominibus iusti, intus autem pleni estis hypocrisi et iniquitate. 
\verse Vae vobis, scribae et pharisaei hypocritae, qui aedificatis sepulcra prophetarum et ornatis monumenta iustorum 
\verse et dicitis: “Si fuissemus in diebus patrum nostrorum, non essemus socii eorum in sanguine prophetarum”!  
\verse Itaque testimonio estis vobismetipsis quia filii estis eorum, qui prophetas occiderunt. 
\verse Et vos implete mensuram patrum vestrorum. 
\verse Serpentes, genimina viperarum, quomodo fugietis a iudicio gehennae? 
\verse Ideo ecce ego mitto ad vos prophetas et sapientes et scribas; ex illis occidetis et crucifigetis et ex eis flagellabitis in synagogis vestris et persequemini de civitate in civitatem, 
\verse ut veniat super vos omnis sanguis iustus, qui effusus est super terram a sanguine Abel iusti usque ad sanguinem Zachariae filii Barachiae, quem occidistis inter templum et altare. 
\verse Amen dico vobis: Venient haec omnia super generationem istam. 
\verse Ierusalem, Ierusalem, quae occidis prophetas et lapidas eos, qui ad te missi sunt, quotiens volui congregare filios tuos, quemadmodum gallina congregat pullos suos sub alas, et noluistis! 
\verse Ecce relinquitur vobis domus vestra deserta! 
\verse Dico enim vobis: Non me videbitis amodo, donec dicatis: "Benedictus, qui venit in nomine Dominil"". 
\end{biblechapter}

\begin{biblechapter}  
\verse Et egressus Iesus de templo ibat, et accesserunt discipuli eius, ut ostenderent ei aedificationes templi; 
\verse ipse autem respondens dixit eis: “Non videtis haec omnia? Amen dico vobis: Non relinquetur hic lapis super lapidem, qui non destruetur". 
\verse Sedente autem eo super montem Oliveti, accesserunt ad eum discipuli secreto dicentes: “Dic nobis: Quando haec erunt, et quod signum adventus tui et consummationis saeculi?". 
\verse Et respondens Iesus dixit eis: “Videte, ne quis vos seducat. 
\verse Multi enim venient in nomine meo dicentes: “Ego sum Christus”, et multos seducent. 
\verse Audituri enim estis proelia et opiniones proeliorum. Videte, ne turbemini; oportet enim fieri, sed nondum est finis. 
\verse Consurget enim gens in gentem, et regnum in regnum, et erunt fames et terrae motus per loca; 
\verse haec autem omnia initia sunt dolorum. 
\verse Tunc tradent vos in tribulationem et occident vos, et eritis odio omnibus gentibus propter nomen meum. 
\verse Et tunc scandalizabuntur multi et invicem tradent et odio habebunt invicem; 
\verse et multi pseudoprophetae surgent et seducent multos. 
\verse Et, quoniam abundavit iniquitas, refrigescet caritas multorum; 
\verse qui autem permanserit usque in finem, hic salvus erit. 
\verse Et praedicabitur hoc evangelium regni in universo orbe in testimonium omnibus gentibus; et tunc veniet consummatio. 
\verse Cum ergo videritis abominationem desolationis, quae dicta est a Daniele propheta, stantem in loco sancto, qui legit, intellegat: 
\verse tunc qui in Iudaea sunt, fugiant ad montes; 
\verse qui in tecto, non descendat tollere aliquid de domo sua; 
\verse et, qui in agro, non revertatur tollere pallium suum. 
\verse Vae autem praegnantibus et nutrientibus in illis diebus! 
\verse Orate autem, ut non fiat fuga vestra hieme vel sabbato: 
\verse erit enim tunc tribulatio magna, qualis non fuit ab initio mundi usque modo neque fiet. 
\verse Et nisi breviati fuissent dies illi, non fieret salva omnis caro; sed propter electos breviabuntur dies illi. 
\verse Tunc si quis vobis dixerit: “Ecce hic Christus” aut: “Hic”, nolite credere. 
\verse Surgent enim pseudochristi et pseudoprophetae et dabunt signa magna et prodigia, ita ut in errorem inducantur, si fieri potest, etiam electi. 
\verse Ecce praedixi vobis. 
\verse Si ergo dixerint vobis: “Ecce in deserto est”, nolite exire; “Ecce in penetralibus”, nolite credere; 
\verse sicut enim fulgur exit ab oriente et paret usque in occidentem, ita erit adventus Filii hominis. 
\verse Ubicumque fuerit corpus, illuc congregabuntur aquilae. 
\verse Statim autem post tribulationem dierum illorum, sol obscurabitur, et luna non dabit lumen suum, et stellae cadent de caelo, et virtutes caelorum commovebuntur. 
\verse Et tunc parebit signum Filii hominis in caelo, et tunc plangent omnes tribus terrae et videbunt Filium hominis venientem in nubibus caeli cum virtute et gloria multa; 
\verse et mittet angelos suos cum tuba magna, et congregabunt electos eius a quattuor ventis, a summis caelorum usque ad terminos eorum. 
\verse Ab arbore autem fici discite parabolam: cum iam ramus eius tener fuerit, et folia nata, scitis quia prope est aestas. 
\verse Ita et vos, cum videritis haec omnia, scitote quia prope est in ianuis. 
\verse Amen dico vobis: Non praeteribit haec generatio, donec omnia haec fiant. 
\verse Caelum et terra transibunt, verba vero mea non praeteribunt. 
\verse De die autem illa et hora nemo scit, neque angeli caelorum neque Filius, nisi Pater solus. 
\verse Sicut enim dies Noe, ita erit adventus Filii hominis. 
\verse Sicut enim erant in diebus ante diluvium comedentes et bibentes, nubentes et nuptum tradentes, usque ad eum diem, quo introivit in arcam Noe, 
\verse et non cognoverunt, donec venit diluvium et tulit omnes, ita erit et adventus Filii hominis. 
\verse Tunc duo erunt in agro: unus assumitur, et unus relinquitur;  
\verse duae molentes in mola: una assumitur, et una relinquitur. 
\verse Vigilate ergo, quia nescitis qua die Dominus vester venturus sit. 
\verse Illud autem scitote quoniam si sciret pater familias qua hora fur venturus esset, vigilaret utique et non sineret perfodi domum suam. 
\verse Ideo et vos estote parati, quia, qua nescitis hora, Filius hominis venturus est. 
\verse Quis putas est fidelis servus et prudens, quem constituit dominus supra familiam suam, ut det illis cibum in tempore? 
\verse Beatus ille servus, quem cum venerit dominus eius, invenerit sic facientem. 
\verse Amen dico vobis quoniam super omnia bona sua constituet eum. 
\verse Si autem dixerit malus servus ille in corde suo: “Moram facit dominus meus venire”, 
\verse et coeperit percutere conservos suos, manducet autem et bibat cum ebriis, 
\verse veniet dominus servi illius in die, qua non sperat, et in hora, qua ignorat, 
\verse et dividet eum partemque eius ponet cum hypocritis; illic erit fletus et stridor dentium. 
\end{biblechapter}

\begin{biblechapter}  
\verse Tunc simile erit regnum caelorum decem virginibus, quae accipientes lampades suas exierunt obviam sponso. 
\verse Quinque autem ex eis erant fatuae, et quinque prudentes. 
\verse Fatuae enim, acceptis lampadibus suis, non sumpserunt oleum secum; 
\verse prudentes vero acceperunt oleum in vasis cum lampadibus suis. 
\verse Moram autem faciente sponso, dormitaverunt omnes et dormierunt. 
\verse Media autem nocte clamor factus est: “Ecce sponsus! Exite obviam ei”. 
\verse Tunc surrexerunt omnes virgines illae et ornaverunt lampades suas. 
\verse Fatuae autem sapientibus dixerunt: “Date nobis de oleo vestro, quia lampades nostrae exstinguuntur”. 
\verse Responderunt prudentes dicentes: “Ne forte non sufficiat nobis et vobis, ite potius ad vendentes et emite vobis”.  
\verse Dum autem irent emere, venit sponsus, et quae paratae erant, intraverunt cum eo ad nuptias; et clausa est ianua. 
\verse Novissime autem veniunt et reliquae virgines dicentes: “Domine, domine, aperi nobis”. 
\verse At ille respondens ait: “Amen dico vobis: Nescio vos”. 
\verse Vigilate itaque, quia nescitis diem neque horam. 
\verse Sicut enim homo peregre proficiscens vocavit servos suos et tradidit illis bona sua. 
\verse Et uni dedit quinque talenta, alii autem duo, alii vero unum, unicuique secundum propriam virtutem, et profectus est. Statim 
\verse abiit, qui quinque talenta acceperat, et operatus est in eis et lucratus est alia quinque; 
\verse similiter qui duo acceperat, lucratus est alia duo. 
\verse Qui autem unum acceperat, abiens fodit in terra et abscondit pecuniam domini sui. 
\verse Post multum vero temporis venit dominus servorum illorum et ponit rationem cum eis. 
\verse Et accedens, qui quinque talenta acceperat, obtulit alia quinque talenta dicens: “Domine, quinque talenta tradidisti mihi; ecce alia quinque superlucratus sum”. 
\verse Ait illi dominus eius: “Euge, serve bone et fidelis. Super pauca fuisti fidelis; supra multa te constituam: intra in gaudium domini tui”. 
\verse Accessit autem et qui duo talenta acceperat, et ait: “Domine, duo talenta tradidisti mihi; ecce alia duo lucratus sum”. 
\verse Ait illi dominus eius: “Euge, serve bone et fidelis. Super pauca fuisti fidelis; supra multa te constituam: intra in gaudium domini tui”. 
\verse Accedens autem et qui unum talentum acceperat, ait: “Domine, novi te quia homo durus es: metis, ubi non seminasti, et congregas, ubi non sparsisti; 
\verse et timens abii et abscondi talentum tuum in terra. Ecce habes, quod tuum est”. 
\verse Respondens autem dominus eius dixit ei: “Serve male et piger! Sciebas quia meto, ubi non seminavi, et congrego, ubi non sparsi? 
\verse Oportuit ergo te mittere pecuniam meam nummulariis, et veniens ego recepissem, quod meum est cum usura. 
\verse Tollite itaque ab eo talentum et date ei, qui habet decem talenta: 
\verse omni enim habenti dabitur, et abundabit; ei autem, qui non habet, et quod habet, auferetur ab eo. 
\verse Et inutilem servum eicite in tenebras exteriores: illic erit fletus et stridor dentium”. 
\verse Cum autem venerit Filius hominis in gloria sua, et omnes angeli cum eo, tunc sedebit super thronum gloriae suae. 
\verse Et congregabuntur ante eum omnes gentes; et separabit eos ab invicem, sicut pastor segregat oves ab haedis,  
\verse et statuet oves quidem a dextris suis, haedos autem a sinistris. 
\verse Tunc dicet Rex his, qui a dextris eius erunt: “Venite, benedicti Patris mei; possidete paratum vobis regnum a constitutione mundi. 
\verse Esurivi enim, et dedistis mihi manducare; sitivi, et dedistis mihi bibere; hospes eram, et collegistis me; 
\verse nudus, et operuistis me; infirmus, et visitastis me; in carcere eram, et venistis ad me”. 
\verse Tunc respondebunt ei iusti dicentes: “Domine, quando te vidimus esurientem et pavimus, aut sitientem et dedimus tibi potum? 
\verse Quando autem te vidimus hospitem et collegimus, aut nudum et cooperuimus? 
\verse Quando autem te vidimus infirmum aut in carcere et venimus ad te?”. 
\verse Et respondens Rex dicet illis: “Amen dico vobis: Quamdiu fecistis uni de his fratribus meis minimis, mihi fecistis”. 
\verse Tunc dicet et his, qui a sinistris erunt: “Discedite a me, maledicti, in ignem aeternum, qui praeparatus est Diabolo et angelis eius. 
\verse Esurivi enim, et non dedistis mihi manducare; sitivi, et non dedistis mihi potum; 
\verse hospes eram, et non collegistis me; nudus, et non operuistis me; infirmus et in carcere, et non visitastis me”. 
\verse Tunc respondebunt et ipsi dicentes: “Domine, quando te vidimus esurientem aut sitientem aut hospitem aut nudum aut infirmum vel in carcere et non ministravimus tibi?”. 
\verse Tunc respondebit illis dicens: “Amen dico vobis: Quamdiu non fecistis uni de minimis his, nec mihi fecistis”. 
\verse Et ibunt hi in supplicium aeternum, iusti autem in vitam aeternam". 
\end{biblechapter}

\begin{biblechapter}  
\verse Et factum est, cum consummasset Iesus sermones hos omnes, dixit discipulis suis: 
\verse “Scitis quia post biduum Pascha fiet, et Filius hominis traditur, ut crucifigatur". 
\verse Tunc congregati sunt principes sacerdotum et seniores populi in aulam principis sacerdotum, qui dicebatur Caiphas, 
\verse et consilium fecerunt, ut Iesum dolo tenerent et occiderent; 
\verse dicebant autem: “Non in die festo, ne tumultus fiat in populo". 
\verse Cum autem esset Iesus in Bethania, in domo Simonis leprosi, 
\verse accessit ad eum mulier habens alabastrum unguenti pretiosi et effudit super caput ipsius recumbentis. 
\verse Videntes autem discipuli, indignati sunt dicentes: “Ut quid perditio haec? 
\verse Potuit enim istud venumdari multo et dari pauperibus".  
\verse Sciens autem Iesus ait illis: “Quid molesti estis mulieri? Opus enim bonum operata est in me; 
\verse nam semper pauperes habetis vobiscum, me autem non semper habetis. 
\verse Mittens enim haec unguentum hoc supra corpus meum, ad sepeliendum me fecit. 
\verse Amen dico vobis: Ubicumque praedicatum fuerit hoc evangelium in toto mundo, dicetur et quod haec fecit in memoriam eius". 
\verse Tunc abiit unus de Duodecim, qui dicebatur Iudas Iscariotes, ad principes sacerdotum 
\verse et ait: “Quid vultis mihi dare, et ego vobis eum tradam?". At illi constituerunt ei triginta argenteos. 
\verse Et exinde quaerebat opportunitatem, ut eum traderet. 
\verse Prima autem Azymorum accesserunt discipuli ad Iesum dicentes: “Ubi vis paremus tibi comedere Pascha?". 
\verse Ille autem dixit: “Ite in civitatem ad quendam et dicite ei: “Magister dicit: Tempus meum prope est; apud te facio Pascha cum discipulis meis”". 
\verse Et fecerunt discipuli, sicut constituit illis Iesus, et paraverunt Pascha. 
\verse Vespere autem facto, discumbebat cum Duodecim. 
\verse Et edentibus illis, dixit: “Amen dico vobis: Unus vestrum me traditurus est". 
\verse Et contristati valde, coeperunt singuli dicere ei: “Numquid ego sum, Domine?".  
\verse At ipse respondens ait: “Qui intingit mecum manum in paropside, hic me tradet. 
\verse Filius quidem hominis vadit, sicut scriptum est de illo; vae autem homini illi, per quem Filius hominis traditur! Bonum erat ei, si natus non fuisset homo ille". 
\verse Respondens autem Iudas, qui tradidit eum, dixit: “Numquid ego sum, Rabbi?". Ait illi: “Tu dixisti". 
\verse Cenantibus autem eis, accepit Iesus panem et benedixit ac fregit deditque discipulis et ait: “Accipite, comedite: hoc est corpus meum". 
\verse Et accipiens calicem, gratias egit et dedit illis dicens: “Bibite ex hoc omnes: 
\verse hic est enim sanguis meus novi testamenti, qui pro multis effunditur in remissionem peccatorum. 
\verse Dico autem vobis: Non bibam amodo de hoc genimine vitis usque in diem illum, cum illud bibam vobiscum novum in regno Patris mei". 
\verse Et hymno dicto, exierunt in montem Oliveti. 
\verse Tunc dicit illis Iesus: “Omnes vos scandalum patiemini in me in ista nocte. Scriptum est enim: "Percutiam pastorem, et dispergentur oves gregis". 
\verse Postquam autem resurrexero, praecedam vos in Galilaeam". 
\verse Respondens autem Petrus ait illi: “Et si omnes scandalizati fuerint in te, ego numquam scandalizabor". 
\verse Ait illi Iesus: “Amen dico tibi: In hac nocte, antequam gallus cantet, ter me negabis". 
\verse Ait illi Petrus: “Etiam si oportuerit me mori tecum, non te negabo". Similiter et omnes discipuli dixerunt. 
\verse Tunc venit Iesus cum illis in praedium, quod dicitur Gethsemani. Et dicit discipulis: “Sedete hic, donec vadam illuc et orem". 
\verse Et assumpto Petro et duobus filiis Zebedaei, coepit contristari et maestus esse. 
\verse Tunc ait illis: “Tristis est anima mea usque ad mortem; sustinete hic et vigilate mecum". 
\verse Et progressus pusillum, procidit in faciem suam orans et dicens: “Pater mi, si possibile est, transeat a me calix iste; verumtamen non sicut ego volo, sed sicut tu". 
\verse Et venit ad discipulos et invenit eos dormientes; et dicit Petro: “Sic non potuistis una hora vigilare mecum? 
\verse Vigilate et orate, ut non intretis in tentationem; spiritus quidem promptus est, caro autem infirma". 
\verse Iterum secundo abiit et oravit dicens: “Pater mi, si non potest hoc transire, nisi bibam illud, fiat voluntas tua". 
\verse Et venit iterum et invenit eos dormientes: erant enim oculi eorum gravati. 
\verse Et relictis illis, iterum abiit et oravit tertio, eundem sermonem iterum dicens.  
\verse Tunc venit ad discipulos et dicit illis: “Dormite iam et requiescite; ecce appropinquavit hora, et Filius hominis traditur in manus peccatorum. 
\verse Surgite, eamus; ecce appropinquavit, qui me tradit". 
\verse Et adhuc ipso loquente, ecce Iudas, unus de Duodecim, venit, et cum eo turba multa cum gladiis et fustibus, missi a principibus sacerdotum et senioribus populi. 
\verse Qui autem tradidit eum, dedit illis signum dicens: “Quemcumque osculatus fuero, ipse est; tenete eum!". 
\verse Et confestim accedens ad Iesum dixit: “Ave, Rabbi!" et osculatus est eum. 
\verse Iesus autem dixit illi: “Amice, ad quod venisti!". Tunc accesserunt et manus iniecerunt in Iesum et tenuerunt eum. 
\verse Et ecce unus ex his, qui erant cum Iesu, extendens manum exemit gladium suum et percutiens servum principis sacerdotum amputavit auriculam eius. 
\verse Tunc ait illi Iesus: “Converte gladium tuum in locum suum. Omnes enim, qui acceperint gladium, gladio peribunt. 
\verse An putas quia non possum rogare Patrem meum, et exhibebit mihi modo plus quam duodecim legiones angelorum?  
\verse Quomodo ergo implebuntur Scripturae quia sic oportet fieri?". 
\verse In illa hora dixit Iesus turbis: “Tamquam ad latronem existis cum gladiis et fustibus comprehendere me? Cotidie sedebam docens in templo, et non me tenuistis".  
\verse Hoc autem totum factum est, ut implerentur scripturae Prophetarum. Tunc discipuli omnes, relicto eo, fugerunt. 
\verse Illi autem tenentes Iesum duxerunt ad Caipham principem sacerdotum, ubi scribae et seniores convenerant. 
\verse Petrus autem sequebatur eum a longe usque in aulam principis sacerdotum; et ingressus intro sede bat cum ministris, ut videret finem. 
\verse Principes autem sacerdotum et omne concilium quaerebant falsum testimonium contra Iesum, ut eum morti traderent, 
\verse et non invenerunt, cum multi falsi testes accessissent. Novissime autem venientes duo  
\verse dixerunt: “Hic dixit: “Possum destruere templum Dei et post triduum aedificare illud”". 
\verse Et surgens princeps sacerdotum ait illi: “Nihil respondes? Quid isti adversum te testificantur?". 
\verse Iesus autem tacebat. Et princeps sacerdotum ait illi: “Adiuro te per Deum vivum, ut dicas nobis, si tu es Christus Filius Dei". 
\verse Dicit illi Iesus: “Tu dixisti. Verumtamen dico vobis: Amodo videbitis Filium hominis sedentem a dextris Virtutis et venientem in nubibus caeli". 
\verse Tunc princeps sacerdotum scidit vestimenta sua dicens: “Blasphemavit! Quid adhuc egemus testibus? Ecce nunc audistis blasphemiam. 
\verse Quid vobis videtur?". Illi autem respondentes dixerunt: “Reus est mortis!". 
\verse Tunc exspuerunt in faciem eius et colaphis eum ceciderunt; alii autem palmas in faciem ei dederunt 
\verse dicentes: “Prophetiza nobis, Christe: Quis est, qui te percussit?". 
\verse Petrus vero sedebat foris in atrio; et accessit ad eum una ancilla dicens: “Et tu cum Iesu Galilaeo eras!". 
\verse At ille negavit coram omnibus dicens: “Nescio quid dicis!". 
\verse Exeunte autem illo ad ianuam, vidit eum alia et ait his, qui erant ibi: “Hic erat cum Iesu Nazareno!". 
\verse Et iterum negavit cum iuramento: “Non novi hominem!". 
\verse Post pusillum autem accesserunt, qui stabant, et dixerunt Petro: “Vere et tu ex illis es, nam et loquela tua manifestum te facit". 
\verse Tunc coepit detestari et iurare: “Non novi hominem!". Et continuo gallus cantavit; 
\verse et recordatus est Petrus verbi Iesu, quod dixerat: “Priusquam gallus cantet, ter me negabis". Et egressus foras ploravit amare. 
\end{biblechapter}

\begin{biblechapter}  
\verse Mane autem facto, consilium inierunt omnes principes sacerdotum et seniores populi adversus Iesum, ut eum morti traderent. 
\verse Et vinctum adduxerunt eum et tradiderunt Pilato praesidi. 
\verse Tunc videns Iudas, qui eum tradidit, quod damnatus esset, paenitentia ductus, rettulit triginta argenteos principibus sacerdotum et senioribus 
\verse dicens: “Peccavi tradens sanguinem innocentem". At illi dixerunt: “Quid ad nos? Tu videris!". 
\verse Et proiectis argenteis in templo, recessit et abiens laqueo se suspendit. 
\verse Principes autem sacerdotum, acceptis argenteis, dixerunt: “Non licet mittere eos in corbanam, quia pretium sanguinis est". 
\verse Consilio autem inito, emerunt ex illis agrum Figuli in sepulturam peregrinorum. 
\verse Propter hoc vocatus est ager ille ager Sanguinis usque in hodiernum diem. 
\verse Tunc impletum est quod dictum est per Ieremiam prophetam di centem: “Et acceperunt triginta argenteos, pretium appretiati quem appretiaverunt a filiis Israel, 
\verse et dederunt eos in agrum Figuli, sicut constituit mihi Dominus". 
\verse Iesus autem stetit ante praesidem; et interrogavit eum praeses dicens: “Tu es Rex Iudaeorum?". Dixit autem Iesus: “Tu dicis". 
\verse Et cum accusaretur a principibus sacerdotum et senioribus, nihil respondit. 
\verse Tunc dicit illi Pilatus: “Non audis quanta adversum te dicant testimonia?". 
\verse Et non respondit ei ad ullum verbum, ita ut miraretur praeses vehementer. 
\verse Per diem autem sollemnem consueverat praeses dimittere turbae unum vinctum, quem voluissent. 
\verse Habebant autem tunc vinctum insignem, qui dicebatur Barabbas. 
\verse Congregatis ergo illis dixit Pilatus: “Quem vultis dimittam vobis: Barabbam an Iesum, qui dicitur Christus?". 
\verse Sciebat enim quod per invidiam tradidissent eum. 
\verse Sedente autem illo pro tribunali, misit ad illum uxor eius dicens: “Nihil tibi et iusto illi. Multa enim passa sum hodie per visum propter eum". 
\verse Principes autem sacerdotum et seniores persuaserunt turbis, ut peterent Barabbam, Iesum vero perderent. 
\verse Respondens autem praeses ait illis: “Quem vultis vobis de duobus dimittam?". At illi dixerunt: “Barabbam!". 
\verse Dicit illis Pilatus: “Quid igitur faciam de Iesu, qui dicitur Christus?". Dicunt omnes: “Crucifigatur!". 
\verse Ait autem: “Quid enim mali fecit?". At illi magis clamabant dicentes: “Crucifigatur!". 
\verse Videns autem Pilatus quia nihil proficeret, sed magis tumultus fieret, accepta aqua, lavit manus coram turba dicens: “Innocens ego sum a sanguine hoc; vos videritis!". 
\verse Et respondens universus populus dixit: “Sanguis eius super nos et super filios nostros". 
\verse Tunc dimisit illis Barabbam; Iesum autem flagellatum tradidit, ut crucifigeretur. 
\verse Tunc milites praesidis suscipientes Iesum in praetorio congregaverunt ad eum universam cohortem. 
\verse Et exuentes eum, clamydem coccineam circumdederunt ei  
\verse et plectentes coronam de spinis posuerunt super caput eius et arundinem in dextera eius et, genu flexo ante eum, illudebant ei dicentes: “Ave, rex Iudaeorum!". 
\verse Et exspuentes in eum acceperunt arundinem et percutiebant caput eius. 
\verse Et postquam illuserunt ei, exuerunt eum clamyde et induerunt eum vestimentis eius et duxerunt eum, ut crucifigerent. 
\verse Exeuntes autem invenerunt hominem Cyrenaeum nomine Simonem; hunc angariaverunt, ut tolleret crucem eius. 
\verse Et venerunt in locum, qui dicitur Golgotha, quod est Calvariae locus, 
\verse et dederunt ei vinum bibere cum felle mixtum; et cum gustasset, noluit bibere. 
\verse Postquam autem crucifixerunt eum, diviserunt vestimenta eius sortem mittentes 
\verse et sedentes servabant eum ibi. 
\verse Et imposuerunt super caput eius causam ipsius scriptam: “Hic est Iesus Rex Iudaeorum". 
\verse Tunc crucifiguntur cum eo duo latrones: unus a dextris, et unus a sinistris. 
\verse Praetereuntes autem blasphemabant eum moventes capita sua 
\verse et dicentes: “Qui destruis templum et in triduo illud reaedificas, salva temetipsum; si Filius Dei es, descende de cruce!". 
\verse Similiter et principes sacerdotum illudentes cum scribis et senioribus dicebant: 
\verse “Alios salvos fecit, seipsum non potest salvum facere. Rex Israel est; descendat nunc de cruce, et credemus in eum. 
\verse Confidit in Deo; liberet nunc, si vult eum. Dixit enim: “Dei Filius sum”". 
\verse Idipsum autem et latrones, qui crucifixi erant cum eo, improperabant ei. 
\verse A sexta autem hora tenebrae factae sunt super universam terram usque ad horam nonam. 
\verse Et circa horam nonam clamavit Iesus voce magna dicens: “Eli, Eli, lema sabacthani?", hoc est: “Deus meus, Deus meus, ut quid dereliquisti me?".  
\verse Quidam autem ex illic stantibus audientes dicebant: “Eliam vocat iste".  
\verse Et continuo currens unus ex eis acceptam spongiam implevit aceto et imposuit arundini et dabat ei bibere. 
\verse Ceteri vero dicebant: “Sine, videamus an veniat Elias liberans eum". 
\verse Iesus autem iterum clamans voce magna emisit spiritum. 
\verse Et ecce velum templi scissum est a summo usque deorsum in duas partes, et terra mota est, et petrae scissae sunt; 
\verse et monumenta aperta sunt, et multa corpora sanctorum, qui dormierant, surrexerunt 
\verse et exeuntes de monumentis post resurrectionem eius venerunt in sanctam civitatem et apparuerunt multis. 
\verse Centurio autem et, qui cum eo erant custodientes Iesum, viso terrae motu et his, quae fiebant, timuerunt valde dicentes: “Vere Dei Filius erat iste!". 
\verse Erant autem ibi mulieres multae a longe aspicientes, quae secutae erant Iesum a Galilaea ministrantes ei; 
\verse inter quas erat Maria Magdalene et Maria Iacobi et Ioseph mater et mater filiorum Zebedaei. 
\verse Cum sero autem factum esset, venit homo dives ab Arimathaea nomine Ioseph, qui et ipse discipulus erat Iesu. 
\verse Hic accessit ad Pilatum et petiit corpus Iesu. Tunc Pilatus iussit reddi. 
\verse Et accepto corpore, Ioseph involvit illud in sindone munda 
\verse et posuit illud in monumento suo novo, quod exciderat in petra, et advolvit saxum magnum ad ostium monumenti et abiit.  
\verse Erat autem ibi Maria Magdalene et altera Maria sedentes contra sepulcrum. 
\verse Altera autem die, quae est post Parascevem, convenerunt principes sacerdotum et pharisaei ad Pilatum 
\verse dicentes: “Domine, recordati sumus quia seductor ille dixit adhuc vivens: “Post tres dies resurgam”. 
\verse Iube ergo custodiri sepulcrum usque in diem tertium, ne forte veniant discipuli eius et furentur eum et dicant plebi: “Surrexit a mortuis”, et erit novissimus error peior priore".  
\verse Ait illis Pilatus: “Habetis custodiam; ite, custodite, sicut scitis".  
\verse Illi autem abeuntes munierunt sepulcrum, signantes lapidem, cum custodia. 
\end{biblechapter}

\begin{biblechapter}  
\verse Sero autem post sabbatum, cum illucesceret in primam sabbati, venit Maria Magdalene et altera Maria videre sepulcrum. 
\verse Et ecce terrae motus factus est magnus: angelus enim Domini descendit de caelo et accedens revolvit lapidem et sedebat super eum. 
\verse Erat autem aspectus eius sicut fulgur, et vestimentum eius candidum sicut nix. 
\verse Prae timore autem eius exterriti sunt custodes et facti sunt velut mortui. 
\verse Respondens autem angelus dixit mulieribus: “Nolite timere vos! Scio enim quod Iesum, qui crucifixus est, quaeritis. 
\verse Non est hic: surrexit enim, sicut dixit. Venite, videte locum, ubi positus erat. 
\verse Et cito euntes dicite discipulis eius: “Surrexit a mortuis et ecce praecedit vos in Galilaeam; ibi eum videbitis”. Ecce dixi vobis". 
\verse Et exeuntes cito de monumento cum timore et magno gaudio cucurrerunt nuntiare discipulis eius. 
\verse Et ecce Iesus occurrit illis dicens: “Avete". Illae autem accesserunt et tenuerunt pedes eius et adoraverunt eum. 
\verse Tunc ait illis Iesus: “Nolite timere; ite, nuntiate fratribus meis, ut eant in Galilaeam et ibi me videbunt". 
\verse Quae cum abiissent, ecce quidam de custodia venerunt in civitatem et nuntiaverunt principibus sacerdotum omnia, quae facta fuerant. 
\verse Et congregati cum senioribus, consilio accepto, pecuniam copiosam dederunt militibus 
\verse dicentes: “Dicite: “Discipuli eius nocte venerunt et furati sunt eum, nobis dormientibus”. 
\verse Et si hoc auditum fuerit a praeside, nos suadebimus ei et securos vos faciemus". 
\verse At illi, accepta pecunia, fecerunt, sicut erant docti. Et divulgatum est verbum istud apud Iudaeos usque in hodiernum diem. 
\verse Undecim autem discipuli abierunt in Galilaeam, in montem ubi constituerat illis Iesus, 
\verse et videntes eum adoraverunt; quidam autem dubitaverunt.  
\verse Et accedens Iesus locutus est eis dicens: “Data est mihi omnis potestas in caelo et in terra. 
\verse Euntes ergo docete omnes gentes, baptizantes eos in nomine Patris et Filii et Spiritus Sancti, 
\verse docentes eos servare omnia, quaecumque mandavi vobis. Et ecce ego vobiscum sum omnibus diebus usque ad consummationem saeculi".
\end{biblechapter}
