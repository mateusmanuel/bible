\biblebook{ Epistula ad Romanos}
\begin{biblechapter}
 \verse Paulus servus Christi Iesu, vocatus apostolus, segregatus in evangelium Dei, 
\verse quod ante promiserat per prophetas suos in Scripturis sanctis 
\verse de Filio suo, qui factus est ex semine David secundum carnem, 
\verse qui constitutus est Filius Dei in virtute secundum Spiritum sanctificationis ex resurrectione mortuorum, Iesu Christo Domino nostro, 
\verse per quem accepimus gratiam et apostolatum ad oboeditionem fidei in omnibus gentibus pro nomine eius, 
\verse in quibus estis et vos vocati Iesu Christi, 
 \verse omnibus, qui sunt Romae dilectis Dei, vocatis sanctis: gratia vobis et pax a Deo Patre nostro et Domino Iesu Christo.
 \verse Primum quidem gratias ago Deo meo per Iesum Christum pro omnibus vobis, quia fides vestra annuntiatur in universo mundo; 
\verse testis enim mihi est Deus, cui servio in spiritu meo in evangelio Filii eius, quomodo sine intermissione memoriam vestri faciam 
\verse semper in orationibus meis obsecrans, si quo modo tandem aliquando prosperum iter habeam in voluntate Dei veniendi ad vos. 
\verse Desidero enim videre vos, ut aliquid impertiar gratiae vobis spiritalis ad confirmandos vos, 
\verse id est una vobiscum consolari per eam, quae invicem est, fidem vestram atque meam. 
\verse Nolo autem vos ignorare, fratres, quia saepe proposui venire ad vos et prohibitus sum usque adhuc, ut aliquem fructum habeam et in vobis, sicut et in ceteris gentibus. 
\verse Graecis ac barbaris, sapientibus et insipientibus debitor sum. 
\verse Itaque, quod in me est, promptus sum et vobis, qui Romae estis, evangelizare.
 \verse Non enim erubesco evangelium: virtus enim Dei est in salutem omni credenti, Iudaeo primum et Graeco. 
\verse Iustitia enim Dei in eo revelatur ex fide in fidem, sicut scriptum est: “ Iustus autem ex fide vivet ”.
 \verse Revelatur enim ira Dei de caelo super omnem impietatem et iniustitiam hominum, qui veritatem in iniustitia detinent, 
\verse quia, quod noscibile est Dei, manifestum est in illis; Deus enim illis manifestavit. 
\verse Invisibilia enim ipsius a creatura mundi per ea, quae facta sunt, intellecta conspiciuntur, sempiterna eius et virtus et divinitas, ut sint inexcusabiles; 
\verse quia, cum cognovissent Deum, non sicut Deum glorificaverunt aut gratias egerunt, sed evanuerunt in cogitationibus suis, et obscuratum est insipiens cor eorum. 
\verse Dicentes se esse sapientes, stulti facti sunt, 
\verse et mutaverunt gloriam incorruptibilis Dei in similitudinem imaginis corruptibilis hominis et volucrum et quadrupedum et serpentium.
 \verse Propter quod tradidit illos Deus in concupiscentiis cordis eorum in immunditiam, ut ignominia afficiant corpora sua in semetipsis, 
\verse qui commutaverunt veritatem Dei in mendacio et coluerunt et servierunt creaturae potius quam Creatori, qui est benedictus in saecula. Amen.
 \verse Propterea tradidit illos Deus in passiones ignominiae. Nam et feminae eorum immutaverunt naturalem usum in eum, qui est contra naturam; 
\verse similiter et masculi, relicto naturali usu feminae, exarserunt in desideriis suis in invicem, masculi in masculos turpitudinem operantes et mercedem, quam oportuit, erroris sui in semetipsis recipientes. 
\verse Et sicut non probaverunt Deum habere in notitia, tradidit eos Deus in reprobum sensum, ut faciant, quae non conveniunt, 
\verse repletos omni iniquitate, malitia, avaritia, nequitia, plenos invidia, homicidio, contentione, dolo, malignitate, susurrones, 
\verse detractores, Deo odibiles, contumeliosos, superbos, elatos, inventores malorum, parentibus non oboedientes, 
\verse insipientes, incompositos, sine affectione, sine misericordia. 
\verse Qui cum iudicium Dei cognovissent, quoniam qui talia agunt, digni sunt morte, non solum ea faciunt, sed et consentiunt facientibus.
 
\begin{biblechapter}
\verse Propter quod inexcusabilis es, o homo omnis, qui iudicas. In quo enim iudicas alterum, teipsum condemnas; eadem enim agis, qui iudicas. 
\verse Scimus enim quoniam iudicium Dei est secundum veritatem in eos, qui talia agunt. 
\verse Existimas autem hoc, o homo, qui iudicas eos, qui talia agunt, et facis ea, quia tu effugies iudicium Dei? 
\verse An divitias benignitatis eius et patientiae et longanimitatis contemnis, ignorans quoniam benignitas Dei ad paenitentiam te adducit? 
\verse Secundum duritiam autem tuam et impaenitens cor thesaurizas tibi iram in die irae et revelationis iusti iudicii Dei, 
\verse qui reddet unicuique secundum opera eius: 
\verse his quidem, qui secundum patientiam boni operis gloriam et honorem et incorruptionem quaerunt, vitam aeternam; 
\verse his autem, qui ex contentione et non oboediunt veritati, oboediunt autem iniquitati, ira et indignatio. 
\verse Tribulatio et angustia in omnem animam hominis operantis malum, Iudaei primum et Graeci; 
\verse gloria autem et honor et pax omni operanti bonum, Iudaeo primum et Graeco. 
\verse Non est enim personarum acceptio apud Deum!
 \verse Quicumque enim sine lege peccaverunt, sine lege et peribunt; et, quicumque in lege peccaverunt, per legem iudicabuntur. 
\verse Non enim auditores legis iusti sunt apud Deum, sed factores legis iustificabuntur. 
\verse Cum enim gentes, quae legem non habent, naturaliter, quae legis sunt, faciunt, eiusmodi legem non habentes ipsi sibi sunt lex; 
\verse qui ostendunt opus legis scriptum in cordibus suis, testimonium simul reddente illis conscientia ipsorum, et inter se invicem cogitationibus accusantibus aut etiam defendentibus, 
\verse in die, cum iudicabit Deus occulta hominum secundum evangelium meum per Christum Iesum. 
 \verse Si autem tu Iudaeus cognominaris et requiescis in lege et gloriaris in Deo, 
 \verse et nosti Voluntatem et discernis potiora instructus per legem, 
\verse et confidis teipsum ducem esse caecorum, lumen eorum, qui in tenebris sunt, 
\verse eruditorem insipientium, magistrum infantium, habentem formam scientiae et veritatis in lege. 
\verse Qui ergo alium doces, teipsum non doces? Qui praedicas non furandum, furaris? 
\verse Qui dicis non moechandum, moecharis? Qui abominaris idola, templa spolias? 
\verse Qui in lege gloriaris, per praevaricationem legis Deum inhonoras? 
\verse “ Nomen enim Dei propter vos blasphematur inter gentes ”, sicut scriptum est.
 \verse Circumcisio quidem prodest, si legem observes; si autem praevaricator legis sis, circumcisio tua praeputium facta est. 
\verse Si igitur praeputium iustitias legis custodiat, nonne praeputium illius in circumcisionem reputabitur? 
\verse Et iudicabit, quod ex natura est praeputium legem consummans, te, qui per litteram et circumcisionem praevaricator legis es. 
\verse Non enim qui manifesto Iudaeus est, neque quae manifesto in carne circumcisio, 
\verse sed qui in abscondito Iudaeus est, et circumcisio cordis in spiritu non littera, cuius laus non ex hominibus sed ex Deo est.
 
\begin{biblechapter}
\verse Quid ergo amplius est Iudaeo, aut quae utilitas circumcisionis?
 \verse Multum per omnem modum. Primum quidem, quia credita sunt illis eloquia Dei.
 \verse Quid enim, si quidam non crediderunt? Numquid incredulitas illorum fidem Dei evacuabit?
 \verse Absit! Exstet autem Deus verax, omnis autem homo mendax, sicut scriptum est: “ Ut iustificeris in sermonibus tuis et vincas cum iudicaris ”.
 \verse Si autem iniustitia nostra iustitiam Dei commendat, quid dicemus? Numquid iniustus Deus, qui infert iram? Secundum hominem dico.
 \verse Absit! Alioquin quomodo iudicabit Deus mundum?
 \verse Si enim veritas Dei in meo mendacio abundavit in gloriam ipsius, quid adhuc et ego tamquam peccator iudicor? 
\verse Et non, sicut blasphemamur, et sicut aiunt quidam nos dicere: “ Faciamus mala, ut veniant bona ”? Quorum damnatio iusta est. 
\verse Quid igitur? Praecellimus eos? Nequaquam! Antea enim causati sumus Iudaeos et Graecos omnes sub peccato esse, 
\verse sicut scriptum est:
 “ Non est iustus quisquam,
 \verse non est intellegens, non est requirens Deum.
 \verse Omnes declinaverunt, simul inutiles facti sunt;
 non est qui faciat bonum, non est usque ad unum.
 \verse Sepulcrum patens est guttur eorum,
 linguis suis dolose agebant,
 venenum aspidum sub labiis eorum,
 \verse quorum os maledictione et amaritudine plenum est;
 \verse veloces pedes eorum ad effundendum sanguinem,
 \verse contritio et infelicitas in viis eorum,
 \verse et viam pacis non cognoverunt.
 \verse Non est timor Dei ante oculos eorum ”.
 \verse Scimus autem quoniam, quaecumque lex loquitur, his, qui in lege sunt, loquitur, ut omne os obstruatur, et obnoxius fiat omnis mundus Deo; 
\verse quia ex operibus legis non iustificabitur omnis caro coram illo, per legem enim cognitio peccati.
 \verse Nunc autem sine lege iustitia Dei manifestata est, testificata a Lege et Prophetis, 
\verse iustitia autem Dei per fidem Iesu Christi, in omnes, qui credunt. Non enim est distinctio: 
\verse omnes enim peccaverunt et egent gloria Dei, 
\verse iustificati gratis per gratiam ipsius per redemptionem, quae est in Christo Iesu; 
\verse quem proposuit Deus propitiatorium per fidem in sanguine ipsius ad ostensionem iustitiae suae, cum praetermisisset praecedentia delicta 
 \verse in sustentatione Dei, ad ostensionem iustitiae eius in hoc tempore, ut sit ipse iustus et iustificans eum, qui ex fide est Iesu.
 \verse Ubi est ergo gloriatio? Exclusa est. Per quam legem? Operum? Non, sed per legem fidei. 
\verse Arbitramur enim iustificari hominem per fidem sine operibus legis. 
\verse An Iudaeorum Deus tantum? Nonne et gentium? Immo et gentium, 
 \verse quoniam quidem unus Deus, qui iustificabit circumcisionem ex fide et praeputium per fidem. 
\verse Legem ergo destruimus per fidem? Absit, sed legem statuimus.
 
\begin{biblechapter}
\verse Quid ergo dicemus invenisse Abraham progenitorem no strum secundum carnem? 
\verse Si enim Abraham ex operibus iustificatus est, habet gloriam sed non apud Deum. 
\verse Quid enim Scriptura dicit? “ Credidit autem Abraham Deo, et reputatum est illi ad iustitiam ”.
 \verse Ei autem, qui operatur, merces non reputatur secundum gratiam sed secundum debitum; 
\verse ei vero, qui non operatur, sed credit in eum, qui iustificat impium, reputatur fides eius ad iustitiam, 
\verse sicut et David dicit beatitudinem hominis, cui Deus reputat iustitiam sine operibus:
 \verse “ Beati, quorum remissae sunt iniquitates,
 et quorum tecta sunt peccata.
 \verse Beatus vir, cui non imputabit Dominus peccatum ”.
 \verse Beatitudo ergo haec in circumcisione an etiam in praeputio? Dicimus enim: “ Reputata est Abrahae fides ad iustitiam ”. 
\verse Quomodo ergo reputata est? In circumcisione an in praeputio? Non in circumcisione sed in praeputio: 
\verse et signum accepit circumcisionis, signaculum iustitiae fidei, quae fuit in praeputio, ut esset pater omnium credentium per praeputium, ut reputetur illis iustitia, 
\verse et pater circumcisionis his non tantum, qui ex circumcisione sunt, sed et qui sectantur vestigia eius, quae fuit in praeputio, fidei patris nostri Abrahae.
 \verse Non enim per legem promissio Abrahae aut semini eius, ut heres esset mundi, sed per iustitiam fidei; 
\verse si enim qui ex lege heredes sunt, exinanita est fides, et abolita est promissio. 
\verse Lex enim iram operatur; ubi autem non est lex, nec praevaricatio. 
\verse Ideo ex fide, ut secundum gratiam, ut firma sit promissio omni semini, non ei, qui ex lege est solum, sed et ei, qui ex fide est Abrahae — qui est pater omnium nostrum, 
\verse sicut scriptum est: “ Patrem multarum gentium posui te ” C, ante Deum, cui credidit, qui vivificat mortuos et vocat ea, quae non sunt, quasi sint; 
\verse qui contra spem in spe credidit, ut fieret pater multarum gentium, secundum quod dictum est: “ Sic erit semen tuum ”. 
\verse Et non infirmatus fide consideravit corpus suum iam emortuum, cum fere centum annorum esset, et emortuam vulvam Sarae; 
\verse in repromissione autem Dei non haesitavit diffidentia, sed confortatus est fide, dans gloriam Deo, 
 \verse et plenissime sciens quia, quod promisit, potens est et facere. 
\verse Ideo et reputatum est illi ad iustitiam.
 \verse Non est autem scriptum tantum propter ipsum: reputatum est illi, 
\verse sed et propter nos, quibus reputabitur, credentibus in eum, qui suscitavit Iesum Dominum nostrum a mortuis, 
\verse qui traditus est propter delicta nostra et suscitatus est propter iustificationem nostram.
 
\begin{biblechapter}
\verse Iustificati igitur ex fide, pacem habemus ad Deum per Domi num nostrum Iesum Christum, 
\verse per quem et accessum habemus fide in gratiam istam, in qua stamus et gloriamur in spe gloriae Dei. 
\verse Non solum autem, sed et gloriamur in tribulationibus, scientes quod tribulatio patientiam operatur, 
\verse patientia autem probationem, probatio vero spem; 
\verse spes autem non confundit, quia caritas Dei diffusa est in cordibus nostris per Spiritum Sanctum, qui datus est nobis.
 \verse Adhuc enim Christus, cum adhuc infirmi essemus, secundum tempus pro impiis mortuus est. 
\verse Vix enim pro iusto quis moritur; nam pro bono forsitan quis et audeat mori. 
\verse Commendat autem suam caritatem Deus in nos, quoniam, cum adhuc peccatores essemus, Christus pro nobis mortuus est. 
\verse Multo igitur magis iustificati nunc in sanguine ipsius, salvi erimus ab ira per ipsum! 
\verse Si enim, cum inimici essemus, reconciliati sumus Deo per mortem Filii eius, multo magis reconciliati salvi erimus in vita ipsius; 
\verse non solum autem, sed et gloriamur in Deo per Dominum nostrum Iesum Christum, per quem nunc reconciliationem accepimus.
 \verse Propterea, sicut per unum hominem peccatum in hunc mundum intravit, et per peccatum mors, et ita in omnes homines mors pertransiit, eo quod omnes peccaverunt. 
\verse Usque ad legem enim peccatum erat in mundo; peccatum autem non imputatur, cum lex non est, 
\verse sed regnavit mors ab Adam usque ad Moysen etiam in eos, qui non peccaverunt in similitudine praevaricationis Adae, qui est figura futuri.
 \verse Sed non sicut delictum, ita et donum; si enim unius delicto multi mortui sunt, multo magis gratia Dei et donum in gratia unius hominis Iesu Christi in multos abundavit. 
\verse Et non sicut per unum, qui peccavit, ita et donum; nam iudicium ex uno in condemnationem, gratia autem ex multis delictis in iustificationem. 
\verse Si enim unius delicto mors regnavit per unum, multo magis, qui abundantiam gratiae et donationis iustitiae accipiunt, in vita regnabunt per unum Iesum Christum.
 \verse Igitur sicut per unius delictum in omnes homines in condemnationem, sic et per unius iustitiam in omnes homines in iustificationem vitae; 
\verse sicut enim per inoboedientiam unius hominis peccatores constituti sunt multi, ita et per unius oboeditionem iusti constituentur multi.
 \verse Lex autem subintravit, ut abundaret delictum; ubi autem abundavit peccatum, superabundavit gratia, 
\verse ut sicut regnavit peccatum in morte, ita et gratia regnet per iustitiam in vitam aeternam per Iesum Christum Dominum nostrum.
 
\begin{biblechapter}
\verse Quid ergo dicemus? Perma nebimus in peccato, ut gratia abundet?
 \verse Absit! Qui enim mortui sumus peccato, quomodo adhuc vivemus in illo? 
\verse An ignoratis quia, quicumque baptizati sumus in Christum Iesum, in mortem ipsius baptizati sumus? 
\verse Consepulti ergo sumus cum illo per baptismum in mortem, ut quemadmodum suscitatus est Christus a mortuis per gloriam Patris, ita et nos in novitate vitae ambulemus. 
\verse Si enim complantati facti sumus similitudini mortis eius, sed et resurrectionis erimus; 
\verse hoc scientes quia vetus homo noster simul crucifixus est, ut destruatur corpus peccati, ut ultra non serviamus peccato. 
\verse Qui enim mortuus est, iustificatus est a peccato.
 \verse Si autem mortui sumus cum Christo, credimus quia simul etiam vivemus cum eo; 
 \verse scientes quod Christus suscitatus ex mortuis iam non moritur, mors illi ultra non dominatur. 
\verse Quod enim mortuus est, peccato mortuus est semel; quod autem vivit, vivit Deo. 
\verse Ita et vos existimate vos mortuos quidem esse peccato, viventes autem Deo in Christo Iesu.
 \verse Non ergo regnet peccatum in vestro mortali corpore, ut oboediatis concupiscentiis eius, 
\verse neque exhibeatis membra vestra arma iniustitiae peccato, sed exhibete vos Deo tamquam ex mortuis viventes et membra vestra arma iustitiae Deo. 
\verse Peccatum enim vobis non dominabitur; non enim sub lege estis sed sub gratia.
 \verse Quid ergo? Peccabimus, quoniam non sumus sub lege sed sub gratia? Absit! 
 \verse Nescitis quoniam, cui exhibetis vos servos ad oboedientiam, servi estis eius, cui oboeditis, sive peccati ad mortem, sive oboeditionis ad iustitiam? 
\verse Gratias autem Deo quod fuistis servi peccati, oboedistis autem ex corde in eam formam doctrinae, in quam traditi estis, 
\verse liberati autem a peccato servi facti estis iustitiae.
 \verse Humanum dico propter infirmitatem carnis vestrae. Sicut enim exhibuistis membra vestra servientia immunditiae et iniquitati ad iniquitatem, ita nunc exhibete membra vestra servientia iustitiae ad sanctificationem. 
\verse Cum enim servi essetis peccati, liberi eratis iustitiae. 
\verse Quem ergo fructum habebatis tunc, in quibus nunc erubescitis? Nam finis illorum mors! 
\verse Nunc vero liberati a peccato, servi autem facti Deo, habetis fructum vestrum in sanctificationem, finem vero vitam aeternam! 
\verse Stipendia enim peccati mors, donum autem Dei vita aeterna in Christo Iesu Domino nostro.
 
\begin{biblechapter}
\verse An ignoratis, fratres — scienti bus enim legem loquor — quia lex in homine dominatur, quanto tempore vivit? 
\verse Nam quae sub viro est mulier, viventi viro alligata est lege; si autem mortuus fuerit vir, soluta est a lege viri. 
\verse Igitur, vivente viro, vocabitur adultera, si fuerit alterius viri; si autem mortuus fuerit vir, libera est a lege, ut non sit adultera, si fuerit alterius viri. 
\verse Itaque, fratres mei, et vos mortificati estis legi per corpus Christi, ut sitis alterius, eius qui ex mortuis suscitatus est, ut fructificaremus Deo. 
\verse Cum enim essemus in carne, passiones peccatorum, quae per legem sunt, operabantur in membris nostris, ut fructificarent morti; 
\verse nunc autem soluti sumus a lege, mortui ei, in qua detinebamur, ita ut serviamus in novitate Spiritus et non in vetustate litterae.
 \verse Quid ergo dicemus? Lex peccatum est? Absit! Sed peccatum non cognovi, nisi per legem; nam concupiscentiam nescirem, nisi lex diceret: “ Non concupisces ”. 
\verse Occasione autem accepta, peccatum per mandatum operatum est in me omnem concupiscentiam; sine lege enim peccatum mortuum erat. 
\verse Ego autem vivebam sine lege aliquando; sed, cum venisset mandatum, peccatum revixit, 
\verse ego autem mortuus sum; et inventum est mihi mandatum, quod erat ad vitam, hoc esse ad mortem; 
\verse nam peccatum, occasione accepta, per mandatum seduxit me et per illud occidit. 
\verse Itaque lex quidem sancta, et mandatum sanctum et iustum et bonum. 
\verse Quod ergo bonum est, mihi factum est mors? Absit! Sed peccatum, ut appareat peccatum, per bonum mihi operatum est mortem; ut fiat supra modum peccans peccatum per mandatum.
 \verse Scimus enim quod lex spiritalis est; ego autem carnalis sum, venumdatus sub peccato. 
\verse Quod enim operor, non intellego; non enim, quod volo, hoc ago, sed quod odi, illud facio. 
\verse Si autem, quod nolo, illud facio, consentio legi quoniam bona. 
\verse Nunc autem iam non ego operor illud, sed, quod habitat in me, peccatum. 
\verse Scio enim quia non habitat in me, hoc est in carne mea, bonum; nam velle adiacet mihi, operari autem bonum, non! 
\verse Non enim, quod volo bonum, facio, sed, quod nolo malum, hoc ago. 
\verse Si autem, quod nolo, illud facio, iam non ego operor illud, sed, quod habitat in me, peccatum. 
\verse Invenio igitur hanc legem volenti mihi facere bonum, quoniam mihi malum adiacet. 
 \verse Condelector enim legi Dei secundum interiorem hominem; 
\verse video autem aliam legem in membris meis repugnantem legi mentis meae et captivantem me in lege peccati, quae est in membris meis.
 \verse Infelix ego homo! Quis me liberabit de corpore mortis huius? 
\verse Gratias autem Deo per Iesum Christum Dominum nostrum! Igitur ego ipse mente servio legi Dei, carne autem legi peccati.
 
\begin{biblechapter}
\verse Nihil ergo nunc damnationis est his, qui sunt in Christo Iesu; 
\verse lex enim Spiritus vitae in Christo Iesu liberavit te a lege peccati et mortis. 
\verse Nam, quod impossibile erat legi, in quo infirmabatur per carnem, Deus Filium suum mittens in similitudine carnis peccati et pro peccato, damnavit peccatum in carne, 
\verse ut iustitia legis impleretur in nobis, qui non secundum carnem ambulamus sed secundum Spiritum.
 \verse Qui enim secundum carnem sunt, quae carnis sunt, sapiunt; qui vero secundum Spiritum, quae sunt Spiritus. 
\verse Nam sapientia carnis mors, sapientia autem Spiritus vita et pax; 
\verse quoniam sapientia carnis inimicitia est in Deum, legi enim Dei non subicitur nec enim potest. 
\verse Qui autem in carne sunt, Deo placere non possunt.
 \verse Vos autem in carne non estis sed in Spiritu, si tamen Spiritus Dei habitat in vobis. Si quis autem Spiritum Christi non habet, hic non est eius. 
\verse Si autem Christus in vobis est, corpus quidem mortuum est propter peccatum, Spiritus vero vita propter iustitiam. 
\verse Quod si Spiritus eius, qui suscitavit Iesum a mortuis, habitat in vobis, qui suscitavit Christum a mortuis vivificabit et mortalia corpora vestra per inhabitantem Spiritum suum in vobis.
 \verse Ergo, fratres, debitores sumus non carni, ut secundum carnem vivamus. 
\verse Si enim secundum carnem vixeritis, moriemini; si autem Spiritu opera corporis mortificatis, vivetis. 
\verse Quicumque enim Spiritu Dei aguntur, hi filii Dei sunt. 
\verse Non enim accepistis spiritum servitutis iterum in timorem, sed accepistis Spiritum adoptionis filiorum, in quo clamamus: “ Abba, Pater! ”. 
\verse Ipse Spiritus testimonium reddit una cum spiritu nostro, quod sumus filii Dei. 
\verse Si autem filii, et heredes: heredes quidem Dei, coheredes autem Christi, si tamen compatimur, ut et conglorificemur. 
\verse Existimo enim quod non sunt condignae passiones huius temporis ad futuram gloriam, quae revelanda est in nobis.
 \verse Nam exspectatio creaturae revelationem filiorum Dei exspectat; 
\verse vanitati enim creatura subiecta est, non volens sed propter eum, qui subiecit, in spem, 
\verse quia et ipsa creatura liberabitur a servitute corruptionis in libertatem gloriae filiorum Dei. 
\verse Scimus enim quod omnis creatura congemiscit et comparturit usque adhuc; 
\verse non solum autem, sed et nos ipsi primitias Spiritus habentes, et ipsi intra nos gemimus adoptionem filiorum exspectantes, redemptionem corporis nostri. 
\verse Spe enim salvi facti sumus; spes autem, quae videtur, non est spes; nam, quod videt, quis sperat? 
\verse Si autem, quod non videmus, speramus, per patientiam exspectamus.
 \verse Similiter autem et Spiritus adiuvat infirmitatem nostram; nam quid oremus, sicut oportet, nescimus, sed ipse Spiritus interpellat gemitibus inenarrabilibus; 
\verse qui autem scrutatur corda, scit quid desideret Spiritus, quia secundum Deum postulat pro sanctis.
 \verse Scimus autem quoniam diligentibus Deum omnia cooperantur in bonum, his, qui secundum propositum vocati sunt. 
\verse Nam, quos praescivit, et praedestinavit conformes fieri imaginis Filii eius, ut sit ipse primogenitus in multis fratribus; 
\verse quos autem praedestinavit, hos et vocavit; et quos vocavit, hos et iustificavit; quos autem iustificavit, illos et glorificavit.
 \verse Quid ergo dicemus ad haec? Si Deus pro nobis, quis contra nos? 
\verse Qui Filio suo non pepercit, sed pro nobis omnibus tradidit illum, quomodo non etiam cum illo omnia nobis donabit? 
\verse Quis accusabit adversus electos Dei? Deus, qui iustificat? 
\verse Quis est qui condemnet? Christus Iesus, qui mortuus est, immo qui suscitatus est, qui et est ad dexteram Dei, qui etiam interpellat pro nobis?
 \verse Quis nos separabit a caritate Christi? Tribulatio an angustia an persecutio an fames an nuditas an periculum an gladius? 
\verse Sicut scriptum est:
 “ Propter te mortificamur tota die,
 aestimati sumus ut oves occisionis ”.
 \verse Sed in his omnibus supervincimus per eum, qui dilexit nos.
 \verse Certus sum enim quia neque mors neque vita neque angeli neque principatus neque instantia neque futura neque virtutes 
\verse neque altitudo neque profundum neque alia quaelibet creatura poterit nos separare a caritate Dei, quae est in Christo Iesu Domino nostro.
 
\begin{biblechapter}
\verse Veritatem dico in Christo, non mentior, testimonium mihi per hibente conscientia mea in Spiritu Sancto, 
\verse quoniam tristitia est mihi magna, et continuus dolor cordi meo. 
\verse Optarem enim ipse ego anathema esse a Christo pro fratribus meis, cognatis meis secundum carnem, 
\verse qui sunt Israelitae, quorum adoptio est filiorum et gloria et testamenta et legislatio et cultus et promissiones, 
\verse quorum sunt patres, et ex quibus Christus secundum carnem: qui est super omnia Deus benedictus in saecula. Amen.
 \verse Non autem quod exciderit verbum Dei. Non enim omnes, qui ex Israel, hi sunt Israel; 
\verse neque quia semen sunt Abrahae, omnes filii, sed: “ In Isaac vocabitur tibi semen ”. 
\verse Id est, non qui filii carnis, hi filii Dei, sed qui filii sunt promissionis, aestimantur semen; 
\verse promissionis enim verbum hoc est: “ Secundum hoc tempus veniam, et erit Sarae filius ”. 
\verse Non solum autem, sed et Rebecca ex uno concubitum habens, Isaac patre nostro; 
\verse cum enim nondum nati fuissent aut aliquid egissent bonum aut malum, ut secundum electionem propositum Dei maneret, 
\verse non ex operibus sed ex vocante dictum est ei: “ Maior serviet minori ”; 
\verse sicut scriptum est: “ Iacob dilexi, Esau autem odio habui ”.
 \verse Quid ergo dicemus? Numquid iniustitia apud Deum? Absit! 
\verse Moysi enim dicit: “ Miserebor, cuius misereor, et misericordiam praestabo, cui misericordiam praesto ”.
 \verse Igitur non volentis neque currentis sed miserentis Dei. 
\verse Dicit enim Scriptura pharaoni: “ In hoc ipsum excitavi te, ut ostendam in te virtutem meam, et ut annuntietur nomen meum in universa terra ”. 
\verse Ergo, cuius vult, miseretur et, quem vult, indurat.
 \verse Dices itaque mihi: “ Quid ergo adhuc queritur? Voluntati enim eius quis restitit? ”. 
\verse O homo, sed tu quis es, qui respondeas Deo? Numquid dicet figmentum ei, qui se finxit: “ Quid me fecisti sic? ”. 
\verse An non habet potestatem figulus luti ex eadem massa facere aliud quidem vas in honorem, aliud vero in ignominiam? 
\verse Quod si volens Deus ostendere iram et notam facere potentiam suam, sustinuit in multa patientia vasa irae aptata in interitum; 
 \verse et ut ostenderet divitias gloriae suae in vasa misericordiae, quae praeparavit in gloriam, 
\verse quos et vocavit nos non solum ex Iudaeis sed etiam ex gentibus? 
\verse Sicut et in Osee dicit:
 “ Vocabo Non plebem meam Plebem meam
 et Non dilectam Dilectam. 
\verse Et erit: in loco, ubi dictum est eis:
 “Non plebs mea vos”,
 ibi vocabuntur Filii Dei vivi ”.
 \verse Isaias autem clamat pro Israel: “ Si fuerit numerus filiorum Israel tamquam arena maris, reliquiae salvae fient. 
\verse Verbum enim consummans et brevians faciet Dominus super terram ”.
 \verse Et sicut praedixit Isaias:
 “ Nisi Dominus Sabaoth reliquisset nobis semen,
 sicut Sodoma facti essemus
 et sicut Gomorra similes fuissemus ”.
 \verse Quid ergo dicemus? Quod gentes, quae non sectabantur iustitiam, apprehenderunt iustitiam, iustitiam autem, quae ex fide est; 
\verse Israel vero sectans legem iustitiae in legem non pervenit. 
\verse Quare? Quia non ex fide sed quasi ex operibus; offenderunt in lapidem offensionis, 
\verse sicut scriptum est:
 “ Ecce pono in Sion lapidem offensionis et petram scandali;
 et, qui credit in eo, non confundetur ”.
 
\begin{biblechapter}
\verse Fratres, voluntas quidem cordis mei et obsecratio ad Deum pro illis in salutem. 
\verse Testimonium enim perhibeo illis quod aemulationem Dei habent sed non secundum scientiam; 
\verse ignorantes enim Dei iustitiam et suam iustitiam quaerentes statuere, iustitiae Dei non sunt subiecti; 
\verse finis enim legis Christus ad iustitiam omni credenti.
 \verse Moyses enim scribit de iustitia, quae ex lege est: “ Qui fecerit homo, vivet in eis ”. 
\verse Quae autem ex fide est iustitia, sic dicit: “ Ne dixeris in corde tuo: Quis ascendet in caelum?”, id est Christum deducere; 
\verse aut: “ Quis descendet in abyssum? ”, hoc est Christum ex mortuis revocare. 
\verse Sed quid dicit? “ Prope te est verbum, in ore tuo et in corde tuo ”; hoc est verbum fidei, quod praedicamus. 
\verse Quia si confitearis in ore tuo: “ Dominum Iesum! ”, et in corde tuo credideris quod Deus illum excitavit ex mortuis, salvus eris. 
 \verse Corde enim creditur ad iustitiam, ore autem confessio fit in salutem. 
\verse Dicit enim Scriptura:
 “ Omnis, qui credit in illo, non confundetur ”.
 \verse Non enim est distinctio Iudaei et Graeci, nam idem Dominus omnium, dives in omnes, qui invocant illum:
 \verse Omnis enim, quicumque invocaverit nomen Domini, salvus erit.
 \verse Quomodo ergo invocabunt, in quem non crediderunt? Aut quomodo credent ei, quem non audierunt? Quomodo autem audient sine praedicante? 
\verse Quomodo vero praedicabunt nisi mittantur? Sicut scriptum est:
 “ Quam speciosi pedes evangelizantium bona ”.
 \verse Sed non omnes oboedierunt evangelio; Isaias enim dicit:
 “ Domine, quis credidit auditui nostro? ”. 
\verse Ergo fides ex auditu, auditus autem per verbum Christi.
 \verse Sed dico: Numquid non audierunt? Quin immo,
 in omnem terram exiit sonus eorum,
 et in fines orbis terrae verba eorum.
 \verse Sed dico: Numquid Israel non cognovit? Primus Moyses dicit:
 “ Ego ad aemulationem vos adducam per Non gentem:
 per gentem insipientem ad iram vos provocabo ”.
 \verse Isaias autem audet et dicit: “ Inventus sum in non quaerentibus me; palam apparui his, qui me non interrogabant ”.
 \verse Ad Israel autem dicit: “ Tota die expandi manus meas ad populum non credentem et contradicentem ”.
 
\begin{biblechapter}
\verse Dico ergo: Numquid repulit Deus populum suum? Absit! Nam et ego Israelita sum, ex semine Abraham, tribu Beniamin. 
\verse Non reppulit Deus plebem suam, quam praescivit. An nescitis in Elia quid dicit Scriptura? Quemadmodum interpellat Deum adversus Israel: 
\verse “ Domine, prophetas tuos occiderunt, altaria tua suffoderunt, et ego relictus sum solus, et quaerunt animam meam ”.
 \verse Sed quid dicit illi responsum divinum?
 “ Reliqui mihi septem milia virorum, qui non curvaverunt genu Baal ”.
 \verse Sic ergo et in hoc tempore reliquiae secundum electionem gratiae factae sunt. \verse Si autem gratia, iam non ex operibus, alioquin gratia iam non est gratia.
 \verse Quid ergo? Quod quaerit Israel, hoc non est consecutus, electio autem consecuta est; ceteri vero excaecati sunt, 
\verse sicut scriptum est:
 “ Dedit illis Deus spiritum soporis,
 oculos, ut non videant,
 et aures, ut non audiant,
 usque in hodiernum diem ”.
 \verse Et David dicit:
 “ Fiat mensa eorum in laqueum et in captionem
 et in scandalum et in retributionem illis. 
\verse Obscurentur oculi eorum, ne videant,
 et dorsum illorum semper incurva! ”.
 \verse Dico ergo: Numquid sic offenderunt, ut caderent? Absit! Sed illorum casu salus gentibus, ut illi ad aemulationem adducantur. 
\verse Quod si casus illorum divitiae sunt mundi, et deminutio eorum divitiae gentium, quanto magis plenitudo eorum!
 \verse Vobis autem dico gentibus: Quantum quidem ego sum gentium apostolus, ministerium meum honorifico, 
\verse si quo modo ad aemulandum provocem carnem meam et salvos faciam aliquos ex illis.
 \verse Si enim amissio eorum reconciliatio est mundi, quae assumptio, nisi vita ex mortuis? 
\verse Quod si primitiae sanctae sunt, et massa; et si radix sancta, et rami. 
\verse Quod si aliqui ex ramis fracti sunt, tu autem, cum oleaster esses, insertus es in illis et consocius radicis pinguedinis olivae factus es, 
\verse noli gloriari adversus ramos; quod si gloriaris, non tu radicem portas, sed radix te.
 \verse Dices ergo: “ Fracti sunt rami, ut ego inserar ”. 
\verse Bene; incredulitate fracti sunt, tu autem fide stas. Noli altum sapere, sed time: 
\verse si enim Deus naturalibus ramis non pepercit, ne forte nec tibi parcat.
 \verse Vide ergo bonitatem et severitatem Dei: in eos quidem, qui ceciderunt, severitatem; in te autem bonitatem Dei, si permanseris in bonitate, alioquin et tu excideris. 
\verse Sed et illi, si non permanserint in incredulitate, inserentur; potens est enim Deus iterum inserere illos! 
\verse Nam si tu ex naturali excisus es oleastro et contra naturam insertus es in bonam olivam, quanto magis hi, qui secundum naturam sunt, inserentur suae olivae. 
\verse Nolo enim vos ignorare, fratres, mysterium hoc, ut non sitis vobis ipsis sapientes, quia caecitas ex parte contigit in Israel, donec plenitudo gentium intraret, 
 \verse et sic omnis Israel salvus fiet, sicut scriptum est:
 “ Veniet ex Sion, qui eripiat,
 avertet impietates ab Iacob;
 \verse et hoc illis a me testamentum,
 cum abstulero peccata eorum ”.
 \verse Secundum evangelium quidem inimici propter vos, secundum electionem autem carissimi propter patres; 
\verse sine paenitentia enim sunt dona et vocatio Dei! 
 \verse Sicut enim aliquando vos non credidistis Deo, nunc autem misericordiam consecuti estis propter illorum incredulitatem, 
\verse ita et isti nunc non crediderunt propter vestram misericordiam, ut et ipsi nunc misericordiam consequantur. 
\verse Conclusit enim Deus omnes in incredulitatem, ut omnium misereatur!
 \verse O altitudo divitiarum et sapientiae et scientiae Dei! Quam incomprehensibilia sunt iudicia eius, et investigabiles viae eius!
 \verse Quis enim cognovit sensum Domini?
 Aut quis consiliarius eius fuit?
 \verse Aut quis prior dedit illi,
 et retribuetur ei?
 \verse Quoniam ex ipso et per ipsum et in ipsum omnia. Ipsi gloria in saecula. Amen.
 
\begin{biblechapter}
\verse Obsecro itaque vos, fratres, per misericordiam Dei, ut exhibeatis corpora vestra hostiam viventem, sanctam, Deo placentem, rationabile obsequium vestrum; 
 \verse et nolite conformari huic saeculo, sed transformamini renovatione mentis, ut probetis quid sit voluntas Dei, quid bonum et bene placens et perfectum.
 \verse Dico enim per gratiam, quae data est mihi, omnibus, qui sunt inter vos, non altius sapere quam oportet sapere, sed sapere ad sobrietatem, unicuique sicut Deus divisit mensuram fidei. 
\verse Sicut enim in uno corpore multa membra habemus, omnia autem membra non eundem actum habent, 
\verse ita multi unum corpus sumus in Christo, singuli autem alter alterius membra. 
\verse Habentes autem donationes secundum gratiam, quae data est nobis, differentes: sive prophetiam, secundum rationem fidei; 
\verse sive ministerium, in ministrando; sive qui docet, in doctrina; 
\verse sive qui exhortatur, in exhortando; qui tribuit, in simplicitate; qui praeest, in sollicitudine; qui miseretur, in hilaritate.
 \verse Dilectio sine simulatione. Odientes malum, adhaerentes bono; 
\verse caritate fraternitatis invicem diligentes, honore invicem praevenientes, 
\verse sollicitudine non pigri, spiritu ferventes, Domino servientes, 
\verse spe gaudentes, in tribulatione patientes, orationi instantes, 
\verse necessitatibus sanctorum communicantes, hospitalitatem sectantes. 
\verse Benedicite persequentibus; benedicite et nolite maledicere! 
\verse Gaudere cum gaudentibus, flere cum flentibus. 
\verse Idipsum invicem sentientes, non alta sapientes, sed humilibus consentientes. Nolite esse prudentes apud vosmetipsos.
 \verse Nulli malum pro malo reddentes; providentes bona coram omnibus hominibus; 
 \verse si fieri potest, quod ex vobis est, cum omnibus hominibus pacem habentes; 
 \verse non vosmetipsos vindicantes, carissimi, sed date locum irae, scriptum est enim: “ Mihi vindicta, ego retribuam ”, dicit Dominus. 
\verse Sed si esurierit inimicus tuus, ciba illum; si sitit, potum da illi. Hoc enim faciens, carbones ignis congeres super caput eius. 
\verse Noli vinci a malo, sed vince in bono malum.
 
\begin{biblechapter}
\verse Omnis anima potestatibus sublimioribus subdita sit. Non est enim potestas nisi a Deo; quae autem sunt, a Deo ordinatae sunt. 
\verse Itaque, qui resistit potestati, Dei ordinationi resistit; qui autem resistunt ipsi, sibi damnationem acquirent. 
\verse Nam principes non sunt timori bono operi sed malo. Vis autem non timere potestatem? Bonum fac, et habebis laudem ex illa; 
\verse Dei enim ministra est tibi in bonum. Si autem malum feceris, time; non enim sine causa gladium portat; Dei enim ministra est, vindex in iram ei, qui malum agit. 
\verse Ideo necesse est subditos esse, non solum propter iram sed et propter conscientiam. 
\verse Ideo enim et tributa praestatis; ministri enim Dei sunt in hoc ipsum instantes. 
\verse Reddite omnibus debita: cui tributum tributum, cui vectigal vectigal, cui timorem timorem, cui honorem honorem.
 \verse Nemini quidquam debeatis, nisi ut invicem diligatis: qui enim diligit proximum, legem implevit. 
\verse Nam: Non adulterabis, Non occides, Non furaberis, Non concupisces, et si quod est aliud mandatum, in hoc verbo recapitulatur: Diliges proximum tuum tamquam teipsum. 
\verse Dilectio proximo malum non operatur; plenitudo ergo legis est dilectio.
 \verse Et hoc scientes tempus, quia hora est iam vos de somno surgere; nunc enim propior est nobis salus quam cum credidimus. 
\verse Nox processit, dies autem appropiavit. Abiciamus ergo opera tenebrarum et induamur arma lucis. 
\verse Sicut in die honeste ambulemus: non in comissationibus et ebrietatibus, non in cubilibus et impudicitiis, non in contentione et aemulatione; 
\verse sed induite Dominum Iesum Christum et carnis curam ne feceritis in concupiscentiis.
 
\begin{biblechapter}
\verse Infirmum autem in fide assumite, non in disceptatio nibus cogitationum. 
\verse Alius enim credit manducare omnia; qui autem infirmus est, holus manducat. 
 \verse Is qui manducat, non manducantem non spernat; et, qui non manducat, manducantem non iudicet, Deus enim illum assumpsit. 
\verse Tu quis es, qui iudices alienum servum? Suo domino stat aut cadit; stabit autem, potens est enim Dominus statuere illum.
 \verse Nam alius iudicat inter diem et diem, alius iudicat omnem diem; unusquisque in suo sensu abundet. 
\verse Qui sapit diem, Domino sapit; et, qui manducat, Domino manducat, gratias enim agit Deo; et, qui non manducat, Domino non manducat et gratias agit Deo. 
\verse Nemo enim nostrum sibi vivit, et nemo sibi moritur; \verse sive enim vivimus, Domino vivimus, sive morimur, Domino morimur. Sive ergo vivimus, sive morimur, Domini sumus. 
\verse In hoc enim Christus et mortuus est et vixit, ut et mortuorum et vivorum dominetur.
 \verse Tu autem, quid iudicas fratrem tuum? Aut tu, quare spernis fratrem tuum? Omnes enim stabimus ante tribunal Dei; 
\verse scriptum est enim:
 “ Vivo ego, dicit Dominus,
 mihi flectetur omne genu,
 et omnis lingua confitebitur Deo ”.
 \verse Itaque unusquisque nostrum pro se rationem reddet Deo. 
\verse Non ergo amplius invicem iudicemus, sed hoc iudicate magis, ne ponatis offendiculum fratri vel scandalum.
 \verse Scio et certus sum in Domino Iesu, quia nihil commune per seipsum, nisi ei, qui existimat quid commune esse, illi commune est. 
\verse Si enim propter cibum frater tuus contristatur, iam non secundum caritatem ambulas. Noli cibo tuo illum perdere, pro quo Christus mortuus est! 
\verse Non ergo blasphemetur bonum vestrum! 
\verse Non est enim regnum Dei esca et potus, sed iustitia et pax et gaudium in Spiritu Sancto; 
\verse qui enim in hoc servit Christo, placet Deo et probatus est hominibus. 
\verse Itaque, quae pacis sunt, sectemur et quae aedificationis sunt in invicem. 
\verse Noli propter escam destruere opus Dei! Omnia quidem munda sunt, sed malum est homini, qui per offendiculum manducat. 
 \verse Bonum est non manducare carnem et non bibere vinum neque id, in quo frater tuus offendit.
 \verse Tu, quam fidem habes, penes temetipsum habe coram Deo. Beatus, qui non iudicat semetipsum in eo quod probat. 
\verse Qui autem discernit si manducaverit, damnatus est, quia non ex fide; omne autem, quod non ex fide, peccatum est.
 
\begin{biblechapter}
\verse Debemus autem nos fir miores imbecillitates infir morum sustinere et non nobis placere. 
\verse Unusquisque nostrum proximo placeat in bonum ad aedificationem; 
\verse etenim Christus non sibi placuit, sed sicut scriptum est: “ Improperia improperantium tibi ceciderunt super me ”. 
\verse Quaecumque enim antea scripta sunt, ad nostram doctrinam scripta sunt, ut per patientiam et consolationem Scripturarum spem habeamus. 
\verse Deus autem patientiae et solacii det vobis idipsum sapere in alterutrum secundum Christum Iesum, 
\verse ut unanimes uno ore glorificetis Deum et Patrem Domini nostri Iesu Christi.
 \verse Propter quod suscipite invicem, sicut et Christus suscepit vos, in gloriam Dei. 
\verse Dico enim Christum ministrum fuisse circumcisionis propter veritatem Dei ad confirmandas promissiones patrum; 
\verse gentes autem propter misericordiam glorificare Deum, sicut scriptum est:
 “ Propter hoc confitebor tibi in gentibus et nomini tuo cantabo ”.
 \verse Et iterum dicit: “ Laetamini, gentes, cum plebe eius ”.
 \verse Et iterum:
 “ Laudate, omnes gentes, Dominum,
 et magnificent eum omnes populi ”.
 \verse Et rursus Isaias ait:
 “ Erit radix Iesse,
 et qui exsurget regere gentes:
 in eo gentes sperabunt ”.
 \verse Deus autem spei repleat vos omni gaudio et pace in credendo, ut abundetis in spe in virtute Spiritus Sancti.
 \verse Certus sum autem, fratres mei, et ego ipse de vobis, quoniam et ipsi pleni estis bonitate, repleti omni scientia, ita ut possitis et alterutrum monere. 
 \verse Audacius autem scripsi vobis ex parte, tamquam in memoriam vos reducens propter gratiam, quae data est mihi a Deo, 
\verse ut sim minister Christi Iesu ad gentes, consecrans evangelium Dei, ut fiat oblatio gentium accepta, sanctificata in Spiritu Sancto. 
\verse Habeo igitur gloriationem in Christo Iesu ad Deum; 
\verse non enim audebo aliquid loqui eorum, quae per me non effecit Christus in oboedientiam gentium, verbo et factis, 
\verse in virtute signorum et prodigiorum, in virtute Spiritus, ita ut ab Ierusalem et per circuitum usque in Illyricum repleverim evangelium Christi, 
\verse sic autem contendens praedicare evangelium, non ubi nominatus est Christus, ne super alienum fundamentum aedificarem, 
\verse sed sicut scriptum est:
 “ Quibus non est annuntiatum de eo, videbunt;
 et, qui non audierunt, intellegent ”.
 \verse Propter quod et impediebar plurimum venire ad vos; 
\verse nunc vero ulterius locum non habens in his regionibus, cupiditatem autem habens veniendi ad vos ex multis iam annis, 
\verse cum in Hispaniam proficisci coepero, spero enim quod praeteriens videam vos et a vobis deducar illuc, si vobis primum ex parte fruitus fuero.
 \verse Nunc autem proficiscor in Ierusalem ministrare sanctis; 
\verse probaverunt enim Macedonia et Achaia communicationem aliquam facere in pauperes sanctorum, qui sunt in Ierusalem. 
\verse Placuit enim eis, et debitores sunt eorum; nam si spiritalibus eorum communicaverunt gentes, debent et in carnalibus ministrare eis. 
\verse Hoc igitur cum consummavero et assignavero eis fructum hunc, proficiscar per vos in Hispaniam; 
\verse scio autem quoniam veniens ad vos, in abundantia benedictionis Christi veniam. 
\verse Obsecro autem vos, fratres, per Dominum nostrum Iesum Christum et per caritatem Spiritus, ut concertemini mecum in orationibus pro me ad Deum, 
\verse ut liberer ab infidelibus, qui sunt in Iudaea, et ministerium meum pro Ierusalem acceptum sit sanctis, 
\verse ut veniens ad vos in gaudio per voluntatem Dei refrigerer vobiscum. 
\verse Deus autem pacis sit cum omnibus vobis. Amen.
 
\begin{biblechapter}
\verse Commendo autem vobis Phoebem sororem nostram, quae est ministra ecclesiae, quae est Cenchreis, 
\verse ut eam suscipiatis in Domino digne sanctis et assistatis ei in quocumque negotio vestri indiguerit; etenim ipsa astitit multis et mihi ipsi. 
\verse Salutate Priscam et Aquilam adiutores meos in Christo Iesu, 
\verse qui pro anima mea suas cervices supposuerunt, quibus non solus ego gratias ago sed et cunctae ecclesiae gentium; 
\verse et domesticam eorum ecclesiam.
 Salutate Epaenetum dilectum mihi, primitias Asiae in Christo. 
\verse Salutate Mariam, quae multum laboravit in vobis. 
\verse Salutate Andronicum et Iuniam cognatos meos et concaptivos meos, qui sunt nobiles in apostolis, qui et ante me fuerunt in Christo. 
\verse Salutate Ampliatum dilectissimum mihi in Domino. 
\verse Salutate Urbanum adiutorem nostrum in Christo et Stachyn dilectum meum. 
\verse Salutate Apellem probatum in Christo. Salutate eos, qui sunt ex Aristobuli. 
 \verse Salutate Herodionem cognatum meum. Salutate eos, qui sunt ex Narcissi, qui sunt in Domino. 
\verse Salutate Tryphaenam et Tryphosam, quae laborant in Domino. Salutate Persidam carissimam, quae multum laboravit in Domino. 
\verse Salutate Rufum electum in Domino et matrem eius et meam. 
\verse Salutate Asyncritum, Phlegonta, Hermen, Patrobam, Hermam et, qui cum eis sunt, fratres. 
 \verse Salutate Philologum et Iuliam, Nereum et sororem eius et Olympam et omnes, qui cum eis sunt, sanctos. 
\verse Salutate invicem in osculo sancto. Salutant vos omnes ecclesiae Christi.
 \verse Rogo autem vos, fratres, ut observetis eos, qui dissensiones et offendicula praeter doctrinam, quam vos didicistis, faciunt, et declinate ab illis; 
\verse huiusmodi enim Domino nostro Christo non serviunt sed suo ventri, et per dulces sermones et benedictiones seducunt corda innocentium.
 \verse Vestra enim oboedientia ad omnes pervenit; gaudeo igitur in vobis, sed volo vos sapientes esse in bono et simplices in malo. 
\verse Deus autem pacis conteret Satanam sub pedibus vestris velociter.
 Gratia Domini nostri Iesu vobiscum.
 \verse Salutat vos Timotheus adiutor meus et Lucius et Iason et Sosipater cognati mei. 
\verse Saluto vos ego Tertius, qui scripsi epistulam in Domino. 
\verse Salutat vos Gaius hospes meus et universae ecclesiae. Salutat vos Erastus arcarius civitatis et Quartus frater. 
(\verse) \verse Ei autem, qui potens est vos confirmare iuxta evangelium meum et praedicationem Iesu Christi secundum revelationem mysterii temporibus aeternis taciti, 
\verse manifestati autem nunc, et per scripturas Prophetarum secundum praeceptum aeterni Dei ad oboeditionem fidei in cunctis gentibus patefacti, 
\verse soli sapienti Deo per Iesum Christum, cui gloria in saecula. Amen.
\end{biblechapter}
